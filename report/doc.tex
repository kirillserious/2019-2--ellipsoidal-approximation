\documentclass[a4paper, 12pt]{article}
\usepackage{a4wide}
% Необходимые пакеты для компиляции русского языка, картинок и прочего

\usepackage[utf8]{inputenc}                % Кодировка
\usepackage[main=russian, english]{babel}  % Русский язык
\usepackage[pdftex]{graphicx}              % Картинки
\usepackage{indentfirst}                   % Отступ перед абзацами

\usepackage{amsmath}  % Математические 
\usepackage{amssymb}  % формулы

\usepackage{tikz}                % Векторная графика
                                 %
\usepackage{pgfplots}            % % Нужно для вставки графиков из 
\pgfplotsset{compat=newest}      % % matlab2tikz
\usetikzlibrary{plotmarks}       % %
\usetikzlibrary{arrows.meta}     % %
\usepgfplotslibrary{patchplots}  % %
\usepackage{grffile}             % %

\usepackage{caption} % Чтобы можно было вставлять формулы к подписям рисунков

\usepackage[unicode]{hyperref}                                         % Ссылки и русские закладки
\hypersetup {                                                          %
    pdftitle={Отчёт по практикуму},                                    % Название документа
    pdfsubject={Динамическое программирование и процессы управления},  % Тема документа
    pdfauthor={Егоров Кирилл Юлианович},                 % Автор документа
    pdfcreator={Кафедра системного анализа ВМК МГУ},     % Создатель документа
    pdfproducer={LaTeX},                                 % Программа, создавшая документ
    hidelinks                                            % Скрывает рамку вокруг ссылок
}


\usepackage{nicefrac}
\usepackage{amsthm}  % Красивый внешний вид теорем, определений и доказательств
\newtheoremstyle{def}
        {\topsep}
        {\topsep}
        {\normalfont}
        {\parindent}
        {\bfseries}
        {.}
        {.5em}
        {}
\theoremstyle{def}
\newtheorem{definition}{Определение}
\newtheorem{example}{Пример}

\newtheoremstyle{th}
        {\topsep}
        {\topsep}
        {\itshape}
        {\parindent}
        {\bfseries}
        {.}
        {.5em}
        {}
\theoremstyle{th}
\newtheorem{theorem}{Теорема}
\newtheorem{lemma}{Лемма}
\newtheorem{assertion}{Утверждение}

\newtheoremstyle{rem}
        {0.5\topsep}
        {0.5\topsep}
        {\normalfont}
        {\parindent}
        {\itshape}
        {.}
        {.5em}
        {}
\theoremstyle{rem}
\newtheorem{remark}{Замечание}

% Новое доказательство
\renewenvironment{proof}{\parД о к а з а т е л ь с т в о.}{\hfill$\blacksquare$}
\newcommand{\T}{^\mathrm{T}}
\newcommand{\setR}{\mathbb{R}}
\newcommand{\Varepsilon}{\mathcal{E}}
\begin{document}
\documentclass[a4paper, 12pt]{article}
\usepackage{a4wide}
% Необходимые пакеты для компиляции русского языка, картинок и прочего

\usepackage[utf8]{inputenc}                % Кодировка
\usepackage[main=russian, english]{babel}  % Русский язык
\usepackage[pdftex]{graphicx}              % Картинки
\usepackage{indentfirst}                   % Отступ перед абзацами

\usepackage{amsmath}  % Математические 
\usepackage{amssymb}  % формулы

\usepackage{tikz}                % Векторная графика
                                 %
\usepackage{pgfplots}            % % Нужно для вставки графиков из 
\pgfplotsset{compat=newest}      % % matlab2tikz
\usetikzlibrary{plotmarks}       % %
\usetikzlibrary{arrows.meta}     % %
\usepgfplotslibrary{patchplots}  % %
\usepackage{grffile}             % %

\usepackage{caption} % Чтобы можно было вставлять формулы к подписям рисунков

\usepackage[unicode]{hyperref}                                         % Ссылки и русские закладки
\hypersetup {                                                          %
    pdftitle={Отчёт по практикуму},                                    % Название документа
    pdfsubject={Динамическое программирование и процессы управления},  % Тема документа
    pdfauthor={Егоров Кирилл Юлианович},                 % Автор документа
    pdfcreator={Кафедра системного анализа ВМК МГУ},     % Создатель документа
    pdfproducer={LaTeX},                                 % Программа, создавшая документ
    hidelinks                                            % Скрывает рамку вокруг ссылок
}


\usepackage{nicefrac}
\usepackage{amsthm}  % Красивый внешний вид теорем, определений и доказательств
\newtheoremstyle{def}
        {\topsep}
        {\topsep}
        {\normalfont}
        {\parindent}
        {\bfseries}
        {.}
        {.5em}
        {}
\theoremstyle{def}
\newtheorem{definition}{Определение}
\newtheorem{example}{Пример}

\newtheoremstyle{th}
        {\topsep}
        {\topsep}
        {\itshape}
        {\parindent}
        {\bfseries}
        {.}
        {.5em}
        {}
\theoremstyle{th}
\newtheorem{theorem}{Теорема}
\newtheorem{lemma}{Лемма}
\newtheorem{assertion}{Утверждение}

\newtheoremstyle{rem}
        {0.5\topsep}
        {0.5\topsep}
        {\normalfont}
        {\parindent}
        {\itshape}
        {.}
        {.5em}
        {}
\theoremstyle{rem}
\newtheorem{remark}{Замечание}

% Новое доказательство
\renewenvironment{proof}{\parД о к а з а т е л ь с т в о.}{\hfill$\blacksquare$}
\newcommand{\T}{^\mathrm{T}}
\newcommand{\setR}{\mathbb{R}}
\newcommand{\Varepsilon}{\mathcal{E}}
\begin{document}
\documentclass[a4paper, 12pt]{article}
\usepackage{a4wide}
% Необходимые пакеты для компиляции русского языка, картинок и прочего

\usepackage[utf8]{inputenc}                % Кодировка
\usepackage[main=russian, english]{babel}  % Русский язык
\usepackage[pdftex]{graphicx}              % Картинки
\usepackage{indentfirst}                   % Отступ перед абзацами

\usepackage{amsmath}  % Математические 
\usepackage{amssymb}  % формулы

\usepackage{tikz}                % Векторная графика
                                 %
\usepackage{pgfplots}            % % Нужно для вставки графиков из 
\pgfplotsset{compat=newest}      % % matlab2tikz
\usetikzlibrary{plotmarks}       % %
\usetikzlibrary{arrows.meta}     % %
\usepgfplotslibrary{patchplots}  % %
\usepackage{grffile}             % %

\usepackage{caption} % Чтобы можно было вставлять формулы к подписям рисунков

\usepackage[unicode]{hyperref}                                         % Ссылки и русские закладки
\hypersetup {                                                          %
    pdftitle={Отчёт по практикуму},                                    % Название документа
    pdfsubject={Динамическое программирование и процессы управления},  % Тема документа
    pdfauthor={Егоров Кирилл Юлианович},                 % Автор документа
    pdfcreator={Кафедра системного анализа ВМК МГУ},     % Создатель документа
    pdfproducer={LaTeX},                                 % Программа, создавшая документ
    hidelinks                                            % Скрывает рамку вокруг ссылок
}


\usepackage{nicefrac}
\usepackage{amsthm}  % Красивый внешний вид теорем, определений и доказательств
\newtheoremstyle{def}
        {\topsep}
        {\topsep}
        {\normalfont}
        {\parindent}
        {\bfseries}
        {.}
        {.5em}
        {}
\theoremstyle{def}
\newtheorem{definition}{Определение}
\newtheorem{example}{Пример}

\newtheoremstyle{th}
        {\topsep}
        {\topsep}
        {\itshape}
        {\parindent}
        {\bfseries}
        {.}
        {.5em}
        {}
\theoremstyle{th}
\newtheorem{theorem}{Теорема}
\newtheorem{lemma}{Лемма}
\newtheorem{assertion}{Утверждение}

\newtheoremstyle{rem}
        {0.5\topsep}
        {0.5\topsep}
        {\normalfont}
        {\parindent}
        {\itshape}
        {.}
        {.5em}
        {}
\theoremstyle{rem}
\newtheorem{remark}{Замечание}

% Новое доказательство
\renewenvironment{proof}{\parД о к а з а т е л ь с т в о.}{\hfill$\blacksquare$}
\newcommand{\T}{^\mathrm{T}}
\newcommand{\setR}{\mathbb{R}}
\newcommand{\Varepsilon}{\mathcal{E}}
\begin{document}
\input{formats/report}
\input{set}
\input{decl}
\begin{document}
\input{title_page/doc}
\tableofcontents
\clearpage
%%%%%%%%%%%%%%%%%%%%%%%%%%%%%%%%%%%%%%%%%%%%%%%%%%%%%%%%%%%%%%%%%%%%%%%%%%%%%%%%        


%%  Document start


%%%%%%%%%%%%%%%%%%%%%%%%%%%%%%%%%%%%%%%%%%%%%%%%%%%%%%%%%%%%%%%%%%%%%%%%%%%%%%%%

\section{Об эллипсоидах и сумме Минковского}

\begin{definition}
        Назовём \textit{эллипсоидом} множество
$$
        \mathcal{E}(q,\,Q) = \{
x \in \setR^n \,:\, \langle x-q,\,Q^{-1}(x-q) \rangle \leqslant 1
        \},
        \qquad \mbox{где } Q= Q\T > 0.
$$
\end{definition}

\begin{assertion}
        Опорная функция и опорный вектор эллипсоида имеют вид:
$$
        \rho(l\,|\,\Varepsilon(q,\,Q)) =
        \langle l,\,q \rangle + 
        \langle l,\,Ql \rangle^{\nicefrac{1}{2}},
$$
$$
        x(l) = q + \frac{Ql}{\langle l,\,Ql\rangle^{\nicefrac12}}.
$$
\end{assertion}
\begin{proof}

Будем доказывать для случая $q = 0$. Иначе --- аналогично.

Так как по определению
$
        \rho(l\,|\,A) = \sup_{x\in A}\langle l,\,x\rangle,
$
то мы должны решать задачу максимизации скалярного произведения
$
        \langle l,\,x \rangle
$
при условии, что
$
        \langle x,\,Q^{-1}x \rangle = 1.
$
Запишем функцию Лагранжа для этой задачи:
$$
        \mathcal{L}(l,\,x,\,\lambda)
        =
        \langle l,\,x \rangle
        +
        \lambda(\langle x,\,Q^{-1}x \rangle - 1).
$$
Тогда
$$
        \frac{\partial \mathcal{L}}{\partial x}
        =
        l + 2\lambda Q^{-1} x 
        = 0
        \quad
        \Longrightarrow
        \quad
        x(l) = -\frac{1}{2\lambda}Ql.
$$
Подставим получившееся выражение для опорного вектора в условие:
$$
        \left\langle
-\frac{1}{2\lambda}Ql
,\,
-\frac{1}{2\lambda}Q^{-1}Ql
        \right\rangle
        = 1
        \;\Longrightarrow\;
        \lambda
        =
        -\frac12 \langle l,\,Ql \rangle^{\nicefrac12}
        \,\Longrightarrow\,
        x(l) = \frac{Ql}{\langle l,\,Ql\rangle^{\nicefrac12}}.
$$
В таком случае опорная функция в направлении $l \neq 0$ равна
$$
        \rho(l\,|\,\Varepsilon(0,\,Q))
        =
        \left\langle
l
,\,
\frac{Ql}{\langle l,\,Ql\rangle^{\nicefrac12}}
        \right\rangle
        =
        \langle l,\,Ql \rangle^{\nicefrac{1}{2}}.
$$
\end{proof}

\begin{definition}
        \textit{Суммой Минковского} множеств $A$ и $B$ называется множество
$$
        A + B
        =
        \{\,
x = a + b \::\: a \in A,\,b \in B 
        \,\}.
$$
\end{definition}

\begin{assertion}
        Опорная функция суммы Минковского равна сумме опорных функций каждого из множеств, то есть
$$
        \rho\left(l\,\left|\,\sum_{i=1}^n A_i\right.\right) = \sum_{i=1}^n \rho\,(l\,|\,A_i).
$$
\end{assertion}
\clearpage
\begin{figure}[t]
        \centering
        \input{img/ellipse}
        \caption{Эллипсоид с центром $q = \protect\begin{bmatrix}1\\2\protect\end{bmatrix}$ и матрицей $Q = \protect\begin{bmatrix}5&3\\3&2\protect\end{bmatrix}.$}
\end{figure}
\begin{figure}[b]
        \centering
        \input{img/sum_ellipse}
        \caption{Сумма двух эллипсоидов.}
\end{figure}

%%%%%%%%%%%%%%%%%%%%%%%%%%%%%%%%%%%%%%%%%%%%%%%%%%%%%%%%%%%%%%%%%%%%%%%%%%%%%%%%
\clearpage
\section{Внешняя оценка суммы эллипсоидов}

\begin{theorem}
        Для суммы Минковского эллипсоидов справедлива следующая внешняя оценка
$$
        \sum\limits_{i=1}^{n} \Varepsilon(q_i,\,Q_i)
        =
        \bigcap\limits_{\| l \| = 1} \Varepsilon(q_+(l),\,Q_+(l)),
$$
где
$$
        \begin{aligned}
q_+(l) &= \sum_{i=1}^{n} q_i,
\\
Q_+(l) &= \sum_{i=1}^n p_i \cdot \sum_{i=1}^{n} \frac{Q_i}{p_i},
\quad
\mbox{где }
p_i = \langle l,\,Q_i l \rangle^{\nicefrac12}.
        \end{aligned}
$$
\end{theorem}

\begin{proof}

Будем доказывать для случая $q_i = 0,$ $i = \overline{1,\,n}$. Случай с произвольными центрами~--- аналогично.

Распишем квадрат опорной функции эллипсоида $\Varepsilon(0,\,Q_+(l))$:
\begin{multline*}
        \rho^2(l\,|\,\Varepsilon(0,\,Q_+(l)))
        =
        \sum_{i=1}^{n}
        \langle
        l,\,Q_il
        \rangle
        +
        \sum_{i < j}
        \left\langle
l,\,\left(
\frac{p_i}{p_j}Q_j + \frac{p_j}{p_i}Q_i
\right)l
        \right\rangle
        \geqslant\\\geqslant
        \left\{
\frac{a+b}{2} \geqslant \sqrt{ab}
        \right\}
        \geqslant
        \sum_{i=1}^{n}\langle l,\,Ql \rangle
        +
        2\sum_{i < j}
        \langle l,\,Q_il \rangle^{\nicefrac12}
        \langle l,\,Q_jl \rangle^{\nicefrac12}
        =\\=
        \left(
\sum_{i=1}^n\langle l,\,Q_il\rangle^{\nicefrac12}
        \right)^2
        =
        \rho^2\left(
l\left|
        \sum_{i=1}^n\Varepsilon(0, Q_i)
\right.
        \right).
\end{multline*}
Таким образом, получили, что для любого $l \neq 0$
$$
        \sum_{i=1}^n \Varepsilon(0,\,Q_i) \subseteq \Varepsilon(0,\,Q_+(l)),
$$
причем, так как равенство опорных функций достигается при
$
        p_i = \langle l,\,Q_i l \rangle^{\nicefrac12},
$
то в направлении $l \neq 0$ эллипсоид $\Varepsilon(0,\,Q_+)$ касается суммы $\sum_{i = 0}^{n} \Varepsilon(0,\,Q_i).$

\end{proof}

\clearpage
\begin{figure}[t]
        \centering
        \input{img/outer_sum_10}
        \caption{Эллипсоидальные аппроксимации для 10 направлений.}
\end{figure}
\begin{figure}[b]
        \centering
        \input{img/outer_sum_100}
        \caption{Эллипсоидальные аппроксимации для 100 направлений.}
\end{figure}

%%%%%%%%%%%%%%%%%%%%%%%%%%%%%%%%%%%%%%%%%%%%%%%%%%%%%%%%%%%%%%%%%%%%%%%%%%%%%%%%
\clearpage
\section{Внутренняя оценка суммы эллипсоидов}

\begin{definition}
        \textit{Сингулярным разложением} матрицы $A \in \setR^{n \times m}$ называется представление матрицы в виде
$$
        A = V \mathit{\Sigma} U^*, 
$$
        где
$$
\begin{aligned}
&V \in \setR^{n \times n}\::\: V^* = V^{-1},\\
&U \in \setR^{m \times m}\::\: U^* = U^{-1},\\
&\mathit{\Sigma} = \mathrm{diag}\left(\sigma_1,\,\ldots,\,\sigma_{\min\{n,\,m\}}\right) \in \setR^{n \times m}\::\:\sigma_1 \geqslant \sigma_2 \geqslant \ldots \geqslant \sigma_{\min\{n,\,m\}}.
\end{aligned}
$$ 
\end{definition}

\begin{theorem}
        Сингулярное разложение
        $A = V \mathit{\Sigma} U^*$
        существует для любой комплексной матрицы $A$.
        Если матрица $A$ вещественная, то матрицы $V$, $\mathit{\Sigma}$ и $U$ также можно выбрать вещественными. 
\end{theorem}

\begin{theorem}
        Старшее сингулярное число $\sigma_1$ матрицы $A = V \mathit{\Sigma} U^*$ является её нормой.
\end{theorem}

\begin{definition}
        Назовём линейное преобразование $\mathcal{A}$ \textit{ортогональным}, если оно сохраняет скалярное произведение, то есть
$$
        \langle \mathcal{A}(x),\,\mathcal{A}(y)\rangle = \langle x,\,y\rangle.
$$
\end{definition}

\begin{theorem}\label{th:unitarnost}
        Необходимым и достаточным условием ортогональности линейного преобразования $\mathcal{A}$ в конечномерном пространстве является унитарность матрицы преобразования $A$, то есть
$$
        A^* = A^{-1}.
$$
\end{theorem}

\begin{assertion}
        Для произвольных векторов $a,\,b\in\setR^{n}$ таких, что $\|a\| = \|b\|$, существует матрица ортогонального преобразования, переводящего $a$ в $b$.
\end{assertion}
\begin{proof}
        
        Построим сингулярное разложение для векторов $a$ и $b$:
$$
        a = V_a \mathit{\Sigma}_a u_a,
        \qquad
        b = V_b \mathit{\Sigma}_b u_b,
$$        
причем $V_a,\,V_b \in \setR^{n \times n}$~--- унитарные матрицы, $u_a,\,u_b\in\{-1,\,1\}\in\setR^1$,
$$
        \mathit{\Sigma}_a = [\sigma_a,\,0,\,\ldots,\,0]\T\in\setR^{n \times 1},\quad
        \mathit{\Sigma}_b = [\sigma_b,\,0,\,\ldots,\,0]\T\in\setR^{n \times 1},\quad
        \sigma_a,\,\sigma_b > 0.
$$
Согласно Теореме~\ref{th:unitarnost} $\sigma_a = \sigma_b$.
Тогда преобразуем выражение для вектора $b$:
\begin{multline*}
        b
        =
        V_b \mathit{\Sigma}_b u_b
        =
        V_b (V_a\T V_a)\mathit{\Sigma}_b u_b
        =
        V_bV_a\T V_a \left( \mathit{\Sigma}_a \frac{\sigma_b}{\sigma_a} \right) \left( u_a\frac{u_b}{u_a} \right)
        =\\=
        V_bV_a\T \frac{\sigma_b\cdot u_b}{\sigma_a\cdot u_a}V_a\mathit{\Sigma}_au_a
        =
        \left(V_b V_a\T \frac{\sigma_b\cdot u_b}{\sigma_a\cdot u_a}\right)a.
\end{multline*}
Так как произведение унитарных матриц есть унитарная матрица, теорема доказана.

\end{proof}

\textbf{Следствие 1.} Далее под ортогональным преобразованием из вектора $a$ в вектор $b$ таких, что $\|a\| = \|b\|$, будем понимать
$$
        \mathrm{Orth}(a,\,b) = u_au_bV_bV_a\T.
$$

\begin{assertion}
        Для суммы Минковского эллипсоидов справедлива следующая оценка
$$
        \sum_{i=1}^n \Varepsilon(q_i,\,Q_i) = 
        \bigcup\limits_{\|l\|=1}\Varepsilon(q_-(l),\,Q_-(l)),
$$
        где
$$
        \begin{aligned}
q_-(l) &= \sum_{i=1}^n q_i,
\\
Q_-(l) &= Q_*\T(l) Q_*(l),
\quad
Q_*(l) = \sum_{i=1}^{n}S_i(l)Q_i^{\nicefrac12},
\\
S_i(l) &= \mathrm{Orth}(Q_i^{\nicefrac12}l,\,\lambda_iQ_1^{\nicefrac12}l),
\quad
\lambda_i = \frac{\langle l,\,Q_il\rangle^{\nicefrac12}}{\langle l,\,Q_1l\rangle^{\nicefrac12}}.
        \end{aligned}
$$
\end{assertion}

\begin{proof}

Будем доказывать для случая $q_i = 0$, $i=\overline{1,\,n}$.
Случай с произвольными центрами~--- аналогично.

Итак рассмотрим эллипсоид $\Varepsilon_- = \Varepsilon(0,\,Q_-)$, $Q_- = Q_*\T Q_*$,
$$
        Q_* = \sum_{i=1}^n S_i Q_i^{\nicefrac12},
$$
где $S_i$~--- некоторые унитарные матрицы. Распишем квадрат опорной функции этого эллипсоида:
\begin{multline*}
        \rho^2(l\,|\,\Varepsilon_-)
        =
        \langle l,\,Q_-l \rangle
        =
        \langle Q_*l,\,Q_*l \rangle
        =
        \sum_{i=1}^n \langle l,\,Q_il \rangle
        +
        \sum_{i \neq j} \left \langle
        S_i Q_i^{\nicefrac12}l,\, S_j Q_j^{\nicefrac12}l
        \right\rangle
        \leqslant\\\leqslant
        \{
        \mbox{Неравенство Коши--Буняковского}
        \}
        \leqslant\\\leqslant
        \sum_{i=1}^n \langle l,\,Q_il \rangle
        +
        \sum_{i \neq j}
        \langle l,\,Q_il \rangle^{\nicefrac12}
        \langle l,\,Q_jl \rangle^{\nicefrac12}
        =
        \left(
        \sum_{i=1}^n \langle l,\,Q_il \rangle^{\nicefrac12}
        \right)^2
        =
        \rho^2\left(
        l\left|
        \sum_{i=1}^n \Varepsilon(q_i,\,Q_i)
        \right.
        \right).
\end{multline*}
Таким образом, получили, что $\Varepsilon_-\subseteq\sum_{i=1}^n\Varepsilon(q_i,\,Q_i)$.

Заметим, что равенство в последней формуле при фиксированном направлении $l \neq 0$ достигается при
$$
        S_iQ_i^{\nicefrac12}l = \lambda_i S_1Q_1^{\nicefrac12}l, 
$$
где $\lambda_i$~--- произвольные неотрицательные константы. Если положить $S_1 = I$, а $\lambda_i$ выбирать, исходя из условий нормировки ($\|Q_i^{\nicefrac12}l\| = \|\lambda_iQ_1^{\nicefrac12}l\|$):
$$
        \lambda_i = \frac{\langle l,\,Q_il\rangle^{\nicefrac12}}{\langle l,\,Q_1l\rangle^{\nicefrac12}},
$$
то получим утверждение теоремы.

\end{proof}
\vfill
\begin{figure}[h]

        \centering
        \input{img/inner_sum_10}
        \caption{Эллипсоидальные аппроксимации для 10 направлений.}
\end{figure}
\clearpage
\begin{figure}[t]

        \centering
        \input{img/inner_sum_100}
        \caption{Эллипсоидальные аппроксимации для 100 направлений.}
\end{figure}
\begin{figure}[b]

        \centering
        \input{img/outer_and_inner_sum_50}
        \caption{Эллипсоидальные аппроксимации для 50 направлений.}
\end{figure}
%%%%%%%%%%%%%%%%%%%%%%%%%%%%%%%%%%%%%%%%%%%%%%%%%%%%%%%%%%%%%%%%%%%%%%%%%%%%%%%%


%% Document end


%%%%%%%%%%%%%%%%%%%%%%%%%%%%%%%%%%%%%%%%%%%%%%%%%%%%%%%%%%%%%%%%%%%%%%%%%%%%%%%%
\clearpage
\begin{thebibliography}{9}
\input{bib}
\end{thebibliography}
\end{document}
\tableofcontents
\clearpage
%%%%%%%%%%%%%%%%%%%%%%%%%%%%%%%%%%%%%%%%%%%%%%%%%%%%%%%%%%%%%%%%%%%%%%%%%%%%%%%%        


%%  Document start


%%%%%%%%%%%%%%%%%%%%%%%%%%%%%%%%%%%%%%%%%%%%%%%%%%%%%%%%%%%%%%%%%%%%%%%%%%%%%%%%

\section{Об эллипсоидах и сумме Минковского}

\begin{definition}
        Назовём \textit{эллипсоидом} множество
$$
        \mathcal{E}(q,\,Q) = \{
x \in \setR^n \,:\, \langle x-q,\,Q^{-1}(x-q) \rangle \leqslant 1
        \},
        \qquad \mbox{где } Q= Q\T > 0.
$$
\end{definition}

\begin{assertion}
        Опорная функция и опорный вектор эллипсоида имеют вид:
$$
        \rho(l\,|\,\Varepsilon(q,\,Q)) =
        \langle l,\,q \rangle + 
        \langle l,\,Ql \rangle^{\nicefrac{1}{2}},
$$
$$
        x(l) = q + \frac{Ql}{\langle l,\,Ql\rangle^{\nicefrac12}}.
$$
\end{assertion}
\begin{proof}

Будем доказывать для случая $q = 0$. Иначе --- аналогично.

Так как по определению
$
        \rho(l\,|\,A) = \sup_{x\in A}\langle l,\,x\rangle,
$
то мы должны решать задачу максимизации скалярного произведения
$
        \langle l,\,x \rangle
$
при условии, что
$
        \langle x,\,Q^{-1}x \rangle = 1.
$
Запишем функцию Лагранжа для этой задачи:
$$
        \mathcal{L}(l,\,x,\,\lambda)
        =
        \langle l,\,x \rangle
        +
        \lambda(\langle x,\,Q^{-1}x \rangle - 1).
$$
Тогда
$$
        \frac{\partial \mathcal{L}}{\partial x}
        =
        l + 2\lambda Q^{-1} x 
        = 0
        \quad
        \Longrightarrow
        \quad
        x(l) = -\frac{1}{2\lambda}Ql.
$$
Подставим получившееся выражение для опорного вектора в условие:
$$
        \left\langle
-\frac{1}{2\lambda}Ql
,\,
-\frac{1}{2\lambda}Q^{-1}Ql
        \right\rangle
        = 1
        \;\Longrightarrow\;
        \lambda
        =
        -\frac12 \langle l,\,Ql \rangle^{\nicefrac12}
        \,\Longrightarrow\,
        x(l) = \frac{Ql}{\langle l,\,Ql\rangle^{\nicefrac12}}.
$$
В таком случае опорная функция в направлении $l \neq 0$ равна
$$
        \rho(l\,|\,\Varepsilon(0,\,Q))
        =
        \left\langle
l
,\,
\frac{Ql}{\langle l,\,Ql\rangle^{\nicefrac12}}
        \right\rangle
        =
        \langle l,\,Ql \rangle^{\nicefrac{1}{2}}.
$$
\end{proof}

\begin{definition}
        \textit{Суммой Минковского} множеств $A$ и $B$ называется множество
$$
        A + B
        =
        \{\,
x = a + b \::\: a \in A,\,b \in B 
        \,\}.
$$
\end{definition}

\begin{assertion}
        Опорная функция суммы Минковского равна сумме опорных функций каждого из множеств, то есть
$$
        \rho\left(l\,\left|\,\sum_{i=1}^n A_i\right.\right) = \sum_{i=1}^n \rho\,(l\,|\,A_i).
$$
\end{assertion}
\clearpage
\begin{figure}[t]
        \centering
        % This file was created by matlab2tikz.
%
%The latest updates can be retrieved from
%  http://www.mathworks.com/matlabcentral/fileexchange/22022-matlab2tikz-matlab2tikz
%where you can also make suggestions and rate matlab2tikz.
%
\definecolor{mycolor1}{rgb}{0.00000,0.44700,0.74100}%
%
\begin{tikzpicture}

\begin{axis}[%
width=0.618\linewidth,
height=0.471\linewidth,
at={(0\linewidth,0\linewidth)},
scale only axis,
xmin=-2,
xmax=4,
xlabel style={font=\color{white!15!black}},
xlabel={$x_1$},
ymin=0.5,
ymax=3.5,
ylabel style={font=\color{white!15!black}},
ylabel={$x_2$},
axis background/.style={fill=white},
axis x line*=bottom,
axis y line*=left,
xmajorgrids,
ymajorgrids,
legend style={legend cell align=left, align=left, draw=white!15!black}
]
\addplot [color=mycolor1]
  table[row sep=crcr]{%
3.12132034355964	3.41421356237309\\
3.14093309737706	3.41361883293874\\
3.15610096472476	3.41218966892045\\
3.16813059909176	3.41027848839136\\
3.17787846987421	3.40808691034947\\
3.18592472730893	3.40573192203561\\
3.19267367435337	3.40328137559688\\
3.19841408644868	3.40077372446074\\
3.20335666517915	3.3982293272274\\
3.20765799352175	3.39565707404995\\
3.21143626106343	3.3930583315506\\
3.2147818220386	3.39042930139934\\
3.217764420063	3.38776240921808\\
3.22043820622192	3.38504707754847\\
3.22284525774146	3.38227008706717\\
3.2250180482594	3.37941564172179\\
3.22698115917429	3.37646519817436\\
3.22875241567428	3.37339708240557\\
3.23034355770318	3.37018588734886\\
3.23176050074785	3.3668016188298\\
3.23300319375062	3.36320852788002\\
3.23406503318587	3.35936353046923\\
3.23493173445265	3.35521406411806\\
3.2355794824484	3.35069515469921\\
3.23597206441911	3.34572534988779\\
3.23605649958035	3.34020099103578\\
3.23575636783317	3.33398799537805\\
3.23496150515886	3.32690982030174\\
3.23351178703941	3.31872942300207\\
3.23117098985666	3.3091215094024\\
3.22758343607311	3.29762858411142\\
3.22219964903172	3.28358902636704\\
3.21414389031876	3.26601495419557\\
3.20196759721768	3.24337600812861\\
3.18316711789886	3.21319839168308\\
3.15318699112585	3.17128288128172\\
3.10323675000489	3.11010066257133\\
3.01525609198129	3.01537616591489\\
2.85016840053085	2.85895352273754\\
2.52458200959445	2.58760093991863\\
1.90440137584336	2.13363920441521\\
0.985501334090361	1.54409660598879\\
0.130059671215813	1.06605503500581\\
-0.409288268560127	0.807214909116197\\
-0.70597634609565	0.687302123652136\\
-0.872419406680929	0.632086042700855\\
-0.97199649313969	0.605972330977308\\
-1.03563236326191	0.593558335040314\\
-1.07869104959228	0.587977176231719\\
-1.10924316645516	0.585982787778395\\
-1.13178875034531	0.58594871840975\\
-1.14897766153775	0.587018094070618\\
-1.1624464899118	0.588720827478583\\
-1.17324845072647	0.59079102809827\\
-1.18208575929697	0.593075264744444\\
-1.18944113039835	0.595484336841729\\
-1.19565527927026	0.597966914051153\\
-1.20097426544368	0.600494649340231\\
-1.20557934247511	0.603053545598106\\
-1.20960630798951	0.605638849748727\\
-1.21315835534862	0.608252001222227\\
-1.2163147887704	0.610898815342025\\
-1.21913703504296	0.613588435376933\\
-1.22167284234597	0.616332784416996\\
-1.22395923010075	0.619146362645509\\
-1.22602455119113	0.622046304964338\\
-1.22788989784863	0.625052658896365\\
-1.2295699951167	0.628188875042512\\
-1.23107366292106	0.631482529497297\\
-1.23240387740984	0.634966324917864\\
-1.23355741534321	0.638679449344891\\
-1.23452401346765	0.642669415179332\\
-1.23528490798758	0.646994563037611\\
-1.23581052306342	0.651727509156087\\
-1.23605692848604	0.656959961411399\\
-1.23596044514103	0.662809563576559\\
-1.23542936997175	0.669429813383256\\
-1.23433108371332	0.677024752794653\\
-1.23247153021665	0.685871266354724\\
-1.22956168278891	0.696353870420598\\
-1.22516102360426	0.709020692660253\\
-1.21857881304302	0.724676739372734\\
-1.2086944178179	0.744545507733446\\
-1.19361473557335	0.770561635854139\\
-1.16998584185419	0.805927250653545\\
-1.13152846224427	0.856225279395141\\
-1.06573985695198	0.931754081541382\\
-0.946173527310501	1.05251473853723\\
-0.715145074514351	1.25784691099334\\
-0.25915963680298	1.61407245297346\\
0.533529697525656	2.15749204989669\\
1.47643270260674	2.72280409766538\\
2.17825862936543	3.08704498818123\\
2.57979469508172	3.26437318172905\\
2.80048557324685	3.34548951140256\\
2.92812298873437	3.38335485119594\\
3.00711025420764	3.40139640666682\\
3.05911603477277	3.40983029938284\\
3.09519171166257	3.41334200179773\\
3.12132034355964	3.41421356237309\\
};

\end{axis}

\begin{axis}[%
width=0.798\linewidth,
height=0.578\linewidth,
at={(-0.104\linewidth,-0.064\linewidth)},
scale only axis,
xmin=0,
xmax=1,
ymin=0,
ymax=1,
axis line style={draw=none},
ticks=none,
axis x line*=bottom,
axis y line*=left,
legend style={legend cell align=left, align=left, draw=white!15!black}
]
\end{axis}
\end{tikzpicture}%
        \caption{Эллипсоид с центром $q = \protect\begin{bmatrix}1\\2\protect\end{bmatrix}$ и матрицей $Q = \protect\begin{bmatrix}5&3\\3&2\protect\end{bmatrix}.$}
\end{figure}
\begin{figure}[b]
        \centering
        % This file was created by matlab2tikz.
%
%The latest updates can be retrieved from
%  http://www.mathworks.com/matlabcentral/fileexchange/22022-matlab2tikz-matlab2tikz
%where you can also make suggestions and rate matlab2tikz.
%
\definecolor{mycolor1}{rgb}{0.00000,0.44700,0.74100}%
\definecolor{mycolor2}{rgb}{0.85000,0.32500,0.09800}%
%
\begin{tikzpicture}

\begin{axis}[%
width=0.618\linewidth,
height=0.487\linewidth,
at={(0\linewidth,0\linewidth)},
scale only axis,
xmin=-4,
xmax=4,
xlabel style={font=\color{white!15!black}},
xlabel={$x_1$},
ymin=-3,
ymax=3,
ylabel style={font=\color{white!15!black}},
ylabel={$x_2$},
axis background/.style={fill=white},
axis x line*=bottom,
axis y line*=left,
xmajorgrids,
ymajorgrids,
legend style={at={(0.03,0.97)}, anchor=north west, legend cell align=left, align=left, draw=white!15!black}
]
\addplot [color=mycolor1]
  table[row sep=crcr]{%
0	1.41421356237309\\
0.0448926526261715	1.41278777581078\\
0.0898751836254369	1.40849029202781\\
0.135035408616089	1.40126046002867\\
0.180456834323975	1.39099628392442\\
0.226216037819367	1.37755312364591\\
0.272379465039817	1.36074202332744\\
0.31899942276683	1.3403278466662\\
0.366109017608597	1.31602749760457\\
0.413715781186158	1.28750864260985\\
0.461793724526318	1.25438953757444\\
0.510273605374737	1.21624080482269\\
0.559031297446442	1.17259030225851\\
0.607874358269208	1.12293255768884\\
0.656527248025466	1.06674455480223\\
0.704616200693154	1.00350985019654\\
0.751655514474458	0.932752901426886\\
0.797037975951299	0.854084849287771\\
0.840033114401921	0.767260538159248\\
0.879797685207568	0.672244052563361\\
0.915402708139832	0.569276526707822\\
0.945879950278728	0.458935986082397\\
0.970287525247814	0.342175739492072\\
0.987789436744396	0.22032715972476\\
0.997738518429436	0.0950562869276414\\
0.999748302867315	-0.0317279345033261\\
0.993739043738288	-0.15800451227805\\
0.979947497005068	-0.281790358648065\\
0.958898165695011	-0.401283709678683\\
0.931342671496431	-0.514977335908855\\
0.898180416909459	-0.62172652940075\\
0.860375718733436	-0.720768510152771\\
0.818884246741863	-0.81170017917703\\
0.774596669241484	-0.894427190999916\\
0.728302096399996	-0.969098608377261\\
0.68066980893543	-1.03603919926208\\
0.632245459597382	-1.09568761863817\\
0.583457251434522	-1.14854484958009\\
0.534627983128064	-1.19513423485098\\
0.485989745067013	-1.23597246546167\\
0.437699042301328	-1.2715498797676\\
0.38985098707616	-1.30231809315216\\
0.342491860563771	-1.32868305133133\\
0.295629790778837	-1.35100187031999\\
0.249243569363955	-1.36958215754347\\
0.203289781078724	-1.3846828264328\\
0.157708488746422	-1.39651568740012\\
0.112427735812955	-1.40524731219809\\
0.0673671215351546	-1.41100082986231\\
0.0224406851852784	-1.41385742962182\\
-0.0224406851852783	-1.41385742962182\\
-0.0673671215351544	-1.41100082986231\\
-0.112427735812955	-1.40524731219809\\
-0.157708488746422	-1.39651568740012\\
-0.203289781078724	-1.3846828264328\\
-0.249243569363954	-1.36958215754347\\
-0.295629790778837	-1.35100187031999\\
-0.342491860563771	-1.32868305133133\\
-0.389850987076159	-1.30231809315216\\
-0.437699042301328	-1.2715498797676\\
-0.485989745067013	-1.23597246546167\\
-0.534627983128064	-1.19513423485098\\
-0.583457251434522	-1.14854484958009\\
-0.632245459597382	-1.09568761863817\\
-0.68066980893543	-1.03603919926208\\
-0.728302096399996	-0.969098608377261\\
-0.774596669241483	-0.894427190999916\\
-0.818884246741864	-0.811700179177029\\
-0.860375718733436	-0.720768510152772\\
-0.898180416909458	-0.621726529400751\\
-0.931342671496431	-0.514977335908856\\
-0.958898165695011	-0.401283709678683\\
-0.979947497005068	-0.281790358648066\\
-0.993739043738288	-0.15800451227805\\
-0.999748302867315	-0.0317279345033253\\
-0.997738518429436	0.0950562869276414\\
-0.987789436744396	0.22032715972476\\
-0.970287525247814	0.34217573949207\\
-0.945879950278728	0.458935986082396\\
-0.915402708139832	0.569276526707822\\
-0.879797685207568	0.672244052563361\\
-0.840033114401921	0.767260538159247\\
-0.797037975951299	0.85408484928777\\
-0.751655514474458	0.932752901426886\\
-0.704616200693154	1.00350985019654\\
-0.656527248025466	1.06674455480223\\
-0.607874358269208	1.12293255768884\\
-0.559031297446443	1.17259030225851\\
-0.510273605374738	1.21624080482269\\
-0.461793724526318	1.25438953757444\\
-0.413715781186158	1.28750864260985\\
-0.366109017608597	1.31602749760457\\
-0.318999422766831	1.3403278466662\\
-0.272379465039818	1.36074202332744\\
-0.226216037819367	1.37755312364591\\
-0.180456834323975	1.39099628392442\\
-0.135035408616089	1.40126046002867\\
-0.0898751836254374	1.40849029202781\\
-0.0448926526261721	1.41278777581078\\
-1.73191211247099e-16	1.41421356237309\\
};
\addlegendentry{Эллипсоиды}

\addplot [color=mycolor1, forget plot]
  table[row sep=crcr]{%
2.12132034355964	1.41421356237309\\
2.14093309737706	1.41361883293874\\
2.15610096472476	1.41218966892045\\
2.16813059909176	1.41027848839136\\
2.17787846987421	1.40808691034947\\
2.18592472730893	1.40573192203561\\
2.19267367435337	1.40328137559688\\
2.19841408644868	1.40077372446074\\
2.20335666517915	1.39822932722739\\
2.20765799352175	1.39565707404995\\
2.21143626106343	1.3930583315506\\
2.2147818220386	1.39042930139934\\
2.217764420063	1.38776240921808\\
2.22043820622192	1.38504707754847\\
2.22284525774146	1.38227008706717\\
2.2250180482594	1.37941564172179\\
2.22698115917429	1.37646519817436\\
2.22875241567428	1.37339708240557\\
2.23034355770318	1.37018588734886\\
2.23176050074785	1.36680161882979\\
2.23300319375062	1.36320852788002\\
2.23406503318587	1.35936353046923\\
2.23493173445265	1.35521406411806\\
2.2355794824484	1.35069515469921\\
2.23597206441911	1.34572534988779\\
2.23605649958035	1.34020099103578\\
2.23575636783317	1.33398799537805\\
2.23496150515886	1.32690982030174\\
2.23351178703941	1.31872942300207\\
2.23117098985666	1.3091215094024\\
2.22758343607311	1.29762858411142\\
2.22219964903172	1.28358902636704\\
2.21414389031876	1.26601495419557\\
2.20196759721768	1.24337600812861\\
2.18316711789886	1.21319839168308\\
2.15318699112585	1.17128288128172\\
2.10323675000489	1.11010066257133\\
2.01525609198129	1.01537616591488\\
1.85016840053085	0.858953522737536\\
1.52458200959445	0.587600939918628\\
0.904401375843364	0.133639204415214\\
-0.0144986659096387	-0.455903394011212\\
-0.869940328784187	-0.933944964994194\\
-1.40928826856013	-1.1927850908838\\
-1.70597634609565	-1.31269787634786\\
-1.87241940668093	-1.36791395729915\\
-1.97199649313969	-1.39402766902269\\
-2.03563236326191	-1.40644166495969\\
-2.07869104959228	-1.41202282376828\\
-2.10924316645516	-1.4140172122216\\
-2.13178875034531	-1.41405128159025\\
-2.14897766153775	-1.41298190592938\\
-2.1624464899118	-1.41127917252142\\
-2.17324845072647	-1.40920897190173\\
-2.18208575929697	-1.40692473525556\\
-2.18944113039835	-1.40451566315827\\
-2.19565527927026	-1.40203308594885\\
-2.20097426544368	-1.39950535065977\\
-2.20557934247511	-1.39694645440189\\
-2.20960630798951	-1.39436115025127\\
-2.21315835534862	-1.39174799877777\\
-2.2163147887704	-1.38910118465798\\
-2.21913703504296	-1.38641156462307\\
-2.22167284234597	-1.383667215583\\
-2.22395923010075	-1.38085363735449\\
-2.22602455119113	-1.37795369503566\\
-2.22788989784863	-1.37494734110363\\
-2.2295699951167	-1.37181112495749\\
-2.23107366292106	-1.3685174705027\\
-2.23240387740984	-1.36503367508214\\
-2.23355741534321	-1.36132055065511\\
-2.23452401346765	-1.35733058482067\\
-2.23528490798758	-1.35300543696239\\
-2.23581052306342	-1.34827249084391\\
-2.23605692848604	-1.3430400385886\\
-2.23596044514103	-1.33719043642344\\
-2.23542936997175	-1.33057018661674\\
-2.23433108371332	-1.32297524720535\\
-2.23247153021665	-1.31412873364528\\
-2.22956168278891	-1.3036461295794\\
-2.22516102360426	-1.29097930733975\\
-2.21857881304302	-1.27532326062727\\
-2.2086944178179	-1.25545449226655\\
-2.19361473557335	-1.22943836414586\\
-2.16998584185419	-1.19407274934646\\
-2.13152846224427	-1.14377472060486\\
-2.06573985695198	-1.06824591845862\\
-1.9461735273105	-0.947485261462767\\
-1.71514507451435	-0.742153089006664\\
-1.25915963680298	-0.385927547026545\\
-0.466470302474344	0.15749204989669\\
0.476432702606736	0.72280409766538\\
1.17825862936543	1.08704498818123\\
1.57979469508172	1.26437318172905\\
1.80048557324685	1.34548951140256\\
1.92812298873437	1.38335485119594\\
2.00711025420764	1.40139640666682\\
2.05911603477277	1.40983029938284\\
2.09519171166257	1.41334200179773\\
2.12132034355964	1.4142135623731\\
};
\addplot [color=mycolor2]
  table[row sep=crcr]{%
2.12132034355964	2.82842712474619\\
2.18582575000323	2.82640660874952\\
2.2459761483502	2.82067996094826\\
2.30316600770785	2.81153894842004\\
2.35833530419819	2.79908319427389\\
2.4121407651283	2.78328504568152\\
2.46505313939319	2.76402339892432\\
2.51741350921551	2.74110157112695\\
2.56946568278775	2.71425682483196\\
2.62137377470791	2.68316571665979\\
2.67322998558975	2.64744786912504\\
2.72505542741334	2.60667010622203\\
2.77679571750944	2.56035271147659\\
2.82831256449113	2.50797963523731\\
2.87937250576692	2.44901464186941\\
2.92963424895255	2.38292549191834\\
2.97863667364875	2.30921809960124\\
3.02579039162557	2.22748193169334\\
3.0703766721051	2.13744642550811\\
3.11155818595542	2.03904567139316\\
3.14840590189045	1.93248505458784\\
3.1799449834646	1.81829951655162\\
3.20521925970046	1.69738980361013\\
3.2233689191928	1.57102231442397\\
3.23371058284854	1.44078163681543\\
3.23580480244767	1.30847305653245\\
3.22949541157146	1.1759834831\\
3.21490900216393	1.04511946165368\\
3.19240995273442	0.917445713323391\\
3.16251366135309	0.794144173493542\\
3.12576385298257	0.675902054710668\\
3.08257536776515	0.562820516214273\\
3.03302813706062	0.454314775018543\\
2.97656426645917	0.348948817128694\\
2.91146921429886	0.244099783305823\\
2.83385680006128	0.135243682019644\\
2.73548220960228	0.0144130439331602\\
2.59871334341581	-0.133168683665202\\
2.38479638365891	-0.336180712113444\\
2.01057175466146	-0.648371525543042\\
1.34210041814469	-1.13791067535239\\
0.375352321166521	-1.75822148716338\\
-0.527448468220416	-2.26262801632552\\
-1.11365847778129	-2.5437869612038\\
-1.4567327767317	-2.68228003389134\\
-1.6691296256022	-2.75259678373195\\
-1.81428800439327	-2.79054335642282\\
-1.92320462744895	-2.81168897715777\\
-2.01132392805712	-2.82302365363059\\
-2.08680248126988	-2.82787464184343\\
-2.15422943553058	-2.82790871121207\\
-2.21634478307291	-2.8239827357917\\
-2.27487422572476	-2.8165264847195\\
-2.33095693947289	-2.80572465930185\\
-2.3853755403757	-2.79160756168836\\
-2.43868469976231	-2.77409782070174\\
-2.49128507004909	-2.75303495626884\\
-2.54346612600745	-2.7281884019911\\
-2.59543032955127	-2.69926454755406\\
-2.64730535029083	-2.66591103001887\\
-2.69914810041563	-2.62772046423944\\
-2.75094277189846	-2.58423541950896\\
-2.80259428647749	-2.53495641420315\\
-2.85391830194335	-2.47935483422118\\
-2.90462903903618	-2.41689283661657\\
-2.95432664759112	-2.34705230341292\\
-3.00248656709011	-2.26937453210355\\
-3.04845424185856	-2.18351130413452\\
-3.0914493816545	-2.08928598065547\\
-3.13058429431929	-1.98676020448289\\
-3.16490008683964	-1.87629788656397\\
-3.19342217916266	-1.75861429449935\\
-3.21523240499265	-1.63479579561046\\
-3.22954956680171	-1.50627700312196\\
-3.23580523135335	-1.37476797309193\\
-3.23369896357047	-1.2421341494958\\
-3.22321880671614	-1.11024302689198\\
-3.20461860896113	-0.980799507713277\\
-3.17835148049538	-0.85519274756288\\
-3.14496439092874	-0.734369602871579\\
-3.10495870881183	-0.618735254776386\\
-3.05861192744495	-0.508062722468019\\
-3.0057323937692	-0.401369642978784\\
-2.94527025004781	-0.296685462718974\\
-2.87460204254734	-0.190562899149911\\
-2.78805571026974	-0.0770301658026245\\
-2.67361421522119	0.0546866392302208\\
-2.50520482475694	0.225105040795738\\
-2.22541867988909	0.474087715816021\\
-1.7209533613293	0.868461990547897\\
-0.880186083660502	1.44500069250654\\
0.110323684998138	2.03883159526995\\
0.859259206598599	2.42737283484743\\
1.30741523004191	2.62511520505649\\
1.57426953542748	2.72304263504847\\
1.74766615441039	2.77435113512036\\
1.87207484559155	2.80265686669549\\
1.96924085114733	2.81832059141065\\
2.0502990590364	2.82612977760851\\
2.12132034355964	2.82842712474619\\
};
\addlegendentry{Сумма Минковского}

\end{axis}

\begin{axis}[%
width=0.798\linewidth,
height=0.597\linewidth,
at={(-0.104\linewidth,-0.066\linewidth)},
scale only axis,
xmin=0,
xmax=1,
ymin=0,
ymax=1,
axis line style={draw=none},
ticks=none,
axis x line*=bottom,
axis y line*=left,
legend style={legend cell align=left, align=left, draw=white!15!black}
]
\end{axis}
\end{tikzpicture}%
        \caption{Сумма двух эллипсоидов.}
\end{figure}

%%%%%%%%%%%%%%%%%%%%%%%%%%%%%%%%%%%%%%%%%%%%%%%%%%%%%%%%%%%%%%%%%%%%%%%%%%%%%%%%
\clearpage
\section{Внешняя оценка суммы эллипсоидов}

\begin{theorem}
        Для суммы Минковского эллипсоидов справедлива следующая внешняя оценка
$$
        \sum\limits_{i=1}^{n} \Varepsilon(q_i,\,Q_i)
        =
        \bigcap\limits_{\| l \| = 1} \Varepsilon(q_+(l),\,Q_+(l)),
$$
где
$$
        \begin{aligned}
q_+(l) &= \sum_{i=1}^{n} q_i,
\\
Q_+(l) &= \sum_{i=1}^n p_i \cdot \sum_{i=1}^{n} \frac{Q_i}{p_i},
\quad
\mbox{где }
p_i = \langle l,\,Q_i l \rangle^{\nicefrac12}.
        \end{aligned}
$$
\end{theorem}

\begin{proof}

Будем доказывать для случая $q_i = 0,$ $i = \overline{1,\,n}$. Случай с произвольными центрами~--- аналогично.

Распишем квадрат опорной функции эллипсоида $\Varepsilon(0,\,Q_+(l))$:
\begin{multline*}
        \rho^2(l\,|\,\Varepsilon(0,\,Q_+(l)))
        =
        \sum_{i=1}^{n}
        \langle
        l,\,Q_il
        \rangle
        +
        \sum_{i < j}
        \left\langle
l,\,\left(
\frac{p_i}{p_j}Q_j + \frac{p_j}{p_i}Q_i
\right)l
        \right\rangle
        \geqslant\\\geqslant
        \left\{
\frac{a+b}{2} \geqslant \sqrt{ab}
        \right\}
        \geqslant
        \sum_{i=1}^{n}\langle l,\,Ql \rangle
        +
        2\sum_{i < j}
        \langle l,\,Q_il \rangle^{\nicefrac12}
        \langle l,\,Q_jl \rangle^{\nicefrac12}
        =\\=
        \left(
\sum_{i=1}^n\langle l,\,Q_il\rangle^{\nicefrac12}
        \right)^2
        =
        \rho^2\left(
l\left|
        \sum_{i=1}^n\Varepsilon(0, Q_i)
\right.
        \right).
\end{multline*}
Таким образом, получили, что для любого $l \neq 0$
$$
        \sum_{i=1}^n \Varepsilon(0,\,Q_i) \subseteq \Varepsilon(0,\,Q_+(l)),
$$
причем, так как равенство опорных функций достигается при
$
        p_i = \langle l,\,Q_i l \rangle^{\nicefrac12},
$
то в направлении $l \neq 0$ эллипсоид $\Varepsilon(0,\,Q_+)$ касается суммы $\sum_{i = 0}^{n} \Varepsilon(0,\,Q_i).$

\end{proof}

\clearpage
\begin{figure}[t]
        \centering
        % This file was created by matlab2tikz.
%
%The latest updates can be retrieved from
%  http://www.mathworks.com/matlabcentral/fileexchange/22022-matlab2tikz-matlab2tikz
%where you can also make suggestions and rate matlab2tikz.
%
\definecolor{mycolor1}{rgb}{0.00000,0.44700,0.74100}%
\definecolor{mycolor2}{rgb}{0.85000,0.32500,0.09800}%
%
\begin{tikzpicture}

\begin{axis}[%
width=0.618\linewidth,
height=0.471\linewidth,
at={(0\linewidth,0\linewidth)},
scale only axis,
xmin=-8,
xmax=8,
xlabel style={font=\color{white!15!black}},
xlabel={$x_1$},
ymin=-5,
ymax=5,
ylabel style={font=\color{white!15!black}},
ylabel={$x_2$},
axis background/.style={fill=white},
axis x line*=bottom,
axis y line*=left,
xmajorgrids,
ymajorgrids,
legend style={at={(0.03,0.97)}, anchor=north west, legend cell align=left, align=left, draw=white!15!black}
]
\addplot [color=mycolor1, forget plot]
  table[row sep=crcr]{%
1.1711787112841	3.34230777773576\\
1.34444389618673	3.33686722689733\\
1.50648318534277	3.32144701348819\\
1.65806242150429	3.29723890999844\\
1.79995419924852	3.26523166869947\\
1.93290799118999	3.22622794766654\\
2.05763032437368	3.18086276334402\\
2.17477223041259	3.12962150018857\\
2.28492171716205	3.07285634836238\\
2.3885995032216	3.01080060567632\\
2.48625667387058	2.94358064140543\\
2.57827324651315	2.87122553999274\\
2.66495688065443	2.79367457178694\\
2.74654114431786	2.71078271324095\\
2.82318287015103	2.62232448808267\\
2.8949582146523	2.52799644473262\\
2.96185708693385	2.42741864049226\\
3.02377565347494	2.3201355847161\\
3.0805066685131	2.20561721602438\\
3.13172744617751	2.08326066796132\\
3.17698540702193	1.95239382914097\\
3.21568133490354	1.81228204209558\\
3.24705081945703	1.66213971753183\\
3.27014489780722	1.50114915910447\\
3.28381172002077	1.32848945666784\\
3.28668221628109	1.14337881243048\\
3.2771642749897	0.945133918403077\\
3.25345178932588	0.733249674250213\\
3.21355684151227	0.507501133889645\\
3.15537469570401	0.268066487797808\\
3.07679115926015	0.015664562009573\\
2.97583883725379	-0.248307401701393\\
2.85090136682411	-0.521659161442773\\
2.70095216778915	-0.801347823435408\\
2.52579793500058	-1.08350783899399\\
2.32628156602984	-1.3635980584876\\
2.10439191407115	-1.63667268179468\\
1.86323617293213	-1.89775167737354\\
1.60685736899554	-2.14223401489281\\
1.33991789027605	-2.3662777508081\\
1.06730585933085	-2.56707514274059\\
0.793739425298697	-2.74297851870391\\
0.523437853608312	-2.89347215049445\\
0.259902596754298	-3.0190207516634\\
0.00581903236219554	-3.1208447467499\\
-0.2369372555121	-3.20067372983419\\
-0.467220354507	-3.2605178574641\\
-0.684493265340969	-3.30248032509504\\
-0.888694234930048	-3.32861897022119\\
-1.0801092196493	-3.34085455680139\\
-1.25926106809304	-3.34091776153004\\
-1.42681950330378	-3.33032515794113\\
-1.58353177449541	-3.31037507736533\\
-1.73017158467123	-3.2821558685242\\
-1.86750300244094	-3.24656096722397\\
-1.99625599202063	-3.20430689704777\\
-2.11711054093165	-3.15595168681459\\
-2.23068687011842	-3.10191219445658\\
-2.33753972875556	-3.04247952074268\\
-2.43815523449964	-2.97783215102365\\
-2.5329490934295	-2.90804674702345\\
-2.6222653208492	-2.83310668006404\\
-2.70637479428206	-2.75290849531896\\
-2.78547311726505	-2.66726655590999\\
-2.85967737154948	-2.5759161597878\\
-2.92902140016315	-2.47851546990414\\
-2.99344930844609	-2.37464666482809\\
-3.05280690958282	-2.26381681735669\\
-3.10683089311493	-2.14545915779927\\
-3.15513558252945	-2.01893559215682\\
-3.19719730262603	-1.88354163878489\\
-3.23233664163192	-1.73851533219976\\
-3.25969932318791	-1.58305212075492\\
-3.27823706849968	-1.41632833369867\\
-3.28669080350863	-1.23753634540448\\
-3.28357990805066	-1.04593497683649\\
-3.26720291066679	-0.840918687724783\\
-3.23565696131938	-0.622108318461704\\
-3.18688516768166	-0.389463978954261\\
-3.11876167995196	-0.143416533926881\\
-3.02922302993614	0.11499241610004\\
-2.91644915653916	0.383980836121862\\
-2.77908754849337	0.660934801326638\\
-2.61649915361415	0.942389492277687\\
-2.42898799999709	1.22411526429499\\
-2.21796402213223	1.50132757866017\\
-1.9859883878281	1.76901289011117\\
-1.7366683581316	2.02232922526914\\
-1.47440268220478	2.25701269019761\\
-1.20401775520347	2.46971278707818\\
-0.930363383198716	2.65819606075749\\
-0.657943143023997	2.82139307813531\\
-0.390637102663735	2.95930311865273\\
-0.131543811712393	3.07279931782965\\
0.117062449314586	3.1633871238911\\
0.353684227850852	3.23296276733506\\
0.577496523037587	3.28360333410188\\
0.788216624026908	3.3174035782658\\
0.985971690059736	3.33636162404406\\
1.1711787112841	3.34230777773576\\
};
\addplot [color=mycolor1, forget plot]
  table[row sep=crcr]{%
2.17430603473841	2.82928793073052\\
2.33044535777815	2.82444897168392\\
2.46594297362775	2.81160678704818\\
2.58415716708103	2.79277030050891\\
2.68789254387122	2.76940595677011\\
2.77946699484521	2.7425714670914\\
2.86078573906653	2.71301902411803\\
2.93341104171975	2.68127265050854\\
2.9986230753596	2.64768508851882\\
3.05747086442469	2.61247907686435\\
3.11081384046105	2.57577692251115\\
3.15935513715665	2.53762134677768\\
3.20366787001532	2.49798979992626\\
3.24421554049886	2.45680381827628\\
3.28136751247959	2.41393452310087\\
3.31541029254594	2.36920500026929\\
3.3465551303234	2.32239002374647\\
3.37494224630124	2.27321336944935\\
3.40064178827692	2.22134278883848\\
3.42365140325324	2.16638255933344\\
3.44389007534897	2.10786339134407\\
3.46118760463784	2.04522934489396\\
3.47526876652081	1.97782129527289\\
3.48573077423729	1.90485640152273\\
3.49201214853071	1.82540300920684\\
3.49335047065297	1.73835053234855\\
3.48872578597209	1.64237424725429\\
3.47678575017332	1.53589584679112\\
3.45574827300754	1.41704249382237\\
3.42327810810487	1.28361072542021\\
3.37633699194203	1.13304806285973\\
3.31101526097114	0.962476172308744\\
3.22237097655495	0.768796474817825\\
3.10433695290387	0.548942043211334\\
2.9498123210435	0.300361978709517\\
2.75112793000477	0.0218247814546395\\
2.50112454582504	-0.285440449235098\\
2.19500744060268	-0.616443644580409\\
1.83279348396173	-0.961489600474761\\
1.42152136958137	-1.3063970008595\\
0.975845648074582	-1.63450693799913\\
0.515964639544959	-1.93017020234628\\
0.0633211215265111	-2.18224450162264\\
-0.363949894265142	-2.38592186002566\\
-0.753796552421497	-2.54230740615597\\
-1.10060884720806	-2.65651208684665\\
-1.40389862561362	-2.73547347665808\\
-1.66643431787696	-2.78630322905355\\
-1.89260401811819	-2.81535915112738\\
-2.08728476369964	-2.82789001873507\\
-2.25518878499693	-2.82801992101541\\
-2.40055085490442	-2.81888822167606\\
-2.52702294235802	-2.80283518923491\\
-2.63767930270744	-2.78157977328078\\
-2.73507199115711	-2.75636932685319\\
-2.82130341222054	-2.72809764441867\\
-2.89809909341037	-2.69739456722559\\
-2.96687326596811	-2.66469244542933\\
-3.02878476984102	-2.63027467163603\\
-3.08478318424195	-2.5943106773748\\
-3.13564609696506	-2.55688082262324\\
-3.18200873955121	-2.51799374438073\\
-3.22438719765583	-2.47759802849779\\
-3.26319624623718	-2.43558952510388\\
-3.29876265010897	-2.39181521440426\\
-3.33133455291616	-2.34607421503113\\
-3.36108736562728	-2.29811628370925\\
-3.38812635908966	-2.24763796026885\\
-3.4124859564092	-2.1942763488638\\
-3.43412549761435	-2.13760038238197\\
-3.45292099547021	-2.0770992851559\\
-3.4686520985225	-2.01216782796055\\
-3.48098310405704	-1.94208786685168\\
-3.48943639793565	-1.86600559716315\\
-3.49335612455861	-1.7829039871982\\
-3.49185921454444	-1.69157008370894\\
-3.4837701826888	-1.59055749156644\\
-3.46753554642006	-1.47814566124369\\
-3.44111377096973	-1.35230025189567\\
-3.40183832761321	-1.21064372066799\\
-3.34625674726662	-1.05045381725873\\
-3.26996105481103	-0.868721532579536\\
-3.16745032441434	-0.662320310385892\\
-3.03211101926641	-0.428362470131523\\
-2.85646733273374	-0.164834179270465\\
-2.63292320715374	0.128425970177643\\
-2.35522203042063	0.448465239159584\\
-2.02065569486654	0.788006644772256\\
-1.63254200363741	1.13497995420302\\
-1.20180828192215	1.47359901929041\\
-0.746311741639337	1.78722909211069\\
-0.287482595077459	2.06211021974623\\
0.15439697822222	2.29018412674959\\
0.56404269287998	2.46977369323667\\
0.932710877487586	2.60426859699022\\
1.2575759170864	2.69994355321456\\
1.54000858568913	2.76398340384577\\
1.78376187044044	2.80319756768445\\
1.99357716381973	2.82340732367622\\
2.17430603473841	2.82928793073052\\
};
\addplot [color=mycolor1, forget plot]
  table[row sep=crcr]{%
6.81839938705513	4.98558555137348\\
6.9451056567848	4.98172857837679\\
7.04503269745632	4.97230404998413\\
7.12550382874078	4.95951351377152\\
7.19150742751882	4.9446702051191\\
7.24652621591938	4.92856445823852\\
7.29304670485787	4.91167081831979\\
7.33288010703728	4.89426840679286\\
7.36736941959477	4.87651242497647\\
7.39752608589151	4.8584773879902\\
7.42412204545158	4.84018356931731\\
7.44775286792895	4.82161320692916\\
7.46888172297768	4.802720283049\\
7.48787036298415	4.78343612628583\\
7.50500109471872	4.76367216612502\\
7.52049232706941	4.74332061039937\\
7.53450938011309	4.72225345843491\\
7.54717163259053	4.70032001509007\\
7.5585566510167	4.67734287814631\\
7.56870160579294	4.65311219540491\\
7.57760198044974	4.62737779738012\\
7.5852072695652	4.59983857492861\\
7.59141298466069	4.57012814868072\\
7.59604777339064	4.53779541139837\\
7.59885369623389	4.5022778262537\\
7.59945651723111	4.46286428877193\\
7.59732094276784	4.41864266581239\\
7.59168253759535	4.36842439739232\\
7.58144256176967	4.31063406338187\\
7.56500233907657	4.243144292049\\
7.5399964148077	4.16302350528493\\
7.50285171749803	4.0661415588715\\
7.44803942928042	3.9465388293111\\
7.36677017908548	3.79539498589725\\
7.24466050473369	3.59931675028097\\
7.0574886748104	3.33749314960869\\
6.76352025145179	2.97714793958209\\
6.29053702849613	2.46732264221061\\
5.51919290384651	1.7351657602429\\
4.28518881238173	0.704044860696217\\
2.47617548846006	-0.62399326502574\\
0.26472488379459	-2.04447711678708\\
-1.85854763216566	-3.22872807539079\\
-3.51179701572037	-4.01952314056496\\
-4.65672460540954	-4.48084729701715\\
-5.42192597149296	-4.73406098849309\\
-5.93815571883776	-4.86916023516302\\
-6.29610684807415	-4.93885917516955\\
-6.55240778900606	-4.97201605639051\\
-6.74175384000817	-4.98434219551015\\
-6.88568139038962	-4.98454045434875\\
-6.99789141017824	-4.97754800336823\\
-7.08734546190676	-4.96623193256517\\
-7.16006824322224	-4.95228978176482\\
-7.22021562299617	-4.93673979314651\\
-7.27072257215752	-4.92019511301591\\
-7.31370624925992	-4.90302121434563\\
-7.3507230387261	-4.88542843600515\\
-7.3829363263936	-4.86752747846997\\
-7.41122838809586	-4.84936318429306\\
-7.43627646256458	-4.8309352520156\\
-7.45860534888868	-4.81221087003346\\
-7.47862427236261	-4.79313219481558\\
-7.49665296586647	-4.77362040406707\\
-7.51294017121452	-4.75357734217863\\
-7.52767665013585	-4.73288533123189\\
-7.54100405805473	-4.71140542676794\\
-7.55302052460794	-4.68897418296435\\
-7.56378340709999	-4.66539881155098\\
-7.57330937097375	-4.64045043926517\\
-7.58157165270455	-4.61385495942758\\
-7.58849402465152	-4.58528069952081\\
-7.5939405470587	-4.55432174128212\\
-7.59769957191291	-4.52047516178699\\
-7.59945951764889	-4.48310960021399\\
-7.59877242863057	-4.44142120888268\\
-7.59499885959066	-4.39437090303387\\
-7.58722344586879	-4.34059333582325\\
-7.5741232741605	-4.27826223117822\\
-7.55375828026879	-4.20488688476054\\
-7.52322939077674	-4.11699767437352\\
-7.4781062033147	-4.00964867067367\\
-7.4114422958755	-3.87561300911386\\
-7.31203520475341	-3.70405586877619\\
-7.16128309678768	-3.47832412658061\\
-6.9274595685231	-3.17231678512378\\
-6.55558188850301	-2.7449831897496\\
-5.95177163129806	-2.13426805552596\\
-4.97054626054604	-1.26027984798751\\
-3.45135195503113	-0.0699980423053237\\
-1.39392514759134	1.34380755585939\\
0.840019649925826	2.68258056595531\\
2.75314581466553	3.67259137928405\\
4.14099226701179	4.28347990938291\\
5.07776315797388	4.62683679623043\\
5.70454533578041	4.81235986820016\\
6.13273164742407	4.90997324701889\\
6.4344036995944	4.95880562065331\\
6.65387188453432	4.98012287766274\\
6.81839938705513	4.98558555137348\\
};
\addplot [color=mycolor1, forget plot]
  table[row sep=crcr]{%
1.42632528730357	3.05419055216814\\
1.59409885697327	3.04894192904819\\
1.74764607607782	3.03434727439337\\
1.88831768585849	3.01189675254528\\
2.01740026430351	2.98279289019926\\
2.13608416558777	2.94798796969752\\
2.24544811665552	2.90822043063386\\
2.34645448710049	2.86404765904994\\
2.43995094101877	2.81587408211711\\
2.52667551525986	2.76397436887494\\
2.60726315074524	2.7085119950182\\
2.68225238936669	2.64955362911531\\
2.75209140658278	2.58707985082535\\
2.81714284020386	2.52099268940941\\
2.8776870472672	2.45112041621127\\
2.93392350905193	2.37721996369354\\
2.98597013463019	2.29897729217642\\
3.03386020403657	2.2160059952028\\
3.07753665694582	2.12784443619997\\
3.11684338412892	2.0339517557063\\
3.15151313143606	1.93370319745457\\
3.18115160055983	1.82638539774865\\
3.20521736059578	1.71119259976722\\
3.22299732316753	1.58722523687895\\
3.23357786670449	1.45349302920453\\
3.23581235158743	1.3089257071312\\
3.2282869316483	1.15239574159971\\
3.20928847715888	0.982758977142026\\
3.176781336107	0.798920617843454\\
3.12840373601196	0.59993508496799\\
3.06149968552777	0.385147822534311\\
2.97320730405123	0.154383494569105\\
2.86062722894035	-0.0918240253086883\\
2.72109091958337	-0.351980889021321\\
2.55253274510375	-0.623406293831111\\
2.3539375047955	-0.902099020098742\\
2.12578916758129	-1.1827809605491\\
1.87040277410249	-1.45918525535049\\
1.59200788692453	-1.72459900904301\\
1.29649516293625	-1.97258135199058\\
0.990837210384203	-2.197696081303\\
0.682309937938043	-2.39607365215453\\
0.377709629271056	-2.56567186636937\\
0.0827467276445912	-2.70621088488075\\
-0.198285197101545	-2.81885870156143\\
-0.462566309044838	-2.90579312858928\\
-0.70860369444456	-2.96975905169869\\
-0.935974346036652	-3.01369849093327\\
-1.14503867672096	-3.04048466100225\\
-1.33667724745256	-3.05275737054036\\
-1.51207595388432	-3.05283983637206\\
-1.67256535682076	-3.04271255194964\\
-1.81950894294363	-3.02402253127754\\
-1.95423070636569	-2.99811154608017\\
-2.07797191745059	-2.96605230356283\\
-2.19186825631256	-2.92868580072754\\
-2.29694038051205	-2.8866561327032\\
-2.3940928301155	-2.84044100242064\\
-2.48411769306366	-2.7903773569584\\
-2.56770060761846	-2.73668222109411\\
-2.64542750422993	-2.67946910987757\\
-2.71779105258215	-2.61876051726236\\
-2.7851961466965	-2.55449698645495\\
-2.84796398665464	-2.48654322508971\\
-2.9063344415271	-2.4146916680026\\
-2.96046643468488	-2.33866383248366\\
-3.01043610126227	-2.25810976873771\\
-3.05623244371903	-2.1726058926298\\
-3.09775016794598	-2.08165151011889\\
-3.13477933218787	-1.98466441800741\\
-3.16699140132554	-1.88097611459704\\
-3.19392129598201	-1.76982740563066\\
-3.21494510178343	-1.65036558457997\\
-3.22925332680441	-1.52164495079917\\
-3.23582007028983	-1.38263325815509\\
-3.2333693489984	-1.23222780416891\\
-3.22034133184322	-1.06928627425662\\
-3.19486361413493	-0.892679032711539\\
-3.15473614964176	-0.701370955594937\\
-3.09744308405396	-0.494541380754748\\
-3.02020998960036	-0.271748964553961\\
-2.92012930110728	-0.0331421889446313\\
-2.79437683989129	0.220296299061972\\
-2.64053313743929	0.486503595447298\\
-2.45699943943893	0.762146269718005\\
-2.24345818717511	1.04256505003492\\
-2.00128002875728	1.3219347994068\\
-1.73374676631364	1.59368386301881\\
-1.44597222977026	1.85114025281027\\
-1.14447704690877	2.08828044420864\\
-0.836487911591588	2.30039820704417\\
-0.529131242108844	2.48452621841593\\
-0.228720064349343	2.63952985558765\\
0.0597204337400604	2.76590334036294\\
0.332641313986094	2.86537701210694\\
0.58791360706013	2.94046425247022\\
0.824614911507097	2.99404850797217\\
1.0427491736426	3.02906396360623\\
1.24296707789219	3.04828205038319\\
1.42632528730357	3.05419055216814\\
};
\addplot [color=mycolor1, forget plot]
  table[row sep=crcr]{%
0.878355311988031	4.00104451440901\\
1.06350692625045	3.99520620111796\\
1.24105208545559	3.97828669634494\\
1.41132245401614	3.95107065481952\\
1.57466773303343	3.91420199851153\\
1.73143696912751	3.86819041177193\\
1.88196376005733	3.81341874894101\\
2.02655438296571	3.750150547048\\
2.16547798030495	3.67853714869157\\
2.29895805486653	3.59862418003358\\
2.42716463522509	3.51035731500992\\
2.55020657118865	3.41358740789436\\
2.668123506243	3.30807520909464\\
2.78087715546415	3.19349600831664\\
2.88834160146377	3.06944468768286\\
2.99029241959652	2.93544182533574\\
3.08639457260405	2.79094167459111\\
3.17618919386929	2.63534305689275\\
3.25907963118477	2.46800444241896\\
3.33431747555457	2.28826473051858\\
3.4009897767428	2.09547144282054\\
3.45800926415817	1.88901813486057\\
3.50411013946209	1.6683927090189\\
3.53785283151403	1.43323782151323\\
3.55764187603864	1.18342353411735\\
3.5617615690935	0.919130581462076\\
3.54843389392288	0.64093999291339\\
3.51590199238786	0.349921399335943\\
3.46253971391698	0.0477085835925056\\
3.38698331236318	-0.263452424287396\\
3.28827546184747	-0.580698823062855\\
3.16600547552711	-0.900604275852195\\
3.02042477853998	-1.21929774783176\\
2.85251553485022	-1.53264516950835\\
2.66399457594665	-1.83647819729149\\
2.45724457303916	-2.12684262395052\\
2.23517770285111	-2.4002325211065\\
2.00105011016354	-2.65377765196434\\
1.7582542402097	-2.88536099227669\\
1.51011809336617	-3.09365738536341\\
1.25973568625162	-3.27809894883646\\
1.00984379818778	-3.43878382182336\\
0.762749729660402	-3.57635010468598\\
0.520306129482393	-3.69183655722441\\
0.283923478606458	-3.78654748590043\\
0.0546086990229475	-3.86193343438586\\
-0.166981171541117	-3.91949359001469\\
-0.380479311033542	-3.96070128664519\\
-0.585749985503425	-3.986950972031\\
-0.782834069573656	-3.99952339235443\\
-0.971901342639129	-3.99956518582877\\
-1.15321025667837	-3.98807919312425\\
-1.32707494440725	-3.96592226831757\\
-1.49383870905093	-3.93380798674879\\
-1.65385299334717	-3.89231225949612\\
-1.80746076448412	-3.84188040852438\\
-1.95498329637778	-3.78283470631474\\
-2.09670942756051	-3.71538173904544\\
-2.23288648811713	-3.63961922624244\\
-2.36371220332813	-3.55554214043776\\
-2.48932698599055	-3.46304813674846\\
-2.60980612184367	-3.36194244219558\\
-2.7251514361164	-3.25194248386318\\
-2.83528211072562	-3.1326826674037\\
-2.9400244113109	-3.00371986447725\\
-3.03910019488404	-2.86454033832207\\
-3.13211422011443	-2.71456903567785\\
-3.21854049496589	-2.55318239915177\\
-3.29770819576308	-2.37972609464072\\
-3.3687881039304	-2.19353927459823\\
-3.43078105286486	-1.99398715573653\\
-3.4825105630904	-1.78050369093681\\
-3.52262264058648	-1.55264583001233\\
-3.54959653315705	-1.31016012225037\\
-3.56177090540339	-1.05306102369323\\
-3.55739011243071	-0.781718072012295\\
-3.53467462163135	-0.496946044577514\\
-3.49191769492277	-0.200088546569051\\
-3.42760684601089	0.106918179217524\\
-3.34056332736365	0.421516953800008\\
-3.23008662410292	0.74055253109619\\
-3.09608510202537	1.06035723112041\\
-2.93917069281702	1.37690232801292\\
-2.76069690826706	1.68600603833541\\
-2.56272661017962	1.98357610732913\\
-2.34792790075655	2.26585543441595\\
-2.11941020973985	2.52963648855715\\
-1.88052401217249	2.77241584827913\\
-1.63465314874106	2.9924724548048\\
-1.38502715711598	3.18886813413911\\
-1.13457364935907	3.3613821295424\\
-0.885820579981077	3.51039961632549\\
-0.640848456119384	3.63677651896027\\
-0.40128537454523	3.7417004733795\\
-0.16833402832988	3.82656253933673\\
0.0571808702239901	3.89284832757713\\
0.274754939025249	3.94205200602846\\
0.48414407094491	3.975612869128\\
0.685307253194799	3.99487187487964\\
0.87835531198803	4.00104451440901\\
};
\addplot [color=mycolor1, forget plot]
  table[row sep=crcr]{%
1.02375040531617	3.61290204042887\\
1.20176097639549	3.60730045239441\\
1.37036957674768	3.59124369131131\\
1.53008538748134	3.56572536946192\\
1.68143780878129	3.53157387649472\\
1.82495211979655	3.48946246279246\\
1.96113166891982	3.43992080417065\\
2.09044497609687	3.38334668835468\\
2.2133163581027	3.32001698205598\\
2.33011892490929	3.25009741061438\\
2.44116901027864	3.17365094361483\\
2.54672128027864	3.09064476183925\\
2.64696390782588	3.0009559112872\\
2.74201331386758	2.90437585191723\\
2.83190806411647	2.8006142012692\\
2.91660158441303	2.68930207223581\\
2.99595343022608	2.56999552386096\\
3.06971893197026	2.44217979639812\\
3.13753715736318	2.30527519718943\\
3.19891731008915	2.15864574879307\\
3.25322395208505	2.00161200452337\\
3.29966183225684	1.83346976493728\\
3.33726166727844	1.65351675400746\\
3.36486898352328	1.46108956040721\\
3.38113910113713	1.25561318766688\\
3.38454247282657	1.03666518752547\\
3.3733857285037	0.804055302209506\\
3.34585461031475	0.55791950056818\\
3.30008499547509	0.29882399643962\\
3.23426669333464	0.027870249673398\\
3.14678092088188	-0.253213474226809\\
3.03636585661394	-0.542013456154021\\
2.90229582008168	-0.835426867441589\\
2.74455015014256	-1.12973284012124\\
2.56394095281318	-1.42074875525535\\
2.36216843685569	-1.70406624141488\\
2.14178131502429	-1.97534198104143\\
1.90603727640003	-2.23060186778484\\
1.65868024595527	-2.46651004688063\\
1.40366973892964	-2.68056058438075\\
1.14490648466575	-2.87116725842797\\
0.885995072210844	-3.03764980487017\\
0.63007116427807	-3.18013500544318\\
0.379703604907356	-3.29940273110009\\
0.136866392078723	-3.3967091096898\\
-0.0970344061660403	-3.47361363503207\\
-0.3210974496125	-3.53182808856116\\
-0.534844239226391	-3.57309603977806\\
-0.738132419461264	-3.5991045027063\\
-0.931073626071126	-3.61142473381268\\
-1.11396114323217	-3.61147691260704\\
-1.28720951056091	-3.60051289518032\\
-1.45130614583367	-3.57961168548513\\
-1.60677385886146	-3.54968318739327\\
-1.75414258331887	-3.51147682884627\\
-1.89392851680262	-3.46559259474307\\
-2.02661895578985	-3.41249278629328\\
-2.15266131796345	-3.35251342660259\\
-2.27245508285067	-3.28587467397293\\
-2.38634561026136	-3.21268991747198\\
-2.49461899453625	-3.13297344723352\\
-2.59749727486108	-3.04664674493076\\
-2.69513344957367	-2.95354355339843\\
-2.78760584167458	-2.85341397942644\\
-2.87491144258746	-2.74592797770453\\
-2.95695793279452	-2.63067867162488\\
-3.0335541551112	-2.5071861012065\\
-3.10439891645093	-2.37490216114041\\
-3.16906813919749	-2.23321771157066\\
-3.22700060200421	-2.0814731142732\\
-3.2774828366435	-1.91897376091784\\
-3.31963422182456	-1.74501249358455\\
-3.35239397487088	-1.55890111606585\\
-3.37451261146532	-1.36001335729562\\
-3.38455150677088	-1.14784151232587\\
-3.38089535402736	-0.922068315058348\\
-3.36178335378977	-0.682654091339456\\
-3.32536546776439	-0.429936596356688\\
-3.26978941824833	-0.164736978685911\\
-3.1933215530012	0.111539791837043\\
-3.09449957081791	0.396827137223658\\
-2.97230730339444	0.688362960794408\\
-2.82635227231545	0.982721291257162\\
-2.65701807848947	1.27592551447855\\
-2.46555953737673	1.5636464007621\\
-2.25411243852087	1.84146980592806\\
-2.02560326847436	2.10520006333872\\
-1.78356466168654	2.35115257139234\\
-1.53188345629107	2.57638853279399\\
-1.27452259365677	2.77885734455166\\
-1.01526081426038	2.95743342231848\\
-0.757485202333998	3.11185660361474\\
-0.504055616196481	3.24260160764969\\
-0.257243124584251	3.35070882318175\\
-0.0187316848425002	3.43760661534103\\
0.210334443627698	3.50494770426823\\
0.429276150050858	3.55447278851852\\
0.637793727556865	3.58790627445286\\
0.835881553665213	3.6068830687125\\
1.02375040531617	3.61290204042887\\
};
\addplot [color=mycolor1, forget plot]
  table[row sep=crcr]{%
1.65642263207936	2.91541373390173\\
1.82037784583777	2.91030086390922\\
1.9677141607887	2.89631048846228\\
2.10039858637834	2.87514661502634\\
2.22021829502882	2.84814149164205\\
2.32876139289674	2.81631933318551\\
2.42741778412894	2.78045311346391\\
2.51739080453339	2.74111237919629\\
2.5997137273383	2.69870202037562\\
2.67526759617419	2.65349283915425\\
2.74479836712207	2.60564506995236\\
2.80893228619176	2.55522600914459\\
2.86818898383772	2.50222278100859\\
2.92299207190501	2.44655108756695\\
2.97367717316179	2.38806060833533\\
3.02049735848801	2.32653755200115\\
3.06362594816039	2.26170472357171\\
3.1031565727605	2.19321936030627\\
3.13910029784126	2.12066890980134\\
3.17137950106921	2.04356487823619\\
3.19981805563009	1.96133487528602\\
3.22412722545868	1.87331304194841\\
3.24388652917448	1.77872919811096\\
3.25851870767061	1.67669733555935\\
3.26725788924465	1.56620458170098\\
3.26911018723029	1.44610257455019\\
3.26280646739801	1.31510446242669\\
3.24674818526298	1.17179264275761\\
3.21894948375011	1.01464503971421\\
3.17698281835873	0.842091215225841\\
3.11794201723986	0.652613547633569\\
3.03844644930004	0.444911865172638\\
2.93472234736788	0.218149481119256\\
2.8028090465519	-0.0277104952854745\\
2.63894062157472	-0.291494298734734\\
2.44013251083137	-0.570393290038137\\
2.20494150359352	-0.859649139467996\\
1.93426251361171	-1.1525223052927\\
1.63191099575297	-1.44070778788785\\
1.30469689836056	-1.71524417896233\\
0.961809545903872	-1.96775111774133\\
0.613598070559529	-2.19163828849998\\
0.270106034005761	-2.3829008079898\\
-0.0601842625425428	-2.54029310737039\\
-0.371081514574716	-2.66493929156151\\
-0.658830666399524	-2.75962282515189\\
-0.921821623365125	-2.82802513033402\\
-1.16008213079777	-2.87409551910905\\
-1.37473982056242	-2.90162188662688\\
-1.56756118680114	-2.91399083853101\\
-1.740606418096	-2.91408981741843\\
-1.89599662931821	-2.90429937494905\\
-2.03577159460937	-2.88653403208306\\
-2.16181254714747	-2.86230368386329\\
-2.27580800711821	-2.83277894819268\\
-2.37924602595903	-2.79885186349478\\
-2.47342135895941	-2.76118829145959\\
-2.55945010266752	-2.72027114215685\\
-2.63828719565115	-2.67643490915497\\
-2.71074408814828	-2.6298925648727\\
-2.77750509407517	-2.58075599703199\\
-2.83914166821686	-2.52905108890512\\
-2.89612426683683	-2.47472838286705\\
-2.94883166501475	-2.41767008273216\\
-2.99755769310528	-2.35769397601212\\
-3.04251536440935	-2.2945547054575\\
-3.08383832442057	-2.22794269467252\\
-3.12157947480033	-2.15748093717595\\
-3.15570652122286	-2.08271979484268\\
-3.18609406848149	-2.00312992664664\\
-3.21251174361749	-1.91809349492053\\
-3.23460767710368	-1.82689389684362\\
-3.25188653209408	-1.72870448128615\\
-3.26368118112669	-1.62257709444266\\
-3.26911716301003	-1.50743193969673\\
-3.2670693455275	-1.38205126127052\\
-3.2561110053532	-1.24508092609879\\
-3.23445719275566	-1.09504625615184\\
-3.19990733869133	-0.930391566459728\\
-3.14979732421527	-0.749556666920957\\
-3.08097940364488	-0.5511073661317\\
-2.98985962964534	-0.333938851596169\\
-2.87253514113668	-0.0975668422736795\\
-2.72508233437513	0.157494529622942\\
-2.54404001527801	0.429303881850831\\
-2.32709234919699	0.714103779021443\\
-2.07387200165371	1.00612432375832\\
-1.78668666913353	1.29775933191409\\
-1.47088122335579	1.58023484821087\\
-1.1345756241589	1.84471730760219\\
-0.787717671493854	2.083588282538\\
-0.440683840723205	2.2914878559795\\
-0.102865668160095	2.46580854778831\\
0.218346993157528	2.60656514302512\\
0.517995089490965	2.71580904103266\\
0.793451069451776	2.79686253352453\\
1.04399255122457	2.85360765026289\\
1.27026126265922	2.88995393049057\\
1.47375493321376	2.90950846059762\\
1.65642263207936	2.91541373390173\\
};
\addplot [color=mycolor1, forget plot]
  table[row sep=crcr]{%
5.18712583895256	4.03643886090506\\
5.31525725204707	4.03252906981332\\
5.41757620024171	4.02287294925828\\
5.50079469346955	4.00964175117464\\
5.5696003253642	3.99416555895309\\
5.62733170440595	3.97726378963646\\
5.67641122072661	3.95943940624426\\
5.71862685754273	3.94099514264964\\
5.75531895681641	3.92210424785202\\
5.78750618287741	3.90285415339594\\
5.81597190070615	3.88327366803006\\
5.84132430607785	3.86334991685115\\
5.8640388204713	3.84303872581938\\
5.88448826356221	3.82227067715433\\
5.90296441651425	3.80095417392495\\
5.91969336093515	3.77897630111696\\
5.9348461628453	3.75620191182305\\
5.94854590917154	3.73247111600344\\
5.96087169458886	3.70759515234812\\
5.97185983006285	3.68135044432882\\
5.98150224897495	3.65347044950186\\
5.98974177522711	3.6236346761894\\
5.99646353807814	3.59145392663468\\
6.00148130126427	3.55645037850838\\
6.00451671496219	3.51803045859987\\
6.00516833049201	3.4754474699895\\
6.00286535774335	3.42774940781586\\
5.99679810904926	3.37370501352264\\
5.98581200843584	3.31169733754317\\
5.96824343302717	3.23956802578995\\
5.94166076599006	3.1543857977506\\
5.90244799924475	3.05209698328066\\
5.84512255933808	2.92699173597258\\
5.76120034149497	2.77088497318124\\
5.63729389255025	2.57187402037809\\
5.45196389052451	2.31254430268633\\
5.170800179951	1.96774488397845\\
4.74005681044081	1.50317000806675\\
4.08343851340225	0.879405714822638\\
3.11873275843801	0.0725212336633277\\
1.82240158153292	-0.879979006481791\\
0.3208772432074	-1.84474570715855\\
-1.13257542859024	-2.65499655952802\\
-2.33491184292571	-3.22951294326548\\
-3.23551674549088	-3.5919659263711\\
-3.8819331864792	-3.80563197939625\\
-4.34333653284971	-3.92625706921822\\
-4.67714451079244	-3.99118894206398\\
-4.92384191511931	-4.02306768443975\\
-5.1104782906461	-4.03519717120396\\
-5.25494229682946	-4.03538413725133\\
-5.36916620850968	-4.02825870448253\\
-5.46124197072875	-4.01660612024233\\
-5.53676482188745	-4.00212386009999\\
-5.59968129721263	-3.98585566318484\\
-5.65282887921241	-3.96844430635513\\
-5.69828447180154	-3.95028150863634\\
-5.73759331009454	-3.93159842884888\\
-5.77192201193149	-3.91252116361816\\
-5.80216268002389	-3.89310518437901\\
-5.82900484455859	-3.8733568160797\\
-5.8529858815526	-3.85324654641236\\
-5.87452674587498	-3.83271703316872\\
-5.89395747634478	-3.81168753720132\\
-5.91153540716701	-3.79005581256549\\
-5.92745802288954	-3.7676980440536\\
-5.94187172094541	-3.74446712558521\\
-5.95487727028482	-3.72018935421026\\
-5.9665323937902	-3.69465943029842\\
-5.97685159715728	-3.66763347237637\\
-5.98580306939653	-3.63881954565161\\
-5.99330214181357	-3.60786493364055\\
-5.99920035424942	-3.57433900902988\\
-6.00326855541121	-3.53771001937137\\
-6.00517152686637	-3.49731329719506\\
-6.00443015115258	-3.45230717604644\\
-6.00036477491604	-3.40161098959502\\
-5.99200950644002	-3.34381653301228\\
-5.9779805973346	-3.27705958700445\\
-5.95627075586493	-3.19883042356599\\
-5.92392157032137	-3.10568985900584\\
-5.87649171664733	-2.99283785316287\\
-5.80717843048081	-2.85345220038719\\
-5.70534840422258	-2.67767671923539\\
-5.5540823371721	-2.45111316334733\\
-5.32619170660676	-2.15275947247519\\
-4.97839825424011	-1.75289683024479\\
-4.44544133267055	-1.21346613054153\\
-3.64360737117788	-0.498615041134729\\
-2.50836858312705	0.391708787828822\\
-1.08269748007528	1.37202389763564\\
0.427899320710524	2.27720604166756\\
1.77169676503285	2.97202628862457\\
2.82116844859093	3.43344452188135\\
3.58608777976602	3.71348668836244\\
4.131761062643	3.87482760873836\\
4.52329054454194	3.96399345419352\\
4.80943067044692	4.01026330629437\\
5.02337853892604	4.03101761132268\\
5.18712583895256	4.03643886090506\\
};
\addplot [color=mycolor1, forget plot]
  table[row sep=crcr]{%
2.11655100682591	2.82843428941261\\
2.27354611471052	2.82356578471402\\
2.41025665966359	2.81060627378123\\
2.52989546066145	2.79154091646479\\
2.63516721135814	2.76782903818278\\
2.72832261743308	2.74053005988702\\
2.81122263331923	2.71040195994312\\
2.88540101761024	2.67797585950846\\
2.95212010046617	2.64361138065595\\
3.01241816078134	2.60753712579265\\
3.06714850265728	2.56987987342384\\
3.11701103132388	2.530685279376\\
3.16257733553198	2.48993216479353\\
3.20431024779221	2.44754189978502\\
3.2425787120241	2.40338394581532\\
3.27766860712889	2.35727827727643\\
3.30978998395011	2.30899513789073\\
3.33908098109372	2.25825237864351\\
3.36560849027125	2.20471045325069\\
3.38936543605966	2.14796500184279\\
3.41026430598958	2.08753682547208\\
3.42812629933144	2.02285894077584\\
3.44266513926722	1.95326031123863\\
3.4534641954334	1.87794579759002\\
3.4599450789137	1.79597189595953\\
3.46132530302978	1.70621801868249\\
3.45656199668418	1.60735356718522\\
3.44427815481841	1.49780211181684\\
3.4226678564112	1.37570607723411\\
3.38937801254489	1.23889915632012\\
3.34136798234704	1.08490031045564\\
3.27475744922555	0.910953996822627\\
3.18469151120958	0.71415720554415\\
3.06528542191618	0.491733786396986\\
2.90976311744834	0.241532573386507\\
2.71096460636707	-0.0371839138929645\\
2.4624261097362	-0.342670722307781\\
2.16014017496843	-0.6695545814815\\
1.80476531683354	-1.00810768327461\\
1.40349383662699	-1.34464591286619\\
0.97038036758317	-1.66351810247334\\
0.524318870350239	-1.95029936328905\\
0.085159888355619	-2.19486019379004\\
-0.330303447862874	-2.39290064579353\\
-0.710776629391997	-2.54551613559592\\
-1.05080767266494	-2.65747804245967\\
-1.34965578360068	-2.7352745046498\\
-1.60964878916396	-2.78560491585113\\
-1.83470724848301	-2.81451245994852\\
-2.02929746385226	-2.82703310171971\\
-2.19780572461383	-2.82716004636509\\
-2.34422376368673	-2.81795933766786\\
-2.47202961915923	-2.80173490909645\\
-2.58417664940125	-2.78019148732509\\
-2.68313477180989	-2.75457447721281\\
-2.77095171368762	-2.72578186937574\\
-2.84931736926041	-2.69445020124993\\
-2.91962332499729	-2.66101894855914\\
-2.98301449583141	-2.625777943375\\
-3.0404322803138	-2.5889018140239\\
-3.0926497659045	-2.55047463225179\\
-3.14029993083631	-2.51050718663972\\
-3.18389785141002	-2.46894866039765\\
-3.22385782311156	-2.42569398502503\\
-3.26050613704066	-2.38058775033286\\
-3.29409006498849	-2.3334252506136\\
-3.32478341468164	-2.28395101236323\\
-3.35268882400787	-2.23185496103844\\
-3.37783676405674	-2.17676622764621\\
-3.4001810051745	-2.11824445995762\\
-3.41959005405378	-2.05576838234722\\
-3.43583377655189	-1.98872124382334\\
-3.44856406235342	-1.91637271697959\\
-3.45728794729811	-1.83785678961846\\
-3.46133108014034	-1.75214528356065\\
-3.45978882279539	-1.65801695123747\\
-3.45146169164846	-1.55402283821256\\
-3.43477150758744	-1.43845010349195\\
-3.40765503359	-1.3092893324724\\
-3.36743408852029	-1.16421545119853\\
-3.31066711786081	-1.00060088864916\\
-3.23300037864963	-0.815592941179428\\
-3.12906223935289	-0.606305635533384\\
-2.99248669002533	-0.370196004944904\\
-2.81621104383475	-0.105701784839631\\
-2.59324586139854	0.186819618571972\\
-2.31809609238649	0.503941380582688\\
-1.98880682986173	0.838150307963187\\
-1.60913736983443	1.17759478204281\\
-1.18981009540952	1.5072611728684\\
-0.747698446788933	1.81168182170006\\
-0.30271621218734	2.07826646569063\\
0.126382115899771	2.29973691404187\\
0.525382985478411	2.47465037794311\\
0.885985247779934	2.60619273420778\\
1.20528688360715	2.70022002649834\\
1.48428693993919	2.7634737488266\\
1.72626932379742	2.80239670394494\\
1.93552996856662	2.8225480577959\\
2.11655100682591	2.82843428941261\\
};
\addplot [color=mycolor1]
  table[row sep=crcr]{%
0.948683298050514	3.79473319220206\\
1.12996891013208	3.78902241071535\\
1.3027838103604	3.77255915923418\\
1.46753552344581	3.74623053454947\\
1.62465224482371	3.71077288416293\\
1.77456152903207	3.66677970030574\\
1.91767403561802	3.61471073626391\\
2.05437110063153	3.55490128334521\\
2.18499505421831	3.4875709448019\\
2.30984136685525	3.41283153872394\\
2.42915185734506	3.33069397942336\\
2.54310832592638	3.24107414976038\\
2.65182608463458	3.14379790634541\\
2.75534694820212	3.03860547381077\\
2.8536313296431	2.92515559896292\\
2.9465491655584	2.80302996392973\\
3.0338694906698	2.67173851056894\\
3.11524860652315	2.53072651479627\\
3.19021696761039	2.3793844730251\\
3.25816516606007	2.21706211963439\\
3.31832976399147	2.0430881668871\\
3.36978023105031	1.85679760784185\\
3.41140891541445	1.65756857671014\\
3.44192680709927	1.44487070344488\\
3.45986879009691	1.21832646083672\\
3.46361298678702	0.977785965288651\\
3.45141941180798	0.723413822791541\\
3.4214930614047	0.455783738589644\\
3.37207522993717	0.175972765268488\\
3.30156372141992	-0.114357323644244\\
3.20865740947072	-0.412907981729085\\
3.0925136091427	-0.716739496073036\\
2.95289924526556	-1.02233055985344\\
2.79031115637498	-1.3257107709346\\
2.60603988731933	-1.6226624623078\\
2.40215712665582	-1.90897292279738\\
2.18141953804948	-2.18070442957845\\
1.94709823579504	-2.43444260737754\\
1.70275852491129	-2.66748659951055\\
1.45202370644265	-2.877956852001\\
1.19835703486066	-3.0648138241024\\
0.944888096380122	-3.22779787311511\\
0.694297500839398	-3.36731218002033\\
0.448761194715288	-3.48427490957016\\
0.209946190700892	-3.57996468520442\\
-0.0209555786883461	-3.65587741213202\\
-0.243169602253506	-3.71360522906444\\
-0.456269901983927	-3.7547419037936\\
-0.660108423391693	-3.78081428495334\\
-0.854748940791145	-3.79323662458319\\
-1.04040889805107	-3.79328335955738\\
-1.21741050351997	-3.78207578896275\\
-1.38614100991934	-3.76057854694004\\
-1.54702130890241	-3.72960250007616\\
-1.70048159286015	-3.6898114747465\\
-1.84694273429099	-3.64173092446581\\
-1.98680208954437	-3.58575723170233\\
-2.1204225680783	-3.52216679568137\\
-2.24812396922559	-3.45112440165093\\
-2.37017574651009	-3.37269062063008\\
-2.48679050050333	-3.2868281760587\\
-2.59811762050387	-3.19340735751618\\
-2.70423659456836	-3.09221068144532\\
-2.80514959232286	-2.98293711145668\\
-2.90077300437504	-2.86520627071399\\
-2.99092770791458	-2.73856321838213\\
-3.07532793559953	-2.60248453106578\\
-3.1535687736533	-2.45638663503084\\
-3.22511252958097	-2.29963757630341\\
-3.28927451889835	-2.13157368366421\\
-3.34520925490937	-1.95152284702656\\
-3.39189861348128	-1.7588363466938\\
-3.42814429814894	-1.5529312348605\\
-3.45256782598978	-1.33334504716437\\
-3.46362220116165	-1.0998039137315\\
-3.45962024522194	-0.852303713618117\\
-3.43878487541956	-0.591201557879526\\
-3.3993259814164	-0.317311504170883\\
-3.3395463824176	-0.0319942042762583\\
-3.25797518618734	0.262774139257094\\
-3.15352067565559	0.564371272065384\\
-3.02562738470585	0.869554764908886\\
-2.87441511977517	1.17455760801215\\
-2.70077402394131	1.47525676358082\\
-2.50639202484484	1.767403513355\\
-2.29370038309167	2.04688929885519\\
-2.06573813670639	2.31000991273335\\
-1.8259528645897	2.5536887433386\\
-1.57796799985389	2.77562773795799\\
-1.32535178123783	2.9743703475303\\
-1.07141885643386	3.14927865398809\\
-0.819084909373598	3.30044159903781\\
-0.570781691446472	3.42853925898355\\
-0.328428492520302	3.53468897189284\\
-0.093448740680559	3.62029468577783\\
0.13318251344032	3.68691393725669\\
0.350873927242831	3.73614982594868\\
0.55934713845833	3.76956970364108\\
0.758567824726602	3.78864856747597\\
0.948683298050513	3.79473319220206\\
};
\addlegendentry{Аппроксимации}

\addplot [color=mycolor2]
  table[row sep=crcr]{%
2.12132034355964	2.82842712474619\\
2.18582575000323	2.82640660874952\\
2.2459761483502	2.82067996094826\\
2.30316600770785	2.81153894842004\\
2.35833530419819	2.79908319427389\\
2.4121407651283	2.78328504568152\\
2.46505313939319	2.76402339892432\\
2.51741350921551	2.74110157112695\\
2.56946568278775	2.71425682483196\\
2.62137377470791	2.68316571665979\\
2.67322998558975	2.64744786912504\\
2.72505542741334	2.60667010622203\\
2.77679571750944	2.56035271147659\\
2.82831256449113	2.50797963523731\\
2.87937250576692	2.44901464186941\\
2.92963424895255	2.38292549191834\\
2.97863667364875	2.30921809960124\\
3.02579039162557	2.22748193169334\\
3.0703766721051	2.13744642550811\\
3.11155818595542	2.03904567139316\\
3.14840590189045	1.93248505458784\\
3.1799449834646	1.81829951655162\\
3.20521925970046	1.69738980361013\\
3.2233689191928	1.57102231442397\\
3.23371058284854	1.44078163681543\\
3.23580480244767	1.30847305653245\\
3.22949541157146	1.1759834831\\
3.21490900216393	1.04511946165368\\
3.19240995273442	0.917445713323391\\
3.16251366135309	0.794144173493542\\
3.12576385298257	0.675902054710668\\
3.08257536776515	0.562820516214273\\
3.03302813706062	0.454314775018543\\
2.97656426645917	0.348948817128694\\
2.91146921429886	0.244099783305823\\
2.83385680006128	0.135243682019644\\
2.73548220960228	0.0144130439331602\\
2.59871334341581	-0.133168683665202\\
2.38479638365891	-0.336180712113444\\
2.01057175466146	-0.648371525543042\\
1.34210041814469	-1.13791067535239\\
0.375352321166521	-1.75822148716338\\
-0.527448468220416	-2.26262801632552\\
-1.11365847778129	-2.5437869612038\\
-1.4567327767317	-2.68228003389134\\
-1.6691296256022	-2.75259678373195\\
-1.81428800439327	-2.79054335642282\\
-1.92320462744895	-2.81168897715777\\
-2.01132392805712	-2.82302365363059\\
-2.08680248126988	-2.82787464184343\\
-2.15422943553058	-2.82790871121207\\
-2.21634478307291	-2.8239827357917\\
-2.27487422572476	-2.8165264847195\\
-2.33095693947289	-2.80572465930185\\
-2.3853755403757	-2.79160756168836\\
-2.43868469976231	-2.77409782070174\\
-2.49128507004909	-2.75303495626884\\
-2.54346612600745	-2.7281884019911\\
-2.59543032955127	-2.69926454755406\\
-2.64730535029083	-2.66591103001887\\
-2.69914810041563	-2.62772046423944\\
-2.75094277189846	-2.58423541950896\\
-2.80259428647749	-2.53495641420315\\
-2.85391830194335	-2.47935483422118\\
-2.90462903903618	-2.41689283661657\\
-2.95432664759112	-2.34705230341292\\
-3.00248656709011	-2.26937453210355\\
-3.04845424185856	-2.18351130413452\\
-3.0914493816545	-2.08928598065547\\
-3.13058429431929	-1.98676020448289\\
-3.16490008683964	-1.87629788656397\\
-3.19342217916266	-1.75861429449935\\
-3.21523240499265	-1.63479579561046\\
-3.22954956680171	-1.50627700312196\\
-3.23580523135335	-1.37476797309193\\
-3.23369896357047	-1.2421341494958\\
-3.22321880671614	-1.11024302689198\\
-3.20461860896113	-0.980799507713277\\
-3.17835148049538	-0.85519274756288\\
-3.14496439092874	-0.734369602871579\\
-3.10495870881183	-0.618735254776386\\
-3.05861192744495	-0.508062722468019\\
-3.0057323937692	-0.401369642978784\\
-2.94527025004781	-0.296685462718974\\
-2.87460204254734	-0.190562899149911\\
-2.78805571026974	-0.0770301658026245\\
-2.67361421522119	0.0546866392302208\\
-2.50520482475694	0.225105040795738\\
-2.22541867988909	0.474087715816021\\
-1.7209533613293	0.868461990547897\\
-0.880186083660502	1.44500069250654\\
0.110323684998138	2.03883159526995\\
0.859259206598599	2.42737283484743\\
1.30741523004191	2.62511520505649\\
1.57426953542748	2.72304263504847\\
1.74766615441039	2.77435113512036\\
1.87207484559155	2.80265686669549\\
1.96924085114733	2.81832059141065\\
2.0502990590364	2.82612977760851\\
2.12132034355964	2.82842712474619\\
};
\addlegendentry{Сумма Минковского}

\end{axis}

\begin{axis}[%
width=0.798\linewidth,
height=0.578\linewidth,
at={(-0.104\linewidth,-0.064\linewidth)},
scale only axis,
xmin=0,
xmax=1,
ymin=0,
ymax=1,
axis line style={draw=none},
ticks=none,
axis x line*=bottom,
axis y line*=left,
legend style={legend cell align=left, align=left, draw=white!15!black}
]
\end{axis}
\end{tikzpicture}%
        \caption{Эллипсоидальные аппроксимации для 10 направлений.}
\end{figure}
\begin{figure}[b]
        \centering
        % This file was created by matlab2tikz.
%
%The latest updates can be retrieved from
%  http://www.mathworks.com/matlabcentral/fileexchange/22022-matlab2tikz-matlab2tikz
%where you can also make suggestions and rate matlab2tikz.
%
\definecolor{mycolor1}{rgb}{0.00000,0.44700,0.74100}%
\definecolor{mycolor2}{rgb}{0.85000,0.32500,0.09800}%
%
\begin{tikzpicture}

\begin{axis}[%
width=0.618\linewidth,
height=0.487\linewidth,
at={(0\linewidth,0\linewidth)},
scale only axis,
xmin=-10,
xmax=10,
xlabel style={font=\color{white!15!black}},
xlabel={$x_1$},
ymin=-6,
ymax=6,
ylabel style={font=\color{white!15!black}},
ylabel={$x_2$},
axis background/.style={fill=white},
axis x line*=bottom,
axis y line*=left,
xmajorgrids,
ymajorgrids,
legend style={at={(0.03,0.97)}, anchor=north west, legend cell align=left, align=left, draw=white!15!black}
]
\addplot [color=mycolor1, forget plot]
  table[row sep=crcr]{%
1.1711787112841	3.34230777773576\\
1.34444389618673	3.33686722689733\\
1.50648318534277	3.32144701348819\\
1.65806242150429	3.29723890999844\\
1.79995419924852	3.26523166869947\\
1.93290799118999	3.22622794766654\\
2.05763032437368	3.18086276334402\\
2.17477223041259	3.12962150018857\\
2.28492171716205	3.07285634836238\\
2.3885995032216	3.01080060567632\\
2.48625667387058	2.94358064140543\\
2.57827324651315	2.87122553999274\\
2.66495688065443	2.79367457178694\\
2.74654114431786	2.71078271324095\\
2.82318287015103	2.62232448808267\\
2.8949582146523	2.52799644473262\\
2.96185708693385	2.42741864049226\\
3.02377565347494	2.3201355847161\\
3.0805066685131	2.20561721602438\\
3.13172744617751	2.08326066796132\\
3.17698540702193	1.95239382914097\\
3.21568133490354	1.81228204209558\\
3.24705081945703	1.66213971753183\\
3.27014489780722	1.50114915910447\\
3.28381172002077	1.32848945666784\\
3.28668221628109	1.14337881243048\\
3.2771642749897	0.945133918403077\\
3.25345178932588	0.733249674250213\\
3.21355684151227	0.507501133889645\\
3.15537469570401	0.268066487797808\\
3.07679115926015	0.015664562009573\\
2.97583883725379	-0.248307401701393\\
2.85090136682411	-0.521659161442773\\
2.70095216778915	-0.801347823435408\\
2.52579793500058	-1.08350783899399\\
2.32628156602984	-1.3635980584876\\
2.10439191407115	-1.63667268179468\\
1.86323617293213	-1.89775167737354\\
1.60685736899554	-2.14223401489281\\
1.33991789027605	-2.3662777508081\\
1.06730585933085	-2.56707514274059\\
0.793739425298697	-2.74297851870391\\
0.523437853608312	-2.89347215049445\\
0.259902596754298	-3.0190207516634\\
0.00581903236219554	-3.1208447467499\\
-0.2369372555121	-3.20067372983419\\
-0.467220354507	-3.2605178574641\\
-0.684493265340969	-3.30248032509504\\
-0.888694234930048	-3.32861897022119\\
-1.0801092196493	-3.34085455680139\\
-1.25926106809304	-3.34091776153004\\
-1.42681950330378	-3.33032515794113\\
-1.58353177449541	-3.31037507736533\\
-1.73017158467123	-3.2821558685242\\
-1.86750300244094	-3.24656096722397\\
-1.99625599202063	-3.20430689704777\\
-2.11711054093165	-3.15595168681459\\
-2.23068687011842	-3.10191219445658\\
-2.33753972875556	-3.04247952074268\\
-2.43815523449964	-2.97783215102365\\
-2.5329490934295	-2.90804674702345\\
-2.6222653208492	-2.83310668006404\\
-2.70637479428206	-2.75290849531896\\
-2.78547311726505	-2.66726655590999\\
-2.85967737154948	-2.5759161597878\\
-2.92902140016315	-2.47851546990414\\
-2.99344930844609	-2.37464666482809\\
-3.05280690958282	-2.26381681735669\\
-3.10683089311493	-2.14545915779927\\
-3.15513558252945	-2.01893559215682\\
-3.19719730262603	-1.88354163878489\\
-3.23233664163192	-1.73851533219976\\
-3.25969932318791	-1.58305212075492\\
-3.27823706849968	-1.41632833369867\\
-3.28669080350863	-1.23753634540448\\
-3.28357990805066	-1.04593497683649\\
-3.26720291066679	-0.840918687724783\\
-3.23565696131938	-0.622108318461704\\
-3.18688516768166	-0.389463978954261\\
-3.11876167995196	-0.143416533926881\\
-3.02922302993614	0.11499241610004\\
-2.91644915653916	0.383980836121862\\
-2.77908754849337	0.660934801326638\\
-2.61649915361415	0.942389492277687\\
-2.42898799999709	1.22411526429499\\
-2.21796402213223	1.50132757866017\\
-1.9859883878281	1.76901289011117\\
-1.7366683581316	2.02232922526914\\
-1.47440268220478	2.25701269019761\\
-1.20401775520347	2.46971278707818\\
-0.930363383198716	2.65819606075749\\
-0.657943143023997	2.82139307813531\\
-0.390637102663735	2.95930311865273\\
-0.131543811712393	3.07279931782965\\
0.117062449314586	3.1633871238911\\
0.353684227850852	3.23296276733506\\
0.577496523037587	3.28360333410188\\
0.788216624026908	3.3174035782658\\
0.985971690059736	3.33636162404406\\
1.1711787112841	3.34230777773576\\
};
\addplot [color=mycolor1, forget plot]
  table[row sep=crcr]{%
1.21926508344108	3.27334196367036\\
1.39130630412839	3.26794365173349\\
1.55153598090393	3.25269915059364\\
1.70081822737307	3.22886110922975\\
1.84001535490823	3.19746467488806\\
1.96995663540462	3.15934744514424\\
2.09141838442226	3.11517076291109\\
2.20511208961792	3.06544020144585\\
2.31167798429415	3.01052405216791\\
2.41168207975524	2.95066926128589\\
2.50561517604179	2.88601465321008\\
2.59389276130835	2.81660150697173\\
2.67685499694353	2.74238167588435\\
2.75476618649088	2.66322350375262\\
2.82781326027155	2.57891582390079\\
2.89610289157071	2.489170352485\\
2.95965690978143	2.39362282222959\\
3.0184057055742	2.29183326184168\\
3.07217934834312	2.18328592468469\\
3.12069617535093	2.0673895238837\\
3.16354868972977	1.94347865783223\\
3.20018675510097	1.81081762911891\\
3.22989834610506	1.66860828812922\\
3.25178857180767	1.51600407734397\\
3.26475841566826	1.35213309692555\\
3.26748572490109	1.17613369082119\\
3.25841251405898	0.987206612265718\\
3.23574463765734	0.784687977211719\\
3.19747218836968	0.568146474359487\\
3.14142113656216	0.337505988781863\\
3.06534780932005	0.0931901220106009\\
2.96708629627689	-0.163722532576174\\
2.84475284042693	-0.431353836986617\\
2.69699912472874	-0.706924876445341\\
2.52328828199352	-0.986738253877743\\
2.32414719016614	-1.26628199746165\\
2.1013342676447	-1.54047573935053\\
1.85786382933463	-1.80404664367384\\
1.59785273337257	-2.05198218219746\\
1.32619884624731	-2.27997573381088\\
1.04814809460687	-2.48477539672332\\
0.768837071389945	-2.66437165107218\\
0.492898365330516	-2.81800515148501\\
0.224188427726118	-2.94602190329025\\
-0.0343421486742361	-3.04963199342895\\
-0.280651418753673	-3.13063387286929\\
-0.513517918101939	-3.19115408703827\\
-0.732402138426053	-3.23343245666966\\
-0.937291976094646	-3.25966380617091\\
-1.12855566644943	-3.27189398375029\\
-1.30681506667752	-3.27196083427941\\
-1.47284399601866	-3.26146857518149\\
-1.62749116596371	-3.24178474838236\\
-1.77162454720074	-3.21405096074922\\
-1.9060930438003	-3.1792009422838\\
-2.03170136752072	-3.13798151469425\\
-2.14919451031851	-3.09097368125037\\
-2.25924887975171	-3.03861221534132\\
-2.36246781585791	-2.9812029127911\\
-2.45937977071431	-2.91893717481279\\
-2.55043787954686	-2.85190389003105\\
-2.63601998899356	-2.78009875378493\\
-2.71642844964106	-2.70343125192797\\
-2.79188914508071	-2.62172958112562\\
-2.86254933659929	-2.53474380432833\\
-2.92847396752118	-2.44214756807608\\
-2.98964010899711	-2.34353875337005\\
-3.04592925441029	-2.23843950866587\\
-3.09711719913452	-2.12629623737968\\
-3.14286129724365	-2.00648030014085\\
-3.18268499588943	-1.87829046258472\\
-3.21595975211415	-1.74095849136247\\
-3.24188479216075	-1.59365978768918\\
-3.25946575421678	-1.43553154684198\\
-3.26749415178163	-1.26570160623257\\
-3.26453089777461	-1.0833317902911\\
-3.24889889583809	-0.887679962191258\\
-3.21869188468844	-0.678184770435691\\
-3.17180903470924	-0.454575648787598\\
-3.10602656951161	-0.217007231912446\\
-3.01911769523831	0.0337887596613096\\
-2.90902859638195	0.296350206020716\\
-2.77410929927924	0.568356995772149\\
-2.6133828822338	0.846566357802756\\
-2.42681658880744	1.12685188938682\\
-2.21553984909621	1.4043777730154\\
-1.98194673587055	1.67391391702028\\
-1.72963332752344	1.93025930170482\\
-1.4631559553519	2.16870275608031\\
-1.18764436113277	2.38543046078226\\
-0.908344887791558	2.57779963933086\\
-0.630184750931516	2.74443551343009\\
-0.357433580063127	2.88515703813491\\
-0.0935024920494292	3.00077594658033\\
0.15911854673327	3.09283095590222\\
0.398812914783312	3.163314718129\\
0.624719128954104	3.21443380598977\\
0.836580651685356	3.24842176843127\\
1.03459285132093	3.26740886591368\\
1.21926508344108	3.27334196367036\\
};
\addplot [color=mycolor1, forget plot]
  table[row sep=crcr]{%
1.27235524758866	3.20606889606907\\
1.44317107064595	3.20071319568392\\
1.60153774787983	3.18564973372891\\
1.74843510276585	3.16219597058339\\
1.88482914833813	3.13143492326719\\
2.01163976581522	3.09423891990115\\
2.12972112934266	3.05129435069619\\
2.23985098586104	3.00312508327947\\
2.34272577688468	2.95011332159164\\
2.43895935773047	2.89251739452003\\
2.52908368509872	2.8304863814774\\
2.61355030737859	2.76407171387989\\
2.69273182444275	2.69323600549816\\
2.76692271100404	2.61785941119026\\
2.83633904449915	2.53774382694731\\
2.9011167660551	2.45261524824953\\
2.96130814896022	2.36212461590775\\
3.01687616843364	2.26584751343373\\
3.06768647413494	2.16328315185123\\
3.11349667956523	2.05385320276079\\
3.15394272225835	1.93690123748885\\
3.18852214664256	1.81169382176359\\
3.21657436300479	1.67742472593303\\
3.23725830595004	1.53322426152362\\
3.24952854266518	1.37817645273807\\
3.25211187660099	1.21134756681076\\
3.24348797643739	1.03183035661999\\
3.22187962263265	0.838808978297762\\
3.18526077526704	0.631649497785332\\
3.13139352539076	0.410019487872334\\
3.05790730118758	0.174036479049343\\
2.96243394314779	-0.0755620455059427\\
2.84280816668981	-0.337245310360568\\
2.69733198476799	-0.608544401674519\\
2.52508258389448	-0.885980316488405\\
2.32621826608086	-1.16511378557182\\
2.10221443243537	-1.44075380334538\\
1.85595416779634	-1.70732887684435\\
1.59161769905144	-1.95937674660199\\
1.31436302269743	-2.19206276802241\\
1.02985073295835	-2.40161735788381\\
0.743712717332933	-2.58560229945124\\
0.46107506075	-2.74296710880307\\
0.186217675282633	-2.87391600087286\\
-0.0775960985361585	-2.97964801275228\\
-0.328133772499326	-3.06204567327999\\
-0.564094671509743	-3.12337551626293\\
-0.784948102500155	-3.16603958225219\\
-0.990751904935546	-3.19239302335658\\
-1.18198074976573	-3.20462570620811\\
-1.35937986129559	-3.20469658668081\\
-1.52384947555703	-3.19430684199975\\
-1.67635893937152	-3.17489872235242\\
-1.81788618426142	-3.14766969123728\\
-1.94937729908481	-3.11359431862355\\
-2.07172112440674	-3.07344891393212\\
-2.18573453476242	-3.02783582028248\\
-2.29215496469956	-2.9772056520116\\
-2.39163756765905	-2.92187665091016\\
-2.48475509068474	-2.86205088696231\\
-2.57199908552444	-2.79782734529923\\
-2.65378147146904	-2.72921210646848\\
-2.73043574186991	-2.65612590224109\\
-2.80221729049535	-2.57840935563713\\
-2.86930244872764	-2.49582622010879\\
-2.93178588931728	-2.40806493900825\\
-2.98967608320614	-2.31473886823903\\
-3.04288850753444	-2.21538555650091\\
-3.09123631078617	-2.10946557375074\\
-3.13441816403804	-1.99636153685547\\
-3.17200309148303	-1.87537822273494\\
-3.20341221640447	-1.74574500735886\\
-3.22789763501203	-1.60662234752231\\
-3.24451911637134	-1.45711464736294\\
-3.25212012271858	-1.29629261534778\\
-3.24930586973196	-1.12322905938835\\
-3.23442791317944	-0.937052827376804\\
-3.20558210147849	-0.737025951103928\\
-3.16062953150577	-0.522648426972396\\
-3.09725286895748	-0.293792623174874\\
-3.01306191087989	-0.0508640093726965\\
-2.90576066218136	0.205024084380883\\
-2.77338096163089	0.471886516761707\\
-2.61457267711254	0.746751816909297\\
-2.42891789191419	1.0256452840926\\
-2.21721135786237	1.30371511604229\\
-1.98163270338355	1.57552461067301\\
-1.72574098366183	1.83549137118179\\
-1.45425689642249	2.0784046669254\\
-1.17265530960846	2.29991677896501\\
-0.886647911507271	2.49690345800275\\
-0.601666181623873	2.66762625184561\\
-0.322445205446415	2.81168834633155\\
-0.0527673783806232	2.92982878700886\\
0.204625871021466	3.02362775785022\\
0.447985415526008	3.09519465143329\\
0.676417888047168	3.14689079713794\\
0.889709449657395	3.1811133975024\\
1.08814645535659	3.20014614291162\\
1.27235524758866	3.20606889606907\\
};
\addplot [color=mycolor1, forget plot]
  table[row sep=crcr]{%
1.331122526514	3.1411519504109\\
1.50070265021491	3.13583953954657\\
1.65714137817264	3.12096355391375\\
1.801554491142	3.09791011124431\\
1.93502746499673	3.06781117661229\\
2.05858258751093	3.03157307872426\\
2.1731603825899	2.98990544455629\\
2.27961069804984	2.94334806244517\\
2.37868996212446	2.89229446064749\\
2.47106207995261	2.83701177144793\\
2.55730118991524	2.7776568956931\\
2.63789504769877	2.71428921024775\\
2.71324818860979	2.64688015759524\\
2.78368427326967	2.57532008083767\\
2.849447181013	2.49942265679744\\
2.9107005052992	2.41892725977025\\
2.96752514659502	2.33349957604737\\
3.01991470673442	2.24273079836124\\
3.06776837901516	2.14613577319474\\
3.11088101529588	2.04315056769814\\
3.14893005419316	1.93313008597277\\
3.18145904053263	1.81534662115847\\
3.20785759670683	1.68899060993897\\
3.22733798348076	1.5531753925286\\
3.23890890115111	1.40694850199169\\
3.24134805532164	1.24931291790325\\
3.23317639837383	1.07926277119951\\
3.2126390158541	0.895839016149664\\
3.177700448064	0.69821122718357\\
3.1260656987857	0.48579126058317\\
3.05524169695212	0.258381986555632\\
2.96265615170331	0.0163582778932624\\
2.84584911120226	-0.239133300220675\\
2.70274374574876	-0.505985173879308\\
2.53198394173171	-0.780996403060881\\
2.33329738953371	-1.05985624741079\\
2.10781090040156	-1.33729898614381\\
1.85822480693181	-1.60745595567134\\
1.58876376753239	-1.86437606022391\\
1.30487103163859	-2.10262351755484\\
1.01268938097015	-2.31782176529302\\
0.718441030482152	-2.50702035799616\\
0.427846180930232	-2.66881735136476\\
0.145694042649061	-2.80324567414949\\
-0.124380790790276	-2.91149239657131\\
-0.379927228454498	-2.99554341560088\\
-0.619563635809872	-3.05783481070723\\
-0.842784978801955	-3.10096237588935\\
-1.04974635506385	-3.12746975963674\\
-1.24106000163502	-3.13971312950987\\
-1.41762489959441	-3.13978845934957\\
-1.58049474935395	-3.12950408936179\\
-1.73078211578909	-3.11038263397954\\
-1.86959284659487	-3.0836797413193\\
-1.99798391664057	-3.0504108901298\\
-2.11693835508858	-3.01138053033187\\
-2.22735200674953	-2.96721019972459\\
-2.33002807564449	-2.9183638378905\\
-2.42567646575971	-2.86516952714163\\
-2.51491579139968	-2.80783748683269\\
-2.59827657406231	-2.74647447073001\\
-2.67620460331178	-2.68109487045736\\
-2.74906375329484	-2.61162888258028\\
-2.81713774976441	-2.53792809983284\\
-2.88063050418508	-2.45976886910422\\
-2.93966469482067	-2.37685374065141\\
-2.99427829770052	-2.28881132968064\\
-3.04441876799606	-2.19519493611852\\
-3.08993455892186	-2.09548033507462\\
-3.13056365740489	-1.98906327620029\\
-3.1659188360687	-1.87525743658402\\
-3.19546940308262	-1.75329388623641\\
-3.21851942578628	-1.6223235794633\\
-3.23418278584247	-1.48142501135315\\
-3.24135609946731	-1.32961999626986\\
-3.23869164592522	-1.16590151836407\\
-3.22457415197589	-0.989278675191041\\
-3.19710771688668	-0.798844633232678\\
-3.15412234119223	-0.593873733391723\\
-3.09321312554347	-0.373952564769022\\
-3.01182828769036	-0.139145727453714\\
-2.90742279527531	0.109811261632613\\
-2.77768964509709	0.371312723440716\\
-2.62086721620327	0.642715147198648\\
-2.43609684116487	0.920255137140879\\
-2.22377291482778	1.19911285221477\\
-1.9857996409858	1.4736651303669\\
-1.72566175604194	1.73792940542773\\
-1.44824671829608	1.98613763849639\\
-1.15942195970072	2.2133243081796\\
-0.865448289650424	2.41579463270279\\
-0.572362115157663	2.59137293020573\\
-0.285459358621382	2.73940143022293\\
-0.0089668275124522	2.8605320594217\\
0.254079449410444	2.95639689740252\\
0.50178376295928	3.02924769275413\\
0.73322946004012	3.08163193886222\\
0.948268430835706	3.11614082010529\\
1.14730794379744	3.13523684905876\\
1.331122526514	3.1411519504109\\
};
\addplot [color=mycolor1, forget plot]
  table[row sep=crcr]{%
1.3964102868975	3.07930591740507\\
1.56473037248086	3.07403794379094\\
1.71915971789631	3.05935744921679\\
1.86097350375265	3.03672287758242\\
1.99139432743895	3.00731572039262\\
2.1115596313916	2.9720749730111\\
2.22250508572583	2.93173110760911\\
2.32515837474101	2.88683696403915\\
2.42033934512417	2.83779441843637\\
2.50876368745401	2.78487654478706\\
2.59104822992295	2.72824544210295\\
2.66771656628559	2.66796611120257\\
2.73920417543268	2.60401683410806\\
2.8058624705604	2.53629650265557\\
2.86796138470457	2.46462930281241\\
2.92569018902551	2.38876711316769\\
2.97915627449035	2.30838993674501\\
3.0283816240354	2.2231046670328\\
3.07329667483836	2.13244250358555\\
3.11373123222949	2.03585539346854\\
3.14940206395745	1.93271200046933\\
3.17989679896267	1.82229391931966\\
3.20465381435363	1.70379319033865\\
3.22293797458151	1.57631267238219\\
3.2338124753162	1.4388715448039\\
3.23610777024232	1.29041917122538\\
3.22838979925308	1.12986177408304\\
3.20893170831586	0.956107756545554\\
3.17569616377437	0.768138804989481\\
3.12633929237158	0.565114530656926\\
3.05825190772653	0.346517306048897\\
2.96865788840053	0.112339502298826\\
2.85479090633259	-0.136694414172483\\
2.71416515318386	-0.398894887802234\\
2.54493847995357	-0.671409304753301\\
2.34633470428385	-0.950126228446477\\
2.11905007755647	-1.22975688596338\\
1.86553297853278	-1.50414781259021\\
1.59002111748132	-1.76682066303184\\
1.29826723729962	-2.01165403512721\\
0.996976978743987	-2.23355486779611\\
0.693081911192092	-2.42895471339356\\
0.39302449262342	-2.59602245879899\\
0.10221242031689	-2.73458150354873\\
-0.175275587441964	-2.84580580466324\\
-0.436737284630285	-2.93180948107383\\
-0.680711669681773	-2.99523570280546\\
-0.90674417217064	-3.03891333463364\\
-1.11512406441372	-3.0656088453726\\
-1.30664144946907	-3.07787111261669\\
-1.48238709584989	-3.07795134920049\\
-1.64360097670177	-3.06777622825643\\
-1.79156543558501	-3.04895448590184\\
-1.92753469255854	-3.02280191024239\\
-2.05269169243189	-2.99037438471845\\
-2.16812430727995	-2.95250254880701\\
-2.27481451443136	-2.90982444674297\\
-2.37363578578764	-2.862814384836\\
-2.46535529269494	-2.81180734627586\\
-2.5506385879597	-2.75701894637208\\
-2.63005519501394	-2.69856122981593\\
-2.70408406639507	-2.6364547413632\\
-2.77311822563175	-2.57063732609075\\
-2.83746812660903	-2.50097008809051\\
-2.89736339022059	-2.42724088988245\\
-2.95295263753188	-2.34916572943973\\
-3.00430115212401	-2.26638830150595\\
-3.05138608715343	-2.17847804645976\\
-3.09408889814955	-2.08492702572269\\
-3.13218464480151	-1.98514605333694\\
-3.16532778277964	-1.87846068021922\\
-3.19303408855744	-1.76410789964738\\
-3.21465847136983	-1.64123485723529\\
-3.22936869755953	-1.50890145057942\\
-3.23611559058606	-1.36608953760508\\
-3.23360122780567	-1.21172256498737\\
-3.22024823272812	-1.04470074878479\\
-3.19417568159924	-0.863958330476281\\
-3.15319057074906	-0.668550491360247\\
-3.09480814848275	-0.457777436782653\\
-3.01631904392544	-0.231350619837756\\
-2.91492429939803	0.0104007841530797\\
-2.78795789248472	0.266298052535966\\
-2.63320550966211	0.534090230068579\\
-2.44930392305496	0.81029787181832\\
-2.23616740772478	1.0901969999926\\
-1.99534603198641	1.36801221596138\\
-1.73019696643741	1.63734829722721\\
-1.44576953465053	1.89181673044\\
-1.14837734553415	2.12573379650668\\
-0.844933408739004	2.33472294072841\\
-0.542206513864935	2.51607716502315\\
-0.246175073832186	2.66881927116423\\
0.0383973476672991	2.79349537142056\\
0.308127665128874	2.89180296092848\\
0.560959304573922	2.96616893137416\\
0.795966987499965	3.01936648760927\\
1.01310046780864	3.05421819894996\\
1.21292743436782	3.07339593495123\\
1.39641028689749	3.07930591740507\\
};
\addplot [color=mycolor1, forget plot]
  table[row sep=crcr]{%
1.46927735143059	3.02133831480419\\
1.63629310035011	3.01611658260144\\
1.7886098202728	3.00164167558635\\
1.9276891077562	2.97944772464369\\
2.05491029732378	2.95076569259784\\
2.17153957212107	2.91656520547484\\
2.27871666322874	2.8775945552278\\
2.37745251949422	2.8344162564902\\
2.46863330682164	2.78743718747514\\
2.55302761411155	2.73693325892181\\
2.63129483361326	2.68306900537496\\
2.70399342855486	2.62591267004484\\
2.77158828747314	2.56544738150251\\
2.83445666632126	2.5015789733937\\
2.89289239235212	2.43414092191058\\
2.94710808822543	2.36289679566997\\
2.99723519856857	2.28754054406744\\
3.04332158332015	2.2076949033788\\
3.08532639615547	2.12290818421282\\
3.1231119034702	2.03264973086929\\
3.15643183233097	1.93630442907388\\
3.18491578251768	1.83316680739732\\
3.20804922766571	1.72243556373827\\
3.22514871272181	1.60320979852764\\
3.23533211017806	1.47448891063472\\
3.23748435336648	1.33517907761217\\
3.23022011209626	1.18411055554444\\
3.21184667676275	1.02007169815298\\
3.18033318732	0.841867471807365\\
3.13329656201969	0.648411901057839\\
3.06802005852571	0.43886436819431\\
2.9815266205418	0.212817298608521\\
2.87073386701779	-0.0294650388365179\\
2.73271642082416	-0.286772864596821\\
2.56508773964061	-0.556683859649172\\
2.36648135624935	-0.835374737329632\\
2.13706031610444	-1.11760600143702\\
1.87892800807281	-1.39696779352388\\
1.59628548819868	-1.66641963371706\\
1.29521584343495	-1.91905712943754\\
0.98308441078536	-2.14893516111894\\
0.667683177910749	-2.35173150036812\\
0.356342600548117	-2.52508420102095\\
0.0552298920758284	-2.6685567979865\\
-0.231040238055092	-2.78330877698785\\
-0.499476806935808	-2.87161508239866\\
-0.748546728386371	-2.93637438496716\\
-0.977882288979001	-2.98069824932831\\
-1.18795708828636	-3.00761818760242\\
-1.37979150940508	-3.01990728770175\\
-1.55471582553872	-3.01999292769046\\
-1.71419607931498	-3.00993230317797\\
-1.85971553954364	-2.99142602962007\\
-1.99269995310453	-2.96585147839687\\
-2.11447465948285	-2.93430374220686\\
-2.22624346296074	-2.89763701989019\\
-2.32908151045466	-2.85650259944695\\
-2.42393660051112	-2.81138175736593\\
-2.51163509895577	-2.76261313414686\\
-2.59288993246338	-2.71041479938398\\
-2.66830903791877	-2.65490151430437\\
-2.73840325102495	-2.59609778961128\\
-2.80359300360421	-2.53394731925019\\
-2.86421343036985	-2.46831930479252\\
-2.92051761046401	-2.39901210446869\\
-2.9726777203323	-2.32575456506696\\
-3.02078387548822	-2.24820533607854\\
-3.06484040527515	-2.16595043299995\\
-3.10475924890094	-2.07849932066814\\
-3.14035009427886	-1.98527984148405\\
-3.17130681823211	-1.88563243717909\\
-3.19718975024394	-1.77880433519281\\
-3.21740331018235	-1.66394473219286\\
-3.23116872653173	-1.54010256139314\\
-3.23749192767323	-1.40622924192809\\
-3.23512747185839	-1.26118994304902\\
-3.22254076830271	-1.10378839066771\\
-3.19787313688991	-0.932812045608432\\
-3.1589177725706	-0.747106330769343\\
-3.10311960877118	-0.545687819133067\\
-3.02761813373429	-0.327905611498936\\
-2.92935806841843	-0.0936553921639545\\
-2.80529528362486	0.156360933207452\\
-2.6527188844722	0.420357817462845\\
-2.46968808889038	0.695227527519055\\
-2.25554013643435	0.976425937428937\\
-2.01136913889677	1.25807902966462\\
-1.74032911995783	1.53337706969685\\
-1.44761450798646	1.7952432948003\\
-1.14004500386212	2.03715506205079\\
-0.825313975647545	2.25391343942017\\
-0.511086370392609	2.44215796957082\\
-0.204180704567995	2.6005151780265\\
0.089984434113084	2.72940082914642\\
0.367614630809603	2.83059573675472\\
0.626479062175471	2.90674459359535\\
0.865667727030579	2.96089679720825\\
1.08527323667551	2.99615293675265\\
1.28607612949814	3.0154312519322\\
1.46927735143059	3.02133831480419\\
};
\addplot [color=mycolor1, forget plot]
  table[row sep=crcr]{%
1.55106012395173	2.96820108237492\\
1.71670039509252	2.96302826671365\\
1.866773650437	2.94877167105386\\
2.00295881409983	2.92704398092938\\
2.12681348158948	2.89912477506173\\
2.23974677558376	2.86601146149039\\
2.34301123972137	2.82846644198549\\
2.43770593604595	2.78705801960205\\
2.52478548784315	2.74219439080212\\
2.60507169099129	2.69415100565756\\
2.67926560970047	2.64309199157726\\
2.74795891918622	2.5890864527319\\
2.81164378731278	2.53212042352181\\
2.87072089988808	2.47210515402108\\
2.92550540208963	2.40888228458754\\
2.97623060040506	2.34222635006831\\
3.02304927747868	2.27184495383945\\
3.06603243671452	2.19737687562467\\
3.10516522738603	2.11838833098142\\
3.14033971357704	2.03436759302091\\
3.17134405045729	1.94471823184124\\
3.19784753184586	1.8487513458686\\
3.21938089574431	1.74567738509342\\
3.23531125925981	1.63459854816662\\
3.24481117124365	1.5145033422472\\
3.24682163905332	1.38426581574507\\
3.2400097972802	1.24265331190513\\
3.22272343233275	1.0883484320068\\
3.19294725989111	0.919993240239861\\
3.14827014535604	0.736266352737939\\
3.0858787298984	0.536005698720147\\
3.00260100298442	0.318389812049431\\
2.89503167884312	0.0831856446549773\\
2.75977558920631	-0.168942950911744\\
2.59383787432449	-0.436098516778841\\
2.39516019521871	-0.714856398931109\\
2.16324375154413	-1.00012576884616\\
1.89972165804939	-1.28529226677395\\
1.60868151094332	-1.56272681971187\\
1.29654972154919	-1.82463070149011\\
0.971467852367241	-2.06403750195299\\
0.642282701040046	-2.27569441529627\\
0.317430689649746	-2.4565734498626\\
0.00402370353224809	-2.60591096109272\\
-0.292670312621192	-2.72485050727559\\
-0.569328782256929	-2.81587147438732\\
-0.824363369645803	-2.88219145532937\\
-1.05754664046762	-2.92726816036025\\
-1.26960505851823	-2.95445057042329\\
-1.46185776232636	-2.96677376570594\\
-1.63593450851817	-2.96686534407497\\
-1.79357561533683	-2.95692623777213\\
-1.93650161604717	-2.9387545377024\\
-2.06633574417822	-2.9137899458279\\
-2.18456334326995	-2.88316473659085\\
-2.29251540962804	-2.84775327892353\\
-2.3913668893472	-2.80821624292687\\
-2.48214327273579	-2.76503805721257\\
-2.56573125059349	-2.7185575147283\\
-2.6428907665372	-2.66899206563575\\
-2.71426685168592	-2.61645657716661\\
-2.78040030166351	-2.5609773681715\\
-2.84173666627119	-2.50250225126656\\
-2.89863325499027	-2.44090720078849\\
-2.95136397640314	-2.3760001438983\\
-3.00012186641745	-2.30752226257338\\
-3.04501914441794	-2.23514710521278\\
-3.08608458434066	-2.15847774454589\\
-3.12325790984976	-2.07704219053697\\
-3.15638082793169	-1.99028728365994\\
-3.18518421352558	-1.89757137239653\\
-3.20927086577889	-1.79815624601589\\
-3.22809320374143	-1.69119909028635\\
-3.24092530881614	-1.57574571829855\\
-3.24682894495659	-1.45072708034643\\
-3.24461374770692	-1.31496217351149\\
-3.23279290849798	-1.16717205316684\\
-3.20953773849238	-1.0060117507856\\
-3.17263792532288	-0.830129432068425\\
-3.11947955211359	-0.638264643741304\\
-3.04706020690436	-0.429398878067323\\
-2.95206903691371	-0.202969729290941\\
-2.83106667794031	0.0408488926212685\\
-2.68079965641057	0.300817909394284\\
-2.49866665674673	0.574306299043266\\
-2.2833102427322	0.857058927982916\\
-2.03523660271512	1.14318345421905\\
-1.75728959879587	1.42547108532208\\
-1.45477290398947	1.69608663675106\\
-1.13507778488407	1.94752294706826\\
-0.806838945162423	2.17357884107625\\
-0.478831266741026	2.37007933016134\\
-0.158921038277483	2.53515197102043\\
0.14666436956898	2.66904955862832\\
0.433639963808184	2.77366049616377\\
0.699590316977137	2.85190376286616\\
0.943658792568718	2.90717031589308\\
1.16614417848768	2.94289755467673\\
1.36810982104501	2.9622952925113\\
1.55106012395173	2.96820108237492\\
};
\addplot [color=mycolor1, forget plot]
  table[row sep=crcr]{%
1.64345852363055	2.92105776339129\\
1.80761732204454	2.9159376568551\\
1.95528246259752	2.90191530665662\\
2.08838519009964	2.88068407505593\\
2.20868456958566	2.85357029991398\\
2.31774709165881	2.82159539265171\\
2.41694629827407	2.78553142969586\\
2.50747327914833	2.74594812184625\\
2.59035220798831	2.70325101153669\\
2.66645738332475	2.65771166454735\\
2.7365297390966	2.60949094897271\\
2.80119172506743	2.55865651553041\\
2.86096001123987	2.50519547473524\\
2.91625577678492	2.44902309758856\\
2.96741249260937	2.38998819246268\\
3.01468115556098	2.32787565263928\\
3.05823291728477	2.26240653502024\\
3.09815899310691	2.19323592423288\\
3.13446764802029	2.11994876026648\\
3.16707794468208	2.04205376673256\\
3.19580980735125	1.95897562065497\\
3.22036981287264	1.87004557129875\\
3.24033197908955	1.77449087509657\\
3.25511271117103	1.67142371408152\\
3.26393904345144	1.55983077873778\\
3.26580948222163	1.4385655265798\\
3.25944729491222	1.30634641319336\\
3.24324730307265	1.16176629450234\\
3.21521958208466	1.00332085639153\\
3.1729375897116	0.829467334288683\\
3.11350486983964	0.638728528193248\\
3.03356407943737	0.429859907970045\\
2.92938400748147	0.202096599130512\\
2.79707107094353	-0.0445127188891665\\
2.63295314192722	-0.308703031259661\\
2.43416129871963	-0.58758430600421\\
2.19937376746793	-0.876349005913475\\
1.92958437026846	-1.1682643309048\\
1.62865018731794	-1.45510280891482\\
1.30333820181883	-1.72804614070483\\
0.962706283169578	-1.97889373554604\\
0.616909840592004	-2.20122852165362\\
0.275783131818425	-2.39117340478182\\
-0.0523698660120398	-2.54754598280501\\
-0.361476459509195	-2.67147265521413\\
-0.647834840576828	-2.76569685549902\\
-0.909834199478792	-2.8338396057494\\
-1.14746606281625	-2.8797869376726\\
-1.36180400226045	-2.90727098253691\\
-1.55455572035072	-2.91963433147983\\
-1.72772588709667	-2.91973241682908\\
-1.88338728987833	-2.90992407129012\\
-2.02353978634332	-2.89211005318921\\
-2.15003280356409	-2.86779221172766\\
-2.26453015630305	-2.83813698352557\\
-2.36850105812257	-2.80403468242865\\
-2.46322609254644	-2.76615089014963\\
-2.54981079323019	-2.72496898281352\\
-2.62920226484629	-2.68082419945641\\
-2.7022061438885	-2.63393023468613\\
-2.76950238990311	-2.58439948452057\\
-2.83165912241808	-2.53225801075129\\
-2.88914413537482	-2.47745613762484\\
-2.94233393935166	-2.41987541944139\\
-2.991520274945	-2.35933254995108\\
-3.03691405420067	-2.29558063773091\\
-3.07864664870852	-2.22830815131749\\
-3.11676836891342	-2.15713574628113\\
-3.15124387830048	-2.08161112701245\\
-3.18194416388457	-2.00120207572529\\
-3.20863454636729	-1.915287813493\\
-3.23095806911326	-1.82314896648223\\
-3.24841347485075	-1.72395663315592\\
-3.26032690319308	-1.61676144428311\\
-3.2658164975705	-1.50048416440397\\
-3.2637494350632	-1.37391042204906\\
-3.25269172194939	-1.23569372924949\\
-3.23085280738982	-1.08437321816084\\
-3.19603021020376	-0.918415572441364\\
-3.14556463730224	-0.736294298285524\\
-3.0763241683432	-0.536622987018219\\
-2.984747049609	-0.31836059341864\\
-2.86698469390163	-0.0811021282462965\\
-2.71919399936644	0.174548415407669\\
-2.53801971706134	0.446560906457461\\
-2.32126736186821	0.731109536982981\\
-2.06868395448563	1.02240047830783\\
-1.78265173563995	1.31286886198096\\
-1.4685171815477	1.5938532425014\\
-1.13431162016923	1.85668637703656\\
-0.789815291758204	2.09393194184063\\
-0.445196810880552	2.3003844084033\\
-0.109648904470953	2.47353264858748\\
0.209589152140003	2.61342251208492\\
0.507643648353963	2.72208375164278\\
0.781911203897705	2.80278585825734\\
1.03164857637783	2.85934727107597\\
1.25745038849592	2.89561713910808\\
1.46075663711784	2.91515243254314\\
1.64345852363055	2.92105776339129\\
};
\addplot [color=mycolor1, forget plot]
  table[row sep=crcr]{%
1.74865638965985	2.88137384176368\\
1.9111836710517	2.87631163289344\\
2.05623578848918	2.8625432733052\\
2.18603530742935	2.84184384607306\\
2.30256716606395	2.81558335823543\\
2.40756928437339	2.78480240242788\\
2.50254327140952	2.75027755034177\\
2.58877477830147	2.71257508773336\\
2.66735724515472	2.67209371664647\\
2.73921552628005	2.62909765426629\\
2.80512756595767	2.58374173106424\\
2.86574328648835	2.53608997072186\\
2.92160039629944	2.48612890195269\\
2.97313709926905	2.43377659779317\\
3.0207017959453	2.37888820127053\\
3.06455987921545	2.32125849205571\\
3.10489767976282	2.26062187935964\\
3.14182353134469	2.19665007013035\\
3.17536581272197	2.12894755657744\\
3.20546768582882	2.05704499373496\\
3.23197808934203	1.98039050162045\\
3.25463836423213	1.89833894078851\\
3.27306368856977	1.81013929990314\\
3.28671829938547	1.71492054271635\\
3.29488331912033	1.61167665999203\\
3.29661596543324	1.49925236882212\\
3.29069916429219	1.37633205880754\\
3.27558139366417	1.24143642297809\\
3.24930844435586	1.09293399751025\\
3.20945245739726	0.929078802553234\\
3.15305014452637	0.748090410689812\\
3.07657268871799	0.548298317047832\\
2.97596499191847	0.328376066452439\\
2.84680977757094	0.0876869626864989\\
2.68468495329439	-0.17325657939976\\
2.48577362462204	-0.452264497626377\\
2.24772972631639	-0.744992585231754\\
1.97067943599865	-1.04472501879821\\
1.65807220993474	-1.34265634595364\\
1.31698343105269	-1.62881221959049\\
0.957554514883364	-1.89348839583847\\
0.591585930754306	-2.12878970482027\\
0.230708359684339	-2.32973727022693\\
-0.115245097273801	-2.49460300242127\\
-0.439198726834851	-2.62449590283836\\
-0.73702122592129	-2.72250681712574\\
-1.0071386753611	-2.7927748429065\\
-1.24987806314248	-2.83972219698163\\
-1.46679045121205	-2.86754714735139\\
-1.66009022809612	-2.87995479744129\\
-1.83225170660669	-2.88005998154867\\
-1.98574995545417	-2.87039434462988\\
-2.12291225381591	-2.85296575291794\\
-2.24584534304923	-2.82933679992632\\
-2.35641026393456	-2.80070390154217\\
-2.45622465215348	-2.76796819319801\\
-2.54667927319293	-2.7317951045581\\
-2.62896065345727	-2.69266242953424\\
-2.70407508028692	-2.6508980274472\\
-2.77287140616133	-2.60670872452633\\
-2.8360613932267	-2.56020198090403\\
-2.89423707595181	-2.51140169625282\\
-2.9478850124543	-2.46025927635858\\
-2.99739747596487	-2.40666083432898\\
-3.04308069233051	-2.3504311788191\\
-3.08516020825859	-2.29133505490838\\
-3.12378340691881	-2.22907595079856\\
-3.15901908747461	-2.16329266303052\\
-3.19085389967043	-2.0935537234282\\
-3.21918527572757	-2.01934973515528\\
-3.24381033001434	-1.94008365202977\\
-3.26441000484119	-1.85505908316613\\
-3.28052753819491	-1.76346684744097\\
-3.29154014235314	-1.66437029480382\\
-3.29662266873162	-1.55669044235931\\
-3.29470210917163	-1.4391928762441\\
-3.28440226177932	-1.31047983674861\\
-3.26397914640617	-1.16899318217089\\
-3.23125040615276	-1.01303728294905\\
-3.18352689318838	-0.840835481318406\\
-3.11756307229887	-0.650639262158281\\
-3.02955578265117	-0.440914290992723\\
-2.91523790632757	-0.210628333221629\\
-2.77013032758312	0.0403443027574127\\
-2.59002029256205	0.310719099018591\\
-2.37170493501425	0.597277998955034\\
-2.11394990123552	0.894492241963369\\
-1.81846128678003	1.19452690062365\\
-1.49051303930177	1.48783794592996\\
-1.13884197202703	1.76438739211616\\
-0.774635244305867	2.01519847780868\\
-0.409838890092488	2.23374014663527\\
-0.055353551683624	2.4166687307997\\
0.280296632940744	2.5637629907927\\
0.591527576299578	2.67724232125039\\
0.875558910164217	2.76083170511072\\
1.13185505571184	2.8188917889414\\
1.36143427945912	2.85578004516305\\
1.56623969593983	2.87546934518756\\
1.74865638965985	2.88137384176368\\
};
\addplot [color=mycolor1, forget plot]
  table[row sep=crcr]{%
1.86949259182206	2.851041894416\\
2.03018408192628	2.84604446976994\\
2.17237182230191	2.83255422748916\\
2.29861218587825	2.81242751649097\\
2.41114071012953	2.78707343834094\\
2.51187956915807	2.75754580885987\\
2.60246366961384	2.72461978989047\\
2.68427383880935	2.68885303783657\\
2.75847075137008	2.65063311987221\\
2.82602639165477	2.61021350747583\\
2.88775166141764	2.56774038583839\\
2.94431972289547	2.52327219864455\\
2.99628515378765	2.47679346786944\\
3.0440991929524	2.42822406951643\\
3.0881213996135	2.37742483755954\\
3.12862800618561	2.32420011403568\\
3.16581715426406	2.2682976576057\\
3.19981108386009	2.20940615768815\\
3.2306552047691	2.14715046917842\\
3.2583138162204	2.08108457955932\\
3.28266205296936	2.01068224645031\\
3.30347341769176	1.9353252070536\\
3.32040200762152	1.85428888022718\\
3.3329582615423	1.766725592993\\
3.3404767620074	1.67164562940943\\
3.34207438265767	1.56789692668809\\
3.33659699577093	1.45414520093235\\
3.32255330074806	1.3288579322199\\
3.29803556898633	1.19029834444202\\
3.26063004462054	1.0365397390164\\
3.20732570918268	0.865516700806141\\
3.13444095158236	0.675137754701349\\
3.03760531141148	0.463492569463746\\
2.91185824557558	0.229191149003763\\
2.751953616705	-0.0281375813465989\\
2.55296987173708	-0.307201442265731\\
2.31128645067285	-0.604357222621918\\
2.02585325470708	-0.913112149382156\\
1.69944506937969	-1.22415553322009\\
1.33936031707961	-1.52621725105204\\
0.957020192028156	-1.8077470468533\\
0.566323354814297	-2.05894306961403\\
0.181253089060065	-2.27336840108622\\
-0.186359555833534	-2.44856971763358\\
-0.528173197105508	-2.58564108209243\\
-0.839579847232978	-2.68814030184989\\
-1.11914634816221	-2.76088306328742\\
-1.36770876357337	-2.80897125440314\\
-1.58747699763472	-2.83717491568213\\
-1.78132765176004	-2.84962817416705\\
-1.95232315093529	-2.84974105301399\\
-2.10342373972285	-2.84023327533267\\
-2.23733814474857	-2.8232230272202\\
-2.35646319745511	-2.80033068410671\\
-2.46287530736805	-2.77277712795278\\
-2.55834911181294	-2.7414682639622\\
-2.64438817914622	-2.70706376792262\\
-2.72225912478861	-2.67003109368689\\
-2.79302456833141	-2.63068688472884\\
-2.85757276629238	-2.58922811280272\\
-2.91664309970992	-2.54575503873085\\
-2.97084729611412	-2.50028772566202\\
-3.02068658959441	-2.45277745981182\\
-3.0665651341151	-2.40311409839735\\
-3.10879997935108	-2.35113008330869\\
-3.14762784831883	-2.29660163007258\\
-3.18320884957155	-2.23924741727532\\
-3.21562712602563	-2.17872495371844\\
-3.24488829079953	-2.11462468320623\\
-3.27091332569106	-2.04646179793107\\
-3.29352841538703	-1.97366567471089\\
-3.31244995552533	-1.89556683644361\\
-3.32726370464709	-1.8113814004094\\
-3.33739675935027	-1.72019315351482\\
-3.34208075361804	-1.62093377446207\\
-3.3403045017562	-1.5123624388513\\
-3.33075439751556	-1.39304730665817\\
-3.31174159508735	-1.2613535150509\\
-3.28111695661172	-1.11544570406821\\
-3.23617900766813	-0.953318255767754\\
-3.17358828797476	-0.772873598414232\\
-3.08931551102783	-0.572077543716153\\
-2.97867232411474	-0.349227927358609\\
-2.83650045946101	-0.103371682617185\\
-2.6576173160146	0.165117691616635\\
-2.43760740047999	0.453853738244781\\
-2.17396722216338	0.757806234807604\\
-1.86742258058554	1.06902256126122\\
-1.52298134004091	1.37704825677867\\
-1.15013513350289	1.67022508059427\\
-0.761815116812467	1.93763085392071\\
-0.372273819119186	2.17099814937115\\
0.00534641066062597	2.36587592682938\\
0.360863004955486	2.52169235164974\\
0.687831398489772	2.64092765973912\\
0.983332883262683	2.72791009701897\\
1.24718773674731	2.78769820053344\\
1.48102129027608	2.82528353001849\\
1.68745078808525	2.84514024371596\\
1.86949259182206	2.851041894416\\
};
\addplot [color=mycolor1, forget plot]
  table[row sep=crcr]{%
2.00970785227068	2.83255951582902\\
2.16829459506053	2.82763576051588\\
2.30731523094934	2.81445258780785\\
2.4297046947207	2.79494515119588\\
2.53797278103981	2.77055534078189\\
2.6342363110595	2.74234305753359\\
2.72026598535531	2.71107549953793\\
2.79753589144043	2.67729624992662\\
2.86726977473392	2.641377482169\\
2.93048162747608	2.60355872384335\\
2.98800995892579	2.56397518676149\\
3.04054597826645	2.52267808539351\\
3.0886562731758	2.47964880147187\\
3.13280064745708	2.43480827137261\\
3.1733457264768	2.38802258405018\\
3.21057482112467	2.33910547032397\\
3.24469439457135	2.2878181226942\\
3.27583731546928	2.23386659248464\\
3.30406290887543	2.17689685470301\\
3.3293536276346	2.11648750120062\\
3.35160795356739	2.05213991488855\\
3.37062888832095	1.98326569362213\\
3.38610709619862	1.90917104296972\\
3.39759740635128	1.8290378685784\\
3.40448696981579	1.74190142248968\\
3.40595292314895	1.64662468673838\\
3.40090701464311	1.54187037394823\\
3.3879244932748	1.42607275936865\\
3.36515505297591	1.29741396377849\\
3.33021556567402	1.15381342249475\\
3.28006917739702	0.992945935825488\\
3.21090547265683	0.812313688080073\\
3.11805519292571	0.609410871367224\\
2.9960036943785	0.382033114912436\\
2.83860917803556	0.128788191766161\\
2.63967063205985	-0.150161553980915\\
2.39398493840761	-0.452182928665707\\
2.09890926629675	-0.7713121847596\\
1.75614168558561	-1.09789400392915\\
1.37302381558774	-1.41923798734738\\
0.962478354853678	-1.72151318483291\\
0.541122414584741	-1.99241543448791\\
0.126086512136677	-2.22353565354694\\
-0.268133107704887	-2.41143559577134\\
-0.631577552867782	-2.55720324292457\\
-0.959131116260233	-2.66503948191669\\
-1.24968021080543	-2.74066012508232\\
-1.50483046760844	-2.79004009696308\\
-1.72770386342792	-2.81865646246515\\
-1.92204364433203	-2.83115261151491\\
-2.09164728934969	-2.83127375429125\\
-2.24005708169665	-2.82194264110306\\
-2.37042229740948	-2.80538913525812\\
-2.48546291352305	-2.78328649808138\\
-2.58748692100441	-2.75687309892156\\
-2.67843186444601	-2.72705265847209\\
-2.75991406419349	-2.69447306987477\\
-2.8332769777984	-2.65958660284275\\
-2.8996347939691	-2.62269497831406\\
-2.95990987145525	-2.58398257526998\\
-3.01486390614539	-2.54354048910545\\
-3.06512327954946	-2.50138357319334\\
-3.11119923529598	-2.45746206919823\\
-3.15350353100543	-2.41166899718123\\
-3.19236012037668	-2.36384413054485\\
-3.22801328528932	-2.3137751086922\\
-3.26063248329872	-2.26119602514444\\
-3.29031400960702	-2.20578365590005\\
-3.31707939300565	-2.1471513506101\\
-3.34087024559752	-2.08484049055914\\
-3.36153905617021	-2.01830932075474\\
-3.37883514501698	-1.94691889450658\\
-3.39238467255464	-1.86991584538964\\
-3.40166320955607	-1.78641176052254\\
-3.40595894338724	-1.69535914022167\\
-3.40432415968643	-1.59552441581267\\
-3.39551232876152	-1.48545947032062\\
-3.37789822385948	-1.36347491041274\\
-3.34937958137217	-1.22762150438802\\
-3.30726196979925	-1.07569147765284\\
-3.24813565478921	-0.905259609032158\\
-3.16776724190762	-0.7137958141245\\
-3.06105331900569	-0.498894981689488\\
-2.92212022887542	-0.258680560012784\\
-2.74469760922797	0.00756902553045162\\
-2.5229174330319	0.298574963947673\\
-2.25263627569165	0.610127618005103\\
-1.93316830403618	0.934409971105092\\
-1.5689363762796	1.26008813656112\\
-1.17019380909004	1.57359617920111\\
-0.752052372940286	1.86152328722302\\
-0.331816374923651	2.11328116118183\\
0.0743649903603743	2.32291239975837\\
0.454129450740461	2.48937682241393\\
0.799989615652643	2.61552393552942\\
1.10897926826103	2.70649812614107\\
1.38150841585277	2.76827057497105\\
1.62007426223507	2.80663226408797\\
1.82819971892363	2.82666489950488\\
2.00970785227068	2.83255951582902\\
};
\addplot [color=mycolor1, forget plot]
  table[row sep=crcr]{%
2.17430603473841	2.82928793073052\\
2.33044535777815	2.82444897168392\\
2.46594297362775	2.81160678704818\\
2.58415716708103	2.79277030050891\\
2.68789254387122	2.76940595677011\\
2.77946699484521	2.7425714670914\\
2.86078573906653	2.71301902411803\\
2.93341104171975	2.68127265050854\\
2.9986230753596	2.64768508851882\\
3.05747086442469	2.61247907686435\\
3.11081384046105	2.57577692251115\\
3.15935513715665	2.53762134677768\\
3.20366787001532	2.49798979992626\\
3.24421554049886	2.45680381827628\\
3.28136751247959	2.41393452310087\\
3.31541029254594	2.36920500026929\\
3.3465551303234	2.32239002374647\\
3.37494224630124	2.27321336944935\\
3.40064178827692	2.22134278883848\\
3.42365140325324	2.16638255933344\\
3.44389007534897	2.10786339134407\\
3.46118760463784	2.04522934489396\\
3.47526876652081	1.97782129527289\\
3.48573077423729	1.90485640152273\\
3.49201214853071	1.82540300920684\\
3.49335047065297	1.73835053234855\\
3.48872578597209	1.64237424725429\\
3.47678575017332	1.53589584679112\\
3.45574827300754	1.41704249382237\\
3.42327810810487	1.28361072542021\\
3.37633699194203	1.13304806285973\\
3.31101526097114	0.962476172308744\\
3.22237097655495	0.768796474817825\\
3.10433695290387	0.548942043211334\\
2.9498123210435	0.300361978709517\\
2.75112793000477	0.0218247814546395\\
2.50112454582504	-0.285440449235098\\
2.19500744060268	-0.616443644580409\\
1.83279348396173	-0.961489600474761\\
1.42152136958137	-1.3063970008595\\
0.975845648074582	-1.63450693799913\\
0.515964639544959	-1.93017020234628\\
0.0633211215265111	-2.18224450162264\\
-0.363949894265142	-2.38592186002566\\
-0.753796552421497	-2.54230740615597\\
-1.10060884720806	-2.65651208684665\\
-1.40389862561362	-2.73547347665808\\
-1.66643431787696	-2.78630322905355\\
-1.89260401811819	-2.81535915112738\\
-2.08728476369964	-2.82789001873507\\
-2.25518878499693	-2.82801992101541\\
-2.40055085490442	-2.81888822167606\\
-2.52702294235802	-2.80283518923491\\
-2.63767930270744	-2.78157977328078\\
-2.73507199115711	-2.75636932685319\\
-2.82130341222054	-2.72809764441867\\
-2.89809909341037	-2.69739456722559\\
-2.96687326596811	-2.66469244542933\\
-3.02878476984102	-2.63027467163603\\
-3.08478318424195	-2.5943106773748\\
-3.13564609696506	-2.55688082262324\\
-3.18200873955121	-2.51799374438073\\
-3.22438719765583	-2.47759802849779\\
-3.26319624623718	-2.43558952510388\\
-3.29876265010897	-2.39181521440426\\
-3.33133455291616	-2.34607421503113\\
-3.36108736562728	-2.29811628370925\\
-3.38812635908966	-2.24763796026885\\
-3.4124859564092	-2.1942763488638\\
-3.43412549761435	-2.13760038238197\\
-3.45292099547021	-2.0770992851559\\
-3.4686520985225	-2.01216782796055\\
-3.48098310405704	-1.94208786685168\\
-3.48943639793565	-1.86600559716315\\
-3.49335612455861	-1.7829039871982\\
-3.49185921454444	-1.69157008370894\\
-3.4837701826888	-1.59055749156644\\
-3.46753554642006	-1.47814566124369\\
-3.44111377096973	-1.35230025189567\\
-3.40183832761321	-1.21064372066799\\
-3.34625674726662	-1.05045381725873\\
-3.26996105481103	-0.868721532579536\\
-3.16745032441434	-0.662320310385892\\
-3.03211101926641	-0.428362470131523\\
-2.85646733273374	-0.164834179270465\\
-2.63292320715374	0.128425970177643\\
-2.35522203042063	0.448465239159584\\
-2.02065569486654	0.788006644772256\\
-1.63254200363741	1.13497995420302\\
-1.20180828192215	1.47359901929041\\
-0.746311741639337	1.78722909211069\\
-0.287482595077459	2.06211021974623\\
0.15439697822222	2.29018412674959\\
0.56404269287998	2.46977369323667\\
0.932710877487586	2.60426859699022\\
1.2575759170864	2.69994355321456\\
1.54000858568913	2.76398340384577\\
1.78376187044044	2.80319756768445\\
1.99357716381973	2.82340732367622\\
2.17430603473841	2.82928793073052\\
};
\addplot [color=mycolor1, forget plot]
  table[row sep=crcr]{%
2.37008966809432	2.84583467340354\\
2.52336101304862	2.84109395545007\\
2.65493105339172	2.82863117278086\\
2.76862267778487	2.81052085769044\\
2.86755001510834	2.7882437769117\\
2.95423266046411	2.76284622438936\\
3.03070422684652	2.73505813212284\\
3.09860634829873	2.70537870361567\\
3.15926627203745	2.6741376613985\\
3.2137591926997	2.64153863503644\\
3.26295751751183	2.60768960954319\\
3.30756937735905	2.57262400672931\\
3.34816845266683	2.53631493276348\\
3.38521681902747	2.49868435450491\\
3.41908214635702	2.45960840435614\\
3.45005024766046	2.41891960171226\\
3.47833367691025	2.37640647248396\\
3.50407681296056	2.33181081137422\\
3.5273576240679	2.28482263812596\\
3.54818606830286	2.23507272898644\\
3.56649882899361	2.18212244331108\\
3.58214978799726	2.12545040199865\\
3.59489527503244	2.06443540369702\\
3.60437266468315	1.99833478863803\\
3.6100702845173	1.92625729534473\\
3.6112858071673	1.84712934765145\\
3.60706929830707	1.7596537595386\\
3.59614590411288	1.66226026082249\\
3.57681194832122	1.55304843364005\\
3.54679746572038	1.42972636836512\\
3.50308914992433	1.28955395370645\\
3.44171303509782	1.1293104537239\\
3.35749134537993	0.945325142230009\\
3.24382230805877	0.733640633043858\\
3.09259867497051	0.490420643488118\\
2.89448951307269	0.212751666019624\\
2.63993722127874	-0.100030692797326\\
2.321252018958	-0.44454182182079\\
1.93586348388899	-0.811582534465944\\
1.48987037994912	-1.18554016734986\\
0.999850167481938	-1.54625462069504\\
0.490800302083166	-1.87351819056015\\
-0.00976760886881658	-2.15229736504357\\
-0.47862074310768	-2.3758291948706\\
-0.900969397609134	-2.54529082015793\\
-1.27082671096515	-2.66712019821288\\
-1.58885251183779	-2.74994862101959\\
-1.85957398047222	-2.80238751830318\\
-2.08913881445424	-2.83189831408148\\
-2.28390274429081	-2.84444872664284\\
-2.44970755124551	-2.84458774491333\\
-2.59160185077621	-2.83568206418269\\
-2.71380175301886	-2.82017758069955\\
-2.81976180515058	-2.79982915091515\\
-2.91228461517198	-2.77588316498957\\
-2.99363402781255	-2.74921520687656\\
-3.06563694687861	-2.72043083707034\\
-3.12976911307429	-2.68993811207581\\
-3.18722482786806	-2.65799919115049\\
-3.2389724585397	-2.62476673107416\\
-3.28579804751296	-2.59030927637789\\
-3.32833924397085	-2.55462866090615\\
-3.36711145022197	-2.51767153878045\\
-3.40252769945223	-2.47933650369906\\
-3.43491342463749	-2.43947777474164\\
-3.46451696149501	-2.39790607275902\\
-3.49151635019761	-2.35438704353505\\
-3.51602275007773	-2.30863737162493\\
-3.53808054304273	-2.2603185491965\\
-3.55766395638719	-2.20902810013721\\
-3.57466976262089	-2.15428789845941\\
-3.58890528732447	-2.09552905336005\\
-3.60007054506754	-2.03207265890987\\
-3.60773279122859	-1.96310553255902\\
-3.61129108299517	-1.8876499221355\\
-3.60992754867974	-1.80452611441937\\
-3.60254096123919	-1.71230707561059\\
-3.58765697617668	-1.60926498692926\\
-3.56330832071048	-1.49331136579171\\
-3.52687810256052	-1.36193640716485\\
-3.47490208222548	-1.21216100573125\\
-3.40283508129398	-1.04052939130352\\
-3.30480996090834	-0.843194970645823\\
-3.17346649956951	-0.61618907910191\\
-3.00001523271263	-0.356005884048084\\
-2.77482637545278	-0.060655994624172\\
-2.48893587499026	0.26874259104347\\
-2.13675360062321	0.626078821425667\\
-1.71965020354387	0.998892456188959\\
-1.24894356722181	1.36887963300529\\
-0.745958132188205	1.71518129242515\\
-0.237835878231387	2.01959717603921\\
0.249302172258792	2.27105681113961\\
0.696170609995157	2.46700138361849\\
1.09254058306598	2.61163994282778\\
1.43609585729356	2.71285279072919\\
1.72975951920885	2.77946657892918\\
1.97909740102934	2.81960054669284\\
2.19048887725582	2.83997842985783\\
2.37008966809432	2.84583467340354\\
};
\addplot [color=mycolor1, forget plot]
  table[row sep=crcr]{%
2.6064589828494	2.88862616640401\\
2.75637109414822	2.883999158983\\
2.88357674785248	2.87195708876679\\
2.99239608382493	2.85462841786627\\
3.08625831948619	2.83349617714752\\
3.16787772828815	2.80958545644448\\
3.23940444300088	2.7835968756565\\
3.30254662301602	2.75600014215947\\
3.35866654942507	2.72709908273117\\
3.40885502240302	2.69707659985484\\
3.4539884661676	2.66602556057904\\
3.4947725378337	2.63396979088962\\
3.53177528488427	2.60087803167415\\
3.56545219557076	2.56667278660369\\
3.59616489562369	2.53123534419147\\
3.62419476499703	2.49440779820407\\
3.64975236089754	2.4559925584995\\
3.67298321301724	2.41574959242231\\
3.69397027804979	2.37339143202941\\
3.71273307794191	2.32857580000093\\
3.72922327387838	2.28089552774105\\
3.74331611674247	2.22986524679513\\
3.7547968298526	2.17490411565226\\
3.76334047684362	2.11531358795446\\
3.76848318927127	2.05024893201245\\
3.76958170216264	1.9786828892658\\
3.76575688652169	1.89935956348535\\
3.7558152982188	1.81073649949425\\
3.73814067326371	1.71091325142572\\
3.71054500789473	1.59754622004815\\
3.67006719259592	1.46775350112306\\
3.61270839144689	1.31802251130379\\
3.53310287780543	1.14415182510341\\
3.42415232901896	0.941294042417582\\
3.27672169828017	0.704226331451283\\
3.07963497410675	0.428057684036333\\
2.82043165537474	0.109644908416964\\
2.48755709893795	-0.250105954072952\\
2.07450524984844	-0.643388213434865\\
1.58529271977403	-1.05349246283719\\
1.03845463107716	-1.455972350082\\
0.465526666727522	-1.82428685022561\\
-0.0971822994778088	-2.13769698101945\\
-0.619092708293362	-2.38657010343355\\
-1.0818079135853	-2.57227964755975\\
-1.47930913785865	-2.70326146416607\\
-1.81429707356905	-2.79054540480154\\
-2.09396044839245	-2.84474516538712\\
-2.32688763075065	-2.87470949979353\\
-2.52134887775033	-2.88725599524887\\
-2.68456326331782	-2.88740426406207\\
-2.82251915336173	-2.8787542171408\\
-2.94005166132089	-2.8638482597678\\
-3.04101234893106	-2.84446473367481\\
-3.12845234049856	-2.82183799344284\\
-3.20478679703347	-2.79681700764569\\
-3.27193131730941	-2.76997725891164\\
-3.3314105736454	-2.74169880644514\\
-3.38444298534991	-2.71222039102835\\
-3.43200595019206	-2.68167673487756\\
-3.47488577361502	-2.65012405459818\\
-3.51371571737217	-2.61755724492019\\
-3.54900485010229	-2.58392108518757\\
-3.58115973412663	-2.54911704627594\\
-3.61050044905449	-2.513006731903\\
-3.63727202218022	-2.47541259993116\\
-3.6616519849773	-2.43611632205393\\
-3.68375447858502	-2.39485491537756\\
-3.70363106361525	-2.35131458846271\\
-3.72126812544317	-2.3051220656886\\
-3.73658047820497	-2.25583296979466\\
-3.74940042748902	-2.20291663828086\\
-3.75946111404373	-2.14573651277383\\
-3.76637237832512	-2.08352496403139\\
-3.76958659341788	-2.01535110261679\\
-3.76835083198556	-1.94007980586714\\
-3.76164027613148	-1.85631995225746\\
-3.74806589202947	-1.76235990295578\\
-3.72574714829292	-1.65608906519836\\
-3.6921384105946	-1.53490686388163\\
-3.64379699476674	-1.3956265130413\\
-3.57608526073647	-1.23439409674593\\
-3.48281657465312	-1.04666947244277\\
-3.35590135639156	-0.827362343558006\\
-3.18515089500215	-0.571289645864595\\
-2.95857988607527	-0.274202847881349\\
-2.6637901042153	0.0653566355405698\\
-2.2911130842731	0.443383637526595\\
-1.83861873047926	0.847729220696273\\
-1.31734186160238	1.25738786218882\\
-0.752965148538522	1.64591896811454\\
-0.180742070492837	1.98874283474401\\
0.364702770050638	2.27033687843499\\
0.858482795179477	2.48690166765121\\
1.28869281512832	2.64393862627122\\
1.6542297268519	2.75166978874569\\
1.96051424731082	2.82117963527452\\
2.21572948658655	2.86228446899281\\
2.42844674742143	2.88280838763335\\
2.6064589828494	2.88862616640401\\
};
\addplot [color=mycolor1, forget plot]
  table[row sep=crcr]{%
2.89659432980286	2.96676193805005\\
3.04261530234076	2.96226512107439\\
3.16502460886113	2.95068440839662\\
3.26865741175649	2.93418704719984\\
3.35725109593054	2.91424503832653\\
3.43369849912562	2.89185260649569\\
3.50024831591357	2.86767479814833\\
3.55865857421896	2.84214816144364\\
3.61031232001395	2.81554874916724\\
3.65630421859564	2.78803798830021\\
3.69750524874502	2.7596935331198\\
3.7346110320036	2.73052984690876\\
3.76817793593031	2.70051165407789\\
3.7986499818713	2.66956232868485\\
3.82637874357441	2.63756856028971\\
3.85163778682918	2.60438214155117\\
3.87463271664979	2.56981937111724\\
3.89550751961209	2.53365830398773\\
3.91434757504091	2.49563387047219\\
3.93117942554675	2.45543069532303\\
3.94596711272854	2.41267325741374\\
3.95860456359512	2.36691281659495\\
3.96890311675038	2.31761027725621\\
3.97657275161087	2.26411383557913\\
3.98119485614645	2.20562984672729\\
3.98218333748952	2.14118483016385\\
3.97872940632414	2.06957590343883\\
3.96972327094597	1.9893062437335\\
3.95364306233142	1.89850159302278\\
3.9283974564574	1.79480380868164\\
3.89110391335995	1.675239124805\\
3.83778061354591	1.53606460590245\\
3.76293148666367	1.37261143308965\\
3.6590218454198	1.17917817429086\\
3.51590365944118	0.949097715740115\\
3.32040545205894	0.675227100275456\\
3.05662035949933	0.351281984348361\\
2.70790143678588	-0.0254718849548195\\
2.26183065135216	-0.45005683315673\\
1.71833479132094	-0.905539269966142\\
1.09751946508794	-1.36238128704534\\
0.439952295888921	-1.78508392229778\\
-0.204892158233907	-2.14427478985174\\
-0.795522309411559	-2.42598247372705\\
-1.30879486149514	-2.63205385457764\\
-1.73948386246382	-2.7740324841352\\
-2.09385275020469	-2.86641317063372\\
-2.38310831218506	-2.9225058993971\\
-2.61920035292473	-2.95290136813909\\
-2.81284514872881	-2.96541205374497\\
-2.97290932641795	-2.96556939715673\\
-3.10644015794655	-2.95720537493408\\
-3.2189319875517	-2.94294495808261\\
-3.31463614585934	-2.92457530690133\\
-3.39683944813556	-2.90330721565073\\
-3.46809074661824	-2.87995516111223\\
-3.53037679839086	-2.85505969363724\\
-3.58525568559304	-2.82897019776391\\
-3.63395695620088	-2.80190075278642\\
-3.67745648029622	-2.77396777398898\\
-3.71653236360908	-2.74521525198787\\
-3.75180672247609	-2.71563145415037\\
-3.78377686886483	-2.68515963854271\\
-3.81283848422717	-2.65370444876066\\
-3.83930262805466	-2.62113505968558\\
-3.86340787361186	-2.58728572894365\\
-3.88532843746716	-2.55195410850564\\
-3.90517882758428	-2.51489743876753\\
-3.92301524024991	-2.47582655031098\\
-3.93883365621706	-2.43439741085335\\
-3.95256428852306	-2.39019975498695\\
-3.96406168175765	-2.34274210155407\\
-3.97308930908288	-2.29143217652475\\
-3.9792968971866	-2.23555139539052\\
-3.98218784472244	-2.17422159649788\\
-3.98107286747296	-2.1063616421403\\
-3.97500424384076	-2.0306308336959\\
-3.96268255309857	-1.9453554158036\\
-3.94232441658273	-1.84843405923733\\
-3.91147547523812	-1.73721883569902\\
-3.86674834272132	-1.6083714994132\\
-3.80346313929779	-1.45770448476151\\
-3.71517576571336	-1.28003921712422\\
-3.59311423455181	-1.06916436795909\\
-3.42564401367706	-0.81807283846141\\
-3.1981127735367	-0.519810584526548\\
-2.89383572578272	-0.169431726839988\\
-2.49743742084217	0.232526788235511\\
-2.00153691664825	0.675524368527635\\
-1.41545564256883	1.1360006086817\\
-0.770243446554317	1.58012503409236\\
-0.11298263180075	1.9739034807845\\
0.508902205780387	2.29501428786976\\
1.06251140744219	2.53789012609105\\
1.53425404867556	2.71015448202869\\
1.92556178584946	2.82553485172154\\
2.24585827135187	2.89826474559884\\
2.50708931975046	2.9403670821331\\
2.72072963620371	2.9610001758028\\
2.89659432980286	2.96676193805005\\
};
\addplot [color=mycolor1, forget plot]
  table[row sep=crcr]{%
3.25912552532344	3.09324266531364\\
3.40075479465786	3.08889121951157\\
3.51801223395664	3.07780502924075\\
3.61624480213526	3.06217246485952\\
3.69947648861053	3.04344120313106\\
3.77075341827375	3.02256613476829\\
3.83239965728367	3.00017203474404\\
3.88620378798641	2.97666014161672\\
3.93355428317857	2.95227813953397\\
3.97553779282755	2.92716623650056\\
4.01301075784132	2.90138751585974\\
4.04665182531094	2.87494781248438\\
4.07700035875947	2.84780848553431\\
4.1044847677202	2.8198942489043\\
4.12944326369463	2.79109743186516\\
4.1521388515986	2.76127951769643\\
4.17276978659718	2.73027044613358\\
4.19147629157254	2.69786590048875\\
4.20834398499086	2.66382258803285\\
4.22340416900177	2.62785133053263\\
4.23663083508319	2.58960758460892\\
4.24793392093488	2.5486787841831\\
4.25714795241982	2.50456761286711\\
4.26401466840201	2.45666994036201\\
4.26815746931866	2.40424565290605\\
4.2690444268092	2.34637992245714\\
4.26593495471885	2.28193153358199\\
4.2578027961164	2.20946366815217\\
4.2432243386947	2.12715103461714\\
4.22021593992867	2.03265559398637\\
4.18599646687352	1.92296203525389\\
4.13664176158023	1.79416546599789\\
4.06658864013277	1.64121227545014\\
3.96794662741513	1.45762193816308\\
3.82961331960112	1.23528635760583\\
3.63633210909192	0.96459705871368\\
3.36821531474038	0.63543950469671\\
3.002045454138	0.239976599825101\\
2.51667707823438	-0.221836445673655\\
1.9044993812272	-0.734699889847041\\
1.18580916303899	-1.2634372751351\\
0.413743826998447	-1.7597067026252\\
-0.341863310285406	-2.18064657921402\\
-1.02281270170815	-2.50552986704684\\
-1.59982705916651	-2.73729161093054\\
-2.07028016133366	-2.89245873154349\\
-2.4465644340315	-2.99061032245346\\
-2.74589996190596	-3.04869727856406\\
-2.98479716449934	-3.07948042799385\\
-3.17703046646929	-3.09191777405463\\
-3.33338405234074	-3.09208364407099\\
-3.46206006354589	-3.08403218295096\\
-3.56922864873687	-3.07045261226997\\
-3.65952583873925	-3.05312518023861\\
-3.73644959712913	-3.03322632582248\\
-3.802657834974	-3.01152960911262\\
-3.86018713294046	-2.98853736603995\\
-3.91061169893487	-2.96456703941671\\
-3.95515868121247	-2.93980794958082\\
-3.99479201962409	-2.91435870000864\\
-4.03027367842802	-2.8882517715398\\
-4.06220855779067	-2.8614695133002\\
-4.09107752686464	-2.83395423114886\\
-4.11726169670518	-2.80561409923711\\
-4.14106010818103	-2.77632597920563\\
-4.16270233220879	-2.74593579823306\\
-4.18235698062868	-2.71425683029612\\
-4.20013674200496	-2.68106599098774\\
-4.21610023890133	-2.64609805777553\\
-4.23025071160572	-2.60903753620112\\
-4.24253123028556	-2.569507683344\\
-4.25281578214716	-2.52705594759409\\
-4.2608951209982	-2.48113475918942\\
-4.26645563286624	-2.43107617280046\\
-4.2690485600191	-2.37605827577271\\
-4.26804558369017	-2.31506047765258\\
-4.26257476548153	-2.24680372935623\\
-4.25142785944008	-2.16967034780763\\
-4.23292558549316	-2.08159650663929\\
-4.20472109273816	-1.97992895305211\\
-4.16351325898368	-1.86123725761094\\
-4.1046316041567	-1.72107692250579\\
-4.02144842120085	-1.55371432541402\\
-3.90458758851443	-1.35186824245691\\
-3.74097753180596	-1.106627367739\\
-3.51303771445789	-0.807919823449194\\
-3.19886052999548	-0.446266543352548\\
-2.77522862261228	-0.0168551754644445\\
-2.22596970505169	0.473625417843501\\
-1.55583841724168	0.99997826760293\\
-0.802144912027912	1.51868967653404\\
-0.0296675510206517	1.98150862124943\\
0.694232425793066	2.35537571235375\\
1.32497545153508	2.63219319355454\\
1.84780947670892	2.82320517706321\\
2.26914351855745	2.94750772659703\\
2.60476612527971	3.02376569855963\\
2.87197017953517	3.06686310061952\\
3.08600600689612	3.08755613156197\\
3.25912552532344	3.09324266531364\\
};
\addplot [color=mycolor1, forget plot]
  table[row sep=crcr]{%
3.72011336764885	3.28650261533118\\
3.85702413168351	3.28230597003519\\
3.96897610105512	3.27172810514178\\
4.06180092705263	3.25696084177994\\
4.13977420248845	3.23941639939732\\
4.20606357291101	3.22000459090589\\
4.26304295070732	3.19930775068244\\
4.31251211391209	3.17769171474465\\
4.35585073130903	3.15537674917136\\
4.39412715861489	3.13248318681217\\
4.42817594083798	3.10906089241272\\
4.45865349172613	3.08510821856777\\
4.4860783837653	3.06058398783418\\
4.51086062960552	3.03541471430641\\
4.5333229443513	3.00949844243116\\
4.55371602420306	2.98270603911308\\
4.57222920919513	2.95488040947898\\
4.58899741369647	2.92583384332133\\
4.60410483656924	2.89534348991946\\
4.61758565067858	2.86314476919085\\
4.62942157532395	2.82892232882523\\
4.63953591305895	2.79229792284285\\
4.6477832358126	2.75281428543453\\
4.65393336975607	2.70991366424717\\
4.65764755998012	2.66290910319778\\
4.65844355059981	2.61094574676467\\
4.65564456679094	2.55294826337533\\
4.64830447353098	2.48754880315085\\
4.63509714703282	2.412987529111\\
4.61415146693176	2.32697452337006\\
4.58280309607122	2.2264978124928\\
4.53721893257655	2.10755820649293\\
4.47182920549984	1.96481080674113\\
4.37847981232864	1.79110508686486\\
4.24521490284242	1.57696626547966\\
4.05468868211603	1.31021198654074\\
3.78258157057486	0.976271917354255\\
3.39746211800831	0.560513847144141\\
2.86564603019587	0.0547349303129109\\
2.16628095956492	-0.530918998666906\\
1.3163458188637	-1.15601715113801\\
0.386364271571009	-1.75373532161103\\
-0.521395033430425	-2.25951955673995\\
-1.32245507455195	-2.64185412289414\\
-1.98003247030003	-2.90611385282926\\
-2.49783304165879	-3.0770017070695\\
-2.89858578292377	-3.18160571774392\\
-3.20833386187208	-3.24175809995871\\
-3.44961337829407	-3.27287671472173\\
-3.63990028487811	-3.28520643597026\\
-3.79212961182057	-3.28537994521807\\
-3.9157126484899	-3.27765525513644\\
-4.01748172736179	-3.26476547925274\\
-4.10242429214688	-3.24846956642585\\
-4.17421535716773	-3.22990138708828\\
-4.23559313594463	-3.20978984403993\\
-4.28862168962215	-3.18859808518074\\
-4.3348747062288	-3.16661214278255\\
-4.37556482043913	-3.14399778752436\\
-4.41163533937276	-3.12083719473372\\
-4.44382586715106	-3.09715260383444\\
-4.4727196324573	-3.07292144245103\\
-4.49877782605937	-3.04808571249083\\
-4.52236456680204	-3.02255738720676\\
-4.54376496468835	-2.99622089810873\\
-4.56319795420717	-2.96893334830397\\
-4.5808250056787	-2.94052278162212\\
-4.59675540197106	-2.91078460550761\\
-4.61104843177575	-2.87947606989276\\
-4.62371255163821	-2.84630851359366\\
-4.63470126528994	-2.81093687707772\\
-4.64390511629923	-2.77294571693644\\
-4.6511387335686	-2.73183060765709\\
-4.65612123157329	-2.68697333269246\\
-4.65844733256287	-2.63760858249132\\
-4.65754516584234	-2.58277889731557\\
-4.65261452453361	-2.52127318585437\\
-4.64253597186815	-2.45154214564333\\
-4.62573588562742	-2.37158111951525\\
-4.59998426947178	-2.2787672414254\\
-4.56208956747593	-2.16963347584623\\
-4.5074365468492	-2.03955904725652\\
-4.4292905327985	-1.88235946304682\\
-4.31777414606879	-1.6897858888492\\
-4.15845144678625	-1.4510336699434\\
-3.93065071450564	-1.15260296148236\\
-3.60631549513373	-0.779398394422836\\
-3.15176018832689	-0.318841953790946\\
-2.53704200326684	0.229845489251054\\
-1.75695286026187	0.84232819540724\\
-0.855158528695751	1.46283364501566\\
0.0765427825210307	2.02106832642988\\
0.938738827138494	2.46648514783927\\
1.66964433031892	2.78741035581351\\
2.25519674352221	3.00145883703757\\
2.71117451008825	3.13606823051264\\
3.06331051275912	3.21613443636322\\
3.33632262894028	3.26020434931857\\
3.55022760940807	3.28090751767623\\
3.72011336764885	3.28650261533118\\
};
\addplot [color=mycolor1, forget plot]
  table[row sep=crcr]{%
4.31376385210612	3.5712909289112\\
4.44604796470272	3.56724523047818\\
4.55294229853712	3.55715130579748\\
4.64071758107381	3.54319152926674\\
4.71386006446686	3.52673700116767\\
4.77562771826157	3.50865147131496\\
4.82842228145861	3.48947632770434\\
4.8740398886074	3.46954452840045\\
4.91384175266373	3.44905163095831\\
4.94887186762898	3.42810053785964\\
4.97993921391307	3.40672986044454\\
5.00767585842429	3.38493186964506\\
5.03257843421039	3.36266366895516\\
5.05503796211655	3.33985381874632\\
5.07536132941632	3.31640577504021\\
5.09378664609597	3.29219895681292\\
5.11049395508259	3.26708789225794\\
5.12561224768768	3.24089963630617\\
5.13922334338563	3.21342944799483\\
5.15136287276803	3.18443453110304\\
5.16201830606558	3.15362544550125\\
5.17112365225563	3.12065456076874\\
5.17855006487048	3.08510061308643\\
5.18409106277195	3.04644799415585\\
5.1874403073849	3.00405877844646\\
5.18815871459962	2.95713457915007\\
5.18562586284048	2.90466395578317\\
5.17896776172049	2.84534903574844\\
5.16694835218675	2.77750189542916\\
5.14780441920348	2.69889655486705\\
5.11899093564329	2.6065555205818\\
5.07678312764305	2.4964401358647\\
5.01564854053493	2.36300223925064\\
4.92725373028048	2.1985462606597\\
4.79891330831509	1.99236703555086\\
4.61127879316853	1.72973582654781\\
4.33531671753969	1.39118557929181\\
3.92978196480634	0.953580933677038\\
3.3437375136301	0.396514247226397\\
2.53427804952791	-0.280980476147483\\
1.50745677047191	-1.03587163166222\\
0.357068633485824	-1.7751575057098\\
-0.761980801568648	-2.39880021333496\\
-1.72327572086905	-2.85783412550436\\
-2.48205251320246	-3.16295654865162\\
-3.05554184320146	-3.35235574201659\\
-3.48329478537669	-3.46408909594419\\
-3.80382171578637	-3.52638354906937\\
-4.04728934136811	-3.55781365822627\\
-4.23545614256618	-3.57002395842887\\
-4.38356160756193	-3.57020410987645\\
-4.50222697748644	-3.56279421365319\\
-4.59890521232152	-3.55055423665639\\
-4.67889059104945	-3.53521283874192\\
-4.74599897837135	-3.51785831702342\\
-4.80302249625893	-3.49917539843617\\
-4.85203452262181	-3.47959019054209\\
-4.89459620783079	-3.45936000860568\\
-4.93189798638166	-3.4386296892856\\
-4.9648577892612	-3.41746720159537\\
-4.99419005699258	-3.39588623180777\\
-5.02045477689393	-3.37386039493518\\
-5.04409263240669	-3.35133191832602\\
-5.06545031740138	-3.32821654269527\\
-5.08479872897419	-3.30440569935721\\
-5.1023458529469	-3.27976657861271\\
-5.11824553461315	-3.25414040114044\\
-5.1326028775606	-3.22733897851624\\
-5.14547666374989	-3.19913945845492\\
-5.1568788852583	-3.16927696331457\\
-5.16677117692064	-3.13743461832273\\
-5.17505759304233	-3.1032301976982\\
-5.18157272299476	-3.06619825240833\\
-5.18606350855219	-3.02576606630659\\
-5.18816218546643	-2.98122103397904\\
-5.1873463224102	-2.93166593613698\\
-5.18287964143246	-2.87595691119277\\
-5.17372362286784	-2.81261638748794\\
-5.15840389644149	-2.73970941162479\\
-5.13480555360966	-2.65466608790162\\
-5.09985528731257	-2.55402457940664\\
-5.04902196155985	-2.43305815523723\\
-4.97552653095822	-2.28523843610815\\
-4.86909697091442	-2.1014864433282\\
-4.71405748155353	-1.86921149932266\\
-4.48661682319537	-1.57134726483217\\
-4.1518034654525	-1.18623914751836\\
-3.66256272570685	-0.690777535072004\\
-2.96831088898074	-0.0714274511682935\\
-2.04438402146404	0.653640890152905\\
-0.938508424831904	1.41435936861618\\
0.21598448538672	2.10611182803394\\
1.26708430107625	2.64931391731234\\
2.1278446066936	3.02747233197286\\
2.78963433717024	3.26955278862576\\
3.28506386419243	3.41591365444406\\
3.65486208112682	3.50005904778578\\
3.93365070187683	3.54509915839146\\
4.14720426646698	3.56579116524099\\
4.31376385210611	3.5712909289112\\
};
\addplot [color=mycolor1, forget plot]
  table[row sep=crcr]{%
5.07461298996009	3.9742534137051\\
5.20312201465524	3.97033115483588\\
5.30587092362685	3.96063385017361\\
5.38952296477647	3.94733331450053\\
5.45874344806197	3.93176352930561\\
5.51686185932693	3.91474824761516\\
5.56629796059966	3.89679421172553\\
5.60884022191911	3.8782071291593\\
5.64583086106727	3.85916244128577\\
5.67829092663376	3.83974909822166\\
5.70700623307129	3.81999687063234\\
5.7325872750885	3.79989339005687\\
5.75551152455045	3.77939461209223\\
5.77615356449002	3.75843092844573\\
5.79480664213944	3.73691026877469\\
5.81169800874334	3.71471898236091\\
5.82699960588411	3.69172093049256\\
5.84083510008146	3.66775496851056\\
5.85328385958774	3.6426307988632\\
5.86438214150385	3.61612299648009\\
5.87412146180624	3.58796281533438\\
5.88244380882355	3.55782714999972\\
5.88923298014295	3.52532371137192\\
5.89430080465441	3.48997102991825\\
5.89736625162972	3.45117124520797\\
5.89802426051472	3.40817265543968\\
5.8956992693481	3.36001749023562\\
5.88957539716316	3.30546801732987\\
5.87849020807931	3.24290038135406\\
5.86077046633799	3.17014965928821\\
5.8339736340321	3.08428016763865\\
5.79447338218908	2.98124009511028\\
5.73678308902399	2.85533667481716\\
5.65243596636619	2.69843657867155\\
5.52812160933873	2.49876569105719\\
5.34263239614403	2.23920506114815\\
5.0621707267555	1.89525219867204\\
4.63447728325225	1.4339410702098\\
3.98645789817976	0.818301459391054\\
3.04127562108193	0.0276824728203648\\
1.78004297196272	-0.899091935467874\\
0.325218964179175	-1.83387368454933\\
-1.08389811375681	-2.61937894138237\\
-2.2550301337491	-3.17893916732926\\
-3.13784810473174	-3.53419885722625\\
-3.77536201908882	-3.74490148296817\\
-4.23270744586419	-3.86445431115576\\
-4.56488181034869	-3.92906223056567\\
-4.81111173084384	-3.96087712458478\\
-4.99782490940931	-3.97300961597196\\
-5.14260703237204	-3.97319579578444\\
-5.25724330095103	-3.96604388570524\\
-5.34975483269134	-3.95433565851878\\
-5.42570363473993	-3.93977138106006\\
-5.48902167942809	-3.92339911305248\\
-5.54254114935149	-3.90586575046038\\
-5.58833817113287	-3.88756639621381\\
-5.62795931121304	-3.86873478129887\\
-5.66257339361963	-3.84949884281418\\
-5.69307497569185	-3.82991527850158\\
-5.72015597962227	-3.8099911362125\\
-5.74435596365189	-3.78969721327136\\
-5.76609779204375	-3.76897613123127\\
-5.78571311862739	-3.74774681542691\\
-5.80346059509227	-3.72590641372493\\
-5.81953872845096	-3.70333024722295\\
-5.83409464438102	-3.6798700882352\\
-5.84722954031131	-3.65535084147328\\
-5.85900125249416	-3.62956551949584\\
-5.86942405656107	-3.60226822087717\\
-5.87846552319394	-3.57316460983039\\
-5.88603991148949	-3.54189912649133\\
-5.89199714349145	-3.50803778459862\\
-5.89610578039421	-3.47104487502588\\
-5.89802748328929	-3.43025109272021\\
-5.89727897404888	-3.38480938741684\\
-5.89317515025065	-3.33363295680044\\
-5.88474311936696	-3.27530684970659\\
-5.87059038716489	-3.20795996555619\\
-5.84869928000546	-3.1290767595092\\
-5.81610037050964	-3.03521604508351\\
-5.76834404664527	-2.92158566067602\\
-5.69863131208933	-2.78139436731751\\
-5.59636883937504	-2.60486868121386\\
-5.44477406956813	-2.37780660366194\\
-5.21703948257889	-2.07964643193155\\
-4.87085181653827	-1.68161072853984\\
-4.34317610222974	-1.1474913730259\\
-3.55463859109555	-0.444439617948892\\
-2.44646178577775	0.424728851617969\\
-1.06299018604471	1.37607382544733\\
0.400171324256592	2.25282459691835\\
1.70555026237845	2.92773748712853\\
2.73097404485681	3.37854084526397\\
3.48314877041734	3.65388972130317\\
4.02274125320431	3.81341724852021\\
4.41164031661734	3.90197566701296\\
4.69683755618252	3.94808845837674\\
4.91064267037425	3.96882631493131\\
5.07461298996009	3.9742534137051\\
};
\addplot [color=mycolor1, forget plot]
  table[row sep=crcr]{%
5.99541349123009	4.49732482894193\\
6.12198373456767	4.49346810979022\\
6.22232276431917	4.48400228275453\\
6.30345828935967	4.47110453976137\\
6.37022710499521	4.45608804678951\\
6.4260340040194	4.43975081575307\\
6.47332606041965	4.42257641347924\\
6.51389546554197	4.40485202361184\\
6.54907708213389	4.38673928808308\\
6.57987997837295	4.36831750635314\\
6.60707667311733	4.34961025530518\\
6.63126471753926	4.33060181272328\\
6.65290979885519	4.31124713321961\\
6.67237623661826	4.29147760548794\\
6.68994867800905	4.27120391812361\\
6.70584748329541	4.25031680783956\\
6.72023943094676	4.2286861073336\\
6.73324478583477	4.20615826222468\\
6.74494135221224	4.18255229293414\\
6.7553658016758	4.15765400075927\\
6.7645122701438	4.13120802717579\\
6.77232790850611	4.10290714059211\\
6.77870469585667	4.07237780676337\\
6.7834663114413	4.03916064288895\\
6.78634810483478	4.0026836765805\\
6.7869670294593	3.96222529387516\\
6.7847765162843	3.9168621418823\\
6.77899814289203	3.86539467577519\\
6.76851666415351	3.80623886471428\\
6.7517158001861	3.73726568768493\\
6.72621592257996	3.65555853467875\\
6.68844539018337	3.55703918497539\\
6.63292334288572	3.43588027230273\\
6.55103235852934	3.28356844963774\\
6.42888013525645	3.0874023791601\\
6.24355374627446	2.82812556715442\\
5.95672477318787	2.47646662632\\
5.50486682757952	1.98928488869036\\
4.78954650692568	1.31006109479641\\
3.68844777342875	0.389582967520768\\
2.13844260427236	-0.748782620037423\\
0.291673312658474	-1.93519244771774\\
-1.48840841691493	-2.92778854675826\\
-2.91487958197734	-3.60977739845915\\
-3.93942991406759	-4.02237595996972\\
-4.64665575647416	-4.2562875645141\\
-5.13582834199775	-4.38424753063528\\
-5.48132614055454	-4.45149222266906\\
-5.73208249096043	-4.48391637079128\\
-5.91920627616184	-4.49608929107534\\
-6.06253170237649	-4.49628172862736\\
-6.17493002609912	-4.4892744801395\\
-6.26494796344672	-4.47788510802943\\
-6.33839878081503	-4.46380206027429\\
-6.39932959310621	-4.44804860908601\\
-6.45061973895545	-4.43124671036897\\
-6.4943587403697	-4.41377052971911\\
-6.53209025416744	-4.3958376866492\\
-6.56497287364746	-4.37756447947498\\
-6.5938882314731	-4.35899976779666\\
-6.61951498853594	-4.34014589512689\\
-6.64238027283283	-4.32097153382069\\
-6.6628958968002	-4.30141933996902\\
-6.68138407285814	-4.28141013959875\\
-6.69809570437103	-4.26084466461084\\
-6.71322326879524	-4.23960341595104\\
-6.72690960312022	-4.21754493771128\\
-6.73925340887719	-4.19450257061826\\
-6.7503119251	-4.17027957233026\\
-6.76010090999522	-4.14464231222923\\
-6.76859177539645	-4.11731104005872\\
-6.77570538291145	-4.08794745695008\\
-6.78130157695893	-4.05613793861392\\
-6.7851629118082	-4.0213707060733\\
-6.7869700919149	-3.98300440271407\\
-6.78626516134385	-3.94022424412106\\
-6.78239605797812	-3.89197987008238\\
-6.77443209693362	-3.83689575601213\\
-6.76103300288552	-3.77313969088716\\
-6.74024193420233	-3.69822593956794\\
-6.7091511345993	-3.60871475364321\\
-6.66334908680843	-3.49974462133492\\
-6.59598477240038	-3.36429144842522\\
-6.49615044266738	-3.19198199938406\\
-6.34604973540744	-2.96719982612318\\
-6.11607156316345	-2.66617766747122\\
-5.7566940389327	-2.25311741435295\\
-5.18768700672088	-1.67742587824769\\
-4.2942798350915	-0.881332086703665\\
-2.966737669024	0.159264763485881\\
-1.23176837138047	1.35185575181467\\
0.630117465771918	2.46760048475806\\
2.25388472598717	3.30754348402831\\
3.47340789742028	3.84405640500079\\
4.32607619065736	4.15642304228316\\
4.91317670196399	4.33011777856789\\
5.32296814448572	4.42349646802813\\
5.61627328687972	4.47095329859258\\
5.83215503416641	4.49191078227295\\
5.99541349123009	4.49732482894193\\
};
\addplot [color=mycolor1, forget plot]
  table[row sep=crcr]{%
6.89736886606275	5.03319437558776\\
7.02416387175193	5.02933501139698\\
7.12411966905462	5.01990796379786\\
7.20458777221079	5.0071180347061\\
7.27057166197707	4.9922792439007\\
7.32556233563215	4.97618178754688\\
7.37205090732628	4.9592997824019\\
7.4118511675274	4.94191188332185\\
7.44630755102136	4.924172880178\\
7.47643228783712	4.90615695916026\\
7.50299772990644	4.88788414863349\\
7.52659964487822	4.86933651746032\\
7.54770128007571	4.85046794499134\\
7.56666440163349	4.8312097142689\\
7.58377130067985	4.81147325938556\\
7.59924036284171	4.79115083779699\\
7.61323689171352	4.77011454015385\\
7.6258802666304	4.74821380263283\\
7.63724808003991	4.72527139393899\\
7.64737756120027	4.70107767299109\\
7.65626429327812	4.67538272270993\\
7.66385792005916	4.64788572851158\\
7.67005416198436	4.61822064710494\\
7.67468194685035	4.58593674444558\\
7.67748369861844	4.55047188167056\\
7.67808563840317	4.51111534899351\\
7.67595302539693	4.46695534627469\\
7.6703220523474	4.41680346793679\\
7.66009460612644	4.35908403918405\\
7.64367242880821	4.2916685701503\\
7.61868976481584	4.21162259560055\\
7.58157130548474	4.11480947629323\\
7.52678116568926	3.99525565555512\\
7.44551109794936	3.84411120881695\\
7.32332970899458	3.64791936884771\\
7.13589448163364	3.38573012706076\\
6.84116389135927	3.02445608149658\\
6.36613618695305	2.51243804049203\\
5.58954013189169	1.77531831863544\\
4.34304282563732	0.73379839849817\\
2.50920940090667	-0.612411235566368\\
0.262259739171652	-2.05567829798883\\
-1.89433615075502	-3.25854162383184\\
-3.56922904333947	-4.05972382111383\\
-4.72545819139513	-4.5256232886552\\
-5.49608724978678	-4.7806438240618\\
-6.01489616881318	-4.91642316290733\\
-6.37409111441823	-4.98636678638294\\
-6.6309998647971	-5.01960357672328\\
-6.82064113847222	-5.03194962969862\\
-6.96470528987139	-5.03214847701271\\
-7.07696932594897	-5.0251529030354\\
-7.16643373595251	-5.01383567684412\\
-7.23914378704781	-4.99989606987022\\
-7.29926650130696	-4.98435252965248\\
-7.34974301291436	-4.96781787150778\\
-7.39269392434695	-4.95065710246054\\
-7.42967754762933	-4.93308011617047\\
-7.46185833587123	-4.91519724170708\\
-7.49011913873529	-4.89705303525015\\
-7.51513748514114	-4.87864698942531\\
-7.5374383030822	-4.85994615784343\\
-7.55743085815116	-4.84089262376764\\
-7.57543487892197	-4.82140754519059\\
-7.59169908559255	-4.80139279414265\\
-7.60641421853607	-4.78073076348686\\
-7.6197219241111	-4.75928262020958\\
-7.63172034425345	-4.73688506958777\\
-7.64246687771582	-4.71334551425335\\
-7.65197826816	-4.68843531256141\\
-7.66022787530608	-4.66188063126278\\
-7.66713964921773	-4.63335011340417\\
-7.67257789316159	-4.60243819627687\\
-7.67633127945071	-4.568642344819\\
-7.67808863581077	-4.53133159948159\\
-7.67740251220767	-4.48970248650128\\
-7.67363405872692	-4.44271618478672\\
-7.66586855333436	-4.38900733663324\\
-7.65278364435701	-4.32674905720569\\
-7.6324394206162	-4.25344879525755\\
-7.60193576333479	-4.16563255672918\\
-7.55683815140872	-4.05834487081539\\
-7.49018848824136	-3.92433857229832\\
-7.3907541462704	-3.75273558519102\\
-7.23985724583102	-3.52678910568387\\
-7.00557905309622	-3.22019061556202\\
-6.63244583047343	-2.79142193565118\\
-6.02533913185054	-2.17738838513611\\
-5.03589622488247	-1.29611144069466\\
-3.49852803761304	-0.0916384912261362\\
-1.4099049355907	1.34356442438593\\
0.860303057242737	2.70407592263365\\
2.80131087132802	3.70855023314097\\
4.20507350080996	4.32647318144255\\
5.14971166906231	4.6727291307529\\
5.78023129590174	4.85936587721496\\
6.21020475115491	4.95739022037701\\
6.51274555585935	5.00636499346055\\
6.7326384758569	5.02772444162467\\
6.89736886606275	5.03319437558776\\
};
\addplot [color=mycolor1, forget plot]
  table[row sep=crcr]{%
7.29591158957958	5.27513025289535\\
7.42332987221212	5.27125335301717\\
7.52358643507382	5.26179883012366\\
7.60417597996328	5.2489901777062\\
7.67018025597365	5.23414719531595\\
7.72513429683464	5.21806073965346\\
7.77155458382699	5.20120373405925\\
7.81126978887302	5.1838531464475\\
7.8456331896464	5.16616213164784\\
7.8756623093767	5.14820348838183\\
7.90213267362373	5.12999615354589\\
7.9256419455298	5.11152139069478\\
7.94665449572159	5.09273253083439\\
7.96553275207919	5.07356053360328\\
7.98255940086198	5.05391670628169\\
7.99795308180025	5.03369335345425\\
8.01187929539751	5.01276276994595\\
8.02445761963065	4.99097474117082\\
8.03576589188746	4.96815252170294\\
8.04584166962416	4.9440870861721\\
8.05468098115947	4.91852925510848\\
8.06223406538229	4.8911790599331\\
8.06839742085364	4.86167138534677\\
8.07300096778685	4.8295564552579\\
8.07578835965867	4.79427301852752\\
8.07638728205741	4.75511099372161\\
8.07426462963591	4.71115859635049\\
8.06865819541017	4.66122616456551\\
8.05847090822105	4.60373425702524\\
8.04210376970811	4.5365457484206\\
8.01718571494063	4.4567080860975\\
7.98012522319577	4.3600479551804\\
7.92534468945407	4.24051775681551\\
7.84393410430332	4.08911620168565\\
7.72121595557031	3.89206964324699\\
7.53224428080095	3.62774426898471\\
7.23346459228597	3.26153245470666\\
6.74802758508566	2.73834698777122\\
5.94514374297721	1.97638429548909\\
4.63617708355683	0.882869012273487\\
2.67725454219386	-0.554912272002079\\
0.250106303551813	-2.11382512863947\\
-2.07544543435575	-3.41106365837185\\
-3.85922221824205	-4.26450734222697\\
-5.07193654273235	-4.7532753562224\\
-5.8697099806853	-5.01733122425154\\
-6.40155209947301	-5.15654607646189\\
-6.76717349051142	-5.22775289084373\\
-7.02734471168533	-5.26141775845063\\
-7.21867447336467	-5.2738769789914\\
-7.36361282690222	-5.27407889076835\\
-7.47631503422351	-5.26705713646384\\
-7.56597768604665	-5.25571554687432\\
-7.63875149643691	-5.24176419093408\\
-7.6988620035609	-5.22622413513902\\
-7.74928367714008	-5.2097076762349\\
-7.7921565112804	-5.19257827760396\\
-7.82905027501155	-5.1750441323721\\
-7.86113625917744	-5.15721404607183\\
-7.88930141267163	-5.13913133431034\\
-7.91422571902979	-5.12079454432989\\
-7.93643556700714	-5.10217005697309\\
-7.95634108450513	-5.08319952348469\\
-7.97426250931867	-5.06380387995354\\
-7.99044887366695	-5.0438849610277\\
-8.00509113481198	-5.02332528700023\\
-8.01833113061329	-5.00198630280175\\
-8.03026721982775	-4.97970513216102\\
-8.04095708344062	-4.9562897293411\\
-8.05041784766114	-4.93151213056845\\
-8.05862338820194	-4.90509929667592\\
-8.065498337632	-4.87672076222637\\
-8.07090788086842	-4.84597191630921\\
-8.07464179934598	-4.81235116353054\\
-8.07639027069993	-4.77522833395339\\
-8.07570740910741	-4.73380033467176\\
-8.07195602234769	-4.68702783425895\\
-8.06422280598895	-4.63354317114868\\
-8.05118578101769	-4.57151365771521\\
-8.03090251436506	-4.49843416256824\\
-8.00046327374921	-4.410804886061\\
-7.95540723953655	-4.30361828179464\\
-7.88871095532438	-4.16952157079232\\
-7.78898364202658	-3.99741843465297\\
-7.63716302068276	-3.77009842557475\\
-7.40037751408439	-3.46023676319357\\
-7.02074134537897	-3.02403193890216\\
-6.39703409109933	-2.39328418225762\\
-5.36654591193065	-1.47560069962743\\
-3.73799675272195	-0.199935111082237\\
-1.49146927378901	1.3435635047952\\
0.962926008148577	2.81449051151017\\
3.04480000395319	3.8920465842516\\
4.52832215317503	4.54522389310579\\
5.5122497122189	4.90595764984086\\
6.16153292543111	5.09818439883339\\
6.6006229495706	5.19830404295129\\
6.90772781078717	5.24802599683915\\
7.12996437498889	5.26961747647854\\
7.29591158957958	5.27513025289535\\
};
\addplot [color=mycolor1, forget plot]
  table[row sep=crcr]{%
6.81839938705513	4.98558555137348\\
6.9451056567848	4.98172857837679\\
7.04503269745632	4.97230404998413\\
7.12550382874078	4.95951351377152\\
7.19150742751882	4.9446702051191\\
7.24652621591938	4.92856445823852\\
7.29304670485787	4.91167081831979\\
7.33288010703728	4.89426840679286\\
7.36736941959477	4.87651242497647\\
7.39752608589151	4.8584773879902\\
7.42412204545158	4.84018356931731\\
7.44775286792895	4.82161320692916\\
7.46888172297768	4.802720283049\\
7.48787036298415	4.78343612628583\\
7.50500109471872	4.76367216612502\\
7.52049232706941	4.74332061039937\\
7.53450938011309	4.72225345843491\\
7.54717163259053	4.70032001509007\\
7.5585566510167	4.67734287814631\\
7.56870160579294	4.65311219540491\\
7.57760198044974	4.62737779738012\\
7.5852072695652	4.59983857492861\\
7.59141298466069	4.57012814868072\\
7.59604777339064	4.53779541139837\\
7.59885369623389	4.5022778262537\\
7.59945651723111	4.46286428877193\\
7.59732094276784	4.41864266581239\\
7.59168253759535	4.36842439739232\\
7.58144256176967	4.31063406338187\\
7.56500233907657	4.243144292049\\
7.5399964148077	4.16302350528493\\
7.50285171749803	4.0661415588715\\
7.44803942928042	3.9465388293111\\
7.36677017908548	3.79539498589725\\
7.24466050473369	3.59931675028097\\
7.0574886748104	3.33749314960869\\
6.76352025145179	2.97714793958209\\
6.29053702849613	2.46732264221061\\
5.51919290384651	1.7351657602429\\
4.28518881238173	0.704044860696217\\
2.47617548846006	-0.62399326502574\\
0.26472488379459	-2.04447711678708\\
-1.85854763216566	-3.22872807539079\\
-3.51179701572037	-4.01952314056496\\
-4.65672460540954	-4.48084729701715\\
-5.42192597149296	-4.73406098849309\\
-5.93815571883776	-4.86916023516302\\
-6.29610684807415	-4.93885917516955\\
-6.55240778900606	-4.97201605639051\\
-6.74175384000817	-4.98434219551015\\
-6.88568139038962	-4.98454045434875\\
-6.99789141017824	-4.97754800336823\\
-7.08734546190676	-4.96623193256517\\
-7.16006824322224	-4.95228978176482\\
-7.22021562299617	-4.93673979314651\\
-7.27072257215752	-4.92019511301591\\
-7.31370624925992	-4.90302121434563\\
-7.3507230387261	-4.88542843600515\\
-7.3829363263936	-4.86752747846997\\
-7.41122838809586	-4.84936318429306\\
-7.43627646256458	-4.8309352520156\\
-7.45860534888868	-4.81221087003346\\
-7.47862427236261	-4.79313219481558\\
-7.49665296586647	-4.77362040406707\\
-7.51294017121452	-4.75357734217863\\
-7.52767665013585	-4.73288533123189\\
-7.54100405805473	-4.71140542676794\\
-7.55302052460794	-4.68897418296435\\
-7.56378340709999	-4.66539881155098\\
-7.57330937097375	-4.64045043926517\\
-7.58157165270455	-4.61385495942758\\
-7.58849402465152	-4.58528069952081\\
-7.5939405470587	-4.55432174128212\\
-7.59769957191291	-4.52047516178699\\
-7.59945951764889	-4.48310960021399\\
-7.59877242863057	-4.44142120888268\\
-7.59499885959066	-4.39437090303387\\
-7.58722344586879	-4.34059333582325\\
-7.5741232741605	-4.27826223117822\\
-7.55375828026879	-4.20488688476054\\
-7.52322939077674	-4.11699767437352\\
-7.4781062033147	-4.00964867067367\\
-7.4114422958755	-3.87561300911386\\
-7.31203520475341	-3.70405586877619\\
-7.16128309678768	-3.47832412658061\\
-6.9274595685231	-3.17231678512378\\
-6.55558188850301	-2.7449831897496\\
-5.95177163129806	-2.13426805552596\\
-4.97054626054604	-1.26027984798751\\
-3.45135195503113	-0.0699980423053237\\
-1.39392514759134	1.34380755585939\\
0.840019649925826	2.68258056595531\\
2.75314581466553	3.67259137928405\\
4.14099226701179	4.28347990938291\\
5.07776315797388	4.62683679623043\\
5.70454533578041	4.81235986820016\\
6.13273164742407	4.90997324701889\\
6.4344036995944	4.95880562065331\\
6.65387188453432	4.98012287766274\\
6.81839938705513	4.98558555137348\\
};
\addplot [color=mycolor1, forget plot]
  table[row sep=crcr]{%
5.81106869281575	4.39030195368767\\
5.9378306742482	4.38643831889657\\
6.03846583405338	4.37694387713713\\
6.1199335771122	4.36399287535022\\
6.1870374509301	4.34890071949983\\
6.24316659297427	4.33246893432805\\
6.29076134204276	4.31518444620006\\
6.33161167052361	4.29733720124044\\
6.36705246181677	4.27909093792948\\
6.39809385319555	4.26052644680076\\
6.42550985478205	4.24166828271736\\
6.44989960323473	4.22250127616143\\
6.47173029631398	4.20298058067738\\
6.49136760427196	4.18303748327818\\
6.50909732396471	4.16258230635481\\
6.52514074436214	4.14150517702381\\
6.53966534011105	4.11967508309843\\
6.55279182890572	4.09693738654155\\
6.56459820933533	4.07310977119143\\
6.57512106561025	4.04797642449512\\
6.58435412975198	4.02128006254521\\
6.59224378227706	3.99271117302261\\
6.59868079617576	3.96189353339659\\
6.60348711570957	3.92836460753721\\
6.60639570497698	3.89154874944605\\
6.60702032865372	3.85072011495395\\
6.60481024568603	3.80495058240551\\
6.59898169346763	3.75303544587264\\
6.58841279772551	3.69338554863195\\
6.57147948699918	3.62386780739903\\
6.54579401358252	3.54156491815036\\
6.50777897262767	3.44240636935973\\
6.45195745624496	3.32059186998236\\
6.36974484521994	3.16767858486497\\
6.24735944954143	2.97113266238818\\
6.06220161960239	2.7120821831487\\
5.77677395539875	2.3621237276036\\
5.32965500631877	1.88001801110771\\
4.62732032063472	1.21306234745563\\
3.55676427257078	0.318016299685843\\
2.06476136417372	-0.777861068382086\\
0.298062509953041	-1.91287086600401\\
-1.40643183116124	-2.86326345521889\\
-2.78185360802044	-3.52076771016897\\
-3.77869138961688	-3.9221517335401\\
-4.47246549395942	-4.15158450527908\\
-4.95546408019367	-4.27791426206472\\
-5.2982738775063	-4.34462797714021\\
-5.54798757103047	-4.37691313728978\\
-5.73484389100408	-4.38906632495553\\
-5.87826329687165	-4.38925751520342\\
-5.99091736913433	-4.38223347418573\\
-6.08125526720893	-4.37080307140861\\
-6.15504250345002	-4.35665515142711\\
-6.21630322685912	-4.34081614461725\\
-6.26790632881065	-4.32391153902328\\
-6.31193723841499	-4.3063185848153\\
-6.34993871357164	-4.28825732814437\\
-6.38306999766929	-4.26984585027003\\
-6.41221405884161	-4.25113423359834\\
-6.43805112511992	-4.23212557690578\\
-6.46110988715076	-4.21278892058008\\
-6.48180359518745	-4.19306696380742\\
-6.5004557152727	-4.17288029514566\\
-6.51731819170392	-4.15212915680827\\
-6.53258431563611	-4.13069332223909\\
-6.54639750008312	-4.10843037237793\\
-6.55885677247197	-4.08517244024534\\
-6.5700194288572	-4.06072131195114\\
-6.5799009870112	-4.03484159215411\\
-6.58847227898074	-4.00725143358615\\
-6.59565318817315	-3.97761005983719\\
-6.60130210178703	-3.94550093310744\\
-6.60519953099088	-3.91040886721914\\
-6.60702341415294	-3.87168855594162\\
-6.60631213863073	-3.82852070749267\\
-6.60240890778542	-3.77984996581519\\
-6.5943770580994	-3.72429558219703\\
-6.58086905971169	-3.66002056526266\\
-6.55991993495031	-3.58453639081319\\
-6.52861445599441	-3.49440592415262\\
-6.48253879024475	-3.38478309582889\\
-6.41485667012296	-3.24868830269853\\
-6.31472279447289	-3.07585768979424\\
-6.16452915602238	-2.8509292437166\\
-5.93517781718945	-2.55071472514717\\
-5.57847827374839	-2.14070828646252\\
-5.01746429854773	-1.57305770883352\\
-4.14439668688545	-0.795006619829821\\
-2.86036399139633	0.211598326002595\\
-1.19674002935348	1.3552324973526\\
0.583584041481237	2.42208766959095\\
2.14300486862341	3.22866929909522\\
3.3241693684594	3.74823849458138\\
4.15735160444676	4.05342574908404\\
4.73528772043855	4.22438788118071\\
5.14097069607059	4.31681952704536\\
5.43256333691566	4.3639936027741\\
5.6478623113483	4.38489141310636\\
5.81106869281575	4.39030195368767\\
};
\addplot [color=mycolor1, forget plot]
  table[row sep=crcr]{%
4.79211818889952	3.82077342471787\\
4.92177086224927	3.81681362251261\\
5.02579382855331	3.80699436771221\\
5.11071934761554	3.79349020396659\\
5.18115322605272	3.77764668970029\\
5.24040110153725	3.76030015906999\\
5.29087648834215	3.74196825006895\\
5.334370015191	3.72296522291877\\
5.37222975797709	3.70347281736717\\
5.40548392042828	3.68358434168994\\
5.43492556274751	3.66333232196307\\
5.46117193263937	3.64270583274064\\
5.48470650120408	3.62166118565911\\
5.50590899835184	3.60012820367493\\
5.52507694347959	3.57801342936336\\
5.54244099201972	3.55520106522052\\
5.55817563103771	3.53155208318496\\
5.57240620970301	3.50690168662155\\
5.58521288765786	3.481055108402\\
5.59663176028572	3.453781546994\\
5.60665312406216	3.42480584883557\\
5.61521653142038	3.39379730983344\\
5.62220190131706	3.36035465530512\\
5.62741543027179	3.32398581588658\\
5.63056828640992	3.28408047206688\\
5.63124490194165	3.23987237622389\\
5.62885583532081	3.19038699505496\\
5.62256719023864	3.13436775439726\\
5.61119365519057	3.07017064517139\\
5.59303397095464	2.99561142824242\\
5.56561361540334	2.90774105071726\\
5.5252755607558	2.80251167617108\\
5.46651940192237	2.67427662899367\\
5.38092287164628	2.51504408972986\\
5.25538259842559	2.31339080669481\\
5.06931449840661	2.05299768168736\\
4.7905529760893	1.71109084870631\\
4.37072655265967	1.25819696264862\\
3.74488371386983	0.663512848732174\\
2.84933918050736	-0.0857440839287663\\
1.67583031501588	-0.948209188546892\\
0.336492051431472	-1.80883640449244\\
-0.962811281776992	-2.53305604756679\\
-2.05571937770858	-3.05513154500464\\
-2.89325465498746	-3.39208229568967\\
-3.50782571836996	-3.59514896413235\\
-3.95467324606576	-3.71192766817774\\
-4.28267967718778	-3.77570835930012\\
-4.52781894824707	-3.80737297697914\\
-4.71488412494648	-3.81952287066889\\
-4.86065518138394	-3.81970699278071\\
-4.97652334114615	-3.81247611664875\\
-5.07031940568377	-3.80060392687166\\
-5.14751628170736	-3.78579935958603\\
-5.2120073467488	-3.76912310516208\\
-5.26661129858299	-3.7512339630798\\
-5.31340301878347	-3.73253677799184\\
-5.35393349155295	-3.71327267409269\\
-5.38937825513544	-3.69357487338636\\
-5.42063916980341	-3.67350359035917\\
-5.44841520493515	-3.65306793974812\\
-5.47325231470367	-3.63223959432685\\
-5.49557894254391	-3.61096105269788\\
-5.51573145198247	-3.58915025213298\\
-5.53397233142163	-3.56670256895459\\
-5.55050306209345	-3.54349080645135\\
-5.5654728855542	-3.51936347099641\\
-5.57898424168559	-3.4941414155384\\
-5.59109529174929	-3.46761274293533\\
-5.60181963693514	-3.43952567750913\\
-5.61112304392667	-3.40957890264328\\
-5.61891664770849	-3.37740859296992\\
-5.6250456597327	-3.34257099954288\\
-5.62927198358847	-3.30451891450158\\
-5.63124820150209	-3.26256955533503\\
-5.63047893279206	-3.21586022250144\\
-5.62626322571677	-3.16328626637663\\
-5.6176078196875	-3.10341308010106\\
-5.60309475075204	-3.03434942572295\\
-5.58067602824568	-2.95356249238552\\
-5.54734979529474	-2.85760436722309\\
-5.49864116120367	-2.741703537316\\
-5.42775866109708	-2.59915315257045\\
-5.32421527535131	-2.42040586979525\\
-5.17159688773638	-2.19179337220612\\
-4.94411730786096	-1.89393757040694\\
-4.60201321517407	-1.50054537619279\\
-4.08799717243774	-0.980163705403888\\
-3.333539748464	-0.307359363168784\\
-2.29356464353244	0.508483010097138\\
-1.01486676146027	1.38789111961271\\
0.331085521457898	2.1943914998068\\
1.54082044438049	2.81975256886592\\
2.50549061640488	3.24374496810197\\
3.22499695397131	3.50706584023681\\
3.74886645174885	3.66190485532824\\
4.13098470470159	3.74889693772828\\
4.41383400597597	3.79461769413917\\
4.62740908872181	3.81532611953117\\
4.79211818889952	3.82077342471787\\
};
\addplot [color=mycolor1, forget plot]
  table[row sep=crcr]{%
3.95999323486845	3.39757254170735\\
4.09484250813319	3.39344321228675\\
4.2045297265259	3.38308210600781\\
4.29508387125841	3.36867800128559\\
4.37087714532624	3.35162544588445\\
4.43511993062576	3.33281395952028\\
4.49020046207001	3.31280761045635\\
4.53791827504927	3.29195743937878\\
4.57964576454014	3.27047251437929\\
4.61644106500604	3.24846521707147\\
4.64912774304262	3.22598024130648\\
4.67835162694857	3.20301311246613\\
4.70462167883089	3.17952181318685\\
4.72833955405471	3.15543373939353\\
4.74982098673809	3.13064935989219\\
4.76931112259044	3.10504340685299\\
4.78699521765181	3.07846405916893\\
4.80300561850361	3.05073031928915\\
4.81742555817475	3.02162757703464\\
4.83028998579573	2.99090116593392\\
4.84158335147927	2.95824752002809\\
4.85123394759627	2.92330230365497\\
4.8591040140495	2.88562458039237\\
4.86497428238551	2.84467566638736\\
4.86852086469722	2.79979071572883\\
4.86928123764474	2.75014022072918\\
4.86660428864334	2.69467734480257\\
4.85957658975848	2.63206514756602\\
4.84691261115477	2.56057504602007\\
4.82678946546552	2.47794396053811\\
4.79659542946469	2.38117226334323\\
4.75254376940376	2.26623814365281\\
4.68907718339365	2.12769852720028\\
4.59795460683918	1.95815040051454\\
4.46688732676326	1.74756400797157\\
4.27764254367676	1.48263691142847\\
4.00387072308215	1.14670604183995\\
3.61005153759665	0.721636574198892\\
3.05560765177234	0.194451774755332\\
2.31137241334251	-0.428642285968568\\
1.39083488530141	-1.10555625327907\\
0.373895154446771	-1.75913136062439\\
-0.617367064691563	-2.31148893173218\\
-1.48259799127861	-2.7245327663766\\
-2.18148886039909	-3.00546910573645\\
-2.72247859519357	-3.18406267255354\\
-3.1346724468484	-3.29168669210835\\
-3.44905165555013	-3.35275913373484\\
-3.69127710003747	-3.38401246048074\\
-3.88062555859937	-3.3962893159387\\
-4.03102237946802	-3.39646583053762\\
-4.15240860493939	-3.38888182308754\\
-4.25189304055679	-3.37628371252528\\
-4.33460233967622	-3.36041786217831\\
-4.40427697195088	-3.34239825578999\\
-4.46368138722626	-3.32293419830987\\
-4.51488543859349	-3.30247223687471\\
-4.55945837766816	-3.2812854333876\\
-4.59860372563644	-3.25953004009571\\
-4.63325399543124	-3.23728172962425\\
-4.66413790900644	-3.2145587924539\\
-4.69182854788939	-3.19133686150793\\
-4.71677809651419	-3.16755798725861\\
-4.7393429950025	-3.14313581339344\\
-4.75980208289272	-3.11795792443445\\
-4.7783694726755	-3.09188599300709\\
-4.7952033007698	-3.06475404861714\\
-4.81041106952656	-3.03636496063061\\
-4.8240519512485	-3.00648503446068\\
-4.83613612420751	-2.97483643062173\\
-4.84662090779453	-2.94108690350011\\
-4.85540311393893	-2.90483609045282\\
-4.8623065785354	-2.86559722460702\\
-4.86706320060209	-2.82277264472673\\
-4.86928487803478	-2.77562075794744\\
-4.86842229710538	-2.72321106614295\\
-4.86370430180861	-2.6643623334864\\
-4.85404804085803	-2.59755672352673\\
-4.83792446013161	-2.52081946986195\\
-4.81315470961862	-2.43154904828298\\
-4.77659879836072	-2.32627679894911\\
-4.72367605630488	-2.20032845352253\\
-4.64762643990108	-2.04735745318607\\
-4.53838863733007	-1.85873574998009\\
-4.38096984835562	-1.62286298531766\\
-4.1533355323116	-1.32469184340526\\
-3.8244975574566	-0.946371254917345\\
-3.35530815035159	-0.47108481288752\\
-2.7077897400668	0.106751713865016\\
-1.86973007851861	0.764620138831268\\
-0.887073673343907	1.44068914574646\\
0.132452543813392	2.05155563779595\\
1.06971589193757	2.5358237813157\\
1.85310797649221	2.87987586763252\\
2.47009835668971	3.10548017401548\\
2.94265469363239	3.24502618750772\\
3.30233078271589	3.32683325735637\\
3.57783719557537	3.37132200537939\\
3.79158168965869	3.39201965933028\\
3.95999323486845	3.39757254170735\\
};
\addplot [color=mycolor1, forget plot]
  table[row sep=crcr]{%
3.31964373338877	3.11680725665986\\
3.46059685762262	3.11247807026117\\
3.57708127662116	3.10146599881294\\
3.67451705414661	3.08596097142581\\
3.75696740880465	3.06740608789205\\
3.82749830257685	3.04674991857918\\
3.888442678605	3.02461109511954\\
3.94159191694989	3.001385629976\\
3.98833402720211	2.97731709974738\\
4.02975355906996	2.95254269648813\\
4.06670414816545	2.92712347062241\\
4.09986146259606	2.90106408349867\\
4.12976201336457	2.87432546960569\\
4.15683165226114	2.84683257984022\\
4.1814064231642	2.81847858065534\\
4.20374761128889	2.7891263560364\\
4.22405224219013	2.75860779629214\\
4.24245983997726	2.72672109262395\\
4.25905590451528	2.69322604436327\\
4.27387226537608	2.65783719413068\\
4.28688417700432	2.62021440850668\\
4.29800369589844	2.5799502930124\\
4.30706848081821	2.53655354267434\\
4.31382462055043	2.48942694947274\\
4.31790133406429	2.43783827225292\\
4.31877427644065	2.38088146767593\\
4.31571252742472	2.3174248153883\\
4.30770184951786	2.24604117926284\\
4.29333306429863	2.16491400132116\\
4.27063886204853	2.07171075141813\\
4.23685446381317	1.96341402908087\\
4.1880672114508	1.83610108349573\\
4.11870939430507	1.68466979149208\\
4.02084615921213	1.5025342124434\\
3.88324316513264	1.28138011224131\\
3.69033637688843	1.01122646403587\\
3.4216152440466	0.681343386361434\\
3.05276596181384	0.283009552357022\\
2.56107231208726	-0.184793005303621\\
1.9374093620001	-0.707248384379817\\
1.20187957948651	-1.24835241324659\\
0.40982998738374	-1.75746107812609\\
-0.365070108068133	-2.18915793181554\\
-1.06149298312255	-2.52144033937862\\
-1.64914225121043	-2.75749030492595\\
-2.12602172492128	-2.91479010698421\\
-2.50572442956788	-3.01384258513672\\
-2.80656397105716	-3.07222755005006\\
-3.04583480008297	-3.10306288904879\\
-3.2378125671698	-3.11548637709263\\
-3.39358276868888	-3.11565342567214\\
-3.5215221365268	-3.1076492923641\\
-3.62789922876217	-3.094170883302\\
-3.71740391909101	-3.07699615540657\\
-3.79356234576415	-3.05729574550448\\
-3.85904598931447	-3.03583683613959\\
-3.91589681262369	-3.01311602974166\\
-3.96568992832198	-2.98944609620787\\
-4.00965105193101	-2.96501280481785\\
-4.04874158413868	-2.93991225009947\\
-4.08372054805974	-2.91417532469041\\
-4.11518990057731	-2.88778359109191\\
-4.14362779027714	-2.86067927062269\\
-4.16941295659567	-2.83277108062239\\
-4.19284249089347	-2.80393700477406\\
-4.21414448452044	-2.77402464624158\\
-4.23348657969749	-2.74284950605136\\
-4.25098104927468	-2.71019129526367\\
-4.26668671068013	-2.67578819118601\\
-4.28060768645165	-2.63932875635692\\
-4.2926887203984	-2.60044102900901\\
-4.30280640314065	-2.55867803931154\\
-4.31075520152593	-2.51349867667858\\
-4.31622655132413	-2.46424239171588\\
-4.31877835652204	-2.4100956127698\\
-4.3177908838693	-2.35004692972493\\
-4.31240301138506	-2.28282697679007\\
-4.30141973612445	-2.20682747567492\\
-4.28317728538454	-2.11999210763607\\
-4.25534552110689	-2.01967006216038\\
-4.21463814624494	-1.90242229035895\\
-4.15639015745231	-1.76377354814919\\
-4.07395353425142	-1.59791747429286\\
-3.95787205311072	-1.39742388831325\\
-3.79486863777062	-1.15310163350004\\
-3.56691807683934	-0.854393647214405\\
-3.25126970548047	-0.491066383999079\\
-2.82334993058864	-0.057334736614125\\
-2.26532838218795	0.440940424271642\\
-1.58091915229243	0.978480466449083\\
-0.808389310351763	1.51014074160763\\
-0.0157820009785098	1.98502234731836\\
0.725768143862515	2.36801909836561\\
1.36956393520291	2.65058247727238\\
1.90079146402238	2.84467600208762\\
2.32689533909663	2.97039684409577\\
2.66486093238406	3.04719472764852\\
2.93292528781976	3.09043587806671\\
3.14697196975999	3.11113324325111\\
3.31964373338877	3.11680725665986\\
};
\addplot [color=mycolor1, forget plot]
  table[row sep=crcr]{%
2.8295327054462	2.9466008779749\\
2.97642610855213	2.94207499671692\\
3.09988970543158	2.93039295654103\\
3.20464856418559	2.91371517181489\\
3.29437580196887	2.89351712903362\\
3.37192808911391	2.87080038913761\\
3.43953494245618	2.84623802679657\\
3.4989453355168	2.82027388436162\\
3.55153916974506	2.79319002120546\\
3.59841129928026	2.76515244759292\\
3.64043465239194	2.73624202568855\\
3.67830760138359	2.70647516533049\\
3.71258947966441	2.6758173996928\\
3.74372712841474	2.64419188113415\\
3.77207456717094	2.61148412790016\\
3.79790728059861	2.5775438631922\\
3.82143215068827	2.54218444063185\\
3.84279369705492	2.50518009028548\\
3.86207698134672	2.46626100924553\\
3.87930725265861	2.42510613230112\\
3.89444612840362	2.38133322903582\\
3.90738378625799	2.3344857638394\\
3.91792624803269	2.28401570541228\\
3.92577631430403	2.22926116270718\\
3.93050598965004	2.16941733623002\\
3.93151722677788	2.10349879512476\\
3.92798638587754	2.03029053102864\\
3.91878579530399	1.94828466613264\\
3.90237305398243	1.855599309194\\
3.87663518959904	1.74987637235859\\
3.83867086386778	1.62815736097154\\
3.78449111775298	1.48674276649606\\
3.70862241492482	1.32105686703013\\
3.60361677607863	1.12557497438373\\
3.45953832854664	0.893939279584528\\
3.26365014721632	0.619506570978022\\
3.00082463472069	0.29671869953972\\
2.65561419860998	-0.0762714533761386\\
2.21706286502895	-0.493729627568006\\
1.68619671892557	-0.938656377115806\\
1.08285674531277	-1.38265831675536\\
0.445432578270779	-1.79241778806565\\
-0.179885110508808	-2.14072410941843\\
-0.754317926290556	-2.41469139112317\\
-1.25584181744111	-2.61602985214543\\
-1.67895470657398	-2.75549727534181\\
-2.02899033676413	-2.84673786287883\\
-2.31615813868755	-2.9024182014246\\
-2.55160205772157	-2.93272494003078\\
-2.74546895585324	-2.9452462777605\\
-2.90625184398168	-2.94540172336058\\
-3.0407635978437	-2.9369743970697\\
-3.15435598513096	-2.92257310831705\\
-3.25119590115092	-2.90398444924689\\
-3.33452168008777	-2.88242517854157\\
-3.40685557962428	-2.85871771507956\\
-3.47017090376185	-2.83341037990418\\
-3.52602004104249	-2.80685924361002\\
-3.57563130948749	-2.77928368215287\\
-3.6199817995631	-2.75080399443141\\
-3.65985205939119	-2.72146673256919\\
-3.69586712014066	-2.69126152679475\\
-3.72852722176749	-2.6601319168179\\
-3.7582307012757	-2.6279818414424\\
-3.78529081653201	-2.59467885057009\\
-3.80994775181099	-2.56005469350327\\
-3.83237664138479	-2.52390363935476\\
-3.85269211510605	-2.48597865435748\\
-3.87094958091437	-2.44598536461657\\
-3.88714318232659	-2.40357354660641\\
-3.9012000727569	-2.35832569061562\\
-3.91297029725293	-2.30974195490434\\
-3.92221112123775	-2.25722055102696\\
-3.92856403549743	-2.20003225407342\\
-3.93152181524005	-2.13728729900023\\
-3.9303818073913	-2.06789240167837\\
-3.92417992018739	-1.99049506249897\\
-3.91159742794709	-1.90341179254178\\
-3.89082955793276	-1.8045367757561\\
-3.85940100403848	-1.69122852929405\\
-3.81390986646538	-1.56017608957687\\
-3.74968089091811	-1.40725677726019\\
-3.66031887909048	-1.22742176886083\\
-3.53719150856458	-1.01469576302685\\
-3.3689732107208	-0.76246914035007\\
-3.14160337218006	-0.464400090228629\\
-2.83938580218212	-0.116368668956703\\
-2.44832025295945	0.280211398732743\\
-1.96241815215748	0.71430772072177\\
-1.39154278411818	1.16286197831489\\
-0.765517221975054	1.59379223854703\\
-0.1285047127005	1.97543774602584\\
0.475270323667959	2.28718535710219\\
1.01486948515288	2.52389873550162\\
1.47704271994541	2.69265356832579\\
1.86252585744045	2.80630441153345\\
2.17972734440079	2.87832260557126\\
2.43967596952195	2.92021193495773\\
2.65316087503536	2.9408255990419\\
2.8295327054462	2.9466008779749\\
};
\addplot [color=mycolor1, forget plot]
  table[row sep=crcr]{%
2.44984468888118	2.85779721664325\\
2.60196738533444	2.8530955341335\\
2.73201868937063	2.84077925552878\\
2.84399722427826	2.82294384498009\\
2.941130399199	2.80107235469133\\
3.026009095818	2.77620459012577\\
3.10071052533583	2.7490607249949\\
3.1669019408754	2.72012984881909\\
3.22592470970097	2.68973265485586\\
3.27886093326873	2.65806546150423\\
3.32658553570199	2.62523087231385\\
3.36980663514464	2.59125886400063\\
3.40909660163786	2.55612095481477\\
3.44491572902982	2.51973927917976\\
3.47763000266974	2.48199180044322\\
3.50752405799971	2.44271446448007\\
3.53481009681523	2.40170078042435\\
3.55963324491399	2.35869907193627\\
3.58207357934416	2.31340744425876\\
3.6021448055327	2.26546633725287\\
3.61978930212512	2.21444836608594\\
3.63486894898146	2.15984497665486\\
3.6471507808511	2.10104925358721\\
3.65628602800719	2.03733401243435\\
3.66178046940091	1.96782409416737\\
3.66295318110665	1.89146159371687\\
3.65887966452276	1.80696267972253\\
3.64831397236724	1.71276488449446\\
3.62958291927769	1.60696463932235\\
3.60044417174799	1.48724713914983\\
3.55790006148047	1.35081571892447\\
3.49796295032051	1.19433822014812\\
3.41538138482406	1.01394710682261\\
3.3033694664129	0.805363083825396\\
3.15345104039838	0.564260916986537\\
2.95565145013395	0.287049461849504\\
2.69942978217378	-0.0277551991169861\\
2.37583077534834	-0.377545202656914\\
1.98105202249717	-0.75349516448623\\
1.52061048003519	-1.13953849261922\\
1.01179793653135	-1.51406798747553\\
0.481729687576415	-1.85483926486565\\
-0.039299878752971	-2.14502112512604\\
-0.525728760766699	-2.37694655936211\\
-0.961583347612355	-2.55184349143257\\
-1.34080776834139	-2.67677340867116\\
-1.66466717479823	-2.76113363997588\\
-1.93852205206722	-2.81418918953832\\
-2.16930815024689	-2.84386430177315\\
-2.36401462859855	-2.85641646336314\\
-2.52894863991	-2.8565588079033\\
-2.66948238030248	-2.8477415602283\\
-2.7900488682189	-2.83244662824804\\
-2.89424415876269	-2.81243887826061\\
-2.98496100689268	-2.78896168880488\\
-3.06451923449325	-2.76288202714237\\
-3.13477937039516	-2.7347952654056\\
-3.19723637238164	-2.70509977050788\\
-3.25309462510342	-2.67404948529879\\
-3.30332692445318	-2.64179071027718\\
-3.34872037797878	-2.60838758321233\\
-3.38991184884979	-2.5738394337785\\
-3.42741510892757	-2.53809221853144\\
-3.46164139897658	-2.50104554062317\\
-3.49291467668763	-2.46255625454029\\
-3.52148247681545	-2.42243928901094\\
-3.54752300405808	-2.3804660457283\\
-3.57114881240569	-2.33636051412372\\
-3.59240717557897	-2.2897930581589\\
-3.61127700091922	-2.24037166112528\\
-3.62766186012688	-2.18763024423159\\
-3.64137837682092	-2.13101349361061\\
-3.65213878904816	-2.06985743237638\\
-3.65952595267066	-2.00336476214815\\
-3.66295831902879	-1.93057379188937\\
-3.66164145467123	-1.85031962616379\\
-3.65450143563783	-1.76118632626276\\
-3.64009397438109	-1.661449251915\\
-3.61648164448542	-1.54900826961948\\
-3.58107074079176	-1.42131599087942\\
-3.53040086531508	-1.27531251851639\\
-3.45988808475131	-1.10739240636343\\
-3.36354413203345	-0.913455098139927\\
-3.23374299194595	-0.689131260121456\\
-3.06120008839267	-0.430331666141548\\
-2.83547558989729	-0.134305568090806\\
-2.54646084629997	0.198661406724477\\
-2.18725706829663	0.563087783868987\\
-1.75823788126407	0.946519661247175\\
-1.2707262246671	1.329691649195\\
-0.747503230270968	1.68991502715918\\
-0.218296874824636	2.0069639827157\\
0.288086702015426	2.26836939515059\\
0.750566434261	2.47117499192935\\
1.15832749650522	2.61998617758171\\
1.50938451881729	2.72342291650602\\
1.80742617457819	2.7910408610157\\
2.05885144943164	2.83151927338772\\
2.27075651633564	2.85195298808745\\
2.44984468888117	2.85779721664325\\
};
\addplot [color=mycolor1, forget plot]
  table[row sep=crcr]{%
2.15083773120232	2.82869718267466\\
2.30732436468486	2.82384624621737\\
2.4433118420125	2.81095668645537\\
2.56209976980254	2.79202803374775\\
2.66645262443518	2.76852402002715\\
2.75866143124573	2.74150316040845\\
2.84061379406736	2.71172005533557\\
2.91386069740201	2.67970163413908\\
2.97967530357787	2.64580343658018\\
3.03910245765468	2.61025057614152\\
3.09299924612863	2.57316716626594\\
3.14206760332443	2.53459711237977\\
3.18688011313304	2.49451841871177\\
3.22790007742325	2.45285255800203\\
3.26549675106933	2.40946998904446\\
3.29995644163108	2.36419255373392\\
3.33148996622944	2.31679321381851\\
3.36023675623487	2.26699337395324\\
3.38626569865824	2.21445786304212\\
3.40957259224826	2.15878749628598\\
3.43007386289986	2.09950900667044\\
3.44759591049805	2.03606201314232\\
3.46185912854562	1.96778258734543\\
3.47245522847033	1.89388290774894\\
3.47881599515294	1.81342648696486\\
3.48017099572729	1.72529860068735\\
3.47549109541112	1.6281719775772\\
3.46341402797019	1.52046878617669\\
3.44214804512423	1.40032192500813\\
3.40935054063824	1.26554232467367\\
3.3619819544403	1.11360553644566\\
3.29614389481357	0.941681805088982\\
3.2069287312535	0.746750454613378\\
3.08834196896464	0.525861154298206\\
2.93341318106459	0.276624430611918\\
2.7346792123074	-0.00199005653599627\\
2.48526439126132	-0.308540843598514\\
2.18069789166611	-0.637876894190381\\
1.82126405509475	-0.980283585224052\\
1.41407042577293	-1.32177794963178\\
0.973529098553298	-1.64611233998187\\
0.519302179515138	-1.93814165635475\\
0.0721749089467226	-2.18714232225843\\
-0.350271619857278	-2.38851645064079\\
-0.736296205869651	-2.54336468492444\\
-1.08035082362415	-2.65665726702385\\
-1.3818401671288	-2.73514641622193\\
-1.64334934794763	-2.7857745520406\\
-1.86907351690406	-2.81477095858355\\
-2.0637212393411	-2.82729792441939\\
-2.23187175557143	-2.82742663992165\\
-2.37766114657535	-2.81826702500692\\
-2.50467106829572	-2.80214488610198\\
-2.61592711340404	-2.78077361439386\\
-2.71394841871074	-2.75539991373785\\
-2.80081557583602	-2.72691936427101\\
-2.87823996979546	-2.6959645656945\\
-2.94762690053014	-2.66297077414461\\
-3.01012976176274	-2.62822399478868\\
-3.06669497313819	-2.59189576090128\\
-3.11809842125759	-2.55406793083113\\
-3.16497452169875	-2.51475000986999\\
-3.20783902965818	-2.47389082687288\\
-3.24710659152563	-2.43138586657657\\
-3.28310383789386	-2.38708115401469\\
-3.31607861296981	-2.340774278353\\
-3.3462057315549	-2.29221290360805\\
-3.37358945379951	-2.2410909216505\\
-3.39826266307694	-2.18704224225232\\
-3.4201825120606	-2.12963207414236\\
-3.43922205140252	-2.06834542350988\\
-3.45515705631335	-2.00257242185593\\
-3.46764689873355	-1.93159000279873\\
-3.4762078570285	-1.85453940282288\\
-3.48017669902106	-1.7703990179856\\
-3.47866172938115	-1.6779524117519\\
-3.47047783053574	-1.5757519309959\\
-3.45406155302938	-1.4620797890869\\
-3.42736251064461	-1.33491119630422\\
-3.3877092317964	-1.19188908423251\\
-3.33165320346531	-1.03032851580001\\
-3.2548076381554	-0.847282536978453\\
-3.15172288291488	-0.639720740842309\\
-3.01588443066694	-0.404894147748275\\
-2.8399829976851	-0.140972001347189\\
-2.61666888523843	0.151994797623094\\
-2.33999732459845	0.470856797629973\\
-2.00757182915667	0.808234985349818\\
-1.62289674730295	1.15214255424824\\
-1.1968205700236	1.48710610175714\\
-0.746798289601564	1.7969696755341\\
-0.293634844758991	2.06845614829498\\
0.143018691654359	2.29382996891854\\
0.548319090065009	2.47151065383526\\
0.913702541845487	2.60480312130623\\
1.23630864336571	2.69980902609968\\
1.51735250612403	2.76353078512541\\
1.76039282780355	2.80262768372586\\
1.9699877541269	2.82281419935843\\
2.15083773120232	2.82869718267466\\
};
\addplot [color=mycolor1, forget plot]
  table[row sep=crcr]{%
1.91158019794265	2.84376903183464\\
2.07163776438853	2.83879386724661\\
2.21285807640053	2.82539747440454\\
2.33791246105122	2.80546152865478\\
2.44912082680197	2.7804062830365\\
2.54846600359532	2.75128830803495\\
2.63762577791934	2.71888098047477\\
2.71801084845853	2.68373809671895\\
2.79080242003554	2.64624280460785\\
2.85698641658264	2.60664449345616\\
2.91738312264264	2.5650861096821\\
2.9726720213868	2.52162396989751\\
3.02341205008208	2.47624170916187\\
3.07005766382338	2.42885960651774\\
3.11297111497144	2.37934019668158\\
3.15243129157618	2.32749080597833\\
3.18863935121423	2.27306343349177\\
3.22172125493937	2.21575222433704\\
3.25172715531842	2.15518864183414\\
3.27862742163927	2.0909343333344\\
3.30230488909748	2.02247159947225\\
3.32254269007925	1.94919132453424\\
3.33900675849467	1.87037822283741\\
3.35122179204887	1.78519333555837\\
3.35853912660402	1.69265393427859\\
3.3600946677081	1.59161145435502\\
3.35475484794567	1.48072896213269\\
3.34104876892245	1.35846122100487\\
3.31708569791958	1.22304305571737\\
3.28045974439537	1.07249594527396\\
3.22814921378437	0.904669156349769\\
3.15642885792177	0.717340480465908\\
3.06083139967709	0.508411707762838\\
2.93622152863079	0.276241139701024\\
2.7770770319997	0.020149459472978\\
2.57809096843142	-0.258902162895629\\
2.3351780468082	-0.557553080740128\\
2.04683497078149	-0.869439194448154\\
1.71554714297952	-1.18511784885378\\
1.34865220465666	-1.49288119946767\\
0.958021807609145	-1.78050890078937\\
0.558338454421943	-2.03748111201489\\
0.164481555358333	-2.25680166315308\\
-0.21095810717441	-2.43573832577295\\
-0.559170187615061	-2.57538181366409\\
-0.875387966569343	-2.67947102389358\\
-1.15825539247116	-2.75307858502423\\
-1.40881879363976	-2.80155900902423\\
-1.62954531860348	-2.82988991803198\\
-1.82356115063245	-2.84235727819136\\
-1.99414405767677	-2.8424727142264\\
-2.14442741054747	-2.8330186505085\\
-2.27725282196893	-2.81614859152382\\
-2.39511589422416	-2.79350028839508\\
-2.50016473838548	-2.76630098678658\\
-2.59422508977545	-2.73545669331649\\
-2.6788363688267	-2.70162402151971\\
-2.75528999541142	-2.66526614123614\\
-2.82466552950899	-2.62669536187234\\
-2.88786267206039	-2.58610494914208\\
-2.94562849673728	-2.54359245794157\\
-2.99857995368833	-2.49917643438195\\
-3.04722197671781	-2.45280791977808\\
-3.09196160662951	-2.404377823958\\
-3.13311851320763	-2.35372093401608\\
-3.17093220955621	-2.30061708200279\\
-3.20556613189311	-2.24478980073838\\
-3.23710861645723	-2.18590264080907\\
-3.26557064483797	-2.12355319619622\\
-3.29088004627111	-2.05726478724901\\
-3.31287163385656	-1.98647567989964\\
-3.33127250427905	-1.91052568967319\\
-3.34568144341957	-1.82864005190303\\
-3.35554105860475	-1.73991058053854\\
-3.3601009295058	-1.64327446592406\\
-3.35836980714755	-1.53749170939389\\
-3.34905485717965	-1.42112337286191\\
-3.33048646848317	-1.29251486138864\\
-3.30052884841376	-1.1497918177212\\
-3.25648057704918	-0.990881452334094\\
-3.19497719768095	-0.813579719624154\\
-3.11192207726117	-0.615694432781554\\
-3.00249429743133	-0.395303860048757\\
-2.8613125677913	-0.151172713601745\\
-2.68286279053364	0.11665152416492\\
-2.46229728022884	0.40610058150193\\
-2.19663792258006	0.712364312934084\\
-1.88621757371078	1.02749962503193\\
-1.53590108495125	1.34076629312927\\
-1.15542361646115	1.63993480818183\\
-0.758353994407265	1.91336173347405\\
-0.359817457823202	2.15211845573736\\
0.0261883777012717	2.35132769387831\\
0.38885595919089	2.51028408905373\\
0.721431866190724	2.63157073461392\\
1.02096906501805	2.71974731916771\\
1.28744399189816	2.7801346433684\\
1.52272459753807	2.81795725135526\\
1.72968641530442	2.83786903536155\\
1.91158019794265	2.84376903183464\\
};
\addplot [color=mycolor1, forget plot]
  table[row sep=crcr]{%
1.71745758679747	2.89173975132996\\
1.8804644346493	2.88666054980203\\
2.02627889269441	2.87281811389132\\
2.15703696678287	2.85196438812595\\
2.27465720289688	2.82545742453582\\
2.38082766665929	2.79433294485042\\
2.47701312392245	2.75936681771373\\
2.56447232708102	2.72112681113683\\
2.64427925055575	2.68001399481002\\
2.71734473025739	2.63629501379885\\
2.78443660076244	2.59012667981921\\
2.84619740298685	2.54157425144665\\
2.90315928776839	2.49062457766663\\
2.95575602681665	2.43719505069377\\
3.00433216544909	2.38113909602147\\
3.04914937549984	2.32224873683445\\
3.09039002973783	2.26025461070945\\
3.12815794227413	2.19482368873312\\
3.16247611357899	2.12555485016604\\
3.19328118862104	2.05197240139194\\
3.22041418435718	1.97351760285347\\
3.24360687108098	1.88953829664735\\
3.26246300883696	1.79927683719634\\
3.27643346532593	1.70185676287416\\
3.28478411983947	1.59626907917125\\
3.28655547768413	1.48135976265904\\
3.28051324744219	1.35582129585754\\
3.26509006168158	1.21819291246397\\
3.23832053313869	1.06687700254827\\
3.1977756659706	0.900182964836113\\
3.14050925870732	0.716414564032025\\
3.06303931277295	0.514021630017732\\
2.96140177152125	0.291839165060381\\
2.831329743771	0.049431359624268\\
2.66862073969067	-0.212463672502666\\
2.46974101058715	-0.491439304059168\\
2.23265724939228	-0.782998939691912\\
1.95776837223382	-1.08040459012058\\
1.64866160836398	-1.37500967383526\\
1.31233089204491	-1.65718103512082\\
0.958592565229468	-1.91767088860956\\
0.598741748159326	-2.14903977073193\\
0.243852707791889	-2.34665123377138\\
-0.0967147415893769	-2.50894703232191\\
-0.416191996672758	-2.63704100182664\\
-0.710574296148952	-2.73391547893574\\
-0.978269118005141	-2.80354919199455\\
-1.2194938979658	-2.85019993873972\\
-1.43564881129259	-2.87792454612846\\
-1.62879099653703	-2.89031939215795\\
-1.80125105253692	-2.8904225079943\\
-1.95538246568011	-2.88071512412685\\
-2.09341468931788	-2.86317442696131\\
-2.21737842377636	-2.839346057336\\
-2.32907704654099	-2.81041845139936\\
-2.43008526104953	-2.77729025756574\\
-2.52176231073087	-2.74062749478147\\
-2.60527181932884	-2.70091001046648\\
-2.68160355032981	-2.65846814186684\\
-2.75159446004058	-2.61351096804578\\
-2.81594769357519	-2.56614758502651\\
-2.87524891462383	-2.51640268507215\\
-2.92997976318596	-2.46422750077732\\
-2.98052842983881	-2.40950694815937\\
-3.02719740245919	-2.35206359744722\\
-3.07020843125574	-2.29165892510769\\
-3.10970469913453	-2.22799215727647\\
-3.14575009217691	-2.16069690234515\\
-3.17832534667139	-2.08933568928717\\
-3.20732070773014	-2.01339248233482\\
-3.23252457198269	-1.93226324211468\\
-3.25360740779271	-1.84524466839372\\
-3.27010006373279	-1.75152142513656\\
-3.28136541998481	-1.65015247165899\\
-3.2865622713529	-1.54005769300379\\
-3.28460047717549	-1.42000696964363\\
-3.27408699466891	-1.28861533321583\\
-3.25326380716356	-1.14435014645906\\
-3.21994156970062	-0.985559539435554\\
-3.17143788509773	-0.81053568915016\\
-3.1045375211571	-0.61763149825431\\
-3.01550430731523	-0.405453196239712\\
-2.90019008781384	-0.173150576686815\\
-2.7543001876026	0.0791858516391993\\
-2.57387541216259	0.350044816530449\\
-2.35601728856905	0.636015792033466\\
-2.09979503696684	0.931474644994217\\
-1.80713233468106	1.22865077950936\\
-1.48333918421578	1.51825424663493\\
-1.13694684041011	1.79065828945332\\
-0.778714076415862	2.03735792720416\\
-0.42003797068775	2.25223275015673\\
-0.0712897935506828	2.43219826408154\\
0.259402493638034	2.57711605372954\\
0.566669600263679	2.6891458577117\\
0.847778827249968	2.77187104282629\\
1.10212364312125	2.82948518216731\\
1.33058671380521	2.86619067025825\\
1.53495371890201	2.88583489506905\\
1.71745758679747	2.89173975132996\\
};
\addplot [color=mycolor1, forget plot]
  table[row sep=crcr]{%
1.55816621582833	2.96411764280911\\
1.72369026381853	2.9589489634012\\
1.87357395094473	2.94471081198243\\
2.00951539727676	2.92302238097717\\
2.13308772686911	2.89516714394286\\
2.24571234639	2.8621446196781\\
2.34865136405567	2.8247181742914\\
2.44301121000513	2.78345639595214\\
2.52975215041391	2.73876741695496\\
2.60970030106496	2.69092650013397\\
2.68356005581673	2.64009761362272\\
2.75192569999174	2.58634982836324\\
2.81529151105137	2.52966933299994\\
2.87405996181646	2.46996775540036\\
2.9285478083254	2.40708735530879\\
2.97898991488656	2.34080353268013\\
3.02554067531358	2.27082499337512\\
3.06827285217283	2.196791835172\\
3.1071735880505	2.11827176852309\\
3.14213725341473	2.03475467638285\\
3.17295469322562	1.94564575898457\\
3.19929833125135	1.85025762396374\\
3.22070250845181	1.74780190281111\\
3.23653840868177	1.63738135042231\\
3.24598303048299	1.51798398475037\\
3.24798201588985	1.38848173911886\\
3.24120693772994	1.24763743617153\\
3.22400916860176	1.09412574273013\\
3.19437511598683	0.926576139898226\\
3.14989189658064	0.74364862494752\\
3.087738838116	0.54415514122812\\
3.00472841444392	0.327240017128452\\
2.89742883428089	0.0926281100414999\\
2.76240535369208	-0.159064235610916\\
2.59661056566265	-0.42598689659216\\
2.39792476504199	-0.704753322834074\\
2.16578860862381	-0.990290191232043\\
1.90179012891402	-1.27596971649145\\
1.61000317878624	-1.55411412857669\\
1.2968823130403	-1.8168464843066\\
0.970638174303505	-2.0571084413693\\
0.640213756305188	-2.26956197274868\\
0.314147274549137	-2.45111751934481\\
-0.000362505488881228	-2.60098113264671\\
-0.297988270820512	-2.720295012052\\
-0.575378380735594	-2.81155756868797\\
-0.830941166451551	-2.87801576920823\\
-1.06446367709236	-2.92315885645935\\
-1.27669672435658	-2.95036437111437\\
-1.46898741728075	-2.96269063235406\\
-1.64299300668765	-2.96278271928243\\
-1.80047857881911	-2.95285388917795\\
-1.94318577414053	-2.93471041232882\\
-2.07275514816701	-2.90979707621904\\
-2.19068588816658	-2.87924906665681\\
-2.29831984755406	-2.843942221781\\
-2.39684037593059	-2.80453778741935\\
-2.48727941435114	-2.76152026863806\\
-2.57052858919337	-2.71522830970021\\
-2.64735163136145	-2.66587917238847\\
-2.71839651149636	-2.61358761803144\\
-2.78420636050438	-2.55838002010735\\
-2.84522865642899	-2.5002044539993\\
-2.90182239101521	-2.43893739131148\\
-2.95426304297833	-2.37438750172038\\
-3.00274522025381	-2.30629695283276\\
-3.04738281614693	-2.23434050698934\\
-3.08820647051653	-2.15812264950087\\
-3.12515804747891	-2.07717295221539\\
-3.15808174399543	-1.990939889959\\
-3.1867113389556	-1.89878340212204\\
-3.2106529959443	-1.79996665418033\\
-3.22936297363835	-1.69364774480912\\
-3.24211962779781	-1.57887258222754\\
-3.24798929893982	-1.45457089871028\\
-3.24578622146153	-1.31955848477936\\
-3.23402770229988	-1.17255030912029\\
-3.21088785082031	-1.01219131125092\\
-3.17415655487047	-0.837114232777981\\
-3.12121566819313	-0.646036467539452\\
-3.04905171671751	-0.437909464786748\\
-2.95433316852084	-0.212132517993081\\
-2.83358778416639	0.0311657486123974\\
-2.68351580771023	0.290794598039761\\
-2.50145812401623	0.564167055530497\\
-2.28599477919338	0.847057253091699\\
-2.03757711193584	1.13357592830069\\
-1.75901774854111	1.41648320142122\\
-1.45562685451183	1.68787901637399\\
-1.13484527909285	1.94016869125852\\
-0.805392760759659	2.16705995139283\\
-0.476146011004946	2.36430279892949\\
-0.155067833387058	2.52997857399797\\
0.151540333209896	2.66432503572871\\
0.439349481079671	2.76924068531352\\
0.705928858253586	2.84766989253353\\
0.950428454900413	2.90303490669397\\
1.17316738454906	2.93880362351463\\
1.37523579774972	2.95821190730456\\
1.55816621582833	2.96411764280911\\
};
\addplot [color=mycolor1, forget plot]
  table[row sep=crcr]{%
1.42632528730357	3.05419055216814\\
1.59409885697326	3.04894192904819\\
1.74764607607782	3.03434727439337\\
1.88831768585849	3.01189675254528\\
2.01740026430351	2.98279289019926\\
2.13608416558777	2.94798796969752\\
2.24544811665552	2.90822043063386\\
2.34645448710049	2.86404765904994\\
2.43995094101876	2.81587408211711\\
2.52667551525986	2.76397436887494\\
2.60726315074524	2.7085119950182\\
2.68225238936669	2.64955362911531\\
2.75209140658278	2.58707985082535\\
2.81714284020386	2.52099268940941\\
2.8776870472672	2.45112041621127\\
2.93392350905193	2.37721996369354\\
2.98597013463019	2.29897729217643\\
3.03386020403657	2.2160059952028\\
3.07753665694582	2.12784443619997\\
3.11684338412892	2.0339517557063\\
3.15151313143606	1.93370319745457\\
3.18115160055983	1.82638539774865\\
3.20521736059578	1.71119259976722\\
3.22299732316753	1.58722523687895\\
3.23357786670449	1.45349302920453\\
3.23581235158743	1.3089257071312\\
3.2282869316483	1.15239574159971\\
3.20928847715888	0.982758977142025\\
3.176781336107	0.798920617843453\\
3.12840373601196	0.599935084967989\\
3.06149968552777	0.38514782253431\\
2.97320730405123	0.154383494569103\\
2.86062722894035	-0.0918240253086897\\
2.72109091958337	-0.351980889021323\\
2.55253274510374	-0.623406293831113\\
2.35393750479549	-0.902099020098744\\
2.12578916758129	-1.1827809605491\\
1.87040277410249	-1.45918525535049\\
1.59200788692452	-1.72459900904301\\
1.29649516293625	-1.97258135199058\\
0.990837210384202	-2.197696081303\\
0.682309937938043	-2.39607365215453\\
0.377709629271056	-2.56567186636938\\
0.0827467276445914	-2.70621088488075\\
-0.198285197101544	-2.81885870156143\\
-0.462566309044838	-2.90579312858928\\
-0.708603694444559	-2.96975905169869\\
-0.935974346036651	-3.01369849093327\\
-1.14503867672095	-3.04048466100225\\
-1.33667724745256	-3.05275737054036\\
-1.51207595388432	-3.05283983637206\\
-1.67256535682076	-3.04271255194964\\
-1.81950894294362	-3.02402253127754\\
-1.95423070636569	-2.99811154608017\\
-2.07797191745059	-2.96605230356283\\
-2.19186825631255	-2.92868580072754\\
-2.29694038051205	-2.8866561327032\\
-2.3940928301155	-2.84044100242064\\
-2.48411769306366	-2.7903773569584\\
-2.56770060761846	-2.73668222109411\\
-2.64542750422993	-2.67946910987757\\
-2.71779105258215	-2.61876051726236\\
-2.7851961466965	-2.55449698645495\\
-2.84796398665464	-2.48654322508971\\
-2.9063344415271	-2.4146916680026\\
-2.96046643468488	-2.33866383248366\\
-3.01043610126227	-2.25810976873771\\
-3.05623244371903	-2.1726058926298\\
-3.09775016794598	-2.08165151011889\\
-3.13477933218787	-1.98466441800741\\
-3.16699140132554	-1.88097611459704\\
-3.19392129598201	-1.76982740563066\\
-3.21494510178343	-1.65036558457997\\
-3.2292533268044	-1.52164495079917\\
-3.23582007028983	-1.38263325815509\\
-3.2333693489984	-1.23222780416891\\
-3.22034133184322	-1.06928627425662\\
-3.19486361413493	-0.892679032711538\\
-3.15473614964176	-0.701370955594936\\
-3.09744308405396	-0.494541380754747\\
-3.02020998960036	-0.271748964553959\\
-2.92012930110728	-0.0331421889446298\\
-2.79437683989129	0.220296299061973\\
-2.64053313743929	0.4865035954473\\
-2.45699943943893	0.762146269718007\\
-2.24345818717511	1.04256505003492\\
-2.00128002875728	1.3219347994068\\
-1.73374676631364	1.59368386301882\\
-1.44597222977026	1.85114025281027\\
-1.14447704690877	2.08828044420864\\
-0.836487911591588	2.30039820704417\\
-0.529131242108844	2.48452621841593\\
-0.228720064349343	2.63952985558765\\
0.05972043374006	2.76590334036294\\
0.332641313986093	2.86537701210694\\
0.587913607060129	2.94046425247022\\
0.824614911507096	2.99404850797217\\
1.0427491736426	3.02906396360623\\
1.24296707789219	3.04828205038319\\
1.42632528730357	3.05419055216814\\
};
\addplot [color=mycolor1, forget plot]
  table[row sep=crcr]{%
1.31655348510211	3.15637921267897\\
1.48642982758109	3.15105639663561\\
1.6433349255677	3.13613505069415\\
1.78835014147596	3.11298458113963\\
1.92253093721614	3.08272520919187\\
2.04687406090608	3.04625525378925\\
2.1622986584819	3.00427899608015\\
2.2696368592467	2.95733267011823\\
2.36963046158222	2.9058073600355\\
2.46293125979115	2.84996834893613\\
2.55010326760306	2.78997090464124\\
2.63162562069553	2.72587271692795\\
2.70789531091329	2.65764330257214\\
2.7792291529804	2.5851707245444\\
2.8458645409221	2.50826596733315\\
2.90795864101281	2.42666529630212\\
2.96558571049066	2.34003092262652\\
3.01873224249287	2.24795031068524\\
3.06728963207688	2.14993451539684\\
3.11104405126304	2.04541603839254\\
3.14966323256332	1.93374686338077\\
3.18267991879898	1.81419759669302\\
3.20947188448626	1.68595902752834\\
3.22923873353934	1.54814796410214\\
3.24097621960527	1.39981992001385\\
3.24344974021805	1.23999211740366\\
3.23517007051758	1.06768127718753\\
3.21437646873718	0.881961600425187\\
3.17903506682906	0.682048823881183\\
3.12686378827857	0.467415571580784\\
3.05539826716717	0.237940380069017\\
2.9621149441898	-0.00591354243759519\\
2.8446252500433	-0.262904611596984\\
2.70094538850893	-0.530834152049075\\
2.52982713787425	-0.806428962524813\\
2.33110705889611	-1.08534184026105\\
2.10600184068736	-1.36232084982776\\
1.8572609218841	-1.63156753781995\\
1.58910066947964	-1.88725091001963\\
1.30689405569773	-2.12408573046757\\
1.01666210925465	-2.33784925911224\\
0.724476500123249	-2.52572184134781\\
0.435905312060046	-2.68639164379551\\
0.155607403891534	-2.819935532752\\
-0.112877160118432	-2.92754352702366\\
-0.367152187414235	-3.01117485959625\\
-0.605855357627645	-3.07322211607211\\
-0.828473614608814	-3.11623165194633\\
-1.03513592590833	-3.14269931239132\\
-1.22641845594262	-3.15493938537975\\
-1.40318038950259	-3.15501361429271\\
-1.56643609265679	-3.14470381180652\\
-1.71726173041519	-3.12551291097201\\
-1.85673089174345	-3.09868249350657\\
-1.98587279537864	-3.06521830682594\\
-2.10564706617362	-3.02591824464424\\
-2.21693006990254	-2.98139949136202\\
-2.32050890884273	-2.93212306146033\\
-2.41708018607659	-2.87841494577717\\
-2.50725146172323	-2.82048366252365\\
-2.59154394175497	-2.75843433336419\\
-2.67039538429476	-2.69227956197409\\
-2.7441625133768	-2.62194745266871\\
-2.81312242894797	-2.54728711579714\\
-2.87747262214401	-2.46807199484482\\
-2.93732926870396	-2.3840013380924\\
-2.99272349836275	-2.29470014050124\\
-3.04359533907467	-2.19971791269805\\
-3.08978502654407	-2.09852670780088\\
-3.13102136894168	-1.99051897056593\\
-3.16690688734356	-1.87500598855174\\
-3.19689954892627	-1.7512180482722\\
-3.22029112401107	-1.61830786078233\\
-3.2361826053834	-1.47535944913705\\
-3.24345783448158	-1.32140549834562\\
-3.24075761963484	-1.15545713000458\\
-3.22645835672939	-0.976551065658248\\
-3.1986615884	-0.783819914013145\\
-3.15520403655315	-0.576591333807192\\
-3.09370104580571	-0.354520223017647\\
-3.01163908232814	-0.117753668347263\\
-2.90653301678303	0.13288011034839\\
-2.77615849394102	0.39568075177871\\
-2.61885563696615	0.667920990033036\\
-2.43387633491964	0.94578097461004\\
-2.22171710630155	1.2244281019515\\
-1.98435422762312	1.49828116328787\\
-1.72529430245151	1.76145443214308\\
-1.4493851392521	2.00831828601761\\
-1.16239594134893	2.23406294420578\\
-0.870448039756891	2.4351388225132\\
-0.579423174651817	2.60948215525256\\
-0.294473622727724	2.7565021318834\\
-0.0197126607919893	2.87687296629366\\
0.241898300129466	2.97221327049281\\
0.488499802108194	3.04473818059727\\
0.719179682800395	3.09694755689515\\
0.933771636258873	3.1313832330886\\
1.13265092126358	3.15046252545726\\
1.31655348510211	3.15637921267897\\
};
\addplot [color=mycolor1, forget plot]
  table[row sep=crcr]{%
1.2248568044801	3.26583710005082\\
1.39676323735686	3.2604434592983\\
1.55679049023156	3.24521862229108\\
1.70581451982452	3.221422185182\\
1.84470834389625	3.19009450224251\\
1.97431066725544	3.15207701567812\\
2.09540597774779	3.10803389929256\\
2.20871277539915	3.05847284285746\\
2.31487729233147	3.00376378484091\\
2.41447069016377	2.94415504221871\\
2.50798823846962	2.87978668139907\\
2.59584937599353	2.81070120318579\\
2.67839784791313	2.73685173789504\\
2.75590131620158	2.65810800822904\\
2.82854997541634	2.57426034850786\\
2.89645379062483	2.48502209186076\\
2.95963802332905	2.3900306692737\\
3.01803673979666	2.28884782091383\\
3.07148401943925	2.18095941565358\\
3.11970261709477	2.06577552528384\\
3.16228990657625	1.94263162365297\\
3.19870107764335	1.81079209738987\\
3.2282298224622	1.66945768180885\\
3.24998719594592	1.51777898202722\\
3.26288005083467	1.35487889119504\\
3.26559152860871	1.17988741327663\\
3.25656761648345	0.991992989171771\\
3.23401578224707	0.790514626554382\\
3.19592403882969	0.57499846910328\\
3.140111030269	0.345340222336652\\
3.06431894751212	0.101930268413435\\
2.96635975428256	-0.154189284546743\\
2.84431936093831	-0.421176800253831\\
2.69681232712494	-0.696285159889052\\
2.52326143687634	-0.97583839918236\\
2.32415570313803	-1.25533021869393\\
2.10122515708008	-1.52966670070289\\
1.85747174494847	-1.79354229005829\\
1.59701994460964	-2.04189681903875\\
1.32479505155844	-2.27036877288047\\
1.04608570702848	-2.47565308395311\\
0.766079005791641	-2.65569655093843\\
0.489457644572723	-2.80971028876742\\
0.220121090286436	-2.9380259158805\\
-0.0389481193804397	-3.04185234037724\\
-0.285688301139277	-3.12299646489722\\
-0.518869419207026	-3.18359900414887\\
-0.737952297696529	-3.22591629846879\\
-0.942931433327034	-3.2521596106048\\
-1.1341854907516	-3.26438966961406\\
-1.31234862482834	-3.26445694460304\\
-1.47820740793431	-3.25397586124015\\
-1.63262284267026	-3.23432191795935\\
-1.77647420393349	-3.20664275108662\\
-1.91062047148805	-3.17187657033859\\
-2.03587515139584	-3.13077349442778\\
-2.15299081170959	-3.08391696630888\\
-2.26265034563407	-3.0317436138914\\
-2.36546264670328	-2.97456072077281\\
-2.46196095618675	-2.9125609786977\\
-2.55260259969955	-2.84583449679776\\
-2.63776917271905	-2.77437821224611\\
-2.71776647992242	-2.69810293469696\\
-2.79282370044644	-2.61683829985855\\
-2.86309135893114	-2.53033593213455\\
-2.92863774711611	-2.4382711419608\\
-2.98944347778396	-2.3402435260626\\
-3.04539387669787	-2.23577691297095\\
-3.0962689453209	-2.12431921703683\\
-3.14173067841532	-2.00524294895247\\
-3.1813076247535	-1.87784739843053\\
-3.214376776523	-1.74136387446161\\
-3.24014321977739	-1.59496587513795\\
-3.25761854918551	-1.43778666219383\\
-3.26559993659443	-1.26894740114535\\
-3.2626530391848	-1.0875996971195\\
-3.24710370237166	-0.892986799456058\\
-3.21704561516177	-0.684527589669214\\
-3.17037344658949	-0.461926123467113\\
-3.1048528720396	-0.225306196851926\\
-3.01823906938497	0.024635631298619\\
-2.90845193438096	0.286474238038438\\
-2.77380746201065	0.557924329770409\\
-2.61328940594513	0.835770476163347\\
-2.42682507743038	1.11590042265069\\
-2.21550988553163	1.39347465647061\\
-1.98171681985069	1.66323968274924\\
-1.72903932359748	1.91995352116603\\
-1.46205156141205	2.15885263362157\\
-1.18591914600036	2.37606806816528\\
-0.905936065855446	2.56890781903211\\
-0.62708081388796	2.73596014692679\\
-0.353670263989849	2.87702212913656\\
-0.0891533241573329	2.99289809453567\\
0.163954213030953	3.0851308914121\\
0.404021800262236	3.15572495189461\\
0.630184158086769	3.2069025595404\\
0.842188542214752	3.24091398383363\\
1.04023911444572	3.25990527538251\\
1.2248568044801	3.26583710005082\\
};
\addplot [color=mycolor1, forget plot]
  table[row sep=crcr]{%
1.14821861877544	3.37822194099113\\
1.32211422966463	3.37275975697398\\
1.48506577373199	3.35725102144531\\
1.63779453771928	3.33285774558015\\
1.78103188774723	3.30054550382355\\
1.91549014928952	3.26109905551922\\
2.04184294580084	3.21513954137717\\
2.16071243679431	3.1631413777036\\
2.27266135883972	3.10544775489936\\
2.37818821564684	3.04228417982872\\
2.47772434226298	2.97376984761565\\
2.57163187060185	2.89992684386039\\
2.66020185224264	2.82068730874675\\
2.74365196007394	2.73589877482412\\
2.82212330574859	2.64532794657016\\
2.89567598835189	2.54866324231546\\
2.96428304456024	2.44551648425036\\
3.02782251609899	2.33542421486215\\
3.08606740302247	2.21784925275971\\
3.13867335214308	2.09218329214815\\
3.18516406667791	1.9577516125178\\
3.22491465336956	1.81382130969792\\
3.25713349629051	1.65961488879327\\
3.28084382325812	1.49433155665004\\
3.29486697834341	1.31717906207993\\
3.29781058888808	1.12741933821346\\
3.28806632819085	0.924431288084098\\
3.2638237265217	0.707793477853093\\
3.22310816963416	0.477387781564748\\
3.16385222464998	0.233521602091018\\
3.08400873558892	-0.0229392581382749\\
2.98171041137278	-0.290442608907457\\
2.85547265898182	-0.566650817741802\\
2.70442391469964	-0.848401476834022\\
2.52853252816493	-1.13175937750553\\
2.32878609395836	-1.41218193840917\\
2.10727494205935	-1.68479879629618\\
1.86714241956969	-1.9447765374198\\
1.6123914789193	-2.18771138208102\\
1.34757265013792	-2.40997843292828\\
1.07740931147432	-2.60897387674382\\
0.806429669689267	-2.78321434775047\\
0.538666472900795	-2.93229411050522\\
0.277460863216762	-3.05673154096928\\
0.0253774981168844	-3.15775215772752\\
-0.215785267024473	-3.23705505106691\\
-0.444918584837936	-3.29659819762263\\
-0.661489845233607	-3.33842296110853\\
-0.865417916930842	-3.36452454422826\\
-1.05695405637369	-3.37676584910671\\
-1.23657808233503	-3.37682731799263\\
-1.40491358433021	-3.36618381742783\\
-1.56266213479399	-3.34610016813901\\
-1.71055440301189	-3.31763840952904\\
-1.84931521840959	-3.28167160219613\\
-1.97963952727928	-3.23890053206647\\
-2.10217647246083	-3.18987093478668\\
-2.21751926609763	-3.1349897903404\\
-2.3261989864738	-3.07453988823723\\
-2.42868084382842	-3.0086922954671\\
-2.52536180090137	-2.93751663369261\\
-2.61656869841968	-2.86098924012431\\
-2.70255623138843	-2.7789993881692\\
-2.78350426090654	-2.69135380942585\\
-2.8595140414808	-2.59777981075905\\
-2.9306030087679	-2.49792733707459\\
-2.996697820998	-2.39137040748995\\
-3.0576253941722	-2.2776084643912\\
-3.11310173490182	-2.15606833597676\\
-3.16271847950798	-2.02610773800262\\
-3.2059272260731	-1.88702154284204\\
-3.24202204076467	-1.73805243153392\\
-3.27012098635517	-1.57840801168737\\
-3.28914822470641	-1.40728699800874\\
-3.29781925092471	-1.22391753054802\\
-3.29463316440711	-1.02761098651421\\
-3.27787753672069	-0.817834448416722\\
-3.24565321045131	-0.594303910063707\\
-3.19592780817642	-0.357097800483401\\
-3.12662703308031	-0.106785954479507\\
-3.0357708082547	0.155437482067335\\
-2.92165557654961	0.427637018465533\\
-2.78307377857452	0.707062532913073\\
-2.61954727411969	0.990151953958752\\
-2.43153652126608	1.27263826333063\\
-2.22057774153498	1.54977366982352\\
-1.98930312676034	1.81665734322728\\
-1.74131842508857	2.06862261920746\\
-1.48094487598371	2.30161695110491\\
-1.21286748687648	2.51250421836134\\
-0.941755014781069	2.69923773288845\\
-0.671919466907365	2.86088622283792\\
-0.4070651975853	2.99753033745599\\
-0.150149193850236	3.11007118332234\\
0.0966532046284112	3.19999973697445\\
0.331900943995047	3.26916921812658\\
0.55478867041782	3.31959838882448\\
0.765025060922061	3.35331887615604\\
0.962709256903624	3.3722680462706\\
1.14821861877544	3.37822194099113\\
};
\addplot [color=mycolor1, forget plot]
  table[row sep=crcr]{%
1.08432098958412	3.48958906878031\\
1.26016263041394	3.48406059583351\\
1.42584987037089	3.46828664202598\\
1.58198865342118	3.44334418104502\\
1.72920180552377	3.41013078867507\\
1.86810226553408	3.36937710519764\\
1.99927420417125	3.32166089092418\\
2.12326000419845	3.26742106314437\\
2.2405513984182	3.20697074047129\\
2.35158338521673	3.14050876766226\\
2.45672982678386	3.06812949587693\\
2.55629986906307	2.98983079230571\\
2.65053450445137	2.90552038458823\\
2.73960273450841	2.81502073768703\\
2.82359688986924	2.71807273618282\\
2.90252673922986	2.6143385213636\\
2.97631208113869	2.50340392539018\\
3.04477357591937	2.38478106818833\\
3.10762165833728	2.25791184958972\\
3.16444349775839	2.12217329129565\\
3.21468817177836	1.97688596873151\\
3.25765053035063	1.82132712217\\
3.29245469720726	1.65475043399235\\
3.31803883456972	1.47641485992724\\
3.33314372828983	1.28562521396671\\
3.33630894286411	1.08178726562651\\
3.32588168137908	0.864479657741579\\
3.30004485434708	0.633543639329572\\
3.25687178373545	0.389189022073703\\
3.19441473769967	0.132110575790638\\
3.1108321596272	-0.136396726777785\\
3.00455406714634	-0.414341923537122\\
2.87447621685277	-0.698985909754783\\
2.72016211379556	-0.986859236316647\\
2.54202061056272	-1.27387130908329\\
2.34142036981858	-1.55551849505863\\
2.12070586008893	-1.82717689529902\\
1.8830952937858	-2.08444177624029\\
1.63246628524926	-2.32345876586698\\
1.37306191215871	-2.5411897987414\\
1.10916823294834	-2.73557159187355\\
0.844817383951105	-2.90555079510593\\
0.583558255283464	-3.05100778424301\\
0.32831573119089	-3.17260093950908\\
0.0813380657245664	-3.27157072675212\\
-0.155783455734006	-3.34953920535941\\
-0.382045527197197	-3.40833035585505\\
-0.596937432007435	-3.44982485664703\\
-0.800337824945961	-3.47585307111208\\
-0.992416369819534	-3.48812343909048\\
-1.17354739216084	-3.48818010731902\\
-1.34423845207748	-3.47738266288519\\
-1.50507393048241	-3.4569013050777\\
-1.6566721384329	-3.42772193538039\\
-1.7996537604388	-3.39065695771374\\
-1.93461930124317	-3.34635878733215\\
-2.06213336918966	-3.29533405368629\\
-2.18271392744576	-3.23795723094195\\
-2.29682497433818	-3.17448296668221\\
-2.40487142121653	-3.10505674818091\\
-2.50719519661724	-3.02972379116863\\
-2.60407181297479	-2.94843619727989\\
-2.69570679019512	-2.8610585351157\\
-2.78223144719621	-2.76737208114765\\
-2.86369765836433	-2.66707803048689\\
-2.94007123848177	-2.55980007038668\\
-3.0112236804157	-2.44508681580146\\
-3.07692204077251	-2.32241474987444\\
-3.1368168701386	-2.1911925053699\\
-3.19042824292162	-2.05076757615838\\
-3.2371301915545	-1.90043686597723\\
-3.27613423396141	-1.73946285848749\\
-3.30647325143702	-1.56709759995265\\
-3.32698777524983	-1.38261705772837\\
-3.33631780438843	-1.18536862902833\\
-3.33290458143599	-0.974834417079117\\
-3.3150081689874	-0.750712059173906\\
-3.28074788496511	-0.513012992199026\\
-3.22817310695987	-0.262174680466433\\
-3.1553708019304	0.00082170216403196\\
-3.06061238059654	0.274343052643648\\
-2.94253533527052	0.556027086251549\\
-2.80034470250325	0.842761973870159\\
-2.6340074807505	1.13074781218675\\
-2.44440357023709	1.41565556019632\\
-2.23339472179801	1.69288079372895\\
-2.00378251961062	1.95786588536709\\
-1.75914769036507	2.20644272094582\\
-1.50359042146601	2.43513784071188\\
-1.2414151827585	2.64138844283209\\
-0.976814721288001	2.82363934648277\\
-0.713602903748065	2.98131945678464\\
-0.455028473577867	3.11472105262322\\
-0.203679564761306	3.22481906642957\\
0.0385301600539593	3.31306897526129\\
0.270314178817421	3.38121429414528\\
0.490928261198941	3.43112311143594\\
0.700069202602977	3.46466199268161\\
0.897772817418163	3.4836072979851\\
1.08432098958412	3.48958906878031\\
};
\addplot [color=mycolor1, forget plot]
  table[row sep=crcr]{%
1.03135213633093	3.59637090839305\\
1.20907004504027	3.59077914425909\\
1.37729149238138	3.57475983814494\\
1.53653710018345	3.54931719978665\\
1.68734744553568	3.51528855270082\\
1.83025841110878	3.47335468467441\\
1.96578322832601	3.42405169926548\\
2.09439954870802	3.36778297747839\\
2.21654011806895	3.30483039105101\\
2.33258587487359	3.2353642907107\\
2.44286051696477	3.15945205909332\\
2.54762576750443	3.07706520209026\\
2.64707671968337	2.98808508293693\\
2.74133675496009	2.89230750409191\\
2.83045161912324	2.78944643230707\\
2.91438231446358	2.67913725852437\\
2.99299653716238	2.56094010047323\\
3.06605847196424	2.43434380459931\\
3.13321687072209	2.29877149590738\\
3.19399151237957	2.1535887667974\\
3.24775840081393	1.99811588966944\\
3.29373444147692	1.83164577081624\\
3.33096288903687	1.65346970017156\\
3.35830161214142	1.46291322151251\\
3.37441719098545	1.25938452546964\\
3.37778900764212	1.0424374575076\\
3.36672866651923	0.811850263579898\\
3.33942099285791	0.567719246035073\\
3.29399299369215	0.310563285224406\\
3.22861581888925	0.0414306205472954\\
3.14164114942691	-0.238006225302083\\
3.03176704475577	-0.525386795849033\\
2.89821937358575	-0.81765279173427\\
2.74092516593776	-1.11111257881753\\
2.56064673823679	-1.40159202394692\\
2.35904427078028	-1.68466770128535\\
2.13864271113069	-1.95595869568967\\
1.90269631859855	-2.21143571861846\\
1.65496642560124	-2.44769807494091\\
1.39944760586285	-2.66217435664607\\
1.14008737146357	-2.85322028953649\\
0.880541753683481	-3.02011052134225\\
0.623995949698562	-3.16294213169387\\
0.37306151489212	-3.2824802789669\\
0.12974553557614	-3.37997902331034\\
-0.104523279302319	-3.45700515846479\\
-0.328830742118102	-3.51528377119382\\
-0.542693766928892	-3.55657483654339\\
-0.745971785698193	-3.58258266842174\\
-0.938782731174552	-3.59489523402033\\
-1.1214290670792	-3.59494797211015\\
-1.29433609777269	-3.58400615651164\\
-1.45800263195444	-3.56316030107577\\
-1.61296283916448	-3.53333004490172\\
-1.75975757063237	-3.49527301601026\\
-1.89891327739221	-3.44959614620652\\
-2.03092676136875	-3.39676771434008\\
-2.15625421059277	-3.33712901411439\\
-2.27530321811131	-3.27090499549261\\
-2.38842672143102	-3.19821354850303\\
-2.49591800484421	-3.11907331953905\\
-2.59800607440349	-3.03341010428896\\
-2.69485084646195	-2.94106197440155\\
-2.78653769193132	-2.84178338833293\\
-2.87307095877536	-2.73524862823342\\
-2.9543661659553	-2.62105500931138\\
-3.03024063679595	-2.498726439141\\
-3.10040243574305	-2.36771807355573\\
-3.16443761172733	-2.22742303238647\\
-3.22179596191571	-2.07718240693662\\
-3.27177584626739	-1.91630010695898\\
-3.31350904610332	-1.74406443604791\\
-3.34594730888101	-1.5597785997573\\
-3.36785308394414	-1.36280254438075\\
-3.37779802106989	-1.15260843741755\\
-3.37417399067421	-0.92885149910718\\
-3.35522247816075	-0.691456475521483\\
-3.31908880564762	-0.440717488068642\\
-3.26390712736859	-0.177405091985249\\
-3.18791975818069	0.0971307505286315\\
-3.08962942159058	0.380878683043629\\
-2.96797524458198	0.671126334464415\\
-2.82251367488618	0.96448540313711\\
-2.65357638041745	1.25699868441593\\
-2.46237236697735	1.54433379386422\\
-2.25100483013732	1.82204987931364\\
-2.02238635691151	2.08590404271753\\
-1.7800567491228	2.33215056177113\\
-1.52792968019001	2.55778436909587\\
-1.27000986510104	2.76069222248298\\
-1.01012604337771	2.93969646714663\\
-0.751716524415859	3.09449943478099\\
-0.49768777443629	3.22555383260815\\
-0.250348979705656	3.33389203355759\\
-0.0114119588916324	3.42094545037425\\
0.217961434923565	3.48837750643975\\
0.437083598214068	3.53794407781902\\
0.645653665675828	3.57138665281924\\
0.843670278731899	3.59035728676248\\
1.03135213633093	3.59637090839305\\
};
\addplot [color=mycolor1, forget plot]
  table[row sep=crcr]{%
0.987871498742845	3.69541322260165\\
1.16735455738493	3.68976238204575\\
1.33788136567027	3.67352014927281\\
1.49990992033289	3.64762963289456\\
1.6539191227879	3.61287609205577\\
1.80038590003472	3.56989587477314\\
1.93976812181382	3.51918672685299\\
2.07249189596827	3.46111825705069\\
2.19894201178193	3.39594180113781\\
2.31945449842671	3.32379926151298\\
2.43431044689757	3.24473073832322\\
2.54373039816002	3.15868094038059\\
2.64786872602072	3.06550449381061\\
2.74680754427043	2.96497037394905\\
2.84054975122996	2.85676578883048\\
2.92901090095079	2.74049995545068\\
3.01200967177911	2.61570834557892\\
3.0892568060592	2.48185814693752\\
3.16034254016137	2.33835589572287\\
3.22472275782997	2.18455848992647\\
3.28170441353351	2.01978908193851\\
3.33043122107849	1.84335964832785\\
3.36987121904565	1.6546022912774\\
3.39880862692961	1.45291144505109\\
3.41584337586375	1.23779899173702\\
3.41940274611399	1.00896362031364\\
3.40777046478862	0.766374328897531\\
3.37913904597866	0.510365512535147\\
3.33169055130225	0.241737460153689\\
3.26370865971377	-0.0381485193979178\\
3.17372040069302	-0.327295984552708\\
3.06065902694148	-0.623039582008411\\
2.92403111682985	-0.922071362753897\\
2.76406319399621	-1.2205420859526\\
2.58179906418554	-1.51424124241489\\
2.37912182003006	-1.79884340078391\\
2.1586854567446	-2.07019150701435\\
1.9237584818392	-2.32457581984514\\
1.67800075380296	-2.5589652969476\\
1.42520869896009	-2.77115794478507\\
1.16906838814842	-2.95983473943839\\
0.912949989095865	-3.12452187550422\\
0.659764068320027	-3.26548192892061\\
0.411885248544655	-3.38356235916136\\
0.171136380764908	-3.48002953470973\\
-0.0611809602427222	-3.55641062829667\\
-0.284226117261211	-3.61435758803426\\
-0.497544137563553	-3.65553962958098\\
-0.700987315142112	-3.68156479488022\\
-0.894641229489225	-3.69392746799862\\
-1.07875959830396	-3.69397702581169\\
-1.25370967242688	-3.68290245745213\\
-1.41992818321254	-3.66172824258476\\
-1.57788686152857	-3.63131759631898\\
-1.72806608400622	-3.59238008450817\\
-1.87093507956281	-3.5454814332458\\
-2.0069372018364	-3.49105403593258\\
-2.13647893981229	-3.42940718912599\\
-2.25992153593609	-3.36073648046014\\
-2.37757427242732	-3.28513203497942\\
-2.48968865497345	-3.20258552877778\\
-2.59645286285159	-3.11299602701529\\
-2.69798594721313	-3.01617481964689\\
-2.79433135048951	-2.91184953141287\\
-2.88544939831458	-2.79966788876174\\
-2.97120849235176	-2.67920164911807\\
-3.05137482238792	-2.54995134904115\\
-3.12560053738497	-2.41135271672331\\
-3.19341049118051	-2.26278582631016\\
-3.25418793804369	-2.10358834462691\\
-3.30715992998904	-1.93307451972076\\
-3.35138369660574	-1.75056184753966\\
-3.38573599681602	-1.55540755584059\\
-3.40890832391732	-1.34705704199601\\
-3.41941187170032	-1.12510601210963\\
-3.41559718901055	-0.889377051198837\\
-3.39569418203983	-0.640009436839153\\
-3.3578781132558	-0.377557973486523\\
-3.30036588172902	-0.103092462559662\\
-3.22154350942313	0.181715443294424\\
-3.12012001649417	0.474535749374782\\
-2.99529505715178	0.772374077051107\\
-2.84691925271312	1.07163416394608\\
-2.67561978752376	1.36825898805647\\
-2.4828628388947	1.65794660849366\\
-2.27093128469621	1.93641959476136\\
-2.0428107979076	2.19971170207586\\
-1.80199636098332	2.44442807993632\\
-1.55224837432126	2.66793932616076\\
-1.29733702121988	2.86848425234937\\
-1.04081254973412	3.04517622282719\\
-0.785829042578658	3.19792659241119\\
-0.535034581828675	3.32731083218641\\
-0.29052660598202	3.43440654295955\\
-0.0538612669716209	3.52062911387546\\
0.173899109718393	3.58758339731566\\
0.392116549741104	3.63694155623125\\
0.600499387624035	3.67035030112556\\
0.799025320798523	3.68936596235326\\
0.987871498742845	3.69541322260165\\
};
\addplot [color=mycolor1, forget plot]
  table[row sep=crcr]{%
0.952714132411607	3.78404114737501\\
1.13380397064677	3.77833685920762\\
1.30637325103211	3.76189732263229\\
1.47083442317104	3.73561543393371\\
1.62762067567458	3.70023265778367\\
1.77716446579211	3.65634701927888\\
1.91988117948408	3.60442233459398\\
2.05615667234629	3.54479760321261\\
2.18633759746216	3.47769588836797\\
2.31072359156383	3.40323231173144\\
2.42956054432884	3.32142100801071\\
2.54303430814367	3.23218104894099\\
2.65126431602905	3.13534147560688\\
2.75429666745447	3.03064569156601\\
2.85209632277312	2.91775558261401\\
2.94453812726338	2.79625585571411\\
3.03139647872044	2.66565924095871\\
3.11233357545849	2.52541338509484\\
3.18688635625902	2.37491048747444\\
3.25445249690282	2.213500986042\\
3.31427619007051	2.04051287582365\\
3.36543493760153	1.85527849771146\\
3.40682924953868	1.65717080131025\\
3.43717797286668	1.44565104805399\\
3.45502291556962	1.22032951214688\\
3.45874735579992	0.9810397392692\\
3.44661367496507	0.72792509839229\\
3.41682531964459	0.461533528993223\\
3.36761704032181	0.182912558011353\\
3.29737431525226	-0.106307761422098\\
3.20477771495196	-0.403860382744142\\
3.08896096889273	-0.706833750006439\\
2.9496639046841	-1.01172793986564\\
2.78735554857161	-1.31458405779946\\
2.60330137337451	-1.61118403748095\\
2.39955419368833	-1.89730252117866\\
2.17886067707563	-2.16897849116513\\
1.94449207431254	-2.42276690311203\\
1.70002351197028	-2.65593308604974\\
1.44909583639802	-2.86656477541579\\
1.19519465816116	-3.05359425813677\\
0.941473578325101	-3.21674038215368\\
0.690636114193532	-3.3563922133266\\
0.444878017363584	-3.47346077781157\\
0.205881893719113	-3.56922337966759\\
-0.0251492633383242	-3.64517894479444\\
-0.24743415390647	-3.70292549841588\\
-0.460543675595968	-3.74406429249515\\
-0.664329558251052	-3.77013028278857\\
-0.858857509148538	-3.78254577658923\\
-1.04434837522275	-3.78259279981239\\
-1.22112868110194	-3.77139956280173\\
-1.38959048251925	-3.74993686590292\\
-1.55015965532614	-3.71902102360354\\
-1.70327135469562	-3.67932067358308\\
-1.84935127458584	-3.63136555260085\\
-1.98880139529637	-3.57555591462119\\
-2.12198904416514	-3.51217173044483\\
-2.24923825863018	-3.44138115671302\\
-2.37082260203892	-3.36324801856241\\
-2.48695872612035	-3.27773823906836\\
-2.59780009525661	-3.18472529276192\\
-2.70343038817182	-3.08399487987252\\
-2.80385617807281	-2.97524912958049\\
-2.89899857143374	-2.85811075903503\\
-2.98868357032124	-2.73212775263374\\
-3.07263102866869	-2.59677929319657\\
-3.15044221867331	-2.45148387999543\\
-3.22158623375341	-2.29561080920482\\
-3.28538575814983	-2.1284964612851\\
-3.34100316177998	-1.9494671113347\\
-3.38742846086942	-1.75787020032102\\
-3.42347143418947	-1.55311608678694\\
-3.44776108074811	-1.33473209991477\\
-3.45875656165306	-1.10243004134154\\
-3.45477459667135	-0.856186900710389\\
-3.4340386540989	-0.596336235620857\\
-3.39475469997954	-0.323664315135675\\
-3.33521618761754	-0.0395009185994836\\
-3.25393688541677	0.254209713866721\\
-3.14980398001913	0.554875578344262\\
-3.02223637374841	0.85927944122605\\
-2.87132601507902	1.16367107638014\\
-2.69793615489473	1.46393315962507\\
-2.50373234616036	1.75581037118509\\
-2.29113115679563	2.03517586494238\\
-2.0631666435215	2.29829796170304\\
-1.82329152448334	2.54206725605282\\
-1.57514323706355	2.76415193623645\\
-1.32231036646699	2.96306465758262\\
-1.06813112802755	3.13814248527204\\
-0.815544970130462	3.28945653769512\\
-0.567005194255521	3.41767637308333\\
-0.324448874733253	3.52391527433429\\
-0.0893127649146833	3.60957823091694\\
0.137419219282995	3.67622740769931\\
0.355150868005853	3.72547273073984\\
0.563602246342536	3.75888945002311\\
0.762739981900453	3.77796071070406\\
0.952714132411606	3.78404114737501\\
};
\addplot [color=mycolor1, forget plot]
  table[row sep=crcr]{%
0.924922571603334	3.86012979383972\\
1.10741517226221	3.85437906114762\\
1.28173030615979	3.83777101375592\\
1.44824772692772	3.81115839519889\\
1.6073672638724	3.77524698969403\\
1.75948843022384	3.73060297968216\\
1.90499467336231	3.67766142563773\\
2.04424113250968	3.61673489540518\\
2.17754490465196	3.5480216365273\\
2.30517696374031	3.47161295997233\\
2.42735501310555	3.38749970936446\\
2.54423666888277	3.29557784790108\\
2.6559124724642	3.19565332509177\\
2.76239831665528	3.08744650423698\\
2.86362695087446	2.97059655376326\\
2.95943831615204	2.84466634353788\\
3.04956856500963	2.70914855109376\\
3.13363776267843	2.56347387892083\\
3.21113646713818	2.40702251364427\\
3.28141167295689	2.23914021248628\\
3.34365300705965	2.05916065760012\\
3.39688060976846	1.8664359259029\\
3.43993683501123	1.66037699706607\\
3.4714847415984	1.44050603636683\\
3.49001724665632	1.20652156428516\\
3.49388160346017	0.958376347328325\\
3.48132425384463	0.696365710316007\\
3.45056065889319	0.42122087860642\\
3.39987291230763	0.13419806142575\\
3.32773433350307	-0.162850132918554\\
3.23295470419749	-0.467437565315586\\
3.11483292281914	-0.776459007045048\\
2.97329712335472	-1.08626991353548\\
2.8090081060877	-1.39283663973601\\
2.62340278296963	-1.69194920344484\\
2.41866172249558	-1.97947427848262\\
2.19759801761047	-2.25161475590067\\
1.96348025985427	-2.50513826980121\\
1.71981555962489	-2.73754254492922\\
1.47012509843908	-2.94713875607942\\
1.21774297344864	-3.13305084521394\\
0.96566055867123	-3.29514366021501\\
0.716426820712345	-3.43390207285486\\
0.472103823814196	-3.55028577351698\\
0.234268654575758	-3.64558147105733\\
0.00404905387826575	-3.7212681828598\\
-0.21782038296065	-3.77890458056491\\
-0.430932716495946	-3.82004159770682\\
-0.635142282220024	-3.84615943395138\\
-0.830502859775551	-3.85862573905873\\
-1.01721333322895	-3.85867078542238\\
-1.19557193450195	-3.84737538802329\\
-1.36593894731248	-3.82566779777615\\
-1.5287070495612	-3.79432647920966\\
-1.68427814337166	-3.75398639801371\\
-1.83304543344168	-3.70514708831453\\
-1.97537956488262	-3.64818130356677\\
-2.11161775084459	-3.58334347413291\\
-2.24205496316544	-3.51077751254264\\
-2.36693640038687	-3.43052374495339\\
-2.48645057430092	-3.34252492663097\\
-2.60072246497577	-3.24663144100706\\
-2.70980628690403	-3.14260590427168\\
-2.81367749143409	-3.03012751626585\\
-2.91222371213947	-2.90879662713068\\
-3.00523445264759	-2.77814013900335\\
-3.09238943676117	-2.63761854151964\\
-3.17324570879828	-2.48663559317992\\
-3.24722381276431	-2.32455190463207\\
-3.31359372064852	-2.15070393914319\\
-3.37146165151542	-1.96443018467537\\
-3.41975954570869	-1.76510640611786\\
-3.45723973290234	-1.5521918499892\\
-3.48247821581105	-1.3252878936088\\
-3.4938908649245	-1.08420970731213\\
-3.48976745481902	-0.829069814385136\\
-3.46832850481223	-0.560369821260565\\
-3.42780882966715	-0.279093054192476\\
-3.36656904634427	0.0132132762966951\\
-3.28323169253416	0.314381602606185\\
-3.17683227421202	0.621610598879662\\
-3.04696849954554	0.931510987921317\\
-2.89392516352604	1.24022063163617\\
-2.71875019385723	1.54358792361473\\
-2.52326142158784	1.83740829534198\\
-2.3099741650074	2.11768516038246\\
-2.08195461616956	2.38087853731976\\
-1.84261899636801	2.62410531352001\\
-1.59550869855381	2.8452649189445\\
-1.34407405027175	3.04307988961685\\
-1.09149384707052	3.21705724794761\\
-0.840547139155294	3.36738904976286\\
-0.593541831202277	3.49481633599458\\
-0.352294833901325	3.6004802503094\\
-0.118152527498448	3.68577925135342\\
0.107961758343429	3.75224470509597\\
0.325485658990213	3.80144075774801\\
0.534150980927119	3.83488943420737\\
0.733919086341772	3.85401872217122\\
0.924922571603333	3.86012979383972\\
};
\addplot [color=mycolor1, forget plot]
  table[row sep=crcr]{%
0.903698258268103	3.92215764008513\\
1.08735132504351	3.91636861736943\\
1.2630878670561	3.89962346304805\\
1.43126467080252	3.8727440060535\\
1.59225780420536	3.83640815616314\\
1.74644301896131	3.79115683907415\\
1.89418044405055	3.73740195984527\\
2.03580251378513	3.67543449883245\\
2.17160419604085	3.60543218458413\\
2.30183471682192	3.52746644617713\\
2.4266900999022	3.44150854364669\\
2.54630594847205	3.34743492958732\\
2.66074998944189	3.2450320258736\\
2.77001398473036	3.13400072270291\\
2.874004695425	3.01396103650767\\
2.97253367537735	2.88445751021317\\
3.06530578539447	2.74496611261371\\
3.15190647658124	2.5949035980394\\
3.231788114712	2.43364052099142\\
3.30425593346166	2.26051934980577\\
3.36845464039736	2.07487935748161\\
3.42335727770903	1.87609012852151\\
3.46775866388929	1.66359551304482\\
3.50027657951629	1.43696954395447\\
3.51936471072105	1.1959850265647\\
3.52334202990972	0.940694008201649\\
3.51044345244203	0.671516969034837\\
3.47889581887818	0.389334317916928\\
3.42702102801173	0.0955698956418503\\
3.35336413039243	-0.207747597431574\\
3.25683844568718	-0.517961781933052\\
3.13687308352426	-0.831820439695171\\
2.99354231586669	-1.14557322156846\\
2.82765341052591	-1.45513677753876\\
2.64077198130094	-1.7563157285933\\
2.43517250879938	-2.04505456852969\\
2.21371504654298	-2.31768641400818\\
1.97966364228776	-2.57114312091857\\
1.73647315857763	-2.80309866965773\\
1.48757553920183	-3.01203162287412\\
1.23619331302362	-3.19720832758851\\
0.985199174442602	-3.3586016427204\\
0.737029308745312	-3.49676734226096\\
0.493648113057673	-3.61270149016879\\
0.256555180764672	-3.70769850472153\\
0.0268223699175176	-3.78322365819072\\
-0.194851198997808	-3.8408075074894\\
-0.408076437012607	-3.88196457610608\\
-0.612711373450221	-3.90813504631275\\
-0.808802834085554	-3.92064622583874\\
-0.996535261072368	-3.92068977991008\\
-1.17618757834346	-3.90931075018121\\
-1.3480979305358	-3.88740485343349\\
-1.5126355067774	-3.85572120270058\\
-1.67017837235513	-3.81486825892942\\
-1.82109615533191	-3.76532141722852\\
-1.96573648222838	-3.7074311251094\\
-2.10441416474785	-3.64143081852443\\
-2.23740226869434	-3.56744425820473\\
-2.36492432410724	-3.48549207330571\\
-2.48714705137577	-3.39549749225424\\
-2.60417307874457	-3.2972913812052\\
-2.71603321453662	-3.19061683572763\\
-2.82267791884308	-3.07513369606113\\
-2.92396770392946	-2.95042349313354\\
-3.01966229315015	-2.81599549175349\\
-3.10940850153282	-2.67129468598716\\
-3.19272698840363	-2.51571282150219\\
-3.2689982985768	-2.3486037636964\\
-3.33744898135135	-2.16930477750443\\
-3.39713908163803	-1.9771654914995\\
-3.44695294995837	-1.77158641000755\\
-3.48559610618886	-1.55206869578192\\
-3.51160175340777	-1.318276409936\\
-3.5233513285425	-1.07011126691083\\
-3.51911393753557	-0.80779804351817\\
-3.49710926858435	-0.531975958161062\\
-3.45559712848183	-0.243787708484773\\
-3.39299365328631	0.0550461005874151\\
-3.30800931705898	0.362182708775455\\
-3.19979749246997	0.674659892579683\\
-3.06809574540893	0.988959834364201\\
-2.91333735231228	1.30114111617248\\
-2.73671011286166	1.60703409649741\\
-2.54014507190236	1.90248118283486\\
-2.32622907275073	2.18359171546618\\
-2.09804955338222	2.44697556903457\\
-1.85899337556969	2.68992265269857\\
-1.61252952337273	2.91050660041041\\
-1.36200597763492	3.10760646104273\\
-1.11048459324002	3.28085520275383\\
-0.860627278989088	3.43053430254579\\
-0.6146358272448	3.5574378624554\\
-0.374239182449423	3.66272820938912\\
-0.140717047647816	3.74779986315749\\
0.0850526535298473	3.81416242323357\\
0.302533810177599	3.86334709699072\\
0.511468552957436	3.89683720004627\\
0.711816190771982	3.91602021218667\\
0.903698258268102	3.92215764008513\\
};
\addplot [color=mycolor1, forget plot]
  table[row sep=crcr]{%
0.888367104169478	3.96922705110388\\
1.07291122390955	3.96340870005969\\
1.24972660934698	3.94655953164109\\
1.41915454831557	3.91947891877675\\
1.58155489367523	3.88282430170261\\
1.73728701994619	3.83711784890623\\
1.88669482939918	3.78275407669708\\
2.03009480218316	3.72000758549893\\
2.16776619914711	3.64904039405054\\
2.2999426476819	3.5699085994557\\
2.42680445556867	3.48256828096067\\
2.54847109980377	3.38688071757307\\
2.6649934270774	3.28261712152967\\
2.77634518475087	3.16946321640947\\
2.88241358406765	3.04702412345997\\
2.98298869260379	2.91483017324689\\
3.07775157595266	2.7723444398493\\
3.16626127875783	2.61897300468039\\
3.24794097593904	2.4540791921464\\
3.32206396272324	2.27700326274185\\
3.38774061280826	2.08708926419992\\
3.44390803562623	1.88372086279525\\
3.4893249028563	1.66636790177474\\
3.52257474521719	1.43464501549463\\
3.54208182626729	1.18838268134413\\
3.546144261553	0.927709422610597\\
3.53298902860703	0.653141346565991\\
3.50085246228754	0.365671846300655\\
3.44808729507601	0.0668504672978017\\
3.37329300996861	-0.241163543734945\\
3.2754604206344	-0.55558933153358\\
3.15411492963611	-0.873069141527941\\
3.0094376299469	-1.18977885737679\\
2.8423415906143	-1.50160330875996\\
2.65448421822659	-1.80436222226359\\
2.4482059418933	-2.0940602942511\\
2.22639883879704	-2.36712731776046\\
1.99232247360235	-2.6206146492307\\
1.74939393345407	-2.85232289414203\\
1.50098193132137	-3.06084981396987\\
1.25023064039342	-3.24556257819414\\
0.999929787427645	-3.40651029606547\\
0.752436848006519	-3.54429883294141\\
0.509647993557539	-3.65995016654044\\
0.273008470238771	-3.7547646023417\\
0.0435506210950787	-3.83019828138013\\
-0.178051930487216	-3.88776249702837\\
-0.391423447866059	-3.92894655792853\\
-0.596425502401252	-3.95516271243238\\
-0.793101019687933	-3.96770989035063\\
-0.981625239981682	-3.96775237649092\\
-1.1622643685979	-3.95630961403404\\
-1.33534170842412	-3.93425381000624\\
-1.50121050516149	-3.90231264186172\\
-1.66023247507985	-3.86107499730855\\
-1.8127609182453	-3.81099824347687\\
-1.95912736560332	-3.75241598788524\\
-2.09963080922159	-3.68554566155298\\
-2.23452868539888	-3.61049553716092\\
-2.36402889969611	-3.52727101104556\\
-2.48828229154454	-3.43578014660196\\
-2.60737503158122	-3.33583861669606\\
-2.72132052997725	-3.22717431025802\\
-2.83005051531157	-3.10943199750812\\
-2.93340503094563	-2.98217859123723\\
-3.03112120282013	-2.84490970765218\\
-3.12282077645936	-2.69705842527188\\
-3.20799662322782	-2.53800736405932\\
-3.28599870174384	-2.36710544939466\\
-3.35602035661871	-2.18369096083397\\
-3.41708636655713	-1.98712264415913\\
-3.46804482675066	-1.7768207039597\\
-3.50756574495878	-1.55231926827857\\
-3.53415006926062	-1.31333125914399\\
-3.54615358389633	-1.05982531792969\\
-3.54183042910619	-0.792112345344705\\
-3.51940052341957	-0.510936253052693\\
-3.4771434253147	-0.217559861614279\\
-3.41351776998375	0.0861669099851317\\
-3.32730027834201	0.397772267523406\\
-3.21773200376542	0.714177318684968\\
-3.08465333840251	1.03177309149074\\
-2.92860545811216	1.34656437398015\\
-2.75087657637592	1.65437294940559\\
-2.5534779083619	1.95107955455206\\
-2.3390459754472	2.23287342098895\\
-2.11068193469523	2.4964744238251\\
-1.87175078567116	2.7392974959319\\
-1.62566981795802	2.95954082059142\\
-1.37571486537172	3.15619453949885\\
-1.12486587587972	3.32898062712203\\
-0.87570296296231	3.4782436754916\\
-0.630353853834017	3.60481537243616\\
-0.390485948581672	3.70987334533868\\
-0.157332027587421	3.79480985536257\\
0.0682623078150467	3.86111972866702\\
0.285780096344525	3.91031146769596\\
0.494971613979965	3.94384147128421\\
0.695795704407208	3.96306883484295\\
0.888367104169477	3.96922705110388\\
};
\addplot [color=mycolor1, forget plot]
  table[row sep=crcr]{%
0.878355311988031	4.00104451440901\\
1.06350692625045	3.99520620111796\\
1.24105208545559	3.97828669634494\\
1.41132245401614	3.95107065481952\\
1.57466773303343	3.91420199851153\\
1.73143696912751	3.86819041177193\\
1.88196376005733	3.81341874894101\\
2.02655438296571	3.750150547048\\
2.16547798030495	3.67853714869157\\
2.29895805486653	3.59862418003358\\
2.42716463522509	3.51035731500992\\
2.55020657118865	3.41358740789436\\
2.668123506243	3.30807520909464\\
2.78087715546415	3.19349600831664\\
2.88834160146377	3.06944468768286\\
2.99029241959652	2.93544182533574\\
3.08639457260405	2.79094167459111\\
3.17618919386929	2.63534305689275\\
3.25907963118477	2.46800444241896\\
3.33431747555457	2.28826473051858\\
3.4009897767428	2.09547144282054\\
3.45800926415817	1.88901813486057\\
3.50411013946209	1.6683927090189\\
3.53785283151403	1.43323782151323\\
3.55764187603864	1.18342353411735\\
3.5617615690935	0.919130581462076\\
3.54843389392288	0.64093999291339\\
3.51590199238786	0.349921399335943\\
3.46253971391698	0.0477085835925056\\
3.38698331236318	-0.263452424287396\\
3.28827546184747	-0.580698823062855\\
3.16600547552711	-0.900604275852195\\
3.02042477853998	-1.21929774783176\\
2.85251553485022	-1.53264516950835\\
2.66399457594665	-1.83647819729149\\
2.45724457303916	-2.12684262395052\\
2.23517770285111	-2.4002325211065\\
2.00105011016354	-2.65377765196434\\
1.7582542402097	-2.88536099227669\\
1.51011809336617	-3.09365738536341\\
1.25973568625162	-3.27809894883646\\
1.00984379818778	-3.43878382182336\\
0.762749729660402	-3.57635010468598\\
0.520306129482393	-3.69183655722441\\
0.283923478606458	-3.78654748590043\\
0.0546086990229475	-3.86193343438586\\
-0.166981171541117	-3.91949359001469\\
-0.380479311033542	-3.96070128664519\\
-0.585749985503425	-3.986950972031\\
-0.782834069573656	-3.99952339235443\\
-0.971901342639129	-3.99956518582877\\
-1.15321025667837	-3.98807919312425\\
-1.32707494440725	-3.96592226831757\\
-1.49383870905093	-3.93380798674879\\
-1.65385299334717	-3.89231225949612\\
-1.80746076448412	-3.84188040852438\\
-1.95498329637778	-3.78283470631474\\
-2.09670942756051	-3.71538173904544\\
-2.23288648811713	-3.63961922624244\\
-2.36371220332813	-3.55554214043776\\
-2.48932698599055	-3.46304813674846\\
-2.60980612184367	-3.36194244219558\\
-2.7251514361164	-3.25194248386318\\
-2.83528211072562	-3.1326826674037\\
-2.9400244113109	-3.00371986447725\\
-3.03910019488404	-2.86454033832207\\
-3.13211422011443	-2.71456903567785\\
-3.21854049496589	-2.55318239915177\\
-3.29770819576308	-2.37972609464072\\
-3.3687881039304	-2.19353927459823\\
-3.43078105286486	-1.99398715573653\\
-3.4825105630904	-1.78050369093681\\
-3.52262264058648	-1.55264583001233\\
-3.54959653315705	-1.31016012225037\\
-3.56177090540339	-1.05306102369323\\
-3.55739011243071	-0.781718072012295\\
-3.53467462163135	-0.496946044577514\\
-3.49191769492277	-0.200088546569051\\
-3.42760684601089	0.106918179217524\\
-3.34056332736365	0.421516953800008\\
-3.23008662410292	0.74055253109619\\
-3.09608510202537	1.06035723112041\\
-2.93917069281702	1.37690232801292\\
-2.76069690826706	1.68600603833541\\
-2.56272661017962	1.98357610732913\\
-2.34792790075655	2.26585543441595\\
-2.11941020973985	2.52963648855715\\
-1.88052401217249	2.77241584827913\\
-1.63465314874106	2.9924724548048\\
-1.38502715711598	3.18886813413911\\
-1.13457364935907	3.3613821295424\\
-0.885820579981077	3.51039961632549\\
-0.640848456119384	3.63677651896027\\
-0.40128537454523	3.7417004733795\\
-0.16833402832988	3.82656253933673\\
0.0571808702239901	3.89284832757713\\
0.274754939025249	3.94205200602846\\
0.48414407094491	3.975612869128\\
0.685307253194799	3.99487187487964\\
0.87835531198803	4.00104451440901\\
};
\addplot [color=mycolor1, forget plot]
  table[row sep=crcr]{%
0.87317260289467	4.01786189580409\\
1.0586470073743	4.01201298588825\\
1.23657813808843	3.99505628803038\\
1.40729283295192	3.96776882189008\\
1.57113567513125	3.93078746483326\\
1.72845048096189	3.88461536516827\\
1.87956560720388	3.8296292438126\\
2.02478211899208	3.7660867946643\\
2.16436396668982	3.69413370111489\\
2.29852943361614	3.61381002254227\\
2.42744322415422	3.52505588894224\\
2.55120865822076	3.42771659234049\\
2.669859524372	3.32154729687035\\
2.78335122505298	3.20621771999007\\
2.8915509323161	3.08131727771277\\
2.99422657288481	2.94636134697276\\
3.09103459362408	2.80079948506522\\
3.18150664226152	2.64402666092272\\
3.26503555735054	2.47539878863308\\
3.34086142196184	2.29425408892726\\
3.40805892143134	2.09994199696792\\
3.46552787019314	1.89186141208578\\
3.51198952458983	1.66950993600597\\
3.54599211843127	1.43254521754385\\
3.565929810729	1.18085842894172\\
3.57007968075661	0.914658061332085\\
3.55666118804975	0.634559542850044\\
3.52392119282383	0.34167275065185\\
3.47024478877763	0.0376757525251807\\
3.39428765284131	-0.275140018359175\\
3.29511970073396	-0.593869140628072\\
3.17236365200774	-0.915049888989435\\
3.02630751934734	-1.23478733986051\\
2.85796923624723	-1.54893823008124\\
2.6690962381088	-1.85334103266034\\
2.46209281396263	-2.14406336221362\\
2.23988130272687	-2.41763291314587\\
2.00571593672854	-2.67122014574578\\
1.76297642204718	-2.90275057507449\\
1.51496986821874	-3.11093871465674\\
1.26476464219	-3.2952500330042\\
1.01507049410291	-3.45580782062897\\
0.76816912641704	-3.59326672357929\\
0.52589097390353	-3.70867415021093\\
0.289628743614559	-3.80333652933828\\
0.0603763166980937	-3.87870162167307\\
-0.161217833407163	-3.93626249501914\\
-0.37479155748923	-3.97748436629825\\
-0.580210632191441	-4.00375260580504\\
-0.777515021874765	-4.01633865542448\\
-0.966871847477634	-4.01638009184131\\
-1.14853571801903	-4.00487119763804\\
-1.32281618140011	-3.98266087997323\\
-1.49005154074905	-3.95045538220979\\
-1.65058804753554	-3.90882383733263\\
-1.80476342481098	-3.85820524655758\\
-1.95289371835076	-3.79891590799786\\
-2.09526256837972	-3.73115666920935\\
-2.23211210715962	-3.65501964694317\\
-2.36363479941683	-3.57049426544381\\
-2.48996564479346	-3.47747262983988\\
-2.61117425248971	-3.37575439108605\\
-2.72725638116091	-3.2650513891372\\
-2.83812461886204	-3.14499249511081\\
-2.94359796877304	-3.01512922244045\\
-3.04339022056795	-2.87494284998519\\
-3.13709714252516	-2.72385400113866\\
-3.22418274777113	-2.56123584983809\\
-3.30396519460744	-2.38643236363178\\
-3.37560330143226	-2.19878321489474\\
-3.43808521149454	-1.99765713755406\\
-3.49022143481789	-1.78249548733109\\
-3.53064529193214	-1.552867446259\\
-3.55782459317131	-1.30853752562824\\
-3.57008902380522	-1.04954457618762\\
-3.56567787070637	-0.776289257127412\\
-3.54281201375396	-0.489623826818346\\
-3.4997920658988	-0.19093445153481\\
-3.43512084778765	0.117797349155185\\
-3.34764306570541	0.43396984883628\\
-3.23668882152542	0.754388232410177\\
-3.10220192627757	1.07535469351025\\
-2.94483103100244	1.39282366721893\\
-2.7659633660714	1.70261215790338\\
-2.56768831396552	2.00064250305003\\
-2.35269006632684	2.28318584141813\\
-2.1240821328952	2.54707244995374\\
-1.88520740306866	2.78984116869868\\
-1.63943250522551	3.00981256628613\\
-1.38996326771129	3.20608531441715\\
-1.13970056624627	3.37846804420796\\
-0.891145740461128	3.52736675030405\\
-0.646355204967131	3.65364981369306\\
-0.406936982742872	3.75851006021307\\
-0.174078362593084	3.8433380143813\\
0.0514066882572753	3.90961465117128\\
0.26902032040407	3.95882687108028\\
0.478521424067281	3.99240526127457\\
0.679869196887858	4.0116815148737\\
0.873172602894669	4.01786189580409\\
};
\addplot [color=mycolor1, forget plot]
  table[row sep=crcr]{%
0.872400712460433	4.02038721270919\\
1.05792368884529	4.01453670886125\\
1.23591279430132	3.99757442485156\\
1.40669415698166	3.97027624228707\\
1.57061160586457	3.93327798661598\\
1.72800818514876	3.88708182838907\\
1.87921147654295	3.83206356936454\\
2.02452177307966	3.76848002634673\\
2.16420225459164	3.69647603203228\\
2.29847042837004	3.61609080803541\\
2.42749020571817	3.52726364931374\\
2.55136408132319	3.42983900962203\\
2.67012496850639	3.32357121094303\\
2.78372732461065	3.20812913063445\\
2.89203728571882	3.08310136072375\\
2.99482163073449	2.94800249437786\\
3.09173552753474	2.80228138172703\\
3.18230919841028	2.64533241228778\\
3.26593390212375	2.47651111686476\\
3.34184799159715	2.29515561666519\\
3.40912429337584	2.10061563872189\\
3.46666068087803	1.89229089164514\\
3.51317646588389	1.66968044266358\\
3.54721805189474	1.43244420253153\\
3.56717804270875	1.1804765244426\\
3.57133243904973	0.913990076760244\\
3.55790032826914	0.633605457730588\\
3.52512913593394	0.340438584799407\\
3.47140564893363	0.0361741613396976\\
3.39538845990468	-0.276889589273369\\
3.29615156509209	-0.595840886473753\\
3.17332267837953	-0.917212741306342\\
3.02719527215502	-1.23710669588607\\
2.8587926074176	-1.55137815237982\\
2.66986666846752	-1.85586663643602\\
2.46282494966837	-2.14664304453676\\
2.24059128960257	-2.42024009883123\\
2.0064196277991	-2.67383432706057\\
1.76368777428001	-2.9053575740163\\
1.51569973809642	-3.1135302481095\\
1.26552008439842	-3.29782276943657\\
1.01585455977801	-3.45836216129842\\
0.768981076019589	-3.59580552596122\\
0.526726777741583	-3.71120155773919\\
0.290481737747762	-3.8058570046132\\
0.061237901848247	-3.88121921953595\\
-0.160357530756129	-3.93878036750128\\
-0.373943110125655	-3.98000446543796\\
-0.579384844552284	-4.00627553969976\\
-0.77672257159037	-4.01886365278914\\
-0.96612302601515	-4.0189050361166\\
-1.14784025442594	-4.00739269981099\\
-1.32218313135468	-3.98517436751489\\
-1.48948922383381	-3.95295518874916\\
-1.65010401763265	-3.91130328345454\\
-1.80436446095217	-3.8606567064451\\
-1.95258582575122	-3.80133085975595\\
-2.09505098149806	-3.73352572889659\\
-2.23200128836394	-3.65733258789161\\
-2.36362842820677	-3.57274002561814\\
-2.4900665935909	-3.47963931098985\\
-2.61138454585985	-3.37782925445319\\
-2.72757713607705	-3.26702085362826\\
-2.83855596441736	-3.14684214525272\\
-2.94413894473072	-3.01684383515928\\
-3.04403865551544	-2.87650645132992\\
-3.13784951437789	-2.72524996645255\\
-3.22503403219608	-2.56244706340121\\
-3.30490871084637	-2.38744145611349\\
-3.37663057017367	-2.19957289845858\\
-3.43918584588259	-1.99821065819708\\
-3.49138309306751	-1.78279721065399\\
-3.53185372744502	-1.55290358461373\\
-3.55906384378357	-1.30829699957148\\
-3.57134178307268	-1.04901997985141\\
-3.56692607720377	-0.775477865301702\\
-3.54403767512824	-0.488528543722764\\
-3.50097829987952	-0.189564564061972\\
-3.4362530729949	0.1194257760861\\
-3.34871021754107	0.435834095364993\\
-3.23768441700587	0.756459683113159\\
-3.10312477660631	1.07760026181148\\
-2.94568542159817	1.39520778860554\\
-2.76675859942821	1.70509912442116\\
-2.56843763007815	2.00319881562071\\
-2.35340908664954	2.2857822333132\\
-2.12478707505283	2.5496852985695\\
-1.88591335131713	2.79245314539585\\
-1.64015198650535	3.01241253190674\\
-1.39070529283191	3.20866760230002\\
-1.1404701847406	3.38103134977479\\
-0.891944060345442	3.5299128589176\\
-0.647179766393561	3.65618236087359\\
-0.407782351929108	3.76103345507587\\
-0.174936821431493	3.84585659144264\\
0.0505444541971551	3.91213206239664\\
0.268164599667046	3.96134569498887\\
0.477682956765757	3.99492678814681\\
0.679058762338607	4.01420566237649\\
0.872400712460432	4.02038721270919\\
};
\addplot [color=mycolor1, forget plot]
  table[row sep=crcr]{%
0.875685557219478	4.00967763805254\\
1.06100272774934	4.00383388897912\\
1.23874599994084	3.98689529297496\\
1.40924453870376	3.95964257308244\\
1.57284539700775	3.92271602638168\\
1.72989491724704	3.87662197661192\\
1.88072398936467	3.82174012688877\\
2.02563619962134	3.75833101425244\\
2.16489801188383	3.68654307749908\\
2.29873023832086	3.60641908780634\\
2.42730016506097	3.51790187687297\\
2.55071379570723	3.42083944803591\\
2.66900776242509	3.3149896888231\\
2.78214053566533	3.20002503333925\\
2.88998264804146	3.07553756233089\\
2.99230574750499	2.94104518790835\\
3.08877042556289	2.7959997556108\\
3.17891294772867	2.6397981105391\\
3.26213126941631	2.47179740992005\\
3.33767107714397	2.29133620127919\\
3.40461307655313	2.09776298196715\\
3.46186336966171	1.89047404078754\\
3.5081495136968	1.66896224611083\\
3.54202567591895	1.43287793509279\\
3.56189106090967	1.18210199051418\\
3.56602625257259	0.916829382170032\\
3.55265192876722	0.637658790924856\\
3.52001312997204	0.345680512379708\\
3.46648947089461	0.0425510835278295\\
3.39072717737164	-0.269460118123823\\
3.29178292249394	-0.587468353417559\\
3.16926320037092	-0.908029045899134\\
3.0234382680544	-1.22725882065251\\
2.85530871517088	-1.54101882183472\\
2.66660715145907	-1.84514416267025\\
2.45972740504391	-2.13569183046976\\
2.23758690778079	-2.4091731874464\\
2.0034408310195	-2.66273895188084\\
1.76067505595249	-2.89429402127679\\
1.51260680563624	-3.10253369378862\\
1.2623168570532	-3.28690728975612\\
1.01252803382881	-3.4475259237388\\
0.765534418367372	-3.58503623101065\\
0.523177189657266	-3.70048142961837\\
0.286857653948686	-3.79516691484637\\
0.0575760054487964	-3.87054178698562\\
-0.164015131686133	-3.92810206847397\\
-0.377551337644567	-3.969316898912\\
-0.582897646867223	-3.99557603830755\\
-0.780094452261493	-4.00815543066842\\
-0.969310149493292	-4.00819704008281\\
-1.15080119641976	-3.99669929558502\\
-1.32487935016732	-3.97451495749124\\
-1.4918853257357	-3.94235382694221\\
-1.65216788081969	-3.90078832860645\\
-1.806067272198	-3.85026053577205\\
-1.95390207358847	-3.79108965251219\\
-2.09595844071301	-3.72347931960282\\
-2.23248102314662	-3.64752438246993\\
-2.36366483543669	-3.56321696870832\\
-2.48964750319551	-3.47045188848123\\
-2.6105013915586	-3.36903151101306\\
-2.72622520662556	-3.25867040014854\\
-2.83673474211532	-3.13900012521297\\
-2.94185253377562	-3.00957481160171\\
-3.04129629697352	-2.86987816734751\\
-3.13466617617788	-2.71933292200084\\
-3.22143105060679	-2.55731384054737\\
-3.3009144433641	-2.38316571498379\\
-3.37228099786505	-2.19622795969058\\
-3.43452503594903	-1.99586758867153\\
-3.48646340108576	-1.78152234342872\\
-3.52673558716778	-1.55275543868523\\
-3.55381496787173	-1.30932262844642\\
-3.56603559239636	-1.05125087648801\\
-3.56163920527467	-0.778925685763544\\
-3.53884647527735	-0.493181072480737\\
-3.49595442884873	-0.195382502407381\\
-3.43145843393958	0.112510508935222\\
-3.34419179023719	0.427917864513764\\
-3.23346972277805	0.747663938626615\\
-3.09921883410767	1.06806550948267\\
-2.94206996319232	1.38508513324789\\
-2.76339399824633	1.69454034792435\\
-2.56526748071004	1.99234636366853\\
-2.35036682262639	2.27476058800659\\
-2.12180357232166	2.53859494654633\\
-1.8829243026072	2.78136778825985\\
-1.63710397863557	3.00137951287317\\
-1.3875599028346	3.19771094877064\\
-1.13720588438874	3.3701564993994\\
-0.888556135040165	3.5191120805132\\
-0.643678721429679	3.64544004071904\\
-0.404191376720517	3.75033068852255\\
-0.171288844332672	3.8351748005113\\
0.0542097043526339	3.90145558792076\\
0.271803250693252	3.95066346198877\\
0.481249224027468	3.98423321927077\\
0.682506708461911	4.00350103458518\\
0.875685557219478	4.00967763805254\\
};
\addplot [color=mycolor1, forget plot]
  table[row sep=crcr]{%
0.882731909221739	3.98702827620123\\
1.06761539375866	3.98119877047761\\
1.24483900571829	3.96431025452675\\
1.41473856780442	3.93715382213604\\
1.57766817606962	3.90037931935314\\
1.73398135577779	3.85450191085997\\
1.88401618029681	3.79990957904154\\
2.02808336540157	3.73687073334937\\
2.16645646266481	3.66554142460836\\
2.29936339425423	3.58597190173672\\
2.42697868338947	3.4981124361362\\
2.54941583457895	3.40181849054501\\
2.66671940613247	3.29685544149948\\
2.77885639919958	3.18290319271153\\
2.88570667111642	3.05956115349769\\
2.98705217798458	2.92635421240694\\
3.08256497769965	2.78274051879899\\
3.17179409970626	2.62812209688069\\
3.25415163517833	2.46185955216491\\
3.32889874738757	2.28329237097139\\
3.39513277197684	2.09176652072532\\
3.45177718708876	1.88667116454702\\
3.49757697745115	1.66748620191229\\
3.53110274393243	1.4338418888766\\
3.55076769715874	1.18559079145782\\
3.55486219390545	0.922890594589315\\
3.54161038162041	0.646293701855839\\
3.50925236618872	0.356836174514016\\
3.45615266872313	0.0561147602165053\\
3.38093134812654	-0.253662597463924\\
3.28260828478366	-0.56966898255585\\
3.16074475307444	-0.888508003772609\\
3.01556137678958	-1.20632901239781\\
2.84801011498882	-1.51900601649008\\
2.6597818717229	-1.82236525663449\\
2.45324093105079	-2.11243438844237\\
2.23129076023667	-2.3856792747371\\
1.99718904460682	-2.63919536900434\\
1.75433899699584	-2.87082967070344\\
1.50608635773878	-3.07922340030803\\
1.25554697065881	-3.26378036902835\\
1.00548064154205	-3.42457735549414\\
0.758216486533735	-3.5622384135832\\
0.515626080765996	-3.67779497954975\\
0.279135032617231	-3.77254959507967\\
0.0497613395445611	-3.84795521492908\\
-0.171830716992709	-3.90551627241799\\
-0.385270399547987	-3.94671303691297\\
-0.590420759709649	-3.97294769465193\\
-0.787323520997589	-3.98550890671645\\
-0.976150797072024	-3.98555100251722\\
-1.15716437571668	-3.97408406911436\\
-1.33068234754739	-3.95197167189453\\
-1.49705231623919	-3.91993356087358\\
-1.65663017597268	-3.87855133874811\\
-1.80976337831088	-3.82827561962295\\
-1.95677765570536	-3.7694336642546\\
-2.09796626742514	-3.70223683836107\\
-2.23358095111042	-3.62678751815175\\
-2.36382387957633	-3.54308528011456\\
-2.48884002864075	-3.45103237950283\\
-2.60870945550432	-3.35043866190228\\
-2.72343907143695	-3.2410261807742\\
-2.83295357390829	-3.12243392489568\\
-2.93708529193436	-2.99422320488968\\
-3.03556280793928	-2.855884416668\\
-3.12799836739509	-2.70684609687665\\
-3.21387429557463	-2.54648741039426\\
-3.29252893414871	-2.37415545137323\\
-3.36314301552118	-2.18918896964803\\
-3.42472793185169	-1.99095030145124\\
-3.47611803587304	-1.77886730131582\\
-3.51596990654864	-1.5524868133086\\
-3.54277234109924	-1.31154051518176\\
-3.55487152426177	-1.05602262530621\\
-3.55051608963588	-0.78627681059057\\
-3.52792622494695	-0.503086622526053\\
-3.48538912189345	-0.207760120922984\\
-3.42137955892997	0.0978043714961962\\
-3.33469919677839	0.411087109991549\\
-3.22462186239684	0.728966240501443\\
-3.09102612672127	1.04779966508264\\
-2.93449296225586	1.363573149435\\
-2.75634736244044	1.67210627095784\\
-2.55862967980305	1.96929477663697\\
-2.34399429304059	2.25135796132634\\
-2.11554707999374	2.51505648762724\\
-1.87664488694667	2.7578512438258\\
-1.63068614503158	2.97798592410831\\
-1.38092054995231	3.17449110084244\\
-1.13029848048179	3.34712106410228\\
-0.881370570904034	3.49624330839168\\
-0.636237859826358	3.62270319198721\\
-0.396545539721535	3.72768397322806\\
-0.163509402534211	3.81257721008895\\
0.0620367292732862	3.87887249966095\\
0.279582967943426	3.9280702282238\\
0.488882722600083	3.96161712127999\\
0.689894874815934	3.98086202666136\\
0.882731909221738	3.98702827620123\\
};
\addplot [color=mycolor1, forget plot]
  table[row sep=crcr]{%
0.893299774619925	3.95386829659369\\
1.07755214373071	3.94805954177936\\
1.25401542773574	3.93124431583162\\
1.42303580063363	3.90422922774552\\
1.5849782444428	3.86767833420769\\
1.74020732941008	3.82211988601337\\
1.88907212038673	3.76795405568808\\
2.03189418792735	3.70546078719016\\
2.16895781927488	3.63480723708031\\
2.3005016488094	3.55605452683267\\
2.42671104466854	3.46916371784449\\
2.54771069215461	3.37400107359669\\
2.66355690548067	3.27034280494737\\
2.77422928199946	3.1578796201603\\
2.8796213954568	3.03622153427032\\
2.97953031852448	2.90490354378534\\
3.07364488506467	2.76339295064852\\
3.16153276848965	2.61109932753198\\
3.24262668758664	2.44738835125698\\
3.31621038180594	2.2716009766731\\
3.38140545071861	2.08307964485391\\
3.43716074643413	1.88120335412893\\
3.48224674236073	1.66543336983428\\
3.5152581350474	1.43537096426486\\
3.53462875686095	1.19082767771975\\
3.53866347378391	0.931906976742413\\
3.52559177989175	0.659093708520822\\
3.49364683462498	0.373344424108135\\
3.44117125418124	0.0761677941582254\\
3.36674676341693	-0.230319245427688\\
3.26933899140187	-0.543376023084219\\
3.14844215251458	-0.859678674650531\\
3.0042028551991	-1.17542656741677\\
2.83750012747688	-1.48651439077919\\
2.64996194821813	-1.78875659376243\\
2.44390769691871	-2.07813814119094\\
2.22221931485283	-2.35105748741888\\
1.98815791197158	-2.60452746429405\\
1.7451527266885	-2.83630800111545\\
1.49659270031397	-3.0449586627256\\
1.24564701169292	-3.22981436239356\\
0.99513183136759	-3.39089983342567\\
0.747429709962517	-3.52880492315626\\
0.504458560161213	-3.64454330167256\\
0.267680966437529	-3.73941334949917\\
0.0381419077695911	-3.81487407197817\\
-0.183476801654758	-3.87244286466293\\
-0.39679508897941	-3.91361704952985\\
-0.601673307903635	-3.93981777331237\\
-0.798155537809111	-3.95235302598206\\
-0.986419868182171	-3.95239585477658\\
-1.16673648453138	-3.94097391618187\\
-1.33943335949599	-3.91896698158723\\
-1.50486877396498	-3.88710964629557\\
-1.66340962350881	-3.8459971348178\\
-1.81541439567	-3.79609266937436\\
-1.96121974953256	-3.7377353433588\\
-2.10112973204507	-3.67114781585288\\
-2.23540678869332	-3.5964434302804\\
-2.36426384812058	-3.51363257887012\\
-2.48785687111275	-3.42262830465481\\
-2.60627735145271	-3.32325127287698\\
-2.71954434210345	-3.21523437045106\\
-2.82759566144855	-3.09822731991584\\
-2.93027802071645	-2.97180183529007\\
-3.02733591852908	-2.83545801111575\\
-3.11839928890021	-2.68863282891876\\
-3.20297008630592	-2.53071188782891\\
-3.28040827084369	-2.36104570920414\\
-3.34991804508014	-2.17897220451209\\
-3.41053571604363	-1.98384708380746\\
-3.46112122223544	-1.77508403838827\\
-3.50035615831711	-1.55220633353923\\
-3.52675197680331	-1.31491082965675\\
-3.53867278879889	-1.06314421601491\\
-3.53437755233595	-0.797189207100136\\
-3.51208603276816	-0.517755532335079\\
-3.47007127328522	-0.226066895661054\\
-3.40677801058063	0.0760687666748171\\
-3.32096139680575	0.386221402916465\\
-3.21183403917211	0.70134969368578\\
-3.07920308030691	1.01787383287239\\
-2.92357492114615	1.3318155924254\\
-2.74620552404545	1.63899910462536\\
-2.5490804545193	1.93529237337846\\
-2.33482042707061	2.21685862135462\\
-2.10652232685255	2.48038217619461\\
-1.86755824066811	2.72323774414946\\
-1.62136202998576	2.94358355286687\\
-1.37123257662162	3.14037417707846\\
-1.1201759479247	3.31330312954958\\
-0.870798319037104	3.46269482591756\\
-0.625251013653767	3.58936892524313\\
-0.385221052769517	3.69449812986298\\
-0.151956201110977	3.7794753728873\\
0.07368758457424	3.8458001464851\\
0.291187166213739	3.89498815865432\\
0.500290138046111	3.92850437284837\\
0.700955402138476	3.94771693646535\\
0.893299774619924	3.95386829659369\\
};
\addplot [color=mycolor1, forget plot]
  table[row sep=crcr]{%
0.907201964607069	3.91167245534824\\
1.09065777465157	3.90588993340582\\
1.266154057183	3.88916794883912\\
1.43405126957733	3.86233344813099\\
1.594729297961	3.82606897971694\\
1.74856773050552	3.78091969528162\\
1.8959304784888	3.72730139516052\\
2.03715367621373	3.66550870945552\\
2.17253591515043	3.59572285092159\\
2.3023300003203	3.51801863638448\\
2.42673554145718	3.43237067112468\\
2.54589180117484	3.33865874564094\\
2.65987031709079	3.23667262488993\\
2.76866689895426	3.12611653258124\\
2.87219268316559	3.00661376124556\\
2.97026401679408	2.87771198423091\\
3.06259105600252	2.73889001751568\\
3.1487651184107	2.58956698232377\\
3.22824504843374	2.42911505251825\\
3.30034316574832	2.2568772212265\\
3.36421179758033	2.07219175911517\\
3.41883196817198	1.87442520570766\\
3.4630065393329	1.66301574203399\\
3.49536093365565	1.43752849981995\\
3.51435543146834	1.1977235870137\\
3.51831372075252	0.943636145187155\\
3.50547257797936	0.675665425934568\\
3.47405682690306	0.394666641041049\\
3.42238156859522	0.102035452677137\\
3.34897972599436	-0.200228877409782\\
3.25274723000599	-0.509498242008303\\
3.13309144873207	-0.822544621750892\\
2.99006239042379	-1.13563484408611\\
2.82444314390421	-1.44469331517534\\
2.63777820796342	-1.74552174078721\\
2.43232677218777	-2.03405132380219\\
2.2109413525581	-2.30659339061513\\
1.97688688500037	-2.56005258739435\\
1.73362686517079	-2.79207387400802\\
1.48460783432562	-3.00110837186224\\
1.23307049154952	-3.18639914912145\\
0.981906815594165	-3.34790143399911\\
0.733571301996436	-3.48615942365957\\
0.490044211354484	-3.60216321962818\\
0.252837747946268	-3.69720593341672\\
0.0230329004807929	-3.77275501842284\\
-0.198665367807622	-3.8303455570256\\
-0.411864969177006	-3.87149796299146\\
-0.616423224962243	-3.89765891705149\\
-0.812387983253727	-3.91016230237642\\
-0.999945933157816	-3.91020610170607\\
-1.17937904953582	-3.89884123580373\\
-1.3510290044646	-3.8769687935728\\
-1.51526875180609	-3.8453427594907\\
-1.67248019647208	-3.80457601721365\\
-1.8230367819316	-3.75514801179825\\
-1.96728987702298	-3.69741295278498\\
-2.10555795271765	-3.63160783367408\\
-2.23811767079117	-3.55785984339636\\
-2.3651961362982	-3.47619297194968\\
-2.4869636832195	-3.38653378625872\\
-2.60352666456006	-3.28871649299009\\
-2.71491980680991	-3.18248752976885\\
-2.821097770212	-3.06751004995626\\
-2.92192564020713	-2.94336880162888\\
-3.01716817460706	-2.8095760590619\\
-3.10647776214747	-2.66557945213046\\
-3.1893812319927	-2.51077275782587\\
-3.2652659155717	-2.34451096228418\\
-3.33336572959656	-2.16613115111861\\
-3.39274854857961	-1.97498099842778\\
-3.44230678277244	-1.77045672708944\\
-3.48075386369101	-1.5520522900466\\
-3.50663020521454	-1.31942101315796\\
-3.51832301358775	-1.07244984603562\\
-3.51410481113174	-0.811344487469956\\
-3.49219533339304	-0.536720862648344\\
-3.45085007673776	-0.249694813293135\\
-3.38847575070051	0.0480421262739279\\
-3.30376801415931	0.354176392328224\\
-3.19586050306031	0.665772375441081\\
-3.06446749681585	0.97933329473128\\
-2.90999768909297	1.29093043938497\\
-2.73361585877175	1.59639665376012\\
-2.53723454146527	1.89156610234076\\
-2.32342899703125	2.17253026180334\\
-2.09528334141111	2.43587407085777\\
-1.85618936035988	2.67885886947717\\
-1.60962792701334	2.89952967674314\\
-1.35896372133843	3.096739924233\\
-1.10727761787898	3.27010201066058\\
-0.857250548598626	3.41988281855851\\
-0.611101534345176	3.54686777559801\\
-0.37057381810315	3.65221571353813\\
-0.13695797183735	3.73732173818946\\
0.0888606376441422	3.8036989372349\\
0.306341543883091	3.85288383501983\\
0.51522509565988	3.88636602159796\\
0.715470826708424	3.90553956693049\\
0.907201964607069	3.91167245534824\\
};
\addplot [color=mycolor1, forget plot]
  table[row sep=crcr]{%
0.924302577268678	3.86189211630072\\
1.10682793588482	3.85614030118489\\
1.28118346465662	3.8395283562989\\
1.44774822158637	3.8129081250885\\
1.6069213205598	3.77698458517328\\
1.75910156319047	3.73232319280708\\
1.90467170212362	3.67935834691924\\
2.04398620368068	3.61840200227287\\
2.1773615125963	3.54965182669744\\
2.30506796545077	3.47319857177777\\
2.42732263390715	3.38903253178664\\
2.54428249642295	3.29704912361095\\
2.65603743712263	3.19705375039243\\
2.76260265707262	3.08876623051121\\
2.86391016387759	2.97182519594288\\
2.95979909106812	2.84579300228258\\
3.05000470335532	2.71016185681574\\
3.13414608564841	2.56436206748747\\
3.21171271537749	2.40777354542344\\
3.28205040596633	2.23974194812529\\
3.34434751338946	2.05960110511928\\
3.39762284392349	1.86670357364382\\
3.44071740249287	1.66046124474937\\
3.47229295909028	1.4403977306503\\
3.49084130871891	1.20621363425014\\
3.49470888810416	0.957864517924979\\
3.482141794043	0.695649247545377\\
3.45135579210955	0.420303290690675\\
3.40063409282872	0.133087650260337\\
3.32845205279382	-0.164139997232879\\
3.23362241769647	-0.468888574874113\\
3.11544783921875	-0.778048547420278\\
2.97386069163747	-1.08797222096061\\
2.80952605372759	-1.39462440672324\\
2.62388462414642	-1.6937954453718\\
2.41911975881142	-1.98135419210385\\
2.19804596743955	-2.25350728117634\\
1.96393173817926	-2.50702712272158\\
1.7202826576272	-2.73941660557328\\
1.47061727341327	-2.94899183374906\\
1.21826637271547	-3.1348809568265\\
0.966217799225929	-3.2969520196137\\
0.717017161305549	-3.43569199190078\\
0.472723616438722	-3.55206163566387\\
0.234911949930583	-3.6473478781585\\
0.00470825077904838	-3.72302931723577\\
-0.217153924593788	-3.78066377945691\\
-0.430268147238923	-3.82180111000609\\
-0.63448884007801	-3.84792031745743\\
-0.829869553541633	-3.86038785584452\\
-1.01660872145817	-3.86043285839573\\
-1.19500398360006	-3.84913508976578\\
-1.36541495081342	-3.82742185085764\\
-1.52823359176441	-3.7960707542068\\
-1.68386109282635	-3.75571600097849\\
-1.83268995444608	-3.70685643347235\\
-1.97509013772311	-3.64986416981211\\
-2.11139819369793	-3.58499304573975\\
-2.24190845035496	-3.512386405709\\
-2.3668654730398	-3.43208402260333\\
-2.48645714046661	-3.34402810450295\\
-2.60080778702254	-3.24806848861867\\
-2.70997095461662	-3.14396724498773\\
-2.81392137979104	-3.03140303150048\\
-2.91254592337998	-2.90997567075147\\
-3.00563324307241	-2.77921156931496\\
-3.09286212990652	-2.63857077978506\\
-3.17378859834742	-2.48745671942205\\
-3.24783206100307	-2.32522980327834\\
-3.31426126159385	-2.15122650860271\\
-3.37218111214272	-1.96478562561834\\
-3.42052220388863	-1.76528360221187\\
-3.45803553631711	-1.55218085095605\\
-3.4832958913014	-1.32508050224315\\
-3.49471815072211	-1.08380015805423\\
-3.49059148637807	-0.828455510171989\\
-3.46913637521871	-0.559552068116832\\
-3.42858832468511	-0.278077702428641\\
-3.36730952046585	0.0144154025009713\\
-3.28392500818716	0.315754618250104\\
-3.1774736798184	0.623133930516661\\
-3.0475572882158	0.933160270630387\\
-2.89446495101418	1.24196910719445\\
-2.71924869608243	1.54540820624748\\
-2.52372969898562	1.83927427087464\\
-2.31042541409597	2.11957373071161\\
-2.08240269310558	2.38277094052677\\
-1.84307691646843	2.62598784012684\\
-1.59598734938198	2.84712897763138\\
-1.34458128952978	3.04492150669661\\
-1.09203405767341	3.21887617504392\\
-0.841121217041329	3.36918768566299\\
-0.594147512821357	3.49659864745824\\
-0.352927239901517	3.60225082355921\\
-0.118804802241516	3.68754254447156\\
0.107297807599128	3.75400451875631\\
0.324818987254881	3.80319990561513\\
0.533490828574192	3.83664957538831\\
0.733274610028959	3.85578031258039\\
0.924302577268677	3.86189211630072\\
};
\addplot [color=mycolor1, forget plot]
  table[row sep=crcr]{%
0.944516283357556	3.80590651146832\\
1.12600694416265	3.80018893313277\\
1.29907839514465	3.78370091403236\\
1.46413313464144	3.75732354672323\\
1.6215942509449	3.72178786797593\\
1.77188427403367	3.67768265364217\\
1.9154090101981	3.62546341875114\\
2.05254514551271	3.56546157697518\\
2.18363055333876	3.49789310579632\\
2.30895639976892	3.42286635562741\\
2.4287602885957	3.34038885649499\\
2.54321981533404	3.25037313791706\\
2.65244600700955	3.15264170713785\\
2.75647621470372	3.04693144590314\\
2.85526610658436	2.93289780187852\\
2.94868049073801	2.81011928079819\\
3.03648279320481	2.67810290045345\\
3.11832314470376	2.5362914557908\\
3.193725211668	2.38407366912169\\
3.26207217013619	2.22079855604496\\
3.3225925950385	2.04579560759144\\
3.37434755227592	1.85840263152925\\
3.41622085719777	1.65800323681731\\
3.44691529552266	1.44407586624896\\
3.46495853482702	1.2162558124227\\
3.46872334307913	0.974410573845452\\
3.45646730708205	0.718726991762843\\
3.42639709238593	0.449805694754777\\
3.37676087139271	0.168754519994192\\
3.30596933613776	-0.122731768557435\\
3.21274043621382	-0.422321960411291\\
3.09625600043273	-0.727047255616864\\
2.95631104282055	-1.03336438385475\\
2.7934311549163	-1.33729128191246\\
2.60893272668596	-1.63461097082836\\
2.40490683411936	-1.9211240646005\\
2.18412034337852	-2.1929170949095\\
1.9498441360456	-2.44660745683292\\
1.70563334591066	-2.67952921263771\\
1.45509320214317	-2.88983651530382\\
1.20166398360216	-3.07651881288057\\
0.948450624488358	-3.23933857772801\\
0.698110229976512	-3.37871350278845\\
0.452798421644794	-3.49556910274425\\
0.21416619587588	-3.59118538035426\\
-0.0166058971438045	-3.66705516544324\\
-0.238750359501091	-3.72476457513914\\
-0.451844476043196	-3.76589970664256\\
-0.655740473954405	-3.79197908652604\\
-0.850500202401142	-3.80440868486485\\
-1.03633766529251	-3.80445512238523\\
-1.21357068491732	-3.79323256638883\\
-1.38258161808189	-3.77169927444621\\
-1.54378626382428	-3.74066046616579\\
-1.69760973338298	-3.70077496828251\\
-1.84446795318847	-3.65256377189301\\
-1.98475352691003	-3.59641921584209\\
-2.11882481403556	-3.53261396042429\\
-2.24699723997037	-3.46130925477345\\
-2.3695360075457	-3.38256225194533\\
-2.48664951816332	-3.29633231161098\\
-2.59848292814463	-3.20248637363108\\
-2.70511136386072	-3.10080360597876\\
-2.80653240361764	-2.99097964416911\\
-2.90265751394226	-2.87263086079165\\
-2.99330221482033	-2.74529924502634\\
-3.07817485806523	-2.60845864283539\\
-3.15686405504991	-2.46152331494399\\
-3.22882500894723	-2.30386001167888\\
-3.29336532118994	-2.13480503046703\\
-3.34963128286441	-1.95368798485662\\
-3.39659625583826	-1.75986421743488\\
-3.43305350587418	-1.55275783808721\\
-3.45761674382456	-1.3319171192781\\
-3.46873256610547	-1.09708323512193\\
-3.46470976061236	-0.848271860648108\\
-3.44377071778722	-0.585864743501096\\
-3.40412947457992	-0.310704947477907\\
-3.34409866432677	-0.0241852744818788\\
-3.26222340618926	0.271684877588597\\
-3.15743393924866	0.574251940885052\\
-3.02920140236346	0.880247522848288\\
-2.87767444134446	1.18588745718983\\
-2.70377096196827	1.4870430624743\\
-2.50920191889443	1.77947271618314\\
-2.29641362803012	2.05908698282915\\
-2.06845016011193	2.32221017542621\\
-1.82875372350786	2.56579959516712\\
-1.58093329845906	2.7875920105064\\
-1.32853619394178	2.98616258070545\\
-1.07485285323216	3.16089911709276\\
-0.822774572692031	3.31190888410948\\
-0.574710991668611	3.43988277640292\\
-0.332563142706253	3.54594231826279\\
-0.0977407564306555	3.6314904143393\\
0.128790253767559	3.69807987412364\\
0.346443592488374	3.74730680680177\\
0.554942570560637	3.78073046282245\\
0.754251950804218	3.79981746875353\\
0.944516283357555	3.80590651146832\\
};
\addplot [color=mycolor1, forget plot]
  table[row sep=crcr]{%
0.967808437787213	3.74499261078458\\
1.14818689311289	3.73931195079239\\
1.31985791757415	3.72295916374077\\
1.48325294871124	3.69684878707962\\
1.63882433054309	3.66174127259313\\
1.78702324306083	3.61825136430489\\
1.92828304146005	3.56685777405959\\
2.06300668643723	3.50791301948369\\
2.19155711540948	3.44165271477233\\
2.31424958279424	3.36820391851422\\
2.43134516225289	3.28759237058299\\
2.54304474522393	3.19974861744842\\
2.6494829867059	3.1045131547088\\
2.75072174486073	3.00164082654933\\
2.84674264284519	2.89080483060273\\
2.9374384594573	2.77160079737071\\
3.02260314284291	2.64355155793964\\
3.1019203554607	2.50611339179397\\
3.17495062009433	2.35868476397706\\
3.24111737238403	2.20061881692842\\
3.29969256622918	2.03124116458242\\
3.34978295744406	1.84987481269934\\
3.39031883591177	1.65587423738532\\
3.42004779543775	1.4486706874312\\
3.43753708839786	1.22783047483062\\
3.44118909537926	0.993127164611371\\
3.4292752135284	0.744626918972085\\
3.39999364017396	0.482783573532793\\
3.35155555677723	0.208536270438194\\
3.28230149779729	-0.0766020530382409\\
3.19084478476633	-0.370481450901351\\
3.07623192928899	-0.670295958881667\\
2.93810193520605	-0.972626900863537\\
2.77681969261661	-1.27356044058473\\
2.5935561610029	-1.56887933595467\\
2.39029241707874	-1.85431300653857\\
2.16973656449054	-2.12581468117063\\
1.935159563803	-2.37982503876469\\
1.69017314560574	-2.61348247365794\\
1.43848440406202	-2.82475130603174\\
1.18366381336087	-3.01245716703307\\
0.928956384295275	-3.17623731692343\\
0.677152922916439	-3.31642727135342\\
0.430524619113761	-3.43391106100985\\
0.190813353526649	-3.52996117362653\\
-0.0407360190398877	-3.60608824569472\\
-0.263316741814673	-3.66391288558087\\
-0.476488298868083	-3.70506493338691\\
-0.680102117264968	-3.73111020158552\\
-0.874231767747263	-3.74350154438241\\
-1.0591115050854	-3.74354965140351\\
-1.23508466119144	-3.73240872042629\\
-1.40256186232585	-3.71107262287069\\
-1.56198815148524	-3.68037794588043\\
-1.71381767872152	-3.64101112654664\\
-1.85849450947297	-3.59351765219552\\
-1.99643816425042	-3.53831192954726\\
-2.12803265189593	-3.47568691584796\\
-2.25361793603441	-3.40582297170541\\
-2.37348294778796	-3.32879566284923\\
-2.48785941152722	-3.24458243257702\\
-2.59691587926138	-3.153068212415\\
-2.70075147467725	-3.05405015647705\\
-2.79938893547174	-2.94724179295157\\
-2.89276662118796	-2.83227699924786\\
-2.9807292348313	-2.70871433859584\\
-3.06301710498207	-2.57604245620105\\
-3.13925400991203	-2.4336874304032\\
-3.20893372022145	-2.28102321172531\\
-3.27140572068634	-2.1173865544282\\
-3.32586097793594	-1.9420981301299\\
-3.37131917984663	-1.75449176545173\\
-3.40661960543466	-1.5539538829359\\
-3.43041868081252	-1.33997511444028\\
-3.44119826803717	-1.11221550915271\\
-3.43728965053286	-0.870583535237989\\
-3.41691871122378	-0.615326928416679\\
-3.37827747551848	-0.347130215287896\\
-3.31962541754045	-0.0672095307517391\\
-3.23942014602516	0.222609279273533\\
-3.13647108052289	0.519847122444371\\
-3.01010204621649	0.821381790707592\\
-2.86030101476354	1.12352752006707\\
-2.6878302144381	1.42219058941629\\
-2.49427053077934	1.71309325270415\\
-2.28198236475224	1.99204203709281\\
-2.05398010648345	2.2552034280972\\
-1.81373520525708	2.49934532639181\\
-1.56493772823386	2.72200884185793\\
-1.31125332476364	2.92159021948065\\
-1.05610981641013	3.09733170608087\\
-0.802537173781546	3.24923679019982\\
-0.553070814292143	3.37793516318734\\
-0.309715500973054	3.48452486394514\\
-0.0739585488149666	3.57041506006226\\
0.153182698290488	3.6371857336636\\
0.371093942046129	3.68647293783938\\
0.579489767029112	3.71988203757484\\
0.778340934540154	3.73892713446705\\
0.967808437787212	3.74499261078458\\
};
\addplot [color=mycolor1, forget plot]
  table[row sep=crcr]{%
0.994196135033692	3.68031066273118\\
1.17340809817122	3.67466886909633\\
1.34358529454558	3.65846042759474\\
1.50519440257395	3.63263740579726\\
1.65872294953164	3.59799278088386\\
1.80465617574299	3.55516956627888\\
1.94345982046772	3.50467133034416\\
2.07556737826532	3.44687286752932\\
2.20137056907467	3.38203024968933\\
2.32121196908053	3.31028982587883\\
2.43537893622874	3.23169598202777\\
2.54409812292944	3.14619764586289\\
2.64752999734826	3.05365365219381\\
2.7457628975869	2.95383719024497\\
2.83880622732637	2.84643965569918\\
2.92658247713054	2.73107434051203\\
3.00891783529258	2.60728052632482\\
3.08553125196808	2.47452871344698\\
3.15602196091464	2.33222792507424\\
3.21985567036081	2.1797362787155\\
3.2763499398936	2.01637630730141\\
3.32465969933645	1.84145681803885\\
3.36376447264931	1.65430334738335\\
3.39245966613	1.45429941414018\\
3.40935525226816	1.24094064278145\\
3.41288624655233	1.01390321439251\\
3.40134033834497	0.773126737678835\\
3.37290854193795	0.518909246585072\\
3.3257642423025	0.252008462058861\\
3.25817385791012	-0.026261159208295\\
3.16863793304269	-0.313951047084765\\
3.05605464214126	-0.608440208278517\\
2.91988919546258	-0.906456213524509\\
2.7603245152051	-1.20417127277554\\
2.57836396307739	-1.4973783244928\\
2.37585911180771	-1.78173589999365\\
2.15544620475775	-2.05305307612738\\
1.92039246005548	-2.30757306327229\\
1.67437275902528	-2.54221124732962\\
1.42121197663355	-2.7547126655596\\
1.16463328497573	-2.94371199518092\\
0.908047180755667	-3.1086997798198\\
0.654402882025848	-3.2499151684208\\
0.406108377881043	-3.36819391757691\\
0.165012567418598	-3.46480052850891\\
-0.0675648393948347	-3.54126761569399\\
-0.290772489374853	-3.59925732074702\\
-0.504151061163646	-3.640451598681\\
-0.70755343146076	-3.66647209103852\\
-0.901069341070793	-3.67882649143326\\
-1.08495903597741	-3.6788765088859\\
-1.25959768192474	-3.66782215616716\\
-1.42543057397612	-3.64669754584442\\
-1.58293813815797	-3.61637421068568\\
-1.73260924409234	-3.57756888264478\\
-1.87492122205869	-3.53085350535999\\
-2.01032505403617	-3.47666595152244\\
-2.13923438109869	-3.41532045672481\\
-2.26201717324251	-3.34701718184587\\
-2.3789891052139	-3.27185060417864\\
-2.49040785532301	-3.18981664267737\\
-2.59646768779841	-3.10081857168896\\
-2.69729379438629	-3.00467189329604\\
-2.79293596339599	-2.90110844015344\\
-2.88336122293346	-2.78978008465047\\
-2.96844518093896	-2.67026255030781\\
-3.04796187215268	-2.54205996954056\\
-3.12157203946577	-2.40461101806195\\
-3.18880994740434	-2.25729768628089\\
-3.24906907751574	-2.09945802108844\\
-3.30158742310312	-1.93040447377617\\
-3.34543361998514	-1.74944978628918\\
-3.37949585047893	-1.55594256956421\\
-3.40247634696937	-1.34931475687185\\
-3.4128953566482	-1.12914277315562\\
-3.40910947746106	-0.895223306830814\\
-3.38935006572105	-0.647662725689275\\
-3.35178750025212	-0.386976208024451\\
-3.29462585142282	-0.114188522962037\\
-3.21622928049041	0.169076531734969\\
-3.11527584916467	0.460535709474934\\
-2.99092658770654	0.757235257041731\\
-2.84298902572669	1.05560799962333\\
-2.67204759944406	1.35160972666908\\
-2.47953180906159	1.64093219927022\\
-2.26769940087452	1.91927262561283\\
-2.03952636078154	2.1826235633948\\
-1.79851474015648	2.42753895198215\\
-1.54844718051117	2.65133526219359\\
-1.29312730776821	2.85220102774083\\
-1.036144754668	3.02920829215682\\
-0.780693638554805	3.1822388174508\\
-0.529458403888326	3.31185067720291\\
-0.284566353757399	3.41911497970214\\
-0.0475957291863381	3.50544923079226\\
0.18037683768094	3.57246640440727\\
0.398706003313474	3.6218503749631\\
0.60709838598585	3.65526119870713\\
0.805534219195617	3.6742687731098\\
0.994196135033692	3.68031066273118\\
};
\addplot [color=mycolor1, forget plot]
  table[row sep=crcr]{%
1.02375040531617	3.61290204042887\\
1.20176097639549	3.60730045239441\\
1.37036957674768	3.59124369131131\\
1.53008538748134	3.56572536946192\\
1.68143780878129	3.53157387649472\\
1.82495211979655	3.48946246279246\\
1.96113166891982	3.43992080417065\\
2.09044497609687	3.38334668835468\\
2.2133163581027	3.32001698205598\\
2.33011892490929	3.25009741061438\\
2.44116901027864	3.17365094361483\\
2.54672128027864	3.09064476183925\\
2.64696390782588	3.0009559112872\\
2.74201331386758	2.90437585191723\\
2.83190806411648	2.8006142012692\\
2.91660158441303	2.68930207223581\\
2.99595343022608	2.56999552386096\\
3.06971893197026	2.44217979639812\\
3.13753715736318	2.30527519718943\\
3.19891731008915	2.15864574879307\\
3.25322395208505	2.00161200452337\\
3.29966183225684	1.83346976493728\\
3.33726166727844	1.65351675400746\\
3.36486898352328	1.46108956040721\\
3.38113910113713	1.25561318766688\\
3.38454247282657	1.03666518752547\\
3.3733857285037	0.804055302209507\\
3.34585461031475	0.557919500568181\\
3.30008499547509	0.298823996439621\\
3.23426669333464	0.0278702496733986\\
3.14678092088188	-0.253213474226809\\
3.03636585661394	-0.54201345615402\\
2.90229582008168	-0.835426867441588\\
2.74455015014256	-1.12973284012124\\
2.56394095281318	-1.42074875525535\\
2.36216843685569	-1.70406624141488\\
2.14178131502429	-1.97534198104143\\
1.90603727640003	-2.23060186778484\\
1.65868024595527	-2.46651004688063\\
1.40366973892964	-2.68056058438075\\
1.14490648466575	-2.87116725842797\\
0.885995072210844	-3.03764980487017\\
0.63007116427807	-3.18013500544318\\
0.379703604907356	-3.29940273110009\\
0.136866392078723	-3.3967091096898\\
-0.0970344061660405	-3.47361363503207\\
-0.321097449612501	-3.53182808856116\\
-0.534844239226391	-3.57309603977806\\
-0.738132419461264	-3.5991045027063\\
-0.931073626071126	-3.61142473381267\\
-1.11396114323217	-3.61147691260704\\
-1.28720951056091	-3.60051289518032\\
-1.45130614583367	-3.57961168548513\\
-1.60677385886146	-3.54968318739327\\
-1.75414258331887	-3.51147682884627\\
-1.89392851680262	-3.46559259474307\\
-2.02661895578985	-3.41249278629328\\
-2.15266131796345	-3.35251342660259\\
-2.27245508285067	-3.28587467397293\\
-2.38634561026136	-3.21268991747198\\
-2.49461899453625	-3.13297344723351\\
-2.59749727486108	-3.04664674493076\\
-2.69513344957367	-2.95354355339843\\
-2.78760584167458	-2.85341397942644\\
-2.87491144258746	-2.74592797770453\\
-2.95695793279452	-2.63067867162487\\
-3.0335541551112	-2.5071861012065\\
-3.10439891645093	-2.37490216114041\\
-3.16906813919749	-2.23321771157066\\
-3.22700060200421	-2.0814731142732\\
-3.2774828366435	-1.91897376091784\\
-3.31963422182456	-1.74501249358455\\
-3.35239397487088	-1.55890111606585\\
-3.37451261146532	-1.36001335729562\\
-3.38455150677088	-1.14784151232587\\
-3.38089535402736	-0.922068315058349\\
-3.36178335378977	-0.682654091339456\\
-3.32536546776439	-0.429936596356689\\
-3.26978941824833	-0.164736978685912\\
-3.1933215530012	0.111539791837042\\
-3.09449957081791	0.396827137223658\\
-2.97230730339444	0.688362960794408\\
-2.82635227231545	0.982721291257161\\
-2.65701807848947	1.27592551447855\\
-2.46555953737673	1.5636464007621\\
-2.25411243852087	1.84146980592806\\
-2.02560326847436	2.10520006333872\\
-1.78356466168654	2.35115257139234\\
-1.53188345629107	2.57638853279399\\
-1.27452259365677	2.77885734455166\\
-1.01526081426038	2.95743342231848\\
-0.757485202333998	3.11185660361474\\
-0.504055616196481	3.24260160764968\\
-0.257243124584251	3.35070882318175\\
-0.0187316848424999	3.43760661534103\\
0.210334443627698	3.50494770426823\\
0.429276150050858	3.55447278851852\\
0.637793727556865	3.58790627445286\\
0.835881553665213	3.6068830687125\\
1.02375040531617	3.61290204042887\\
};
\addplot [color=mycolor1, forget plot]
  table[row sep=crcr]{%
1.05659982423101	3.5436962097753\\
1.23338953524892	3.5381356863494\\
1.40036900034433	3.52223658455641\\
1.55809830241556	3.49703803845282\\
1.70715629008817	3.46340652800132\\
1.84811490673	3.42204718070233\\
1.98152076859	3.37351662858223\\
2.10788216348064	3.31823592487422\\
2.22765991767207	3.25650260540557\\
2.34126085936369	3.18850139185262\\
2.44903285812511	3.11431331704303\\
2.55126062779082	3.03392324367229\\
2.64816164416347	2.94722587889718\\
2.73988165365188	2.85403048404157\\
2.82648934309907	2.7540645618524\\
2.90796981500203	2.64697689049896\\
2.98421657841733	2.53234037859128\\
3.05502183896928	2.4096553521991\\
3.12006497023276	2.27835406514218\\
3.17889919744524	2.1378074573444\\
3.23093675399576	1.98733547741162\\
3.27543311994892	1.82622262905518\\
3.31147146435916	1.65374077137479\\
3.33794913309673	1.46918154206162\\
3.3535689795292	1.27190097051325\\
3.35683951089544	1.06137872310933\\
3.34608911292561	0.837293708330638\\
3.31950076128171	0.599616122860203\\
3.27517415625753	0.348713088675428\\
3.21122140990007	0.0854606072336505\\
3.12589939848037	-0.188651157300746\\
3.01777592707658	-0.471438308117994\\
2.88591787080211	-0.759992854618088\\
2.73007874651697	-1.05072513220936\\
2.550853793361	-1.33949555086233\\
2.34976699558954	-1.62183705326857\\
2.12926057137599	-1.89324869827937\\
1.89257432820553	-2.14952018695187\\
1.64352619912048	-2.38703493485975\\
1.38622833938899	-2.60300137176016\\
1.12478689562998	-2.79557870125465\\
0.863033287022711	-2.96388830141094\\
0.604321973242455	-3.10792613548574\\
0.351410388033882	-3.22840742802162\\
0.106418236504346	-3.32657959408979\\
-0.129149325311288	-3.40403480820078\\
-0.354335519552628	-3.46254393948293\\
-0.568643112585792	-3.50392309778006\\
-0.771939191240562	-3.5299354786536\\
-0.964364609538357	-3.54222559917473\\
-1.14625434955744	-3.54228020380336\\
-1.3180713423304	-3.53140935223569\\
-1.48035384250214	-3.51074166070109\\
-1.63367505571405	-3.48122869904976\\
-1.77861308794007	-3.44365471792408\\
-1.91572914317403	-3.39864895889987\\
-2.04555202384444	-3.34669868759013\\
-2.16856723967643	-3.28816176780067\\
-2.28520931521275	-3.22327808645115\\
-2.39585615477706	-3.15217948163206\\
-2.50082455396636	-3.07489805892091\\
-2.60036613240358	-2.99137293864105\\
-2.69466310590541	-2.90145558780369\\
-2.78382342447307	-2.80491397812694\\
-2.86787488513555	-2.70143589453793\\
-2.94675789701699	-2.59063181274185\\
-3.02031664354309	-2.4720378837501\\
-3.08828847005424	-2.34511972036689\\
-3.15029144544934	-2.20927788655847\\
-3.20581023122645	-2.06385625298164\\
-3.25418067517839	-1.90815470103258\\
-3.29457397191594	-1.74144801953847\\
-3.32598184364106	-1.56301320296016\\
-3.34720503075629	-1.37216764621141\\
-3.35684845374742	-1.16832079568533\\
-3.35332766149975	-0.951041435213579\\
-3.33489244351797	-0.720141650946591\\
-3.29967439518837	-0.475776270276219\\
-3.24576518889597	-0.218552900262679\\
-3.17133050273471	0.0503574420655213\\
-3.07476013326221	0.329125177290405\\
-2.95484728379189	0.615204118239174\\
-2.81097990419437	0.905334800269474\\
-2.6433164241271	1.19563026176681\\
-2.45291113117283	1.48175420485149\\
-2.24175529051304	1.75918285353335\\
-2.01271173396278	2.02352008155767\\
-1.76934182203281	2.27081800201832\\
-1.51564830327697	2.49784970293208\\
-1.25577690828014	2.70229053142342\\
-0.993726460657979	2.88278613974326\\
-0.733110174569144	3.03891122275533\\
-0.476993794902001	3.17104359983949\\
-0.22781652556404	3.28018863279807\\
0.0126152938222836	3.36778859164235\\
0.243080823721761	3.4355438678002\\
0.462864883201927	3.48526240878233\\
0.671664208211566	3.51874399905155\\
0.86949350295471	3.53769889760678\\
1.05659982423101	3.5436962097753\\
};
\addplot [color=mycolor1, forget plot]
  table[row sep=crcr]{%
1.09293589616732	3.47352416739806\\
1.26849674134541	3.46800521628701\\
1.4337963726869	3.45226881742003\\
1.58945551384844	3.42740359121524\\
1.73611105723776	3.39431658087325\\
1.87438896106105	3.35374610094779\\
2.00488523863961	3.30627618560696\\
2.1281529475438	3.25235098875703\\
2.24469342574854	3.19228814442275\\
2.35495035945798	3.12629055407149\\
2.45930556407635	3.05445637538651\\
2.55807560223686	2.97678718834121\\
2.6515085506791	2.89319444595849\\
2.73978036775067	2.80350440791691\\
2.82299041518313	2.70746182798343\\
2.90115576282974	2.60473273944519\\
2.97420396577533	2.49490677185651\\
3.04196406409613	2.37749955163049\\
3.10415563430801	2.25195590161078\\
3.16037584105218	2.11765477296779\\
3.21008462779589	1.97391712583436\\
3.25258848496511	1.82001832512479\\
3.28702369044987	1.65520702208577\\
3.31234058338146	1.47873290964321\\
3.32729135394503	1.28988608325564\\
3.33042502619716	1.08805085086839\\
3.32009472087404	0.872776463495758\\
3.294483716037	0.643866026764014\\
3.25165786063622	0.401482380030563\\
3.18965183998718	0.14626562297628\\
3.10659467762955	-0.120548837136475\\
3.00087467000739	-0.397029733799231\\
2.87133514004594	-0.680491177724248\\
2.71748067453288	-0.967502756385573\\
2.53966159761159	-1.25399141701655\\
2.3391970671875	-1.53544458360951\\
2.11839966185456	-1.80720207873993\\
1.88047959687191	-2.06479972792575\\
1.6293324405293	-2.304309147934\\
1.36924230153405	-2.52261466175929\\
1.10455236597759	-2.71758238549025\\
0.839358910264985	-2.88810324984138\\
0.577273118980962	-3.03402069412792\\
0.321273502343611	-3.15597497606478\\
0.0736493239613713	-3.25520446646513\\
-0.163980901938123	-3.33334094448337\\
-0.39059948088467	-3.39222550860005\\
-0.605691971291168	-3.43375953298264\\
-0.809141323680796	-3.4597947909185\\
-1.00112695072322	-3.47205997306241\\
-1.18203618954599	-3.47211728526765\\
-1.35239117552056	-3.46134177342606\\
-1.51279121505116	-3.44091649946167\\
-1.66386909976731	-3.41183787617206\\
-1.80625908431625	-3.37492682851767\\
-1.94057411061837	-3.33084269798519\\
-2.06739003805033	-3.28009782656068\\
-2.18723495344339	-3.22307152820622\\
-2.30058198023905	-3.1600227071581\\
-2.40784432622723	-3.09110075955144\\
-2.50937157959258	-3.01635464438066\\
-2.60544647762741	-2.93574017215796\\
-2.69628153531545	-2.84912566758599\\
-2.78201504058541	-2.75629624183129\\
-2.86270601000842	-2.65695698090751\\
-2.93832776493263	-2.55073543601379\\
-3.00875984686818	-2.43718390414292\\
-3.07377805825877	-2.31578212656641\\
-3.1330425106563	-2.18594122180271\\
-3.18608371292067	-2.04700991947117\\
-3.23228697158644	-1.89828447842317\\
-3.27087574721622	-1.73902405258735\\
-3.30089516523149	-1.56847368625895\\
-3.32119766893653	-1.38589751586511\\
-3.3304338617257	-1.19062500815314\\
-3.32705290150082	-0.982112972744464\\
-3.30931826690162	-0.760025345617177\\
-3.27534601764221	-0.524330951425817\\
-3.2231732680184	-0.275416197277719\\
-3.15086363930317	-0.0142046695419724\\
-3.05665291922995	0.257730879547194\\
-2.93913116800754	0.538085541447675\\
-2.7974470431082	0.823794593647349\\
-2.6315078179007	1.11108723868343\\
-2.44213824832725	1.3956391351759\\
-2.23115840500554	1.67282304544363\\
-2.00134937353323	1.93803265140908\\
-1.75629691134506	2.18703182923845\\
-1.50013135080415	2.41626989391422\\
-1.2372073006177	2.62310873735567\\
-0.971779315032066	2.80592927048401\\
-0.707725490017997	2.96411384119377\\
-0.448353209970221	3.09792740271686\\
-0.196298235066255	3.20833524870649\\
0.0464916587598717	3.29679723218257\\
0.278709355828535	3.36507081509076\\
0.499601885991418	3.41504341430274\\
0.708866665431189	3.44860294222669\\
0.906546832754389	3.46754676618677\\
1.09293589616732	3.47352416739806\\
};
\addplot [color=mycolor1, forget plot]
  table[row sep=crcr]{%
1.13302068675688	3.40313637942729\\
1.30735216851495	3.39765928538817\\
1.47092677483942	3.38209011309542\\
1.62443718206064	3.3575709367022\\
1.7685880866297	3.32505162330178\\
1.90406761415	3.28530464075765\\
2.03152780257099	3.23894145950706\\
2.15157173658891	3.18642873356713\\
2.26474533572043	3.12810319314109\\
2.37153220990307	3.06418469332344\\
2.47235035129693	2.99478719894024\\
2.56754971585957	2.91992769738027\\
2.6574099652991	2.83953316231655\\
2.74213779835695	2.7534457746188\\
2.82186341190383	2.66142666773136\\
2.89663570968929	2.56315852299072\\
2.96641593285578	2.45824741206679\\
3.03106943574792	2.34622438362293\\
3.09035538994984	2.22654743366613\\
3.14391429031312	2.09860469807455\\
3.19125328747383	1.96171997457725\\
3.23172961998089	1.81516202908499\\
3.26453281477108	1.65815956638465\\
3.28866692652729	1.48992422469463\\
3.30293495665252	1.30968442431625\\
3.30592877774352	1.11673323346959\\
3.29602938262015	0.910493382839516\\
3.27142395306259	0.690601814115529\\
3.23014776659083	0.457014216525408\\
3.17015968197974	0.210126377221273\\
3.08945885590666	-0.0490964927735401\\
2.98624619175463	-0.318998769080705\\
2.85912575571581	-0.597146044523929\\
2.70732901701836	-0.880299346639518\\
2.53093039362746	-1.16448129438572\\
2.33101101524119	-1.44515283591611\\
2.10972539158081	-1.71749742055327\\
1.87023802609507	-1.97678093821641\\
1.61652380359922	-2.21873030170006\\
1.35305954468467	-2.43986251032453\\
1.08446177591869	-2.63770588575249\\
0.815136413441826	-2.81088291357719\\
0.548996485816688	-2.95905851749179\\
0.289280196144789	-3.08278558254861\\
0.0384743594774492	-3.18329306991735\\
-0.201672979692613	-3.26226070706034\\
-0.430077793684999	-3.32161311525602\\
-0.646211432851508	-3.36335192736009\\
-0.849981410527051	-3.38943187493426\\
-1.04161760406947	-3.40167823750354\\
-1.22157284792635	-3.40173856002849\\
-1.39044147947608	-3.39106017274081\\
-1.54889585570102	-3.37088556392582\\
-1.69763890928502	-3.34225904988895\\
-1.83736999643794	-3.30603979435721\\
-1.9687611709434	-3.2629176975686\\
-2.09244126968614	-3.21342986163187\\
-2.20898559685683	-3.1579762239611\\
-2.31890942029672	-3.09683357292348\\
-2.42266387964347	-3.03016757652347\\
-2.52063322625479	-2.95804272260561\\
-2.61313256500743	-2.88043023576772\\
-2.70040545437466	-2.79721413980557\\
-2.78262085456982	-2.7081957039486\\
-2.85986900638624	-2.6130965685346\\
-2.93215588849615	-2.51156090882488\\
-2.99939595191547	-2.40315707968205\\
-3.06140288241337	-2.28737930319375\\
-3.11787821370082	-2.16365013027391\\
-3.16839773091541	-2.03132463968469\\
-3.212395798176	-1.88969764544751\\
-3.24914805925617	-1.73801557117162\\
-3.27775345127651	-1.57549510699851\\
-3.29711719989476	-1.40135125129047\\
-3.30593748850052	-1.2148377624613\\
-3.302699839467	-1.01530323260925\\
-3.28568485651335	-0.802265657690554\\
-3.25299663615765	-0.575507104717066\\
-3.2026203827126	-0.3351873569578\\
-3.13251772553263	-0.0819707903550062\\
-3.04076577657179	0.182845918157791\\
-2.92573984772423	0.457225623874559\\
-2.7863293168359	0.73832973233008\\
-2.62216232277006	1.02253511756338\\
-2.43380124155044	1.30555436200757\\
-2.22286316939781	1.58266832264814\\
-1.99202429442627	1.84905394833601\\
-1.74488709228206	2.10016182470575\\
-1.48572084946671	2.33207840649466\\
-1.21911829403769	2.54180703771746\\
-0.949631265581961	2.72742165398387\\
-0.681448676680737	2.88807983140649\\
-0.41816212230173	3.02391448273191\\
-0.162637548043815	3.1358447873217\\
0.0830137649937929	3.22535264164017\\
0.317387867035559	3.29426385042997\\
0.539693124365956	3.34455979598504\\
0.749633818797704	3.3782314325697\\
0.947292108645063	3.3971767483328\\
1.13302068675687	3.40313637942729\\
};
\addplot [color=mycolor1, forget plot]
  table[row sep=crcr]{%
1.17719733975312	3.33322396506955\\
1.35030249818025	3.32778891865049\\
1.51210809846406	3.31239138794923\\
1.6633919221548	3.28823087405312\\
1.80493755041589	3.25630209514873\\
1.93750430270129	3.21741226946189\\
2.06180738253903	3.17219990803701\\
2.17850539478485	3.12115312410239\\
2.28819294015768	3.06462632145195\\
2.3913965003925	3.00285469798302\\
2.48857225515869	2.93596636593233\\
2.58010480873367	2.86399211193033\\
2.66630605624465	2.7868729486618\\
2.74741359916657	2.7044656838033\\
2.82358824267196	2.61654677904956\\
2.89491018808665	2.52281481350526\\
2.96137358622634	2.42289191838283\\
3.02287915605375	2.31632462878386\\
3.07922461390126	2.20258471810196\\
3.13009272134828	2.08107075675129\\
3.17503687128785	1.95111138561789\\
3.21346432814191	1.81197163079968\\
3.24461756879648	1.66286401890881\\
3.26755469912438	1.50296677519668\\
3.28113072177179	1.33145196182033\\
3.2839825776602	1.14752694432796\\
3.27452241765133	0.950492869956774\\
3.25094543127714	0.739823575020654\\
3.21126052595	0.515267021592148\\
3.15335365277612	0.276968374618114\\
3.07509361635764	0.0256085736204264\\
2.97448735595804	-0.237455475206265\\
2.84988440258159	-0.510072370381124\\
2.70021767939671	-0.789231277761984\\
2.52525123765674	-1.07108606918227\\
2.32578937578246	-1.35109731252547\\
2.1037934386845	-1.62430061525686\\
1.8623602868618	-1.88567821596643\\
1.60554296119306	-2.13057743858749\\
1.33803325211188	-2.35509891946982\\
1.06476312386227	-2.55638059451322\\
0.790501560057985	-2.73273083095373\\
0.519517860924475	-2.88360440588051\\
0.255356471029191	-3.00945165230741\\
0.000735039895534709	-3.11149167834557\\
-0.242450364521467	-3.19146232558214\\
-0.473044209789634	-3.25138778815474\\
-0.690508675550763	-3.29338783178077\\
-0.894787541396454	-3.3195370096192\\
-1.08617634430937	-3.33177145469118\\
-1.2652096649494	-3.33183511510098\\
-1.43256970375565	-3.32125551528476\\
-1.58901598471793	-3.301339723258\\
-1.73533370960581	-3.2731828908662\\
-1.87229737577021	-3.23768367258399\\
-2.00064620563421	-3.19556257903416\\
-2.1210682989065	-3.14738071624703\\
-2.23419094197467	-3.09355738515112\\
-2.34057504222673	-3.03438572111461\\
-2.44071212577514	-2.97004601408423\\
-2.53502271940783	-2.90061663612471\\
-2.6238552305647	-2.82608267277113\\
-2.70748465315566	-2.7463424518198\\
-2.78611057654608	-2.66121222069601\\
-2.85985407502429	-2.57042926552956\\
-2.92875311990893	-2.47365381018884\\
-2.99275620019862	-2.37047009744903\\
-3.05171387521637	-2.26038715189157\\
-3.10536803169253	-2.14283987010183\\
-3.15333870101788	-2.01719129415084\\
-3.19510844104602	-1.88273721512157\\
-3.23000454330044	-1.738714637411\\
-3.25717974709943	-1.58431611467405\\
-3.27559279693591	-1.41871252496896\\
-3.28399114506574	-1.24108742257173\\
-3.28089944041944	-1.05068654847698\\
-3.26461916262123	-0.846886146933993\\
-3.23324672514285	-0.629283014845219\\
-3.18471920149778	-0.397807139638174\\
-3.11689775481201	-0.152853719678765\\
-3.02769764519051	0.104575151663589\\
-2.91526879688354	0.372737614777059\\
-2.7782210069923	0.649055916963807\\
-2.61587298585537	0.930091699417381\\
-2.42848727696386	1.21162640459507\\
-2.21743989233247	1.48886716772943\\
-1.98527245900548	1.75677187288989\\
-1.73559192587236	2.01045298137057\\
-1.47281714099461	2.24559094462114\\
-1.20181194350925	2.4587783360375\\
-0.92747448166092	2.64773182655615\\
-0.654359680035902	2.81134496592553\\
-0.386394751691459	2.94959521024363\\
-0.126716165951527	3.06334822411246\\
0.122375429000432	3.15411339977998\\
0.359367735470225	3.22379856286806\\
0.583430829221697	3.27449645974263\\
0.794284661203453	3.30831872767862\\
0.992064213675017	3.32727966867312\\
1.17719733975312	3.33322396506955\\
};
\addplot [color=mycolor1, forget plot]
  table[row sep=crcr]{%
1.22590434247989	3.26444255553571\\
1.39778568519362	3.2590497847381\\
1.55777518892862	3.24382861499667\\
1.70675103836143	3.22003994081109\\
1.84558826292861	3.18872508758141\\
1.97512733776895	3.15072621278902\\
2.09615428269976	3.10670801538788\\
2.20938891075182	3.05717857651015\\
2.31547857593696	3.00250813821333\\
2.41499540189816	2.94294526954143\\
2.50843549202283	2.87863026509742\\
2.59621902113075	2.80960585033293\\
2.67869040132478	2.73582539079166\\
2.75611791888507	2.65715886373023\\
2.82869237462659	2.5733968811675\\
2.89652434460227	2.48425307600197\\
2.9596397271147	2.38936519468617\\
3.01797327033446	2.28829529598714\\
3.07135979769443	2.1805295503353\\
3.11952288389663	2.06547828426429\\
3.16206080708689	1.94247713761692\\
3.19842974646615	1.81079051711406\\
3.22792445709442	1.66961895656359\\
3.24965710022171	1.51811254070199\\
3.26253562201684	1.35539320285037\\
3.26524415210251	1.18058940520403\\
3.25622942186491	0.992887307773702\\
3.23369920606965	0.791602743226298\\
3.19564113848408	0.576277664196529\\
3.1398725068913	0.346802528699445\\
3.06413287471086	0.103561521425096\\
2.96623008252873	-0.152410070565631\\
2.84424437601905	-0.419277452491931\\
2.69678336750397	-0.694299485893786\\
2.52326227368228	-0.973804263907594\\
2.32416298853439	-1.25328660214706\\
2.10121018814479	-1.52765009372196\\
1.85740346435531	-1.79158308598752\\
1.59686869882091	-2.04001649265165\\
1.32453632558024	-2.26857849595714\\
1.04570301329607	-2.4739540329236\\
0.765565360812775	-2.65408168115743\\
0.488815500179686	-2.80816699289228\\
0.219360997648381	-2.93653887741222\\
-0.0398095987651686	-3.04040602242973\\
-0.286630899201536	-3.12157692434183\\
-0.519871267646057	-3.18219496718836\\
-0.738991605563387	-3.22451960078847\\
-0.943987665821066	-3.25076517916732\\
-1.13524008710841	-3.26299522676806\\
-1.3133853463107	-3.26306258127946\\
-1.47921241980635	-3.25258358098722\\
-1.63358462390113	-3.23293521265411\\
-1.77738335876483	-3.20526623833188\\
-1.9114694962231	-3.17051570223822\\
-2.03665819186413	-3.12943433562294\\
-2.15370343312798	-3.08260603329976\\
-2.26328932688233	-3.03046776553468\\
-2.36602580463572	-2.9733270897908\\
-2.4624470016689	-2.91137693497489\\
-2.55301102484149	-2.84470763450985\\
-2.63810016768621	-2.77331635409199\\
-2.71802087730646	-2.69711414748705\\
-2.79300294516599	-2.61593091635425\\
-2.86319750178541	-2.52951857428171\\
-2.92867346026797	-2.43755274047984\\
-2.98941209046259	-2.33963333071361\\
-3.04529942914146	-2.23528448666135\\
-3.09611625826001	-2.12395440523951\\
-3.14152543401717	-2.00501581362534\\
-3.18105645264084	-1.87776810279971\\
-3.21408733492589	-1.7414425017944\\
-3.23982425672907	-1.59521216135885\\
-3.25727992165527	-1.43820961964884\\
-3.26525255654388	-1.26955481037257\\
-3.26230870533485	-1.08839744737734\\
-3.24677476675413	-0.893978070280039\\
-3.21674442802203	-0.685711889067924\\
-3.17011152831026	-0.463298237743598\\
-3.10463978567584	-0.226855164446353\\
-3.01808102245287	0.0229273383947193\\
-2.9083502301115	0.284631072839721\\
-2.77375704158983	0.555977281428288\\
-2.61327784344003	0.833755689427438\\
-2.42683244890199	1.11385674149264\\
-2.2155098612699	1.3914402852622\\
-1.98167908157629	1.66124848376536\\
-1.72893302723864	1.91803170286154\\
-1.46184919114113	2.15701658680652\\
-1.18559991910093	2.37432382629505\\
-0.905488161095217	2.56725215472164\\
-0.626502063598137	2.73438287412226\\
-0.352967397025358	2.87550894157846\\
-0.0883401830366873	2.99143329249239\\
0.164858930307504	3.08369955520424\\
0.404996785704091	3.1543143820327\\
0.631207403458537	3.20550301505091\\
0.843238767849654	3.23951886961585\\
1.04129670838009	3.25851096417827\\
1.22590434247988	3.26444255553571\\
};
\addplot [color=mycolor1, forget plot]
  table[row sep=crcr]{%
1.27969473451899	3.19743891413949\\
1.45034994159423	3.19208882280089\\
1.608468718383	3.17704946054004\\
1.75504743713589	3.15364704231165\\
1.8910667842996	3.12297092862\\
2.01745932978799	3.08589794332656\\
2.13509003936657	3.04311762731975\\
2.24474574897865	2.99515607355915\\
2.34713052878886	2.9423971209665\\
2.44286465667115	2.88510040147726\\
2.53248555223779	2.82341615911886\\
2.61644949636217	2.75739699170234\\
2.6951332998723	2.68700677794949\\
2.76883531595009	2.61212709684994\\
2.8377753393484	2.53256145660749\\
2.90209302353936	2.44803765162519\\
2.96184449223774	2.35820857506451\\
3.01699683974581	2.2626518460952\\
3.06742021994134	2.16086867923668\\
3.11287723304133	2.05228254415422\\
3.15300935413325	1.93623835707118\\
3.18732023860967	1.81200323241128\\
3.21515593194735	1.67877023059018\\
3.23568236933861	1.53566708793343\\
3.24786116349761	1.38177261724521\\
3.25042565989046	1.21614429781399\\
3.24186071298692	1.03786143510928\\
3.22039170350251	0.846088937327736\\
3.18399096149264	0.640166798123835\\
3.13041270376352	0.41972908786561\\
3.05727006314035	0.184852660212058\\
2.9621682737045	-0.0637711837261371\\
2.84290427252501	-0.324659745035805\\
2.69773225686422	-0.595388289159807\\
2.5256756024321	-0.872510538970974\\
2.32684009585234	-1.15160055593432\\
2.10265963161525	-1.42745523771516\\
1.85599667089474	-1.69446398913854\\
1.59103854112513	-1.94710290900718\\
1.31297908263676	-2.18046319213452\\
1.02753779962426	-2.39070140666667\\
0.740417862665125	-2.57531756857995\\
0.456816757094041	-2.73321902000345\\
0.18107562986538	-2.8645894324229\\
-0.0834972651340051	-2.97062633628157\\
-0.33464256235706	-3.05322456841679\\
-0.571049054307578	-3.11467098787108\\
-0.79218830185694	-3.15739101298888\\
-0.998129124263412	-3.18376270669748\\
-1.18936222632091	-3.19599631795903\\
-1.36665104937862	-3.19606775514251\\
-1.53091423172332	-3.1856915971373\\
-1.68313847021417	-3.16632026979512\\
-1.82431733797418	-3.13915871316964\\
-1.95541060374083	-3.10518684935743\\
-2.07731882889597	-3.06518475312069\\
-2.19086879991828	-3.01975740852929\\
-2.296806278529	-2.96935732324368\\
-2.39579341215939	-2.91430418010914\\
-2.48840886072844	-2.85480126234141\\
-2.57514924628002	-2.79094870596757\\
-2.65643093499139	-2.72275379750876\\
-2.73259144246289	-2.65013860775138\\
-2.80388993994359	-2.57294527616556\\
-2.87050645497965	-2.4909392638613\\
-2.93253942459908	-2.40381089613441\\
-2.99000128898612	-2.31117553424823\\
-3.04281182340003	-2.21257276418519\\
-3.09078891113382	-2.10746508255407\\
-3.13363647922882	-1.99523671409207\\
-3.17092937706462	-1.87519343226531\\
-3.20209511295751	-1.74656459867964\\
-3.22639262956776	-1.60850911330203\\
-3.24288877161496	-1.46012759407702\\
-3.25043388083961	-1.30048387741061\\
-3.24763916545357	-1.12863979540267\\
-3.232860253651	-0.943707989005588\\
-3.20419371017733	-0.744927940467152\\
-3.15949614900126	-0.531769892733092\\
-3.09643841910547	-0.304069017672216\\
-3.01260905935427	-0.062187036124715\\
-2.90567988916881	0.192810438976774\\
-2.77363964444937	0.458985248496227\\
-2.61508666850671	0.733405387358675\\
-2.4295487824329	1.01212011812496\\
-2.21777243410176	1.29027878622215\\
-1.98190506393249	1.56241892651132\\
-1.72549842501278	1.82290682446117\\
-1.45329498813935	2.06646233416912\\
-1.17081808859522	2.28866210291941\\
-0.883846036343302	2.48631281300606\\
-0.597883133052734	2.65762344314827\\
-0.317731837641074	2.8021658944279\\
-0.0472280991005781	2.92066872817214\\
0.210850791495053	3.01471826027117\\
0.454737366582832	3.08644089581483\\
0.683534623061324	3.13822035187157\\
0.89703583690876	3.17247731940366\\
1.09554128372098	3.19151731174063\\
1.27969473451899	3.19743891413949\\
};
\addplot [color=mycolor1, forget plot]
  table[row sep=crcr]{%
1.3392619362044	3.13288106114262\\
1.5086791795128	3.12757437885239\\
1.6648604132956	3.11272344007181\\
1.80894084011687	3.08972360883537\\
1.94202280659676	3.05971330019967\\
2.06514290229219	3.02360320058587\\
2.1792535778672	2.98210580929765\\
2.28521452921726	2.93576279359985\\
2.3837902847826	2.88496894959092\\
2.47565142870222	2.82999235401166\\
2.56137766034903	2.77099073935933\\
2.64146145066806	2.70802435097411\\
2.71631144520293	2.6410656385359\\
2.78625502182879	2.5700061549625\\
2.85153957186433	2.49466102162765\\
2.91233216336754	2.41477129524368\\
2.96871728581953	2.33000455594079\\
3.02069238245767	2.23995404162502\\
3.06816086448863	2.14413669371357\\
3.11092228510539	2.04199056894032\\
3.14865934933596	1.93287223022889\\
3.18092147510529	1.81605498124403\\
3.2071047420893	1.6907291845516\\
3.22642832936629	1.55600643641962\\
3.23790804050797	1.41093009280062\\
3.24032837021464	1.2544955614675\\
3.23221593702453	1.08568485209026\\
3.21181915823305	0.903520955896151\\
3.17710188324506	0.707148356743944\\
3.12576223284096	0.495945692402419\\
3.05529155810392	0.269674225576345\\
2.96309086907791	0.0286599408519614\\
2.84666081340954	-0.226003578863136\\
2.70387285481623	-0.492260043094357\\
2.5333104918282	-0.766949791157519\\
2.33464000483891	-1.0457837492775\\
2.10893700227236	-1.3234898634351\\
1.85887326877483	-1.59416127410092\\
1.58867727353884	-1.85178014035832\\
1.30383136652625	-2.09082616151757\\
1.01054697329997	-2.30683587515119\\
0.715131634073073	-2.49678465263894\\
0.423391948519504	-2.6592193276381\\
0.140191149224581	-2.79414784538079\\
-0.13078263680191	-2.90275563289121\\
-0.387047456258826	-2.98704378778889\\
-0.627211121354136	-3.04947311479465\\
-0.850773576984367	-3.09266743431044\\
-1.05790511290341	-3.11919740795147\\
-1.24923868994443	-3.13144278304647\\
-1.42569600205082	-3.13151872715371\\
-1.58835308692518	-3.12124838514936\\
-1.73834309104347	-3.10216529525421\\
-1.87679002952083	-3.0755328613566\\
-2.00476644935078	-3.04237188331492\\
-2.12326846130743	-3.00349035732629\\
-2.23320275723049	-2.95951214245916\\
-2.33538147355553	-2.91090271021194\\
-2.43052186422163	-2.85799121830943\\
-2.51924862748709	-2.80098875199537\\
-2.60209739095236	-2.74000289964855\\
-2.67951832872522	-2.67504898055312\\
-2.75187920382206	-2.60605829388496\\
-2.81946733449019	-2.53288375738476\\
-2.88249010557909	-2.45530328282913\\
-2.94107370907848	-2.3730212138732\\
-2.9952598198123	-2.28566814510265\\
-3.04499990804278	-2.19279946220562\\
-3.09014687456381	-2.09389300588201\\
-3.13044368203036	-1.98834638334865\\
-3.16550867093336	-1.87547465280863\\
-3.19481732271165	-1.75450941559799\\
-3.21768041602521	-1.62460080063886\\
-3.23321888992378	-1.48482445001928\\
-3.24033638637087	-1.33419643647127\\
-3.23769153638757	-1.17170005268329\\
-3.22367374567674	-0.996329519905232\\
-3.196388676233	-0.807156629135423\\
-3.15366283961751	-0.603426658440192\\
-3.09308042758429	-0.384688747459116\\
-3.01206879249482	-0.150961996233866\\
-2.90804995608205	0.0970695671635621\\
-2.7786711381308	0.357853290762011\\
-2.62211397341698	0.628793108926508\\
-2.43745768365259	0.906158296836139\\
-2.22503879901498	1.1851375243136\\
-1.98672015936148	1.46008544100662\\
-1.72597326286571	1.72496609922759\\
-1.44770719319412	1.97393409200938\\
-1.15784449555035	2.20193615907363\\
-0.862724864969265	2.40519530789025\\
-0.5684724898963	2.58147229590912\\
-0.280464895163095	2.7300712736549\\
-0.0029932945470816	2.85163151150354\\
0.260864143836346	2.94779278933954\\
0.509191499615886	3.02082769660696\\
0.741070000077793	3.07331076844402\\
0.956362264850219	3.10786112278658\\
1.15549403945334	3.12696676719541\\
1.3392619362044	3.13288106114262\\
};
\addplot [color=mycolor1, forget plot]
  table[row sep=crcr]{%
1.4054745831124	3.07149337387774\\
1.57362727100623	3.06623132298505\\
1.72778699106429	3.05157706491179\\
1.8692516647565	3.02899874981702\\
1.99926321162977	2.99968435119239\\
2.11897510688696	2.96457699823885\\
2.22943611038759	2.92440967730864\\
2.33158448622435	2.87973669499959\\
2.42624859489868	2.83096077860544\\
2.51415099035098	2.77835555376001\\
2.59591408402611	2.72208359692615\\
2.67206609429429	2.66221046842641\\
2.74304644182516	2.59871519613369\\
2.80921003520959	2.53149766884962\\
2.87083006084297	2.46038335395687\\
2.92809898034762	2.38512570196443\\
2.98112747194105	2.30540655755974\\
3.02994104686825	2.22083487486481\\
3.07447404219802	2.13094404515007\\
3.1145606499368	2.03518820178653\\
3.14992260504253	1.93293798778814\\
3.1801531438731	1.82347648080454\\
3.20469689494574	1.70599630232067\\
3.22282553140833	1.57959943447058\\
3.23360938613093	1.44330197737494\\
3.23588593502282	1.29604704519222\\
3.22822727176805	1.1367302319388\\
3.20891065145785	0.964243506987811\\
3.17589909545075	0.777544777654567\\
3.12684302710255	0.575761118613226\\
3.05911867262978	0.358332765458058\\
2.96992343272682	0.125200762736179\\
2.85645018416743	-0.122968392686797\\
2.71615742706347	-0.384544218725802\\
2.54713532170004	-0.656725403355817\\
2.34853577548911	-0.935432686045723\\
2.12099170299253	-1.21537911606742\\
1.86691228769943	-1.49037570040727\\
1.59053375100839	-1.75387251044781\\
1.29765092469468	-1.99965166553425\\
0.995049756673487	-2.22251711528282\\
0.689764855128191	-2.41881039084985\\
0.388344336714448	-2.58663740439976\\
0.0962863203254775	-2.72579076860021\\
-0.182266134892198	-2.83744263071393\\
-0.444575305476755	-2.92372607293893\\
-0.689170200651332	-2.98731462456856\\
-0.915605230675081	-3.03107101158091\\
-1.124190568135	-3.05779374085243\\
-1.3157431266952	-3.07005907314908\\
-1.49138202257451	-3.07013998665209\\
-1.65237434032767	-3.05997949315592\\
-1.80002680080641	-3.04119800758308\\
-1.93561465998037	-3.0151192949233\\
-2.06033850655863	-2.98280444559781\\
-2.17530072283779	-2.94508734198385\\
-2.28149506482953	-2.90260795655498\\
-2.37980449691982	-2.85584170843552\\
-2.4710038297239	-2.80512424893919\\
-2.55576479613425	-2.75067168453415\\
-2.6346619863128	-2.69259656245031\\
-2.70817860414535	-2.63092006979291\\
-2.77671136436757	-2.56558091705725\\
-2.84057407142502	-2.4964413451172\\
-2.8999995470633	-2.42329064404795\\
-2.95513963227719	-2.34584652296658\\
-3.00606300125895	-2.26375463621684\\
-3.05275050582436	-2.17658656405072\\
-3.09508773151852	-2.08383657758257\\
-3.13285440499613	-1.9849176037046\\
-3.16571026463317	-1.87915696707727\\
-3.19317702051026	-1.76579275225261\\
-3.21461613019455	-1.64397203747039\\
-3.2292023730239	-1.51275284866754\\
-3.23589372448983	-1.37111251535963\\
-3.23339896779886	-1.217966211236\\
-3.22014603594558	-1.05220081037912\\
-3.19425648770722	-0.872730640045606\\
-3.15353496995275	-0.678582877771494\\
-3.09548696227414	-0.469020438077985\\
-3.01738292443531	-0.243707879537554\\
-2.91639048564559	-0.00291924116312545\\
-2.78979527494392	0.252226179261229\\
-2.63532064441046	0.519533917556674\\
-2.45153225954353	0.795567761858333\\
-2.23827491409184	1.07562197911916\\
-1.99704545399365	1.35390475240838\\
-1.73117936849501	1.62396651620491\\
-1.44574621356274	1.87933276613553\\
-1.14712218591709	2.11421751592803\\
-0.842314357326939	2.32414547803164\\
-0.538198389740975	2.50633192203147\\
-0.240852599292319	2.65975270557451\\
0.0448812356255614	2.78493845106198\\
0.315570224341204	2.88359640994968\\
0.569135749148115	2.9581792577659\\
0.804652925319656	3.01149314457779\\
1.02208728745503	3.04639408810963\\
1.22203103342413	3.06558388836308\\
1.4054745831124	3.07149337387774\\
};
\addplot [color=mycolor1, forget plot]
  table[row sep=crcr]{%
1.47942380987548	3.01409903169868\\
1.64626455039693	3.00888351732135\\
1.79829624689677	2.9944363478465\\
1.93700704846841	2.97230176776496\\
2.06379831826375	2.94371715953913\\
2.1799540989066	2.90965595947843\\
2.28662837119145	2.87086852888969\\
2.38484333686672	2.82791837021459\\
2.4754940038472	2.78121274986893\\
2.55935591460565	2.73102770910772\\
2.63709397460015	2.67752789168202\\
2.70927109620344	2.62078178779733\\
2.77635586641289	2.56077301411612\\
2.83872875041704	2.49740819628071\\
2.89668651596338	2.43052193867583\\
2.95044464676317	2.35987928153766\\
3.0001375352307	2.28517597293728\\
3.04581622485987	2.20603683256147\\
3.08744342398849	2.12201246457777\\
3.12488544666132	2.03257459934331\\
3.15790066486958	1.93711042432436\\
3.18612399696444	1.83491642705397\\
3.20904693798338	1.72519255157135\\
3.22599270701752	1.60703791200326\\
3.23608632413714	1.47944997405672\\
3.23821996159076	1.34133007776124\\
3.23101493250038	1.19149949585151\\
3.21278345253694	1.0287319156817\\
3.1814961642855	0.851810177806806\\
3.13476565527705	0.659616892903152\\
3.06986188544866	0.451269262738913\\
2.9837819155047	0.226306340358796\\
2.87340149427001	-0.0150705104270993\\
2.73573556174386	-0.271718997598551\\
2.56832184339207	-0.54127972878473\\
2.36970959446515	-0.819974718732148\\
2.1399833195045	-1.10257758631277\\
1.88119288598873	-1.38264819930625\\
1.59752959516814	-1.65307042307391\\
1.29512089242035	-1.90682964704465\\
0.981426447742972	-2.13785775176816\\
0.66436393646992	-2.34172200086941\\
0.351394605016278	-2.51598196277505\\
0.0487974547586737	-2.6601626822905\\
-0.23873474837032	-2.77542164655584\\
-0.508170243751502	-2.86405775042834\\
-0.757967507119653	-2.92900735035789\\
-0.98777367530101	-2.9734232879994\\
-1.1980910269562	-3.00037532671187\\
-1.38997469205245	-3.01266848879099\\
-1.56479036970873	-3.01275487347464\\
-1.72403700215334	-3.00270968394413\\
-1.86922667587088	-2.98424595906764\\
-2.00180940524786	-2.95874918876373\\
-2.12313041210999	-2.92731945832026\\
-2.23440947949695	-2.89081381148429\\
-2.33673442944079	-2.84988499437422\\
-2.43106303818853	-2.80501492086142\\
-2.5182295086471	-2.75654245540378\\
-2.59895294951886	-2.70468576332112\\
-2.67384623600738	-2.64955976796782\\
-2.74342424195536	-2.59118933749998\\
-2.80811082315821	-2.52951880014098\\
-2.86824416379404	-2.46441831516597\\
-2.92408022172711	-2.39568754127205\\
-2.97579405823449	-2.32305696396324\\
-3.02347883677504	-2.24618718088355\\
-3.06714223968626	-2.16466640771873\\
-3.10669999303777	-2.07800646707122\\
-3.14196611940757	-1.98563757183635\\
-3.17263947018963	-1.88690233258323\\
-3.19828604522639	-1.78104963392229\\
-3.21831662507747	-1.66722937843925\\
-3.23195938213613	-1.54448964299202\\
-3.2382275020994	-1.41177859804979\\
-3.23588259406218	-1.26795467637459\\
-3.22339602477535	-1.11180998860894\\
-3.198912583081	-0.94211383125201\\
-3.16022439810807	-0.757685076645846\\
-3.10476802094913	-0.557503640031831\\
-3.02966381164673	-0.340870796261394\\
-2.93182298174123	-0.107623705902376\\
-2.80815068243951	0.141601790059955\\
-2.65586777743815	0.405086766923669\\
-2.47295211775567	0.679779419408755\\
-2.25865735059234	0.961166577758489\\
-2.01400900809958	1.24336659938685\\
-1.7421274693308	1.51951626565132\\
-1.44822398926211	1.78244372089397\\
-1.13918912623464	2.02550654974069\\
-0.822829286010876	2.24338606724704\\
-0.506941025352666	2.43262553182077\\
-0.198466686990418	2.59179277046852\\
0.0970790994492219	2.72128431288835\\
0.375842686436026	2.82289348564613\\
0.635569619905188	2.89929725677801\\
0.87535433088247	2.95358556276458\\
1.09531230173884	2.98889935870533\\
1.29625660842899	3.00819221393258\\
1.47942380987548	3.01409903169868\\
};
\addplot [color=mycolor1, forget plot]
  table[row sep=crcr]{%
1.56248802181173	2.9616734001184\\
1.72794160058687	2.95650722867695\\
1.87771037776087	2.94228025725477\\
2.01350416846367	2.9206156108536\\
2.13690554077677	2.89279910679247\\
2.24934335930265	2.85983152504823\\
2.3520855292164	2.8224767991473\\
2.44624293856636	2.781303675091\\
2.5327792614432	2.73672023362352\\
2.61252321755089	2.68900161357266\\
2.68618120369028	2.63831167779904\\
2.75434907197669	2.5847194697673\\
2.81752236361814	2.52821126532338\\
2.87610462006534	2.46869891581417\\
2.93041355954928	2.40602505151122\\
2.98068497658418	2.33996559239673\\
3.02707422747102	2.27022990888279\\
3.06965512668631	2.19645889483955\\
3.10841601017396	2.1182211652924\\
3.14325263091391	2.03500757937128\\
3.17395744815529	1.94622432861669\\
3.20020476631742	1.85118494273159\\
3.22153109376454	1.74910178229258\\
3.23731006382553	1.63907795993805\\
3.24672135907247	1.52010122767342\\
3.24871342237838	1.3910422798955\\
3.24196051612137	1.25066125707407\\
3.22481619765923	1.09762809066295\\
3.19526792680432	0.930564723553199\\
3.15090180614554	0.748119965368382\\
3.08889279489712	0.549090102380487\\
3.00604403733272	0.332598796784509\\
2.89890774126572	0.0983453980010447\\
2.76402519326072	-0.153082636643624\\
2.59831705952781	-0.419864091461348\\
2.39962622519607	-0.698635848076194\\
2.16735629533455	-0.98433558391937\\
1.90306755215678	-1.27032769523082\\
1.6108251747217	-1.54890500107748\\
1.29710068781251	-1.81214296080079\\
0.970146867704863	-2.0529270739582\\
0.638965572146003	-2.26586713629577\\
0.312157381743099	-2.44783584723404\\
-0.00302571524577319	-2.59802067678299\\
-0.301220094054039	-2.71756300043051\\
-0.579056423883941	-2.80897290177942\\
-0.8349411321204	-2.87551534463542\\
-1.06867026275776	-2.92069886383828\\
-1.28100962445009	-2.94791844465975\\
-1.47332349419031	-2.96024657501622\\
-1.6472859563713	-2.96033897061529\\
-1.80467722973043	-2.95041637068623\\
-1.94725182593645	-2.93228999711988\\
-2.07666084446663	-2.90740770495249\\
-2.19441188468987	-2.87690642664304\\
-2.30185338601962	-2.84166287301235\\
-2.40017378689372	-2.80233862274819\\
-2.4904089277545	-2.75941821344266\\
-2.57345341291102	-2.7132401876885\\
-2.6500732538513	-2.66402168351149\\
-2.72091818762069	-2.61187739006263\\
-2.78653274547894	-2.55683370715729\\
-2.84736555949864	-2.49883886357561\\
-2.90377662674676	-2.43776962713736\\
-2.95604236355122	-2.37343511295737\\
-3.00435831666581	-2.30557808202369\\
-3.04883937980577	-2.23387402918455\\
-3.08951730923333	-2.15792829371994\\
-3.12633525131421	-2.07727139352432\\
-3.15913889648398	-1.99135279573921\\
-3.18766376747544	-1.89953340913903\\
-3.21151805059608	-1.80107724325244\\
-3.23016031570991	-1.69514296658646\\
-3.24287149488152	-1.58077657029018\\
-3.24872069156883	-1.45690708487812\\
-3.24652492274259	-1.32234840752889\\
-3.23480399458236	-1.17581188348314\\
-3.21173373099997	-1.01593641696128\\
-3.17510417730993	-0.841345494965563\\
-3.12229467922138	-0.650743181776849\\
-3.05028512889501	-0.443062800463219\\
-2.95573153358211	-0.217680470676936\\
-2.835141774389	0.0253026699225968\\
-2.68518801965275	0.284725337264344\\
-2.50317597893971	0.558027535137331\\
-2.28764751658583	0.841001509445815\\
-2.03902032112432	1.12776023726286\\
-1.76008767943735	1.41104522436188\\
-1.45616345687017	1.6829170560622\\
-1.13471863967304	1.93572767581246\\
-0.804524934801826	2.16312908782645\\
-0.474521414348072	2.36082533570033\\
-0.152730019317042	2.52686942081269\\
0.154502462670652	2.66148988749968\\
0.44282014922353	2.76659143926723\\
0.709783032080508	2.84513401290853\\
0.954545295336056	2.90055901688932\\
1.17743860785017	2.93635299009173\\
1.37956959771753	2.95576769492973\\
1.56248802181173	2.9616734001184\\
};
\addplot [color=mycolor1, forget plot]
  table[row sep=crcr]{%
1.65642263207936	2.91541373390173\\
1.82037784583778	2.91030086390921\\
1.9677141607887	2.89631048846228\\
2.10039858637834	2.87514661502634\\
2.22021829502882	2.84814149164205\\
2.32876139289674	2.81631933318551\\
2.42741778412894	2.78045311346391\\
2.51739080453339	2.74111237919629\\
2.5997137273383	2.69870202037562\\
2.67526759617419	2.65349283915425\\
2.74479836712207	2.60564506995236\\
2.80893228619176	2.55522600914459\\
2.86818898383772	2.50222278100859\\
2.92299207190501	2.44655108756695\\
2.97367717316179	2.38806060833533\\
3.02049735848801	2.32653755200114\\
3.06362594816039	2.26170472357171\\
3.1031565727605	2.19321936030627\\
3.13910029784126	2.12066890980134\\
3.17137950106921	2.04356487823619\\
3.19981805563009	1.96133487528602\\
3.22412722545868	1.87331304194841\\
3.24388652917448	1.77872919811096\\
3.25851870767061	1.67669733555935\\
3.26725788924465	1.56620458170098\\
3.26911018723029	1.44610257455019\\
3.26280646739801	1.3151044624267\\
3.24674818526298	1.17179264275761\\
3.21894948375011	1.01464503971421\\
3.17698281835873	0.842091215225842\\
3.11794201723986	0.652613547633571\\
3.03844644930004	0.444911865172639\\
2.93472234736788	0.218149481119258\\
2.8028090465519	-0.0277104952854724\\
2.63894062157472	-0.291494298734732\\
2.44013251083137	-0.570393290038135\\
2.20494150359352	-0.859649139467993\\
1.93426251361171	-1.1525223052927\\
1.63191099575297	-1.44070778788784\\
1.30469689836056	-1.71524417896233\\
0.961809545903871	-1.96775111774133\\
0.613598070559528	-2.19163828849998\\
0.27010603400576	-2.3829008079898\\
-0.0601842625425439	-2.54029310737039\\
-0.371081514574717	-2.66493929156151\\
-0.658830666399525	-2.75962282515189\\
-0.921821623365127	-2.82802513033402\\
-1.16008213079778	-2.87409551910905\\
-1.37473982056242	-2.90162188662688\\
-1.56756118680114	-2.91399083853101\\
-1.74060641809601	-2.91408981741843\\
-1.89599662931821	-2.90429937494904\\
-2.03577159460937	-2.88653403208306\\
-2.16181254714747	-2.86230368386328\\
-2.27580800711821	-2.83277894819267\\
-2.37924602595903	-2.79885186349478\\
-2.47342135895941	-2.76118829145959\\
-2.55945010266753	-2.72027114215685\\
-2.63828719565115	-2.67643490915497\\
-2.71074408814828	-2.6298925648727\\
-2.77750509407517	-2.58075599703199\\
-2.83914166821686	-2.52905108890512\\
-2.89612426683683	-2.47472838286705\\
-2.94883166501475	-2.41767008273216\\
-2.99755769310528	-2.35769397601212\\
-3.04251536440935	-2.2945547054575\\
-3.08383832442057	-2.22794269467252\\
-3.12157947480033	-2.15748093717595\\
-3.15570652122286	-2.08271979484268\\
-3.18609406848149	-2.00312992664664\\
-3.21251174361749	-1.91809349492053\\
-3.23460767710368	-1.82689389684362\\
-3.25188653209408	-1.72870448128615\\
-3.26368118112669	-1.62257709444266\\
-3.26911716301003	-1.50743193969673\\
-3.2670693455275	-1.38205126127052\\
-3.2561110053532	-1.24508092609879\\
-3.23445719275566	-1.09504625615185\\
-3.19990733869133	-0.930391566459729\\
-3.14979732421527	-0.749556666920958\\
-3.08097940364488	-0.551107366131702\\
-2.98985962964534	-0.333938851596171\\
-2.87253514113668	-0.0975668422736816\\
-2.72508233437513	0.15749452962294\\
-2.54404001527801	0.429303881850829\\
-2.32709234919699	0.714103779021441\\
-2.07387200165371	1.00612432375832\\
-1.78668666913353	1.29775933191409\\
-1.47088122335579	1.58023484821087\\
-1.1345756241589	1.84471730760218\\
-0.787717671493854	2.083588282538\\
-0.440683840723204	2.2914878559795\\
-0.102865668160094	2.46580854778831\\
0.218346993157529	2.60656514302512\\
0.517995089490967	2.71580904103266\\
0.793451069451778	2.79686253352453\\
1.04399255122457	2.85360765026289\\
1.27026126265922	2.88995393049057\\
1.47375493321376	2.90950846059762\\
1.65642263207936	2.91541373390173\\
};
\addplot [color=mycolor1, forget plot]
  table[row sep=crcr]{%
1.76348601971647	2.87683329369742\\
1.92578633299117	2.87177911807686\\
2.07047954639434	2.85804562529419\\
2.19982936085868	2.837418579363\\
2.3158520581634	2.81127338959993\\
2.42030894180586	2.78065273973874\\
2.51471878827486	2.74633336758708\\
2.60037969658333	2.70888073371149\\
2.67839404839091	2.66869233009578\\
2.74969308449871	2.62603116006154\\
2.81505931225171	2.58105106779765\\
2.87514595113622	2.53381545319279\\
2.93049316546575	2.48431065802551\\
2.98154109959006	2.43245504247299\\
3.02863983365929	2.3781045253445\\
3.07205638379692	2.32105515081392\\
3.11197881832259	2.26104307036364\\
3.14851747228343	2.19774218859202\\
3.18170312587335	2.13075961284405\\
3.21148187177012	2.05962896925004\\
3.23770623227883	1.98380160652951\\
3.2601218996854	1.90263571639291\\
3.27834926667941	1.8153834799699\\
3.2918587030151	1.7211765459532\\
3.29993835675765	1.61901052787068\\
3.30165319257391	1.5077298845134\\
3.29579418014286	1.38601568238664\\
3.28081729572738	1.25238055969162\\
3.25477378510873	1.10517799928342\\
3.21523672977476	0.94263703848744\\
3.15923545962907	0.762938836079086\\
3.08322002447145	0.56435741322315\\
2.98309348302936	0.345491101696972\\
2.85436853747411	0.105608547675051\\
2.692519632201	-0.154885635179985\\
2.49359485450611	-0.433906731071035\\
2.25509698278377	-0.727187237009924\\
1.97701804067147	-1.02802696071195\\
1.66273686318081	-1.32754885415733\\
1.31936433370759	-1.61561712920537\\
0.957197314032151	-1.88230756971129\\
0.588281686375872	-2.1195031919381\\
0.224519787448563	-2.32205758013215\\
-0.124024317679309	-2.4881594718107\\
-0.450126094146457	-2.61891568901138\\
-0.749594467833043	-2.71747033472558\\
-1.02086678096851	-2.78804076186645\\
-1.26432516705138	-2.83512893618276\\
-1.48159516891929	-2.86300127331954\\
-1.67496794109156	-2.87541488118393\\
-1.84698734463292	-2.87552103698405\\
-2.00018683612315	-2.86587509023157\\
-2.13694034232468	-2.84849917106608\\
-2.25939062132157	-2.82496362814055\\
-2.36942584813439	-2.79646841722182\\
-2.46868372951739	-2.7639156604951\\
-2.55856966932492	-2.72797036453293\\
-2.64028075426689	-2.68910924597426\\
-2.71483083160932	-2.64765891538533\\
-2.78307414907135	-2.60382507704837\\
-2.84572633842907	-2.55771437365284\\
-2.90338226450221	-2.50935029296217\\
-2.95653064787545	-2.45868428788424\\
-3.00556554356136	-2.40560300220309\\
-3.05079480570215	-2.34993226542908\\
-3.09244564168406	-2.29143832806084\\
-3.13066728644761	-2.22982665188924\\
-3.16553072400175	-2.1647384458182\\
-3.19702525437404	-2.09574504440137\\
-3.22505155192867	-2.02234016598363\\
-3.24941068500344	-1.94393006826753\\
-3.26978836880001	-1.85982165915236\\
-3.28573351194609	-1.76920875235269\\
-3.29662991623451	-1.67115693541895\\
-3.30165985357059	-1.56458803023886\\
-3.29975828568289	-1.44826600751467\\
-3.28955692138947	-1.32078766195834\\
-3.26931849695565	-1.18058362230308\\
-3.23686423711171	-1.02593864805765\\
-3.18950234636825	-0.85504484502446\\
-3.12397381588454	-0.666107183445228\\
-3.03644493809001	-0.457526194514793\\
-2.92259354736716	-0.228184376895331\\
-2.77785413884563	0.0221464900115145\\
-2.59789378652418	0.292291090070833\\
-2.37936353476511	0.579126249471295\\
-2.12088114755301	0.877173423796475\\
-1.8240442075674	1.17857189841392\\
-1.49410500651416	1.47365939389888\\
-1.13989674277931	1.75220125656733\\
-0.772813421464587	2.00499206780641\\
-0.405068595964772	2.22530026810592\\
-0.0478214785723391	2.40965525281262\\
0.290210933590949	2.55779532341061\\
0.603341316210892	2.67196929045974\\
0.888768223076226	2.75597143720336\\
1.14599270169202	2.81424369783125\\
1.37610047475835	2.85121851829582\\
1.58111190506633	2.870929013894\\
1.76348601971647	2.87683329369742\\
};
\addplot [color=mycolor1, forget plot]
  table[row sep=crcr]{%
1.88661887654611	2.84789220388966\\
2.04705214990411	2.8429038537427\\
2.18884443214041	2.82945197986441\\
2.31459898637225	2.80940341620734\\
2.42658588194401	2.78417194600725\\
2.52675220198436	2.75481261074247\\
2.61675052305756	2.72209991644837\\
2.69797402742826	2.68658999318215\\
2.77159191642958	2.64866863606242\\
2.83858199221868	2.60858767046551\\
2.89975909593142	2.56649197280371\\
2.95579906312272	2.52243912845706\\
3.00725833086018	2.47641330663042\\
3.05458952043593	2.42833455819933\\
3.0981533527615	2.37806442389918\\
3.13822720222123	2.32540847912831\\
3.17501049763042	2.27011623126171\\
3.20862705450401	2.21187861648089\\
3.23912427775346	2.15032320768242\\
3.26646900783157	2.08500713814001\\
3.29053959190897	2.0154076671603\\
3.31111353902201	1.94091027094747\\
3.32784985983523	1.86079415202803\\
3.34026489971346	1.77421515866877\\
3.34770016613014	1.68018635367781\\
3.34928037988859	1.57755697444958\\
3.34385986297503	1.46499145273588\\
3.32995565774689	1.34095177579936\\
3.30566691605795	1.20368914956853\\
3.268582925055	1.05125515563213\\
3.21568798943698	0.881548851538667\\
3.14328218917604	0.692424616921975\\
3.04695489446798	0.481894709881565\\
2.92167355656706	0.248465993719188\\
2.7620789756982	-0.00835812346349713\\
2.56309271743613	-0.287419179604642\\
2.32090628856909	-0.585186685965562\\
2.03428782115633	-0.895217074015581\\
1.70589780178838	-1.20814309277256\\
1.34305318204809	-1.51251547659411\\
0.957358301848464	-1.79651289090998\\
0.563027461849464	-2.05004471948193\\
0.17440368490501	-2.2664497757554\\
-0.196376909748047	-2.44316293361382\\
-0.540783844466607	-2.58127675633976\\
-0.854143981342034	-2.68442154501285\\
-1.13505368197173	-2.75751618959368\\
-1.38443246395149	-2.80576439146061\\
-1.6045931848711	-2.83402015795313\\
-1.79851286989534	-2.84647927607051\\
-1.9693409729063	-2.8465932033257\\
-2.1201079802089	-2.8371073567179\\
-2.25357672915568	-2.8201544840852\\
-2.37218440879142	-2.79736219431427\\
-2.47803682426109	-2.76995408295331\\
-2.57292963265337	-2.73883618320411\\
-2.65838120293001	-2.70466698118174\\
-2.73566843032449	-2.66791222186143\\
-2.80586098660883	-2.62888680496751\\
-2.8698519182274	-2.58778620568981\\
-2.92838384723987	-2.54470959153192\\
-2.98207071911846	-2.49967641573398\\
-3.03141535246996	-2.45263787417719\\
-3.07682314525913	-2.40348426509281\\
-3.11861227649762	-2.35204900150335\\
-3.15702066482802	-2.29810979171628\\
-3.19220983321692	-2.24138731469277\\
-3.2242656938994	-2.18154156579224\\
-3.253196112476	-2.11816592758243\\
-3.27892493199109	-2.05077892738059\\
-3.3012819315108	-1.97881358102903\\
-3.31998795355617	-1.90160420286395\\
-3.33463415867616	-1.81837061019379\\
-3.34465406183439	-1.72819981365247\\
-3.34928670602053	-1.63002564346745\\
-3.34752911347362	-1.52260744991088\\
-3.33807619606171	-1.40451024639017\\
-3.31924694264404	-1.27409075461728\\
-3.28889755562535	-1.12949720141945\\
-3.24432634438246	-0.968695913179166\\
-3.18218323873547	-0.789545104072829\\
-3.09841088038382	-0.589945317049548\\
-2.98826611939397	-0.368104162226417\\
-2.84649908086911	-0.122953273533075\\
-2.66779180783707	0.145266116939086\\
-2.4475534338266	0.434295408854172\\
-2.18308925931526	0.739191127217544\\
-1.87496839310639	1.05200132842453\\
-1.52814377342763	1.36215315008865\\
-1.15220751368633	1.65775617088943\\
-0.760348434732937	1.9275974139927\\
-0.367169777737965	2.16314400211131\\
0.0138428598720332	2.3597740592165\\
0.37225558768193	2.51686213953497\\
0.701498660578425	2.63692954475243\\
0.998640491179626	2.72439733159422\\
1.26356280992866	2.78442955309\\
1.4979877098556	2.82211183597127\\
1.70463610269356	2.84199118101627\\
1.88661887654611	2.84789220388966\\
};
\addplot [color=mycolor1, forget plot]
  table[row sep=crcr]{%
2.02970341638092	2.83118406656601\\
2.18799151583129	2.82627070826438\\
2.32657319718337	2.81313005572053\\
2.4484361969554	2.79370724827128\\
2.55612775499061	2.76944789457504\\
2.65179070273404	2.74141210242071\\
2.73721348154227	2.71036551716844\\
2.81388206297824	2.67684947612192\\
2.88302799853583	2.6412338361512\\
2.94567029440224	2.60375608343852\\
3.0026506013816	2.56454984358704\\
3.05466205328004	2.52366528406634\\
3.10227241407705	2.48108331006773\\
3.14594225428358	2.43672495627237\\
3.18603880658966	2.39045697724693\\
3.22284602136534	2.34209432522433\\
3.2565711878713	2.29139995779901\\
3.28734832050702	2.23808222232966\\
3.31523833276256	2.18178990450667\\
3.34022582962197	2.12210489554508\\
3.36221213263002	2.05853232050682\\
3.38100389863481	1.99048788022667\\
3.39629639027895	1.91728210122766\\
3.40765009260943	1.83810118658649\\
3.41445894347145	1.75198426532106\\
3.41590797658147	1.65779713944551\\
3.41091773876497	1.55420328624735\\
3.39807261966851	1.43963415894118\\
3.37553062049951	1.31226317838362\\
3.34091387918683	1.16999188138802\\
3.29118392127937	1.01046337054864\\
3.22251556622729	0.831128393084588\\
3.13020222573058	0.629403191943986\\
3.00865666611958	0.402973082376185\\
2.85161512835336	0.150302156920767\\
2.65269571955597	-0.128613728236073\\
2.40646243195864	-0.431300484550089\\
2.11002654386438	-0.751892842889294\\
1.7649061071668	-1.08070904660359\\
1.37842750844447	-1.40486610785543\\
0.963726805587811	-1.71019729738016\\
0.537833723209402	-1.98401574313531\\
0.118366070380709	-2.21760513799863\\
-0.279767988968165	-2.40737349899909\\
-0.646366232116422	-2.55440928108333\\
-0.976246163067648	-2.66301463115588\\
-1.26835500452852	-2.73904412880406\\
-1.52442446289068	-2.78860444565441\\
-1.74771927502018	-2.81727691382763\\
-1.94211413490662	-2.82977819743403\\
-2.11151509006866	-2.8299004587552\\
-2.2595483536233	-2.82059402181255\\
-2.38942531946137	-2.80410331288318\\
-2.50391055697638	-2.78210802442187\\
-2.60534333966609	-2.75584821494047\\
-2.69568274109955	-2.72622676262518\\
-2.77655963538476	-2.69388956052156\\
-2.84932713683109	-2.65928653977993\\
-2.91510570900307	-2.62271721150604\\
-2.97482169203059	-2.5843641256496\\
-3.02923924618228	-2.54431705456848\\
-3.07898625410271	-2.50259008809168\\
-3.12457489493291	-2.45913327921625\\
-3.16641758628866	-2.4138400312764\\
-3.20483888407323	-2.36655106269949\\
-3.24008378529377	-2.31705550766154\\
-3.27232271764382	-2.26508949195961\\
-3.30165332831405	-2.2103323474067\\
-3.32809900115148	-2.1524004828699\\
-3.35160382843044	-2.09083880791846\\
-3.37202353024845	-2.02510950346722\\
-3.38911153809806	-1.95457785774221\\
-3.40249912689535	-1.87849485180717\\
-3.41166808383927	-1.79597622221363\\
-3.41591394977336	-1.70597791654738\\
-3.41429740138974	-1.60726831190391\\
-3.40558097926308	-1.49839849557875\\
-3.38814838024225	-1.37767365741986\\
-3.35990448915265	-1.24313175674567\\
-3.31815730919366	-1.09254088240715\\
-3.25948987861662	-0.923435062714006\\
-3.17964414953854	-0.733220371008295\\
-3.07346350523359	-0.519398117047068\\
-2.93497869715061	-0.279964343057266\\
-2.75776845829473	-0.0140400219106444\\
-2.53575536604124	0.277264088159463\\
-2.26454919752634	0.5898750285627\\
-1.94323913720542	0.916019429230686\\
-1.5761386446494	1.24425583883913\\
-1.17359697122873	1.56074624024008\\
-0.751049416477331	1.85170522131172\\
-0.326269606485461	2.10618551936316\\
0.0841269569430639	2.31799429981231\\
0.467444213254643	2.48601907070544\\
0.816043340955449	2.61316846626642\\
1.12696299533261	2.70471399004408\\
1.40071392257016	2.76676605369003\\
1.63993250907036	2.80523492133321\\
1.84828108699536	2.82529081769113\\
2.02970341638092	2.83118406656601\\
};
\addplot [color=mycolor1, forget plot]
  table[row sep=crcr]{%
2.19794348574178	2.83020791239487\\
2.35373363816799	2.82538097677279\\
2.48874200304993	2.81258609601171\\
2.60638535099903	2.79384130177606\\
2.70950805314651	2.77061553326634\\
2.80045446322886	2.74396555105998\\
2.88114703858336	2.71464105191456\\
2.9531589992383	2.68316310588763\\
3.01777725847028	2.64988163896509\\
3.07605480264875	2.61501701171813\\
3.1288532349893	2.57868972809118\\
3.17687675042674	2.54094132969035\\
3.22069888491605	2.50174871427288\\
3.26078324797071	2.46103347785424\\
3.29749923425388	2.41866739360158\\
3.33113347916659	2.37447477349825\\
3.36189759804843	2.32823217865962\\
3.38993253319128	2.27966572467875\\
3.41530962187418	2.22844604886074\\
3.4380282810242	2.17418085132003\\
3.45800996505544	2.11640478115467\\
3.47508777476958	2.0545663069574\\
3.48899075557473	1.98801108946729\\
3.49932149906526	1.91596127624999\\
3.50552513068025	1.83749009698622\\
3.50684711571555	1.75149122253605\\
3.50227656741248	1.65664269587682\\
3.49047099763168	1.55136609748902\\
3.46965800078646	1.43378341593225\\
3.437509876863	1.30167761211687\\
3.39099009386595	1.15246929499121\\
3.32617849345752	0.983232960249207\\
3.23809999368771	0.790793683555821\\
3.1206161475555	0.571969256244199\\
2.96649683732306	0.324047654852989\\
2.76786675583152	0.0455943566014359\\
2.51727988560229	-0.262379028283973\\
2.2096098960078	-0.595051692367061\\
1.84459494234405	-0.942756585655447\\
1.42919527126162	-1.29111794491241\\
0.978307570666708	-1.62306038301653\\
0.512678782781185	-1.92241778446513\\
0.0544287312810153	-2.17761606270126\\
-0.377737554718123	-2.38363051745455\\
-0.771451366904411	-2.54157158742407\\
-1.12104353606049	-2.65669577496405\\
-1.42614017964794	-2.73613114161499\\
-1.68970031090223	-2.78716214054464\\
-1.91631039667646	-2.81627692693882\\
-2.11101924854228	-2.82881137685641\\
-2.27867372333326	-2.82894245377662\\
-2.42360785275526	-2.81983869614429\\
-2.54954389456128	-2.80385452952917\\
-2.65960446621492	-2.78271420765073\\
-2.75637414654071	-2.75766555114021\\
-2.84197675809744	-2.72960045336035\\
-2.91815161790832	-2.69914593213989\\
-2.98632159096612	-2.66673140099171\\
-3.04765072082669	-2.63263763015077\\
-3.10309155195078	-2.59703195032721\\
-3.15342321866503	-2.5599932276488\\
-3.19928164481707	-2.52152923413244\\
-3.24118314601982	-2.48158831164077\\
-3.27954254164904	-2.44006666857769\\
-3.31468665714298	-2.39681222612689\\
-3.34686386761021	-2.35162561087716\\
-3.37625011359407	-2.30425864380888\\
-3.40295160773395	-2.25441047833786\\
-3.42700423840378	-2.20172137452279\\
-3.44836945014586	-2.14576394973124\\
-3.46692612433319	-2.08603160999408\\
-3.48245767719466	-2.02192373879323\\
-3.49463321342869	-1.95272710773176\\
-3.50298109834676	-1.87759289802314\\
-3.50685272079441	-1.79550873151772\\
-3.50537351343399	-1.70526530223622\\
-3.49737752988289	-1.605417758054\\
-3.48132122789405	-1.49424324164873\\
-3.45517202082913	-1.3696985434041\\
-3.41626862149497	-1.22938661008532\\
-3.36115520018392	-1.07054914816206\\
-3.28540356604819	-0.890116606399041\\
-3.18346285077177	-0.684867785709255\\
-3.04862189585315	-0.451777235719651\\
-2.87323907188057	-0.188647424477684\\
-2.64947050202814	0.104898760550868\\
-2.37074235958796	0.426112149877728\\
-2.03402508173049	0.76782684360933\\
-1.64243859568701	1.11789622291151\\
-1.20698322528366	1.46022099380214\\
-0.745924559696358	1.77767795717415\\
-0.281335260344847	2.0560103396237\\
0.165852337997601	2.28682675361663\\
0.579901420918831	2.46835078353251\\
0.951887237654244	2.60406024107767\\
1.27902455765975	2.70040825839983\\
1.56284739585687	2.76476655539025\\
1.80730949491724	2.80409733637165\\
2.01733982975625	2.8243298222766\\
2.19794348574178	2.83020791239487\\
};
\addplot [color=mycolor1, forget plot]
  table[row sep=crcr]{%
2.39842872978907	2.84977139256201\\
2.55129038431519	2.84504461674373\\
2.68231539555117	2.83263442291828\\
2.79539001740285	2.81462313009287\\
2.89366946536777	2.79249252018157\\
2.97969935245457	2.76728667816391\\
3.05552932001567	2.73973209652949\\
3.12281051005268	2.71032437359527\\
3.18287548552932	2.67938999343328\\
3.23680211219273	2.64712995242337\\
3.28546384805857	2.61365029122884\\
3.3295689325236	2.57898318346077\\
3.36969066335952	2.54310115785129\\
3.40629054629358	2.50592623984897\\
3.43973570423691	2.46733522432841\\
3.47031157804237	2.42716187302171\\
3.49823064269921	2.3851965200136\\
3.523637592907	2.34118332956168\\
3.5466112048444	2.29481525526136\\
3.56716283849509	2.2457265776743\\
3.58523128640436	2.19348273351238\\
3.60067337611301	2.13756698200104\\
3.61324936578264	2.07736327643932\\
3.62260170039184	2.01213452177731\\
3.62822507741097	1.94099521647983\\
3.62942496089109	1.86287734047253\\
3.62526064607836	1.77648835582449\\
3.61446772309498	1.68026053263946\\
3.59535345920811	1.57229189592498\\
3.56565767806044	1.45028166408724\\
3.5223723437118	1.31146848252927\\
3.46151792285882	1.15259035574255\\
3.37788913343578	0.969904397791377\\
3.26481670291628	0.759336217353002\\
3.11405964046072	0.51687337923087\\
2.91605591056187	0.239360678696166\\
2.66089845199008	-0.074154951409486\\
2.34045290561423	-0.420557366703982\\
1.95172262498189	-0.790768853476641\\
1.50061335174333	-1.16900621117368\\
1.00396182899994	-1.53459576559659\\
0.487507470558486	-1.86661795835332\\
-0.0202704812606559	-2.14941502129083\\
-0.495328724565367	-2.37591010083617\\
-0.922461096163118	-2.54729676726565\\
-1.29564998398071	-2.67022881755567\\
-1.61576083610479	-2.75360467736886\\
-1.88760925020119	-2.80626531035516\\
-2.11761850849402	-2.83583585301951\\
-2.31236767058961	-2.84838727307023\\
-2.47786376359186	-2.848527498087\\
-2.61927104117742	-2.83965348790494\\
-2.74088359664343	-2.82422436669264\\
-2.84620715031017	-2.80399881931045\\
-2.93807710626956	-2.78022231052697\\
-3.01877781024656	-2.75376742051602\\
-3.09014858164517	-2.72523609470473\\
-3.15367233792879	-2.69503292377865\\
-3.21054721822161	-2.6634171134397\\
-3.26174334772701	-2.63053902736006\\
-3.30804728020505	-2.59646561565443\\
-3.35009648208242	-2.56119780348917\\
-3.38840584828985	-2.52468198992928\\
-3.42338783163099	-2.4868171327911\\
-3.45536738907869	-2.44745840594938\\
-3.48459261736701	-2.40641805663432\\
-3.51124166291223	-2.3634638195184\\
-3.53542623466593	-2.3183150301698\\
-3.55719180613415	-2.27063639909317\\
-3.57651434509614	-2.22002924128097\\
-3.59329313427756	-2.16601979160752\\
-3.607338917041	-2.10804406434049\\
-3.61835618706849	-2.04542853193148\\
-3.62591790130348	-1.97736571032573\\
-3.62943018671176	-1.90288357081217\\
-3.62808369083953	-1.82080761576549\\
-3.62078707436777	-1.72971459653246\\
-3.60607682103516	-1.6278774991293\\
-3.58199631503049	-1.51320312714434\\
-3.5459367630778	-1.38316739339852\\
-3.49443481646271	-1.2347610883137\\
-3.42293053187514	-1.0644733062168\\
-3.32551203519009	-0.868364747381815\\
-3.19472224053972	-0.642321746624063\\
-3.02159306783709	-0.38262930064068\\
-2.79620535592164	-0.0870280333253533\\
-2.50919050985094	0.243654883901909\\
-2.1545017961401	0.603522232975445\\
-1.73316922104735	0.980104720301319\\
-1.25652350100294	1.35475173726096\\
-0.746404409624323	1.70596083943897\\
-0.230855365319286	2.01482665889464\\
0.263069460758329	2.26979350025063\\
0.71545763148392	2.46816383184406\\
1.1158702518671	2.61428313674225\\
1.46209994357525	2.71628873977033\\
1.75733287011997	2.78326240843799\\
2.00742483932292	2.82352076593156\\
2.21900696162006	2.84391929265026\\
2.39842872978907	2.84977139256201\\
};
\addplot [color=mycolor1, forget plot]
  table[row sep=crcr]{%
2.64097641089213	2.89659444528762\\
2.79041062611745	2.8919835091847\\
2.91701342264999	2.8799994826167\\
3.02517161377208	2.86277682362462\\
3.11835537351822	2.84179789268256\\
3.1993034241174	2.8180842826645\\
3.27017992279921	2.79233229640008\\
3.33270056385315	2.76500748982383\\
3.38823118982876	2.73641014297704\\
3.43786378858543	2.7067203746822\\
3.48247461290933	2.67602905414433\\
3.52276843328127	2.64435875960573\\
3.55931210575014	2.61167768001073\\
3.59255988695773	2.57790840977447\\
3.62287230601135	2.54293292803773\\
3.65052990345537	2.50659459017085\\
3.67574274773775	2.46869762432826\\
3.69865631155523	2.42900437234479\\
3.71935400691495	2.3872303083034\\
3.73785641225198	2.34303668445357\\
3.75411695064979	2.29602047291274\\
3.76801346544322	2.24570107596631\\
3.77933475253769	2.19150305315141\\
3.78776060210338	2.13273384775694\\
3.79283321636605	2.06855518320083\\
3.79391692835111	1.99794645060503\\
3.79014185654502	1.9196580684176\\
3.78032540163229	1.83215258867449\\
3.76286329027734	1.73353154789707\\
3.73557937374125	1.62144735702884\\
3.69552135443048	1.4930032209677\\
3.63869022202073	1.34465276673472\\
3.5596997724219	1.1721294490384\\
3.45139077633049	0.970471418888199\\
3.3044941123845	0.73426928073975\\
3.10758058775252	0.458352527766974\\
2.84777069524612	0.139206421099586\\
2.51291985474115	-0.222666080392999\\
2.09587767575504	-0.619732372235386\\
1.60026445800111	-1.03518853022456\\
1.04485624976115	-1.44396712379425\\
0.462207844468628	-1.81852813787821\\
-0.109945569927732	-2.13720202313305\\
-0.639837694079034	-2.38988812122926\\
-1.1085225069396	-2.57800121726277\\
-1.51002341235624	-2.71030781968856\\
-1.84740273810277	-2.79822029892015\\
-2.12828543924942	-2.852660444426\\
-2.3616410819168	-2.88268286233461\\
-2.55602732535301	-2.89522665708\\
-2.71886286896576	-2.89537612907735\\
-2.85626797288029	-2.88676174916544\\
-2.9731620129911	-2.87193760572698\\
-3.07344897627211	-2.85268406443714\\
-3.16021175941515	-2.83023305489948\\
-3.23588415401786	-2.80542946599412\\
-3.3023921495358	-2.77884446518797\\
-3.3612657187078	-2.75085422974431\\
-3.41372549818282	-2.72169432633904\\
-3.46074929415268	-2.69149708850984\\
-3.50312281966151	-2.66031711633317\\
-3.54147825685777	-2.62814841171454\\
-3.57632343641523	-2.59493552952458\\
-3.60806373863459	-2.56058033584831\\
-3.63701826208773	-2.52494541321332\\
-3.66343135939488	-2.48785476032272\\
-3.687480279284	-2.44909214452755\\
-3.70927935164872	-2.40839723914093\\
-3.72888088108065	-2.36545948564737\\
-3.74627264792341	-2.31990944066658\\
-3.7613716263694	-2.27130718105428\\
-3.77401318424574	-2.21912713209767\\
-3.7839345887977	-2.16273843990084\\
-3.79075105493889	-2.10137972025599\\
-3.79392176922668	-2.03412668286445\\
-3.79270221906001	-1.95985077568743\\
-3.78607765707419	-1.87716669740717\\
-3.77267056209203	-1.78436658148074\\
-3.75061256297117	-1.67933930236004\\
-3.71736886968593	-1.55947562288976\\
-3.66950212780627	-1.42156566010487\\
-3.60236624409406	-1.26170792856632\\
-3.50973711671127	-1.07527511122444\\
-3.38343266281118	-0.857029400844818\\
-3.21307688276914	-0.601556618540798\\
-2.98635308825682	-0.304279960318684\\
-2.69035381736796	0.0366596101033578\\
-2.31476776430595	0.417622393762846\\
-1.85709586746691	0.826579657652298\\
-1.32825248141111	1.24217289287799\\
-0.754563161249114	1.63710924267664\\
-0.172579790492442	1.98578138140815\\
0.381690366566272	2.27193718950993\\
0.882472466847537	2.49158054055026\\
1.31763497668127	2.65043265612539\\
1.68631313491539	2.75909577113312\\
1.9943501924068	2.82900808155476\\
2.25034735327705	2.87024233480355\\
2.46321131062424	2.89078292609621\\
2.64097641089213	2.89659444528762\\
};
\addplot [color=mycolor1, forget plot]
  table[row sep=crcr]{%
2.93937327811104	2.98020791618488\\
3.0848469729724	2.97572929737168\\
3.20660057920926	2.96421158105103\\
3.30953681559854	2.94782581088244\\
3.39743205735299	2.92804154463718\\
3.47320078612098	2.90584830807318\\
3.53910295079154	2.88190610921701\\
3.59690170272777	2.85664696662268\\
3.64798167114937	2.83034322980455\\
3.69343713544351	2.80315351614966\\
3.73413767026049	2.77515352399524\\
3.77077704972821	2.74635653760725\\
3.80390969856187	2.71672679942007\\
3.83397781176107	2.68618783037745\\
3.8613313854702	2.65462704500914\\
3.88624274436105	2.62189750703973\\
3.9089166545751	2.58781731858184\\
3.92949672483139	2.5521668737934\\
3.94806848002368	2.51468399639427\\
3.9646592060098	2.47505679034342\\
3.97923437833746	2.43291384066253\\
3.99169016663867	2.38781118544366\\
4.00184110903122	2.33921521879601\\
4.00940152307047	2.28648035436313\\
4.01395848707556	2.22881985483185\\
4.0149331832171	2.16526769114884\\
4.01152589598762	2.09462862606547\\
4.00263781427355	2.01541295408357\\
3.98675976919101	1.92575162362794\\
3.96181398478126	1.82328724568355\\
3.92492998739264	1.70503781680153\\
3.87213126655526	1.56723527884344\\
3.79790934549623	1.40515550036124\\
3.69467810189503	1.21299017713548\\
3.55216050847545	0.983882265427233\\
3.35691612825008	0.71037699308149\\
3.09254740660601	0.385728441595871\\
2.74163772771271	0.00662483549291108\\
2.29080897278802	-0.422469058064585\\
1.73923812242842	-0.884699757276557\\
1.10716516270388	-1.34981282010608\\
0.436573322336864	-1.78088447824577\\
-0.220890687156659	-2.14710992718977\\
-0.821943562709433	-2.43379883759398\\
-1.34272331037627	-2.64289484339832\\
-1.7782116219387	-2.78646452955348\\
-2.13529852485433	-2.87956050057985\\
-2.42584722998334	-2.93590877128599\\
-2.6623291416762	-2.96635774621973\\
-2.85582398510105	-2.97886103559161\\
-3.01543395442862	-2.97901953234204\\
-3.14835247008457	-2.97069499888843\\
-3.26016183347867	-2.95652191845138\\
-3.35516490155065	-2.93828744324002\\
-3.43667771452072	-2.9171984584452\\
-3.507264902453	-2.89406441611592\\
-3.56892100214669	-2.86942101856488\\
-3.6232072007064	-2.84361351437841\\
-3.67135349433301	-2.81685272698547\\
-3.71433476709828	-2.78925269730199\\
-3.75292744395917	-2.7608558587592\\
-3.78775171277993	-2.73164965769039\\
-3.81930298044209	-2.70157719211839\\
-3.84797521331808	-2.67054354674246\\
-3.87407805236539	-2.63841889729279\\
-3.89784902335864	-2.60503903927504\\
-3.9194617271411	-2.57020369456018\\
-3.93903054716445	-2.53367271663447\\
-3.95661211382123	-2.49516011774674\\
-3.9722034834123	-2.45432565273035\\
-3.98573669073324	-2.41076349279229\\
-3.99706898085955	-2.36398728674946\\
-4.00596757094343	-2.31341061456523\\
-4.01208717353537	-2.25832146405978\\
-4.01493764077979	-2.19784888121129\\
-4.01383783984167	-2.13091933860427\\
-4.00785007513615	-2.05619964169997\\
-3.99568682003507	-1.97202242746716\\
-3.97557799561631	-1.87628976447428\\
-3.94508247899672	-1.76635072010641\\
-3.90082250269903	-1.63885163641798\\
-3.83811651857233	-1.48956683168124\\
-3.75049206492978	-1.31323991930616\\
-3.62909313531356	-1.10351573998024\\
-3.46209560616133	-0.853141210902829\\
-3.2344777859539	-0.554776905634855\\
-2.92892665923736	-0.202946176203328\\
-2.52917471338455	0.202394248066321\\
-2.02691275660062	0.651054593927904\\
-1.43106859787255	1.11918464768549\\
-0.773466768821263	1.57182874168429\\
-0.103109774127866	1.97345457244795\\
0.530458453168599	2.30060631642212\\
1.0930536622269	2.54743514594387\\
1.57088555125187	2.7219330138421\\
1.96586550132787	2.8384039978474\\
2.2880922624006	2.91157789220269\\
2.55010990667103	2.95381098597267\\
2.76383367615874	2.97445489318156\\
2.93937327811104	2.98020791618488\\
};
\addplot [color=mycolor1, forget plot]
  table[row sep=crcr]{%
3.31310804065758	3.11423285912486\\
3.45413337104492	3.10990129700141\\
3.57070011418171	3.09888133309348\\
3.66822059229077	3.08336274896067\\
3.75075389077549	3.06478914220523\\
3.82136389596575	3.04410976075739\\
3.88238263370365	3.02194389134628\\
3.93560120648659	2.99868810193405\\
3.98240768763913	2.97458640468319\\
4.02388686053869	2.94977631080212\\
4.06089266641142	2.92431908538716\\
4.09410109529664	2.8982195133334\\
4.12404896518883	2.87143857341009\\
4.15116240252091	2.84390119084381\\
4.17577768368916	2.81550044301811\\
4.19815627842688	2.78609906534202\\
4.21849534453176	2.75552874147828\\
4.23693448191354	2.72358739709812\\
4.25355920468701	2.69003450423134\\
4.26840128824111	2.65458421163407\\
4.2814358549939	2.61689591897394\\
4.29257473888013	2.57656168398216\\
4.30165526881013	2.53308956451831\\
4.30842307490455	2.48588161816088\\
4.31250676187696	2.43420476743595\\
4.31338118330189	2.37715203405867\\
4.31031439599432	2.31359068433281\\
4.30229088841038	2.24209254415909\\
4.28789994881676	2.16084011080941\\
4.2651725267102	2.06750024045705\\
4.23134208942419	1.95905570936229\\
4.18249472269963	1.83158559607804\\
4.11306313531554	1.67999283802569\\
4.01511709631591	1.49770258622065\\
3.87743618320911	1.27642243790266\\
3.68448903879646	1.00621107253835\\
3.41583178233321	0.676404659142812\\
3.04726885144551	0.278377622413228\\
2.55625498552266	-0.188781270660875\\
1.93383218683997	-0.710200982381025\\
1.20012725724235	-1.24996491816525\\
0.410247293308147	-1.75767968050705\\
-0.362558776364174	-2.1882089706272\\
-1.05730575048697	-2.51968996339351\\
-1.64380701589454	-2.75527700362356\\
-2.11999550471781	-2.91234747622804\\
-2.49933220685415	-3.01130348963465\\
-2.80001154620597	-3.06965670617956\\
-3.03924310506355	-3.10048655263343\\
-3.23124858898985	-3.11291154929754\\
-3.38708122729526	-3.11307847353079\\
-3.51509913996024	-3.10506929500743\\
-3.62156044024057	-3.09158012403946\\
-3.711149313784	-3.07438917578723\\
-3.78738893952191	-3.05466771209734\\
-3.8529493853219	-3.03318359710622\\
-3.90987206489963	-3.01043404356493\\
-3.95973201171652	-2.98673231744526\\
-4.00375510589819	-2.96226456418639\\
-4.04290302683582	-2.93712714354745\\
-4.07793511653314	-2.91135111566008\\
-4.10945364919197	-2.88491812560413\\
-4.13793706580435	-2.85777040301206\\
-4.1637643602983	-2.82981660720521\\
-4.18723283314462	-2.80093460271564\\
-4.20857073457294	-2.77097181506166\\
-4.22794581138425	-2.73974350941688\\
-4.245470382185	-2.70702910090965\\
-4.26120324543763	-2.67256640697519\\
-4.27514843193874	-2.63604356066886\\
-4.2872505100243	-2.59708809391366\\
-4.29738579648496	-2.55525244546419\\
-4.30534836692443	-2.50999481981321\\
-4.31082912432757	-2.46065388244215\\
-4.31338526900716	-2.40641517487225\\
-4.31239615973142	-2.34626630873118\\
-4.30699952897152	-2.27893688278491\\
-4.29599896864385	-2.20281760686899\\
-4.27772905629921	-2.11585134255531\\
-4.24985786814903	-2.01538698375836\\
-4.20909750788577	-1.89798634141617\\
-4.15078234346563	-1.75917735716389\\
-4.06826643508234	-1.59316129396303\\
-3.95210196693554	-1.39252369712639\\
-3.78903362871618	-1.14810314065818\\
-3.56108363900491	-0.84939444890248\\
-3.24559190290084	-0.486245395053664\\
-2.81813173129827	-0.0529767983154305\\
-2.26105422587032	0.444458656327441\\
-1.578189749031	0.980788387203884\\
-0.807703545356659	1.51104381240574\\
-0.0172819665077225	1.98461563751243\\
0.722354115713504	2.36662229951065\\
1.36473839717828	2.6485642939408\\
1.89506161310841	2.84232579337481\\
2.32065368498424	2.96789444026509\\
2.65836901131155	3.0446346460869\\
2.92634210140821	3.08786054169085\\
3.14038817811298	3.10855749810132\\
3.31310804065758	3.11423285912486\\
};
\addplot [color=mycolor1, forget plot]
  table[row sep=crcr]{%
3.78933340392954	3.31791824327719\\
3.9256218828151	3.31374193339205\\
4.03688754026066	3.30322976742082\\
4.12902223330395	3.28857288936556\\
4.2063313864434	3.27117830469308\\
4.27199608767686	3.25194973580832\\
4.32839497011113	3.23146399032309\\
4.37732801465156	3.21008240302836\\
4.42017287967763	3.18802181828246\\
4.45799495636751	3.16540012959222\\
4.49162555254062	3.1422656083244\\
4.52171793409172	3.1186157297649\\
4.54878780123903	3.09440904826234\\
4.57324266131504	3.06957233862602\\
4.59540313353684	3.04400438056539\\
4.61551824784775	3.01757721978827\\
4.63377612142125	2.9901353736854\\
4.65031090648788	2.96149318670523\\
4.6652065283807	2.93143033182382\\
4.67849741922023	2.89968526525472\\
4.69016615633992	2.86594624339662\\
4.70013759293096	2.82983927634141\\
4.70826867265188	2.79091208910161\\
4.71433258516435	2.74861274838639\\
4.71799515078679	2.70226103120105\\
4.71878017350119	2.6510097787119\\
4.71601874132723	2.59379227609881\\
4.70877471281688	2.52924996225507\\
4.69573432228688	2.45563229247299\\
4.67504105577271	2.37065713885981\\
4.64404636968432	2.27131567053681\\
4.59893080163012	2.15360087231286\\
4.53412752008188	2.01213665651605\\
4.4414546834988	1.83969417544177\\
4.30885398769578	1.62662912569426\\
4.11871132685225	1.36042186851634\\
3.84610274133577	1.0258819329934\\
3.4584010973348	0.607359779437747\\
2.91995060592895	0.0953039023489661\\
2.20759156509503	-0.501193725730767\\
1.33742359745292	-1.14114239634577\\
0.382662983474024	-1.75477769362568\\
-0.548910372727928	-2.27384341746798\\
-1.36839053415821	-2.66499197898418\\
-2.03794194892131	-2.93408438541059\\
-2.56253374214651	-3.10722854796537\\
-2.96667026485512	-3.21272556954055\\
-3.27780092330027	-3.27315255507966\\
-3.51936725246729	-3.30431196037223\\
-3.70937444249712	-3.31662595527043\\
-3.86105204278324	-3.31680038675033\\
-3.98397054689743	-3.30911826665799\\
-4.08504620548983	-3.29631702607449\\
-4.16930917359062	-3.28015199262873\\
-4.2404548367953	-3.26175110641758\\
-4.30122979639096	-3.24183736156445\\
-4.35370019480879	-3.22086886829122\\
-4.3994386079345	-3.1991277038773\\
-4.43965506957823	-3.17677672466477\\
-4.47528972515932	-3.15389610687864\\
-4.50707895446667	-3.13050686958774\\
-4.53560296016067	-3.10658587936471\\
-4.56132023664761	-3.08207514272936\\
-4.58459259871904	-3.05688713619861\\
-4.60570327346796	-3.03090725083102\\
-4.6248697490697	-3.0039939853017\\
-4.6422525006617	-2.97597721465299\\
-4.65796028890226	-2.94665463110278\\
-4.67205238859317	-2.91578625810213\\
-4.68453780508563	-2.8830867485093\\
-4.69537123272274	-2.84821496494332\\
-4.70444515773916	-2.81076007592616\\
-4.71157705238056	-2.77022304923311\\
-4.71648996968417	-2.72599193502021\\
-4.71878391210462	-2.67730863644428\\
-4.71789392855401	-2.6232238652005\\
-4.71302870163677	-2.5625355332369\\
-4.70307995437032	-2.49370375141885\\
-4.6864876004763	-2.41473266923938\\
-4.66103707452359	-2.32300543050104\\
-4.62355218403979	-2.21505373816029\\
-4.56942756657358	-2.08623940876509\\
-4.49191974942988	-1.93032717585299\\
-4.38109402030484	-1.73895152208022\\
-4.22234462466892	-1.50106645059843\\
-3.99459081237621	-1.20270969746124\\
-3.66891074591962	-0.827976910678309\\
-3.21003812908939	-0.363074319302198\\
-2.58576302165782	0.194107345369954\\
-1.78898086231441	0.819660721978775\\
-0.864096666789198	1.45603320419475\\
0.0926204481931285	2.029259224117\\
0.976282252524162	2.48578474467102\\
1.72232346382894	2.81337818018193\\
2.3170353021528	3.03079287264475\\
2.77789272839084	3.16685501765346\\
3.13226917842333	3.24743838858308\\
3.4060300034355	3.29163397229516\\
3.61988992172188	3.31233578974474\\
3.78933340392954	3.31791824327719\\
};
\addplot [color=mycolor1, forget plot]
  table[row sep=crcr]{%
4.40309459243992	3.61673537985275\\
4.53481677921992	3.61270800079255\\
4.64110043478882	3.60267248745111\\
4.72826995708325	3.58880955901003\\
4.80083637203317	3.57248498439558\\
4.86206764080882	3.55455676819874\\
4.91436804378816	3.53556130027745\\
4.95953263533969	3.51582758848991\\
4.99891992945788	3.49554826079772\\
5.03357064447642	3.47482417862802\\
5.06429044154428	3.45369265395223\\
5.09170828781157	3.43214527136778\\
5.11631805711692	3.410138959954\\
5.13850839996918	3.38760254418567\\
5.15858423577884	3.36444013421383\\
5.17678210917628	3.34053216635835\\
5.19328089884202	3.3157345415432\\
5.20820883754311	3.28987605209768\\
5.22164740765234	3.26275408446327\\
5.23363235533374	3.23412840086048\\
5.24415177023397	3.20371260744588\\
5.25314086087471	3.17116268062802\\
5.26047266817884	3.13606161193869\\
5.26594343298205	3.09789879743532\\
5.26925056740572	3.05604217020051\\
5.26996001559929	3.00970014813575\\
5.26745796636491	2.95786908053401\\
5.26087896394693	2.89925977346441\\
5.24899772008713	2.8321934711684\\
5.23006411986865	2.75445280529139\\
5.20154796627757	2.66306595096728\\
5.15973861448595	2.5539918319648\\
5.09911009413821	2.42166098728948\\
5.01131013473353	2.25831539476813\\
4.88356677732647	2.05310112519422\\
4.69627993517117	1.79096594601817\\
4.41979118172258	1.45178521240388\\
4.01147087775361	1.01120055205626\\
3.41775468130055	0.446881263255729\\
2.59202807503267	-0.244177181185543\\
1.53803684977493	-1.01899743473361\\
0.353058949636403	-1.780497840141\\
-0.79904577232246	-2.42258252442959\\
-1.78480006806738	-2.89332952841377\\
-2.55848306879809	-3.20447426270419\\
-3.13987999047351	-3.39650357026675\\
-3.57134976436999	-3.5092188842842\\
-3.89333155653455	-3.57180253210422\\
-4.13710369462435	-3.60327573004656\\
-4.32501861242621	-3.61547196053107\\
-4.47262257592968	-3.61565291919622\\
-4.59069228147971	-3.60828113370718\\
-4.68675783778689	-3.59611933784414\\
-4.76615035916567	-3.58089207619238\\
-4.83270185626321	-3.56368187291071\\
-4.88921007291105	-3.54516800989344\\
-4.93774878031748	-3.52577211107219\\
-4.97987706309158	-3.50574806491598\\
-5.01678227739226	-3.4852382414019\\
-5.04937901093761	-3.46430895694594\\
-5.07837847286748	-3.44297291939761\\
-5.10433771159524	-3.42120332425291\\
-5.12769484256259	-3.39894245124507\\
-5.14879439005106	-3.37610650483996\\
-5.16790548484653	-3.35258775419939\\
-5.18523474790841	-3.32825458453661\\
-5.20093506196988	-3.30294976941533\\
-5.21511097995922	-3.27648704866286\\
-5.22782116803274	-3.24864590680683\\
-5.23907797792926	-3.21916426047402\\
-5.24884394262677	-3.18772855137705\\
-5.2570246441864	-3.15396047294106\\
-5.26345695555513	-3.11739919286055\\
-5.26789102729119	-3.07747741355114\\
-5.26996344992716	-3.03348885180034\\
-5.26915757024225	-2.98454358648732\\
-5.26474463940321	-2.92950601583791\\
-5.25569575852446	-2.86690757081248\\
-5.24054850816678	-2.79482237995296\\
-5.21720209561772	-2.71068811292387\\
-5.1825981888099	-2.61104546079613\\
-5.13221728813398	-2.49115771374786\\
-5.05927746750375	-2.34445847477063\\
-4.95346176740859	-2.16177090388398\\
-4.79894273764576	-1.93028302696452\\
-4.57152676326354	-1.63246322477282\\
-4.23529796064144	-1.24574721441874\\
-3.7412462009014	-0.74544591316852\\
-3.03549773002775	-0.115886240823348\\
-2.08987122394223	0.626159587489338\\
-0.952219886815669	1.40870385076199\\
0.237291686538823	2.12144366640694\\
1.31764533683373	2.67979391491851\\
2.19789026299071	3.06654459157377\\
2.87070142789591	3.31267996601176\\
3.37164122166647	3.46068313324343\\
3.74384219945325	3.54538371592389\\
4.02341056241594	3.5905547199245\\
4.23693830634328	3.61124714534204\\
4.40309459243992	3.61673537985275\\
};
\addplot [color=mycolor1, forget plot]
  table[row sep=crcr]{%
5.18712583895256	4.03643886090506\\
5.31525725204707	4.03252906981332\\
5.41757620024171	4.02287294925828\\
5.50079469346955	4.00964175117464\\
5.5696003253642	3.99416555895309\\
5.62733170440595	3.97726378963646\\
5.67641122072661	3.95943940624426\\
5.71862685754273	3.94099514264964\\
5.75531895681641	3.92210424785202\\
5.78750618287741	3.90285415339594\\
5.81597190070615	3.88327366803006\\
5.84132430607785	3.86334991685115\\
5.8640388204713	3.84303872581938\\
5.88448826356221	3.82227067715433\\
5.90296441651425	3.80095417392495\\
5.91969336093515	3.77897630111696\\
5.9348461628453	3.75620191182305\\
5.94854590917154	3.73247111600344\\
5.96087169458886	3.70759515234812\\
5.97185983006285	3.68135044432882\\
5.98150224897495	3.65347044950186\\
5.98974177522711	3.6236346761894\\
5.99646353807814	3.59145392663468\\
6.00148130126427	3.55645037850838\\
6.00451671496219	3.51803045859987\\
6.00516833049201	3.4754474699895\\
6.00286535774335	3.42774940781586\\
5.99679810904926	3.37370501352264\\
5.98581200843584	3.31169733754317\\
5.96824343302717	3.23956802578995\\
5.94166076599006	3.1543857977506\\
5.90244799924475	3.05209698328066\\
5.84512255933808	2.92699173597258\\
5.76120034149497	2.77088497318124\\
5.63729389255025	2.57187402037809\\
5.45196389052451	2.31254430268633\\
5.170800179951	1.96774488397845\\
4.74005681044081	1.50317000806675\\
4.08343851340225	0.879405714822638\\
3.11873275843801	0.0725212336633277\\
1.82240158153292	-0.879979006481791\\
0.3208772432074	-1.84474570715855\\
-1.13257542859024	-2.65499655952802\\
-2.33491184292571	-3.22951294326548\\
-3.23551674549088	-3.5919659263711\\
-3.8819331864792	-3.80563197939625\\
-4.34333653284971	-3.92625706921822\\
-4.67714451079244	-3.99118894206398\\
-4.92384191511931	-4.02306768443975\\
-5.1104782906461	-4.03519717120396\\
-5.25494229682946	-4.03538413725133\\
-5.36916620850968	-4.02825870448253\\
-5.46124197072875	-4.01660612024233\\
-5.53676482188745	-4.00212386009999\\
-5.59968129721263	-3.98585566318484\\
-5.65282887921241	-3.96844430635513\\
-5.69828447180154	-3.95028150863634\\
-5.73759331009454	-3.93159842884888\\
-5.77192201193149	-3.91252116361816\\
-5.80216268002389	-3.89310518437901\\
-5.82900484455859	-3.8733568160797\\
-5.8529858815526	-3.85324654641236\\
-5.87452674587498	-3.83271703316872\\
-5.89395747634478	-3.81168753720132\\
-5.91153540716701	-3.79005581256549\\
-5.92745802288954	-3.7676980440536\\
-5.94187172094541	-3.74446712558521\\
-5.95487727028482	-3.72018935421026\\
-5.9665323937902	-3.69465943029842\\
-5.97685159715728	-3.66763347237637\\
-5.98580306939653	-3.63881954565161\\
-5.99330214181357	-3.60786493364055\\
-5.99920035424942	-3.57433900902988\\
-6.00326855541121	-3.53771001937137\\
-6.00517152686637	-3.49731329719506\\
-6.00443015115258	-3.45230717604644\\
-6.00036477491604	-3.40161098959502\\
-5.99200950644002	-3.34381653301228\\
-5.9779805973346	-3.27705958700445\\
-5.95627075586493	-3.19883042356599\\
-5.92392157032137	-3.10568985900584\\
-5.87649171664733	-2.99283785316287\\
-5.80717843048081	-2.85345220038719\\
-5.70534840422258	-2.67767671923539\\
-5.5540823371721	-2.45111316334733\\
-5.32619170660676	-2.15275947247519\\
-4.97839825424011	-1.75289683024479\\
-4.44544133267055	-1.21346613054153\\
-3.64360737117788	-0.498615041134729\\
-2.50836858312705	0.391708787828822\\
-1.08269748007528	1.37202389763564\\
0.427899320710524	2.27720604166756\\
1.77169676503285	2.97202628862457\\
2.82116844859093	3.43344452188135\\
3.58608777976602	3.71348668836244\\
4.131761062643	3.87482760873836\\
4.52329054454194	3.96399345419352\\
4.80943067044692	4.01026330629437\\
5.02337853892604	4.03101761132268\\
5.18712583895256	4.03643886090506\\
};
\addplot [color=mycolor1, forget plot]
  table[row sep=crcr]{%
6.12157382844716	4.57111927024128\\
6.24806332456751	4.56726569668653\\
6.34824649775386	4.55781500425311\\
6.4291970916542	4.54494694268638\\
6.49577467523898	4.52997365270056\\
6.55139508238138	4.51369115448788\\
6.59851042630908	4.49658102796028\\
6.63891485485238	4.47892879176142\\
6.67394362166934	4.46089480869968\\
6.70460542067214	4.44255745871458\\
6.7316720572432	4.42393970758168\\
6.75574025693286	4.40502547929216\\
6.77727488947267	4.38576958944698\\
6.79663952794232	4.36610347045196\\
6.814118176515	4.34593801528116\\
6.82993067301352	4.32516431250666\\
6.84424340439968	4.30365268873562\\
6.85717638380716	4.28125022700269\\
6.86880731417405	4.25777673640515\\
6.87917293117044	4.23301897186331\\
6.88826762162818	4.20672271277163\\
6.8960390043297	4.17858207435375\\
6.90237978446501	4.14822510695287\\
6.90711468046038	4.1151942809265\\
6.90998046502719	4.07891977302142\\
6.91059598638067	4.03868242701766\\
6.90841714258277	3.99356163051609\\
6.90266864778835	3.94236074878176\\
6.89223910917539	3.8834985298116\\
6.87551668756016	3.81484790313155\\
6.85012617833902	3.73349184818549\\
6.81249851632434	3.63534605823882\\
6.75714866376768	3.51456422659319\\
6.67543564422854	3.36258547306192\\
6.55339203348151	3.16659726668628\\
6.3678959605814	2.9070891579332\\
6.08006719213609	2.55421584701719\\
5.62497593235754	2.06357009602314\\
4.90087775492559	1.37605332592143\\
3.77905242794992	0.438318091531441\\
2.18933297264066	-0.729136310160695\\
0.287381237730008	-1.95096832528054\\
-1.5447403507541	-2.97262224633514\\
-3.00610436096118	-3.6713485534524\\
-4.04944884188627	-4.09155331359193\\
-4.76576569968722	-4.32849181891376\\
-5.25912890942235	-4.45755806731079\\
-5.60648327241671	-4.52516921820955\\
-5.85799505250904	-4.55769374925049\\
-6.0453526545804	-4.56988338160137\\
-6.18866536129684	-4.57007668099744\\
-6.30093745826316	-4.56307784238045\\
-6.39078107110039	-4.5517108744142\\
-6.46404187547533	-4.53766449240715\\
-6.52478291269304	-4.52196027003734\\
-6.57589106657402	-4.50521810715663\\
-6.6194591016604	-4.48781032519455\\
-6.65703171446965	-4.46995307151583\\
-6.68976743548591	-4.45176155084663\\
-6.71854731555456	-4.43328386418837\\
-6.74404922583356	-4.41452187854762\\
-6.76679946214588	-4.39544402454388\\
-6.78720904944798	-4.37599291397698\\
-6.80559950258139	-4.35608949909373\\
-6.82222114061885	-4.33563479105379\\
-6.83726598293421	-4.31450971391857\\
-6.85087654374139	-4.29257337685617\\
-6.86315134649674	-4.26965983229243\\
-6.87414760938744	-4.24557320696391\\
-6.88388124494256	-4.22008091326308\\
-6.89232401999751	-4.19290443992443\\
-6.89939738720861	-4.16370694997184\\
-6.90496206581973	-4.13207653417616\\
-6.90880183133517	-4.09750341183482\\
-6.91059903522101	-4.05934852983483\\
-6.90989788950252	-4.01679970999594\\
-6.90604912324757	-3.96880944003592\\
-6.89812554682906	-3.91400509544899\\
-6.88479106675031	-3.85055695492978\\
-6.86409340019097	-3.77598032518114\\
-6.8331286479579	-3.6868327962381\\
-6.78748542624817	-3.5782416184092\\
-6.72030123773543	-3.44315227413553\\
-6.62062508317728	-3.27111847402897\\
-6.47053413497078	-3.04635546174779\\
-6.24007622704454	-2.74471365640521\\
-5.87884444936142	-2.32953799591968\\
-5.3044219589401	-1.7483980287049\\
-4.3972539283441	-0.940097828098037\\
-3.04005847943889	0.123663394463265\\
-1.25604321060693	1.34990678306991\\
0.662081669560649	2.49936265935053\\
2.33001000851835	3.36220391793696\\
3.57564188357901	3.91025083801778\\
4.44149212444558	4.22747445667198\\
5.03479210309462	4.4030175728086\\
5.44739037766176	4.49704300301994\\
5.74189957370979	4.5446983443192\\
5.95822788276908	4.56570118070013\\
6.12157382844716	4.57111927024128\\
};
\addplot [color=mycolor1, forget plot]
  table[row sep=crcr]{%
6.9909859082623	5.08978129550532\\
7.11790148972651	5.08591861692123\\
7.21790506732202	5.07648728423415\\
7.29838163561349	5.06369615375623\\
7.36435274701211	5.04886033454228\\
7.41931940096902	5.03276997863885\\
7.46577835194663	5.01589878061026\\
7.50554660831771	4.99852490148967\\
7.53997044729718	4.98080268271892\\
7.57006314704963	4.96280594493406\\
7.59659764406972	4.94455443903197\\
7.62017001937801	4.92603003784696\\
7.6412436766496	4.90718649633422\\
7.66018045043824	4.88795503534393\\
7.6772626550723	4.86824708120501\\
7.69270868132481	4.84795493222628\\
7.70668383791504	4.8269507645884\\
7.71930752208815	4.80508414235582\\
7.73065736702177	4.78217800344967\\
7.74077067438982	4.75802291717117\\
7.74964314023736	4.73236921810162\\
7.75722457110038	4.70491638404746\\
7.76341091045485	4.67529870204613\\
7.7680313806868	4.64306579848476\\
7.77082878287473	4.60765590708681\\
7.77142980517594	4.56835866541107\\
7.7693002597077	4.52426252110813\\
7.76367694468062	4.47417907231269\\
7.75346230241894	4.41653212432005\\
7.73705831926242	4.34919159985885\\
7.71209854955049	4.26921930519493\\
7.67500460179675	4.17247056612995\\
7.62023168766171	4.05295499601583\\
7.53894862904738	3.90178743964219\\
7.41666688165914	3.70543622994144\\
7.22890136428397	3.44278820015857\\
6.93324989831278	3.08039162177707\\
6.4557914223663	2.56576625180694\\
5.67298969892936	1.82278227009393\\
4.41173338381147	0.768977579181146\\
2.54848954519755	-0.598763404959222\\
0.259362341041705	-2.06909945961282\\
-1.93680878161845	-3.29406765961223\\
-3.63733063802839	-4.10755022276248\\
-4.80690862843087	-4.57885393338226\\
-5.58394911087071	-4.8360087212446\\
-6.10581601941451	-4.97259431008842\\
-6.46649792202238	-5.04283035884027\\
-6.72414397385616	-5.07616400887524\\
-6.91415248836824	-5.08853476964782\\
-7.05839474350069	-5.08873432244207\\
-7.17073735575507	-5.08173413086652\\
-7.26022690093588	-5.07041390261868\\
-7.33293315759245	-5.0564751412908\\
-7.39303655546847	-5.04093667685731\\
-7.44348572540194	-5.02441103394238\\
-7.48640552596234	-5.00726273880557\\
-7.52335670437388	-4.98970120591377\\
-7.55550510552885	-4.97183635536423\\
-7.58373436459845	-4.95371242221127\\
-7.60872244132046	-4.93532866348408\\
-7.63099448766155	-4.91665197383152\\
-7.65095987493623	-4.89762434431278\\
-7.6689383733506	-4.87816689879263\\
-7.6851787145651	-4.85818152645098\\
-7.69987164223171	-4.83755068372668\\
-7.71315881284425	-4.81613564451565\\
-7.72513839650761	-4.79377326270991\\
-7.73586784746538	-4.77027113072694\\
-7.74536400090288	-4.74540083794309\\
-7.75360035311974	-4.71888882329563\\
-7.76050104570531	-4.69040404176945\\
-7.76593063930396	-4.65954127749152\\
-7.76967814082003	-4.62579836507921\\
-7.77143279958508	-4.58854471138716\\
-7.77074767729164	-4.54697715285237\\
-7.76698451038783	-4.50005701858685\\
-7.75922917705121	-4.44641874011509\\
-7.74615977391762	-4.38423447145167\\
-7.72583628097031	-4.31100918742543\\
-7.69535695951676	-4.22326338726391\\
-7.65028192517443	-4.11602995471153\\
-7.58363868665366	-3.98203740160641\\
-7.48415836494337	-3.81035641884321\\
-7.33307293563603	-3.58413002197802\\
-7.09823734409178	-3.27680652861004\\
-6.7236017538666	-2.84632035718348\\
-6.11259151482577	-2.22835698922053\\
-5.11344545089394	-1.3384735771218\\
-3.55457909703552	-0.117214279941999\\
-1.4289302365554	1.34338348366825\\
0.884372753194159	2.72972904867204\\
2.85844942495496	3.75135840219081\\
4.28103103102118	4.37759901212331\\
5.23495821793396	4.72727792235517\\
5.86989868459124	4.91523188368736\\
6.30199845709865	5.01374509012517\\
6.60558442554296	5.06289109773317\\
6.82599817175736	5.08430221422049\\
6.9909859082623	5.08978129550532\\
};
\addplot [color=mycolor1, forget plot]
  table[row sep=crcr]{%
7.2832163016226	5.26738346659666\\
7.41061043289889	5.263507259095\\
7.51085362238329	5.25405397104058\\
7.59143597922978	5.24124644397224\\
7.65743670387548	5.22640424861722\\
7.71238936855266	5.21031818762825\\
7.75880959249844	5.19346119896369\\
7.79852552685244	5.17611028823901\\
7.8328901282851	5.15841865182427\\
7.86292071882693	5.14045912617052\\
7.88939269615076	5.12225067952556\\
7.91290364110264	5.10377460001331\\
7.93391787068537	5.0849842368701\\
7.95279777727258	5.06581056233232\\
7.96982602300386	5.0461648913656\\
7.98522123048198	5.02593953196896\\
7.99914888712475	5.00500677865477\\
8.01172855981302	4.98321641326176\\
8.02303807533367	4.96039168395227\\
8.03311497988752	4.93632355656157\\
8.04195528894996	4.91076284102306\\
8.04950922619908	4.88340955722362\\
8.0556732719991	4.85389857879987\\
8.06027732503006	4.82178012145223\\
8.06306501409945	4.78649293281249\\
8.06366399836447	4.74732694439775\\
8.06154114814309	4.70337041156567\\
8.05593424328879	4.65343376196269\\
8.0457462307116	4.59593773479463\\
8.02937820478327	4.5287455533124\\
8.00445936209182	4.4489053281845\\
7.9673988192036	4.35224501095417\\
7.91262045065983	4.23271945852377\\
7.83121765478356	4.08133226464549\\
7.70852088001885	3.88431981373448\\
7.51960327357625	3.62006967991628\\
7.22095771012417	3.25402150257812\\
6.73585491905443	2.73119467365785\\
5.9338037090026	1.97001890949518\\
4.62681297446452	0.878148203435068\\
2.67187007308957	-0.556720013437796\\
0.250486289929001	-2.11193355209638\\
-2.0696655194179	-3.40615562364339\\
-3.8499829292279	-4.25793875217054\\
-5.06091117825139	-4.74598363350573\\
-5.85782546953629	-5.00975354786056\\
-6.38925156712959	-5.14885876736711\\
-6.75466431715758	-5.22002459554351\\
-7.01472669261033	-5.25367520056542\\
-7.20599774179902	-5.26613050162035\\
-7.3509037063389	-5.26633231327831\\
-7.46358791204496	-5.25931164731821\\
-7.5532406990084	-5.24797128446114\\
-7.62600937502927	-5.23402089881646\\
-7.68611755561483	-5.21848143480156\\
-7.73653859168646	-5.20196517777155\\
-7.77941180915661	-5.18483562082599\\
-7.81630656912174	-5.16730099817604\\
-7.84839390877197	-5.14947015553103\\
-7.87656061685162	-5.13138644318063\\
-7.90148657445999	-5.11304843630537\\
-7.92369810403299	-5.09442253706724\\
-7.94360528994299	-5.07545041203258\\
-7.96152834084078	-5.05605300734567\\
-7.97771626876498	-5.03613216313046\\
-7.99236001617379	-5.01557040116862\\
-8.00560140905129	-4.99422916439238\\
-8.0175387954673	-4.97194557148382\\
-8.02822984560691	-4.9485275690411\\
-8.03769167373634	-4.92374718350929\\
-8.04589814162964	-4.89733136452143\\
-8.05277386521658	-4.86894963511716\\
-8.05818400956813	-4.83819737436594\\
-8.06191833293965	-4.80457298148844\\
-8.0636669871316	-4.76744629123877\\
-8.0629840602184	-4.72601423503403\\
-8.05923233959752	-4.67923754237978\\
-8.05149851972958	-4.62574867987064\\
-8.03846066931219	-4.56371521155516\\
-8.01817651567367	-4.49063248713586\\
-7.98773673884338	-4.40300162235347\\
-7.94268149303391	-4.2958168297695\\
-7.875989565814	-4.16172878067136\\
-7.77627537215769	-3.9896481243096\\
-7.62448891865551	-3.76237899002894\\
-7.38778859683814	-3.45262826362175\\
-7.00836393412335	-3.01666537601575\\
-6.38518511700294	-2.38644975042891\\
-5.35599446820208	-1.46991717648909\\
-3.7303363840847	-0.19650874428884\\
-1.48884909752624	1.34353339141959\\
0.959651428398393	2.81092659973965\\
3.03703543232128	3.88615305828664\\
4.51803085418353	4.53821345645394\\
5.50071654952238	4.89848960522985\\
6.14940394979527	5.09053888006187\\
6.58820082916082	5.19059117671614\\
6.89515570020845	5.24028859755106\\
7.11731260873144	5.2618722081855\\
7.2832163016226	5.26738346659666\\
};
\addplot [color=mycolor1, forget plot]
  table[row sep=crcr]{%
6.70347763616719	4.91651399485236\\
6.83007664955833	4.91265981889838\\
6.92998138220201	4.90323710271026\\
7.01047416996457	4.89044293379464\\
7.07652162707673	4.87558963231008\\
7.1315946956581	4.85946790418777\\
7.1781734424047	4.84255304116001\\
7.21806555594753	4.82512492910375\\
7.2526121361442	4.80733942500051\\
7.28282364523882	4.78927155807402\\
7.30947155849757	4.77094197770443\\
7.33315126942091	4.75233317462754\\
7.35432592629523	4.73339927747518\\
7.37335733591397	4.71407166956962\\
7.39052788745482	4.69426175477585\\
7.40605607042567	4.67386164292077\\
7.42010726413257	4.65274316775044\\
7.43280087100144	4.63075540303822\\
7.44421443399252	4.60772064972786\\
7.45438504147412	4.58342869094532\\
7.46330802425303	4.55762892140207\\
7.47093263923078	4.53001972152165\\
7.47715405822289	4.50023412488345\\
7.48180046698014	4.46782036326869\\
7.4846133193766	4.43221517819991\\
7.48521760641887	4.39270671784476\\
7.48307708251151	4.34838215388948\\
7.47742619858199	4.29805244526697\\
7.46716503316036	4.24014223168562\\
7.45069394100136	4.17252539686212\\
7.42564742943805	4.09227413291477\\
7.38845507139868	3.99526728048557\\
7.33359759524606	3.87556506932507\\
7.25231233843863	3.72439006851256\\
7.13028518270596	3.52844199409026\\
6.94347119049367	3.2671147221298\\
6.65058787595554	2.90809150816575\\
6.18057119284037	2.40144748284787\\
5.41690092925881	1.67654181760295\\
4.20115170498602	0.660616150378107\\
2.42827484707059	-0.640964198887718\\
0.268347616781196	-2.02837983043246\\
-1.80653486008804	-3.18560397589839\\
-3.42824772437214	-3.96126484223808\\
-4.55665807253282	-4.41590235047736\\
-5.31392618188489	-4.66647502121664\\
-5.82640166433818	-4.80058418077178\\
-6.18255946006111	-4.8699302108425\\
-6.43799920851102	-4.90297376549035\\
-6.62693996389362	-4.91527247467651\\
-6.77069166466908	-4.91546988765032\\
-6.88284379882309	-4.90848067661172\\
-6.9723011509031	-4.89716395384352\\
-7.04505863936063	-4.88321499271203\\
-7.10525615208148	-4.86765193458418\\
-7.15581995336506	-4.85108855312973\\
-7.19886243218019	-4.83389110241934\\
-7.23593737895186	-4.81627063938727\\
-7.2682068052951	-4.79833845040109\\
-7.29655230128934	-4.78013982151605\\
-7.3216508068323	-4.7616747634758\\
-7.344027034824	-4.74291066225017\\
-7.36409023460512	-4.72378977323213\\
-7.38216021207903	-4.70423328747032\\
-7.398485793522	-4.6841429867534\\
-7.41325781261155	-4.66340106094978\\
-7.4266179677889	-4.64186836716448\\
-7.43866439015068	-4.6193811957718\\
-7.44945438564445	-4.59574642809872\\
-7.45900450396759	-4.57073479109756\\
-7.46728778834022	-4.54407170536698\\
-7.47422772475973	-4.51542494970146\\
-7.47968797536858	-4.48438798100889\\
-7.4834563607787	-4.4504571823829\\
-7.48522061203805	-4.41300045211077\\
-7.48453191140727	-4.37121320755525\\
-7.48074977535588	-4.32405574785452\\
-7.47295767181221	-4.27016245942286\\
-7.45983155827868	-4.20770761137826\\
-7.43943073289554	-4.13420278451995\\
-7.40885710156391	-4.04618426431852\\
-7.36368557622261	-3.9387195459219\\
-7.29698595053399	-3.80461097266097\\
-7.19759899571727	-3.63308679491217\\
-7.04703364595124	-3.40763157658516\\
-6.81384596300016	-3.10245055255353\\
-6.44377654815097	-2.67718339857415\\
-5.84477135816693	-2.07130501058155\\
-4.87556119954063	-1.20797396692885\\
-3.38288024296397	-0.0383960540040172\\
-1.37078772986577	1.3443138563634\\
0.810538102279413	2.65154390175893\\
2.68311428953206	3.62052057186414\\
4.04772863334151	4.22114173444051\\
4.97299565420324	4.56025948872589\\
5.59432313215858	4.74415705549701\\
6.01991840236974	4.84117447389263\\
6.32034721556549	4.88980295022609\\
6.53922157908623	4.91106112284334\\
6.70347763616719	4.91651399485236\\
};
\addplot [color=mycolor1, forget plot]
  table[row sep=crcr]{%
5.6711334783885	4.30975043106863\\
5.7981024301589	4.30587961160684\\
5.89901983552652	4.29635798587537\\
5.98079215866699	4.28335819729862\\
6.04819759579739	4.26819796487796\\
6.1046137173402	4.25168198630243\\
6.15247622562267	4.23430012596883\\
6.19357391517587	4.21634470956528\\
6.22924216395473	4.1979812635934\\
6.26049234894254	4.17929183890547\\
6.28809998345381	4.16030180769172\\
6.31266572321334	4.14099645246868\\
6.33465817685667	4.12133107574606\\
6.35444425796681	4.10123685616594\\
6.37231081268117	4.08062378112781\\
6.38847997395563	4.05938143380288\\
6.40311984915975	4.03737805543945\\
6.41635157079522	4.01445805505438\\
6.42825332307538	3.99043794404617\\
6.43886162772311	3.96510049580183\\
6.44816987671727	3.93818673968355\\
6.45612378990955	3.9093851641175\\
6.46261309868026	3.87831718679015\\
6.4674582429467	3.84451749726722\\
6.47039011205888	3.80740720632041\\
6.47101968873904	3.76625671562929\\
6.46879257848685	3.72013363666001\\
6.46292031781962	3.66782858083868\\
6.45227514912502	3.6077476111253\\
6.43522598444519	3.53775356181477\\
6.40937752556968	3.45492756017388\\
6.37114634207824	3.3552040438814\\
6.31505680508197	3.23280294416779\\
6.23254804360725	3.0793361534654\\
6.10992346565928	2.88240170329279\\
5.92482870183606	2.62343174964573\\
5.64041904469585	2.27470730679447\\
5.19691702437587	1.79647488054428\\
4.50458082029795	1.13896519617542\\
3.45742283844856	0.263403896699317\\
2.00941141207743	-0.800244905551356\\
0.303011623706775	-1.89654499842687\\
-1.34449683242487	-2.81512225579304\\
-2.68112629374548	-3.4540231187615\\
-3.65671878469359	-3.84681066198187\\
-4.34013422475085	-4.07279450366802\\
-4.81839601455809	-4.19787328446349\\
-5.15918042587159	-4.26418658504594\\
-5.4081503849663	-4.29637222498502\\
-5.5948639841194	-4.30851423607433\\
-5.7384171486743	-4.30870448503699\\
-5.85132502829035	-4.30166392516054\\
-5.94196078964505	-4.29019538445848\\
-6.01605318700399	-4.27598864975694\\
-6.07760908488791	-4.26007311282492\\
-6.12948986345481	-4.24307738928111\\
-6.17377834682678	-4.22538140368627\\
-6.21201710773861	-4.20720728139212\\
-6.24536633812097	-4.18867461752844\\
-6.27471041969153	-4.16983452283209\\
-6.30073111301828	-4.15069072193919\\
-6.32395858420191	-4.13121254900976\\
-6.34480741542836	-4.11134271879507\\
-6.36360222110575	-4.09100159519299\\
-6.38059589306671	-4.07008897948351\\
-6.39598246150051	-4.04848399971752\\
-6.40990586422851	-4.02604338814606\\
-6.42246543071431	-4.00259821730328\\
-6.43371852153588	-3.97794898337333\\
-6.44368045765436	-3.9518587450571\\
-6.4523215771323	-3.92404381761165\\
-6.4595609210172	-3.89416125156107\\
-6.46525561527158	-3.86179194902714\\
-6.46918439675131	-3.8264177215145\\
-6.47102279443057	-3.78738976774571\\
-6.47030599987355	-3.74388478137508\\
-6.46637306063012	-3.69484290960486\\
-6.45828203146807	-3.63887861096069\\
-6.44467890473077	-3.57415031720324\\
-6.42359128449403	-3.49816635320554\\
-6.39209674307987	-3.40749056189011\\
-6.34577796197681	-3.29728790107074\\
-6.2778083077063	-3.16061280682044\\
-6.17738955106327	-2.98728708213752\\
-6.02706024461078	-2.76214966984903\\
-5.79812536235465	-2.46246989105206\\
-5.44343817795874	-2.05475714247584\\
-4.88856704559779	-1.49328566815426\\
-4.03113780386764	-0.729107711971661\\
-2.78027450561968	0.251580606084027\\
-1.17052585641563	1.35824257218321\\
0.548410181789146	2.38830122955274\\
2.05915259483019	3.16964756291024\\
3.21103173494561	3.67628253450724\\
4.02923013302172	3.97594957924406\\
4.60011313937785	4.14480849271164\\
5.00266458119172	4.23651798591384\\
5.29299196057574	4.28348280355837\\
5.5079043851364	4.3043405849677\\
5.6711334783885	4.30975043106863\\
};
\addplot [color=mycolor1, forget plot]
  table[row sep=crcr]{%
4.67163867468412	3.75659873546539\\
4.80186866683076	3.75262005310516\\
4.90652567250872	3.74274013679767\\
4.99208151308878	3.72913519677954\\
5.06311456176333	3.71315652110489\\
5.12291958259539	3.69564659311496\\
5.17390741933042	3.67712836345684\\
5.21786995034725	3.65792026237507\\
5.2561582181402	3.63820710234224\\
5.28980398174832	3.61808431913829\\
5.31960387782812	3.59758578423176\\
5.34617847729379	3.57670127592357\\
5.3700141988918	3.55538727713922\\
5.39149329896913	3.53357332830043\\
5.41091539240697	3.51116528725187\\
5.4285128029399	3.48804629794914\\
5.44446126337847	3.46407590816154\\
5.45888694392339	3.43908752149046\\
5.47187038633885	3.41288416844244\\
5.48344759855503	3.3852323987665\\
5.49360826829224	3.35585390306671\\
5.50229073971236	3.32441423610317\\
5.50937301255037	3.29050770125132\\
5.51465850025144	3.25363701578832\\
5.51785452024647	3.21318573661633\\
5.51854032300997	3.16838047263847\\
5.51611962807314	3.11823846564039\\
5.50974966984763	3.06149390496678\\
5.49823388315572	2.99649290883732\\
5.47985722789589	2.92104177132475\\
5.45212943924673	2.83218483792799\\
5.41137828575746	2.72587598020497\\
5.35209610574454	2.5964902658839\\
5.26588076293174	2.43610259371712\\
5.13972473458458	2.23345390462815\\
4.95333122733118	1.97259485714987\\
4.67527629662551	1.63153661214883\\
4.2589173556774	1.18235232003381\\
3.64281353614946	0.596872128640378\\
2.76871086709487	-0.134514116493091\\
1.63239985388465	-0.969704806761522\\
0.341481053554475	-1.79923945066372\\
-0.911702846433598	-2.49772316798219\\
-1.97133552712673	-3.00385668656745\\
-2.78928181566971	-3.33288905651447\\
-3.39378724049444	-3.53260625340405\\
-3.83599982110015	-3.64816015023836\\
-4.1621849514401	-3.71157909184766\\
-4.40688862759836	-3.74318310171582\\
-4.59417261718687	-3.75534464528515\\
-4.74045144297392	-3.75552783829145\\
-4.85693590176527	-3.74825749999184\\
-4.9513691171685	-3.73630399961283\\
-5.0291828692432	-3.72138067376052\\
-5.09425280363193	-3.70455440951017\\
-5.14939157249457	-3.68648981444233\\
-5.19667372054293	-3.66759648240025\\
-5.23765253214277	-3.64811914323348\\
-5.27350690411595	-3.62819359810396\\
-5.30514231084921	-3.60788177858723\\
-5.33326119401197	-3.58719380834022\\
-5.35841265244685	-3.5661017859846\\
-5.3810278694075	-3.54454814645157\\
-5.40144551937356	-3.52245033846133\\
-5.41992997189453	-3.49970286447534\\
-5.43668416493028	-3.47617728657543\\
-5.4518583743395	-3.45172050149471\\
-5.46555564428253	-3.42615136553884\\
-5.47783428839028	-3.39925556254538\\
-5.48870756772893	-3.37077842323692\\
-5.49814035214634	-3.34041519337928\\
-5.50604222919097	-3.30779797902985\\
-5.51225608123179	-3.27247822817927\\
-5.51654052399648	-3.23390307935904\\
-5.51854366039925	-3.19138312850961\\
-5.51776414387268	-3.14404799408307\\
-5.51349321652105	-3.09078427355424\\
-5.50472759415137	-3.0301477285399\\
-5.49003678633989	-2.96023725677942\\
-5.46735789106889	-2.87851157009022\\
-5.43367306040936	-2.78151934276526\\
-5.38449473794708	-2.66449872084913\\
-5.31303423010647	-2.52078270542301\\
-5.20885353055482	-2.34093024024107\\
-5.05570892896308	-2.11152138409498\\
-4.82828202528647	-1.8137207970727\\
-4.4879570021524	-1.4223506632844\\
-3.97996628739933	-0.908028977357546\\
-3.24036481558319	-0.248412187594042\\
-2.22956888661681	0.544610566929457\\
-0.994985668439008	1.39372573796315\\
0.301866163761128	2.17079799240933\\
1.47121494724185	2.77523922020516\\
2.40982402831559	3.18773350054884\\
3.11508888893499	3.44581206508271\\
3.63202941986422	3.59858514417259\\
4.01115966441353	3.68488685491627\\
4.29300553502665	3.73043968840615\\
4.50653510037146	3.75114038016021\\
4.67163867468412	3.75659873546539\\
};
\addplot [color=mycolor1, forget plot]
  table[row sep=crcr]{%
3.86621004882867	3.35342706786038\\
4.00183357207157	3.34927247233028\\
4.1123685284783	3.33883024245586\\
4.20377044648071	3.32429056068696\\
4.28037595763327	3.30705474370343\\
4.34537997702568	3.28801997229603\\
4.40116579060115	3.26775716189122\\
4.44953332480904	3.24662287243617\\
4.49185790821677	3.22483033303829\\
4.52920164601461	3.20249486927488\\
4.56239231197835	3.17966308188443\\
4.59207976211996	3.15633153217353\\
4.61877659959918	3.13245850227588\\
4.64288763783899	3.10797105090717\\
4.66473124550015	3.08276873997757\\
4.68455466314305	3.05672486288503\\
4.7025446915998	3.02968563972196\\
4.71883465603935	3.00146758243057\\
4.73350817184596	2.97185302495721\\
4.74659992366957	2.9405836247656\\
4.7580933725364	2.90735144414468\\
4.76791498485876	2.87178698466784\\
4.77592418240633	2.83344324345249\\
4.7818976783785	2.79177444288712\\
4.78550609576224	2.74610749632804\\
4.78627961188332	2.69560342433257\\
4.78355760174661	2.63920470388718\\
4.77641448341741	2.57556273942764\\
4.7635495930259	2.50293705401208\\
4.74312197712771	2.41905414834294\\
4.71250005368641	2.32090912183603\\
4.66787928390382	2.20448757604381\\
4.60369678663558	2.06438162517247\\
4.51174254996496	1.89328081664084\\
4.37985077778534	1.68136173464953\\
4.19012139625965	1.41574376435977\\
3.91697709373688	1.08056317916047\\
3.52647748988007	0.659046311978013\\
2.98075927797431	0.140115068399494\\
2.25401139944964	-0.46838896206813\\
1.36123583027566	-1.1249298141095\\
0.378653360704582	-1.75643635154976\\
-0.579639605518917	-2.29040469240906\\
-1.41967280366853	-2.69138883814888\\
-2.10247401918624	-2.9658295509358\\
-2.63451282620022	-3.14144828765701\\
-3.04232770706418	-3.24791623306647\\
-3.3549485141355	-3.3086392593448\\
-3.59681921560905	-3.33984201416901\\
-3.78652373158095	-3.35213895732227\\
-3.93761030978875	-3.35231435881987\\
-4.05982011941077	-3.34467762409983\\
-4.16015857488049	-3.33197049824379\\
-4.24370077268982	-3.31594426507581\\
-4.31416325657116	-3.29772045747233\\
-4.37430112835846	-3.27801574968895\\
-4.42618242695195	-3.25728289561473\\
-4.47137831709626	-3.23579978582744\\
-4.51109590643123	-3.21372620312733\\
-4.54627186749189	-3.19114022325021\\
-4.5776390756402	-3.16806158910986\\
-4.60577446408777	-3.14446658671418\\
-4.63113362443092	-3.1202972373356\\
-4.65407589608681	-3.09546655614066\\
-4.67488248461812	-3.06986095178151\\
-4.69376932377597	-3.04334039813289\\
-4.71089581438933	-3.01573670290691\\
-4.72637014416942	-2.98684996782189\\
-4.74025155221982	-2.95644314047749\\
-4.7525496017215	-2.92423436818679\\
-4.76322022105016	-2.88988665116764\\
-4.77215792256101	-2.85299402715322\\
-4.77918315360532	-2.81306316485667\\
-4.78402309752955	-2.76948874955416\\
-4.78628330477081	-2.72152033843684\\
-4.78540610917088	-2.66821734199282\\
-4.78060957357864	-2.60838730064551\\
-4.77079723101516	-2.54050046645771\\
-4.7544233768589	-2.46257061045725\\
-4.72928994097639	-2.37198772160957\\
-4.69223734416034	-2.26528291109474\\
-4.63867132226216	-2.13780063274324\\
-4.56184012363635	-1.98325319303706\\
-4.45174957172697	-1.79315260240838\\
-4.29361423929005	-1.55619623735213\\
-4.06591436106207	-1.25792352950098\\
-3.73878545119878	-0.881544558184724\\
-3.27520921049671	-0.411907453801613\\
-2.64040735696895	0.154629335400547\\
-1.82504978681276	0.794725831716961\\
-0.874286234966954	1.44888122767422\\
0.110528333002808	2.0389454945846\\
1.01821992348271	2.50790754880763\\
1.7810827303388	2.84291284876407\\
2.38588406134361	3.06403611085734\\
2.8520687344207	3.20168452259337\\
3.20887045341322	3.28282783173843\\
3.48343404570952	3.32715821301171\\
3.69724199086767	3.34785822823937\\
3.86621004882867	3.35342706786038\\
};
\addplot [color=mycolor1, forget plot]
  table[row sep=crcr]{%
3.24802072526657	3.08898657669455\\
3.3897758927434	3.08463098294626\\
3.50717780454169	3.07353093809921\\
3.60555965346185	3.05787447765259\\
3.68893799237806	3.0391101092248\\
3.7603551517977	3.01819389413152\\
3.82213347265528	2.99575175340309\\
3.87606096630307	2.97218590476173\\
3.92352615805759	2.9477448044097\\
3.96561606674019	2.92256922878291\\
4.00318764128468	2.89672264545511\\
4.03692007441669	2.87021111198516\\
4.06735325513992	2.84299606912493\\
4.09491606564018	2.81500218748499\\
4.11994711853208	2.78612163966091\\
4.14270973683786	2.75621564574238\\
4.16340240243195	2.72511377835292\\
4.18216546548543	2.69261124842913\\
4.19908456276707	2.65846418064867\\
4.21419089312556	2.62238269572208\\
4.22745820611286	2.58402141962884\\
4.23879603612072	2.54296681275648\\
4.24803831449298	2.4987204280975\\
4.25492595624999	2.45067683506364\\
4.25908126159783	2.39809444370101\\
4.2599708704382	2.34005678287332\\
4.25685237487358	2.27542086782332\\
4.2486972575211	2.20274808738622\\
4.23407919947639	2.1202115526856\\
4.21101150918225	2.02547225932618\\
4.17671002598232	1.91551539532524\\
4.12724851717392	1.78643957729673\\
4.05706479291633	1.63320051076463\\
3.95827691705199	1.44933768874049\\
3.8198074555026	1.2267818260611\\
3.62645740478369	0.955994004666752\\
3.35845471980742	0.626973491313833\\
2.99278339242373	0.232044799412283\\
2.50858313759176	-0.228662052705863\\
1.89851323643394	-0.739765058077636\\
1.18289867440378	-1.26624364600204\\
0.414474409831214	-1.76017377325569\\
-0.337616378172249	-2.17915311760037\\
-1.0157375690553	-2.50268408552043\\
-1.59079943986477	-2.73365860485437\\
-2.06006652098701	-2.88843216535393\\
-2.43571632063607	-2.98641656949835\\
-2.73477077015239	-3.04444784505154\\
-2.97359690442859	-3.07522108427288\\
-3.16587677526447	-3.08766094281896\\
-3.32233866490942	-3.08782658981517\\
-3.45115212667339	-3.07976629512333\\
-3.5584688334254	-3.06616779092057\\
-3.6489146448141	-3.04881172004382\\
-3.7259821618184	-3.02887558897982\\
-3.7923266830378	-3.0071341439782\\
-3.84998372656887	-2.98409079310284\\
-3.90052729064399	-2.96006385616357\\
-3.94518476284576	-2.93524332188682\\
-3.98492053793448	-2.90972826726283\\
-4.02049711639897	-2.88355147382037\\
-4.05251993928572	-2.85669544050908\\
-4.0814703778279	-2.82910249060896\\
-4.10772998172893	-2.80068069715055\\
-4.13159815263183	-2.77130671125858\\
-4.15330473483526	-2.74082614489049\\
-4.17301851823932	-2.70905185267998\\
-4.19085226555785	-2.67576022356852\\
-4.20686455867674	-2.64068539442126\\
-4.22105846772521	-2.60351110647912\\
-4.23337674350934	-2.56385971649354\\
-4.24369287854316	-2.52127762252951\\
-4.25179692277218	-2.47521604069441\\
-4.25737430662315	-2.42500563752001\\
-4.25997501365807	-2.36982293811103\\
-4.2589691055279	-2.30864563778352\\
-4.25348260730029	-2.24019288784219\\
-4.24230478577604	-2.16284527211018\\
-4.22375345868195	-2.0745376090272\\
-4.19547866566049	-1.97261627352369\\
-4.1541765623212	-1.85365258466172\\
-4.0951757525708	-1.71320800031408\\
-4.01185252086737	-1.54556278064905\\
-3.8948460055357	-1.34346386421076\\
-3.73112332047878	-1.09805241778276\\
-3.50318686910388	-0.79934673966639\\
-3.18928479391729	-0.438006532001369\\
-2.7664475480644	-0.00939589426369803\\
-2.2188015897552	0.479649837256682\\
-1.551283722985	1.00395488422053\\
-0.801025192159766	1.52030488372181\\
-0.0322148105001834	1.98092643936069\\
0.688464523498724	2.35312769590979\\
1.3168168305411	2.62889283803365\\
1.83810593029394	2.81933767520858\\
2.2585572322313	2.94337777355454\\
2.59374360323828	3.0195352365236\\
2.86078600862169	3.06260564085484\\
3.07481850245181	3.08329773658049\\
3.24802072526657	3.08898657669455\\
};
\addplot [color=mycolor1, forget plot]
  table[row sep=crcr]{%
2.77439331404181	2.93091302742763\\
2.92201679669116	2.92636277086672\\
3.04637157876297	2.91459504312614\\
3.15208826828026	2.89776376165187\\
3.2427841339122	2.87734691491567\\
3.3212839387362	2.85435203999827\\
3.38979992259227	2.82945891805353\\
3.45007259499269	2.80311756223426\\
3.50347860095776	2.77561515658961\\
3.55111252006413	2.74712164781818\\
3.59384861636267	2.71772067013067\\
3.63238736330016	2.6874303278905\\
3.66729043889089	2.65621687058045\\
3.69900694701284	2.62400327690954\\
3.72789287992069	2.59067406893071\\
3.75422526400935	2.55607719458439\\
3.77821198551289	2.52002347276501\\
3.79999793695132	2.48228383665412\\
3.81966782478975	2.44258440184194\\
3.83724570330605	2.40059919840696\\
3.85269101930178	2.35594021887548\\
3.86589063488885	2.30814422775121\\
3.87664590262258	2.25665553496582\\
3.88465334892666	2.2008036375973\\
3.8894768110636	2.13977426643563\\
3.89050787933548	2.07257193041771\\
3.8869101016783	1.99797155235903\\
3.87754047298414	1.91445631756412\\
3.86083913397739	1.82013864395438\\
3.83467496285018	1.71266178182518\\
3.79613135228652	1.58908219207822\\
3.74121472048687	1.44574010822192\\
3.66447316001772	1.27814261737485\\
3.55853593268374	1.08091920886736\\
3.41365109965799	0.847977226098818\\
3.21745098500263	0.573094105004352\\
2.95545729312942	0.251310050265984\\
2.61320031683666	-0.118511197166621\\
2.18088500636221	-0.530058184913609\\
1.66037168352198	-0.966331273249687\\
1.07123681999009	-1.39989519993898\\
0.45011847127832	-1.79917692173613\\
-0.159383005627218	-2.13866734081843\\
-0.720633271641079	-2.40633549571523\\
-1.21252016464745	-2.60379224476195\\
-1.62935814917913	-2.74118014520894\\
-1.97576460223835	-2.83146612569576\\
-2.26115887990112	-2.8867963145686\\
-2.49603422430993	-2.91702541889677\\
-2.69007033677433	-2.9295545520092\\
-2.85144939310223	-2.92970835120029\\
-2.98678676772916	-2.92122770039677\\
-3.10131263512495	-2.90670689464467\\
-3.19912099480982	-2.88793146665785\\
-3.28340775081399	-2.86612289143199\\
-3.35667147783187	-2.84211015751369\\
-3.42087317426777	-2.81644812355084\\
-3.47755971760584	-2.78949854229469\\
-3.52795787563839	-2.76148532437063\\
-3.57304540203354	-2.73253211436099\\
-3.61360464511784	-2.70268768525832\\
-3.65026290958858	-2.6719428606411\\
-3.6835227709447	-2.64024144169147\\
-3.71378470478819	-2.60748677533737\\
-3.74136374106741	-2.57354502183569\\
-3.76650134914323	-2.53824577448667\\
-3.78937336346313	-2.50138038823889\\
-3.81009443551002	-2.4626981440497\\
-3.82871921349276	-2.42190018056349\\
-3.84524017711625	-2.37863093987127\\
-3.85958176020026	-2.33246667967303\\
-3.87159004412699	-2.28290038161987\\
-3.88101685660539	-2.2293221170647\\
-3.88749650543373	-2.17099360012175\\
-3.89051253781423	-2.10701525198191\\
-3.88935073818355	-2.03628362365249\\
-3.88303293072704	-1.95743652160941\\
-3.87022389757597	-1.8687827950562\\
-3.84910078373677	-1.76821383653104\\
-3.81717092845662	-1.65309525059751\\
-3.77102113456042	-1.52014164967862\\
-3.70598200889864	-1.36528878398867\\
-3.61570295942839	-1.1836020632259\\
-3.49167419272179	-0.969310283789998\\
-3.32283524579673	-0.716141548030639\\
-3.09562346027749	-0.418264181886093\\
-2.79515779611711	-0.0722301614851879\\
-2.40854550978567	0.319857789234741\\
-1.93088423023517	0.746616654536766\\
-1.37241539502882	1.18544283033955\\
-0.761944613841044	1.60567594363182\\
-0.141318421918017	1.97750271825346\\
0.447759648128476	2.28165213535834\\
0.975915141814334	2.51333247222445\\
1.43020225350868	2.67919557517685\\
1.81083524976017	2.79140654950542\\
2.12542789170596	2.8628250344706\\
2.38427515070782	2.90453159423087\\
2.59760801138153	2.92512684507289\\
2.77439331404181	2.93091302742763\\
};
\addplot [color=mycolor1, forget plot]
  table[row sep=crcr]{%
2.40672193448848	2.85099006036898\\
2.55946400860147	2.84626734990196\\
2.69033064347883	2.8338724347951\\
2.80322644467541	2.81588983864324\\
2.90131847641568	2.79380159691169\\
2.98715982076598	2.76865112711042\\
3.06280469611034	2.74116390876034\\
3.1299069409141	2.71183448715012\\
3.18980061012433	2.68098840470885\\
3.24356430475479	2.6488258944964\\
3.29207175671717	2.61545243483868\\
3.33603121277982	2.58089983906641\\
3.37601584075997	2.54514046966064\\
3.41248696586263	2.50809636937505\\
3.44581153967493	2.46964452440797\\
3.47627488404781	2.42961905478184\\
3.50408944080944	2.38781081578398\\
3.52939998615142	2.34396465460018\\
3.55228552002567	2.29777437053471\\
3.57275779753005	2.24887525477032\\
3.59075621010789	2.1968339208358\\
3.60613842512094	2.14113496828313\\
3.61866582373807	2.08116384252486\\
3.62798230340243	2.01618506341848\\
3.63358438976255	1.94531480750321\\
3.63477978748752	1.86748668408962\\
3.63063045232983	1.78140953722261\\
3.61987499455317	1.68551643113255\\
3.60082386156014	1.57790502980948\\
3.57121974908707	1.45627211371012\\
3.52805622590764	1.31785035715285\\
3.46735228172388	1.1593660453844\\
3.38389486753738	0.977055647018671\\
3.27099539636765	0.766811080918345\\
3.1203742964426	0.524568862670887\\
2.92240239086533	0.247103233007234\\
2.66707034700497	-0.0666239051903698\\
2.34611261882098	-0.413576567700192\\
1.9564054111336	-0.78471495934714\\
1.50379513917584	-1.16420787544064\\
1.00519331949073	-1.53123118839722\\
0.486558707776251	-1.86465454960816\\
-0.0233418928124208	-2.14863449799717\\
-0.500224441331953	-2.37600079694935\\
-0.92876000791341	-2.54795215360327\\
-1.30292325563112	-2.67120671237276\\
-1.62364176742573	-2.75474212836996\\
-1.89581714913953	-2.80746710454253\\
-2.12595426693129	-2.8370548462651\\
-2.32069791209789	-2.84960647951803\\
-2.48610350192749	-2.84974705246401\\
-2.62736899749821	-2.84088225899685\\
-2.74881112519193	-2.82547500293986\\
-2.85395039459688	-2.80528503157114\\
-2.94563164671655	-2.78155750762754\\
-3.02614509779226	-2.75516411941453\\
-3.09733360141841	-2.72670575288099\\
-3.16068210115339	-2.69658598894361\\
-3.21738980476314	-2.66506317591553\\
-3.26842731238166	-2.6322870135419\\
-3.314581298808	-2.59832399127148\\
-3.35648915611748	-2.56317477226966\\
-3.39466561513653	-2.52678568243184\\
-3.42952294651184	-2.48905578494707\\
-3.46138595739844	-2.44984052919454\\
-3.49050266464898	-2.40895260246941\\
-3.51705123538792	-2.36616034149742\\
-3.54114352770935	-2.32118384593142\\
-3.5628253207578	-2.27368875417122\\
-3.58207307499217	-2.22327747492459\\
-3.59878678753142	-2.16947750238967\\
-3.61277817756754	-2.11172626935765\\
-3.62375302059351	-2.04935180693829\\
-3.63128590837492	-1.98154828765504\\
-3.63478499888611	-1.90734535493057\\
-3.63344339233583	-1.82557004893318\\
-3.62617260376615	-1.73480026336293\\
-3.61151225487164	-1.63330929028357\\
-3.58750883853339	-1.5190026775697\\
-3.55155596273051	-1.38935235741425\\
-3.50019064181572	-1.2413406084902\\
-3.42884882418522	-1.07144080442315\\
-3.3316059040818	-0.875687035934242\\
-3.20097695900251	-0.649923754453545\\
-3.02794219877558	-0.390375127140654\\
-2.80249808934361	-0.094702634584102\\
-2.51515717546199	0.236352716673745\\
-2.15973732910717	0.596958360645233\\
-1.73716603044673	0.974644700231005\\
-1.25877548178621	1.35066068275713\\
-0.746556765376957	1.70331415643677\\
-0.228820724963049	2.01349036870507\\
0.267101046683619	2.26948919152698\\
0.721110320794781	2.46857197377776\\
1.12270709349533	2.61512503245354\\
1.46971763786224	2.71736211276299\\
1.76540696891737	2.78444046005549\\
2.01571702780733	2.82473480138731\\
2.22735322408389	2.84513919980508\\
2.40672193448848	2.85099006036898\\
};
\addplot [color=mycolor1, forget plot]
  table[row sep=crcr]{%
2.11655100682591	2.82843428941261\\
2.27354611471052	2.82356578471402\\
2.41025665966359	2.81060627378123\\
2.52989546066145	2.79154091646479\\
2.63516721135814	2.76782903818278\\
2.72832261743308	2.74053005988702\\
2.81122263331923	2.71040195994312\\
2.88540101761024	2.67797585950846\\
2.95212010046617	2.64361138065595\\
3.01241816078134	2.60753712579265\\
3.06714850265728	2.56987987342384\\
3.11701103132388	2.530685279376\\
3.16257733553198	2.48993216479353\\
3.20431024779221	2.44754189978502\\
3.2425787120241	2.40338394581532\\
3.27766860712889	2.35727827727643\\
3.30978998395011	2.30899513789073\\
3.33908098109372	2.25825237864351\\
3.36560849027125	2.20471045325069\\
3.38936543605966	2.14796500184279\\
3.41026430598958	2.08753682547208\\
3.42812629933144	2.02285894077584\\
3.44266513926722	1.95326031123863\\
3.4534641954334	1.87794579759002\\
3.4599450789137	1.79597189595953\\
3.46132530302978	1.70621801868249\\
3.45656199668418	1.60735356718522\\
3.44427815481841	1.49780211181684\\
3.4226678564112	1.37570607723411\\
3.38937801254489	1.23889915632012\\
3.34136798234704	1.08490031045564\\
3.27475744922555	0.910953996822627\\
3.18469151120958	0.71415720554415\\
3.06528542191618	0.491733786396986\\
2.90976311744834	0.241532573386507\\
2.71096460636707	-0.0371839138929645\\
2.4624261097362	-0.342670722307781\\
2.16014017496843	-0.6695545814815\\
1.80476531683354	-1.00810768327461\\
1.40349383662699	-1.34464591286619\\
0.97038036758317	-1.66351810247334\\
0.524318870350239	-1.95029936328905\\
0.085159888355619	-2.19486019379004\\
-0.330303447862874	-2.39290064579353\\
-0.710776629391997	-2.54551613559592\\
-1.05080767266494	-2.65747804245967\\
-1.34965578360068	-2.7352745046498\\
-1.60964878916396	-2.78560491585113\\
-1.83470724848301	-2.81451245994852\\
-2.02929746385226	-2.82703310171971\\
-2.19780572461383	-2.82716004636509\\
-2.34422376368673	-2.81795933766786\\
-2.47202961915923	-2.80173490909645\\
-2.58417664940125	-2.78019148732509\\
-2.68313477180989	-2.75457447721281\\
-2.77095171368762	-2.72578186937574\\
-2.84931736926041	-2.69445020124993\\
-2.91962332499729	-2.66101894855914\\
-2.98301449583141	-2.625777943375\\
-3.0404322803138	-2.5889018140239\\
-3.0926497659045	-2.55047463225179\\
-3.14029993083631	-2.51050718663972\\
-3.18389785141002	-2.46894866039765\\
-3.22385782311156	-2.42569398502503\\
-3.26050613704066	-2.38058775033286\\
-3.29409006498849	-2.3334252506136\\
-3.32478341468164	-2.28395101236323\\
-3.35268882400787	-2.23185496103844\\
-3.37783676405674	-2.17676622764621\\
-3.4001810051745	-2.11824445995762\\
-3.41959005405378	-2.05576838234722\\
-3.43583377655189	-1.98872124382334\\
-3.44856406235342	-1.91637271697959\\
-3.45728794729811	-1.83785678961846\\
-3.46133108014034	-1.75214528356065\\
-3.45978882279539	-1.65801695123747\\
-3.45146169164846	-1.55402283821256\\
-3.43477150758744	-1.43845010349195\\
-3.40765503359	-1.3092893324724\\
-3.36743408852029	-1.16421545119853\\
-3.31066711786081	-1.00060088864916\\
-3.23300037864963	-0.815592941179428\\
-3.12906223935289	-0.606305635533384\\
-2.99248669002533	-0.370196004944904\\
-2.81621104383475	-0.105701784839631\\
-2.59324586139854	0.186819618571972\\
-2.31809609238649	0.503941380582688\\
-1.98880682986173	0.838150307963187\\
-1.60913736983443	1.17759478204281\\
-1.18981009540952	1.5072611728684\\
-0.747698446788933	1.81168182170006\\
-0.30271621218734	2.07826646569063\\
0.126382115899771	2.29973691404187\\
0.525382985478411	2.47465037794311\\
0.885985247779934	2.60619273420778\\
1.20528688360715	2.70022002649834\\
1.48428693993919	2.7634737488266\\
1.72626932379742	2.80239670394494\\
1.93552996856662	2.8225480577959\\
2.11655100682591	2.82843428941261\\
};
\addplot [color=mycolor1, forget plot]
  table[row sep=crcr]{%
1.8839109593124	2.84837258818049\\
2.04438503565292	2.84338280475722\\
2.18623968056034	2.82992488077221\\
2.31207075292011	2.80986400883025\\
2.42414288002402	2.78461324524576\\
2.524399227141	2.75522744745394\\
2.6144895971911	2.72248123179932\\
2.69580523727765	2.68693097387905\\
2.76951401643886	2.64896275152078\\
2.83659283040392	2.60882865184404\\
2.89785590868346	2.56667375911347\\
2.95397867354868	2.52255579514719\\
3.00551727602874	2.47645898572703\\
3.05292412571317	2.42830335512052\\
3.09655976598816	2.37795033370079\\
3.13670139641996	2.32520530362994\\
3.17354824791263	2.26981749794226\\
3.20722389261628	2.21147750004641\\
3.23777542608711	2.14981245574515\\
3.26516929363055	2.08437900355741\\
3.28928334186076	2.01465385142123\\
3.30989445454165	1.94002188583742\\
3.32666087447042	1.85976171107942\\
3.33909802279223	1.77302861621705\\
3.34654632203126	1.67883521851811\\
3.3481292617274	1.57603053813278\\
3.34269983548856	1.46327919049815\\
3.32877377018421	1.33904400156975\\
3.30444912628243	1.20157803439204\\
3.26731469495041	1.04893624597627\\
3.21435548933653	0.879023236256821\\
3.14187443310187	0.689701858699571\\
3.04546717651756	0.478996523364139\\
2.92011247449501	0.245430336841415\\
2.7604689387904	-0.0114734525998377\\
2.5614829365389	-0.290535148120325\\
2.31937579911317	-0.588206229077695\\
2.03294464193934	-0.898035058199291\\
1.70486833862432	-1.21066306833375\\
1.34246114266892	-1.51466922851812\\
0.957298419245449	-1.7982752131383\\
0.563544258574439	-2.05143637377369\\
0.17548435714616	-2.26752728383446\\
-0.194793772703191	-2.44400064822539\\
-0.538789710898925	-2.58194925525039\\
-0.851840571636949	-2.68499183533562\\
-1.13253788137548	-2.75803083744598\\
-1.38178777086944	-2.80625377472262\\
-1.60188666155828	-2.83450133180103\\
-1.79579559551911	-2.84695953436325\\
-1.96665021039329	-2.84707329656296\\
-2.11746986643662	-2.83758398914435\\
-2.25100884799222	-2.82062207494757\\
-2.3696979777122	-2.79781403418103\\
-2.47563843265873	-2.77038304485132\\
-2.57062256708164	-2.73923512802462\\
-2.65616642914844	-2.70502896323499\\
-2.73354530585091	-2.66823056893121\\
-2.80382776856561	-2.62915512247521\\
-2.86790611868862	-2.58799833682481\\
-2.92652247610629	-2.5448595531361\\
-2.98029044423412	-2.49975832182142\\
-3.02971259851557	-2.45264585394876\\
-3.07519414669671	-2.40341237922406\\
-3.11705309513639	-2.35189115869908\\
-3.15552717907817	-2.29785966662167\\
-3.19077770350021	-2.24103826799102\\
-3.22289030677915	-2.18108656759356\\
-3.25187250472187	-2.11759748601923\\
-3.27764769496605	-2.05008902577164\\
-3.3000450960285	-1.97799362927374\\
-3.31878485592477	-1.90064501222353\\
-3.33345729045611	-1.81726240582165\\
-3.34349490965776	-1.72693230686436\\
-3.34813559491003	-1.62858819671687\\
-3.34637507943646	-1.52098938281574\\
-3.33690693424297	-1.40270135184946\\
-3.3180489099609	-1.27208212003948\\
-3.28765635604558	-1.12728245865763\\
-3.24302759373264	-0.966273063059091\\
-3.18081418964464	-0.786919056262009\\
-3.09696316231815	-0.587131211462139\\
-2.98673994755229	-0.365131324720503\\
-2.84490907694386	-0.119869220079494\\
-2.66617395415224	0.148392918344402\\
-2.44597148905526	0.437376123390119\\
-2.18163739288019	0.742122945956752\\
-1.87376584258502	1.05468104394214\\
-1.52731875729217	1.36449608541033\\
-1.15187248617585	1.65971438859468\\
-0.760574942281166	1.92916920291734\\
-0.367973512747125	2.16436994990962\\
0.0125009114487025	2.36072200208788\\
0.370454450605222	2.51760845619305\\
0.69933719935473	2.637544049508\\
0.996219485266728	2.7249350397717\\
1.2609731326353	2.7849286865175\\
1.49530475588198	2.82259567527393\\
1.70191874368275	2.84247146195521\\
1.8839109593124	2.84837258818049\\
};
\addplot [color=mycolor1, forget plot]
  table[row sep=crcr]{%
1.69485018192866	2.89996807890675\\
1.85820655715354	2.89487647765564\\
2.00457984397636	2.8809797270943\\
2.13604336602734	2.86001243183153\\
2.25446721546826	2.83332346706828\\
2.36150279435773	2.80194460779108\\
2.45858756385122	2.76665089554655\\
2.54696018123358	2.72801094037392\\
2.62767995343108	2.68642736149034\\
2.70164705308327	2.64216844335124\\
2.76962154304058	2.59539234248843\\
2.83224022351806	2.54616513680408\\
2.89003087130571	2.49447383688251\\
2.94342373414113	2.44023526897727\\
2.99276027509175	2.38330153487777\\
3.03829919395036	2.32346257290518\\
3.08021972265549	2.26044619262228\\
3.11862212089304	2.19391583437811\\
3.15352519751962	2.1234662139144\\
3.18486055866431	2.04861695467206\\
3.21246313728203	1.96880429383634\\
3.23605739555129	1.88337098822631\\
3.25523842050502	1.79155467072672\\
3.2694469773572	1.69247516304914\\
3.27793749214014	1.58512170826644\\
3.27973799910182	1.46834185451507\\
3.27360148014203	1.34083495111\\
3.25794903920057	1.20115510414749\\
3.23080747248973	1.04773118604595\\
3.18974772417294	0.878915217825194\\
3.13183736099263	0.693074924434428\\
3.05363038060174	0.488750457842159\\
2.95123130069541	0.264896523752804\\
2.82048480477904	0.0212242347237356\\
2.65734908593136	-0.241365893412844\\
2.45849463889622	-0.520314844299026\\
2.22211017793365	-0.811023372373281\\
1.94878415697773	-1.10674648214036\\
1.6421946299421	-1.39895948782619\\
1.30927104075118	-1.67827753077494\\
0.959596009127865	-1.9357781217093\\
0.604107705613491	-2.1643428143617\\
0.25348980666344	-2.35957492855925\\
-0.0832302884218608	-2.5200350160566\\
-0.399501438849518	-2.64684060282257\\
-0.691411116616583	-2.74289832883057\\
-0.957357474203116	-2.81207428785222\\
-1.19748394163742	-2.85850998165818\\
-1.41308626186256	-2.88616140914034\\
-1.60611127361413	-2.89854677647915\\
-1.77878771692437	-2.89864837370952\\
-1.93338211754963	-2.88891044962224\\
-2.07205337146284	-2.87128738842466\\
-2.19677689735299	-2.84731199527596\\
-2.30931382793077	-2.81816645646392\\
-2.41120717242651	-2.784747251869\\
-2.50379272061424	-2.74772055487814\\
-2.58821691441871	-2.70756750439545\\
-2.66545700910847	-2.66462009255071\\
-2.73634086543492	-2.61908892742578\\
-2.80156496636929	-2.57108420877264\\
-2.86170999100684	-2.52063113127606\\
-2.9172536873189	-2.46768073119723\\
-2.96858098836634	-2.41211698161585\\
-3.01599139254191	-2.35376074751745\\
-3.05970362599114	-2.29237104537152\\
-3.09985755302369	-2.2276439152803\\
-3.13651321368978	-2.1592091074652\\
-3.16964675459914	-2.08662471002761\\
-3.19914288327941	-2.00936980644851\\
-3.22478332091583	-1.92683526041964\\
-3.24623055931555	-1.83831280347719\\
-3.26300606042059	-1.7429827835832\\
-3.27446190405073	-1.63990127816714\\
-3.27974485941134	-1.52798787187569\\
-3.27775205581099	-1.40601637633116\\
-3.26707808478902	-1.27261230077595\\
-3.24595485914421	-1.12626317540111\\
-3.21218847434543	-0.965351063463567\\
-3.16310248510393	-0.788220765195193\\
-3.09550534797973	-0.593301764250026\\
-3.0057118108052	-0.379305157815972\\
-2.88966262082075	-0.145514809122397\\
-2.7431989937066	0.10782186103107\\
-2.5625459681075	0.379032043483445\\
-2.34502296408943	0.664572021693395\\
-2.08991266545217	0.958757334829479\\
-1.79928738607017	1.25387241352455\\
-1.47847542184976	1.54081565734261\\
-1.13585696179675	1.81025596046772\\
-0.781886175584397	2.05402234197167\\
-0.427573956485562	2.26628262034397\\
-0.0829201636283366	2.44413355176673\\
0.244226294998288	2.58749481622467\\
0.548649664549752	2.69848476061002\\
0.827654445755815	2.78058761635831\\
1.0805877294045	2.83787921004732\\
1.3082392155596	2.87445182827387\\
1.51228482589137	2.89406303166017\\
1.69485018192866	2.89996807890675\\
};
\addplot [color=mycolor1, forget plot]
  table[row sep=crcr]{%
1.5395140061444	2.97500957994766\\
1.70534410082583	2.96983000906519\\
1.85572707401358	2.95554327789407\\
1.9923107530755	2.93375139112282\\
2.11662750506519	2.90572748918844\\
2.23006643917711	2.87246545096575\\
2.33386447233891	2.83472602671762\\
2.42910859408973	2.79307698683249\\
2.51674415445317	2.74792657593276\\
2.59758582791122	2.69955050273766\\
2.67232917150208	2.64811311550725\\
2.74156152853298	2.59368353952608\\
2.80577155393159	2.53624752869216\\
2.8653569492562	2.47571569084652\\
2.92063016456976	2.41192863217861\\
2.97182189850544	2.34465945447924\\
3.01908223835145	2.27361394319377\\
3.06247924904098	2.19842871193207\\
3.10199475663549	2.11866752710578\\
3.13751698776756	2.03381603362686\\
3.16882963099094	1.94327515304639\\
3.19559679248079	1.84635355110563\\
3.21734325007464	1.74225980594756\\
3.23342940734364	1.63009530021964\\
3.2430204850141	1.50884947785317\\
3.24504988117975	1.37740003626802\\
3.2381774770256	1.23452196372132\\
3.22074524976044	1.07891115508892\\
3.19073527136989	0.909230628141407\\
3.14573946868521	0.724189852387416\\
3.08295671749822	0.522669626245742\\
2.99924068776572	0.303904659505936\\
2.89122968292588	0.0677307008671348\\
2.75559327064357	-0.185111188979598\\
2.58942210103305	-0.452647179329826\\
2.39075715043921	-0.731391841739698\\
2.15919697674414	-1.01622745388764\\
1.89644649909105	-1.30056297494084\\
1.60661428118054	-1.57684933779702\\
1.29608038339813	-1.83741477251027\\
0.972874658231894	-2.07544116347108\\
0.645689164156102	-2.2858126665623\\
0.32279690444423	-2.46560003908645\\
0.0111702639346649	-2.61408822075381\\
-0.284018382596925	-2.73242298820054\\
-0.559493616072948	-2.82305325505413\\
-0.813673044702863	-2.88914946235601\\
-1.04630650619837	-2.93411858390528\\
-1.25808172423759	-2.96126352480465\\
-1.45027307466865	-2.97358176151219\\
-1.62446613306611	-2.97367251087915\\
-1.78236127963077	-2.96371661956631\\
-1.92564490571912	-2.94549878857511\\
-2.05591214294895	-2.92045034564002\\
-2.17462581182947	-2.88969872908586\\
-2.28309919511411	-2.8541158283054\\
-2.38249348967667	-2.8143613010691\\
-2.47382360087399	-2.77091938932807\\
-2.55796809608302	-2.72412907714267\\
-2.63568066532851	-2.67420807934244\\
-2.70760146949569	-2.62127140011584\\
-2.77426742181663	-2.56534523839753\\
-2.83612085628008	-2.50637695086907\\
-2.89351626999702	-2.44424167588688\\
-2.94672494338748	-2.37874610655234\\
-2.99593728124691	-2.30962979626666\\
-3.04126270453835	-2.23656429522712\\
-3.08272687330215	-2.15915035826483\\
-3.12026594626481	-2.07691344069961\\
-3.15371749158923	-1.98929772057509\\
-3.18280756657785	-1.89565897027643\\
-3.20713339939221	-1.79525677540025\\
-3.22614106377603	-1.68724690463044\\
-3.23909759290579	-1.57067512940798\\
-3.24505722438492	-1.44447455408784\\
-3.24282205851402	-1.30746963988646\\
-3.23089858435342	-1.15839168124217\\
-3.20745362432094	-0.995912562443984\\
-3.17027670028752	-0.818706068731119\\
-3.11676105244137	-0.625548373771311\\
-3.04392265811155	-0.415470422042853\\
-2.94848477357015	-0.18797256088483\\
-2.82706195131493	0.0566974860419996\\
-2.67647622903883	0.317222338979823\\
-2.4942200968804	0.590900251243522\\
-2.27903699278498	0.8734299226913\\
-2.0315209511659	1.15891560518726\\
-1.75456510124819	1.44020027978298\\
-1.45346149726645	1.70955456555951\\
-1.13552077609128	1.95961285942646\\
-0.809240575052896	2.18432063074294\\
-0.48323218558479	2.37962329690684\\
-0.165206978506065	2.54372251001265\\
0.138726712936532	2.67689519113288\\
0.424354947668622	2.78101359892693\\
0.689287467491502	2.85895599889901\\
0.932657573845497	2.91406305971143\\
1.15473186815185	2.94972304820235\\
1.35653090954332	2.96910368814714\\
1.5395140061444	2.97500957994766\\
};
\addplot [color=mycolor1, forget plot]
  table[row sep=crcr]{%
1.41082729830732	3.06696346341668\\
1.57888188824117	3.06170488517268\\
1.73288335955502	3.04706602621272\\
1.87414312688596	3.02452072980028\\
2.00391454723165	2.99526075284613\\
2.12336056424584	2.96023162134034\\
2.2335376505108	2.92016776505852\\
2.33539028955245	2.87562431982674\\
2.42975183496754	2.82700448312965\\
2.51734885600281	2.77458217793819\\
2.59880702224616	2.71852023752744\\
2.67465724389923	2.65888453001565\\
2.74534123038429	2.59565450298443\\
2.81121591551885	2.52873061463361\\
2.87255636770134	2.45793907089664\\
2.92955689257045	2.38303423364092\\
2.98233006803905	2.30369901990077\\
3.0309034452899	2.21954358802503\\
3.07521361811412	2.13010261489511\\
3.1150973197344	2.03483152232268\\
3.15027916623443	1.93310212832522\\
3.1803556508972	1.82419840505262\\
3.20477503873642	1.70731335329514\\
3.22281297056105	1.58154849661164\\
3.23354394743114	1.44591820544066\\
3.23580955861631	1.29936202832266\\
3.22818552015256	1.14076944943835\\
3.20895153448483	0.969022943927854\\
3.17607089684429	0.783066626566984\\
3.12719077837387	0.582008631504728\\
3.05967895816599	0.365264576686387\\
2.9707174033871	0.132745404909382\\
2.85747509860654	-0.114916532506813\\
2.71737778747114	-0.376125734509712\\
2.54847564470256	-0.648111497870774\\
2.34987790098481	-0.92681406401126\\
2.12217966549545	-1.20694813391023\\
1.86776648471642	-1.48230420879851\\
1.59087330425521	-1.74629029409601\\
1.2973196913592	-1.99263140219444\\
0.993939257680404	-2.21607026299848\\
0.687827986807597	-2.41289475473447\\
0.385597020319771	-2.58117320154147\\
0.092798406562557	-2.72067983488085\\
-0.186386490639635	-2.83258573447756\\
-0.449198933908991	-2.91903531731147\\
-0.694162162043553	-2.98272022812905\\
-0.920836124759431	-3.0265233621924\\
-1.12954360328692	-3.05326227094516\\
-1.32111758772385	-3.06552945518219\\
-1.49669412922668	-3.06561076784676\\
-1.65755646084978	-3.05545885686464\\
-1.80502578545109	-3.03670100108363\\
-1.940389825552	-3.01066563361817\\
-2.0648596017813	-2.97841687486326\\
-2.17954605591969	-2.94079047915299\\
-2.28544987622582	-2.8984275184682\\
-2.38345960078487	-2.85180403442792\\
-2.47435451497848	-2.80125604253879\\
-2.55880996272381	-2.74699991262773\\
-2.63740348711924	-2.68914846490745\\
-2.7106207635212	-2.62772324420868\\
-2.77886064751763	-2.56266345208767\\
-2.84243888316447	-2.49383198198069\\
-2.90159014285708	-2.42101894941071\\
-2.95646812835036	-2.34394305783714\\
-3.00714347359018	-2.26225110474146\\
-3.05359916966103	-2.17551592317112\\
-3.09572319326319	-2.08323308317822\\
-3.13329797629251	-1.98481676073562\\
-3.16598632330039	-1.87959533992676\\
-3.19331339329362	-1.76680757652886\\
-3.21464445626887	-1.64560055585467\\
-3.22915838245868	-1.51503127165456\\
-3.23581732988064	-1.37407448527119\\
-3.23333401795761	-1.22164063012467\\
-3.22013951826568	-1.05660889000736\\
-3.19435689680321	-0.877882063447059\\
-3.15378950237545	-0.684471055268767\\
-3.09593718817678	-0.475617033070586\\
-3.01805869148634	-0.250957106870563\\
-2.91730211708451	-0.0107329121178832\\
-2.7909247167741	0.24397134333908\\
-2.63661310096762	0.510994758935142\\
-2.4528908367777	0.786927069310679\\
-2.23956137318445	1.06707387295504\\
-1.9980896606226	1.34563420623355\\
-1.73179791579767	1.61612676971752\\
-1.44576736588601	1.87202631518176\\
-1.14641122706676	2.10748617964671\\
-0.840792519535627	2.31797229868893\\
-0.535850622331257	2.50065357292545\\
-0.237723450808828	2.65447783145559\\
0.0487005982121677	2.77996645491535\\
0.319959039170133	2.87883253552236\\
0.573960279152549	2.95354414673502\\
0.8097798590146	3.00692708029006\\
1.02739289189082	3.04185725774207\\
1.22740630901176	3.06105425092449\\
1.41082729830732	3.06696346341668\\
};
\addplot [color=mycolor1, forget plot]
  table[row sep=crcr]{%
1.30361953236429	3.17036463029885\\
1.47376416606297	3.16503240495085\\
1.63108944317391	3.15007019639534\\
1.7766465540186	3.12683240291195\\
1.91146452429925	3.09642860492775\\
2.03651750336495	3.05974978176111\\
2.1527056347414	3.01749524388881\\
2.26084522591371	2.97019785848818\\
2.36166495027574	2.91824634527061\\
2.45580568463729	2.86190416805311\\
2.54382227054595	2.80132498247682\\
2.62618599578559	2.73656483097968\\
2.70328695170963	2.66759138181632\\
2.77543566424828	2.59429054405119\\
2.84286355022453	2.51647079143163\\
2.90572183974832	2.43386551933816\\
2.96407864894928	2.34613375805254\\
3.01791390089306	2.2528595865683\\
3.06711179070596	2.15355064791606\\
3.11145048955135	2.04763627519511\\
3.15058880159253	1.93446591646719\\
3.18404955764049	1.81330882011404\\
3.21119969208565	1.68335633779026\\
3.23122726951574	1.54372874757138\\
3.24311629403442	1.39348921404227\\
3.2456210665141	1.23166837496241\\
3.23724329235029	1.05730400136355\\
3.21621721229362	0.869501022271874\\
3.18051076818563	0.667517534924947\\
3.12785401697737	0.450881541738476\\
3.05580897864137	0.21954004163846\\
2.96189638125696	-0.0259644596040753\\
2.84379194918649	-0.284305861266947\\
2.69959498153774	-0.553205406163492\\
2.52815280101113	-0.829327537810989\\
2.32939746931219	-1.10829520263138\\
2.10462355751182	-1.38487132898864\\
1.85662197887327	-1.65332173850767\\
1.58960019902782	-1.90792269649361\\
1.30886852536418	-2.14352180023087\\
1.02034111000833	-2.35603101754621\\
0.729958305028817	-2.54274468527235\\
0.443155842501758	-2.70242930019275\\
0.164479140140104	-2.83519990861521\\
-0.10261427537527	-2.94224915593108\\
-0.355776568571634	-3.02551318243085\\
-0.593663001368655	-3.08734677455416\\
-0.815754331765168	-3.13025317655978\\
-1.022157458092	-3.15668638671042\\
-1.21341761347519	-3.16892386956687\\
-1.39035957818846	-3.16899711976838\\
-1.5539635035359	-3.15866437392518\\
-1.7052737242077	-3.13941095678695\\
-1.84533548233189	-3.11246576440986\\
-1.97515349264969	-3.07882568028455\\
-2.09566662361031	-3.03928254929481\\
-2.20773388699658	-2.99444947212902\\
-2.31212797355392	-2.94478466322395\\
-2.4095335250971	-2.89061207068761\\
-2.50054811227694	-2.83213853275303\\
-2.58568448087765	-2.76946756644291\\
-2.66537305925385	-2.70261004411595\\
-2.73996401639047	-2.63149207845245\\
-2.80972835476686	-2.55596045081979\\
-2.87485764102363	-2.47578591157516\\
-2.93546204161393	-2.39066467413014\\
-2.99156635732208	-2.30021843292352\\
-3.04310375458557	-2.20399327251553\\
-3.08990688777381	-2.10145791534584\\
-3.13169611264067	-1.99200189660985\\
-3.16806453130349	-1.87493447766496\\
-3.19845971881004	-1.74948544026469\\
-3.22216221323852	-1.61480937119357\\
-3.23826128142189	-1.46999567551543\\
-3.24562920526706	-1.3140873543616\\
-3.24289650054273	-1.14611251355545\\
-3.22843222084146	-0.965133505438952\\
-3.20033591008952	-0.770319259909192\\
-3.15645078689902	-0.561046193550949\\
-3.09441095916207	-0.337031235685504\\
-3.01173783361475	-0.098495821530246\\
-2.90600046968373	0.153648974685787\\
-2.77504864012543	0.417619031336771\\
-2.6173129607066	0.690614035340361\\
-2.43214277890123	0.968766232625099\\
-2.2201237012431	1.24723435609051\\
-1.98329388368493	1.52047682401108\\
-1.72517741052467	1.78269522605538\\
-1.4505859192436	2.0283827126093\\
-1.16520188757099	2.25286634840677\\
-0.875024671769417	2.45272336342021\\
-0.585801320529994	2.62598737981331\\
-0.302559157014001	2.7721257626261\\
-0.0293125372598301	2.89183212438409\\
0.231042498441196	2.9867134565383\\
0.476678738039672	3.05895312777444\\
0.7066887332391	3.11100953386938\\
0.920891090611341	3.14538139339833\\
1.11963378465583	3.16444637329992\\
1.30361953236428	3.17036463029885\\
};
\addplot [color=mycolor1, forget plot]
  table[row sep=crcr]{%
1.21404429523693	3.28044238815931\\
1.38621268985409	3.27503967250787\\
1.54663278389918	3.25977667517288\\
1.69615770955681	3.23589953360583\\
1.83563983114042	3.20443849817299\\
1.96589963992468	3.16622753638176\\
2.08770582995269	3.12192530571178\\
2.20176333666588	3.07203536120413\\
2.30870677728523	3.01692441575519\\
2.40909733174624	2.9568380968778\\
2.50342159874684	2.89191403251249\\
2.5920913452711	2.82219232605669\\
2.67544335036796	2.74762360551715\\
2.75373874229953	2.66807489620662\\
2.82716136067381	2.5833336011911\\
2.89581475873685	2.49310990092694\\
2.95971751088797	2.39703792042776\\
3.01879652122194	2.29467607384036\\
3.07287805593017	2.18550709727907\\
3.12167626436159	2.06893843718183\\
3.164779035279	1.944303891071\\
3.20163119093747	1.81086771903676\\
3.23151530030509	1.66783287357184\\
3.25353085916163	1.51435553835408\\
3.26657332103525	1.34956880393622\\
3.26931556015483	1.17261897059671\\
3.26019588174115	0.982718497951086\\
3.23741867330031	0.779219721340932\\
3.19897605515409	0.561712644588827\\
3.14270097103519	0.330147720524914\\
3.06636311552953	0.0849797754968957\\
2.96781741455216	-0.172678419947656\\
2.84520857548179	-0.440914677424546\\
2.69722299994229	-0.716920591449242\\
2.52336142265399	-0.996979099102004\\
2.32418586746018	-1.27657335860218\\
2.10148095611961	-1.55063605168546\\
1.85827225065407	-1.81392514924874\\
1.59866930506816	-2.06147264959836\\
1.32754437871465	-2.28902302570026\\
1.05010372376851	-2.49337372784765\\
0.771437131170872	-2.67255571357623\\
0.496130795580913	-2.82583698514912\\
0.228001336931145	-2.95357686792502\\
-0.0300302684969267	-3.05698655331028\\
-0.275940431953769	-3.1378566857644\\
-0.50851573553319	-3.19830070194832\\
-0.727216481893093	-3.24054311834333\\
-0.932024563568576	-3.26676350699635\\
-1.12329864221807	-3.27899388469962\\
-1.30164924092964	-3.27906033897986\\
-1.46783839273164	-3.26855755833863\\
-1.62270342414218	-3.24884563813068\\
-1.76710181361321	-3.22106052479423\\
-1.90187309514551	-3.18613172740119\\
-2.02781378660798	-3.14480294762885\\
-2.14566180775921	-3.0976528684895\\
-2.25608750024786	-3.0451144903595\\
-2.35968899977471	-2.98749218015208\\
-2.45699026122276	-2.92497609629639\\
-2.54844047680928	-2.85765395199859\\
-2.6344139584906	-2.78552024918273\\
-2.71520979391882	-2.70848320565843\\
-2.79105074844663	-2.62636964450667\\
-2.86208099170613	-2.53892814329307\\
-2.92836229218099	-2.4458307708287\\
-2.9898683616798	-2.34667378659936\\
-3.04647705843351	-2.24097775733803\\
-3.09796018939814	-2.12818767179549\\
-3.14397071057272	-2.00767382546431\\
-3.18402723796887	-1.87873452025237\\
-3.21749599223031	-1.74060199786171\\
-3.24357066319546	-1.59245351215947\\
-3.2612512708607	-1.43343004080377\\
-3.26932400463725	-1.2626657993691\\
-3.26634533158282	-1.07933234415153\\
-3.25063542710014	-0.882701412409692\\
-3.22028813827935	-0.672230366860738\\
-3.1732069482427	-0.447672601718421\\
-3.1071780824799	-0.209211783624883\\
-3.01999175754808	0.0423873708727519\\
-2.90961887002595	0.305628153185356\\
-2.77444132743723	0.578158055922668\\
-2.61351892168651	0.856709056758445\\
-2.42685603899915	1.13714194062122\\
-2.21561361070748	1.41462476740585\\
-1.98220507942266	1.6839496369487\\
-1.73022872524769	1.93995393006796\\
-1.46422412897447	2.17797530792202\\
-1.18928764889999	2.39425119110813\\
-0.91062143246285	2.58618445637019\\
-0.633105203016797	2.75243455599225\\
-0.360964850674611	2.89284069661211\\
-0.0975764829442119	3.00822147313603\\
0.154593906607722	3.10011179527205\\
0.393942688293211	3.17049342432048\\
0.619612011186499	3.22155838822615\\
0.831341940748543	3.25552473402513\\
1.02931979977233	3.27450805884036\\
1.21404429523693	3.28044238815931\\
};
\addplot [color=mycolor1, forget plot]
  table[row sep=crcr]{%
1.13918986385114	3.39291030692128\\
1.31334254622258	3.38743932459056\\
1.47666231161076	3.37189486946436\\
1.62985336376191	3.34742713098829\\
1.77363116361104	3.31499238731071\\
1.9086936208948	3.27536813870626\\
2.03570151526558	3.22916982839512\\
2.15526566937106	3.17686731060412\\
2.26793883658075	3.11879998754381\\
2.37421068991487	3.05519005756676\\
2.47450466317591	2.98615365655134\\
2.56917568712316	2.91170988787152\\
2.65850808526946	2.83178786715564\\
2.74271305516713	2.74623199020703\\
2.82192527413863	2.65480569155722\\
2.89619824619334	2.55719401700191\\
2.96549806255904	2.45300540251019\\
3.02969529607241	2.34177314884866\\
3.08855480632043	2.2229572204889\\
3.14172331913164	2.09594719327623\\
3.18871478900932	1.96006744155325\\
3.22889379417839	1.81458600194816\\
3.2614576002984	1.65872897810248\\
3.28541812055037	1.49170283761927\\
3.29958586098608	1.31272743994833\\
3.30255912144019	1.12108299720362\\
3.29272322416275	0.916174187090234\\
3.26826625015252	0.697613960216642\\
3.22721935403153	0.465327738155843\\
3.16753056484358	0.219675153705707\\
3.08718005056154	-0.0384191370860458\\
2.98434084579235	-0.307341574723892\\
2.85758089070568	-0.584696951783404\\
2.70608979312119	-0.867277122662939\\
2.52989900831624	-1.15112144140111\\
2.3300519147241	-1.4316889926073\\
2.10867725524729	-1.70414101309216\\
1.86893120683685	-1.96370287284456\\
1.61480018945704	-2.20604842658807\\
1.35079090761119	-2.42763724664457\\
1.08156304935098	-2.62594429031886\\
0.811571843486864	-2.79954935129738\\
0.544778559945462	-2.94808887382429\\
0.284462885417311	-3.07210182987649\\
0.0331430358728775	-3.17281577583671\\
-0.207412600404753	-3.25191821988336\\
-0.436109895771939	-3.3113472100126\\
-0.652418494402153	-3.35312039216399\\
-0.856250380106214	-3.37920883099783\\
-1.04784395731508	-3.39145301224971\\
-1.22766287057359	-3.39151379982221\\
-1.39631320746139	-3.3808496940076\\
-1.55447908508147	-3.36071226228474\\
-1.70287462151121	-3.3321530445752\\
-1.84220946327544	-3.29603688523293\\
-1.97316492744097	-3.25305814882564\\
-2.09637808130158	-3.20375749061753\\
-2.21243149987578	-3.14853775640611\\
-2.32184688148563	-3.08767822001648\\
-2.42508109894911	-3.02134678958923\\
-2.52252359246908	-2.9496100841685\\
-2.61449426618703	-2.87244144934862\\
-2.70124124044346	-2.78972708358762\\
-2.78293794738014	-2.70127051464626\\
-2.85967915133165	-2.60679572088432\\
-2.9314755405021	-2.50594925269726\\
-2.9982465862832	-2.3983017905733\\
-3.05981141648412	-2.28334969254005\\
-3.11587751739229	-2.16051724954773\\
-3.16602719136474	-2.02916059682455\\
-3.20970188427344	-1.88857453477377\\
-3.24618480397113	-1.73800390071199\\
-3.27458273194635	-1.57666159427501\\
-3.29380864897059	-1.40375585733705\\
-3.30256781250384	-1.21852985551498\\
-3.29935127053315	-1.02031683366934\\
-3.28244242619718	-0.808613839943074\\
-3.24994397454638	-0.583175817221317\\
-3.19983384934379	-0.344129229116059\\
-3.13005892010776	-0.0920998288786608\\
-3.03867289103494	0.171657472809961\\
-2.92401889067576	0.445146795822683\\
-2.78494685465088	0.725565289186103\\
-2.62104180619464	1.00931428445458\\
-2.43282490921739	1.29211422453723\\
-2.22188069184517	1.56923392380142\\
-1.99086776908621	1.83581845968917\\
-1.74339015709075	2.08727071340673\\
-1.4837382877407	2.31962080554035\\
-1.21654221879537	2.52981570238502\\
-0.946400969228186	2.71588066109658\\
-0.677553065468151	2.87693743867862\\
-0.413635536840815	3.0130978833525\\
-0.15755103135502	3.12527387942904\\
0.088563304520537	3.21495096129171\\
0.323288483199085	3.28396596140827\\
0.545827235646421	3.33431531819545\\
0.755885505103046	3.36800638923301\\
0.953552195606826	3.38695306611053\\
1.13918986385114	3.39291030692128\\
};
\addplot [color=mycolor1, forget plot]
  table[row sep=crcr]{%
1.07681390015377	3.50387265912038\\
1.25290541920648	3.49833572423572\\
1.4189356767722	3.48252854212395\\
1.57549793994767	3.45751789326495\\
1.72320292935161	3.4241930275202\\
1.86265234412909	3.3832777999272\\
1.99442009920366	3.33534439184401\\
2.11903930354307	3.2808270371971\\
2.23699331814286	3.22003479545064\\
2.34870954353418	3.15316285038042\\
2.45455486234239	3.08030211046827\\
2.55483188912201	3.0014470836132\\
2.64977535661494	2.91650213026931\\
2.73954810062175	2.82528629261796\\
2.82423620396988	2.72753697483497\\
2.90384293447019	2.62291282871868\\
2.9782811746746	2.51099629514403\\
3.04736410730181	2.39129637879328\\
3.11079400754534	2.26325240417023\\
3.16814912553205	2.12623972620519\\
3.21886884942524	1.97957865614578\\
3.26223766078859	1.82254821170636\\
3.29736887515341	1.65440669150492\\
3.32318985127375	1.47442145935703\\
3.33843129116964	1.28191060678544\\
3.34162444246868	1.07629917335834\\
3.33111137687374	0.857192084217782\\
3.30507483070997	0.624464560729761\\
3.26159491273558	0.378368078115702\\
3.19873959952593	0.119645680832525\\
3.11469341897808	-0.15035532707082\\
3.00792316781258	-0.429591854598313\\
2.87737058282481	-0.715278695886092\\
2.7226505650479	-1.00391302290729\\
2.54422273820986	-1.29138984491412\\
2.34349842605664	-1.5732142596992\\
2.12284929698522	-1.8447947354008\\
1.88550002685379	-2.10177872423388\\
1.63531234685439	-2.34037630997864\\
1.37649370738826	-2.55761666979082\\
1.1132808549074	-2.75149747541878\\
0.849650718702709	-2.9210133765116\\
0.589098633806478	-3.06607653770328\\
0.334503372642803	-3.18736095127435\\
0.0880778652927617	-3.28610893254169\\
-0.148610152184288	-3.36393423857315\\
-0.374569942243917	-3.42264617084894\\
-0.589294217619091	-3.46410761740004\\
-0.792658164053592	-3.49013049208802\\
-0.984823260130143	-3.50240573644614\\
-1.16615279750645	-3.50246184449687\\
-1.337141908017	-3.49164495734264\\
-1.49836219357036	-3.4711140416334\\
-1.65041952382722	-3.44184577780922\\
-1.79392288646242	-3.40464505649268\\
-1.92946203233454	-3.36015815167484\\
-2.05759181094738	-3.30888659825559\\
-2.17882137624947	-3.25120053029595\\
-2.29360675984949	-3.18735076066673\\
-2.40234560498806	-3.11747924444821\\
-2.50537310652952	-3.04162781050253\\
-2.6029584036181	-2.95974520593235\\
-2.69530082563154	-2.87169260754258\\
-2.78252550646702	-2.77724783745605\\
-2.86467796709813	-2.67610859633044\\
-2.94171733334199	-2.5678951135311\\
-3.01350791825165	-2.45215272354561\\
-3.07980897269412	-2.3283550251702\\
-3.14026251404752	-2.19590847669295\\
-3.19437930828088	-2.05415953611136\\
-3.24152333948906	-1.90240577421399\\
-3.28089549603646	-1.73991276179689\\
-3.31151778247802	-1.56593892917875\\
-3.33222017760411	-1.37977094642282\\
-3.3416333263787	-1.18077234614903\\
-3.33819154649102	-0.968447894925771\\
-3.32015200737686	-0.742525306038447\\
-3.28563707691093	-0.50305389184007\\
-3.23270715373773	-0.250516304671474\\
-3.15946997428325	0.0140555549435519\\
-3.06422843948946	0.288975714522802\\
-2.94566174443023	0.571831968071082\\
-2.80302423696485	0.85947189989199\\
-2.63633487342317	1.14807102780864\\
-2.44652124165192	1.43329715793668\\
-2.23548085738443	1.71056661957679\\
-2.00603259452234	1.97536480215196\\
-1.76175240525437	2.22358302957913\\
-1.5067141483345	2.45181490629269\\
-1.24517891109999	2.65756275019291\\
-0.981286178864548	2.83932648752665\\
-0.718794577087978	2.9965751045877\\
-0.460902438757844	3.12962440062159\\
-0.210156918784256	3.23945763471365\\
0.0315577520644397	3.32752657542855\\
0.262972928864498	3.39556279593601\\
0.483352111847407	3.44541778923692\\
0.69239179193768	3.47893974872642\\
0.890121699398104	3.49788690805546\\
1.07681390015377	3.50387265912038\\
};
\addplot [color=mycolor1, forget plot]
  table[row sep=crcr]{%
1.02515979667224	3.60981291354574\\
1.20311562436717	3.6042131621784\\
1.37165196277825	3.58816339150138\\
1.53128009861078	3.56265918140955\\
1.68253149195655	3.52853058333861\\
1.8259333787484	3.48645225249679\\
1.96199093354743	3.43695506518422\\
2.09117436581869	3.38043785585983\\
2.21390955397894	3.31717842954101\\
2.33057106041103	3.24734337988316\\
2.44147658709197	3.17099650562936\\
2.54688211315672	3.08810580046505\\
2.64697710091582	2.99854912171739\\
2.74187926984612	2.90211874503844\\
2.83162852659825	2.79852510431886\\
2.91617971318818	2.68740011465374\\
2.99539390766968	2.56830059516605\\
3.06902809714852	2.44071246018251\\
3.13672316160073	2.30405654196441\\
3.19799028367838	2.15769715263519\\
3.25219616601802	2.00095478670175\\
3.29854783101757	1.8331246947889\\
3.33607833874827	1.65350338667567\\
3.36363552033777	1.46142537282874\\
3.37987679574304	1.25631249938791\\
3.38327427887542	1.03773787322616\\
3.37213551898953	0.805505339863061\\
3.34464607516443	0.559743452978278\\
3.29894015671415	0.301009592182757\\
3.23320408185208	0.0303953020543156\\
3.14581355627779	-0.25038151573407\\
3.03549928853415	-0.538917019771991\\
2.90152659503935	-0.8321166106038\\
2.74386511438148	-1.12626478789704\\
2.56331773962297	-1.41718043855402\\
2.36157729213811	-1.70045233319201\\
2.14118811785052	-1.97173013328136\\
1.90540730117711	-2.22702947852382\\
1.65798200631524	-2.46300249869556\\
1.40287823587041	-2.67713114916541\\
1.14400535857114	-2.8678184835532\\
0.88497745374461	-3.03437591453482\\
0.628939318653582	-3.17692474229225\\
0.378467674378691	-3.29624212085679\\
0.13554262520618	-3.39358379222262\\
-0.0984255763087338	-3.47051059170827\\
-0.322533207424746	-3.52873675093961\\
-0.536300907622331	-3.57000886379482\\
-0.739586635061441	-3.59601713677922\\
-0.932503172145304	-3.60833591349153\\
-1.11534548375721	-3.60838819588231\\
-1.2885300856312	-3.59742832529505\\
-1.45254649037625	-3.57653744089128\\
-1.60791959998377	-3.54662725568829\\
-1.75518136156198	-3.50844872441816\\
-1.89484986610153	-3.46260312808364\\
-2.0274141676177	-3.40955388545592\\
-2.15332330756317	-3.34963800690096\\
-2.27297826969436	-3.28307654969192\\
-2.38672582070896	-3.20998374826903\\
-2.49485339177604	-3.13037471143959\\
-2.5975843192724	-3.04417173168765\\
-2.69507289134802	-2.95120936519182\\
-2.78739874658233	-2.85123853588039\\
-2.87456025092164	-2.74393001033717\\
-2.95646655049722	-2.62887769751419\\
-3.03292807461189	-2.50560236111094\\
-3.10364536250223	-2.37355650455739\\
-3.16819623162121	-2.23213140760739\\
-3.22602152236636	-2.08066756335117\\
-3.27640997908704	-1.91847007885575\\
-3.31848329936319	-1.74483093760243\\
-3.35118304149935	-1.5590603234897\\
-3.37326194829048	-1.36052937479863\\
-3.38328330904302	-1.14872660980604\\
-3.37963314927423	-0.923329607521185\\
-3.36055108527246	-0.684292037527928\\
-3.32418620024484	-0.431943504613437\\
-3.26868367304024	-0.167095722866397\\
-3.19230536147705	0.108856576005162\\
-3.09358244533766	0.393857068754825\\
-2.97149044096019	0.685152865976423\\
-2.82562738894695	0.97932494000532\\
-2.65636726985971	1.27240021362405\\
-2.46495643298639	1.56004880299526\\
-2.25352466310871	1.83785155187066\\
-2.02499590669243	2.10160399929956\\
-1.7829041496629	2.34761020364682\\
-1.53114120525856	2.57291909951605\\
-1.27367773558674	2.77546850554358\\
-1.0143017090026	2.95412322218416\\
-0.756409657405247	3.10861616483928\\
-0.502870033834599	3.23941798981028\\
-0.255960938590626	3.34756760350115\\
-0.017371473825098	3.43449392828752\\
0.211750845111671	3.50185165385534\\
0.430725326573695	3.55138427531142\\
0.639252034058069	3.58481934966929\\
0.837326115959362	3.60379494983452\\
1.02515979667224	3.60981291354574\\
};
\addplot [color=mycolor1, forget plot]
  table[row sep=crcr]{%
0.9828277466891	3.70763537359447\\
1.16253075031462	3.70197719960511\\
1.33334009430001	3.68570766277502\\
1.49570695775239	3.65976271051159\\
1.65010344972799	3.62492141184049\\
1.79699993433113	3.58181474962233\\
1.9368480532624	3.53093576568061\\
2.07006805357902	3.47264986453593\\
2.19703921115515	3.40720453038006\\
2.31809233268834	3.33473804069562\\
2.43350349604434	3.25528699630026\\
2.5434883397133	3.16879265862485\\
2.64819633554898	3.07510621461667\\
2.74770457856682	2.97399319805948\\
2.8420107106544	2.86513740040777\\
2.93102467161595	2.74814471903087\\
3.01455905390315	2.62254752861117\\
3.09231794308403	2.487810332777\\
3.16388427545534	2.3433376651173\\
3.22870596337977	2.18848546303003\\
3.286081359338	2.02257742548568\\
3.33514508587561	1.84492815991116\\
3.37485588221517	1.65487516850558\\
3.40398892519859	1.45182182275065\\
3.42113604999295	1.2352932736457\\
3.42471832924838	1.00500652996698\\
3.41301635536507	0.760954447175441\\
3.38422393697969	0.5035008557295\\
3.33653022560203	0.233480401589032\\
3.26823289024617	-0.0477079186871799\\
3.17788032636966	-0.338029307421637\\
3.0644339748185	-0.634783043541202\\
2.92743353400122	-0.93463305884721\\
2.76714030521927	-1.233713376536\\
2.5846302238386	-1.52781119213057\\
2.38181130086192	-1.81261429556168\\
2.16135144700147	-2.08399295313932\\
1.92652001739265	-2.33827507327527\\
1.6809648359463	-2.57247228271275\\
1.4284597389176	-2.78442464350298\\
1.1726614366354	-2.97284981242987\\
0.916908237129788	-3.13730219312071\\
0.664080193666466	-3.2780628929517\\
0.416525582855244	-3.39598864294165\\
0.176046663487098	-3.49234732127893\\
-0.0560693645323572	-3.56866183959142\\
-0.278990371332361	-3.62657612494709\\
-0.492264951839194	-3.66774934602806\\
-0.695745042793501	-3.69377879807574\\
-0.889513035128932	-3.70614832570495\\
-1.0738175758965	-3.7061975177389\\
-1.24901973286528	-3.69510659111797\\
-1.41554952266051	-3.67389233770175\\
-1.57387183832527	-3.64341131302821\\
-1.72446036017558	-3.6043673252688\\
-1.86777791359634	-3.55732108663302\\
-2.00426180745468	-3.50270055590226\\
-2.13431284872163	-3.44081101890906\\
-2.25828692075098	-3.37184433929873\\
-2.37648819938399	-3.29588709096604\\
-2.48916324566855	-3.21292748404959\\
-2.59649535097355	-3.12286114391619\\
-2.69859862106026	-3.02549591924005\\
-2.79551137592754	-2.92055599967027\\
-2.88718852065041	-2.80768573144182\\
-2.97349262037157	-2.68645364415975\\
-3.05418350461285	-2.55635735543462\\
-3.12890635067281	-2.41683021112011\\
-3.19717837652471	-2.26725075241316\\
-3.25837453921522	-2.10695637406438\\
-3.31171301872151	-1.93526283367979\\
-3.35624180374542	-1.75149155094615\\
-3.39082841090712	-1.55500682278472\\
-3.41415566254215	-1.34526505202943\\
-3.4247274669697	-1.12187765939501\\
-3.42088954046595	-0.884688279919619\\
-3.40087069395281	-0.633862869992579\\
-3.36285022003324	-0.369988265853077\\
-3.30505544949928	-0.0941705558364476\\
-3.22589007815862	0.191880185460561\\
-3.12408805098337	0.485796593290265\\
-2.99888000213569	0.78455200019934\\
-2.85015098904599	1.08452724333672\\
-2.67856222643492	1.38165550846322\\
-2.48560896290965	1.67164033388676\\
-2.27359387819706	1.950224895958\\
-2.0455101495427	2.21347602806993\\
-1.80484699864818	2.45803975348327\\
-1.5553471104622	2.68132970671944\\
-1.30075416578548	2.88162457077831\\
-1.04458727672918	3.05807042243704\\
-0.789968913470217	3.21060202528641\\
-0.539518452389949	3.33980861790815\\
-0.295309740119573	3.44677296070807\\
-0.0588814470986692	3.53290880790546\\
0.168715467301757	3.59981463192038\\
0.386848938890716	3.6491533690256\\
0.595228759293788	3.68256119407468\\
0.793830724627843	3.70158370544942\\
0.9828277466891	3.70763537359447\\
};
\addplot [color=mycolor1]
  table[row sep=crcr]{%
0.948683298050514	3.79473319220206\\
1.12996891013208	3.78902241071535\\
1.3027838103604	3.77255915923418\\
1.46753552344581	3.74623053454947\\
1.62465224482371	3.71077288416293\\
1.77456152903207	3.66677970030574\\
1.91767403561802	3.61471073626391\\
2.05437110063153	3.55490128334521\\
2.18499505421831	3.4875709448019\\
2.30984136685525	3.41283153872394\\
2.42915185734506	3.33069397942336\\
2.54310832592638	3.24107414976038\\
2.65182608463458	3.14379790634541\\
2.75534694820212	3.03860547381077\\
2.8536313296431	2.92515559896292\\
2.9465491655584	2.80302996392973\\
3.0338694906698	2.67173851056894\\
3.11524860652315	2.53072651479627\\
3.19021696761039	2.3793844730251\\
3.25816516606007	2.21706211963439\\
3.31832976399147	2.0430881668871\\
3.36978023105031	1.85679760784185\\
3.41140891541445	1.65756857671014\\
3.44192680709927	1.44487070344488\\
3.45986879009691	1.21832646083672\\
3.46361298678702	0.977785965288651\\
3.45141941180798	0.723413822791541\\
3.4214930614047	0.455783738589644\\
3.37207522993717	0.175972765268488\\
3.30156372141992	-0.114357323644244\\
3.20865740947072	-0.412907981729085\\
3.0925136091427	-0.716739496073036\\
2.95289924526556	-1.02233055985344\\
2.79031115637498	-1.3257107709346\\
2.60603988731933	-1.6226624623078\\
2.40215712665582	-1.90897292279738\\
2.18141953804948	-2.18070442957845\\
1.94709823579504	-2.43444260737754\\
1.70275852491129	-2.66748659951055\\
1.45202370644265	-2.877956852001\\
1.19835703486066	-3.0648138241024\\
0.944888096380122	-3.22779787311511\\
0.694297500839398	-3.36731218002033\\
0.448761194715288	-3.48427490957016\\
0.209946190700892	-3.57996468520442\\
-0.0209555786883461	-3.65587741213202\\
-0.243169602253506	-3.71360522906444\\
-0.456269901983927	-3.7547419037936\\
-0.660108423391693	-3.78081428495334\\
-0.854748940791145	-3.79323662458319\\
-1.04040889805107	-3.79328335955738\\
-1.21741050351997	-3.78207578896275\\
-1.38614100991934	-3.76057854694004\\
-1.54702130890241	-3.72960250007616\\
-1.70048159286015	-3.6898114747465\\
-1.84694273429099	-3.64173092446581\\
-1.98680208954437	-3.58575723170233\\
-2.1204225680783	-3.52216679568137\\
-2.24812396922559	-3.45112440165093\\
-2.37017574651009	-3.37269062063008\\
-2.48679050050333	-3.2868281760587\\
-2.59811762050387	-3.19340735751618\\
-2.70423659456836	-3.09221068144532\\
-2.80514959232286	-2.98293711145668\\
-2.90077300437504	-2.86520627071399\\
-2.99092770791458	-2.73856321838213\\
-3.07532793559953	-2.60248453106578\\
-3.1535687736533	-2.45638663503084\\
-3.22511252958097	-2.29963757630341\\
-3.28927451889835	-2.13157368366421\\
-3.34520925490937	-1.95152284702656\\
-3.39189861348128	-1.7588363466938\\
-3.42814429814894	-1.5529312348605\\
-3.45256782598978	-1.33334504716437\\
-3.46362220116165	-1.0998039137315\\
-3.45962024522194	-0.852303713618117\\
-3.43878487541956	-0.591201557879526\\
-3.3993259814164	-0.317311504170883\\
-3.3395463824176	-0.0319942042762583\\
-3.25797518618734	0.262774139257094\\
-3.15352067565559	0.564371272065384\\
-3.02562738470585	0.869554764908886\\
-2.87441511977517	1.17455760801215\\
-2.70077402394131	1.47525676358082\\
-2.50639202484484	1.767403513355\\
-2.29370038309167	2.04688929885519\\
-2.06573813670639	2.31000991273335\\
-1.8259528645897	2.5536887433386\\
-1.57796799985389	2.77562773795799\\
-1.32535178123783	2.9743703475303\\
-1.07141885643386	3.14927865398809\\
-0.819084909373598	3.30044159903781\\
-0.570781691446472	3.42853925898355\\
-0.328428492520302	3.53468897189284\\
-0.093448740680559	3.62029468577783\\
0.13318251344032	3.68691393725669\\
0.350873927242831	3.73614982594868\\
0.55934713845833	3.76956970364108\\
0.758567824726602	3.78864856747597\\
0.948683298050513	3.79473319220206\\
};
\addlegendentry{Аппроксимации}

\addplot [color=mycolor2]
  table[row sep=crcr]{%
2.12132034355964	2.82842712474619\\
2.18582575000323	2.82640660874952\\
2.2459761483502	2.82067996094826\\
2.30316600770785	2.81153894842004\\
2.35833530419819	2.79908319427389\\
2.4121407651283	2.78328504568152\\
2.46505313939319	2.76402339892432\\
2.51741350921551	2.74110157112695\\
2.56946568278775	2.71425682483196\\
2.62137377470791	2.68316571665979\\
2.67322998558975	2.64744786912504\\
2.72505542741334	2.60667010622203\\
2.77679571750944	2.56035271147659\\
2.82831256449113	2.50797963523731\\
2.87937250576692	2.44901464186941\\
2.92963424895255	2.38292549191834\\
2.97863667364875	2.30921809960124\\
3.02579039162557	2.22748193169334\\
3.0703766721051	2.13744642550811\\
3.11155818595542	2.03904567139316\\
3.14840590189045	1.93248505458784\\
3.1799449834646	1.81829951655162\\
3.20521925970046	1.69738980361013\\
3.2233689191928	1.57102231442397\\
3.23371058284854	1.44078163681543\\
3.23580480244767	1.30847305653245\\
3.22949541157146	1.1759834831\\
3.21490900216393	1.04511946165368\\
3.19240995273442	0.917445713323391\\
3.16251366135309	0.794144173493542\\
3.12576385298257	0.675902054710668\\
3.08257536776515	0.562820516214273\\
3.03302813706062	0.454314775018543\\
2.97656426645917	0.348948817128694\\
2.91146921429886	0.244099783305823\\
2.83385680006128	0.135243682019644\\
2.73548220960228	0.0144130439331602\\
2.59871334341581	-0.133168683665202\\
2.38479638365891	-0.336180712113444\\
2.01057175466146	-0.648371525543042\\
1.34210041814469	-1.13791067535239\\
0.375352321166521	-1.75822148716338\\
-0.527448468220416	-2.26262801632552\\
-1.11365847778129	-2.5437869612038\\
-1.4567327767317	-2.68228003389134\\
-1.6691296256022	-2.75259678373195\\
-1.81428800439327	-2.79054335642282\\
-1.92320462744895	-2.81168897715777\\
-2.01132392805712	-2.82302365363059\\
-2.08680248126988	-2.82787464184343\\
-2.15422943553058	-2.82790871121207\\
-2.21634478307291	-2.8239827357917\\
-2.27487422572476	-2.8165264847195\\
-2.33095693947289	-2.80572465930185\\
-2.3853755403757	-2.79160756168836\\
-2.43868469976231	-2.77409782070174\\
-2.49128507004909	-2.75303495626884\\
-2.54346612600745	-2.7281884019911\\
-2.59543032955127	-2.69926454755406\\
-2.64730535029083	-2.66591103001887\\
-2.69914810041563	-2.62772046423944\\
-2.75094277189846	-2.58423541950896\\
-2.80259428647749	-2.53495641420315\\
-2.85391830194335	-2.47935483422118\\
-2.90462903903618	-2.41689283661657\\
-2.95432664759112	-2.34705230341292\\
-3.00248656709011	-2.26937453210355\\
-3.04845424185856	-2.18351130413452\\
-3.0914493816545	-2.08928598065547\\
-3.13058429431929	-1.98676020448289\\
-3.16490008683964	-1.87629788656397\\
-3.19342217916266	-1.75861429449935\\
-3.21523240499265	-1.63479579561046\\
-3.22954956680171	-1.50627700312196\\
-3.23580523135335	-1.37476797309193\\
-3.23369896357047	-1.2421341494958\\
-3.22321880671614	-1.11024302689198\\
-3.20461860896113	-0.980799507713277\\
-3.17835148049538	-0.85519274756288\\
-3.14496439092874	-0.734369602871579\\
-3.10495870881183	-0.618735254776386\\
-3.05861192744495	-0.508062722468019\\
-3.0057323937692	-0.401369642978784\\
-2.94527025004781	-0.296685462718974\\
-2.87460204254734	-0.190562899149911\\
-2.78805571026974	-0.0770301658026245\\
-2.67361421522119	0.0546866392302208\\
-2.50520482475694	0.225105040795738\\
-2.22541867988909	0.474087715816021\\
-1.7209533613293	0.868461990547897\\
-0.880186083660502	1.44500069250654\\
0.110323684998138	2.03883159526995\\
0.859259206598599	2.42737283484743\\
1.30741523004191	2.62511520505649\\
1.57426953542748	2.72304263504847\\
1.74766615441039	2.77435113512036\\
1.87207484559155	2.80265686669549\\
1.96924085114733	2.81832059141065\\
2.0502990590364	2.82612977760851\\
2.12132034355964	2.82842712474619\\
};
\addlegendentry{Сумма Минковского}

\end{axis}

\begin{axis}[%
width=0.798\linewidth,
height=0.597\linewidth,
at={(-0.104\linewidth,-0.066\linewidth)},
scale only axis,
xmin=0,
xmax=1,
ymin=0,
ymax=1,
axis line style={draw=none},
ticks=none,
axis x line*=bottom,
axis y line*=left,
legend style={legend cell align=left, align=left, draw=white!15!black}
]
\end{axis}
\end{tikzpicture}%
        \caption{Эллипсоидальные аппроксимации для 100 направлений.}
\end{figure}

%%%%%%%%%%%%%%%%%%%%%%%%%%%%%%%%%%%%%%%%%%%%%%%%%%%%%%%%%%%%%%%%%%%%%%%%%%%%%%%%
\clearpage
\section{Внутренняя оценка суммы эллипсоидов}

\begin{definition}
        \textit{Сингулярным разложением} матрицы $A \in \setR^{n \times m}$ называется представление матрицы в виде
$$
        A = V \mathit{\Sigma} U^*, 
$$
        где
$$
\begin{aligned}
&V \in \setR^{n \times n}\::\: V^* = V^{-1},\\
&U \in \setR^{m \times m}\::\: U^* = U^{-1},\\
&\mathit{\Sigma} = \mathrm{diag}\left(\sigma_1,\,\ldots,\,\sigma_{\min\{n,\,m\}}\right) \in \setR^{n \times m}\::\:\sigma_1 \geqslant \sigma_2 \geqslant \ldots \geqslant \sigma_{\min\{n,\,m\}}.
\end{aligned}
$$ 
\end{definition}

\begin{theorem}
        Сингулярное разложение
        $A = V \mathit{\Sigma} U^*$
        существует для любой комплексной матрицы $A$.
        Если матрица $A$ вещественная, то матрицы $V$, $\mathit{\Sigma}$ и $U$ также можно выбрать вещественными. 
\end{theorem}

\begin{theorem}
        Старшее сингулярное число $\sigma_1$ матрицы $A = V \mathit{\Sigma} U^*$ является её нормой.
\end{theorem}

\begin{definition}
        Назовём линейное преобразование $\mathcal{A}$ \textit{ортогональным}, если оно сохраняет скалярное произведение, то есть
$$
        \langle \mathcal{A}(x),\,\mathcal{A}(y)\rangle = \langle x,\,y\rangle.
$$
\end{definition}

\begin{theorem}\label{th:unitarnost}
        Необходимым и достаточным условием ортогональности линейного преобразования $\mathcal{A}$ в конечномерном пространстве является унитарность матрицы преобразования $A$, то есть
$$
        A^* = A^{-1}.
$$
\end{theorem}

\begin{assertion}
        Для произвольных векторов $a,\,b\in\setR^{n}$ таких, что $\|a\| = \|b\|$, существует матрица ортогонального преобразования, переводящего $a$ в $b$.
\end{assertion}
\begin{proof}
        
        Построим сингулярное разложение для векторов $a$ и $b$:
$$
        a = V_a \mathit{\Sigma}_a u_a,
        \qquad
        b = V_b \mathit{\Sigma}_b u_b,
$$        
причем $V_a,\,V_b \in \setR^{n \times n}$~--- унитарные матрицы, $u_a,\,u_b\in\{-1,\,1\}\in\setR^1$,
$$
        \mathit{\Sigma}_a = [\sigma_a,\,0,\,\ldots,\,0]\T\in\setR^{n \times 1},\quad
        \mathit{\Sigma}_b = [\sigma_b,\,0,\,\ldots,\,0]\T\in\setR^{n \times 1},\quad
        \sigma_a,\,\sigma_b > 0.
$$
Согласно Теореме~\ref{th:unitarnost} $\sigma_a = \sigma_b$.
Тогда преобразуем выражение для вектора $b$:
\begin{multline*}
        b
        =
        V_b \mathit{\Sigma}_b u_b
        =
        V_b (V_a\T V_a)\mathit{\Sigma}_b u_b
        =
        V_bV_a\T V_a \left( \mathit{\Sigma}_a \frac{\sigma_b}{\sigma_a} \right) \left( u_a\frac{u_b}{u_a} \right)
        =\\=
        V_bV_a\T \frac{\sigma_b\cdot u_b}{\sigma_a\cdot u_a}V_a\mathit{\Sigma}_au_a
        =
        \left(V_b V_a\T \frac{\sigma_b\cdot u_b}{\sigma_a\cdot u_a}\right)a.
\end{multline*}
Так как произведение унитарных матриц есть унитарная матрица, теорема доказана.

\end{proof}

\textbf{Следствие 1.} Далее под ортогональным преобразованием из вектора $a$ в вектор $b$ таких, что $\|a\| = \|b\|$, будем понимать
$$
        \mathrm{Orth}(a,\,b) = u_au_bV_bV_a\T.
$$

\begin{assertion}
        Для суммы Минковского эллипсоидов справедлива следующая оценка
$$
        \sum_{i=1}^n \Varepsilon(q_i,\,Q_i) = 
        \bigcup\limits_{\|l\|=1}\Varepsilon(q_-(l),\,Q_-(l)),
$$
        где
$$
        \begin{aligned}
q_-(l) &= \sum_{i=1}^n q_i,
\\
Q_-(l) &= Q_*\T(l) Q_*(l),
\quad
Q_*(l) = \sum_{i=1}^{n}S_i(l)Q_i^{\nicefrac12},
\\
S_i(l) &= \mathrm{Orth}(Q_i^{\nicefrac12}l,\,\lambda_iQ_1^{\nicefrac12}l),
\quad
\lambda_i = \frac{\langle l,\,Q_il\rangle^{\nicefrac12}}{\langle l,\,Q_1l\rangle^{\nicefrac12}}.
        \end{aligned}
$$
\end{assertion}

\begin{proof}

Будем доказывать для случая $q_i = 0$, $i=\overline{1,\,n}$.
Случай с произвольными центрами~--- аналогично.

Итак рассмотрим эллипсоид $\Varepsilon_- = \Varepsilon(0,\,Q_-)$, $Q_- = Q_*\T Q_*$,
$$
        Q_* = \sum_{i=1}^n S_i Q_i^{\nicefrac12},
$$
где $S_i$~--- некоторые унитарные матрицы. Распишем квадрат опорной функции этого эллипсоида:
\begin{multline*}
        \rho^2(l\,|\,\Varepsilon_-)
        =
        \langle l,\,Q_-l \rangle
        =
        \langle Q_*l,\,Q_*l \rangle
        =
        \sum_{i=1}^n \langle l,\,Q_il \rangle
        +
        \sum_{i \neq j} \left \langle
        S_i Q_i^{\nicefrac12}l,\, S_j Q_j^{\nicefrac12}l
        \right\rangle
        \leqslant\\\leqslant
        \{
        \mbox{Неравенство Коши--Буняковского}
        \}
        \leqslant\\\leqslant
        \sum_{i=1}^n \langle l,\,Q_il \rangle
        +
        \sum_{i \neq j}
        \langle l,\,Q_il \rangle^{\nicefrac12}
        \langle l,\,Q_jl \rangle^{\nicefrac12}
        =
        \left(
        \sum_{i=1}^n \langle l,\,Q_il \rangle^{\nicefrac12}
        \right)^2
        =
        \rho^2\left(
        l\left|
        \sum_{i=1}^n \Varepsilon(q_i,\,Q_i)
        \right.
        \right).
\end{multline*}
Таким образом, получили, что $\Varepsilon_-\subseteq\sum_{i=1}^n\Varepsilon(q_i,\,Q_i)$.

Заметим, что равенство в последней формуле при фиксированном направлении $l \neq 0$ достигается при
$$
        S_iQ_i^{\nicefrac12}l = \lambda_i S_1Q_1^{\nicefrac12}l, 
$$
где $\lambda_i$~--- произвольные неотрицательные константы. Если положить $S_1 = I$, а $\lambda_i$ выбирать, исходя из условий нормировки ($\|Q_i^{\nicefrac12}l\| = \|\lambda_iQ_1^{\nicefrac12}l\|$):
$$
        \lambda_i = \frac{\langle l,\,Q_il\rangle^{\nicefrac12}}{\langle l,\,Q_1l\rangle^{\nicefrac12}},
$$
то получим утверждение теоремы.

\end{proof}
\vfill
\begin{figure}[h]

        \centering
        % This file was created by matlab2tikz.
%
%The latest updates can be retrieved from
%  http://www.mathworks.com/matlabcentral/fileexchange/22022-matlab2tikz-matlab2tikz
%where you can also make suggestions and rate matlab2tikz.
%
\definecolor{mycolor1}{rgb}{0.00000,0.44700,0.74100}%
\definecolor{mycolor2}{rgb}{0.85000,0.32500,0.09800}%
%
\begin{tikzpicture}

\begin{axis}[%
width=0.618\linewidth,
height=0.487\linewidth,
at={(0\linewidth,0\linewidth)},
scale only axis,
xmin=-4,
xmax=4,
xlabel style={font=\color{white!15!black}},
xlabel={$x_1$},
ymin=-3,
ymax=3,
ylabel style={font=\color{white!15!black}},
ylabel={$x_2$},
axis background/.style={fill=white},
axis x line*=bottom,
axis y line*=left,
xmajorgrids,
ymajorgrids,
legend style={at={(0.03,0.97)}, anchor=north west, legend cell align=left, align=left, draw=white!15!black}
]
\addplot [color=mycolor1, forget plot]
  table[row sep=crcr]{%
2.36712178368749	2.74633057020532\\
2.41188305682678	2.74494651398331\\
2.45036041459025	2.74130089565813\\
2.48383536069915	2.73596692388682\\
2.51327948444918	2.72933446739078\\
2.53943919311047	2.72166753585199\\
2.56289492242467	2.71314181506891\\
2.58410302011837	2.70386940753685\\
2.60342571833248	2.69391520541791\\
2.62115280082523	2.68330766515176\\
2.63751738560017	2.67204572812381\\
2.65270745907852	2.66010298488441\\
2.66687427014079	2.64742976011011\\
2.68013832873481	2.63395351039579\\
2.69259349381253	2.61957772004863\\
2.70430943819101	2.6041793136995\\
2.71533261385189	2.58760445132293\\
2.72568568587575	2.56966240776041\\
2.73536523285703	2.55011704085516\\
2.74433729805113	2.52867508999008\\
2.75253007967915	2.50497017880853\\
2.75982261143016	2.47854086090236\\
2.76602760995085	2.44880025137829\\
2.77086559717422	2.41499358451471\\
2.77392567575526	2.37613820122431\\
2.77460549370625	2.33093765236215\\
2.77201821305984	2.27765729690984\\
2.7648464218491	2.21394231571241\\
2.7511098920125	2.13654986655322\\
2.72779317877431	2.04095569483249\\
2.69024836192134	1.92078707468285\\
2.6312539409082	1.7670493095065\\
2.53961413480675	1.56721869756115\\
2.39837999194539	1.30464987086518\\
2.18360687141246	0.959750006540987\\
1.8667826621648	0.516209188824812\\
1.42701784641989	-0.0238781630653391\\
0.87502831245873	-0.620875059217897\\
0.268936641059173	-1.19901922110702\\
-0.309576813214938	-1.68530797756881\\
-0.802100907580029	-2.04896030059078\\
-1.19206161730571	-2.300440537754\\
-1.49041992551985	-2.46708460024485\\
-1.7168759123608	-2.57532489307401\\
-1.88995458136084	-2.64492059461505\\
-2.02413470299114	-2.68919932969544\\
-2.12995725581116	-2.71680265831965\\
-2.21490837909904	-2.73327951527395\\
-2.284286999297	-2.74220891806166\\
-2.34187032729419	-2.74592425079378\\
-2.39038313346388	-2.74596637420131\\
-2.4318186528206	-2.74336542452957\\
-2.46765634360889	-2.73881708355027\\
-2.49901014208278	-2.7327940719517\\
-2.52673028332805	-2.72561759993438\\
-2.551474024202	-2.71750375208859\\
-2.57375536742722	-2.70859393338899\\
-2.5939804436843	-2.69897499500867\\
-2.61247296631128	-2.688692537203\\
-2.6294927100655	-2.67775958603992\\
-2.64524900273341	-2.66616202895009\\
-2.6599105764423	-2.65386167486455\\
-2.67361268902278	-2.64079746083638\\
-2.68646212009346	-2.62688508668123\\
-2.69854042224194	-2.61201517643097\\
-2.70990563030842	-2.59604990865956\\
-2.72059247497691	-2.57881790224895\\
-2.73061098716967	-2.5601069666803\\
-2.73994319179238	-2.53965410007404\\
-2.74853734053432	-2.51713180921804\\
-2.75629877602211	-2.49212938365471\\
-2.76307597883326	-2.46412710489964\\
-2.76863950239953	-2.43246039507853\\
-2.77265014411951	-2.39626942454063\\
-2.7746104875259	-2.35442742359956\\
-2.77379029091263	-2.3054374557388\\
-2.76911009981251	-2.24728211523106\\
-2.75895730729457	-2.17720283029247\\
-2.74089227231279	-2.0913749275097\\
-2.71117633487659	-1.98443346302753\\
-2.66401914162082	-1.84880387542276\\
-2.59041768625865	-1.67383876852635\\
-2.47652724106683	-1.44496938777841\\
-2.30193857203366	-1.14371563581315\\
-2.0396739695219	-0.750852651889909\\
-1.66265564107253	-0.256803878437201\\
-1.16248546087895	0.319450601512933\\
-0.573614265552006	0.917547528939285\\
0.0284025713818862	1.45675943888985\\
0.5685251002902	1.88250662286626\\
1.00960568842464	2.18713123722861\\
1.35150239782658	2.39258093926768\\
1.61139397123953	2.52710219556684\\
1.80908182046774	2.61398894502504\\
1.96116532773652	2.66959557253549\\
2.08006223005889	2.7046815507453\\
2.17466815864962	2.726172144186\\
2.25127885929965	2.73851896753466\\
2.3143624731471	2.74460741055068\\
2.36712178368749	2.74633057020532\\
};
\addplot [color=mycolor1, forget plot]
  table[row sep=crcr]{%
2.10959639276541	2.82833170033868\\
2.13005277719103	2.82769680200034\\
2.14802287471214	2.82599213309054\\
2.16399176571052	2.82344580125796\\
2.17833205481083	2.82021393645668\\
2.19133411253028	2.81640177024618\\
2.20322718958303	2.81207744870569\\
2.21419439157868	2.80728111824648\\
2.22438344129317	2.80203084312009\\
2.23391449132048	2.79632631929519\\
2.24288582434271	2.79015097768998\\
2.25137799857592	2.78347282702145\\
2.25945680590028	2.7762442174586\\
2.26717527446035	2.7684005761966\\
2.27457484262639	2.75985805103207\\
2.28168573976336	2.75050987848557\\
2.28852651598563	2.7402211492275\\
2.29510255122665	2.72882145096567\\
2.30140322103382	2.71609459233055\\
2.30739716790373	2.70176419574736\\
2.31302476509234	2.68547330303474\\
2.31818626415068	2.66675511304507\\
2.32272310676726	2.64499030425581\\
2.32638812037928	2.61934362489206\\
2.32879716902256	2.58866772920285\\
2.32934906157394	2.5513540908222\\
2.32708969033154	2.5050964728495\\
2.32047561983038	2.44650691445407\\
2.3069520645374	2.37047913665329\\
2.28218252151097	2.26911845209114\\
2.23862573128952	2.12995002321369\\
2.16294815306376	1.93306507749366\\
2.03172785596982	1.64737025756866\\
1.80647115907523	1.22912474961823\\
1.43674425455007	0.635722130761753\\
0.895415675386827	-0.122678493566976\\
0.24506740533721	-0.923290250180105\\
-0.373302970705661	-1.59439810765956\\
-0.861342602045722	-2.06153432057587\\
-1.21051470427877	-2.35582021146555\\
-1.45350302607409	-2.53553747281427\\
-1.62427912108232	-2.6457756283534\\
-1.74735161713476	-2.71454779213463\\
-1.83866892421111	-2.75820072395713\\
-1.90838558633582	-2.78623139232265\\
-1.96302614690974	-2.80425772800326\\
-2.00687082453237	-2.81568963402831\\
-2.04279713811814	-2.82265357129101\\
-2.07278812116456	-2.82650998992637\\
-2.09824304490626	-2.82814934742327\\
-2.12017116470066	-2.82816582669431\\
-2.13931522377064	-2.826961933092\\
-2.15623195525253	-2.82481302044249\\
-2.17134569327871	-2.82190798040682\\
-2.18498480181322	-2.81837538766676\\
-2.19740689106534	-2.81430054547294\\
-2.20881656655742	-2.80973669054728\\
-2.21937810527049	-2.8047123437088\\
-2.22922461613086	-2.79923603208632\\
-2.23846471192486	-2.79329914096155\\
-2.24718737595216	-2.78687735484354\\
-2.25546547731453	-2.7799309467999\\
-2.26335822936359	-2.77240402898462\\
-2.27091276797016	-2.76422275734835\\
-2.27816493006503	-2.75529236858525\\
-2.28513922246038	-2.74549279836288\\
-2.2918478707394	-2.73467246525985\\
-2.2982887089987	-2.72263957559321\\
-2.30444148590314	-2.70914996681804\\
-2.31026187654082	-2.69388999154343\\
-2.31567202720427	-2.67645213420482\\
-2.32054568700573	-2.65629974947181\\
-2.32468465040066	-2.63271516862803\\
-2.32778088753628	-2.60472181958332\\
-2.32935449079792	-2.57096482990594\\
-2.32864968427513	-2.52952378807117\\
-2.32445618534273	-2.47761220956851\\
-2.31479431693115	-2.41108425761014\\
-2.29634617602986	-2.32361020748856\\
-2.26340897464661	-2.20528850710705\\
-2.20596668980778	-2.04035887822781\\
-2.10628907021382	-1.80379023894369\\
-1.93390602004408	-1.45787754842354\\
-1.64266626013909	-0.955835256982624\\
-1.18636447694275	-0.272301910216555\\
-0.575627521932152	0.529268583185396\\
0.0765700150810036	1.28296729173816\\
0.635827895370091	1.85302389135356\\
1.05156023208183	2.22653834661497\\
1.34296588926262	2.45673552212428\\
1.54615322639242	2.59724775265813\\
1.69061445079945	2.68411603245696\\
1.79623883340552	2.73880286760147\\
1.87575398878987	2.77375114085251\\
1.93728218443528	2.79624367598011\\
1.98609242944908	2.81064247686136\\
2.02568323586637	2.81963136032678\\
2.05843611040654	2.82490602040426\\
2.08601207953629	2.8275641952934\\
2.10959639276541	2.82833170033868\\
};
\addplot [color=mycolor1, forget plot]
  table[row sep=crcr]{%
-0.751180563789503	1.29301100092275\\
-0.644509272490707	1.2895621961483\\
-0.525853075685341	1.27816235586954\\
-0.394733067383687	1.25710640312682\\
-0.251255417385498	1.22462204940121\\
-0.0963657956622014	1.17906474936042\\
0.0679313764001297	1.11919457949777\\
0.238479475719274	1.04449792159326\\
0.411096645609142	0.95547160953947\\
0.580968788823447	0.85375980073978\\
0.743253360925444	0.742056725339432\\
0.893734219404683	0.623769327860435\\
1.02933068138786	0.502533729359883\\
1.14832654684568	0.381735283717928\\
1.25030593415255	0.264160768651649\\
1.33588484834454	0.151837028466942\\
1.40636299856342	0.046035698606773\\
1.46339657703116	-0.0526160638365817\\
1.50874510492206	-0.143981414805869\\
1.54410401747092	-0.228260871982706\\
1.57101084154411	-0.305863800112672\\
1.59080459914876	-0.37731143443009\\
1.60461898267407	-0.443169442531953\\
1.61339434135726	-0.50400430761343\\
1.6178983941152	-0.560357233657007\\
1.61874953068363	-0.612730111534175\\
1.61643932306818	-0.661579338697342\\
1.6113526202479	-0.707314476704708\\
1.60378462375957	-0.750299690229399\\
1.59395489259106	-0.790856617640623\\
1.58201848840448	-0.829267814089403\\
1.56807456975614	-0.865780234303465\\
1.55217275313418	-0.900608430316382\\
1.53431752429447	-0.933937265273067\\
1.51447093083552	-0.965924014217765\\
1.49255372986368	-0.996699753941291\\
1.4684451102011	-1.02636994778256\\
1.44198106109596	-1.05501411446045\\
1.41295142229282	-1.08268443616728\\
1.38109562819774	-1.10940311195514\\
1.34609715924291	-1.13515819874194\\
1.30757674909159	-1.15989760548665\\
1.26508448786515	-1.18352082038956\\
1.2180911421837	-1.20586786686862\\
1.16597933286764	-1.22670492454442\\
1.10803574356846	-1.24570606215447\\
1.04344637620195	-1.26243069409395\\
0.971298136519321	-1.27629683490271\\
0.890591822483024	-1.2865512138292\\
0.800273888148872	-1.29223914571393\\
0.699296856787975	-1.29218010554772\\
0.586720020892429	-1.28495946080335\\
0.461861094092619	-1.26895243972215\\
0.3245024552969	-1.24240144182672\\
0.175138600776365	-1.2035682010117\\
0.0152226545865632	-1.15097166888395\\
-0.152664711632565	-1.08369478588514\\
-0.324821834412255	-1.00170023256538\\
-0.496685757869856	-0.90605470141019\\
-0.663344832918517	-0.798955642596777\\
-0.820192630077424	-0.683508161867404\\
-0.963532782997699	-0.563297645942172\\
-1.09095770410823	-0.441888842741164\\
-1.20142500062305	-0.322399917701083\\
-1.29507612370551	-0.207246320149147\\
-1.37291296977355	-0.0980684877283813\\
-1.43645044441771	0.0042020468738688\\
-1.48742259418976	0.0992032593483852\\
-1.52757278250983	0.186984897840184\\
-1.55852539637732	0.267866024583987\\
-1.58172126356109	0.342322104544315\\
-1.59839612803496	0.410903378933661\\
-1.60958478267695	0.474180138513564\\
-1.61613840476429	0.532708621189078\\
-1.61874713092658	0.587011554073861\\
-1.61796325588701	0.637568493346236\\
-1.61422266111825	0.684812373734955\\
-1.60786343226419	0.729129764048409\\
-1.59914138437268	0.770863155573305\\
-1.58824260169102	0.810314202630113\\
-1.57529326680601	0.847747236816572\\
-1.56036710004137	0.883392638422727\\
-1.54349071326817	0.917449811620675\\
-1.52464713656047	0.950089605203132\\
-1.50377772010479	0.981456069358695\\
-1.4807825573749	1.0116674553813\\
-1.45551952420259	1.04081635809509\\
-1.42780198561346	1.06896887513935\\
-1.39739519197381	1.09616261560587\\
-1.3640113739163	1.12240333396511\\
-1.32730356145226	1.14765989469051\\
-1.28685821295664	1.17185719094692\\
-1.24218687070083	1.19486655389881\\
-1.19271730234616	1.21649311338865\\
-1.13778500336415	1.23645953866669\\
-1.07662660918946	1.2543856619263\\
-1.00837780643194	1.26976377842146\\
-0.932079852456276	1.28193010776828\\
-0.846700871404774	1.2900342666388\\
-0.751180563789503	1.29301100092275\\
};
\addplot [color=mycolor1, forget plot]
  table[row sep=crcr]{%
1.7832981678353	1.36546589275169\\
2.05592568115162	1.3571744194569\\
2.26714164491445	1.33728673652804\\
2.43111853478521	1.31126156725633\\
2.55943271347477	1.28244136021363\\
2.66093354729971	1.25276022517939\\
2.74218962233567	1.22327936216145\\
2.80802606279813	1.19453958765605\\
2.86198660051068	1.16677896496198\\
2.90668763683666	1.14006289696381\\
2.94407647731309	1.11436054257093\\
2.97561575220362	1.08958916115493\\
3.00241420566575	1.06563949952862\\
3.02531948618789	1.04239000684265\\
3.04498425031883	1.0197144567338\\
3.06191351815595	0.997485660984366\\
3.07649877708056	0.975576844041151\\
3.08904261974966	0.95386158977852\\
3.09977651875239	0.932212879917087\\
3.10887352086999	0.910501505895431\\
3.11645707010401	0.888593987389955\\
3.12260675718842	0.866350032379446\\
3.12736148556441	0.843619502128469\\
3.13072029732992	0.820238784686037\\
3.13264088455171	0.796026421968093\\
3.13303559280109	0.770777769825975\\
3.13176447555128	0.744258389841893\\
3.12862464535523	0.716195767549336\\
3.12333474415734	0.686268814798392\\
3.1155127546208	0.654094433393716\\
3.10464449980951	0.619210182674226\\
3.09003888625681	0.58105180074941\\
3.07076402681151	0.538923992473511\\
3.04555554798029	0.49196258500875\\
3.01268429798848	0.439086064580252\\
2.96976506994043	0.37893515870206\\
2.91348118717393	0.309801736731583\\
2.83919435906196	0.229555616004797\\
2.74041386938704	0.135595692071535\\
2.60813878932407	0.024891213502135\\
2.43022180279671	-0.105743493774361\\
2.19124961386915	-0.258874318262122\\
1.87411209379186	-0.434804268222937\\
1.46518116380506	-0.628916844714506\\
0.964245059921326	-0.829016773640375\\
0.395207777190676	-1.01569772268623\\
-0.193748734909378	-1.16861428361279\\
-0.747622298683287	-1.2757260230298\\
-1.22738617820585	-1.33743352552652\\
-1.61919514519511	-1.36280865849501\\
-1.92824504599151	-1.36321391987884\\
-2.16827266488954	-1.34828035900592\\
-2.35424440438222	-1.32479177202571\\
-2.49912053823983	-1.29705421238368\\
-2.61307727724595	-1.26762654584206\\
-2.7037556868631	-1.23795184232903\\
-2.77678844889193	-1.20879671834807\\
-2.83630900148409	-1.18052984170452\\
-2.88535963560319	-1.15329067060285\\
-2.92619505892335	-1.12708932719676\\
-2.96050096090471	-1.10186489456276\\
-2.98954936666697	-1.07751903751993\\
-3.01430876229406	-1.0539350695173\\
-3.03552236243188	-1.03098844046375\\
-3.05376401994576	-1.00855215247121\\
-3.06947838995699	-0.98649915687812\\
-3.08300991163942	-0.964702930102648\\
-3.09462374740807	-0.943036918843075\\
-3.10452083500761	-0.921373240822967\\
-3.11284852374633	-0.899580840330464\\
-3.11970778215795	-0.877523178040282\\
-3.12515761091966	-0.855055451989272\\
-3.12921702232672	-0.832021282451107\\
-3.13186471906918	-0.808248735438496\\
-3.13303639024817	-0.783545498449621\\
-3.13261931307637	-0.757692949906779\\
-3.13044367314647	-0.730438772495175\\
-3.1262696548114	-0.701487641376271\\
-3.11976885109067	-0.670489360837328\\
-3.1104978196536	-0.63702361653161\\
-3.09786054958415	-0.600580246600419\\
-3.08105502897536	-0.560533616143727\\
-3.05899676892009	-0.516109340945298\\
-3.03020872281921	-0.466341368002129\\
-2.99266220849364	-0.410017606742597\\
-2.94354713401342	-0.345613712643242\\
-2.87894316526232	-0.271219097624817\\
-2.7933613052015	-0.18447081327201\\
-2.67914274745981	-0.0825377845677899\\
-2.5257783825343	0.0377458731415321\\
-2.31943501649626	0.179398198627669\\
-2.04347928473513	0.34412630878845\\
-1.68159158956786	0.530151902243493\\
-1.22537153655025	0.729304947360203\\
-0.685671303026337	0.925396871033157\\
-0.0996443290375696	1.09741544503237\\
0.478086993696042	1.22814867876394\\
0.998202340115934	1.3118123673045\\
1.43427748430264	1.35394217752062\\
1.78329816783529	1.36546589275169\\
};
\addplot [color=mycolor1, forget plot]
  table[row sep=crcr]{%
2.35571328277463	2.22100511820195\\
2.47826782286047	2.21724110275804\\
2.57945563798566	2.2076760066786\\
2.66391338264297	2.19423739652515\\
2.73516709681387	2.17820355908947\\
2.79590243487287	2.16041750763504\\
2.84817524353824	2.1414300272445\\
2.89356994912398	2.12159441970805\\
2.93331677967345	2.10112910266619\\
2.96837799604573	2.08015897733583\\
2.99951128269362	2.05874275651048\\
3.02731643733883	2.03689093924234\\
3.05226984512698	2.01457747210173\\
3.07474996249443	1.99174706294535\\
3.09505610683312	1.96831941038495\\
3.11342217070063	1.94419114633847\\
3.13002638320828	1.91923597242133\\
3.14499787010008	1.89330324619182\\
3.1584204744478	1.86621510225032\\
3.17033405755072	1.83776204934554\\
3.18073327556166	1.8076968479169\\
3.18956359405484	1.77572632588045\\
3.19671403015796	1.74150061739947\\
3.20200576274209	1.70459909187564\\
3.20517527509796	1.66451195769695\\
3.20585002004533	1.62061615374963\\
3.20351362162625	1.57214365762075\\
3.19745620422277	1.51813972897166\\
3.1867033712263	1.45740789312791\\
3.16991440413369	1.38843778139177\\
3.14523622615357	1.30931167074949\\
3.11009471123801	1.2175867111612\\
3.06090031723225	1.11015476668775\\
2.9926453648678	0.983095758531635\\
2.89838838626637	0.831573980390453\\
2.76868761057062	0.649898587637061\\
2.59122233945366	0.432001663954325\\
2.35121103645622	0.172770931598334\\
2.03377775386478	-0.129248435576604\\
1.62959742771365	-0.46779854288743\\
1.14346100946033	-0.825320879963705\\
0.600821886747986	-1.17394011308631\\
0.0442409056935615	-1.4838164622843\\
-0.48163880163311	-1.73456167138362\\
-0.945768934604242	-1.92088020146989\\
-1.33603704925895	-2.04954181981907\\
-1.65480624610948	-2.13266138371325\\
-1.91166875393097	-2.18249310676955\\
-2.11807157549804	-2.2090843405378\\
-2.28459218053074	-2.21985724993794\\
-2.42000314217803	-2.220001836486\\
-2.53121317335989	-2.21304488666874\\
-2.62353282613246	-2.2013486995232\\
-2.70100594205337	-2.18648395128973\\
-2.76670839564354	-2.16948958718696\\
-2.82298812770595	-2.15104809534126\\
-2.87164830808901	-2.13160199457462\\
-2.91408376440492	-2.11143085198241\\
-2.95138155578939	-2.09070216769097\\
-2.98439489737335	-2.06950500443297\\
-3.01379754605468	-2.04787217232614\\
-3.0401239091731	-2.02579474465893\\
-3.06379868525868	-2.00323135043708\\
-3.08515876112665	-1.98011382155867\\
-3.10446929599717	-1.95635020164888\\
-3.12193534452632	-1.93182574111754\\
-3.13770994317351	-1.90640223807004\\
-3.15189925874524	-1.87991589072489\\
-3.16456513585618	-1.85217367237726\\
-3.17572515028752	-1.82294810191056\\
-3.18535005003472	-1.79197014323174\\
-3.19335821697194	-1.75891980928172\\
-3.19960647651815	-1.7234138533874\\
-3.20387617749746	-1.68498968341194\\
-3.20585289974103	-1.64308431012539\\
-3.20509733712814	-1.59700671612584\\
-3.20100372563178	-1.54590148473199\\
-3.19274046786317	-1.48870085916713\\
-3.17916512589608	-1.4240616742884\\
-3.15870247919262	-1.35028304642604\\
-3.12916978909147	-1.26520095598012\\
-3.08752833431522	-1.16605847020069\\
-3.02953722802248	-1.04935888480526\\
-2.9492923192467	-0.910731054901148\\
-2.83866985696159	-0.744886013906675\\
-2.68680635565555	-0.545843632021727\\
-2.48001009197639	-0.307772295691794\\
-2.20297704173767	-0.0269519369913502\\
-1.84266886104903	0.294767101014322\\
-1.39569223068177	0.645751374552216\\
-0.876809173496224	1.0027034275221\\
-0.321315333824451	1.33532954261279\\
0.224920657710854	1.61726471950061\\
0.722602501490466	1.83555657428617\\
1.15023955592394	1.99171058627205\\
1.50385186179304	2.09599160249212\\
1.79026056897879	2.16104741003003\\
2.02047904511769	2.19817110296771\\
2.20572333459932	2.21607809894601\\
2.35571328277463	2.22100511820195\\
};
\addplot [color=mycolor1, forget plot]
  table[row sep=crcr]{%
2.41112293345451	2.67293497202217\\
2.46888392391395	2.67115131797152\\
2.51815038737103	2.6664855191518\\
2.56067745902692	2.65971096292733\\
2.59779343839778	2.65135197568606\\
2.6305157273557	2.64176311344075\\
2.65963253350677	2.63118102002161\\
2.68576094455292	2.61975863731399\\
2.70938859486226	2.60758789630244\\
2.73090382849039	2.59471475314211\\
2.75061769692551	2.58114902195038\\
2.76878007240405	2.56687056330675\\
2.78559143844458	2.55183281196918\\
2.8012114227226	2.53596424504605\\
2.81576478577869	2.51916812567583\\
2.82934531927977	2.50132065759491\\
2.84201790198104	2.48226751893041\\
2.85381878116153	2.46181858333906\\
2.8647539665374	2.43974046017921\\
2.87479541554687	2.41574626808028\\
2.8838744197952	2.3894817673208\\
2.89187122478307	2.36050657363737\\
2.89859935644222	2.32826859854975\\
2.90378227390621	2.29206901942597\\
2.90701863342752	2.2510138451484\\
2.90773033309651	2.20394632395823\\
2.90508412791903	2.14935179353612\\
2.89787220142401	2.08522283788636\\
2.88432854807521	2.00886770995555\\
2.86184503952644	1.916639659751\\
2.82653307728443	1.80356260538734\\
2.77255796402825	1.66284105249546\\
2.69117522918125	1.4853017815288\\
2.56949523498214	1.25900114520614\\
2.38939363438971	0.969699040009248\\
2.12804201406743	0.603783618187781\\
1.76333966100237	0.155977290279563\\
1.28774343536766	-0.358089056184036\\
0.725951136675671	-0.893477459637415\\
0.136840561367889	-1.38815401247061\\
-0.41314368327355	-1.79385084762324\\
-0.881710561305587	-2.09580917159059\\
-1.25843603279283	-2.30612464980497\\
-1.55288592047271	-2.44682454430528\\
-1.7813060775171	-2.53866033056596\\
-1.9593657775584	-2.59741700772015\\
-2.09975332575718	-2.63403834464104\\
-2.21203071942399	-2.65581837430199\\
-2.3032006562605	-2.66755558712192\\
-2.37835240081262	-2.67240733578158\\
-2.44119571607529	-2.67246443748957\\
-2.49445761919681	-2.66912333754563\\
-2.54016488193194	-2.66332432155596\\
-2.57984205540937	-2.6557040973601\\
-2.61464993480629	-2.64669419424165\\
-2.6454827915952	-2.63658501085421\\
-2.67303717734763	-2.62556789562519\\
-2.69786105735113	-2.61376299788296\\
-2.72038922461991	-2.60123775195561\\
-2.74096903869299	-2.58801907026067\\
-2.75987924735955	-2.57410120073673\\
-2.77734377856985	-2.55945048935761\\
-2.793541793656	-2.54400782159856\\
-2.80861487682785	-2.52768920006705\\
-2.82267193569293	-2.51038468758181\\
-2.83579215890869	-2.491955765212\\
-2.84802618740108	-2.47223099410424\\
-2.85939547818274	-2.45099970448246\\
-2.86988964901186	-2.42800324168044\\
-2.87946135839992	-2.40292305034405\\
-2.8880179592727	-2.37536453831611\\
-2.89540870763175	-2.34483518069009\\
-2.90140561910968	-2.31071462828254\\
-2.90567500103259	-2.27221356472263\\
-2.90773500983351	-2.22831655519435\\
-2.90689191136059	-2.17770193064862\\
-2.90214344552749	-2.11862858948262\\
-2.8920308887691	-2.04877526035613\\
-2.87441080959479	-1.96501245936017\\
-2.84610196161767	-1.86308285568593\\
-2.8023432419264	-1.73716857524885\\
-2.7359855348757	-1.5793538010083\\
-2.63637493654125	-1.37909972675918\\
-2.48809271214244	-1.12315205734959\\
-2.27038916682889	-0.796974956580139\\
-1.95964380346866	-0.38978904595946\\
-1.53877736296124	0.0949035646598871\\
-1.01455397500912	0.626922435424875\\
-0.430388009563244	1.14961064156151\\
0.146485075591148	1.60386358275269\\
0.65890842805611	1.95746634107526\\
1.08124245096797	2.21110290073615\\
1.41497076391917	2.38377830939657\\
1.67431219460888	2.49773918811789\\
1.87576040675745	2.57138888298126\\
2.03360307134768	2.61796802131199\\
2.15891606343668	2.64643668053227\\
2.25989814780917	2.66271445002484\\
2.34251989232733	2.67069163243849\\
2.41112293345451	2.67293497202217\\
};
\addplot [color=mycolor1, forget plot]
  table[row sep=crcr]{%
2.2306613868761	2.81803070297518\\
2.25810690410877	2.81718012527407\\
2.28201516003601	2.81491320421253\\
2.30308950693452	2.81155368741527\\
2.32186667064398	2.80732269213538\\
2.33876212010734	2.80236972560499\\
2.35410163785533	2.79679295696782\\
2.36814362598294	2.79065256203978\\
2.38109507726808	2.78397948565953\\
2.39312313224099	2.77678107656757\\
2.40436349690098	2.76904449916777\\
2.41492657297905	2.76073847420239\\
2.42490186835632	2.75181366374805\\
2.43436105653859	2.74220184468911\\
2.44335990651309	2.73181387613749\\
2.45193918373476	2.72053633609572\\
2.46012451013591	2.70822655989054\\
2.4679250472588	2.69470563459613\\
2.47533070951499	2.67974865995893\\
2.48230739246368	2.66307123355111\\
2.48878936392124	2.64431058778368\\
2.49466743013696	2.6229989883018\\
2.4997706103052	2.59852571315108\\
2.50383757236253	2.57008185871995\\
2.50647153235448	2.53657882915073\\
2.50706783155106	2.49652574468635\\
2.50469535887311	2.44784158954154\\
2.49789834998327	2.38756214474631\\
2.48435833264454	2.31137583767289\\
2.46030778970481	2.21288289339427\\
2.41950575059641	2.08242347523106\\
2.35147553534986	1.90531321212694\\
2.23868402405164	1.65958770416935\\
2.05300806937287	1.31467312543502\\
1.75510971736689	0.836517617528178\\
1.30849126874066	0.211189888458031\\
0.720798419557309	-0.511317529079945\\
0.0779140443342507	-1.20791854545575\\
-0.504310404303506	-1.76449354912077\\
-0.963651051034921	-2.15132519346601\\
-1.30133101338113	-2.40097689803586\\
-1.54462840311249	-2.55800318024844\\
-1.72127575581048	-2.65671069052081\\
-1.85211382956641	-2.71925994897991\\
-1.95136885095432	-2.75917169432432\\
-2.02850063951073	-2.78462216400938\\
-2.08981496010035	-2.80061234470425\\
-2.13957654655819	-2.81026070860493\\
-2.18072560715238	-2.81555396271566\\
-2.21533107455486	-2.81778429561671\\
-2.2448783990183	-2.81780785993293\\
-2.27045467227012	-2.81620060789198\\
-2.29286973444643	-2.81335422309845\\
-2.31273685207127	-2.80953638074975\\
-2.33052744438039	-2.80492929976484\\
-2.34660885364804	-2.79965478587334\\
-2.36127083449701	-2.7937906722762\\
-2.37474439994278	-2.78738164589066\\
-2.38721539229624	-2.78044630443655\\
-2.3988343416865	-2.77298159220663\\
-2.40972365384161	-2.76496532407901\\
-2.41998282343293	-2.75635722071773\\
-2.42969213301304	-2.74709867891067\\
-2.43891512794588	-2.73711134941806\\
-2.44770002625738	-2.72629446298113\\
-2.45608010789502	-2.71452071131373\\
-2.46407301201541	-2.70163033252864\\
-2.47167873369492	-2.68742284384416\\
-2.4788759266952	-2.67164557285018\\
-2.48561584719577	-2.65397770782051\\
-2.49181285049689	-2.63400793094603\\
-2.49732966899515	-2.61120267356518\\
-2.50195456246426	-2.58486040061168\\
-2.50536549382862	-2.554044685875\\
-2.50707310865982	-2.51748448173663\\
-2.5063283000101	-2.47342271799767\\
-2.50196930299109	-2.41938216717487\\
-2.49216346758393	-2.35179720417731\\
-2.47396280197052	-2.26542754675948\\
-2.4425289071623	-2.15242386901544\\
-2.38978363183492	-2.00087480010687\\
-2.30214558738244	-1.79274231006888\\
-2.15720197537847	-1.50172612091604\\
-1.92078092466796	-1.09405563207464\\
-1.55158685210626	-0.541139515258476\\
-1.02868761619051	0.144455732587051\\
-0.398536047886839	0.87154781456737\\
0.226482120535896	1.50768862618519\\
0.750336570209395	1.97786021388668\\
1.14618923806126	2.29038532705206\\
1.43290798360277	2.48861007795899\\
1.63980650934666	2.61301346167155\\
1.79139786468341	2.69150173946714\\
1.90500767470046	2.74143986091333\\
1.99224745257609	2.77333611988895\\
2.06083057765806	2.79357153066645\\
2.11593135374411	2.80608480485869\\
2.16108190935215	2.81335835477259\\
2.19874233023691	2.81699046255361\\
2.2306613868761	2.81803070297518\\
};
\addplot [color=mycolor1, forget plot]
  table[row sep=crcr]{%
1.34916822602105	2.61884380257413\\
1.36255348536219	2.61842535800733\\
1.37480683187091	2.61726037653587\\
1.38612752561835	2.61545290337742\\
1.39667669922158	2.61307332288931\\
1.40658638148471	2.61016585503015\\
1.41596613668646	2.60675348673496\\
1.42490800439879	2.60284105382679\\
1.43349020388433	2.598416912692\\
1.44177991815768	2.59345345180439\\
1.44983536863337	2.5879065527139\\
1.45770731417958	2.58171399239607\\
1.46544004576872	2.57479266390242\\
1.47307188948927	2.56703436125339\\
1.48063516614287	2.55829970555158\\
1.48815547206729	2.54840955221251\\
1.49565002401685	2.53713286811873\\
1.50312461938875	2.52416952842272\\
1.5105684469781	2.50912563214587\\
1.51794544380942	2.49147756217942\\
1.52517994674875	2.47051874931118\\
1.53213268434057	2.4452792850012\\
1.53856001621684	2.41440198359606\\
1.5440434126235	2.37594709462455\\
1.54786479332811	2.32707784393597\\
1.54878118856405	2.26354409964061\\
1.54460916042535	2.17882346641676\\
1.53144942360156	2.06269711309166\\
1.50225566229098	1.89898924372039\\
1.44436890588002	1.66251346478269\\
1.33620870262374	1.31721418768455\\
1.14673141186483	0.824026867167908\\
0.849782608285716	0.175946554623806\\
0.461685962801752	-0.548018956645931\\
0.0565516337693074	-1.20228090782282\\
-0.290753796572131	-1.6917502022023\\
-0.554835633179842	-2.01816874854801\\
-0.746358627160081	-2.22639575742822\\
-0.884817909103233	-2.3589451739538\\
-0.986659923035106	-2.44472049770824\\
-1.0634337309332	-2.50144568703854\\
-1.12282433770986	-2.53974002418693\\
-1.16991128279668	-2.5660218412918\\
-1.20809299884232	-2.58425316968421\\
-1.23968783716858	-2.59694161257033\\
-1.26631202944042	-2.60571445312303\\
-1.28911705407786	-2.61165259675128\\
-1.30894048914882	-2.61548905733766\\
-1.32640356331209	-2.617729744945\\
-1.34197546868493	-2.61872870075034\\
-1.35601659862283	-2.61873598932488\\
-1.36880817441296	-2.61792878298912\\
-1.38057291538259	-2.61643186607089\\
-1.39148970600711	-2.61433131778502\\
-1.40170416735516	-2.6116836843858\\
-1.4113363853036	-2.608522081478\\
-1.42048663037264	-2.6048601311055\\
-1.42923963244693	-2.60069429697314\\
-1.43766779293806	-2.5960049543598\\
-1.44583359306231	-2.59075637031308\\
-1.45379136809174	-2.58489564336501\\
-1.46158854894265	-2.57835053799512\\
-1.46926641317011	-2.57102602853142\\
-1.47686032733914	-2.56279922031462\\
-1.48439939058245	-2.55351211746392\\
-1.49190528944329	-2.54296141961962\\
-1.49939002248124	-2.53088409664441\\
-1.50685190825312	-2.51693681519435\\
-1.5142688787915	-2.50066621316221\\
-1.52158734818893	-2.48146525837629\\
-1.5287036798198	-2.4585079950171\\
-1.53543297058567	-2.43064999568616\\
-1.54145557404778	-2.39627321363943\\
-1.5462236004709	-2.35303882637033\\
-1.54879377227572	-2.29748516474355\\
-1.54752207454328	-2.22436239290059\\
-1.53949639900754	-2.12552421645835\\
-1.51947921042305	-1.98811596811773\\
-1.47800070676288	-1.79185510593083\\
-1.39834292507678	-1.50606847102606\\
-1.25373377570828	-1.09096289236972\\
-1.01217443181423	-0.516876345325173\\
-0.663483593147455	0.185309718307808\\
-0.255650441715128	0.892281146234285\\
0.127141962370146	1.46932437295849\\
0.432991150703737	1.87280614464764\\
0.658442536758492	2.13409571476104\\
0.821074581862312	2.30001032816487\\
0.939487523829889	2.40636173951891\\
1.02762555404293	2.47592484154283\\
1.09493604306106	2.52242215305209\\
1.14766234075684	2.55409174160465\\
1.18995036713984	2.57596124909194\\
1.22459961632259	2.59117267921784\\
1.25354049557583	2.60173993143481\\
1.27813357584593	2.60898559194935\\
1.29935845276156	2.61379764430181\\
1.31793477108438	2.61678384887558\\
1.33440123755647	2.61836680110905\\
1.34916822602105	2.61884380257413\\
};
\addplot [color=mycolor1, forget plot]
  table[row sep=crcr]{%
1.33832785075596	1.05856770426014\\
1.71708873794132	1.04706270790821\\
2.00524308812722	1.01997360948505\\
2.22158293985867	0.98568232028984\\
2.38421685443873	0.949191101343268\\
2.5076145272811	0.91313611260959\\
2.6024663989092	0.87874493050746\\
2.67642270413419	0.846477797541033\\
2.7349083019084	0.816402633286111\\
2.78177963434076	0.78840014646188\\
2.81980118326959	0.762271550607141\\
2.85097669866004	0.737793146276745\\
2.87677601172378	0.714742657350381\\
2.89829006124147	0.692910811688235\\
2.91633737206639	0.672105353139954\\
2.93153778559506	0.652151275517667\\
2.9443639890652	0.632889257135411\\
2.95517784931997	0.614173310232574\\
2.96425621180258	0.595868146084142\\
2.97180927822365	0.577846482438437\\
2.97799364973239	0.559986372597205\\
2.982921432524	0.542168554727318\\
2.98666632910257	0.524273773221959\\
2.9892673015886	0.506179992999476\\
2.99073014005198	0.487759402000392\\
2.99102706084517	0.46887507006363\\
2.99009426746616	0.449377098720827\\
2.9878272023742	0.429098051468598\\
2.98407297275139	0.407847392434451\\
2.97861910870765	0.38540457648988\\
2.97117735555587	0.36151031754284\\
2.96136053270739	0.335855404031068\\
2.94864948746336	0.308066221173719\\
2.93234564125112	0.27768587209012\\
2.91150227341512	0.244149475236976\\
2.88482407104405	0.206751914642895\\
2.8505189770917	0.164606224499872\\
2.80607829448728	0.11659141871253\\
2.74795009456766	0.0612912218850788\\
2.67105933973035	-0.00306711709908142\\
2.56812617464324	-0.0786451112657921\\
2.4287765554169	-0.167922293729583\\
2.23862361750258	-0.273355629826909\\
1.97903645009366	-0.396452773264885\\
1.62946597084443	-0.535843685786365\\
1.17549772978648	-0.684372465629206\\
0.623913899953506	-0.827066784603145\\
0.0156603832950979	-0.944197989256905\\
-0.582713904244943	-1.02082682525928\\
-1.1109120882187	-1.05489053430368\\
-1.54004749962896	-1.05543374616649\\
-1.87136800357853	-1.03485403068786\\
-2.12118690488697	-1.00334708083351\\
-2.30861435903366	-0.967504177395819\\
-2.45007070231361	-0.931008400315427\\
-2.55806939607659	-0.895691241500445\\
-2.64167391994001	-0.862335439946601\\
-2.70732795253916	-0.831170891254812\\
-2.75960186319709	-0.802153688786899\\
-2.80175664512797	-0.775115414285042\\
-2.83614241827917	-0.749840185559272\\
-2.86447275609878	-0.726102868844767\\
-2.88801214554155	-0.703686833934361\\
-2.90770435438771	-0.682391113449945\\
-2.92426093572038	-0.662032191603923\\
-2.9382227944058	-0.642443165421628\\
-2.95000341218912	-0.623471699475948\\
-2.9599194431015	-0.604977491641939\\
-2.96821248644699	-0.58682959233689\\
-2.97506458571853	-0.568903718994405\\
-2.98060916195791	-0.55107959889393\\
-2.98493852032551	-0.533238312675461\\
-2.98810867137945	-0.515259573693341\\
-2.99014191896274	-0.497018851100946\\
-2.99102743998762	-0.47838421890837\\
-2.99071988462755	-0.459212783487904\\
-2.98913583058736	-0.439346503305103\\
-2.98614770441886	-0.418607162070534\\
-2.981574503371	-0.396790184171176\\
-2.97516826859042	-0.373656881793238\\
-2.96659470971629	-0.348924587454162\\
-2.95540556317111	-0.322253943280443\\
-2.94099902838946	-0.293232380082817\\
-2.92256272972903	-0.261352524893812\\
-2.898990733363	-0.22598395477037\\
-2.86876167890882	-0.186336478897755\\
-2.82975838133331	-0.141413299954517\\
-2.77899972415068	-0.08995380351499\\
-2.71224382577774	-0.030370353344629\\
-2.62341239539384	0.0393040537726912\\
-2.50379953054959	0.121410226009802\\
-2.34112351143517	0.218486204153281\\
-2.11880645853817	0.332683514259896\\
-1.81669174283804	0.464386512203611\\
-1.41581099600258	0.609717854256119\\
-0.910199242096203	0.757659551907771\\
-0.32300024499543	0.8900063214995\\
0.289093551297695	0.98803779850949\\
0.858233627317148	1.04278813347991\\
1.33832785075596	1.05856770426014\\
};
\addplot [color=mycolor1]
  table[row sep=crcr]{%
2.22721472973942	1.94935886896179\\
2.3886648425121	1.94441487511101\\
2.5195634483927	1.93205457572182\\
2.62670779849127	1.91501766628131\\
2.71531853364	1.89508786650117\\
2.78936864949369	1.87341097537522\\
2.85187814440773	1.85071228241933\\
2.90515032165589	1.82744075521499\\
2.9509520177711	1.80386332654994\\
2.99064783647547	1.78012618777659\\
3.02529909977035	1.7562945351756\\
3.0557365623221	1.73237827445755\\
3.08261387510984	1.7083485310185\\
3.10644696879361	1.68414807630787\\
3.12764310033082	1.65969765855822\\
3.14652223914943	1.63489950162782\\
3.16333268711427	1.60963876367151\\
3.17826225469757	1.5837834336332\\
3.19144589188063	1.55718292750329\\
3.20297034763997	1.52966548843736\\
3.2128761683912	1.50103436869378\\
3.2211571124149	1.47106265802835\\
3.22775682617807	1.43948650786522\\
3.23256237104755	1.40599637010381\\
3.23539387180292	1.3702257107414\\
3.23598913837185	1.33173645766061\\
3.23398152983158	1.29000018434938\\
3.22886850103885	1.24437370419161\\
3.21996707976175	1.19406735000614\\
3.20635080827997	1.13810376656035\\
3.18676025898309	1.07526465092859\\
3.15947593702972	1.0040228169715\\
3.12213828317303	0.922457920675385\\
3.07149545967596	0.828157721349833\\
3.00305879148097	0.718116213117772\\
2.91065699973037	0.588662002756959\\
2.785925244978	0.435496623074383\\
2.61788823849866	0.254008232649376\\
2.39306798253518	0.0401523940333445\\
2.09699730361163	-0.20771900183087\\
1.7183819635995	-0.485932899271695\\
1.25639683089422	-0.782386182967809\\
0.728319451607428	-1.07597695395358\\
0.17048677629427	-1.34156752261761\\
-0.372344302750219	-1.55919115860543\\
-0.863264272549092	-1.72086595365401\\
-1.28299919742863	-1.8302347441912\\
-1.62890152595063	-1.89731652670458\\
-1.90833376158261	-1.93331843320936\\
-2.13240654410545	-1.94782687646357\\
-2.3122537087605	-1.94803363323748\\
-2.45746747168844	-1.93896359662991\\
-2.57574556593285	-1.92399109460735\\
-2.67305603610964	-1.90533078490532\\
-2.75395564612754	-1.88441460443591\\
-2.82190758462804	-1.86215611801615\\
-2.87954766726586	-1.83912801979308\\
-2.92889157233296	-1.81567886095776\\
-2.97149068204595	-1.79200910352774\\
-3.00854741762708	-1.76822049003994\\
-3.04100007190525	-1.74434802267285\\
-3.0695851473103	-1.7203805908067\\
-3.09488323143276	-1.69627413213334\\
-3.11735281839452	-1.67195981527391\\
-3.13735524534064	-1.64734883071173\\
-3.15517299859553	-1.62233479306363\\
-3.17102297575804	-1.59679437416259\\
-3.18506579905015	-1.570586527561\\
-3.19741190629232	-1.54355048208521\\
-3.20812485602917	-1.51550254281508\\
-3.21722203851139	-1.48623162017347\\
-3.22467275544594	-1.45549329519237\\
-3.23039339092638	-1.42300210774249\\
-3.23423911335776	-1.38842161161915\\
-3.23599118590349	-1.35135156242766\\
-3.23533847055249	-1.31131137681683\\
-3.23185101774822	-1.2677187105326\\
-3.22494264006069	-1.21986163848029\\
-3.2138179373885	-1.16686249014231\\
-3.19739719631547	-1.10763095340298\\
-3.17420973909784	-1.0408037811618\\
-3.142242561071	-0.964668760862181\\
-3.09872682936675	-0.877072578179989\\
-3.0398417586358	-0.77531812311728\\
-2.96031892266926	-0.656071422287709\\
-2.85295425309884	-0.515330753928937\\
-2.70811161177175	-0.348574764250833\\
-2.51348980394932	-0.151314840167112\\
-2.25479111797749	0.0795960527623154\\
-1.91840618384291	0.343563906305597\\
-1.49721915275415	0.633014609293368\\
-0.998726638487032	0.931115827109969\\
-0.450328181715587	1.21375185121395\\
0.105465468340197	1.45718721810745\\
0.625945116714116	1.64701594620063\\
1.08251666630864	1.78154116777267\\
1.46486936946524	1.86834353319716\\
1.77626331377619	1.91854897142989\\
2.02656829443923	1.94275375393752\\
2.22721472973941	1.94935886896179\\
};
\addlegendentry{Аппроксимации}

\addplot [color=mycolor2]
  table[row sep=crcr]{%
2.12132034355964	2.82842712474619\\
2.18582575000323	2.82640660874952\\
2.2459761483502	2.82067996094826\\
2.30316600770785	2.81153894842004\\
2.35833530419819	2.79908319427389\\
2.4121407651283	2.78328504568152\\
2.46505313939319	2.76402339892432\\
2.51741350921551	2.74110157112695\\
2.56946568278775	2.71425682483196\\
2.62137377470791	2.68316571665979\\
2.67322998558975	2.64744786912504\\
2.72505542741334	2.60667010622203\\
2.77679571750944	2.56035271147659\\
2.82831256449113	2.50797963523731\\
2.87937250576692	2.44901464186941\\
2.92963424895255	2.38292549191834\\
2.97863667364875	2.30921809960124\\
3.02579039162557	2.22748193169334\\
3.0703766721051	2.13744642550811\\
3.11155818595542	2.03904567139316\\
3.14840590189045	1.93248505458784\\
3.1799449834646	1.81829951655162\\
3.20521925970046	1.69738980361013\\
3.2233689191928	1.57102231442397\\
3.23371058284854	1.44078163681543\\
3.23580480244767	1.30847305653245\\
3.22949541157146	1.1759834831\\
3.21490900216393	1.04511946165368\\
3.19240995273442	0.917445713323391\\
3.16251366135309	0.794144173493542\\
3.12576385298257	0.675902054710668\\
3.08257536776515	0.562820516214273\\
3.03302813706062	0.454314775018543\\
2.97656426645917	0.348948817128694\\
2.91146921429886	0.244099783305823\\
2.83385680006128	0.135243682019644\\
2.73548220960228	0.0144130439331602\\
2.59871334341581	-0.133168683665202\\
2.38479638365891	-0.336180712113444\\
2.01057175466146	-0.648371525543042\\
1.34210041814469	-1.13791067535239\\
0.375352321166521	-1.75822148716338\\
-0.527448468220416	-2.26262801632552\\
-1.11365847778129	-2.5437869612038\\
-1.4567327767317	-2.68228003389134\\
-1.6691296256022	-2.75259678373195\\
-1.81428800439327	-2.79054335642282\\
-1.92320462744895	-2.81168897715777\\
-2.01132392805712	-2.82302365363059\\
-2.08680248126988	-2.82787464184343\\
-2.15422943553058	-2.82790871121207\\
-2.21634478307291	-2.8239827357917\\
-2.27487422572476	-2.8165264847195\\
-2.33095693947289	-2.80572465930185\\
-2.3853755403757	-2.79160756168836\\
-2.43868469976231	-2.77409782070174\\
-2.49128507004909	-2.75303495626884\\
-2.54346612600745	-2.7281884019911\\
-2.59543032955127	-2.69926454755406\\
-2.64730535029083	-2.66591103001887\\
-2.69914810041563	-2.62772046423944\\
-2.75094277189846	-2.58423541950896\\
-2.80259428647749	-2.53495641420315\\
-2.85391830194335	-2.47935483422118\\
-2.90462903903618	-2.41689283661657\\
-2.95432664759112	-2.34705230341292\\
-3.00248656709011	-2.26937453210355\\
-3.04845424185856	-2.18351130413452\\
-3.0914493816545	-2.08928598065547\\
-3.13058429431929	-1.98676020448289\\
-3.16490008683964	-1.87629788656397\\
-3.19342217916266	-1.75861429449935\\
-3.21523240499265	-1.63479579561046\\
-3.22954956680171	-1.50627700312196\\
-3.23580523135335	-1.37476797309193\\
-3.23369896357047	-1.2421341494958\\
-3.22321880671614	-1.11024302689198\\
-3.20461860896113	-0.980799507713277\\
-3.17835148049538	-0.85519274756288\\
-3.14496439092874	-0.734369602871579\\
-3.10495870881183	-0.618735254776386\\
-3.05861192744495	-0.508062722468019\\
-3.0057323937692	-0.401369642978784\\
-2.94527025004781	-0.296685462718974\\
-2.87460204254734	-0.190562899149911\\
-2.78805571026974	-0.0770301658026245\\
-2.67361421522119	0.0546866392302208\\
-2.50520482475694	0.225105040795738\\
-2.22541867988909	0.474087715816021\\
-1.7209533613293	0.868461990547897\\
-0.880186083660502	1.44500069250654\\
0.110323684998138	2.03883159526995\\
0.859259206598599	2.42737283484743\\
1.30741523004191	2.62511520505649\\
1.57426953542748	2.72304263504847\\
1.74766615441039	2.77435113512036\\
1.87207484559155	2.80265686669549\\
1.96924085114733	2.81832059141065\\
2.0502990590364	2.82612977760851\\
2.12132034355964	2.82842712474619\\
};
\addlegendentry{Сумма Минковского}

\end{axis}

\begin{axis}[%
width=0.798\linewidth,
height=0.597\linewidth,
at={(-0.104\linewidth,-0.066\linewidth)},
scale only axis,
xmin=0,
xmax=1,
ymin=0,
ymax=1,
axis line style={draw=none},
ticks=none,
axis x line*=bottom,
axis y line*=left,
legend style={legend cell align=left, align=left, draw=white!15!black}
]
\end{axis}
\end{tikzpicture}%
        \caption{Эллипсоидальные аппроксимации для 10 направлений.}
\end{figure}
\clearpage
\begin{figure}[t]

        \centering
        % This file was created by matlab2tikz.
%
%The latest updates can be retrieved from
%  http://www.mathworks.com/matlabcentral/fileexchange/22022-matlab2tikz-matlab2tikz
%where you can also make suggestions and rate matlab2tikz.
%
\definecolor{mycolor1}{rgb}{0.00000,0.44700,0.74100}%
\definecolor{mycolor2}{rgb}{0.85000,0.32500,0.09800}%
%
\begin{tikzpicture}

\begin{axis}[%
width=0.618\linewidth,
height=0.487\linewidth,
at={(0\linewidth,0\linewidth)},
scale only axis,
xmin=-4,
xmax=4,
xlabel style={font=\color{white!15!black}},
xlabel={$x_1$},
ymin=-3,
ymax=3,
ylabel style={font=\color{white!15!black}},
ylabel={$x_2$},
axis background/.style={fill=white},
axis x line*=bottom,
axis y line*=left,
xmajorgrids,
ymajorgrids,
legend style={at={(0.03,0.97)}, anchor=north west, legend cell align=left, align=left, draw=white!15!black}
]
\addplot [color=mycolor1, forget plot]
  table[row sep=crcr]{%
2.36712178368749	2.74633057020532\\
2.41188305682678	2.74494651398331\\
2.45036041459025	2.74130089565813\\
2.48383536069915	2.73596692388682\\
2.51327948444918	2.72933446739078\\
2.53943919311047	2.72166753585199\\
2.56289492242467	2.71314181506891\\
2.58410302011837	2.70386940753685\\
2.60342571833248	2.69391520541791\\
2.62115280082523	2.68330766515176\\
2.63751738560017	2.67204572812381\\
2.65270745907852	2.66010298488441\\
2.66687427014079	2.64742976011011\\
2.68013832873481	2.63395351039579\\
2.69259349381253	2.61957772004863\\
2.70430943819101	2.6041793136995\\
2.71533261385189	2.58760445132293\\
2.72568568587575	2.56966240776041\\
2.73536523285703	2.55011704085516\\
2.74433729805113	2.52867508999008\\
2.75253007967915	2.50497017880853\\
2.75982261143016	2.47854086090236\\
2.76602760995085	2.44880025137829\\
2.77086559717422	2.41499358451471\\
2.77392567575526	2.37613820122431\\
2.77460549370625	2.33093765236215\\
2.77201821305984	2.27765729690984\\
2.7648464218491	2.21394231571241\\
2.7511098920125	2.13654986655322\\
2.72779317877431	2.04095569483249\\
2.69024836192134	1.92078707468285\\
2.6312539409082	1.7670493095065\\
2.53961413480675	1.56721869756115\\
2.39837999194539	1.30464987086518\\
2.18360687141246	0.959750006540987\\
1.8667826621648	0.516209188824812\\
1.42701784641989	-0.0238781630653391\\
0.87502831245873	-0.620875059217897\\
0.268936641059173	-1.19901922110702\\
-0.309576813214938	-1.68530797756881\\
-0.802100907580029	-2.04896030059078\\
-1.19206161730571	-2.300440537754\\
-1.49041992551985	-2.46708460024485\\
-1.7168759123608	-2.57532489307401\\
-1.88995458136084	-2.64492059461505\\
-2.02413470299114	-2.68919932969544\\
-2.12995725581116	-2.71680265831965\\
-2.21490837909904	-2.73327951527395\\
-2.284286999297	-2.74220891806166\\
-2.34187032729419	-2.74592425079378\\
-2.39038313346388	-2.74596637420131\\
-2.4318186528206	-2.74336542452957\\
-2.46765634360889	-2.73881708355027\\
-2.49901014208278	-2.7327940719517\\
-2.52673028332805	-2.72561759993438\\
-2.551474024202	-2.71750375208859\\
-2.57375536742722	-2.70859393338899\\
-2.5939804436843	-2.69897499500867\\
-2.61247296631128	-2.688692537203\\
-2.6294927100655	-2.67775958603992\\
-2.64524900273341	-2.66616202895009\\
-2.6599105764423	-2.65386167486455\\
-2.67361268902278	-2.64079746083638\\
-2.68646212009346	-2.62688508668123\\
-2.69854042224194	-2.61201517643097\\
-2.70990563030842	-2.59604990865956\\
-2.72059247497691	-2.57881790224895\\
-2.73061098716967	-2.5601069666803\\
-2.73994319179238	-2.53965410007404\\
-2.74853734053432	-2.51713180921804\\
-2.75629877602211	-2.49212938365471\\
-2.76307597883326	-2.46412710489964\\
-2.76863950239953	-2.43246039507853\\
-2.77265014411951	-2.39626942454063\\
-2.7746104875259	-2.35442742359956\\
-2.77379029091263	-2.3054374557388\\
-2.76911009981251	-2.24728211523106\\
-2.75895730729457	-2.17720283029247\\
-2.74089227231279	-2.0913749275097\\
-2.71117633487659	-1.98443346302753\\
-2.66401914162082	-1.84880387542276\\
-2.59041768625865	-1.67383876852635\\
-2.47652724106683	-1.44496938777841\\
-2.30193857203366	-1.14371563581315\\
-2.0396739695219	-0.750852651889909\\
-1.66265564107253	-0.256803878437201\\
-1.16248546087895	0.319450601512933\\
-0.573614265552006	0.917547528939285\\
0.0284025713818862	1.45675943888985\\
0.5685251002902	1.88250662286626\\
1.00960568842464	2.18713123722861\\
1.35150239782658	2.39258093926768\\
1.61139397123953	2.52710219556684\\
1.80908182046774	2.61398894502504\\
1.96116532773652	2.66959557253549\\
2.08006223005889	2.7046815507453\\
2.17466815864962	2.726172144186\\
2.25127885929965	2.73851896753466\\
2.3143624731471	2.74460741055068\\
2.36712178368749	2.74633057020532\\
};
\addplot [color=mycolor1, forget plot]
  table[row sep=crcr]{%
2.35240505464064	2.76075021334929\\
2.39429606969185	2.75945449074689\\
2.43037411286026	2.75603583797069\\
2.46182071964198	2.75102474749105\\
2.48953226372909	2.7447822797979\\
2.51419796801997	2.7375529548978\\
2.53635434440832	2.72949928507746\\
2.55642367540865	2.72072453208959\\
2.57474154346732	2.71128775569393\\
2.59157672930874	2.70121369370841\\
2.6071457034148	2.69049907088173\\
2.62162321073745	2.67911633833517\\
2.63514996235992	2.66701545720637\\
2.6478381125071	2.65412407506413\\
2.65977495873579	2.64034624885663\\
2.67102511914481	2.62555970842345\\
2.68163128510145	2.60961150375056\\
2.69161349794549	2.59231171409547\\
2.70096672992684	2.57342469308632\\
2.70965633462077	2.55265704944946\\
2.71761062979593	2.52964117359486\\
2.72470942365474	2.50391254885291\\
2.73076659186681	2.4748782286687\\
2.73550368771706	2.44177255370809\\
2.73850973115383	2.40359416717204\\
2.73917927620848	2.35901526004097\\
2.73661574356727	2.30624913708954\\
2.72947837780359	2.24285482082059\\
2.7157367198957	2.16544669277417\\
2.69227293033385	2.06926342687234\\
2.65423709364454	1.94753924024194\\
2.59402035163186	1.79063563981677\\
2.49971323572562	1.58501269692596\\
2.35315024241768	1.31256503039365\\
2.12864777932873	0.952067155541005\\
1.79621437753681	0.486680047250032\\
1.33625613088531	-0.078254314338472\\
0.766437217218612	-0.69465416704619\\
0.154503796467904	-1.27852819912494\\
-0.414737908033144	-1.75715359300286\\
-0.888451837771183	-2.10699420135133\\
-1.25739472197187	-2.34495625430441\\
-1.53680289125242	-2.50103042562183\\
-1.74770650766029	-2.60184210371291\\
-1.90850464434176	-2.66650103010414\\
-2.03309498455462	-2.70761509614372\\
-2.13140980780297	-2.73325953784195\\
-2.21042927392142	-2.7485852910304\\
-2.27506472386636	-2.75690363778087\\
-2.32880559720436	-2.76037053426479\\
-2.37416495660386	-2.76040947101594\\
-2.41298006388656	-2.75797261878991\\
-2.44661469143309	-2.75370354023524\\
-2.4760961740651	-2.74803989675378\\
-2.50220924915406	-2.74127920764433\\
-2.52556111884979	-2.73362152710032\\
-2.54662714872804	-2.72519745309993\\
-2.56578336978446	-2.71608663470736\\
-2.583329856457	-2.70632998808298\\
-2.59950769563699	-2.69593763465725\\
-2.61451137202192	-2.68489382800076\\
-2.6284978033132	-2.67315965738214\\
-2.64159285649488	-2.66067399757508\\
-2.65389589416861	-2.64735294979028\\
-2.66548269162024	-2.63308784501451\\
-2.67640689871448	-2.61774172870901\\
-2.68670007084269	-2.60114409074919\\
-2.69637013715653	-2.58308342333688\\
-2.70539798684275	-2.56329695495395\\
-2.71373160169024	-2.54145658319533\\
-2.72127679579021	-2.51714955916172\\
-2.72788306153676	-2.48985177771614\\
-2.73332213400089	-2.45889047089402\\
-2.73725545152916	-2.42339148023845\\
-2.73918432926723	-2.38220377433743\\
-2.73837272033678	-2.33378998386214\\
-2.73372579881979	-2.27606573156975\\
-2.7235964056328	-2.20616156525457\\
-2.70547282555558	-2.12006886105957\\
-2.67547208091285	-2.01211722390011\\
-2.62752302154944	-1.87422773696123\\
-2.55209340449926	-1.69493840971088\\
-2.43439384879093	-1.45844070952189\\
-2.25251213766288	-1.14463119254633\\
-1.97767200659054	-0.732949898706597\\
-1.5822742236057	-0.214800426101576\\
-1.06187245772022	0.384846887446284\\
-0.460163049994717	0.99612986835765\\
0.139878547911511	1.53372324957911\\
0.664968942177641	1.94772499231274\\
1.08537079974014	2.23812186948185\\
1.40696804069012	2.43139655318979\\
1.64956497932395	2.55697472989305\\
1.83339591262838	2.63777387015065\\
1.97462394608275	2.68941180394685\\
2.08504289043958	2.72199559336845\\
2.17298471767437	2.74197177729419\\
2.24429964729586	2.75346450848608\\
2.30312113316005	2.75914104476399\\
2.35240505464064	2.76075021334929\\
};
\addplot [color=mycolor1, forget plot]
  table[row sep=crcr]{%
2.33637221543638	2.77359127925803\\
2.37555924354049	2.77237880820087\\
2.40937123794229	2.76917454596508\\
2.43889707827725	2.76446924085856\\
2.46496343861871	2.7585971151577\\
2.48820653315317	2.75178450780297\\
2.50912212950084	2.744181630855\\
2.52810088298891	2.73588349316341\\
2.54545360964711	2.7269437243832\\
2.56142955275294	2.71738362781807\\
2.57622968413772	2.70719792374284\\
2.5900164143854	2.69635809673865\\
2.60292063833293	2.68481390212879\\
2.61504673328514	2.67249333997666\\
2.62647590475164	2.65930122141101\\
2.63726810274461	2.64511629832438\\
2.64746258445559	2.62978677865868\\
2.65707705402952	2.61312388268925\\
2.66610514396634	2.59489288554136\\
2.67451178632173	2.57480080444892\\
2.68222571341034	2.55247947808503\\
2.68912786135759	2.52746217653571\\
2.69503371742093	2.49915095925159\\
2.69966647054546	2.46677058143266\\
2.70261587912763	2.42930254372515\\
2.70327451243641	2.38538942141003\\
2.70073749781637	2.33319418508968\\
2.69364247410982	2.27019083236438\\
2.67991042087211	2.19285019863564\\
2.65632152704606	2.09616833205675\\
2.61781990553633	1.97297012766626\\
2.556393576908	1.81293522681722\\
2.4593772940689	1.60143082913913\\
2.30730201772547	1.31876620611291\\
2.07262878699913	0.94196376050997\\
1.72391573965135	0.453789171162566\\
1.24344969741539	-0.136391318768668\\
0.656926099516598	-0.770996753093629\\
0.0417886619641087	-1.35809301456615\\
-0.515623703650036	-1.82689899376119\\
-0.969303541781385	-2.16201421670412\\
-1.31723069543787	-2.38645255455632\\
-1.57829933202544	-2.53229442217931\\
-1.77441694676618	-2.62604232656544\\
-1.92364912764283	-2.68605128040974\\
-2.03924856860398	-2.72419816479556\\
-2.13053605171207	-2.74800903746349\\
-2.20400284943405	-2.76225725612886\\
-2.26419305112633	-2.77000296444939\\
-2.31432632834073	-2.77323664854273\\
-2.35671847301735	-2.77327262353084\\
-2.39306188920309	-2.77099058863201\\
-2.42461308321403	-2.76698563780904\\
-2.45231913795599	-2.76166279258598\\
-2.47690408935424	-2.75529748631566\\
-2.49892872559441	-2.74807481098525\\
-2.51883255679587	-2.74011527302354\\
-2.53696365504808	-2.73149180432721\\
-2.55360011571802	-2.72224097327856\\
-2.56896563426422	-2.71237024001503\\
-2.583240872228	-2.7018624133194\\
-2.59657174112777	-2.69067802547869\\
-2.60907536260046	-2.67875604645876\\
-2.62084420261484	-2.66601314846319\\
-2.63194868411551	-2.65234156671407\\
-2.64243842589445	-2.63760545389755\\
-2.65234211207123	-2.62163547080534\\
-2.66166584405338	-2.60422117073813\\
-2.67038963981693	-2.58510049165977\\
-2.6784614893414	-2.5639453284014\\
-2.68578799793619	-2.54034165851501\\
-2.69222006701896	-2.51376194823722\\
-2.69753113435073	-2.48352642453764\\
-2.70138398314494	-2.44874803365179\\
-2.70327961671456	-2.40825314607774\\
-2.70247745867555	-2.3604657339348\\
-2.69786892269196	-2.3032359796149\\
-2.68777407981268	-2.23358396910862\\
-2.66961041986487	-2.14731447222314\\
-2.63934946132841	-2.03844174415473\\
-2.59063069771097	-1.89835713280234\\
-2.51336604572977	-1.71472897173176\\
-2.3917587955001	-1.47040741966589\\
-2.20228433366304	-1.14352791485559\\
-1.91428524384337	-0.712154976758204\\
-1.49989381707048	-0.169092576727588\\
-0.959545669549715	0.453636680936888\\
-0.346998703073263	1.076089367383\\
0.248235503995792	1.60952777200189\\
0.756340489353792	2.01023492636414\\
1.15552289245518	2.28602185753826\\
1.45720069343643	2.46734487511429\\
1.6832308024427	2.58435439754378\\
1.8539597705315	2.65939683811413\\
1.984991847698	2.70730692674371\\
2.087470984034	2.73754728674876\\
2.16917584595401	2.75610613263204\\
2.23553100450468	2.7667989929561\\
2.29035472161473	2.77208920980101\\
2.33637221543638	2.77359127925803\\
};
\addplot [color=mycolor1, forget plot]
  table[row sep=crcr]{%
2.31899511579583	2.78496710127215\\
2.35562981603182	2.78383324360938\\
2.38729775592243	2.78083185798896\\
2.41500170562571	2.77641662221972\\
2.43950357273108	2.77089669699762\\
2.46139029705334	2.76448142446218\\
2.48111974715084	2.75730951010792\\
2.49905313489982	2.74946824664576\\
2.51547819869842	2.74100620271409\\
2.53062596016773	2.731941508788\\
2.54468292505632	2.72226707646172\\
2.55779998553553	2.71195358311898\\
2.57009886950765	2.70095072299508\\
2.58167669807872	2.6891869960311\\
2.59260900645553	2.676568132452\\
2.60295142313444	2.66297410276083\\
2.61274006261332	2.64825451572018\\
2.6219905463938	2.63222203817396\\
2.63069540277576	2.61464325412882\\
2.63881937832462	2.59522608137787\\
2.64629187947257	2.57360243066363\\
2.65299528241172	2.54930414537656\\
2.65874708838325	2.52172927256302\\
2.66327266338513	2.49009418446174\\
2.66616324468682	2.45336466284387\\
2.66681042018401	2.41015524261165\\
2.66430232883478	2.35858004977787\\
2.65725653684722	2.29602884307349\\
2.64354680791206	2.21882754576371\\
2.61985118912713	2.12172286615602\\
2.58090258556044	1.99711165873042\\
2.51826721495296	1.83394833927724\\
2.41847530624473	1.61641992585627\\
2.26065458999511	1.32310807005003\\
2.01528220827559	0.929154871667727\\
1.64952589629397	0.417120007567265\\
1.1482714408185	-0.198666982227323\\
0.54640905154937	-0.850020916681251\\
-0.0691017291823024	-1.43764535137375\\
-0.612199948864583	-1.89453488182135\\
-1.04479878658608	-2.21414158478757\\
-1.3718153396137	-2.42511771476334\\
-1.61516534393746	-2.56107110241224\\
-1.79722379344277	-2.64810143533772\\
-1.93555021217121	-2.70372546121445\\
-2.04270539109275	-2.73908544849924\\
-2.12740163597375	-2.76117652473781\\
-2.19565920649712	-2.77441387849936\\
-2.25167424964972	-2.78162177519906\\
-2.29841322926979	-2.7846360705932\\
-2.33800775259184	-2.78466928587429\\
-2.37201536576894	-2.78253358634189\\
-2.40159285489621	-2.7787788787706\\
-2.42761273157058	-2.77377972370747\\
-2.45074261926824	-2.76779091832854\\
-2.47150014893828	-2.7609835610781\\
-2.49029146552292	-2.75346871855796\\
-2.50743860077189	-2.74531304937084\\
-2.52319916028231	-2.73654908333665\\
-2.53778061266345	-2.72718184373065\\
-2.55135071340469	-2.71719286899586\\
-2.56404509506734	-2.70654228408706\\
-2.57597271479835	-2.69516929825076\\
-2.58721960988191	-2.68299130905499\\
-2.5978512323804	-2.66990163492385\\
-2.60791348668932	-2.65576575356513\\
-2.61743245645768	-2.64041576869778\\
-2.62641265810546	-2.62364263858406\\
-2.63483347157999	-2.60518544727419\\
-2.64264313965648	-2.5847166411094\\
-2.6497493406152	-2.56182162521395\\
-2.65600473681119	-2.53597031735907\\
-2.66118493374687	-2.50647702921167\\
-2.6649546923366	-2.47244312764962\\
-2.66681556845443	-2.43267390052964\\
-2.66602360944957	-2.38555624036334\\
-2.66145791012851	-2.32887614154429\\
-2.65140730054231	-2.25954319945489\\
-2.63321932490039	-2.17317210160491\\
-2.60271796065926	-2.06345017956284\\
-2.5532428781369	-1.92121005516382\\
-2.47411910018783	-1.73318767435553\\
-2.34847036311905	-1.48077624467292\\
-2.1510355617258	-1.1401956316382\\
-1.84919435755573	-0.688107897947388\\
-1.41514372797696	-0.119250957563104\\
-0.855280865430319	0.526083098359011\\
-0.234160930659898	1.1574202658589\\
0.353373767113646	1.68411073867882\\
0.842696187852734	2.07009612486439\\
1.22026952463506	2.33099564995725\\
1.50246043079355	2.50062249797124\\
1.71263134581766	2.60942771688335\\
1.87096320639038	2.67902252218549\\
1.99240428591782	2.72342573228369\\
2.08743319127747	2.75146710747536\\
2.16328836969436	2.76869663302595\\
2.22498824023365	2.77863875701106\\
2.27605403234581	2.78356586654782\\
2.31899511579583	2.78496710127215\\
};
\addplot [color=mycolor1, forget plot]
  table[row sep=crcr]{%
2.30020669994137	2.79497003360658\\
2.33442779348364	2.79391054443853\\
2.36406366192883	2.79110146412444\\
2.39003684377099	2.78696181192526\\
2.41304890569281	2.78177729330581\\
2.43364086342606	2.77574132765827\\
2.45223524061943	2.7689818362945\\
2.46916576788949	2.76157888616564\\
2.48469862514458	2.75357632318545\\
2.4990477973216	2.74498934528772\\
2.51238625496893	2.73580923500262\\
2.52485410803677	2.72600600801785\\
2.53656450354794	2.71552942862473\\
2.54760777625928	2.70430862911225\\
2.55805417112397	2.69225040614971\\
2.56795530659792	2.67923612396515\\
2.57734441530391	2.66511700844601\\
2.58623526248155	2.6497074455038\\
2.5946194802314	2.6327756742544\\
2.60246183676749	2.61403095386041\\
2.60969263975395	2.59310582709883\\
2.61619597887432	2.56953141746383\\
2.62179172332008	2.542702640977\\
2.62620789476024	2.51182856207531\\
2.62903786604208	2.47586050220534\\
2.62967313221243	2.43338630950981\\
2.62719598509371	2.38247244008694\\
2.62020520353049	2.32042471414226\\
2.60652826894451	2.24342194673642\\
2.58274014246277	2.14595321992017\\
2.54335564670146	2.01996541621218\\
2.47949678751822	1.85363764394082\\
2.37683220232506	1.62987621008494\\
2.21297126498285	1.32537261555615\\
1.95626592344349	0.913253136607868\\
1.57259905293611	0.376138452149806\\
1.05034058210994	-0.265538845068828\\
0.43480429548368	-0.931861720771676\\
-0.178039901245535	-1.5171177814555\\
-0.704415039910225	-1.96005554934772\\
-1.11504702584882	-2.26348951010003\\
-1.42134106334064	-2.46111901777462\\
-1.64759369620293	-2.58752837680277\\
-1.81627967256246	-2.66816836568712\\
-1.94431008888774	-2.7196523161846\\
-2.04352060001263	-2.75239019587104\\
-2.12202234173039	-2.77286498390856\\
-2.185382705902	-2.78515204327266\\
-2.237468123753	-2.79185378693279\\
-2.28100707369837	-2.79466128154541\\
-2.31795885555641	-2.79469191916393\\
-2.34975517450592	-2.7926947809758\\
-2.37745987784314	-2.78917754473946\\
-2.40187600344441	-2.78448627762359\\
-2.42361860980879	-2.7788564533485\\
-2.44316509317433	-2.77244605686927\\
-2.46089046981923	-2.76535730926546\\
-2.47709245601405	-2.75765100206573\\
-2.49200950805543	-2.74935591027353\\
-2.50583391707323	-2.74047482637293\\
-2.51872135965986	-2.73098817820737\\
-2.53079784572123	-2.72085581955979\\
-2.54216469215797	-2.71001732897607\\
-2.55290192959034	-2.69839096767582\\
-2.56307038255415	-2.68587129689785\\
-2.5727125250979	-2.67232531332748\\
-2.58185208201898	-2.65758680599976\\
-2.59049219988251	-2.64144844511845\\
-2.59861182582681	-2.62365085150195\\
-2.606159669656	-2.60386752009051\\
-2.61304472916205	-2.5816839133836\\
-2.6191217364694	-2.55656819154151\\
-2.62416887494901	-2.52782972793532\\
-2.6278534437985	-2.49455948149816\\
-2.62967831816298	-2.4555429826136\\
-2.62889718603392	-2.40913136117688\\
-2.62437806539607	-2.35304728947931\\
-2.61437975905681	-2.28408921703587\\
-2.59618015937168	-2.19767714406857\\
-2.56545255417668	-2.08715767216677\\
-2.51522386318292	-1.94277105292442\\
-2.43419553653303	-1.7502485606065\\
-2.30432773508839	-1.48939355868306\\
-2.09848152744881	-1.1343369241126\\
-1.78199733962051	-0.660334255898569\\
-1.32757728226521	-0.0647399507045235\\
-0.748830647813808	0.60249490401494\\
-0.121708119297749	1.24011936632773\\
0.455176324749901	1.75741345798711\\
0.924069208136567	2.12736739288647\\
1.27977406430857	2.37319242181238\\
1.5429475354994	2.53140131594123\\
1.73794202389831	2.63235402061753\\
1.88453375381058	2.69678935047547\\
1.99693909722587	2.73788853009942\\
2.08496394786265	2.76386257147755\\
2.15532142036926	2.77984277321716\\
2.21264266329337	2.78907880461998\\
2.26016874412744	2.79366392340583\\
2.30020669994137	2.79497003360658\\
};
\addplot [color=mycolor1, forget plot]
  table[row sep=crcr]{%
2.27989423025359	2.80367012813669\\
2.31182909506837	2.80268110738935\\
2.33953609037404	2.80005458825546\\
2.36386281781723	2.79617711440432\\
2.38545448717227	2.79131239293695\\
2.40480920655615	2.78563889722791\\
2.42231645590045	2.77927442324971\\
2.43828425414907	2.7722922606117\\
2.45295859430733	2.76473184231069\\
2.46653749575951	2.7566056535963\\
2.4791812362186	2.74790351148229\\
2.49101981016599	2.73859490116034\\
2.50215831474622	2.72862977367262\\
2.51268072396356	2.71793801000599\\
2.52265233631157	2.70642760149479\\
2.53212104114996	2.69398145776753\\
2.54111742329624	2.68045260898249\\
2.54965359336202	2.66565739609871\\
2.55772047072465	2.64936601366224\\
2.56528302613558	2.63128944511472\\
2.57227266530954	2.61106135187156\\
2.57857542753652	2.58821275073957\\
2.58401385557852	2.56213618744709\\
2.58831904026199	2.53203433579653\\
2.5910870560991	2.4968451036069\\
2.59171006355139	2.45513071157973\\
2.58926545338738	2.40491069026903\\
2.58233419211587	2.34340654738009\\
2.56869787676544	2.26664664449753\\
2.54482641466212	2.16885201665013\\
2.50500751445182	2.04149340089486\\
2.4398917399604	1.87191587913018\\
2.33421854445463	1.64162796522165\\
2.16394499060052	1.32524422830162\\
1.8951439591019	0.893735738631335\\
1.49258625080902	0.330161276340648\\
0.949220739602002	-0.337555860020286\\
0.322047854491551	-1.01666579883714\\
-0.284862839313155	-1.59643693298567\\
-0.79218484411334	-2.02345199049009\\
-1.18011059909213	-2.31015726127132\\
-1.4659367423745	-2.49459919940231\\
-1.67570670773632	-2.61180569021119\\
-1.8316675525631	-2.68636393057746\\
-1.94996553587042	-2.73393415501005\\
-2.04168892856542	-2.76420086393755\\
-2.11435811405785	-2.78315379997754\\
-2.1731056775555	-2.79454577443937\\
-2.22148547198689	-2.80077024555496\\
-2.26200198522569	-2.80338244448942\\
-2.29645300988559	-2.80341066898991\\
-2.3261525625886	-2.80154493910368\\
-2.35207766536885	-2.79825337903581\\
-2.37496647227013	-2.79385534504486\\
-2.39538494038225	-2.78856817949742\\
-2.41377286330605	-2.78253755533497\\
-2.43047614459481	-2.77585739089817\\
-2.44576974145904	-2.7685829824573\\
-2.459874171878	-2.76073961003977\\
-2.47296749859146	-2.75232802354111\\
-2.48519406814302	-2.74332768503392\\
-2.49667086234683	-2.73369829912388\\
-2.50749203283999	-2.7233799285648\\
-2.51773198565752	-2.71229181905291\\
-2.52744722803038	-2.70032991313774\\
-2.53667705929853	-2.68736289432676\\
-2.54544306149733	-2.67322644674672\\
-2.55374720206252	-2.65771521859413\\
-2.56156817541707	-2.64057170660506\\
-2.56885534467735	-2.62147088627847\\
-2.57551924031023	-2.59999882451805\\
-2.58141693080415	-2.57562260863434\\
-2.58632953165119	-2.54764751297536\\
-2.58992736438644	-2.51515507764702\\
-2.59171528196931	-2.47691215235485\\
-2.59094547146427	-2.43123506817258\\
-2.58647587216081	-2.37578350774903\\
-2.57653607048487	-2.30724324881995\\
-2.55833386982738	-2.22083344899758\\
-2.52738724967351	-2.10954318377554\\
-2.47639423807661	-1.96298063156187\\
-2.39338883531493	-1.76578796581867\\
-2.2590688978225	-1.49602457183854\\
-2.04425621765017	-1.12553780210027\\
-1.71218911508775	-0.628210747673746\\
-1.23666007004454	-0.00489431482113671\\
-0.63993220037813	0.68322735642396\\
-0.00973267835069244	1.32417897840899\\
0.553493837725301	1.82937408032667\\
1.00045453225166	2.18209994984005\\
1.33414363399572	2.41274133225413\\
1.57879406457006	2.55982595724772\\
1.75926752282458	2.6532640326178\\
1.89473119297408	2.71280842369238\\
1.99861121015519	2.75079006261327\\
2.08003965083966	2.77481706052269\\
2.14522029762639	2.78962086867523\\
2.19841515595272	2.79819150981641\\
2.24260078315941	2.80245391783868\\
2.27989423025359	2.80367012813669\\
};
\addplot [color=mycolor1, forget plot]
  table[row sep=crcr]{%
2.25788870292592	2.81111255677442\\
2.287654916644	2.81019040218352\\
2.3135286657741	2.80773741291854\\
2.33628740238797	2.8041096403732\\
2.35652357987319	2.79955011947793\\
2.37469512569472	2.79422326875967\\
2.39116055804213	2.78823736127986\\
2.40620378693669	2.78165931984663\\
2.42005186230477	2.77452444794562\\
2.43288781059253	2.766842718315\\
2.44485998250258	2.75860262975216\\
2.45608886382766	2.74977325298756\\
2.46667198535894	2.74030482662775\\
2.47668734784351	2.73012807860742\\
2.48619561592421	2.71915230142912\\
2.49524120452002	2.70726207500777\\
2.50385226155616	2.69431238739626\\
2.51203942295326	2.68012172829357\\
2.5197930568574	2.6644624943301\\
2.52707849316102	2.64704770760462\\
2.53382840332611	2.62751254658506\\
2.53993097497982	2.60538841995103\\
2.54521167952737	2.58006611276808\\
2.54940501935253	2.55074262249637\\
2.55211023338693	2.51634321065097\\
2.55272075042301	2.47540513158836\\
2.55030976292687	2.42590114020182\\
2.5434409975938	2.36496711601297\\
2.5298498499946	2.28847602158638\\
2.50589782519448	2.19036749895199\\
2.46563359806467	2.06160428840486\\
2.399202721541	1.88862746060108\\
2.29033437288762	1.65140967199981\\
2.11317626587243	1.32227278785799\\
1.83135769246533	0.869896633398846\\
1.40881012514373	0.278313060321905\\
0.844418841760294	-0.415371904508185\\
0.208111714600051	-1.10458374955839\\
-0.389353359035489	-1.67551535364681\\
-0.875374687306539	-2.08470429598352\\
-1.23998841711074	-2.35422379466866\\
-1.50565468425842	-2.52567217950787\\
-1.69954564436864	-2.63401111711252\\
-1.84339117402125	-2.70277841441165\\
-1.95247849740747	-2.746644511391\\
-2.03713505805024	-2.77457870618071\\
-2.10430312402153	-2.79209626869318\\
-2.15869821427116	-2.80264369279087\\
-2.20357767997811	-2.80841735787632\\
-2.24123489720429	-2.81084482758961\\
-2.27331599547042	-2.81087078825918\\
-2.30102469660419	-2.80912984941941\\
-2.32525672835316	-2.80605301082385\\
-2.34668950945295	-2.80193453872268\\
-2.36584302602081	-2.7969747305353\\
-2.3831218493064	-2.79130767928843\\
-2.39884459497255	-2.78501949729019\\
-2.41326487171072	-2.77816032594382\\
-2.42658635807579	-2.77075218758484\\
-2.43897375075651	-2.76279396009856\\
-2.45056074742211	-2.75426426916161\\
-2.46145584307613	-2.74512277687487\\
-2.47174645662899	-2.73531012829537\\
-2.48150171724134	-2.72474665468707\\
-2.49077409655662	-2.71332979434373\\
-2.49959995033285	-2.70093005551038\\
-2.50799891171573	-2.68738518956101\\
-2.51597193812646	-2.67249204113197\\
-2.52349762850909	-2.65599526143237\\
-2.53052615923393	-2.63757166082288\\
-2.536969773856	-2.61680835723503\\
-2.54268810059327	-2.59317191853194\\
-2.54746548162895	-2.56596418469749\\
-2.55097565952651	-2.53425802893047\\
-2.55272599747514	-2.49680236516834\\
-2.55196784675265	-2.45187920609796\\
-2.54754975556874	-2.39708483508081\\
-2.53767237367441	-2.32898961584025\\
-2.51947199549834	-2.24260356335451\\
-2.48830469404539	-2.13053737931832\\
-2.43651919657954	-1.98171946016395\\
-2.35142882017189	-1.77960281834186\\
-2.21235184848992	-1.50032235638658\\
-1.98788615858815	-1.11322470622252\\
-1.63913213049626	-0.590916485747844\\
-1.14175486096006	0.0611110024336226\\
-0.528318729159554	0.768681689213856\\
0.101619538049815	1.40957805269479\\
0.648126500738803	1.89991988974157\\
1.07179145063932	2.23433020905319\\
1.38341441913172	2.44974622718435\\
1.61005221608102	2.58601032593121\\
1.77663186292935	2.67225729403111\\
1.90153812700941	2.72716117174515\\
1.99736313946378	2.76219713392125\\
2.07256887160442	2.78438736883833\\
2.13286638976315	2.79808153592387\\
2.18216585835773	2.8060240588645\\
2.22319384147655	2.80998144231508\\
2.25788870292592	2.81111255677442\\
};
\addplot [color=mycolor1, forget plot]
  table[row sep=crcr]{%
2.23394855765595	2.81731297801188\\
2.26165542881003	2.81645433828003\\
2.28578522134835	2.81416644391003\\
2.30704958441646	2.81077666426061\\
2.32599148360855	2.80650857474673\\
2.34303114263776	2.8015133554433\\
2.35849801089891	2.79589030904853\\
2.37265334861332	2.78970036790849\\
2.38570639666023	2.7829749647209\\
2.39782607668333	2.7757217405338\\
2.40914951273934	2.76792800626899\\
2.4197882375849	2.7595625174909\\
2.42983265885312	2.75057588292467\\
2.43935515927449	2.74089975437909\\
2.44841205590173	2.7304448060995\\
2.45704452161019	2.71909738093224\\
2.46527845854013	2.70671453784107\\
2.47312318888068	2.69311705736275\\
2.48056867109479	2.67807971880391\\
2.48758072771249	2.66131781199672\\
2.49409343444114	2.64246831961024\\
2.49999728651665	2.62106339378108\\
2.50512088308885	2.59649247127797\\
2.50920239819958	2.56794731733937\\
2.51184457318115	2.53434093478312\\
2.51244251282485	2.49418572108672\\
2.51006559670834	2.44540696967048\\
2.50326034293511	2.38505228302863\\
2.48971464771938	2.30883202088359\\
2.46567636093457	2.21038701377491\\
2.42493925477907	2.08013241442987\\
2.35710180056824	1.90352032711532\\
2.24478420350019	1.65882250405807\\
2.06013938703502	1.31581788713266\\
1.7641832420183	0.840776377405985\\
1.32043053375636	0.219467138432777\\
0.73538690017579	-0.499760516135056\\
0.0930291108947909	-1.1957590007653\\
-0.491215767880153	-1.75424119406673\\
-0.953776314805486	-2.14377244520633\\
-1.29459482026794	-2.39573992748687\\
-1.54045239114532	-2.55441710429746\\
-1.71905444770675	-2.65421650753045\\
-1.85135954830114	-2.71746709182542\\
-1.95172065990269	-2.75782369546833\\
-2.02969800390966	-2.78355326332276\\
-2.09166991701261	-2.79971502556886\\
-2.14195212072707	-2.80946440618582\\
-2.1835205310336	-2.81481166388316\\
-2.21846928352491	-2.81706417099303\\
-2.24830184335604	-2.81708800387865\\
-2.27411836743984	-2.81546568889455\\
-2.29673831343415	-2.81259331688718\\
-2.31678213074571	-2.8087415448277\\
-2.33472667692994	-2.80409461954302\\
-2.35094346387928	-2.79877572531461\\
-2.36572547954666	-2.79286362401391\\
-2.37930626916953	-2.78640361375265\\
-2.39187367454519	-2.77941467487347\\
-2.4035798143547	-2.77189396598476\\
-2.41454836077782	-2.76381938913056\\
-2.42487981794438	-2.7551506532804\\
-2.43465526858326	-2.74582906428706\\
-2.44393888363475	-2.73577611664615\\
-2.45277935678078	-2.72489082997507\\
-2.46121031049194	-2.71304563903389\\
-2.46924960371401	-2.70008048887152\\
-2.47689733375108	-2.68579458068648\\
-2.48413214009023	-2.66993492373347\\
-2.49090514657346	-2.65218042025219\\
-2.49713045655451	-2.63211955817063\\
-2.50267043458998	-2.60921876941312\\
-2.50731287652463	-2.58277689405653\\
-2.51073524368639	-2.55185857118274\\
-2.51244778648669	-2.51519506832468\\
-2.51170143795885	-2.4710338852086\\
-2.50733563119917	-2.41690644816279\\
-2.49752163633762	-2.34926324313888\\
-2.47932146413312	-2.26289383276216\\
-2.44791996027284	-2.15000411186928\\
-2.39529035629046	-1.99878420638165\\
-2.30795966087572	-1.79137743947912\\
-2.1637256527534	-1.50178070923687\\
-1.9287515730817	-1.09660022170732\\
-1.56201415832366	-0.547363379121762\\
-1.04210332896129	0.134299496204066\\
-0.41373736915889	0.85930176547767\\
0.21212340676912	1.49627049229307\\
0.73880059729606	1.96895776582428\\
1.1379412125118	2.28407109237519\\
1.42753200855995	2.48427879822855\\
1.63667724327522	2.6100323100656\\
1.78996263355016	2.68939756091493\\
1.90484457399082	2.73989491206942\\
1.99304948264513	2.77214413487499\\
2.06237662154612	2.79259916061378\\
2.11806121525891	2.80524509917386\\
2.16367804052527	2.81259382941613\\
2.2017171408998	2.81626251283113\\
2.23394855765595	2.81731297801188\\
};
\addplot [color=mycolor1, forget plot]
  table[row sep=crcr]{%
2.20773459083003	2.8222495289209\\
2.23348473718759	2.82145125522293\\
2.2559548318002	2.8193204935936\\
2.27579468679104	2.81615759288709\\
2.29350076287215	2.81216778393027\\
2.30945784242059	2.80748975993564\\
2.32396803739161	2.80221436187556\\
2.33727129498891	2.79639687695855\\
2.34956008794141	2.79006510231986\\
2.36099004991404	2.78322450845005\\
2.37168772374717	2.77586133018681\\
2.38175620219995	2.7679440877519\\
2.3912791795804	2.75942382039327\\
2.40032374934845	2.75023315421115\\
2.40894214567091	2.74028419309749\\
2.41717251349424	2.72946509451973\\
2.42503868370206	2.71763504945624\\
2.43254880918091	2.70461720498769\\
2.43969256182161	2.6901888180198\\
2.44643636739362	2.6740675635357\\
2.45271581340241	2.65589236827855\\
2.4584238176108	2.63519628236681\\
2.4633922399022	2.61136753834277\\
2.4673630846604	2.5835927402505\\
2.46994277399634	2.550772487407\\
2.4705282381474	2.51139364570898\\
2.46818499718316	2.46333216102615\\
2.46144163796091	2.40354278661851\\
2.44793587476921	2.32756282820322\\
2.42379395539544	2.22871109864616\\
2.38253334906901	2.09680478313707\\
2.31315170340179	1.91620172798163\\
2.19703815751015	1.66327247075068\\
2.00413044856463	1.30496254027981\\
1.69266803865093	0.805057180161984\\
1.22639854495036	0.152164411992385\\
0.621530473963147	-0.591629130270673\\
-0.0230689005423035	-1.29031100961773\\
-0.590042352737044	-1.83246425399385\\
-1.02707637532975	-2.20058405277435\\
-1.34373007916348	-2.43471755813798\\
-1.57016562545039	-2.58086921940254\\
-1.73405441270537	-2.67244926556035\\
-1.85536224159972	-2.73044230004995\\
-1.94744880260296	-2.76747089490889\\
-2.0191063409059	-2.79111442363814\\
-2.07616448903187	-2.80599406743089\\
-2.12255587999007	-2.81498850965447\\
-2.16098911599277	-2.81993202932399\\
-2.19337006048283	-2.82201867867359\\
-2.22106776878537	-2.82204050841864\\
-2.24508498412103	-2.82053101044253\\
-2.26616947929479	-2.8178533986692\\
-2.28488817049593	-2.81425608103213\\
-2.30167738166643	-2.80990816867924\\
-2.3168775422682	-2.80492255517029\\
-2.33075753221499	-2.79937107088254\\
-2.34353201168744	-2.79329445550569\\
-2.3553739069556	-2.78670884188176\\
-2.3664234841957	-2.77960980324398\\
-2.37679496514577	-2.77197461198927\\
-2.38658132134539	-2.76376309272819\\
-2.39585766622337	-2.7549172664856\\
-2.40468350734871	-2.74535983930341\\
-2.41310399834668	-2.73499146118265\\
-2.42115022146144	-2.72368654912106\\
-2.42883841979367	-2.71128730962606\\
-2.4361679632172	-2.69759538525143\\
-2.44311764748936	-2.6823602491983\\
-2.44963965175959	-2.6652630245746\\
-2.45565004906105	-2.64589371824984\\
-2.46101406296593	-2.62371878033168\\
-2.46552308833301	-2.59803416998312\\
-2.46885847617421	-2.56789628085639\\
-2.47053353886647	-2.53201838011155\\
-2.46979887629439	-2.48861229487933\\
-2.46548451503161	-2.4351416266843\\
-2.4557308880001	-2.367930037746\\
-2.43752103026658	-2.28153102169332\\
-2.40585540154001	-2.16771136528395\\
-2.35229751005661	-2.01384958783898\\
-2.26250565626076	-1.80063144759285\\
-2.11258557093622	-1.49966066179349\\
-1.86602728037014	-1.07454491114853\\
-1.4797877932246	-0.496095355052655\\
-0.936805752189655	0.215941648042236\\
-0.295978143629587	0.955565328864142\\
0.321438639117415	1.5841689940824\\
0.825136322831452	2.0363613926554\\
1.19865633278655	2.33130040158839\\
1.46632323755697	2.51636869632571\\
1.65850147464714	2.63192520700021\\
1.79906607079213	2.70470478186873\\
1.90442332523672	2.75101495342029\\
1.98541221796483	2.78062512601975\\
2.0491795001313	2.79943900899214\\
2.10050143774229	2.81109359900658\\
2.14263303395119	2.8178803855583\\
2.17784033336939	2.82127556070905\\
2.20773459083003	2.8222495289209\\
};
\addplot [color=mycolor1, forget plot]
  table[row sep=crcr]{%
2.17877089191157	2.8258491631413\\
2.20266195475253	2.82510825619919\\
2.22355303970033	2.82312699966302\\
2.24203576130143	2.82018025751118\\
2.25856281479918	2.8164559480498\\
2.27348562190955	2.81208097374269\\
2.28708055244381	2.80713818397963\\
2.29956747509971	2.80167753770813\\
2.31112305815506	2.79572340861465\\
2.32189040561136	2.78927923667156\\
2.33198608000072	2.78233027086249\\
2.34150521308389	2.77484485181766\\
2.35052516945291	2.76677448124427\\
2.35910806153742	2.75805277512825\\
2.36730228879221	2.74859327147649\\
2.37514316822329	2.73828593921934\\
2.38265262068332	2.72699209253963\\
2.38983775991052	2.7145372308512\\
2.39668807680493	2.70070106681206\\
2.4031706870237	2.68520362462547\\
2.40922276235853	2.66768571092972\\
2.41473970458509	2.6476811525825\\
2.41955668385167	2.62457674239169\\
2.42341956070791	2.59755346032928\\
2.42593840079165	2.56549858692832\\
2.42651175108008	2.52687163564035\\
2.42420061301231	2.47949555154823\\
2.41751383304105	2.42022482642527\\
2.40403424708072	2.34440848059438\\
2.37975463609059	2.24501166999273\\
2.33788680348285	2.11118778930979\\
2.26675737864986	1.92606679741267\\
2.14637058166322	1.66387081861997\\
1.94418590217523	1.28837621186511\\
1.61553544103642	0.760902998569861\\
1.12539644326509	0.0745020623084606\\
0.502231119522599	-0.692028792653104\\
-0.139898794240548	-1.3883092500649\\
-0.685264967956727	-1.90997587733958\\
-1.09481076588507	-2.25501661218449\\
-1.38703683868072	-2.47111344877806\\
-1.59446713793099	-2.60500512550983\\
-1.74420428353216	-2.68867837342763\\
-1.85503000691998	-2.74165972225064\\
-1.93926554994634	-2.77553050794186\\
-2.00493893990024	-2.79719875117125\\
-2.0573470000794	-2.81086507442919\\
-2.10005537972538	-2.81914490964075\\
-2.13551861595477	-2.82370597694059\\
-2.16546446419626	-2.82563535550257\\
-2.19113517288083	-2.82565529704955\\
-2.21344172129508	-2.82425307049057\\
-2.23306440504989	-2.821760889237\\
-2.2505197612451	-2.81840617041766\\
-2.26620596485001	-2.81434373682659\\
-2.28043418264582	-2.80967676338909\\
-2.29345058827861	-2.80447053516409\\
-2.30545204646153	-2.79876149299468\\
-2.31659742199454	-2.79256309496412\\
-2.32701580304022	-2.78586944178028\\
-2.33681249704063	-2.77865724760119\\
-2.3460733713891	-2.770886495416\\
-2.35486791396654	-2.76249994441075\\
-2.36325124550464	-2.75342152166876\\
-2.37126520234114	-2.74355350788806\\
-2.37893850604596	-2.7327722961707\\
-2.38628592868827	-2.72092234306972\\
-2.39330622982574	-2.70780771495221\\
-2.39997845694397	-2.69318032125385\\
-2.40625592354149	-2.67672345836231\\
-2.41205673908975	-2.65802856431732\\
-2.4172490422457	-2.63656193879927\\
-2.42162786717519	-2.61161633019505\\
-2.42487845667244	-2.58223923626837\\
-2.42651708281877	-2.54712463397145\\
-2.42579363098841	-2.50444610188262\\
-2.42152761816296	-2.45159422103287\\
-2.41182571193496	-2.38475522131539\\
-2.39358448774803	-2.29822459719431\\
-2.36160130059077	-2.18328435227048\\
-2.30698424867367	-2.02640706535026\\
-2.21441724776119	-1.80663567049949\\
-2.05810232219544	-1.49287274945234\\
-1.79859143345776	-1.0454639060061\\
-1.39108370550384	-0.435140869323708\\
-0.82480235487054	0.307613699130504\\
-0.174921739515048	1.05796528414176\\
0.429057793926385	1.67312163418713\\
0.906601218695649	2.10195276089735\\
1.25353586858761	2.37594377011342\\
1.4994538511774	2.5459881258258\\
1.67519385396721	2.65166344876194\\
1.80358749812364	2.71814129356404\\
1.89989066764397	2.7604709173079\\
1.97404145527801	2.78758017016175\\
2.03254641694308	2.8048407202027\\
2.0797395803064	2.81555711454915\\
2.11857098081715	2.82181180533255\\
2.15109432677297	2.82494776723096\\
2.17877089191157	2.8258491631413\\
};
\addplot [color=mycolor1, forget plot]
  table[row sep=crcr]{%
2.14638286171033	2.82796423666817\\
2.1685094971813	2.82727778496317\\
2.18790043752099	2.82543857464301\\
2.20509246287125	2.82269741653787\\
2.22049703837008	2.81922588001395\\
2.23443415506741	2.81513972630638\\
2.24715592449306	2.81051424969982\\
2.25886328880231	2.8053943710103\\
2.2697180143095	2.79980122861067\\
2.27985138828809	2.79373634775355\\
2.28937056027584	2.78718405488052\\
2.29836315520485	2.78011253502649\\
2.30690057317799	2.77247374551488\\
2.315040240036	2.76420225953563\\
2.32282695786662	2.75521299287993\\
2.33029340631105	2.74539764554835\\
2.33745974772071	2.73461954720598\\
2.34433217475774	2.72270640758823\\
2.3509000855707	2.70944020643105\\
2.3571313455368	2.6945430608253\\
2.36296474045027	2.6776572978299\\
2.36829814852421	2.6583169976477\\
2.37296998737321	2.63590671969835\\
2.37672981520411	2.60960056517011\\
2.37919099741342	2.57827042825493\\
2.37975296956227	2.54034492229864\\
2.37747065233213	2.4935876616492\\
2.37082972145653	2.43474115356513\\
2.35735041420007	2.35894366511076\\
2.33287430374968	2.25876240221542\\
2.29026634313273	2.12259835813708\\
2.21708805566767	1.93217933656367\\
2.09176081156176	1.65926844659496\\
1.87895213931415	1.26409107160541\\
1.53103973962933	0.705711375456089\\
1.01576355532262	-0.0160168121401194\\
0.376896810506748	-0.802150579667117\\
-0.256940380314764	-1.48973187510391\\
-0.776081907587097	-1.98647819331637\\
-1.15629517575985	-2.30686986686515\\
-1.423932858916	-2.50480321137639\\
-1.61280149684316	-2.6267181121602\\
-1.74893632450621	-2.70279042246697\\
-1.84977142266498	-2.75099469191492\\
-1.9265562643376	-2.78186855061526\\
-1.98656202171616	-2.80166594608952\\
-2.03456899024968	-2.81418391236825\\
-2.07379132074096	-2.82178736325071\\
-2.10644193588614	-2.82598625428656\\
-2.13407993566315	-2.82776659093309\\
-2.1578277979326	-2.82778475114576\\
-2.17850995701177	-2.82648439457411\\
-2.19674311789018	-2.8241684818417\\
-2.21299635522654	-2.82104461336558\\
-2.22763190433368	-2.81725411473201\\
-2.24093335951189	-2.81289097509428\\
-2.25312549401949	-2.80801429077649\\
-2.26438839561164	-2.80265643842267\\
-2.27486766896483	-2.79682835032\\
-2.28468185966512	-2.79052274220007\\
-2.29392786811972	-2.78371581231931\\
-2.3026848646701	-2.7763677098011\\
-2.31101703952361	-2.76842191179093\\
-2.31897539098827	-2.75980352179609\\
-2.32659865096067	-2.75041638306813\\
-2.33391335056484	-2.74013877136725\\
-2.3409329250638	-2.72881726957966\\
-2.34765562652752	-2.71625820452968\\
-2.35406082818988	-2.70221570273356\\
-2.36010302314961	-2.68637493163734\\
-2.36570236978041	-2.66832832882923\\
-2.37072988989644	-2.64754140238456\\
-2.37498415356	-2.62330269720879\\
-2.37815406021903	-2.59464921277142\\
-2.37975834147586	-2.56025294094827\\
-2.37904510441422	-2.51824449556296\\
-2.37482104878933	-2.46593286163663\\
-2.36515394587688	-2.39935067236393\\
-2.34684223351532	-2.31250396918446\\
-2.31445313585167	-2.19612735400624\\
-2.2585767975166	-2.03566252518764\\
-2.16278387532862	-1.80827193345053\\
-1.99910792181745	-1.47978249385533\\
-1.72488195578284	-1.00704113098572\\
-1.29408696393288	-0.361795492332068\\
-0.704865784088944	0.41126327946243\\
-0.0506195584320581	1.16697280720031\\
0.534227949968228	1.76287629141842\\
0.982439525342405	2.16547356894941\\
1.3019570069277	2.41784782802175\\
1.52636237507873	2.57302658281129\\
1.68619549819523	2.66913835746718\\
1.80294774058144	2.72958802234225\\
1.890643171054	2.76813310488603\\
1.95831241293265	2.79287176878262\\
2.01183572375014	2.80866181552588\\
2.05512133356105	2.81849028502511\\
2.09082835129564	2.82424123560066\\
2.12080904146535	2.82713164121354\\
2.14638286171033	2.82796423666817\\
};
\addplot [color=mycolor1, forget plot]
  table[row sep=crcr]{%
2.10959639276541	2.82833170033868\\
2.13005277719103	2.82769680200034\\
2.14802287471214	2.82599213309054\\
2.16399176571052	2.82344580125796\\
2.17833205481083	2.82021393645668\\
2.19133411253028	2.81640177024618\\
2.20322718958303	2.81207744870569\\
2.21419439157868	2.80728111824648\\
2.22438344129317	2.80203084312009\\
2.23391449132048	2.79632631929519\\
2.24288582434271	2.79015097768998\\
2.25137799857592	2.78347282702145\\
2.25945680590028	2.7762442174586\\
2.26717527446035	2.7684005761966\\
2.27457484262639	2.75985805103207\\
2.28168573976336	2.75050987848557\\
2.28852651598563	2.7402211492275\\
2.29510255122665	2.72882145096567\\
2.30140322103382	2.71609459233055\\
2.30739716790373	2.70176419574736\\
2.31302476509234	2.68547330303474\\
2.31818626415068	2.66675511304507\\
2.32272310676726	2.64499030425581\\
2.32638812037928	2.61934362489206\\
2.32879716902256	2.58866772920285\\
2.32934906157394	2.5513540908222\\
2.32708969033154	2.5050964728495\\
2.32047561983038	2.44650691445407\\
2.3069520645374	2.37047913665329\\
2.28218252151097	2.26911845209114\\
2.23862573128952	2.12995002321369\\
2.16294815306376	1.93306507749366\\
2.03172785596982	1.64737025756866\\
1.80647115907523	1.22912474961823\\
1.43674425455007	0.635722130761753\\
0.895415675386827	-0.122678493566976\\
0.24506740533721	-0.923290250180105\\
-0.373302970705661	-1.59439810765956\\
-0.861342602045722	-2.06153432057587\\
-1.21051470427877	-2.35582021146555\\
-1.45350302607409	-2.53553747281427\\
-1.62427912108232	-2.6457756283534\\
-1.74735161713476	-2.71454779213463\\
-1.83866892421111	-2.75820072395713\\
-1.90838558633582	-2.78623139232265\\
-1.96302614690974	-2.80425772800326\\
-2.00687082453237	-2.81568963402831\\
-2.04279713811814	-2.82265357129101\\
-2.07278812116456	-2.82650998992637\\
-2.09824304490626	-2.82814934742327\\
-2.12017116470066	-2.82816582669431\\
-2.13931522377064	-2.826961933092\\
-2.15623195525253	-2.82481302044249\\
-2.17134569327871	-2.82190798040682\\
-2.18498480181322	-2.81837538766676\\
-2.19740689106534	-2.81430054547294\\
-2.20881656655742	-2.80973669054728\\
-2.21937810527049	-2.8047123437088\\
-2.22922461613086	-2.79923603208632\\
-2.23846471192486	-2.79329914096155\\
-2.24718737595216	-2.78687735484354\\
-2.25546547731453	-2.7799309467999\\
-2.26335822936359	-2.77240402898462\\
-2.27091276797016	-2.76422275734835\\
-2.27816493006503	-2.75529236858525\\
-2.28513922246038	-2.74549279836288\\
-2.2918478707394	-2.73467246525985\\
-2.2982887089987	-2.72263957559321\\
-2.30444148590314	-2.70914996681804\\
-2.31026187654082	-2.69388999154343\\
-2.31567202720427	-2.67645213420482\\
-2.32054568700573	-2.65629974947181\\
-2.32468465040066	-2.63271516862803\\
-2.32778088753628	-2.60472181958332\\
-2.32935449079792	-2.57096482990594\\
-2.32864968427513	-2.52952378807117\\
-2.32445618534273	-2.47761220956851\\
-2.31479431693115	-2.41108425761014\\
-2.29634617602986	-2.32361020748856\\
-2.26340897464661	-2.20528850710705\\
-2.20596668980778	-2.04035887822781\\
-2.10628907021382	-1.80379023894369\\
-1.93390602004408	-1.45787754842354\\
-1.64266626013909	-0.955835256982624\\
-1.18636447694275	-0.272301910216555\\
-0.575627521932152	0.529268583185396\\
0.0765700150810036	1.28296729173816\\
0.635827895370091	1.85302389135356\\
1.05156023208183	2.22653834661497\\
1.34296588926262	2.45673552212428\\
1.54615322639242	2.59724775265813\\
1.69061445079945	2.68411603245696\\
1.79623883340552	2.73880286760147\\
1.87575398878987	2.77375114085251\\
1.93728218443528	2.79624367598011\\
1.98609242944908	2.81064247686136\\
2.02568323586637	2.81963136032678\\
2.05843611040654	2.82490602040426\\
2.08601207953629	2.8275641952934\\
2.10959639276541	2.82833170033868\\
};
\addplot [color=mycolor1, forget plot]
  table[row sep=crcr]{%
2.0669689142298	2.8265001166603\\
2.08585218487773	2.82591377173339\\
2.10248435713509	2.824335791098\\
2.11730176510645	2.82197287100359\\
2.13064039591691	2.81896657267359\\
2.14276272243324	2.81541218008658\\
2.15387647302746	2.8113710713481\\
2.16414796811114	2.80687885374025\\
2.17371172139776	2.80195064490338\\
2.18267741955627	2.79658435510239\\
2.19113501896617	2.79076249383181\\
2.19915845114387	2.78445280557154\\
2.2068082596289	2.77760788518744\\
2.21413336951179	2.77016380211266\\
2.22117209536673	2.76203765196474\\
2.227952408255	2.75312383596871\\
2.23449139325266	2.74328872273134\\
2.24079371916766	2.73236314815236\\
2.24684878895986	2.72013192028773\\
2.25262600707546	2.70631905782578\\
2.25806722804191	2.69056680598815\\
2.26307483266911	2.67240537597125\\
2.26749281998738	2.65120855342423\\
2.27107644143286	2.62612730046412\\
2.27344254173279	2.59598829347994\\
2.27398653889816	2.55913525875812\\
2.27174012871236	2.51317476307575\\
2.26512077120442	2.45455886598506\\
2.25147861067759	2.37788450758131\\
2.22625743325768	2.27469939362629\\
2.18142130782272	2.13147407201847\\
2.10255582239114	1.92633668518003\\
1.96404509251501	1.62482097010485\\
1.72381856839937	1.17882240111403\\
1.329183818134	0.545395040165709\\
0.761790068907563	-0.249787526505068\\
0.106628164730735	-1.05674278201958\\
-0.487505353795216	-1.70185391858681\\
-0.939357911409621	-2.13448330265905\\
-1.25594103658005	-2.40134149815461\\
-1.47431937685956	-2.56286478580134\\
-1.62749848260503	-2.66174359928501\\
-1.73804377195806	-2.72351381077397\\
-1.82030363294048	-2.76283520906365\\
-1.8833232358948	-2.78817183992138\\
-1.93289296053334	-2.80452423009049\\
-1.97280938807871	-2.81493111993595\\
-2.00562765622305	-2.82129200404624\\
-2.03311197406603	-2.82482564197129\\
-2.05651006259846	-2.82633216974524\\
-2.07672408698264	-2.82634706367953\\
-2.09441965051151	-2.82523401181922\\
-2.11009691749222	-2.82324233404957\\
-2.12413805178504	-2.82054327423604\\
-2.13683950901854	-2.81725336930394\\
-2.1484344317427	-2.81344970908028\\
-2.15910844240422	-2.80917997274218\\
-2.16901094268674	-2.80446900297834\\
-2.178263291945	-2.79932300498311\\
-2.18696477081792	-2.79373204110065\\
-2.19519693282776	-2.78767122462506\\
-2.20302674361282	-2.78110083463718\\
-2.21050876536302	-2.77396543908935\\
-2.2176865377223	-2.76619199986685\\
-2.22459321797863	-2.75768682147306\\
-2.23125145805402	-2.74833107563165\\
-2.23767239863978	-2.73797446554853\\
-2.24385353281524	-2.72642635543562\\
-2.24977500441801	-2.71344333660651\\
-2.25539361411527	-2.69871165550837\\
-2.26063332867985	-2.68182206458618\\
-2.26537028309853	-2.66223325496721\\
-2.26940886619963	-2.63921770416699\\
-2.27244298687703	-2.61177982613277\\
-2.27399205537368	-2.57852946857343\\
-2.27329264767083	-2.53748169639891\\
-2.26911034445943	-2.4857320441213\\
-2.25940393141311	-2.41891713013389\\
-2.24071038667139	-2.33030101858926\\
-2.20699631199343	-2.20921528422324\\
-2.1475106522554	-2.03845428181591\\
-2.04296071306462	-1.79036922549699\\
-1.85994625905196	-1.42317525718662\\
-1.5486658362723	-0.886598994775004\\
-1.06463898401359	-0.161388990802826\\
-0.435691160374269	0.6644529412044\\
0.205745366158514	1.40610510429633\\
0.732167458520347	1.9429031736525\\
1.11235206856804	2.28455348115237\\
1.37509630217415	2.492128076491\\
1.55741791816583	2.61821320681005\\
1.68704870813498	2.69616214335981\\
1.78204832273847	2.74534615469033\\
1.85379818103074	2.77687987495302\\
1.90951601160325	2.79724714593898\\
1.95387541865583	2.81033207185613\\
1.98998084771081	2.81852894203929\\
2.01994885409201	2.82335459591496\\
2.04525882008697	2.8257939325355\\
2.0669689142298	2.8265001166603\\
};
\addplot [color=mycolor1, forget plot]
  table[row sep=crcr]{%
2.01629649870323	2.8216947796992\\
2.03371192787167	2.82115372757937\\
2.04909758148215	2.8196937697169\\
2.06284386947368	2.81750145142994\\
2.07525232111241	2.81470461391784\\
2.08655918486678	2.81138915448681\\
2.09695197926938	2.80761003738521\\
2.10658128152195	2.80339853083658\\
2.11556923308719	2.79876688561593\\
2.12401573403612	2.79371120614847\\
2.13200297131505	2.78821297113196\\
2.13959870991857	2.78223946498994\\
2.14685862733202	2.77574324077276\\
2.15382786330377	2.76866062152778\\
2.16054187042183	2.76090914028469\\
2.16702657163593	2.75238370062266\\
2.17329774507153	2.74295109035814\\
2.17935944791232	2.73244227361185\\
2.1852011364565	2.72064158145531\\
2.1907929012884	2.7072714544067\\
2.19607785082874	2.69197065410794\\
2.20096002912548	2.67426267219099\\
2.20528513440281	2.65350909646456\\
2.20880931586443	2.62883936170034\\
2.21114769507717	2.59904253518429\\
2.2116874471933	2.56239654634731\\
2.20943715928125	2.51639175507511\\
2.20275831617271	2.45727185066445\\
2.188872968427	2.37925324006683\\
2.1629384810545	2.27317642871619\\
2.11628700004405	2.1241865211064\\
2.03314705487453	1.90797639336758\\
1.88522789170033	1.58603353217718\\
1.62647712510769	1.10567006058556\\
1.20337195083471	0.426430541886378\\
0.611915909614419	-0.402858236553646\\
-0.0377639343599761	-1.20354789310859\\
-0.597100482331314	-1.81117113185058\\
-1.00757427958943	-2.20428707001593\\
-1.29021696967227	-2.44256084179465\\
-1.48412953405143	-2.58599032846721\\
-1.62023730138053	-2.67384690614635\\
-1.71879227503168	-2.72891443587442\\
-1.79245059734471	-2.76412199699261\\
-1.84914108871065	-2.78691245609935\\
-1.89393410940838	-2.80168792222779\\
-1.93015885351913	-2.81113148167195\\
-1.96006170027521	-2.81692666023866\\
-1.98519848830143	-2.82015798991299\\
-2.00667318546355	-2.82154029086012\\
-2.02528665005883	-2.82155369216577\\
-2.0416316546784	-2.82052532925747\\
-2.05615503547555	-2.81868002036959\\
-2.06919925101121	-2.81617239719858\\
-2.08103074658255	-2.81310765133649\\
-2.09185967675809	-2.80955510860833\\
-2.10185384863354	-2.80555716250382\\
-2.11114872107	-2.80113511420794\\
-2.11985465684564	-2.79629287490345\\
-2.12806221881559	-2.79101911843369\\
-2.13584603650612	-2.78528823438026\\
-2.14326759134354	-2.77906026763665\\
-2.15037714299369	-2.77227990629132\\
-2.15721492382378	-2.76487447169454\\
-2.16381164722527	-2.75675075432964\\
-2.17018829476462	-2.74779040808994\\
-2.17635505229918	-2.73784344117483\\
-2.18230913715818	-2.72671909182738\\
-2.1880310680229	-2.7141730012679\\
-2.19347862762006	-2.69988901191023\\
-2.19857727062877	-2.68345298618764\\
-2.20320488089757	-2.66431451591879\\
-2.20716729498335	-2.64172983860735\\
-2.21015833063649	-2.61467490171091\\
-2.2116931024764	-2.58170984489889\\
-2.21099398493157	-2.54076243038014\\
-2.20679020984049	-2.48877291954632\\
-2.19695552159172	-2.42109700701258\\
-2.17783500661387	-2.33048097653156\\
-2.142969037832	-2.20528818446957\\
-2.08067540438981	-2.02650626200969\\
-1.96972237526817	-1.76327837650438\\
-1.77324535376412	-1.36912188000483\\
-1.43794190280115	-0.791111827159751\\
-0.924545711399306	-0.0216540259654149\\
-0.283953095218948	0.819957185386306\\
0.335043173914783	1.53607343206824\\
0.820645184263823	2.03143236592395\\
1.16236468113286	2.33857029213583\\
1.39605637439492	2.52320245266567\\
1.55792497526828	2.63514208823294\\
1.67327854150371	2.70450306114709\\
1.75815357520141	2.74844279268488\\
1.82254887470734	2.77674235019102\\
1.87278527967306	2.79510456930404\\
1.91295729365938	2.80695333270049\\
1.94579044293073	2.81440658009161\\
1.97314833573224	2.81881138097168\\
1.99633775367937	2.82104590983873\\
2.01629649870323	2.8216947796992\\
};
\addplot [color=mycolor1, forget plot]
  table[row sep=crcr]{%
1.95408713286161	2.81255877444144\\
1.97015665108589	2.81205922536446\\
1.98440383011239	2.81070703591457\\
1.99717590274892	2.80866986239339\\
2.00874206890337	2.80606267207798\\
2.01931402447436	2.80296252225127\\
2.02906040923129	2.79941828627224\\
2.03811713116402	2.79545704113902\\
2.04659483732625	2.79108817309935\\
2.05458436756647	2.78630585336903\\
2.06216074721035	2.78109027757561\\
2.06938608809292	2.77540788802964\\
2.07631163782244	2.76921066958993\\
2.08297912134429	2.76243450284066\\
2.08942144040516	2.75499645363844\\
2.09566272220139	2.74679075818894\\
2.10171762517856	2.73768310713825\\
2.10758970122228	2.72750261231144\\
2.11326845507579	2.71603051211328\\
2.1187244940024	2.70298416525528\\
2.12390175446257	2.68799407744965\\
2.12870510331387	2.67057039352651\\
2.13298040690076	2.65005309663722\\
2.13648199862587	2.62553640960846\\
2.13881848174384	2.5957513300483\\
2.13936021939874	2.55887846075589\\
2.13707705910718	2.51224175080742\\
2.13024522821122	2.4517937944577\\
2.11590219113997	2.37122954964349\\
2.08880702511242	2.26043600118698\\
2.03943638612533	2.10280117567733\\
1.95023384582017	1.87087470725969\\
1.78957421552096	1.52125205780096\\
1.50724163123535	0.997097348433753\\
1.05218124740212	0.266322314398444\\
0.442891559848105	-0.58852021357033\\
-0.185833290433792	-1.36392590129485\\
-0.698009436945026	-1.92057888359518\\
-1.06198920354467	-2.26924086719763\\
-1.30958905743568	-2.47798367576543\\
-1.4792957117902	-2.60350567562873\\
-1.59889578491176	-2.68070188805705\\
-1.68600931275967	-2.72937275373668\\
-1.75152908476803	-2.76068754710666\\
-1.80226698636391	-2.78108311856666\\
-1.84258863188146	-2.79438234601673\\
-1.87537167226418	-2.80292773170737\\
-1.90256666928812	-2.80819742761536\\
-1.92553095322346	-2.81114894635193\\
-1.94523210358837	-2.81241666419805\\
-1.96237516540708	-2.81242866498418\\
-1.97748420843269	-2.81147777919341\\
-1.99095587711757	-2.80976585505906\\
-2.00309533870785	-2.80743194832296\\
-2.01414091135235	-2.80457058999968\\
-2.02428125088093	-2.80124376919405\\
-2.03366754272059	-2.79748882437222\\
-2.0424222723539	-2.79332358755083\\
-2.05064560332622	-2.78874961214502\\
-2.05842004429573	-2.78375399369517\\
-2.06581385882923	-2.77831008210545\\
-2.07288351712786	-2.77237723631783\\
-2.07967537858105	-2.76589965726303\\
-2.08622670867571	-2.75880423126099\\
-2.09256605897104	-2.75099720604875\\
-2.09871296179467	-2.7423593867516\\
-2.10467679770348	-2.73273935563581\\
-2.1104545639566	-2.72194395253888\\
-2.11602707550614	-2.70972484716044\\
-2.12135281425098	-2.69575939801092\\
-2.12635811517624	-2.67962296813396\\
-2.1309214701633	-2.66074817727884\\
-2.13484812202904	-2.63836371385647\\
-2.13782819258103	-2.61140038594983\\
-2.13936610387029	-2.57834332604201\\
-2.13865849025632	-2.53699337538637\\
-2.13437691750635	-2.48407135459853\\
-2.12426956517293	-2.41454448693609\\
-2.10441034382447	-2.32045540155919\\
-2.06775511912624	-2.18887241749459\\
-2.00137768373468	-1.99841461647744\\
-1.8815491498074	-1.71418308494054\\
-1.6673253143189	-1.28445643542233\\
-1.30292284263472	-0.656185427944006\\
-0.760561802985589	0.157087992584297\\
-0.120400018076629	0.99873582485722\\
0.46087903608505	1.67162366241341\\
0.897137918336744	2.11679559453497\\
1.19773552170667	2.38700539604759\\
1.40216485547805	2.54851917659725\\
1.54406119305533	2.64664238340825\\
1.64571156583332	2.70775941662276\\
1.72097033471938	2.74671763256643\\
1.77842930535205	2.7719666864416\\
1.82352311167654	2.78844764991827\\
1.85978356502206	2.79914158907219\\
1.88957179553683	2.80590281834557\\
1.91450981240302	2.80991739269252\\
1.93574032729193	2.81196268811317\\
1.95408713286161	2.81255877444144\\
};
\addplot [color=mycolor1, forget plot]
  table[row sep=crcr]{%
1.87457785195785	2.79663504639062\\
1.88945545218441	2.79617219509606\\
1.90270374359799	2.79491450953976\\
1.91462952242024	2.79301206176028\\
1.92547182633201	2.79056780972743\\
1.93541953820537	2.78765050384051\\
1.94462383638467	2.78430319829366\\
1.95320713164883	2.78054883237584\\
1.96126955988095	2.77639378663779\\
1.96889373707553	2.7718299711776\\
1.97614824742985	2.76683577821289\\
1.98309017679418	2.76137607604001\\
1.98976689233669	2.7554013038897\\
1.99621718508488	2.74884562490728\\
2.0024718207702	2.74162398977357\\
2.00855347402061	2.73362783867479\\
2.01447593856784	2.7247190026417\\
2.02024239483717	2.71472112543319\\
2.02584235012124	2.70340756507238\\
2.03124660207401	2.69048416881407\\
2.03639913691442	2.67556440825667\\
2.04120411953491	2.65813286935502\\
2.04550480009076	2.63749057763998\\
2.04904874009366	2.61267129583654\\
2.05142923430248	2.58231024186075\\
2.05198409645007	2.54443273360228\\
2.04961575014777	2.49610446347231\\
2.04246165430275	2.4328367742712\\
2.02727238009869	2.3475506408997\\
1.99820950966423	2.22874757739577\\
1.9445086891535	2.05733205488968\\
1.84614113374679	1.80162784832694\\
1.66730383173018	1.41247595138888\\
1.35437802897999	0.831416506907343\\
0.866004494455484	0.0466867728527042\\
0.253507304421055	-0.813411613844761\\
-0.332010221993016	-1.53608422884368\\
-0.783291433215933	-2.02675635656377\\
-1.0960575116577	-2.32639806983683\\
-1.30784508177712	-2.50494509822122\\
-1.45374421032671	-2.612849065112\\
-1.55745382918991	-2.67978157783237\\
-1.63370370471678	-2.7223779641566\\
-1.6915732725475	-2.75003312998556\\
-1.73676242378907	-2.76819602222803\\
-1.77294782937584	-2.78012947651889\\
-1.80257071204522	-2.78785003090417\\
-1.82729792907952	-2.79264072957431\\
-1.84829752840052	-2.79533912096034\\
-1.8664076462336	-2.79650397780766\\
-1.8822427616688	-2.79651467256384\\
-1.89626224752837	-2.79563203275335\\
-1.90881565562071	-2.79403651912157\\
-1.92017329436437	-2.79185267951468\\
-1.93054729272853	-2.78916507242008\\
-1.94010637508435	-2.78602874578762\\
-1.94898639167708	-2.78247614163633\\
-1.9572979261832	-2.7785215770886\\
-1.9651318482346	-2.7741640131358\\
-1.9725633874208	-2.76938854442878\\
-1.97965511282255	-2.76416685834055\\
-1.98645907017771	-2.75845677846827\\
-1.99301823301222	-2.75220090018766\\
-1.99936734793394	-2.74532422449441\\
-2.00553318496923	-2.73773058396824\\
-2.0115341292134	-2.72929751211378\\
-2.01737895576195	-2.71986900884221\\
-2.02306449523449	-2.70924536160379\\
-2.02857168889946	-2.69716873091932\\
-2.03385919308024	-2.6833024952447\\
-2.03885311904793	-2.66720119032324\\
-2.04343049559734	-2.64826594721389\\
-2.04739225140634	-2.62567703875512\\
-2.05041821524351	-2.5982893830741\\
-2.05199037599498	-2.56446652853883\\
-2.05125843748935	-2.52180971963489\\
-2.04679724268251	-2.46670334973505\\
-2.03615563241697	-2.39353204771595\\
-2.01499378005436	-2.29330477661769\\
-1.97540492246692	-2.15123132676112\\
-1.90269410937579	-1.94265037912345\\
-1.76980293844462	-1.62748139356961\\
-1.53124519946957	-1.14892440398809\\
-1.13209359195833	-0.460484046162954\\
-0.566808533969557	0.387816394586287\\
0.0520366019978877	1.20215165598319\\
0.576535381829118	1.80967253888786\\
0.954864286568924	2.19582800001589\\
1.21211636398751	2.42708312606015\\
1.38729554854803	2.56547899463559\\
1.50978522747304	2.65017391625239\\
1.59834333757868	2.70341329806673\\
1.66452000788624	2.73766611350783\\
1.71548697704509	2.7600597349861\\
1.7558056309989	2.77479364434049\\
1.7884611523011	2.78442313328132\\
1.81546395096188	2.7905512025093\\
1.83820504074484	2.794211411217\\
1.8576710696949	2.7960861786337\\
1.87457785195785	2.79663504639062\\
};
\addplot [color=mycolor1, forget plot]
  table[row sep=crcr]{%
1.76784387513612	2.76928910145383\\
1.78174633759923	2.76885615522374\\
1.79419693541599	2.76767383044685\\
1.80546484911209	2.76587600985613\\
1.81576126131739	2.76355453796707\\
1.82525415931308	2.76077035293493\\
1.83407887909368	2.75756084426591\\
1.84234572307335	2.75394466141325\\
1.85014552584246	2.74992473191324\\
1.85755374913669	2.7454899539701\\
1.8646334942403	2.74061583435693\\
1.87143768860931	2.73526420470525\\
1.87801060916971	2.72938204000661\\
1.88438883113162	2.72289930259853\\
1.89060162587557	2.7157256259312\\
1.89667076319854	2.70774551650771\\
1.90260958893445	2.69881156524049\\
1.90842113042871	2.68873488479607\\
1.91409480045653	2.67727156911894\\
1.91960097508577	2.66410330775859\\
1.92488222337915	2.6488092124617\\
1.92983910128206	2.63082412809549\\
1.9343068731227	2.60937566180876\\
1.93801667340984	2.58338686500575\\
1.94052922309179	2.55132202917289\\
1.94111869405914	2.51093572559874\\
1.93856325046462	2.45885292681666\\
1.93075569678326	2.38984749944438\\
1.91395891517441	2.29557553517638\\
1.88135412086077	2.16233871312354\\
1.82023064403477	1.96727962024497\\
1.70698513939229	1.67293893640523\\
1.5009626367681	1.22458307939976\\
1.14920561952737	0.571075287988628\\
0.634671529376524	-0.256529150340702\\
0.0487615290360517	-1.08023744885568\\
-0.464160373241134	-1.71379205799029\\
-0.840794541274635	-2.12341792386813\\
-1.09857110524551	-2.37037717746715\\
-1.27425441166773	-2.51846834010297\\
-1.39693091863683	-2.60918252109282\\
-1.48545234652581	-2.66630303200044\\
-1.55147603147699	-2.70318039482592\\
-1.60224137064384	-2.72743650225245\\
-1.64234632461511	-2.74355321834049\\
-1.67479425152706	-2.75425225602386\\
-1.70160378481403	-2.76123825526265\\
-1.72416901355081	-2.76560911350238\\
-1.74347710347502	-2.76808940820817\\
-1.76024319539013	-2.76916723159805\\
-1.77499620104009	-2.7691767210339\\
-1.78813472615172	-2.76834914861335\\
-1.79996436786742	-2.76684528428748\\
-1.8107231218077	-2.76477629836321\\
-1.82059902686968	-2.76221746239001\\
-1.82974263443093	-2.75921720197294\\
-1.83827595617522	-2.75580306113597\\
-1.84629896779417	-2.75198554221772\\
-1.85389438056917	-2.7477604169714\\
-1.86113115584889	-2.7431098675017\\
-1.8680670790099	-2.73800265396649\\
-1.87475059904002	-2.73239338527917\\
-1.88122205763355	-2.72622086635664\\
-1.88751436365687	-2.71940539282717\\
-1.89365310351775	-2.71184474423268\\
-1.89965600355715	-2.70340846884428\\
-1.90553156197023	-2.69392982790673\\
-1.91127652217843	-2.68319442880026\\
-1.91687162919518	-2.67092405027451\\
-1.92227472893751	-2.6567533208856\\
-1.92740961653825	-2.64019553048539\\
-1.93214788588446	-2.62059153274098\\
-1.93627893869243	-2.59703169657767\\
-1.93945940251928	-2.56823380041851\\
-1.9411256975553	-2.53234698407273\\
-1.94033865257595	-2.48662825826946\\
-1.93549900187693	-2.42689383328213\\
-1.92381069939923	-2.34656509677893\\
-1.90024268269791	-2.23498378595543\\
-1.85550126153486	-2.07446712487143\\
-1.77220791794184	-1.83557709268416\\
-1.61884814998775	-1.47187517105885\\
-1.34621301713298	-0.924796279998125\\
-0.909481515083475	-0.170949326882521\\
-0.340914514642232	0.683260693916561\\
0.223422225044149	1.42661131604494\\
0.669486332528397	1.9435442286894\\
0.982190677905218	2.26275433782589\\
1.19459131176185	2.45367579279806\\
1.34083225948148	2.56919246942168\\
1.44460153384385	2.6409310904603\\
1.52074563315668	2.686699595453\\
1.57843171408775	2.71655270800445\\
1.6234100295506	2.73631178759889\\
1.65938328561586	2.74945551357152\\
1.68880511280075	2.75812989491714\\
1.71334758481246	2.76369848551413\\
1.73418031619097	2.76705069509871\\
1.75214124503063	2.76877985050119\\
1.76784387513612	2.76928910145383\\
};
\addplot [color=mycolor1, forget plot]
  table[row sep=crcr]{%
1.61613855159024	2.72141247921817\\
1.62941391966627	2.72099848689358\\
1.64139747806358	2.71986002293162\\
1.65232379011111	2.71811627703685\\
1.66237878254442	2.71584885191853\\
1.67171183356178	2.71311119457135\\
1.68044447792192	2.70993483944171\\
1.68867675842975	2.70633345329975\\
1.69649190694504	2.70230529579381\\
1.70395981265679	2.69783446704066\\
1.71113958433924	2.69289114772267\\
1.71808140792551	2.68743091323758\\
1.72482782235928	2.68139309808504\\
1.73141447211586	2.67469808160963\\
1.73787033314867	2.66724324384313\\
1.74421733919648	2.65889717938977\\
1.75046924315976	2.64949152834833\\
1.75662941219957	2.63880944005132\\
1.76268703975575	2.62656915235071\\
1.76861090034744	2.61240031796586\\
1.77433915991991	2.59580931610619\\
1.77976267171791	2.57612745194278\\
1.78469722220312	2.55243193700056\\
1.78883652472459	2.5234224961629\\
1.79167072701979	2.48722377925767\\
1.79234135213091	2.44106055482288\\
1.789375703746	2.38070977151951\\
1.78018693291654	2.29955531830014\\
1.76011199813813	2.1869387238229\\
1.72055198175909	2.02533260345223\\
1.64553280840373	1.7859655735198\\
1.50647561611139	1.42449133190946\\
1.25979031965253	0.887322769791866\\
0.867422530010792	0.157466147395609\\
0.358701421246773	-0.662129182426281\\
-0.149473375810737	-1.37749528953538\\
-0.557458530087428	-1.88173099747749\\
-0.848618942779954	-2.19841186792069\\
-1.04949403742542	-2.39082010930412\\
-1.18952358701084	-2.50882522014053\\
-1.28985205145128	-2.58299321675379\\
-1.36403888976222	-2.63085119744051\\
-1.42059441277052	-2.66243215945022\\
-1.46492211832912	-2.68360718214946\\
-1.50053416058839	-2.69791494306816\\
-1.52977527810634	-2.70755423580167\\
-1.55425307769521	-2.71393093611635\\
-1.57509782069847	-2.71796727543747\\
-1.5931228018649	-2.72028177271289\\
-1.60892578158929	-2.72129691153713\\
-1.62295469491897	-2.72130530601542\\
-1.63555120422068	-2.72051134533719\\
-1.64698019083996	-2.71905795828844\\
-1.65745012410178	-2.71704410986192\\
-1.66712738619165	-2.71453637411781\\
-1.67614651188316	-2.71157661673854\\
-1.68461761334463	-2.70818704280787\\
-1.69263182725761	-2.70437339026219\\
-1.70026534302248	-2.70012674922752\\
-1.70758238708614	-2.69542428859632\\
-1.71463741304862	-2.69022902975082\\
-1.72147665723247	-2.68448869518187\\
-1.72813914936835	-2.67813355668702\\
-1.73465720626276	-2.67107309642847\\
-1.74105637209399	-2.66319115557991\\
-1.74735469006647	-2.65433905492944\\
-1.75356107936389	-2.64432589293457\\
-1.7596724213162	-2.6329048006232\\
-1.76566868254957	-2.61975326128974\\
-1.7715049366081	-2.60444451716606\\
-1.77709833358423	-2.58640528663938\\
-1.78230661343683	-2.56485196448675\\
-1.78689208331813	-2.5386921773767\\
-1.79045991879808	-2.50636914342008\\
-1.79234981279428	-2.46560917445441\\
-1.79144039174069	-2.41300112137448\\
-1.78578601903851	-2.34327839389174\\
-1.77192472130788	-2.24807066239538\\
-1.74353879464788	-2.11373299263696\\
-1.68889795825026	-1.91774897557392\\
-1.58643016183899	-1.6238749767559\\
-1.39964365988511	-1.18075199110278\\
-1.08242507481589	-0.543631661955198\\
-0.621555837314145	0.25307854518944\\
-0.09641245414283	1.04328968947428\\
0.36874050269807	1.65658250176025\\
0.716347118554585	2.05952788811988\\
0.958339109629879	2.30652938594418\\
1.12558565639423	2.45682608727719\\
1.2436561905534	2.55006469283883\\
1.32958975333439	2.60945667416062\\
1.39412727092238	2.64823864447461\\
1.44403321655362	2.67405908965072\\
1.48364979593202	2.69145857862251\\
1.51583699422256	2.70321614386247\\
1.54253000575099	2.71108399841515\\
1.56507272792322	2.71619739538668\\
1.58442127374257	2.71930968084827\\
1.60127112895018	2.72093100837461\\
1.61613855159024	2.72141247921817\\
};
\addplot [color=mycolor1, forget plot]
  table[row sep=crcr]{%
1.38732790552402	2.63465819500296\\
1.40062943147086	2.6342425157038\\
1.41278139958255	2.63308730039004\\
1.42398694382011	2.63129832534375\\
1.43440982960286	2.62894733458447\\
1.44418383730999	2.62607976906692\\
1.45341964378235	2.62271986095831\\
1.4622099269933	2.61887384173129\\
1.47063318375156	2.61453172508766\\
1.47875659248889	2.60966793076692\\
1.48663814362397	2.60424087250832\\
1.49432817971773	2.59819151597249\\
1.5018704234068	2.5914408000625\\
1.50930251238285	2.58388568931988\\
1.51665599722341	2.57539346488786\\
1.52395567717654	2.56579363848176\\
1.53121803284551	2.55486654492852\\
1.53844833348756	2.54232716514553\\
1.54563569904242	2.52780193788258\\
1.55274489072743	2.51079503816241\\
1.55970271801618	2.49063848845685\\
1.56637535903263	2.46641691101483\\
1.57252996400346	2.43685161615735\\
1.57776839447886	2.40011803141076\\
1.5814103256968	2.35355156732736\\
1.58228214227961	2.29316361847308\\
1.57832723800457	2.21283228039354\\
1.5658755767825	2.10294548415539\\
1.53827929241975	1.94819438992214\\
1.48349357885858	1.72440073080168\\
1.38055319409598	1.39581707532157\\
1.19781585501386	0.920306801172927\\
0.904628098562938	0.280688803757821\\
0.509489430092227	-0.456114835477672\\
0.0855409875837257	-1.14058062935781\\
-0.283257072849337	-1.66029752573636\\
-0.564207735286064	-2.00757771839611\\
-0.766873947036603	-2.22793675491897\\
-0.912235725107974	-2.36710609782894\\
-1.01829550715567	-2.45644119828739\\
-1.09766249484591	-2.51508683724071\\
-1.15866437634191	-2.55442282650588\\
-1.20675941635745	-2.58126903813666\\
-1.24557031288401	-2.59980192357757\\
-1.27755060032258	-2.61264593418094\\
-1.30439982093182	-2.62149347003746\\
-1.32732197226194	-2.62746251368772\\
-1.34718845519265	-2.63130761041243\\
-1.36464270165228	-2.63354740438033\\
-1.38016871998496	-2.63454360949568\\
-1.3941369208368	-2.63455102054594\\
-1.40683536048265	-2.63374982835509\\
-1.41849144100645	-2.63226685697424\\
-1.42928724575547	-2.63018969575471\\
-1.43937055152318	-2.62757615751248\\
-1.44886285155347	-2.62446057350835\\
-1.45786527494397	-2.62085787180965\\
-1.46646299787239	-2.61676602896612\\
-1.47472855005451	-2.61216724990523\\
-1.48272428903746	-2.60702806608776\\
-1.4905042219424	-2.60129841465141\\
-1.49811528344374	-2.59490964867501\\
-1.50559811854367	-2.58777131199924\\
-1.51298735896656	-2.57976637202622\\
-1.52031131173482	-2.57074441656157\\
-1.52759088287588	-2.56051205144964\\
-1.53483741525638	-2.54881933036228\\
-1.54204888849166	-2.53534041788314\\
-1.54920354220679	-2.51964568188709\\
-1.55624931637771	-2.50116077122663\\
-1.56308631896982	-2.47910550075426\\
-1.56953737967292	-2.45240071311581\\
-1.57529774141912	-2.41952321802537\\
-1.57984730402533	-2.37827470462384\\
-1.58229399048494	-2.32540535526911\\
-1.58108764032128	-2.25598889593011\\
-1.5734870898689	-2.16237360962026\\
-1.55455945900639	-2.03243925699405\\
-1.51534183875637	-1.84688071019744\\
-1.43979552430412	-1.57587327333968\\
-1.30141382543751	-1.17872959520849\\
-1.06600085542706	-0.619443701596676\\
-0.716526411785143	0.0840243813567158\\
-0.295063910234367	0.814367939190121\\
0.109186222039112	1.42365553711546\\
0.434684938257419	1.85305282279251\\
0.67402696988451	2.1304579326942\\
0.845463387434966	2.30536939321677\\
0.969271852192695	2.41657625809892\\
1.06071279964974	2.48875193121301\\
1.13006486318988	2.53666300667769\\
1.1840651285775	2.56909994731102\\
1.22715045655147	2.59138317239276\\
1.26229400270659	2.60681253448976\\
1.29153199480616	2.61748891908155\\
1.31629096703128	2.62478392175053\\
1.33759255451442	2.62961371435005\\
1.35618374488957	2.63260257864501\\
1.3726213487659	2.63418297028529\\
1.38732790552402	2.63465819500296\\
};
\addplot [color=mycolor1, forget plot]
  table[row sep=crcr]{%
1.02739733741893	2.47348557476368\\
1.04219991052851	2.47302138058375\\
1.0559950337039	2.47170853917827\\
1.06895728691558	2.46963781625163\\
1.08123207453146	2.46686791514125\\
1.09294215182295	2.46343119833349\\
1.10419258201004	2.45933728611963\\
1.11507453640509	2.45457500278217\\
1.12566822414637	2.44911293646631\\
1.13604514731848	2.44289872233356\\
1.14626980813599	2.43585702324449\\
1.15640093689847	2.42788604669408\\
1.16649225345643	2.41885228009767\\
1.17659271108451	2.40858292391278\\
1.18674608692328	2.39685521926411\\
1.19698965773281	2.38338145237328\\
1.20735150090413	2.36778779134156\\
1.21784563190738	2.34958414090792\\
1.2284636298166	2.32812067278729\\
1.2391604290419	2.30252424743977\\
1.24983023012145	2.2716040012872\\
1.26026537740859	2.23370898637197\\
1.2700854025811	2.18651048837497\\
1.27861313515567	2.12666571765968\\
1.28465631246388	2.04929712833414\\
1.286121874765	1.94719876205682\\
1.2793458872528	1.80969238589744\\
1.25799632015502	1.62123209077316\\
1.21156045628604	1.36058221253479\\
1.12428293704218	1.00343260412228\\
0.977774300260783	0.534507895522493\\
0.762495229105826	-0.0276962808078741\\
0.494580699093191	-0.614415920687728\\
0.21512444262883	-1.13704908249558\\
-0.0363748816684043	-1.54366353098258\\
-0.242170781785461	-1.83370229251349\\
-0.403006881609096	-2.03239387455115\\
-0.5271200897453	-2.16723258118143\\
-0.623437344275255	-2.25937028699492\\
-0.699259962306375	-2.32318702120188\\
-0.760009693129276	-2.36804447342719\\
-0.809580221423752	-2.39998858735189\\
-0.850750354948673	-2.42295559567968\\
-0.885515681979411	-2.43954711168527\\
-0.915326660655649	-2.45151305628743\\
-0.941252924563969	-2.4600514134627\\
-0.96409544920585	-2.46599588098061\\
-0.984463303547178	-2.46993500067531\\
-1.0028266286991	-2.47228901446572\\
-1.0195536418362	-2.47336025691699\\
-1.03493682563864	-2.473366694953\\
-1.0492117187949	-2.47246453206161\\
-1.06257058284439	-2.47076357138599\\
-1.07517247546841	-2.46833767408633\\
-1.08715077022958	-2.46523180445155\\
-1.09861883702821	-2.46146661564033\\
-1.10967437803078	-2.4570411775543\\
-1.1204027631107	-2.4519342071522\\
-1.13087960249329	-2.44610398495604\\
-1.1411727158489	-2.43948699841739\\
-1.15134359471152	-2.43199521991977\\
-1.16144839927243	-2.42351178387252\\
-1.17153847205639	-2.41388465128115\\
-1.18166027864401	-2.4029176125877\\
-1.19185458362132	-2.3903576388808\\
-1.20215451307336	-2.37587708363649\\
-1.21258190036276	-2.35904845940414\\
-1.22314088434211	-2.33930829887762\\
-1.23380699293633	-2.31590468120806\\
-1.24450865180094	-2.28781990495046\\
-1.25509574716695	-2.25365477084764\\
-1.2652856878568	-2.2114528242258\\
-1.27456978008644	-2.1584300336945\\
-1.28204888502655	-2.09055610856803\\
-1.28614302408898	-2.00190929011678\\
-1.28408101466374	-1.88371283442261\\
-1.27103285053692	-1.72302784374413\\
-1.238776102119	-1.50145517093768\\
-1.17419465210873	-1.19547538391306\\
-1.05944501671691	-0.782915478248143\\
-0.878472672005636	-0.26187964957388\\
-0.632896379370723	0.323789932064092\\
-0.353427167312457	0.888326759231871\\
-0.0842104180387018	1.35587313105521\\
0.145234985297924	1.70192712671374\\
0.327761757720484	1.94263941038087\\
0.469069566250933	2.10629884367719\\
0.578250597443543	2.21759988385927\\
0.663530707415728	2.29413829132118\\
0.731244886631844	2.34754665688213\\
0.785996928136564	2.38534607007067\\
0.831075464460832	2.41240712373885\\
0.868832417770454	2.43192315659928\\
0.900966209411553	2.44602308723871\\
0.928720063416563	2.45615169171875\\
0.953017804165661	2.46330641382896\\
0.974556567004055	2.46818659097927\\
0.99387047974921	2.47128894435639\\
1.01137486888268	2.47296968927403\\
1.02739733741893	2.47348557476368\\
};
\addplot [color=mycolor1, forget plot]
  table[row sep=crcr]{%
0.47620713588277	2.18437004943608\\
0.496625391790089	2.18372579956675\\
0.516344844398201	2.18184549881844\\
0.535514439991744	2.17877969559682\\
0.554268779286526	2.17454430135902\\
0.5727315313037	2.16912250662038\\
0.591018288661362	2.16246485952589\\
0.60923897433689	2.15448754932154\\
0.627499870189142	2.14506878020796\\
0.645905296015645	2.13404295309354\\
0.664558918125901	2.12119217378726\\
0.683564598818918	2.10623435253214\\
0.703026598024824	2.0888068213686\\
0.723048782054685	2.06844393344197\\
0.743732243240506	2.0445464723026\\
0.76517032472752	2.01633983223327\\
0.787439373166844	1.98281678410573\\
0.810582443206192	1.94265922636535\\
0.834581400690629	1.89413184551802\\
0.859310076920423	1.83493981906352\\
0.884456983623625	1.76204465156248\\
0.909400721240225	1.6714420062521\\
0.933016540050802	1.55793417073609\\
0.953395887840482	1.41499903073798\\
0.967492159003808	1.23499780370961\\
0.970805297122381	1.01018388457675\\
0.957426852106025	0.735142385460073\\
0.921010892175009	0.410912963375699\\
0.857092982242873	0.0493880011383427\\
0.766035925310678	-0.325839751088052\\
0.654323155910683	-0.6855301599157\\
0.53246034017998	-1.00515751171179\\
0.410885578518104	-1.27203135999729\\
0.296971860217316	-1.48520273938611\\
0.194415318420365	-1.65092506949056\\
0.104138541697818	-1.77801892895423\\
0.0254860864408128	-1.87506077908946\\
-0.0428805888734549	-1.94924200381424\\
-0.102465461955605	-2.00617381507678\\
-0.154689883582295	-2.05008085854172\\
-0.200798345906226	-2.08409273935246\\
-0.241840318683732	-2.11051574146561\\
-0.278684796607555	-2.13105091112399\\
-0.312045929833457	-2.14695799135387\\
-0.342509935822128	-2.15917489270236\\
-0.370559393274054	-2.16840355335845\\
-0.396593767743847	-2.17517133861407\\
-0.420946197254415	-2.17987493192\\
-0.443897002523521	-2.18281175264095\\
-0.465684490818418	-2.18420246095624\\
-0.486513590890022	-2.18420703600205\\
-0.50656277870641	-2.18293614879374\\
-0.525989668611247	-2.18045901049187\\
-0.544935567027176	-2.17680849067284\\
-0.56352922005612	-2.17198401849787\\
-0.581889931748121	-2.16595256483222\\
-0.600130183878707	-2.15864782773647\\
-0.618357847292737	-2.14996758545978\\
-0.636678034922541	-2.13976902101053\\
-0.655194601897781	-2.1278616415685\\
-0.674011240920889	-2.11399719299014\\
-0.693232039511397	-2.09785567790774\\
-0.71296124121597	-2.07902619108133\\
-0.733301755241592	-2.05698074329029\\
-0.754351638642492	-2.0310385007615\\
-0.776197250911278	-2.00031686454427\\
-0.79890092114673	-1.96366452461323\\
-0.822479567427569	-1.91957013046458\\
-0.846868468608206	-1.86603893716692\\
-0.871860950673296	-1.80042997719165\\
-0.89700991179925	-1.71925125011644\\
-0.921471626576344	-1.61792774868017\\
-0.943770128972071	-1.49060270372679\\
-0.961474196303445	-1.33013337698189\\
-0.970838148433354	-1.12862641091917\\
-0.966609375019177	-0.879085111628614\\
-0.942457436254818	-0.578726314182426\\
-0.892610136125284	-0.233577186077621\\
-0.814687491133952	0.138325729779689\\
-0.712153021676426	0.509387866614885\\
-0.593959795072005	0.851454979769253\\
-0.471086238914411	1.14547925137416\\
-0.352660605060765	1.38504616933722\\
-0.244166059846886	1.57346115459866\\
-0.147758427943964	1.7187390553571\\
-0.0634346279898705	1.82980920482635\\
0.009888947335459	1.91462255823226\\
0.0736783576399879	1.97957090125034\\
0.129415354679625	2.02953758502394\\
0.178438527373515	2.06816309229685\\
0.221894380239587	2.09813451538813\\
0.260739303706182	2.12143181788064\\
0.295761446410512	2.1395179456237\\
0.327607743123823	2.15347913236043\\
0.356809782611324	2.16412628696682\\
0.38380626313929	2.1720676188883\\
0.408961606107032	2.17776054426846\\
0.432581034590971	2.18154881346394\\
0.454922659440202	2.18368910282942\\
0.47620713588277	2.18437004943608\\
};
\addplot [color=mycolor1, forget plot]
  table[row sep=crcr]{%
-0.218432368692388	1.75006885244458\\
-0.178478874550534	1.74879469991791\\
-0.137402821732203	1.74486457968917\\
-0.0950083444367945	1.73807094169262\\
-0.0510875079760354	1.72813839612093\\
-0.00542085812119394	1.71471388417246\\
0.0422207735712061	1.69735461930125\\
0.0920724954094568	1.67551384024463\\
0.144368606585523	1.64852472000514\\
0.199331491102878	1.61558333583957\\
0.257154326987754	1.57573256717545\\
0.317975337776623	1.52785033449011\\
0.38184103876284	1.47064791060494\\
0.448656175742909	1.40268720968177\\
0.518119426896689	1.32242971859341\\
0.589647322399636	1.22833301036837\\
0.662295171885714	1.11901109920361\\
0.734693343049372	0.993468019785329\\
0.805028151840547	0.851394813056192\\
0.871103239544788	0.69348622035243\\
0.93051018123024	0.521692030405634\\
0.98090791377875	0.33929151936184\\
1.02036266320661	0.150699125173134\\
1.04765596401809	-0.0390090651407171\\
1.06246077478911	-0.224743756910975\\
1.06532985788274	-0.402041008726533\\
1.05751428953611	-0.567514609618132\\
1.04068841502765	-0.719035767469351\\
1.01667106016879	-0.85566276912545\\
0.987206502897894	-0.97741184673942\\
0.953828795059489	-1.08497118276009\\
0.91780206834694	-1.17943256047259\\
0.880115114284985	-1.26207804852348\\
0.841507293520577	-1.33423012225647\\
0.802507999458383	-1.397157525876\\
0.763478381307864	-1.45202337733812\\
0.724649355479105	-1.49986213387365\\
0.686153487547168	-1.54157460811198\\
0.648050358305878	-1.57793325628698\\
0.610346013277956	-1.60959257302992\\
0.573007458842296	-1.63710138046797\\
0.535973205639494	-1.66091513805353\\
0.499160751098688	-1.68140726081827\\
0.462471736243642	-1.69887895800313\\
0.425795355670333	-1.71356740441454\\
0.389010464018177	-1.7256522148187\\
0.351986713401945	-1.7352602619678\\
0.314584973745266	-1.74246889659897\\
0.276657229172496	-1.7473076147454\\
0.238046105928604	-1.74975818716593\\
0.198584169077412	-1.74975322574803\\
0.158093126445626	-1.74717311794589\\
0.116383100910635	-1.74184121782058\\
0.0732521805527515	-1.73351714785357\\
0.0284865375923655	-1.72188805042904\\
-0.0181384683047321	-1.70655765052496\\
-0.0668556056456335	-1.68703308344085\\
-0.117900656747315	-1.66270965522525\\
-0.171503689684931	-1.63285412029728\\
-0.227875117489127	-1.596587801435\\
-0.287184486792264	-1.55287210664826\\
-0.349529547676552	-1.50050090893681\\
-0.414893083737573	-1.43810700479428\\
-0.483085696016494	-1.36419340466288\\
-0.553674989290907	-1.27720393252153\\
-0.625906368285954	-1.1756498278953\\
-0.69862865302886	-1.05830639631927\\
-0.770248394068362	-0.92448131101201\\
-0.838746521318586	-0.774329508164403\\
-0.901792002438366	-0.609150208681351\\
-0.956969634936358	-0.431563950437059\\
-1.00209899767848	-0.245460409394229\\
-1.03557164603545	-0.0556587079783228\\
-1.0566040611219	0.132673297390674\\
-1.06532245902061	0.314687447316354\\
-1.06265948272779	0.48641850018189\\
-1.05011439250815	0.645103748192245\\
-1.0294659048525	0.789229178833693\\
-1.00251785216747	0.918366425926476\\
-0.970921189464369	1.03290364240365\\
-0.936078507461518	1.13376138894612\\
-0.899114494127462	1.22214923861014\\
-0.860888801271196	1.29938442668872\\
-0.822030537946831	1.36677135618774\\
-0.78297990227181	1.42553035213702\\
-0.74402853215203	1.47676173626029\\
-0.705354597016382	1.52143297586337\\
-0.667051389562434	1.56037963029418\\
-0.629149625407273	1.59431370034004\\
-0.591634288114514	1.62383527509477\\
-0.554457028715936	1.64944500234484\\
-0.51754507661473	1.67155598997075\\
-0.48080747746677	1.69050442247485\\
-0.444139313285142	1.70655857697429\\
-0.407424412692465	1.71992614383045\\
-0.370536936812692	1.7307598657518\\
-0.333342130857583	1.73916154979285\\
-0.295696461168918	1.74518450695214\\
-0.257447309526412	1.74883445104043\\
-0.218432368692388	1.75006885244458\\
};
\addplot [color=mycolor1, forget plot]
  table[row sep=crcr]{%
-0.751180563789503	1.29301100092275\\
-0.644509272490707	1.2895621961483\\
-0.525853075685341	1.27816235586954\\
-0.394733067383687	1.25710640312682\\
-0.251255417385498	1.22462204940121\\
-0.0963657956622014	1.17906474936042\\
0.0679313764001297	1.11919457949777\\
0.238479475719274	1.04449792159326\\
0.411096645609142	0.95547160953947\\
0.580968788823447	0.85375980073978\\
0.743253360925444	0.742056725339432\\
0.893734219404683	0.623769327860435\\
1.02933068138786	0.502533729359883\\
1.14832654684568	0.381735283717928\\
1.25030593415255	0.264160768651649\\
1.33588484834454	0.151837028466942\\
1.40636299856342	0.046035698606773\\
1.46339657703116	-0.0526160638365817\\
1.50874510492206	-0.143981414805869\\
1.54410401747092	-0.228260871982706\\
1.57101084154411	-0.305863800112672\\
1.59080459914876	-0.37731143443009\\
1.60461898267407	-0.443169442531953\\
1.61339434135726	-0.50400430761343\\
1.6178983941152	-0.560357233657007\\
1.61874953068363	-0.612730111534175\\
1.61643932306818	-0.661579338697342\\
1.6113526202479	-0.707314476704708\\
1.60378462375957	-0.750299690229399\\
1.59395489259106	-0.790856617640623\\
1.58201848840448	-0.829267814089403\\
1.56807456975614	-0.865780234303465\\
1.55217275313418	-0.900608430316382\\
1.53431752429447	-0.933937265273067\\
1.51447093083552	-0.965924014217765\\
1.49255372986368	-0.996699753941291\\
1.4684451102011	-1.02636994778256\\
1.44198106109596	-1.05501411446045\\
1.41295142229282	-1.08268443616728\\
1.38109562819774	-1.10940311195514\\
1.34609715924291	-1.13515819874194\\
1.30757674909159	-1.15989760548665\\
1.26508448786515	-1.18352082038956\\
1.2180911421837	-1.20586786686862\\
1.16597933286764	-1.22670492454442\\
1.10803574356846	-1.24570606215447\\
1.04344637620195	-1.26243069409395\\
0.971298136519321	-1.27629683490271\\
0.890591822483024	-1.2865512138292\\
0.800273888148872	-1.29223914571393\\
0.699296856787975	-1.29218010554772\\
0.586720020892429	-1.28495946080335\\
0.461861094092619	-1.26895243972215\\
0.3245024552969	-1.24240144182672\\
0.175138600776365	-1.2035682010117\\
0.0152226545865632	-1.15097166888395\\
-0.152664711632565	-1.08369478588514\\
-0.324821834412255	-1.00170023256538\\
-0.496685757869856	-0.90605470141019\\
-0.663344832918517	-0.798955642596777\\
-0.820192630077424	-0.683508161867404\\
-0.963532782997699	-0.563297645942172\\
-1.09095770410823	-0.441888842741164\\
-1.20142500062305	-0.322399917701083\\
-1.29507612370551	-0.207246320149147\\
-1.37291296977355	-0.0980684877283813\\
-1.43645044441771	0.0042020468738688\\
-1.48742259418976	0.0992032593483852\\
-1.52757278250983	0.186984897840184\\
-1.55852539637732	0.267866024583987\\
-1.58172126356109	0.342322104544315\\
-1.59839612803496	0.410903378933661\\
-1.60958478267695	0.474180138513564\\
-1.61613840476429	0.532708621189078\\
-1.61874713092658	0.587011554073861\\
-1.61796325588701	0.637568493346236\\
-1.61422266111825	0.684812373734955\\
-1.60786343226419	0.729129764048409\\
-1.59914138437268	0.770863155573305\\
-1.58824260169102	0.810314202630113\\
-1.57529326680601	0.847747236816572\\
-1.56036710004137	0.883392638422727\\
-1.54349071326817	0.917449811620675\\
-1.52464713656047	0.950089605203132\\
-1.50377772010479	0.981456069358695\\
-1.4807825573749	1.0116674553813\\
-1.45551952420259	1.04081635809509\\
-1.42780198561346	1.06896887513935\\
-1.39739519197381	1.09616261560587\\
-1.3640113739163	1.12240333396511\\
-1.32730356145226	1.14765989469051\\
-1.28685821295664	1.17185719094692\\
-1.24218687070083	1.19486655389881\\
-1.19271730234616	1.21649311338865\\
-1.13778500336415	1.23645953866669\\
-1.07662660918946	1.2543856619263\\
-1.00837780643194	1.26976377842146\\
-0.932079852456276	1.28193010776828\\
-0.846700871404774	1.2900342666388\\
-0.751180563789503	1.29301100092275\\
};
\addplot [color=mycolor1, forget plot]
  table[row sep=crcr]{%
-0.792711621111321	0.98157953768228\\
-0.524039126534901	0.972855450713703\\
-0.223155287672921	0.943970299352618\\
0.0970845287357334	0.89265178624017\\
0.418361554453311	0.820112091604991\\
0.721568299281959	0.731202980446408\\
0.992069737039553	0.632937470787459\\
1.22249537797474	0.532313651163314\\
1.41221608098798	0.434728026924781\\
1.56495976330159	0.343484831498922\\
1.68635092136143	0.260092913835551\\
1.78224001120135	0.18483850219387\\
1.8578635205525	0.11730988759003\\
1.9175611884238	0.0567670347987813\\
1.96478181381882	0.00236466154353296\\
2.00220485764216	-0.0467286451778812\\
2.03188865709344	-0.0912751891546783\\
2.05540712823268	-0.131948588281725\\
2.07396278328999	-0.169332489372267\\
2.08847528569881	-0.203927707183566\\
2.09964889554858	-0.236162518549415\\
2.10802305513758	-0.266403470902291\\
2.11401000820589	-0.294965496142634\\
2.11792262303556	-0.322120867955179\\
2.11999485572392	-0.348106911291836\\
2.12039666474396	-0.3731325425674\\
2.11924469208138	-0.39738378512693\\
2.11660964448466	-0.421028418889408\\
2.11252101571752	-0.444219913015945\\
2.10696956246702	-0.467100769619532\\
2.09990776123253	-0.489805381061177\\
2.09124831286927	-0.512462475132387\\
2.08086060958534	-0.53519719055088\\
2.06856492130052	-0.558132786625655\\
2.05412387949755	-0.581391940158127\\
2.03723062115011	-0.605097510735529\\
2.01749268582802	-0.629372548500999\\
1.99441041849142	-0.654339154204018\\
1.96734820656221	-0.680115545407622\\
1.93549637849465	-0.706810282115394\\
1.89782106738893	-0.734511980507324\\
1.85299897045652	-0.763271883400435\\
1.79933414963148	-0.793075222898754\\
1.73465580805775	-0.823795289805376\\
1.6562014095423	-0.855121592693686\\
1.56050256136341	-0.886451175769476\\
1.44331848846043	-0.916732514784459\\
1.29971241791409	-0.944260595272382\\
1.12444293603457	-0.966452087976906\\
0.912917778625771	-0.979697607096246\\
0.662923339355622	-0.979494985082252\\
0.37698478839068	-0.961143526917589\\
0.0643945057451018	-0.92113151248099\\
-0.258836596504117	-0.858806199962149\\
-0.573293343729396	-0.77729038059577\\
-0.86154610949324	-0.682778190636289\\
-1.11246542293541	-0.58253462290651\\
-1.32228123826288	-0.482886497173722\\
-1.4928788430727	-0.388183675967567\\
-1.62920072357579	-0.300767744684041\\
-1.73713626895234	-0.221463029054589\\
-1.82229189975114	-0.15015026080482\\
-1.88946780204592	-0.0862173983858334\\
-1.9425473720943	-0.0288514537115337\\
-1.98457653540813	0.0227960100964114\\
-2.01790588850602	0.0695260087674356\\
-2.04433584795107	0.112057504232829\\
-2.06524208259898	0.151018597474137\\
-2.08167602871728	0.186950344449971\\
-2.09444236973088	0.220315931258629\\
-2.10415752737457	0.251511445216503\\
-2.11129331982863	0.280876428860417\\
-2.11620933433931	0.308703443840538\\
-2.11917680800605	0.335246401829789\\
-2.12039612423771	0.360727673755501\\
-2.1200094724428	0.385344097860135\\
-2.11810978260752	0.409272042657603\\
-2.11474671244081	0.432671680630897\\
-2.10993020723372	0.455690611939787\\
-2.10363194826658	0.478466953807131\\
-2.09578483471324	0.50113198435242\\
-2.08628048961741	0.523812399829979\\
-2.07496462746163	0.546632209530777\\
-2.06162995442018	0.569714248797749\\
-2.04600607729657	0.593181230636013\\
-2.02774565683554	0.617156169080839\\
-2.0064057380819	0.641761875216961\\
-1.98142280905275	0.667119021938585\\
-1.95207967357141	0.693341953353095\\
-1.91746169957402	0.720530913984127\\
-1.87639952256073	0.748758596599108\\
-1.82739512479565	0.778047728646369\\
-1.76852903775977	0.808334698837547\\
-1.69734967106284	0.839411914748699\\
-1.61075435885193	0.870838996005181\\
-1.50489091967973	0.901811446283295\\
-1.37514641412866	0.930979107423511\\
-1.21635403674298	0.95622448292148\\
-1.02343225058082	0.974458275051307\\
-0.792711621111322	0.98157953768228\\
};
\addplot [color=mycolor1, forget plot]
  table[row sep=crcr]{%
-0.392413722930891	0.84554181943936\\
0.0574991168375939	0.831154097362173\\
0.505147659949656	0.788519822044919\\
0.911684194589035	0.723753194528262\\
1.25373099685688	0.64685994138717\\
1.52646987318868	0.567132827487189\\
1.73723815412086	0.490727610699746\\
1.89787424528086	0.420675286662048\\
2.02003209129655	0.3578942088356\\
2.11336301864759	0.302168930108651\\
2.18525659578056	0.252792732245178\\
2.24116986978698	0.208915948011435\\
2.28507546096545	0.169710004235972\\
2.31985801824868	0.134432497075667\\
2.34762016050889	0.102444204275014\\
2.36990526938917	0.0732049691631955\\
2.38785567468071	0.0462616886765516\\
2.40232392619075	0.0212344929143888\\
2.41395099003992	-0.00219632298715742\\
2.42322138148395	-0.0243017186686428\\
2.43050222936105	-0.0453135288483911\\
2.43607108364639	-0.0654324372620509\\
2.44013575490199	-0.0848345583017478\\
2.44284842789974	-0.103676892510789\\
2.44431557320258	-0.122101899568361\\
2.44460468239501	-0.140241400069661\\
2.44374849975782	-0.158219985998165\\
2.44174716327257	-0.17615809341235\\
2.43856846387418	-0.194174870617511\\
2.43414625602581	-0.212390960687017\\
2.42837688181337	-0.230931307666362\\
2.42111328258965	-0.249928089576639\\
2.41215624176518	-0.269523875938011\\
2.40124189793098	-0.289875098650625\\
2.38802424562686	-0.311155904824948\\
2.37205073996559	-0.333562413755809\\
2.35272825286137	-0.357317299644011\\
2.32927537032245	-0.382674413614582\\
2.30065521410733	-0.409922742037863\\
2.26548044975565	-0.439388182657007\\
2.22187884264946	-0.471430051093477\\
2.16730403669588	-0.506426266136148\\
2.0982739817986	-0.544735706230507\\
2.01002436055964	-0.586616641754953\\
1.89609176532488	-0.632064808809404\\
1.74792684764033	-0.680515169409561\\
1.55484842567927	-0.730343280848899\\
1.30505799123314	-0.778160890317086\\
0.988919447812327	-0.818148046207319\\
0.605416543083506	-0.842192477576616\\
0.169844949337834	-0.841990436587597\\
-0.284265742050115	-0.81313340510902\\
-0.715481032707434	-0.758309777154651\\
-1.09145467912348	-0.686181762958299\\
-1.39848995418638	-0.60688389900676\\
-1.63889402833189	-0.528262755877788\\
-1.82305794230454	-0.454813385393407\\
-1.96310514827734	-0.388371966541922\\
-2.06978862969178	-0.329187492207117\\
-2.15160841140705	-0.27673952309024\\
-2.21493276229014	-0.230220283082745\\
-2.26442186776805	-0.188777889956994\\
-2.30346035018527	-0.151622578146054\\
-2.33450919511601	-0.118063158297597\\
-2.35936804367039	-0.0875111519170718\\
-2.37936328361864	-0.0594716327281494\\
-2.3954807891641	-0.033529829150519\\
-2.40845918011913	-0.00933749359531244\\
-2.41885544449055	0.0133993926007298\\
-2.42709131700411	0.0349311323748096\\
-2.43348622202874	0.0554730815850008\\
-2.43828076060055	0.0752128930362168\\
-2.44165345782466	0.0943165027594429\\
-2.44373262028527	0.112933115758461\\
-2.44460455599637	0.131199419032673\\
-2.44431899168683	0.149243217004679\\
-2.44289222122969	0.167186655319756\\
-2.44030829102082	0.185149175644964\\
-2.43651834129811	0.203250326852542\\
-2.43143805178636	0.22161254615892\\
-2.42494296339643	0.240364016156838\\
-2.41686124184706	0.259641698268294\\
-2.40696318622118	0.279594636712781\\
-2.39494642873272	0.300387613833229\\
-2.38041526965797	0.322205206779803\\
-2.36285186960453	0.345256226766583\\
-2.34157597558486	0.369778376738065\\
-2.31568834666596	0.396042667919569\\
-2.28399090003332	0.424356551513887\\
-2.24487367983333	0.45506358830937\\
-2.19615515333315	0.488535316920607\\
-2.13485898358148	0.525146943201869\\
-2.05691063622175	0.565221183974557\\
-1.95675047473234	0.608912266997585\\
-1.82691017093183	0.655983984208223\\
-1.65773700274838	0.705418249335526\\
-1.43775505952901	0.75480505417595\\
-1.15564363113835	0.799599025224289\\
-0.805088588744831	0.832719416165173\\
-0.392413722930893	0.84554181943936\\
};
\addplot [color=mycolor1, forget plot]
  table[row sep=crcr]{%
0.127583927346025	0.822354782701324\\
0.648213232803182	0.80605800795337\\
1.10318107339043	0.76304053956409\\
1.46974443062957	0.704857395593081\\
1.7506277306205	0.641835239258246\\
1.96083296670893	0.580446894605021\\
2.1172920353496	0.523753890913169\\
2.23435929047659	0.472709568819359\\
2.32289332314376	0.427209264636958\\
2.39071777840021	0.386710389660878\\
2.44337011656711	0.350545072678569\\
2.48476144153763	0.318059921922197\\
2.51767062198877	0.288669271144002\\
2.54409160763447	0.261868466494899\\
2.56547156713278	0.237230258626693\\
2.58287306167405	0.214395113582127\\
2.5970847403339	0.193060239851003\\
2.60869741614965	0.172969283164263\\
2.61815680955908	0.153903340296842\\
2.62580044199271	0.135673377279496\\
2.63188363663354	0.118113912642419\\
2.63659792076735	0.101077754940444\\
2.64008402751842	0.0844315774446671\\
2.6424409643375	0.0680521305771022\\
2.64373212040189	0.0518229159299322\\
2.64398903950231	0.0356311665529758\\
2.64321323224248	0.0193649933701248\\
2.64137620229844	0.00291056587197173\\
2.63841768691518	-0.0138508038767062\\
2.63424193709751	-0.0310458117512815\\
2.62871166336885	-0.0488126995318836\\
2.62163901919879	-0.0673053570165812\\
2.61277264673791	-0.0866981974870052\\
2.60177931095254	-0.107192055562649\\
2.58821791195275	-0.129021398355355\\
2.57150255840735	-0.152463170319034\\
2.55084970446004	-0.177847574298192\\
2.52520179584149	-0.205570946037465\\
2.49311601422567	-0.236110429802094\\
2.45260104892505	-0.270039025142066\\
2.40087706065533	-0.308036934055735\\
2.3340250863367	-0.350889310674258\\
2.24648755503755	-0.39944821162938\\
2.13040049228801	-0.454512077012021\\
1.97483523001082	-0.516532619024773\\
1.76533290111822	-0.585000852211627\\
1.48484120470863	-0.657355061917321\\
1.11829318779626	-0.727527135003885\\
0.663073259146776	-0.785196445728787\\
0.142231167785234	-0.818072895775999\\
-0.393472173993291	-0.818155783814035\\
-0.886049713082783	-0.787199461778638\\
-1.29779320358844	-0.73511883652189\\
-1.62009373036179	-0.673452659624767\\
-1.86347479476686	-0.610680751481175\\
-2.04480161083383	-0.551419036273427\\
-2.17999142771539	-0.497516541074019\\
-2.28164162385726	-0.449294850545918\\
-2.35900309711399	-0.406375673156854\\
-2.41866449572068	-0.368128112625037\\
-2.46527788530321	-0.33388094484314\\
-2.5021365149178	-0.303011228645238\\
-2.53159107832171	-0.274973555532582\\
-2.55533779171109	-0.249302966880185\\
-2.57461494025345	-0.225607431139841\\
-2.59033677587351	-0.203557141000713\\
-2.6031851909469	-0.182873735592355\\
-2.61367299057704	-0.163320619972174\\
-2.62218795812845	-0.144694689635349\\
-2.62902380288774	-0.126819405365888\\
-2.63440202887595	-0.109539031254651\\
-2.63848741459588	-0.092713817117327\\
-2.641398900009	-0.0762159155355208\\
-2.64321707729127	-0.059925845691677\\
-2.6439890701039	-0.043729338835614\\
-2.64373129273313	-0.0275144184669946\\
-2.64243035866194	-0.0111685801382348\\
-2.64004222456362	0.00542405958235494\\
-2.63648948404545	0.0223857783896487\\
-2.63165654158267	0.0398485720872036\\
-2.62538217455736	0.0579579507065615\\
-2.61744869653946	0.0768773987940208\\
-2.60756652062552	0.0967937282660691\\
-2.59535231747372	0.117923594147524\\
-2.58029806141494	0.140521481901149\\
-2.56172689521631	0.164889487239771\\
-2.53872967067157	0.191389142111881\\
-2.51007287587656	0.220455274617765\\
-2.47406396369724	0.252611164735925\\
-2.42835338315418	0.288482511170148\\
-2.3696439929403	0.328803780544085\\
-2.29327057097972	0.374401992519135\\
-2.19261540300272	0.426125476488849\\
-2.05837353081808	0.484651867778157\\
-1.87786003310808	0.550056379298748\\
-1.635040280013	0.620972440860603\\
-1.31294328821675	0.693264653039546\\
-0.901053209393393	0.758704786362719\\
-0.408199139990258	0.805422528040251\\
0.127583927346023	0.822354782701324\\
};
\addplot [color=mycolor1, forget plot]
  table[row sep=crcr]{%
0.5647960623138	0.853812859999501\\
1.07128673553608	0.838198408636122\\
1.48039777394432	0.799662583060077\\
1.79266817538587	0.750166491000075\\
2.02479185719599	0.698110293912658\\
2.19632768966487	0.648020529098301\\
2.32386018402829	0.601806757171454\\
2.4198235022426	0.559959493407524\\
2.49307769479211	0.522306821137164\\
2.54982610613474	0.488416853654953\\
2.59440759520677	0.457790987159596\\
2.62987906959654	0.429948280836494\\
2.65841847283103	0.404457029515137\\
2.68159732786201	0.380942142385574\\
2.70056417724772	0.359082240673385\\
2.71616855270952	0.338602997986055\\
2.72904546107385	0.319269651181613\\
2.73967355164051	0.300879897931048\\
2.74841557811197	0.283257601678915\\
2.75554679891187	0.266247368715491\\
2.76127503362634	0.249709913689685\\
2.76575483893485	0.233518077021313\\
2.76909744165955	0.21755334638614\\
2.77137751332221	0.201702738965491\\
2.77263748964209	0.185855909231112\\
2.77288986443052	0.169902352837051\\
2.77211767622501	0.153728577765967\\
2.77027322578096	0.1372151074127\\
2.76727488704598	0.120233165192337\\
2.76300167813254	0.102640864674644\\
2.7572850126839	0.0842786907811072\\
2.74989671666541	0.0649640032548598\\
2.7405319138553	0.044484220158523\\
2.72878466789091	0.0225882441383666\\
2.71411318339692	-0.00102442090286279\\
2.69578969943864	-0.0267175372222971\\
2.67282762221409	-0.0549357456712831\\
2.64387443893777	-0.086226620103167\\
2.60705283594023	-0.121267363051761\\
2.55972344891701	-0.160894741313181\\
2.49813067786754	-0.206132626129549\\
2.41688109230449	-0.258201271274281\\
2.30820646225063	-0.31846933337672\\
2.16102794156049	-0.388261527787755\\
1.9600942649918	-0.468349327874165\\
1.68618532970365	-0.557854874809137\\
1.31978937039226	-0.652388661475404\\
0.851701177658143	-0.742091957403252\\
0.2996878226251	-0.812222786327461\\
-0.283403565055562	-0.849311024806142\\
-0.829131909311639	-0.849671124327323\\
-1.28856516732028	-0.820991188416604\\
-1.64777878514603	-0.775658460547183\\
-1.91749270169338	-0.724097527473434\\
-2.11699928820803	-0.67265504472608\\
-2.26472299505105	-0.624376210638953\\
-2.37516131876585	-0.58033858751573\\
-2.45884957200914	-0.540632718202448\\
-2.52320806872325	-0.504922443915455\\
-2.57342209009238	-0.472726917545838\\
-2.61312907901512	-0.44355003443715\\
-2.64490534747896	-0.416933488381197\\
-2.67059795700049	-0.392473816110554\\
-2.69154829428359	-0.36982347867641\\
-2.70874274362947	-0.34868555022563\\
-2.72291490439807	-0.328806414191347\\
-2.73461560117952	-0.309968379463334\\
-2.74426133726505	-0.291982958094855\\
-2.75216816003143	-0.274685010667489\\
-2.75857551538068	-0.257927732765276\\
-2.7636631154935	-0.241578364791952\\
-2.76756282819312	-0.225514479799037\\
-2.7703669221514	-0.209620702756656\\
-2.77213354502348	-0.193785722015356\\
-2.77288999071339	-0.177899461140576\\
-2.77263407401168	-0.161850282759114\\
-2.77133373868317	-0.145522093310267\\
-2.7689248503333	-0.128791207021445\\
-2.76530694288232	-0.111522807349164\\
-2.76033647049569	-0.0935668124380141\\
-2.75381683187417	-0.0747529051583457\\
-2.7454840339232	-0.054884424790499\\
-2.73498627666103	-0.033730733354312\\
-2.7218548619301	-0.01101756388009\\
-2.70546248424923	0.0135852635734658\\
-2.68496288535053	0.040479485056165\\
-2.65920263327697	0.0701581547916423\\
-2.62659082253035	0.103230057451741\\
-2.58490501885585	0.140448374709387\\
-2.53100118982908	0.18274056801885\\
-2.4603826165084	0.23122979984345\\
-2.36657498260569	0.287222701791902\\
-2.24027879008065	0.35210449981766\\
-2.06840652840338	0.42701652250624\\
-1.8335570524236	0.512090553182067\\
-1.51554768948008	0.604966091936343\\
-1.09817415321068	0.698692925584068\\
-0.583632372478345	0.780585967110578\\
-0.00768759183282086	0.835429000391897\\
0.564796062313798	0.853812859999501\\
};
\addplot [color=mycolor1, forget plot]
  table[row sep=crcr]{%
0.890712145070811	0.909450418757661\\
1.35615885261011	0.895223268347426\\
1.7178770133649	0.861204771906555\\
1.98883562135947	0.81827185341504\\
2.18934297244442	0.773305016262608\\
2.33813014010674	0.729851990461745\\
2.44972390595194	0.68940712468328\\
2.5346150158692	0.652382072672868\\
2.60017637536743	0.618678388031736\\
2.65156020319445	0.58798790340672\\
2.69238512568549	0.559939164936878\\
2.72521891863578	0.534163927537106\\
2.75190661716258	0.510324101099737\\
2.77379135916652	0.488119891841291\\
2.79186327696568	0.467289454487744\\
2.80686061899949	0.447605063345819\\
2.81933911290857	0.428868163364543\\
2.82972005140612	0.410904358831404\\
2.83832395807279	0.393558763793993\\
2.84539433942666	0.376691836434651\\
2.85111450317749	0.360175680602173\\
2.85561942298616	0.343890737238368\\
2.85900396567129	0.327722764353783\\
2.86132834493843	0.311559995282265\\
2.86262134708813	0.29529036032451\\
2.86288163617489	0.278798650749535\\
2.86207725184185	0.261963492909228\\
2.8601432339406	0.244653981187701\\
2.85697711819763	0.226725788861691\\
2.85243181853318	0.208016532305813\\
2.84630510839095	0.18834010198256\\
2.83832448478395	0.167479587806611\\
2.82812556800403	0.145178310381336\\
2.81522123766127	0.121128317454752\\
2.79895724360707	0.094955515907978\\
2.7784477619024	0.0662003995574959\\
2.75248083374686	0.0342931637967893\\
2.71937815937494	-0.00147795233693322\\
2.67678546413278	-0.0420055273043307\\
2.62135794787402	-0.0884069185173281\\
2.54829130368562	-0.142064245531875\\
2.45064075021913	-0.204634033811873\\
2.31839972834169	-0.277960726236148\\
2.13747019310304	-0.363747848256886\\
1.88917402634202	-0.46271092344986\\
1.55217347467057	-0.572854125554091\\
1.11031757645939	-0.686931045624751\\
0.5686004428524	-0.790900393956026\\
-0.0317095263207011	-0.867395661633784\\
-0.621049791188676	-0.905112993988739\\
-1.13650530825543	-0.905617152832956\\
-1.5495086819587	-0.87991946706539\\
-1.86345932543428	-0.840329516553253\\
-2.0966362049343	-0.795758438133868\\
-2.26919022981483	-0.751261298010872\\
-2.3978358398636	-0.70921082060183\\
-2.49498482746347	-0.67046583142385\\
-2.56944700736389	-0.635131647095147\\
-2.62738491506972	-0.602979271852111\\
-2.67311188480114	-0.573656849805782\\
-2.70967167159994	-0.546789350398643\\
-2.73923734713559	-0.522021743950802\\
-2.7633805729318	-0.49903476354143\\
-2.78325281465181	-0.477547928176154\\
-2.79970790674062	-0.457317036615965\\
-2.81338569230257	-0.438129586166547\\
-2.82476970026708	-0.419799710568105\\
-2.83422733473163	-0.402163320645339\\
-2.84203813121379	-0.385073692666452\\
-2.84841374148752	-0.368397543383253\\
-2.85351207522891	-0.352011538208387\\
-2.85744721363654	-0.335799140336732\\
-2.86029616421148	-0.319647693949695\\
-2.86210314856951	-0.303445628825438\\
-2.86288184224531	-0.28707966883864\\
-2.8626157732811	-0.270431918524477\\
-2.86125690299825	-0.253376687088789\\
-2.85872223112257	-0.235776885282436\\
-2.8548880620305	-0.217479794390573\\
-2.84958130814093	-0.198311954384588\\
-2.84256684818636	-0.178072845120389\\
-2.83352944016367	-0.156526934314425\\
-2.82204791572865	-0.133393532481749\\
-2.80755820480412	-0.108333723894724\\
-2.78929991967601	-0.0809334392355528\\
-2.76623839708287	-0.050681533377657\\
-2.73694969564132	-0.016941637589268\\
-2.69944929699592	0.0210831417040737\\
-2.65093531208999	0.0643921817503108\\
-2.58740373957987	0.114231187293853\\
-2.5030803834195	0.172122051690198\\
-2.38961812563455	0.239836476890025\\
-2.23508483033654	0.319213739722195\\
-2.0230669203918	0.411616201279232\\
-1.73304317172197	0.516683621167623\\
-1.34476477329517	0.630124557518481\\
-0.850287967294496	0.741278414960865\\
-0.271751240440297	0.833554342235435\\
0.332303788006903	0.891315090999931\\
0.890712145070809	0.909450418757661\\
};
\addplot [color=mycolor1, forget plot]
  table[row sep=crcr]{%
1.13228895415094	0.976232367083907\\
1.55453529055446	0.963380785631541\\
1.87735344896523	0.933035113576346\\
2.1184845775288	0.894825371350644\\
2.29787683748837	0.854585882214751\\
2.43227481251057	0.815326896230432\\
2.53422473485194	0.778370124569441\\
2.61269575478856	0.744139423495457\\
2.67399866107858	0.712620330446479\\
2.72257214445541	0.683604727923391\\
2.76156142008329	0.656814173994275\\
2.79322006604216	0.631958960343927\\
2.81918314674446	0.608764303135775\\
2.84065202094255	0.586980174694858\\
2.85851989743146	0.56638325395648\\
2.87345784029931	0.546775281493775\\
2.88597427105367	0.527979946664411\\
2.8964565493718	0.509839330570172\\
2.90520028440166	0.492210366795147\\
2.91243011729296	0.474961497954436\\
2.91831446402342	0.457969563902499\\
2.92297587905606	0.441116886757648\\
2.92649814196841	0.424288482122503\\
2.92893078207311	0.407369305689946\\
2.93029147526219	0.390241429259922\\
2.93056652851896	0.372781023828083\\
2.92970947811905	0.354855005560475\\
2.92763764022047	0.336317169435942\\
2.92422624024647	0.317003591131751\\
2.91929947813593	0.296727015455653\\
2.91261751677722	0.275269863030624\\
2.90385784701549	0.252375368188296\\
2.89258868678762	0.227736201213942\\
2.87823086126403	0.200979719574153\\
2.86000274508799	0.171648736327014\\
2.83683995383567	0.139176417712259\\
2.8072769916719	0.102853728926342\\
2.76927123580572	0.0617880301729651\\
2.71993968118953	0.0148527389768783\\
2.65516586661613	-0.0393677412042734\\
2.56902248807935	-0.102622103283209\\
2.45296226998788	-0.176981508802543\\
2.29481334781964	-0.264668724270688\\
2.07792820730373	-0.367504549350102\\
1.78167534396632	-0.485598501371426\\
1.38600553881976	-0.614970273259653\\
0.883559945743928	-0.744805485886824\\
0.297063929592292	-0.857551335033007\\
-0.314995550482536	-0.935748231473546\\
-0.881964052990284	-0.972192941185301\\
-1.35656539196237	-0.972743377863418\\
-1.72742597329563	-0.949699043438517\\
-2.00684087248312	-0.914467679869741\\
-2.21475183366217	-0.874719671788787\\
-2.36982151425778	-0.834722796055881\\
-2.48667231671519	-0.796519916364948\\
-2.57594861399704	-0.760908291075883\\
-2.64517968856829	-0.728051258689338\\
-2.69965514074735	-0.697816255793652\\
-2.74310677989146	-0.669949578355601\\
-2.77819289831537	-0.644162341846309\\
-2.80683000624538	-0.620170316931991\\
-2.83041747520835	-0.597710497828228\\
-2.8499897378227	-0.57654625934367\\
-2.86632008078914	-0.55646714197932\\
-2.87999208997297	-0.537286293160247\\
-2.89144933133931	-0.518837048204695\\
-2.90103022890587	-0.500969346659074\\
-2.90899273449339	-0.483546279852304\\
-2.91553183885285	-0.466440864345504\\
-2.9207919572066	-0.449533035674002\\
-2.9248755439451	-0.432706806771718\\
-2.92784882832088	-0.415847509214427\\
-2.9297452356072	-0.398839018717003\\
-2.9305668130403	-0.381560851171685\\
-2.93028377937382	-0.36388499688355\\
-2.92883213178227	-0.345672334683861\\
-2.9261090476084	-0.326768430524965\\
-2.92196558218108	-0.306998472609269\\
-2.91619585082817	-0.286161021550744\\
-2.90852144106672	-0.264020152464737\\
-2.89856915113163	-0.24029542772493\\
-2.88583917141691	-0.214648955648693\\
-2.86965932428903	-0.186668556753896\\
-2.8491186548788	-0.155845786349874\\
-2.8229700613926	-0.121547306806751\\
-2.78948610531244	-0.0829780416000335\\
-2.74624382975354	-0.0391351503135862\\
-2.6898027911544	0.0112456755571244\\
-2.61522702752864	0.0697429611071754\\
-2.5153954382363	0.13827444486632\\
-2.38007896894432	0.21902535174251\\
-2.19493334032315	0.314123658023469\\
-1.9410880248057	0.424762276878524\\
-1.59722619747505	0.549365716412347\\
-1.14760097929895	0.6808116647911\\
-0.597875475194193	0.804535971041456\\
0.0103610704461334	0.901750895964801\\
0.607930437625238	0.959080304254248\\
1.13228895415094	0.976232367083907\\
};
\addplot [color=mycolor1, forget plot]
  table[row sep=crcr]{%
1.31724795584908	1.04879793416756\\
1.70068896800859	1.03714926463591\\
1.99243412563981	1.00972335390571\\
2.21128528948877	0.975035371074232\\
2.37559068430626	0.938170427444379\\
2.50007029924285	0.901800397049627\\
2.59560899674705	0.867161038389809\\
2.66999240803846	0.834708218963402\\
2.72873547749773	0.804501169978652\\
2.77575356376531	0.776411417061493\\
2.81384956025627	0.750231993046598\\
2.84505236265008	0.725732440318642\\
2.87084839353232	0.702685120391255\\
2.89233964894694	0.680876613176355\\
2.91035210944293	0.660111518528291\\
2.92551071394318	0.640212499121538\\
2.93829169714277	0.621018557932336\\
2.9490594557421	0.602382565596242\\
2.95809270468511	0.58416853462776\\
2.96560310047765	0.56624886235795\\
2.97174845934883	0.548501617650293\\
2.97664199416488	0.530807866768432\\
2.98035851181535	0.513048988155423\\
2.98293817100018	0.495103896110834\\
2.98438814435474	0.476846068882062\\
2.9846823200263	0.458140250826432\\
2.98375898607868	0.438838666034884\\
2.98151623928633	0.41877653748519\\
2.9778046186795	0.397766646238988\\
2.9724161465914	0.375592583047378\\
2.96506851384193	0.352000231958747\\
2.95538249317645	0.326686872401422\\
2.94284968615133	0.299287082441922\\
2.92678621646407	0.269354364850047\\
2.90626568799413	0.236337108267457\\
2.88002119272049	0.199547193952261\\
2.84630076925584	0.158119443323986\\
2.80265275945175	0.110960658291323\\
2.74560663983224	0.0566894522876568\\
2.67020291053394	-0.00642470003833229\\
2.56932201190236	-0.0804962544596073\\
2.43279967300743	-0.167962268872052\\
2.24648657562714	-0.271266028968002\\
1.9919207905751	-0.391979782927913\\
1.64840702094397	-0.528949767320616\\
1.20066961434231	-0.675428087235624\\
0.65373091120456	-0.816901827471298\\
0.0466851295588091	-0.933778186550022\\
-0.554257251248188	-1.01071739348441\\
-1.08724660964121	-1.04507959019524\\
-1.52142481475842	-1.04562549335195\\
-1.85690865491003	-1.02478708833223\\
-2.10975536194308	-0.992899466037865\\
-2.29924077089661	-0.956664390850311\\
-2.44204554602821	-0.919821926019723\\
-2.55090826368052	-0.884223183205733\\
-2.63505602361024	-0.850651395512304\\
-2.70104332736356	-0.819329232452809\\
-2.75351347363108	-0.790203558051256\\
-2.79577501936664	-0.76309717269976\\
-2.83020913831105	-0.737786709899157\\
-2.85854980616021	-0.714040992926969\\
-2.88207503608867	-0.691638662656213\\
-2.90173766121912	-0.670375131436804\\
-2.9182553876661	-0.650064167795247\\
-2.93217336191435	-0.630536882903907\\
-2.94390805059626	-0.61163954606285\\
-2.95377827091342	-0.593230944566961\\
-2.9620272587906	-0.575179625835013\\
-2.96883837402682	-0.557361158923949\\
-2.97434618406172	-0.539655444784479\\
-2.97864408732369	-0.521944044947472\\
-2.98178923332578	-0.50410746231682\\
-2.98380520340743	-0.48602228163461\\
-2.98468268775975	-0.467558052713413\\
-2.98437819755078	-0.448573771047592\\
-2.98281065760069	-0.428913773217576\\
-2.9798555075506	-0.408402813770303\\
-2.97533566543103	-0.386840020298865\\
-2.96900833349175	-0.363991327101009\\
-2.96054608874203	-0.339579856150916\\
-2.9495099034091	-0.313273536888723\\
-2.93531053375228	-0.284669024188519\\
-2.91715286645675	-0.253270685689284\\
-2.89395496323989	-0.218463112035864\\
-2.86422917280108	-0.179475359715508\\
-2.82590609659725	-0.135335264769788\\
-2.77607276040422	-0.0848134410459427\\
-2.7105844092644	-0.0263608877946367\\
-2.62349944761116	0.041944146798071\\
-2.5062973391022	0.122395838482514\\
-2.34692584809889	0.217499779025785\\
-2.12902993939484	0.329424887102003\\
-1.83250714465007	0.458686424420379\\
-1.4379374353601	0.601721274169193\\
-0.938024113384351	0.74798013060162\\
-0.353923457943785	0.879607203169263\\
0.258945755246423	0.977741613950653\\
0.832029281899767	1.0328566526933\\
1.31724795584908	1.04879793416756\\
};
\addplot [color=mycolor1, forget plot]
  table[row sep=crcr]{%
1.46475086798591	1.12486267541903\\
1.81455064496186	1.11424173504341\\
2.08101811692771	1.0891848432148\\
2.28239477396513	1.05725633865747\\
2.43516101366466	1.02297096369503\\
2.55222786165662	0.988759043180935\\
2.6431073278131	0.955802910425065\\
2.7146373091078	0.924590286752199\\
2.77170413818777	0.895241476523484\\
2.81781236316433	0.867692314960797\\
2.85549699948198	0.841793134484645\\
2.88661151674617	0.817360852200531\\
2.91252656590965	0.794205432401846\\
2.93426699299933	0.772142515743081\\
2.95260677318198	0.750998663464851\\
2.96813529734919	0.73061271717463\\
2.98130405122357	0.710835161507174\\
2.9924597435533	0.691526494724127\\
3.00186794581836	0.672555128082024\\
3.00972997636107	0.653795067109678\\
3.01619486979958	0.63512347785137\\
3.02136766535384	0.616418154327055\\
3.02531482399341	0.597554849653668\\
3.02806727630516	0.578404395087712\\
3.02962136384206	0.558829498131974\\
3.02993773187217	0.538681075652315\\
3.02893803222613	0.517793934940117\\
3.02649907337657	0.495981559240114\\
3.0224437786017	0.473029678094332\\
3.0165279391859	0.4486881991649\\
3.00842121624705	0.422660937455598\\
2.9976800587588	0.394592389157719\\
2.98370902355612	0.364050550863591\\
2.96570518575104	0.330504479387123\\
2.94257758377414	0.293294950382676\\
2.91282947836327	0.251596311720892\\
2.8743850248728	0.204367746024086\\
2.82433329036807	0.150293442953563\\
2.75855200597052	0.0877155387590932\\
2.67116583336082	0.0145754690441363\\
2.55380702427253	-0.0715921184010218\\
2.39473271379309	-0.173506451243602\\
2.17814159007815	-0.293605490771841\\
1.88475266322618	-0.432752885013195\\
1.49594165548874	-0.587840504337622\\
1.00421008637066	-0.748816520881164\\
0.428204497449267	-0.897960863886098\\
-0.181069342161435	-1.01542826939592\\
-0.758071027725421	-1.08942430520555\\
-1.25383701882624	-1.12144949613573\\
-1.65114711302072	-1.12196791913176\\
-1.95706424890432	-1.10296348206271\\
-2.18873552451611	-1.0737368824625\\
-2.36396063075085	-1.04021870875464\\
-2.49749047958932	-1.00576048059793\\
-2.60046006633146	-0.972081963203389\\
-2.68094737638745	-0.93996520250237\\
-2.74473272823448	-0.909684049981042\\
-2.79595080402652	-0.881250064378457\\
-2.83757850259748	-0.854547535635447\\
-2.87178039260318	-0.829405541040784\\
-2.90014809676444	-0.805635276710233\\
-2.92386524475683	-0.783048536119421\\
-2.94382142623585	-0.761466065933136\\
-2.96069143501722	-0.74072055377238\\
-2.97499083610984	-0.720656820926039\\
-2.98711525579079	-0.701130600286\\
-2.99736835333593	-0.682006626587287\\
-3.00598180523958	-0.66315640643376\\
-3.01312954511842	-0.644455835646819\\
-3.01893776823853	-0.625782717974056\\
-3.02349170434944	-0.607014171680475\\
-3.02683980356726	-0.588023866192023\\
-3.0289957113926	-0.568678996357998\\
-3.02993819078351	-0.548836868510815\\
-3.02960895059597	-0.528340934081635\\
-3.02790813295673	-0.507016057545366\\
-3.02468696775183	-0.484662740045331\\
-3.01973678363577	-0.461049931197972\\
-3.01277312067384	-0.435905940663681\\
-3.00341304428323	-0.408906797741841\\
-2.99114279818484	-0.379661190623549\\
-2.97527147807427	-0.347690840003647\\
-2.95486418674673	-0.312404833834234\\
-2.92864474542825	-0.273066128691932\\
-2.89485294071803	-0.228748292490477\\
-2.85103389105989	-0.178281118439137\\
-2.79372728790955	-0.120186213158118\\
-2.71801409109878	-0.0526109543084696\\
-2.6168773368519	0.0267122368710863\\
-2.48037329926872	0.120411627235978\\
-2.29477385383028	0.231169688316104\\
-2.04231301726284	0.360863536376334\\
-1.7031762285242	0.50873932397908\\
-1.26255393483827	0.668547632236554\\
-0.724234652787264	0.82617364302703\\
-0.12359505347142	0.961691386230613\\
0.477364392125094	1.05806587603998\\
1.01792006988566	1.11014470800911\\
1.46475086798591	1.12486267541903\\
};
\addplot [color=mycolor1, forget plot]
  table[row sep=crcr]{%
1.58697213121126	1.20336136317783\\
1.9076340390308	1.19362316957183\\
2.1529623596323	1.17054474461314\\
2.33998746848524	1.14088153826959\\
2.48338418761784	1.10869017780001\\
2.59449987839305	1.07621039709076\\
2.68170112752524	1.04458260799585\\
2.75104372054698	1.01432015572903\\
2.80689578120153	0.985592670342716\\
2.8524221421894	0.958388406892165\\
2.88993526501501	0.932604821861844\\
2.92114215149815	0.908098078821548\\
2.94731633975157	0.884709440025805\\
2.96941768474312	0.86227875839831\\
2.98817617409695	0.840650810103552\\
3.00415099310331	0.819677676168766\\
3.01777246146683	0.799218958659433\\
3.02937199965124	0.77914081640432\\
3.03920361601012	0.759314352276649\\
3.04745928077584	0.739613624834528\\
3.05427978771138	0.719913405723052\\
3.05976217492766	0.700086712141951\\
3.06396439936458	0.680002083025932\\
3.0669076774521	0.659520521792054\\
3.06857667662451	0.638491987024067\\
3.06891753693403	0.616751267716092\\
3.06783349085648	0.594113025247864\\
3.06517760298044	0.570365713580987\\
3.06074183339464	0.545263994757396\\
3.05424118955854	0.518519139618314\\
3.04529109862083	0.489786733138532\\
3.03337519605724	0.458650778994819\\
3.01779932079387	0.424603012444904\\
2.99762538192359	0.387015895099739\\
2.97157555727582	0.345107440606509\\
2.93789252757304	0.297895896533356\\
2.894134677273	0.244142888401816\\
2.83687646773538	0.182286152050274\\
2.76127581055905	0.110370308801734\\
2.66047170131774	0.0260025409344433\\
2.52481587585663	-0.0735981350095048\\
2.34109787068459	-0.191304729055157\\
2.09235121362609	-0.329248668687806\\
1.75971183833163	-0.48704866481498\\
1.32881027591122	-0.658999800722057\\
0.802036457999092	-0.831568463994606\\
0.210734221357192	-0.984824584999507\\
-0.38777641280512	-1.10035461917923\\
-0.934360738786419	-1.17053852025654\\
-1.39336140266842	-1.20022669008197\\
-1.75779374171375	-1.20070866807381\\
-2.03862293124132	-1.18325589981369\\
-2.25276893646319	-1.15622999995266\\
-2.41635313761951	-1.12492897948904\\
-2.54239200337834	-1.09239593812663\\
-2.64066508308718	-1.06024728385831\\
-2.71829858239608	-1.02926440743518\\
-2.78043476129419	-0.999762345382136\\
-2.83078849227627	-0.971805143099149\\
-2.87206179875188	-0.945327442609912\\
-2.90623884510072	-0.920201620544917\\
-2.93479211592366	-0.896274071926745\\
-2.95882589717626	-0.873384213184259\\
-2.97917639446744	-0.851373868808164\\
-2.99648202677968	-0.830091335297682\\
-3.01123315083976	-0.80939251750509\\
-3.02380748740073	-0.789140465686734\\
-3.03449549213506	-0.769204040001337\\
-3.04351854501094	-0.749456087159715\\
-3.05104190548527	-0.72977131651809\\
-3.05718374616062	-0.710023945644536\\
-3.062021132934	-0.690085111628488\\
-3.06559349611069	-0.669819992783676\\
-3.06790388643737	-0.649084542835442\\
-3.06891809699603	-0.627721697409026\\
-3.06856152709846	-0.605556863823372\\
-3.06671343951736	-0.582392443487743\\
-3.06319798543667	-0.558001054676279\\
-3.05777100043022	-0.532117013905994\\
-3.0501010498453	-0.504425486699422\\
-3.03974243419978	-0.474548522123226\\
-3.02609671962065	-0.442026930521274\\
-3.0083576305549	-0.406296649987609\\
-2.98543152971664	-0.366657904102167\\
-2.95582179344142	-0.322235192168796\\
-2.91745967145321	-0.271926286852041\\
-2.86745638659513	-0.214339737284079\\
-2.80174211919316	-0.147724783229074\\
-2.71455205963207	-0.0699093564136836\\
-2.59773493723347	0.0217100643328273\\
-2.43994355498867	0.130022399139329\\
-2.22603357434645	0.257683214119972\\
-1.93763912498983	0.405860982827894\\
-1.55696130812789	0.571904053643628\\
-1.07620703943236	0.74636557032752\\
-0.511274730013405	0.911924373514858\\
0.0916394559526066	1.04810571255123\\
0.670514117540115	1.14105418007271\\
1.17582961916804	1.18979941964901\\
1.58697213121126	1.20336136317783\\
};
\addplot [color=mycolor1, forget plot]
  table[row sep=crcr]{%
1.69154852785871	1.28370401175513\\
1.98672928216181	1.27473415003994\\
2.21390108424491	1.25335406547444\\
2.3886721671133	1.22562492412593\\
2.5240760054015	1.19521991116207\\
2.63011715172874	1.1642170823112\\
2.71419350409143	1.13371768638542\\
2.78169954664035	1.10425279247967\\
2.83656251862296	1.07603088498327\\
2.88165563630413	1.04908292777186\\
2.91909840184877	1.02334554935167\\
2.9504695155905	0.998707999571044\\
2.97695656811281	0.975038180312408\\
2.99946129654955	0.952196653784043\\
3.01867392699092	0.930043754760351\\
3.03512601603836	0.908442745974919\\
3.04922825187172	0.887260696633115\\
3.06129762746649	0.866368037516756\\
3.07157699757694	0.845637323770712\\
3.08024907326989	0.824941487340937\\
3.08744624698973	0.804151710236768\\
3.0932571761229	0.783134953971837\\
3.0977307141138	0.761751114076917\\
3.10087751639272	0.739849715679729\\
3.10266942737847	0.717266016282757\\
3.10303654479367	0.693816327060933\\
3.10186163011529	0.66929229710414\\
3.0989712573794	0.643453818580479\\
3.09412272589533	0.616020096217404\\
3.0869852483128	0.586658271738282\\
3.07711317949214	0.554968791982241\\
3.06390794727981	0.520466449283701\\
3.04656369490717	0.482555705357621\\
3.0239891727972	0.440498567965223\\
2.99469475660702	0.393373038710646\\
2.95662820936091	0.340020310101148\\
2.90693577599555	0.278980258156785\\
2.84161740355516	0.208419226490437\\
2.75504125426462	0.126065791553188\\
2.63929931271235	0.0291976991417976\\
2.4834665440008	-0.0852183864878517\\
2.27307022427829	-0.220025901007928\\
1.99064539634526	-0.376669674047728\\
1.61917471026681	-0.552941606294356\\
1.15055721309653	-0.740034199829811\\
0.598048773833731	-0.921163600349624\\
0.00314733408814727	-1.07549448712044\\
-0.575841737161633	-1.18736911049668\\
-1.08929438614472	-1.25336201237515\\
-1.51347869615247	-1.28082049737753\\
-1.84866477490039	-1.2812637280361\\
-2.10779694582459	-1.26515094305336\\
-2.3069495686061	-1.24000737687748\\
-2.46060174261895	-1.21059800012478\\
-2.58025054920589	-1.17970719790656\\
-2.67452367686858	-1.1488614252525\\
-2.74974342309568	-1.11883741880872\\
-2.8105111479618	-1.08998157033276\\
-2.86018303277302	-1.06240009352769\\
-2.90122398017439	-1.03606910801023\\
-2.93546093651639	-1.01089725903163\\
-2.964261486689	-0.986760774587864\\
-2.98865933226719	-0.963522655409174\\
-3.00944268677859	-0.941042759048854\\
-3.02721690008306	-0.919182661092823\\
-3.04244911808689	-0.897807516718047\\
-3.05550031833885	-0.876786189958432\\
-3.06664836696297	-0.855990364817512\\
-3.07610458458484	-0.835293028743695\\
-3.08402551471939	-0.814566525998151\\
-3.09052103557451	-0.79368025920963\\
-3.09565956145848	-0.772498038722634\\
-3.09947078479444	-0.750875021230973\\
-3.10194617235881	-0.728654128915337\\
-3.10303721737287	-0.705661788879955\\
-3.10265123399875	-0.681702772720765\\
-3.10064423317521	-0.656553840372643\\
-3.09681010303249	-0.629955793032172\\
-3.09086488530946	-0.601603407626702\\
-3.08242432191236	-0.571132549326719\\
-3.07097193975157	-0.538103528363932\\
-3.05581359250599	-0.501979477312295\\
-3.03601235654665	-0.462098187829636\\
-3.01029466309495	-0.417635526260147\\
-2.97691413610257	-0.367558441195495\\
-2.93345344412972	-0.310566192624902\\
-2.87653678974171	-0.245021002147175\\
-2.80141888517497	-0.168876660325787\\
-2.70141956169065	-0.0796317756926746\\
-2.56721359818832	0.0256249862756282\\
-2.38613071547999	0.149929294633429\\
-2.14200616450009	0.295636668224313\\
-1.81689468557862	0.46271509430863\\
-1.3968259799178	0.646009298609677\\
-0.882832291688758	0.832645321473129\\
-0.302368412278713	1.00289687492031\\
0.29193820929031	1.13726589176048\\
0.842910913972665	1.22582272036213\\
1.31296805875141	1.27120728874838\\
1.69154852785871	1.28370401175513\\
};
\addplot [color=mycolor1, forget plot]
  table[row sep=crcr]{%
1.7832981678353	1.36546589275169\\
2.05592568115162	1.3571744194569\\
2.26714164491445	1.33728673652804\\
2.43111853478521	1.31126156725633\\
2.55943271347477	1.28244136021363\\
2.66093354729971	1.25276022517939\\
2.74218962233567	1.22327936216145\\
2.80802606279813	1.19453958765605\\
2.86198660051068	1.16677896496198\\
2.90668763683666	1.14006289696381\\
2.94407647731309	1.11436054257094\\
2.97561575220362	1.08958916115493\\
3.00241420566575	1.06563949952862\\
3.02531948618789	1.04239000684265\\
3.04498425031883	1.0197144567338\\
3.06191351815595	0.997485660984367\\
3.07649877708056	0.975576844041152\\
3.08904261974966	0.95386158977852\\
3.09977651875239	0.932212879917087\\
3.10887352086999	0.910501505895431\\
3.11645707010401	0.888593987389956\\
3.12260675718842	0.866350032379446\\
3.12736148556441	0.84361950212847\\
3.13072029732992	0.820238784686038\\
3.13264088455171	0.796026421968094\\
3.13303559280109	0.770777769825975\\
3.13176447555128	0.744258389841893\\
3.12862464535523	0.716195767549337\\
3.12333474415734	0.686268814798393\\
3.1155127546208	0.654094433393716\\
3.10464449980951	0.619210182674226\\
3.09003888625681	0.581051800749411\\
3.07076402681151	0.538923992473511\\
3.04555554798029	0.491962585008751\\
3.01268429798848	0.439086064580252\\
2.96976506994043	0.37893515870206\\
2.91348118717393	0.309801736731583\\
2.83919435906196	0.229555616004797\\
2.74041386938704	0.135595692071535\\
2.60813878932407	0.0248912135021348\\
2.43022180279671	-0.105743493774361\\
2.19124961386915	-0.258874318262122\\
1.87411209379186	-0.434804268222938\\
1.46518116380506	-0.628916844714507\\
0.964245059921325	-0.829016773640376\\
0.395207777190675	-1.01569772268623\\
-0.193748734909379	-1.1686142836128\\
-0.747622298683288	-1.2757260230298\\
-1.22738617820585	-1.33743352552652\\
-1.61919514519511	-1.36280865849501\\
-1.92824504599151	-1.36321391987884\\
-2.16827266488954	-1.34828035900592\\
-2.35424440438222	-1.32479177202571\\
-2.49912053823983	-1.29705421238368\\
-2.61307727724595	-1.26762654584206\\
-2.7037556868631	-1.23795184232903\\
-2.77678844889193	-1.20879671834807\\
-2.83630900148409	-1.18052984170452\\
-2.88535963560319	-1.15329067060285\\
-2.92619505892335	-1.12708932719676\\
-2.96050096090471	-1.10186489456276\\
-2.98954936666697	-1.07751903751993\\
-3.01430876229406	-1.0539350695173\\
-3.03552236243188	-1.03098844046375\\
-3.05376401994576	-1.00855215247121\\
-3.06947838995699	-0.98649915687812\\
-3.08300991163942	-0.964702930102649\\
-3.09462374740807	-0.943036918843076\\
-3.10452083500761	-0.921373240822968\\
-3.11284852374633	-0.899580840330465\\
-3.11970778215795	-0.877523178040283\\
-3.12515761091966	-0.855055451989273\\
-3.12921702232672	-0.832021282451107\\
-3.13186471906918	-0.808248735438497\\
-3.13303639024817	-0.783545498449622\\
-3.13261931307637	-0.757692949906779\\
-3.13044367314647	-0.730438772495175\\
-3.1262696548114	-0.701487641376272\\
-3.11976885109067	-0.670489360837329\\
-3.11049781965361	-0.63702361653161\\
-3.09786054958415	-0.600580246600419\\
-3.08105502897536	-0.560533616143728\\
-3.05899676892009	-0.516109340945298\\
-3.03020872281921	-0.466341368002129\\
-2.99266220849364	-0.410017606742598\\
-2.94354713401342	-0.345613712643242\\
-2.87894316526232	-0.271219097624817\\
-2.7933613052015	-0.18447081327201\\
-2.67914274745981	-0.0825377845677898\\
-2.5257783825343	0.0377458731415323\\
-2.31943501649625	0.179398198627669\\
-2.04347928473513	0.344126308788451\\
-1.68159158956786	0.530151902243493\\
-1.22537153655025	0.729304947360204\\
-0.685671303026336	0.925396871033158\\
-0.099644329037568	1.09741544503237\\
0.478086993696043	1.22814867876394\\
0.998202340115935	1.3118123673045\\
1.43427748430264	1.35394217752062\\
1.78329816783529	1.36546589275169\\
};
\addplot [color=mycolor1, forget plot]
  table[row sep=crcr]{%
1.86527941841213	1.44825240370401\\
2.11771325155239	1.44056775786807\\
2.3146438691887	1.42201660055715\\
2.46888019254694	1.39752963749764\\
2.59071684335125	1.37015790542439\\
2.68800519087505	1.34170343400817\\
2.76659677043846	1.31318515783504\\
2.83082010235443	1.2851462271798\\
2.88388025783308	1.25784608835794\\
2.92816364116387	1.23137735726302\\
2.96546102639441	1.20573592840368\\
2.99712789566332	1.1808626349622\\
3.02419904593052	1.15666774659049\\
3.04747055975572	1.13304512390001\\
3.06755865595758	1.10988011611193\\
3.08494214443492	1.08705364341692\\
3.09999317660819	1.06444391896843\\
3.11299954624969	1.0419266711386\\
3.12418079062586	1.01937436463494\\
3.13369963737515	0.996654693924047\\
3.1416698415423	0.973628477145177\\
3.14816109038466	0.950146978784336\\
3.15320137109218	0.926048613223259\\
3.1567769610436	0.901154914561983\\
3.15882998113876	0.875265590354205\\
3.15925322167056	0.848152399268906\\
3.15788167619843	0.819551496607721\\
3.15447986339269	0.789153767778748\\
3.14872352711161	0.756592507909527\\
3.14017360532388	0.721427595068109\\
3.12823933821891	0.683125037112243\\
3.1121258845059	0.641030451681497\\
3.09075960873231	0.594334702650082\\
3.06268100836197	0.542029687526621\\
3.02589079683469	0.482852475934354\\
2.97762897175065	0.415217431606972\\
2.91406095861527	0.33714041809399\\
2.82984383017217	0.246170597184527\\
2.71756319068948	0.139371466952623\\
2.56710335408355	0.0134469599782818\\
2.36521648184592	-0.134794739196671\\
2.09600634147434	-0.307322133291899\\
1.74374612949788	-0.502780130865966\\
1.29973652733544	-0.713622372095883\\
0.772685574497584	-0.924268036294543\\
0.195840532604639	-1.1136404366881\\
-0.379672184980449	-1.26317811794792\\
-0.905127232671636	-1.36486458989047\\
-1.35185024182953	-1.42235333482624\\
-1.71378205463731	-1.44579970039503\\
-1.99931648626235	-1.44616912203778\\
-2.2222613254083	-1.43229003289289\\
-2.39639418110116	-1.41028844269828\\
-2.53330505257749	-1.38406879379923\\
-2.64202280717088	-1.35598825095052\\
-2.72933700426184	-1.32740987741486\\
-2.80028183599667	-1.29908456453855\\
-2.85858020554726	-1.27139510629733\\
-2.90699530226382	-1.24450638015938\\
-2.947592193925	-1.21845598805502\\
-2.98192717794727	-1.19320836075959\\
-3.01118336923205	-1.16868675508286\\
-3.03626759247698	-1.14479193034465\\
-3.05787979721232	-1.1214127859477\\
-3.07656301310694	-1.09843211952335\\
-3.09273946787627	-1.0757293913073\\
-3.10673677727398	-1.05318161540029\\
-3.11880691461837	-1.03066303560194\\
-3.12913982678066	-1.00804395896556\\
-3.13787297099832	-0.985188940451968\\
-3.14509762024053	-0.961954392812934\\
-3.15086246552017	-0.938185609944088\\
-3.15517478867469	-0.913713121998979\\
-3.15799925541525	-0.888348234484449\\
-3.1592541569643	-0.86187753192624\\
-3.15880468027373	-0.83405604094017\\
-3.1564524775992	-0.804598638814661\\
-3.1519203913931	-0.773169152187795\\
-3.14483060713148	-0.739366405369039\\
-3.13467366330692	-0.70270623935128\\
-3.12076451078705	-0.662598226770153\\
-3.1021799921693	-0.618315471439615\\
-3.07766944933406	-0.568955574380272\\
-3.04552637025706	-0.513390775284264\\
-3.00340387231053	-0.450205952538559\\
-2.94805082150653	-0.377625791225276\\
-2.87494099838897	-0.293439685493208\\
-2.77777324861214	-0.194950415666673\\
-2.64785882128024	-0.0790107997470756\\
-2.47353686337519	0.0577132524403142\\
-2.24007102783224	0.217997577923155\\
-1.93107839760373	0.402477340836931\\
-1.53320039607419	0.607063582008598\\
-1.04509831762917	0.820231299118718\\
-0.487511116244679	1.02294809819989\\
0.0954524566832367	1.19419249706947\\
0.651027569613964	1.32000441649981\\
1.13916833508337	1.39857409095482\\
1.54310927986971	1.43761596787514\\
1.86527941841213	1.44825240370401\\
};
\addplot [color=mycolor1, forget plot]
  table[row sep=crcr]{%
1.93943026851774	1.53163957203499\\
2.17360049581859	1.52450375594013\\
2.35755267889386	1.507167450353\\
2.50283805042945	1.48409467494661\\
2.61862701362222	1.45807589297092\\
2.711904720556	1.43078980113568\\
2.78789794972007	1.40321063191535\\
2.85049812106648	1.37587728963447\\
2.90260827294845	1.34906340471532\\
2.94640665985586	1.32288242012228\\
2.98354018983974	1.29735179089412\\
3.01526428573785	1.27243191584892\\
3.04254355050506	1.24804954958427\\
3.06612428546851	1.22411167389182\\
3.08658691232949	1.20051347002365\\
3.10438401945045	1.17714260332082\\
3.1198680486297	1.15388115911178\\
3.1333114246421	1.13060603329294\\
3.14492107221718	1.10718824793783\\
3.15484865623774	1.08349144998392\\
3.16319744069907	1.05936970978761\\
3.17002633210458	1.03466463528547\\
3.17535141046174	1.00920173680874\\
3.17914502232256	0.982785903580248\\
3.18133228639511	0.955195775051067\\
3.1817846142805	0.926176699198573\\
3.18030954356728	0.895431856345077\\
3.17663577468156	0.862610980826365\\
3.17039173647708	0.827295923087385\\
3.16107519214855	0.788982051656446\\
3.14801021191297	0.747054194608803\\
3.13028610628902	0.700755480999907\\
3.10667040258241	0.649147136556571\\
3.0754844031259	0.591057219627259\\
3.03442515971809	0.525016963546477\\
2.98031231722511	0.449186006034351\\
2.9087346617035	0.361274943673765\\
2.81357717269425	0.258490731375111\\
2.68644585098409	0.137567426068143\\
2.51612277515245	-0.00498383250365952\\
2.28846610535659	-0.172159159422162\\
1.98770377648844	-0.364935344947061\\
1.60066040195149	-0.579748947424175\\
1.1249587311468	-0.805731300108648\\
0.578529135762317	-1.02424182226127\\
0.0018151964207621	-1.21369394943156\\
-0.554595484228764	-1.35836326169271\\
-1.04995644620431	-1.4542802652993\\
-1.46493689603643	-1.50770529730677\\
-1.79936987895032	-1.52937259081181\\
-2.06360046551447	-1.52970881922359\\
-2.27107724781439	-1.51678493296196\\
-2.43440000580924	-1.4961417936104\\
-2.56393648952145	-1.47132809713904\\
-2.66771714564749	-1.44451759463789\\
-2.75179187549888	-1.41699534929173\\
-2.82067103513665	-1.38949139261048\\
-2.87771388935971	-1.36239547667343\\
-2.92543291381231	-1.33589101005423\\
-2.96572023314904	-1.3100372813353\\
-3.00001227802128	-1.28481949674121\\
-3.02940845809071	-1.26017900999828\\
-3.05475657477219	-1.23603138924932\\
-3.07671444666484	-1.21227698949568\\
-3.09579454834497	-1.18880686910994\\
-3.11239646055606	-1.16550577220273\\
-3.12683048862806	-1.14225321584285\\
-3.13933478547637	-1.11892329975785\\
-3.15008759391245	-1.09538359128941\\
-3.15921570692029	-1.07149326640162\\
-3.1667998656831	-1.04710056941051\\
-3.1728775236196	-1.02203956524038\\
-3.17744316287597	-0.996126082116127\\
-3.18044612691394	-0.969152667859541\\
-3.18178570066164	-0.940882299647012\\
-3.18130289768291	-0.911040485996277\\
-3.17876806450486	-0.879305271180267\\
-3.17386293502143	-0.845294485638946\\
-3.16615509087293	-0.808549370543012\\
-3.15506180294167	-0.768513432790178\\
-3.13979879576737	-0.724505062774383\\
-3.11930738664225	-0.675682109879071\\
-3.09215045321041	-0.620996384380717\\
-3.05636355479973	-0.559136266347788\\
-3.00924235488592	-0.488457035834185\\
-2.94704246831324	-0.406902965112206\\
-2.86456748527221	-0.311936416048813\\
-2.75463822885288	-0.200514624736791\\
-2.6075037101399	-0.0692071419161662\\
-2.41043964816916	0.0853601783129438\\
-2.14818441113122	0.265428168733131\\
-1.80548913225713	0.470070070247356\\
-1.37333584735421	0.692353225606642\\
-0.858513701023624	0.917297761136096\\
-0.29079596436584	1.12382307127971\\
0.281914664267367	1.29216766341901\\
0.811647335494608	1.41220223974685\\
1.26785920299175	1.48566882945716\\
1.64171802014129	1.52181328657332\\
1.93943026851773	1.53163957203499\\
};
\addplot [color=mycolor1, forget plot]
  table[row sep=crcr]{%
2.00696136401616	1.61515115761543\\
2.2244763751215	1.60851615796231\\
2.39650471451627	1.59229668680998\\
2.53345495378374	1.57054151686175\\
2.64351148367024	1.54580578256746\\
2.73290434414652	1.51965195360396\\
2.80631279325486	1.49300744753006\\
2.86724111811561	1.46640128633536\\
2.91832130806057	1.44011502928095\\
2.96154209634674	1.41427730154465\\
2.99841739547142	1.38892246172996\\
3.03010867183627	1.36402680232611\\
3.057513552894	1.3395307305017\\
3.08133005773032	1.31535217735229\\
3.10210330264744	1.29139447609956\\
3.12025956983251	1.26755070454325\\
3.1361311872701	1.24370571539386\\
3.14997463568489	1.219736596381\\
3.16198356202128	1.19551199641695\\
3.17229784946792	1.17089055449665\\
3.1810095042562	1.14571853095339\\
3.18816581973541	1.11982663858188\\
3.19377003078697	1.09302598566912\\
3.19777944566942	1.06510296096169\\
3.20010080960119	1.03581280137891\\
3.20058238477912	1.00487147697869\\
3.19900188865654	0.971945394114084\\
3.1950489674993	0.936638245959665\\
3.18830022702405	0.898474118681664\\
3.17818389819098	0.856875683725423\\
3.16392984492455	0.811135976447896\\
3.14449863349101	0.760381917347674\\
3.11848055397372	0.703527498527821\\
3.08395162400545	0.639214758512009\\
3.03826882809506	0.565742073837507\\
2.97778233785116	0.480983635635591\\
2.89744247597732	0.382314922117848\\
2.79029578776044	0.266583707551525\\
2.64692727178493	0.130216750300535\\
2.45507544953773	-0.0303578786756252\\
2.20001268640501	-0.217674749787857\\
1.86685064606414	-0.431254689935425\\
1.44620976602339	-0.664782761045597\\
0.943127699983683	-0.903873517789795\\
0.38427994838395	-1.12747132535379\\
-0.185353231066425	-1.31470988115095\\
-0.718599426479986	-1.45343541687645\\
-1.18333170544918	-1.54346315654748\\
-1.5681980712356	-1.59302554230365\\
-1.87733032808084	-1.61305360077976\\
-2.12214127410076	-1.61335939866085\\
-2.31547278170568	-1.60130961030503\\
-2.46880280663759	-1.58192293699056\\
-2.59141439882153	-1.55843016806524\\
-2.69046679613017	-1.53283655796367\\
-2.77136399372475	-1.50635073996917\\
-2.83815504437088	-1.47967750060326\\
-2.89387509055317	-1.45320736984484\\
-2.94081000509277	-1.42713625155903\\
-2.98069325725668	-1.40153995435671\\
-3.01484960365108	-1.37642030837201\\
-3.04429922644163	-1.35173353392588\\
-3.06983315014823	-1.32740752815835\\
-3.09206798947786	-1.3033521953867\\
-3.11148582640992	-1.27946536477639\\
-3.12846332614156	-1.25563585908362\\
-3.14329298033504	-1.2317446693827\\
-3.15619849444608	-1.20766480801879\\
-3.16734571270759	-1.18326016620529\\
-3.17685002186825	-1.15838353860248\\
-3.18478083537304	-1.13287386037785\\
-3.19116349026798	-1.10655261035256\\
-3.19597865600009	-1.07921925153787\\
-3.19915912844725	-1.05064549600183\\
-3.20058363534984	-1.0205680844799\\
-3.20006697823932	-0.988679652526426\\
-3.19734543923824	-0.954617103743766\\
-3.19205583125112	-0.917946715583595\\
-3.18370578514848	-0.878144954382671\\
-3.17163173039152	-0.8345736703227\\
-3.15493937296599	-0.786447997426901\\
-3.13241909538237	-0.732794970696447\\
-3.10242537324393	-0.672400795585811\\
-3.06270493050221	-0.603745367988815\\
-3.01015344216256	-0.524925208081585\\
-2.9404774856664	-0.433572977477883\\
-2.84774445136671	-0.326798395754649\\
-2.7238374474047	-0.201211234613841\\
-2.55793839316163	-0.0531560686324172\\
-2.33641919046957	0.120602410235937\\
-2.04400323109662	0.321404499629473\\
-1.66760176663596	0.546225264813016\\
-1.20384676724017	0.784847999392526\\
-0.66815026467459	1.01902764897822\\
-0.0976764318487073	1.22667643555794\\
0.459022871868458	1.3904123014413\\
0.960698937316442	1.50414893360095\\
1.38575196981399	1.57262394183115\\
1.73160305051423	1.60606634792779\\
2.00696136401616	1.61515115761543\\
};
\addplot [color=mycolor1, forget plot]
  table[row sep=crcr]{%
2.06860660762213	1.6982564075881\\
2.27083390287344	1.69208158484801\\
2.43181623095735	1.67689744800409\\
2.56093236612886	1.65638138520399\\
2.66550083508625	1.63287461855828\\
2.75109090401043	1.60782966645149\\
2.82189989396105	1.58212559838171\\
2.88108778101375	1.55627688899532\\
2.93104242025631	1.53056767567032\\
2.9735794665914	1.50513681595202\\
3.01008960653438	1.48003139638351\\
3.04164595097124	1.45524024098408\\
3.06908218678636	1.43071476405002\\
3.09304952831468	1.4063817792773\\
3.11405833439906	1.3821511459237\\
3.13250858656795	1.35792004598616\\
3.14871219749462	1.3335750035248\\
3.16290923474801	1.30899232412895\\
3.17527950851923	1.28403735238804\\
3.18595050823006	1.25856275768579\\
3.19500232471316	1.23240592553061\\
3.20246991898054	1.20538542817751\\
3.20834286164587	1.17729645745394\\
3.21256243981641	1.14790501164691\\
3.21501578237673	1.11694052610887\\
3.21552635813266	1.08408651333649\\
3.21383981384117	1.04896862160267\\
3.20960358553135	1.01113932017236\\
3.20233795866104	0.970058163217434\\
3.19139515971534	0.925066269822954\\
3.17590148070908	0.875353300549936\\
3.15467517274135	0.819914882186446\\
3.12610969095244	0.757498330656529\\
3.08800775922697	0.686535139968228\\
3.03734714023266	0.605061175825374\\
2.96995616008087	0.510632303323317\\
2.8800827372595	0.400259326796594\\
2.75987252483657	0.270420895487558\\
2.59886936571566	0.117280023737274\\
2.38388581270329	-0.0626631847495668\\
2.10003245751582	-0.271146178427355\\
1.73420453571025	-0.505713869055155\\
1.28204877937458	-0.756817313998955\\
0.756667781044695	-1.00661495204683\\
0.1922733449297	-1.23255116919391\\
-0.364437061939789	-1.41564239051156\\
-0.871800672737549	-1.54769918540689\\
-1.30616008159199	-1.63187423596462\\
-1.66268725551021	-1.67779647802568\\
-1.94852783530275	-1.69631403914339\\
-2.17554581779588	-1.69659209740831\\
-2.35584395382825	-1.6853483502887\\
-2.49985475124989	-1.66713418663842\\
-2.61590047339757	-1.64489452404878\\
-2.71037792764455	-1.62047891864067\\
-2.78812467574338	-1.59502118880521\\
-2.85278177135636	-1.56919735303032\\
-2.90709431663539	-1.54339350921435\\
-2.95314268875547	-1.51781282664167\\
-2.99251442364579	-1.49254305174153\\
-3.02643004080702	-1.46759887639381\\
-3.0558346556689	-1.44294839981031\\
-3.08146467381201	-1.41852950986356\\
-3.10389645791012	-1.39425983034538\\
-3.12358193679098	-1.37004250838304\\
-3.14087468969792	-1.34576925536863\\
-3.15604899636746	-1.32132151185452\\
-3.16931359396485	-1.2965702596509\\
-3.18082134004983	-1.2713747764242\\
-3.19067558088456	-1.24558047154152\\
-3.19893371714453	-1.21901582613657\\
-3.20560820661074	-1.19148836502251\\
-3.21066501481064	-1.16277949856979\\
-3.21401929166534	-1.13263797735477\\
-3.21552778542046	-1.10077159067195\\
-3.21497716899955	-1.06683660114395\\
-3.21206700150163	-1.03042423025091\\
-3.20638541338284	-0.991043282215236\\
-3.1973746947755	-0.948097708047462\\
-3.18428265137816	-0.900857571416836\\
-3.16609369567306	-0.848421520785672\\
-3.14143095431247	-0.789668619176481\\
-3.10841702795366	-0.723197550674446\\
-3.06447656286716	-0.647252563797837\\
-3.00605961778449	-0.559639709938102\\
-2.9282649168792	-0.457647549503845\\
-2.82435759990059	-0.338010483311221\\
-2.68523395877069	-0.197001944847137\\
-2.49904103887176	-0.0308321456585343\\
-2.25149282157391	0.163357481220727\\
-1.92794704359109	0.385569683479203\\
-1.5185982643987	0.630131842732555\\
-1.02683517234253	0.883261618608591\\
-0.476530090381141	1.12394512252728\\
0.0899721270408058	1.33026131171234\\
0.626255381738577	1.48807579992018\\
1.09878526349904	1.59525133632222\\
1.49389074129826	1.65892050951114\\
1.81374487298738	1.68985191952004\\
2.06860660762213	1.6982564075881\\
};
\addplot [color=mycolor1, forget plot]
  table[row sep=crcr]{%
2.12479054205496	1.78038134069434\\
2.31291656161758	1.77463153108908\\
2.46360610366426	1.76041283002961\\
2.58531390643501	1.7410691972296\\
2.68459681276724	1.71874663610339\\
2.76644398208781	1.69479360268786\\
2.83462713558066	1.67003996425573\\
2.8919991948091	1.64498193509097\\
2.94072770888523	1.61990174728722\\
2.98247012200416	1.59494421232009\\
3.01850295537561	1.57016544374821\\
3.04981632875234	1.54556375037565\\
3.0771830307112	1.52109910765688\\
3.10120907214989	1.49670526622932\\
3.12237077455906	1.47229705683084\\
3.1410420104021	1.44777449975262\\
3.15751415986059	1.42302472183765\\
3.17201058532855	1.39792229432523\\
3.18469687103318	1.37232834788918\\
3.19568766556411	1.34608864468236\\
3.20505065072055	1.31903065742965\\
3.21280790299115	1.29095960015476\\
3.21893468318756	1.2616532579701\\
3.22335545718919	1.23085536183297\\
3.22593668726858	1.1982671369072\\
3.22647560444775	1.16353650877242\\
3.22468373243725	1.12624426815375\\
3.22016332006521	1.08588625994957\\
3.2123739627508	1.04185036683857\\
3.20058542991375	0.993386703785246\\
3.18381089780502	0.939569061922303\\
3.16071221931497	0.87924535425997\\
3.12946538640294	0.810974926595604\\
3.08757007304783	0.732951832674429\\
3.03158313835534	0.642917163875137\\
2.95675589737613	0.538073721361076\\
2.85656905904271	0.415039483078827\\
2.72221216084583	0.26992395087485\\
2.54219661269735	0.0986971519466855\\
2.30259833799656	-0.101860489881717\\
1.988914542344	-0.332282706047911\\
1.590833038462	-0.587588465309065\\
1.11013273152794	-0.854634144926016\\
0.568083887970402	-1.11247084944338\\
0.00463629186577767	-1.33814162830174\\
-0.53445543917634	-1.51552597345371\\
-1.01432355133255	-1.6404783228451\\
-1.41910545379632	-1.71894465365314\\
-1.74911013215975	-1.76145651305498\\
-2.01349136861769	-1.7785815895257\\
-2.22414161555226	-1.778834444578\\
-2.39236488665411	-1.76833812291586\\
-2.52763190541574	-1.75122476343541\\
-2.637414179359	-1.73018111639456\\
-2.72743941803059	-1.70691245766884\\
-2.80204676583525	-1.68247968465132\\
-2.86451532605169	-1.65752740762065\\
-2.91733005804378	-1.63243301732164\\
-2.96238471945608	-1.6074024926176\\
-3.00113255564112	-1.58253150520725\\
-3.03469685032081	-1.55784423610873\\
-3.06395171936462	-1.53331792796646\\
-3.08958118440682	-1.50889827723035\\
-3.11212246190421	-1.48450889048134\\
-3.13199774897241	-1.46005683471198\\
-3.14953755486476	-1.43543555342924\\
-3.16499773010165	-1.41052593585689\\
-3.17857169594814	-1.38519601128781\\
-3.19039890215735	-1.35929952905157\\
-3.20057018425654	-1.33267353477869\\
-3.20913041012869	-1.30513493834513\\
-3.21607856529649	-1.27647596930937\\
-3.22136519811773	-1.24645831777657\\
-3.22488690195018	-1.21480565062894\\
-3.22647721994893	-1.18119406369038\\
-3.22589298028387	-1.14523986795464\\
-3.22279455206183	-1.10648390022672\\
-3.21671778036694	-1.06437128407666\\
-3.20703430696839	-1.01822524069456\\
-3.19289546606382	-0.967213175566549\\
-3.17315277669444	-0.910302912228278\\
-3.14624504277518	-0.846206799830856\\
-3.11003815033464	-0.773311962456872\\
-3.06159927613152	-0.689597282062443\\
-2.99688444241334	-0.592544204307182\\
-2.91032342357126	-0.479064016328777\\
-2.79431509370997	-0.345497907382085\\
-2.63873561239191	-0.187811386672545\\
-2.43077687368962	-0.0022104490946153\\
-2.15584030237021	0.213483429925804\\
-1.80070129893057	0.457436150053204\\
-1.35998856707431	0.720810101212633\\
-0.844619331415884	0.986195276047295\\
-0.286098770515454	1.23058945444942\\
0.270545939309583	1.43341860165406\\
0.783237520117522	1.58436045908561\\
1.22637739195548	1.68490620595368\\
1.5930049503019	1.74399940268459\\
1.88877271927099	1.77260242882456\\
2.12479054205496	1.78038134069434\\
};
\addplot [color=mycolor1, forget plot]
  table[row sep=crcr]{%
2.1757455399383	1.86092947588109\\
2.35081968343383	1.85557361347775\\
2.4918814069607	1.84225863208274\\
2.60655824336578	1.8240283222734\\
2.70073573799555	1.80285014400824\\
2.77889236357058	1.77997422491894\\
2.84442307195399	1.75618106444592\\
2.89990649664374	1.73194580475101\\
2.94731124659916	1.70754511313093\\
2.98815020451916	1.68312612891192\\
3.02359428704186	1.6587507692563\\
3.0545558807579	1.63442411379075\\
3.08175001423489	1.61011247806171\\
3.10573928683399	1.58575475408463\\
3.1269669296582	1.56126929111975\\
3.14578113135792	1.5365577545954\\
3.1624528498072	1.51150686354687\\
3.17718866730384	1.48598855560677\\
3.19013976072305	1.45985889229939\\
3.20140769291341	1.43295585051682\\
3.21104744439648	1.40509601867434\\
3.21906786098303	1.37607010761512\\
3.22542946417315	1.3456370815782\\
3.23003932927788	1.31351660071788\\
3.23274245037814	1.27937933174952\\
3.23330864339078	1.2428345148618\\
3.23141353732528	1.20341395994266\\
3.22661149742104	1.16055137087074\\
3.21829731089289	1.11355555601054\\
3.20565200704221	1.06157568726971\\
3.18756610020821	1.00355637933419\\
3.16253064865351	0.938180154314155\\
3.12848273803986	0.86379528743524\\
3.08258772403524	0.778329148142527\\
3.02093765895846	0.679193262511462\\
2.93814948064986	0.563201169264176\\
2.82687253540112	0.426552585933946\\
2.6772961224536	0.265001045394384\\
2.4769450478802	0.0744276433387469\\
2.21143140255826	-0.147838875464467\\
1.86731172809882	-0.400655940343124\\
1.43813288752117	-0.675973742953314\\
0.932668968336354	-0.956875982837009\\
0.379868962039353	-1.21994253881364\\
-0.176755743005223	-1.44298613217454\\
-0.69465073899282	-1.61347070951245\\
-1.1462983201646	-1.73111613176759\\
-1.52265813963826	-1.80408987649808\\
-1.82793983029911	-1.84342025116485\\
-2.07253666415207	-1.85926110979957\\
-2.26808650642126	-1.85949111268943\\
-2.42508096947947	-1.84969048308533\\
-2.55211304014538	-1.83361451173693\\
-2.65590054705101	-1.81371615279375\\
-2.74158252091892	-1.79156687627748\\
-2.81305835960616	-1.7681569245645\\
-2.87328539301801	-1.7440977085159\\
-2.92451497364393	-1.71975452013379\\
-2.96847135624044	-1.69533243440478\\
-3.0064843306294	-1.67093159810833\\
-3.03958667374776	-1.64658270937511\\
-3.06858558161	-1.62226969316863\\
-3.09411507998217	-1.59794405612958\\
-3.11667455875612	-1.57353377393765\\
-3.13665713747485	-1.54894852044796\\
-3.15437050281795	-1.52408237866027\\
-3.17005208114538	-1.49881474013717\\
-3.18387984234486	-1.47300981269993\\
-3.19597961166421	-1.44651495919136\\
-3.20642944476207	-1.41915794600328\\
-3.21526135973226	-1.39074306432466\\
-3.22246048722934	-1.36104598217436\\
-3.2279614680494	-1.32980707750475\\
-3.23164166782781	-1.29672287980269\\
-3.23331045712067	-1.26143509754435\\
-3.23269337812526	-1.22351651876016\\
-3.22940942610948	-1.18245282867667\\
-3.22293882882447	-1.1376190811391\\
-3.21257749123535	-1.08824918889927\\
-3.19737252597229	-1.03339639195026\\
-3.17603082171242	-0.971882330094916\\
-3.14678925483359	-0.902232374478002\\
-3.10723101369695	-0.8225959581938\\
-3.054028521735	-0.730654356005426\\
-2.98259295602703	-0.62352803360588\\
-2.88662269817132	-0.497717904034039\\
-2.75759083133511	-0.349160982008038\\
-2.58434167608116	-0.173565222105404\\
-2.35324979693792	0.0326908767820533\\
-2.04985818817793	0.270734018763883\\
-1.66325731093179	0.536350555225738\\
-1.19359855779719	0.817107480987676\\
-0.659653834827181	1.09216856819398\\
-0.0991467107972801	1.3375460775094\\
0.442715382529131	1.53508043637685\\
0.929717232907285	1.67851643798467\\
1.34384825322769	1.77250800435862\\
1.68360721994237	1.82727950637311\\
1.95708707666518	1.85372676041588\\
2.1757455399383	1.86092947588109\\
};
\addplot [color=mycolor1, forget plot]
  table[row sep=crcr]{%
2.22159676622086	1.93930856982817\\
2.38456348915545	1.93431869239652\\
2.51659941474423	1.92185152071188\\
2.62459243618175	1.90468011902611\\
2.71383537803312	1.88460851340267\\
2.7883560991336	1.86279418799137\\
2.85121539348207	1.83996875348906\\
2.90474663628562	1.81658429638293\\
2.95073880373225	1.79290902967154\\
2.99057285119828	1.76908942219381\\
3.02532224916302	1.7451904488079\\
3.05582684850611	1.72122159542384\\
3.08274717489659	1.69715354787979\\
3.10660441249865	1.67292872537141\\
3.127809887305	1.64846767764792\\
3.14668677481328	1.62367262888872\\
3.16348596110748	1.59842897211503\\
3.17839740498815	1.57260520022359\\
3.19155791838624	1.54605154151156\\
3.20305595341838	1.51859740878168\\
3.21293371855797	1.49004764492252\\
3.22118671215024	1.4601774351994\\
3.227760530906	1.42872564251366\\
3.23254455564754	1.39538619351755\\
3.23536180344011	1.35979698777029\\
3.23595382175374	1.32152560535341\\
3.23395892799329	1.28005083611285\\
3.22888128420597	1.23473873318586\\
3.22004712651428	1.18481150089176\\
3.2065427863991	1.12930708628236\\
3.18712675833704	1.06702695026784\\
3.1601048199004	0.996469414928267\\
3.12315314447153	0.915746872100953\\
3.07307027181208	0.822488485651201\\
3.0054376929282	0.713739031638763\\
2.91417909926155	0.585885607914834\\
2.7910507029328	0.434688447808658\\
2.62521279462519	0.255575061560006\\
2.4032948862381	0.044481007647173\\
2.11081059009323	-0.200384751809641\\
1.73616153777326	-0.475677649798832\\
1.27780703965718	-0.769790461836722\\
0.752028056882593	-1.06208929013006\\
0.194399578661223	-1.32756797160026\\
-0.350316856720065	-1.54593554218786\\
-0.844475622829684	-1.70866935710458\\
-1.26787394401055	-1.81898888345277\\
-1.61719807411477	-1.88673288114333\\
-1.89950611666071	-1.92310517359197\\
-2.12585585919881	-1.93776143897764\\
-2.30744776846662	-1.937970749563\\
-2.453976108978	-1.92881915090876\\
-2.57323699998103	-1.91372272803877\\
-2.67128003518321	-1.89492236072458\\
-2.75272515508203	-1.87386549694145\\
-2.82108293983526	-1.85147437242084\\
-2.87902424937034	-1.82832618816331\\
-2.92859056026966	-1.80477156292672\\
-2.97135227022156	-1.78101165755016\\
-3.00852588244833	-1.75714819470257\\
-3.04106019066816	-1.73321582747918\\
-3.06969959513531	-1.70920299863323\\
-3.09503068404653	-1.68506523966372\\
-3.11751656635825	-1.66073343720329\\
-3.13752218229105	-1.63611867783153\\
-3.1553328871236	-1.61111468952351\\
-3.17116792414091	-1.58559850903179\\
-3.1851899036903	-1.55942974263533\\
-3.19751103107226	-1.53244860321826\\
-3.20819653289102	-1.5044727669215\\
-3.21726548504536	-1.47529297523436\\
-3.22468901655486	-1.44466719677413\\
-3.2303856241741	-1.41231304342215\\
-3.23421305309102	-1.37789799510356\\
-3.235955841838	-1.34102681304299\\
-3.23530714524298	-1.30122529861235\\
-3.2318427684834	-1.25791926980496\\
-3.22498437026108	-1.21040727021301\\
-3.2139473887235	-1.15782510271635\\
-3.1976672355856	-1.09909984365504\\
-3.17469450283523	-1.03289070644772\\
-3.14304623510161	-0.957514408594192\\
-3.09999606227808	-0.870854547522102\\
-3.04178279133616	-0.770260111763278\\
-2.96322004042956	-0.652452227928976\\
-2.85721209015153	-0.513489278488834\\
-2.71425395122918	-0.348902459295057\\
-2.52217422497574	-0.15421940103064\\
-2.26673597004074	0.0737793821462952\\
-1.93418981696836	0.334730225925443\\
-1.51691609518709	0.621483026274706\\
-1.02151355087482	0.917723764851995\\
-0.474421557665009	1.19967161875598\\
0.0822682578220813	1.4434858734613\\
0.60542847324188	1.63428244591121\\
1.06555720769938	1.76985019514753\\
1.45151187150373	1.85746785386418\\
1.76607348285571	1.908183370787\\
2.0189513713455	1.93263715122637\\
2.22159676622086	1.93930856982817\\
};
\addplot [color=mycolor1, forget plot]
  table[row sep=crcr]{%
2.26242433708805	2.01495975701713\\
2.41414617789258	2.01031029942162\\
2.53771272321917	1.99863920907641\\
2.6393508791053	1.9824750944135\\
2.72382923025413	1.96347238228063\\
2.79477724470694	1.94270157923801\\
2.8549589261095	1.92084645526483\\
2.90648822956053	1.89833479687986\\
2.95099201579987	1.8754241778549\\
2.98973104374063	1.85225799090357\\
3.02368912523749	1.82890199066744\\
3.05363870720377	1.80536806266584\\
3.0801891695655	1.78162956416864\\
3.10382246092602	1.75763103475301\\
3.12491940707309	1.73329407085491\\
3.14377907181281	1.70852050670912\\
3.16063284991186	1.68319361541998\\
3.1756544582572	1.65717775520509\\
3.18896660690415	1.63031668309961\\
3.20064483218905	1.60243060602038\\
3.21071872453098	1.57331191267627\\
3.2191705545088	1.54271941140635\\
3.22593106467386	1.51037077388933\\
3.23087192170737	1.47593273907774\\
3.23379397786403	1.43900845148347\\
3.23441002415314	1.39912107769105\\
3.23232006341119	1.35569254873925\\
3.22697619457315	1.30801590090295\\
3.21763284831761	1.25521923424936\\
3.20327617608307	1.19621881822066\\
3.18252367432787	1.12965848973131\\
3.15348149758124	1.05383259621917\\
3.11354260866627	0.966591258648163\\
3.05910531395786	0.865231748365437\\
2.98519334228269	0.746392684964905\\
2.88497752433065	0.605997106085439\\
2.74926341418225	0.43935036706806\\
2.56617333589524	0.241604392998829\\
2.32158625195492	0.00893748363705662\\
2.0013801507348	-0.259164636303382\\
1.59667626378294	-0.556598110954975\\
1.11180299427944	-0.867816262874491\\
0.570628563809619	-1.16878568287357\\
0.0138354865941234	-1.43397669740479\\
-0.51477784008836	-1.64597382440369\\
-0.983585324057899	-1.80041115102751\\
-1.37923667624514	-1.90352622692107\\
-1.70304879787909	-1.96633146823376\\
-1.96406395146819	-1.9999609332631\\
-2.17358366620682	-2.01352459856801\\
-2.3422597299364	-2.01371519107483\\
-2.47902161586027	-2.00516977239792\\
-2.59094452144229	-1.99099881414488\\
-2.68348498162809	-1.97325066716072\\
-2.76080429913143	-1.95325798961672\\
-2.82606866053358	-1.93187800249144\\
-2.88169362637296	-1.90965337745034\\
-2.92953211223144	-1.88691822322425\\
-2.97101501167211	-1.86386742935304\\
-3.00725510852754	-1.84060192483462\\
-3.03912354064474	-1.81715816586222\\
-3.06730606704634	-1.79352725982213\\
-3.09234454601841	-1.76966721413769\\
-3.11466755692612	-1.7455105517505\\
-3.13461298630067	-1.72096872693834\\
-3.15244458048663	-1.6959342476818\\
-3.16836386841219	-1.67028106039175\\
-3.18251841524694	-1.64386351247528\\
-3.19500703061484	-1.61651403425936\\
-3.20588228444575	-1.58803954494798\\
-3.21515044773085	-1.55821646686304\\
-3.22276874634791	-1.52678411218415\\
-3.22863956562872	-1.49343607275455\\
-3.23260093919311	-1.45780908245487\\
-3.23441225649563	-1.41946861840475\\
-3.23373357302114	-1.37789024595265\\
-3.23009612544867	-1.33243537787069\\
-3.22286052918501	-1.28231970232306\\
-3.21115751492079	-1.22657205308672\\
-3.19380375532933	-1.1639810301653\\
-3.16918216277058	-1.09302646845147\\
-3.13507199916185	-1.01179349857178\\
-3.08840988556377	-0.917869845171345\\
-3.02496091257697	-0.808235214543548\\
-2.93888704748241	-0.679171346043859\\
-2.82223666826812	-0.526263839427488\\
-2.66448461949251	-0.344648301373574\\
-2.45249402739633	-0.129781031181872\\
-2.17169849900763	0.120868484719446\\
-1.80972935627272	0.404948643356206\\
-1.3632561340928	0.711840294406582\\
-0.84598563937002	1.02126035664147\\
-0.291314470632887	1.30722853904193\\
0.256386012127547	1.54720538202523\\
0.757911370961256	1.73018016716924\\
1.19073558268672	1.85774057613982\\
1.54966147374486	1.93923803701874\\
1.84070705774828	1.98616599240323\\
2.07456070878453	2.00877850877765\\
2.26242433708805	2.01495975701713\\
};
\addplot [color=mycolor1, forget plot]
  table[row sep=crcr]{%
2.29830749724921	2.08738533347698\\
2.4395811740186	2.08305263854005\\
2.55520061617609	2.07212901265497\\
2.65080211567494	2.05692218109676\\
2.73068984585204	2.03894971670617\\
2.79814044812796	2.01920078004988\\
2.85565401536398	1.99831282384068\\
2.90514813770187	1.97668873039448\\
2.94810347329143	1.9545738782152\\
2.98567147129506	1.9321067243545\\
3.01875369636157	1.90935197166206\\
3.04806022379772	1.88632225308333\\
3.07415270212239	1.86299217817905\\
3.09747616469615	1.83930722377463\\
3.11838252328554	1.81518906403112\\
3.13714782939007	1.79053835531842\\
3.15398476936351	1.7652356059088\\
3.16905140194706	1.73914049674307\\
3.1824568002718	1.71208982970108\\
3.1942639845052	1.68389413204003\\
3.20449029378011	1.65433281748077\\
3.21310511863907	1.62314767854759\\
3.22002467010205	1.59003434623226\\
3.22510316709981	1.55463118714243\\
3.22811944042731	1.51650489916663\\
3.22875742400764	1.47513179685374\\
3.22657825605299	1.42987342926179\\
3.22098063445581	1.37994473301228\\
3.2111445119643	1.32437239859616\\
3.19595098563718	1.2619405822732\\
3.17386813346458	1.19112073535739\\
3.14278851341989	1.10998268755293\\
3.09979955470441	1.01608649672324\\
3.04086530125618	0.906361827493813\\
2.96040347532639	0.776999705219256\\
2.85077245154363	0.623421613075426\\
2.70177576333412	0.440470155704622\\
2.50051308750921	0.223096331657488\\
2.2323179458462	-0.0320410681959813\\
1.88399246213748	-0.323721965763938\\
1.45030338557161	-0.642524096561426\\
0.942220642271537	-0.968737361939567\\
0.390812497285217	-1.2755120554387\\
-0.159963629894488	-1.53794078881014\\
-0.669198713934636	-1.74224053078579\\
-1.11183040302902	-1.88809740401201\\
-1.48062795069045	-1.98423320154642\\
-1.78052030750788	-2.04240516957082\\
-2.02184432018373	-2.0734973663325\\
-2.21584486452425	-2.08605362771345\\
-2.37256441865051	-2.08622730090106\\
-2.50021033879442	-2.0782481673088\\
-2.60520716935758	-2.0649511812546\\
-2.69248456073879	-2.04820986400352\\
-2.76579786844252	-2.02925083248949\\
-2.82800777561613	-2.00886956147172\\
-2.88130214750432	-1.98757446254011\\
-2.92736452028104	-1.96568194613657\\
-2.96749945605449	-1.94337886547424\\
-3.00272500583715	-1.9207634826045\\
-3.03384076001914	-1.8978723035551\\
-3.06147798141842	-1.87469756194609\\
-3.08613661192687	-1.85119844159636\\
-3.10821261761388	-1.82730802935857\\
-3.128018148681	-1.80293727330643\\
-3.14579626615936	-1.77797674994651\\
-3.16173145589966	-1.75229672683094\\
-3.17595675407387	-1.72574578494571\\
-3.18855800163099	-1.69814809973293\\
-3.19957549213643	-1.6692993440335\\
-3.2090030483687	-1.63896105116648\\
-3.21678433039166	-1.60685314597231\\
-3.22280591241952	-1.57264420130932\\
-3.2268863325603	-1.53593879205867\\
-3.22875987206479	-1.4962610814842\\
-3.22805319432644	-1.45303346778393\\
-3.22425207670579	-1.40554872534058\\
-3.21665417134161	-1.35293358966853\\
-3.20430186210206	-1.29410118586178\\
-3.18588664028215	-1.22768920689045\\
-3.15961284078826	-1.15198064948813\\
-3.12300419233677	-1.06480505265109\\
-3.07263255395004	-0.963422490360464\\
-3.00374830420568	-0.844404217321268\\
-2.90980681553143	-0.703551162898204\\
-2.78194086289829	-0.535948802319476\\
-2.60857622479041	-0.336361891275331\\
-2.37570095198536	-0.100320780841988\\
-2.06878304888541	0.173669299639037\\
-1.67760012151999	0.480726745207704\\
-1.20406911546428	0.80630008205186\\
-0.669318227607618	1.12628547167269\\
-0.112520488089901	1.41346196908147\\
0.421789369974737	1.64766180912662\\
0.899662855971952	1.82206522437116\\
1.3053487755271	1.94165717647392\\
1.63860176170692	2.01733702594874\\
1.90778748090552	2.06074259319959\\
2.12409187880228	2.08165636290458\\
2.29830749724921	2.08738533347698\\
};
\addplot [color=mycolor1, forget plot]
  table[row sep=crcr]{%
2.32935350426214	2.15617173173143\\
2.46092075586192	2.15213372489249\\
2.56908832940956	2.14191139927175\\
2.65896471538462	2.12761282504564\\
2.73444205483902	2.1106304941265\\
2.79848402521174	2.09187777804395\\
2.85335580032668	2.07194771005019\\
2.90079919596766	2.05121820007574\\
2.94216311250622	2.02992142129865\\
2.97849976370664	2.00818952500258\\
3.01063548002523	1.98608474349786\\
3.03922285172473	1.96361914362191\\
3.06477921334039	1.94076744429417\\
3.08771509049744	1.91747510467296\\
3.10835519914207	1.89366310208727\\
3.12695383187255	1.86923030050607\\
3.14370591329698	1.84405396201361\\
3.15875459586971	1.81798871118315\\
3.17219595232114	1.79086408294747\\
3.18408106369956	1.76248063963401\\
3.19441557281967	1.7326045112283\\
3.20315654396047	1.70096007769185\\
3.21020621229256	1.66722035772634\\
3.21540188668415	1.63099447862448\\
3.21850084250327	1.59181135878443\\
3.21915844446517	1.54909841790525\\
3.21689688452632	1.50215371941553\\
3.21106068009092	1.45010943213044\\
3.20075328108649	1.39188388794337\\
3.18474656782996	1.32611890190494\\
3.16135148736598	1.25109869757437\\
3.12823359553641	1.16464748759466\\
3.08215268206203	1.06400625588533\\
3.01860416618456	0.945699465708896\\
2.93135083717691	0.805427264160013\\
2.81187975223788	0.638072770612336\\
2.64894909567976	0.438017195115797\\
2.42867976698624	0.200113082811465\\
2.13613065485561	-0.0782098645709827\\
1.75967758187403	-0.393487132995316\\
1.29866158436661	-0.732453427227594\\
0.771202537901096	-1.07121256327827\\
0.214728980558727	-1.38091874404249\\
-0.325494961838313	-1.6384149650196\\
-0.812965102505627	-1.83404646195004\\
-1.22924624190243	-1.97125460517688\\
-1.57235707714977	-2.06070931296474\\
-1.84993959636118	-2.11455784134355\\
-2.07308909793711	-2.14330769833361\\
-2.25278582519434	-2.1549355994704\\
-2.39843774449618	-2.15509398815662\\
-2.51757797964365	-2.14764365112562\\
-2.61604479854629	-2.13517105857273\\
-2.6982992584146	-2.11939098848893\\
-2.76773681442649	-2.1014323062276\\
-2.82694677743785	-2.08203220493066\\
-2.87791380142478	-2.06166558681209\\
-2.92216930744609	-2.04063052932208\\
-2.96090359475513	-2.01910460671347\\
-2.99504838736969	-1.9971819926349\\
-3.02533758060229	-1.97489786432459\\
-3.05235202489591	-1.95224434892787\\
-3.07655261064923	-1.92918075751268\\
-3.09830472024621	-1.90563987805827\\
-3.11789623002166	-1.88153146092839\\
-3.13555059924298	-1.85674360684358\\
-3.15143610824056	-1.83114247836945\\
-3.16567195008156	-1.80457054930332\\
-3.1783315980804	-1.77684344724499\\
-3.1894436318143	-1.74774530867145\\
-3.19898997864918	-1.71702243439424\\
-3.20690128843532	-1.68437489038377\\
-3.21304887521986	-1.64944552896162\\
-3.21723229289842	-1.61180569135058\\
-3.21916110914594	-1.57093657538571\\
-3.21842872934844	-1.52620489105142\\
-3.21447509338645	-1.47683096313165\\
-3.20653357442546	-1.4218468714199\\
-3.19355525520003	-1.36004158880333\\
-3.17410072549746	-1.28988955477375\\
-3.14618550490828	-1.20945917541377\\
-3.10706046072684	-1.11629952448398\\
-3.05290489374579	-1.00730967183723\\
-2.978412885197	-0.878611223092022\\
-2.87627789570386	-0.72548180564268\\
-2.73666064701904	-0.542483349892518\\
-2.54692508885849	-0.324051637474437\\
-2.29232017876867	-0.0659769457238991\\
-1.9587960087572	0.231793697508506\\
-1.53911030580676	0.561283420912842\\
-1.04127224491097	0.903661418785349\\
-0.493744333777135	1.23140511216232\\
0.0600689388895753	1.51714943946215\\
0.577425408082559	1.74399826308549\\
1.03044074659414	1.90937750121675\\
1.40961161396351	2.02117636793413\\
1.71867229167716	2.09137080250741\\
1.96760462743302	2.13151173759376\\
2.16773707616045	2.15085998477536\\
2.32935350426214	2.15617173173143\\
};
\addplot [color=mycolor1, forget plot]
  table[row sep=crcr]{%
2.35571328277463	2.22100511820195\\
2.47826782286047	2.21724110275804\\
2.57945563798566	2.2076760066786\\
2.66391338264297	2.19423739652515\\
2.73516709681387	2.17820355908947\\
2.79590243487287	2.16041750763504\\
2.84817524353824	2.1414300272445\\
2.89356994912398	2.12159441970805\\
2.93331677967345	2.10112910266619\\
2.96837799604573	2.08015897733583\\
2.99951128269362	2.05874275651048\\
3.02731643733883	2.03689093924234\\
3.05226984512698	2.01457747210173\\
3.07474996249443	1.99174706294535\\
3.09505610683312	1.96831941038495\\
3.11342217070063	1.94419114633847\\
3.13002638320828	1.91923597242133\\
3.14499787010008	1.89330324619182\\
3.1584204744478	1.86621510225032\\
3.17033405755072	1.83776204934554\\
3.18073327556166	1.8076968479169\\
3.18956359405484	1.77572632588045\\
3.19671403015796	1.74150061739947\\
3.20200576274209	1.70459909187564\\
3.20517527509796	1.66451195769695\\
3.20585002004533	1.62061615374963\\
3.20351362162625	1.57214365762075\\
3.19745620422277	1.51813972897166\\
3.1867033712263	1.45740789312791\\
3.16991440413369	1.38843778139177\\
3.14523622615357	1.30931167074949\\
3.11009471123801	1.2175867111612\\
3.06090031723225	1.11015476668775\\
2.9926453648678	0.983095758531635\\
2.89838838626637	0.831573980390453\\
2.76868761057062	0.649898587637061\\
2.59122233945366	0.432001663954325\\
2.35121103645622	0.172770931598334\\
2.03377775386478	-0.129248435576604\\
1.62959742771365	-0.46779854288743\\
1.14346100946033	-0.825320879963705\\
0.600821886747986	-1.17394011308631\\
0.0442409056935615	-1.4838164622843\\
-0.48163880163311	-1.73456167138362\\
-0.945768934604242	-1.92088020146989\\
-1.33603704925895	-2.04954181981907\\
-1.65480624610948	-2.13266138371325\\
-1.91166875393097	-2.18249310676955\\
-2.11807157549804	-2.2090843405378\\
-2.28459218053074	-2.21985724993794\\
-2.42000314217803	-2.220001836486\\
-2.53121317335989	-2.21304488666874\\
-2.62353282613246	-2.2013486995232\\
-2.70100594205337	-2.18648395128973\\
-2.76670839564354	-2.16948958718696\\
-2.82298812770595	-2.15104809534126\\
-2.87164830808901	-2.13160199457462\\
-2.91408376440492	-2.11143085198241\\
-2.95138155578939	-2.09070216769097\\
-2.98439489737335	-2.06950500443297\\
-3.01379754605468	-2.04787217232614\\
-3.0401239091731	-2.02579474465893\\
-3.06379868525868	-2.00323135043708\\
-3.08515876112665	-1.98011382155867\\
-3.10446929599717	-1.95635020164888\\
-3.12193534452632	-1.93182574111754\\
-3.13770994317351	-1.90640223807004\\
-3.15189925874524	-1.87991589072489\\
-3.16456513585618	-1.85217367237726\\
-3.17572515028752	-1.82294810191056\\
-3.18535005003472	-1.79197014323174\\
-3.19335821697194	-1.75891980928172\\
-3.19960647651815	-1.7234138533874\\
-3.20387617749746	-1.68498968341194\\
-3.20585289974103	-1.64308431012539\\
-3.20509733712814	-1.59700671612584\\
-3.20100372563178	-1.54590148473199\\
-3.19274046786317	-1.48870085916713\\
-3.17916512589608	-1.4240616742884\\
-3.15870247919262	-1.35028304642604\\
-3.12916978909147	-1.26520095598012\\
-3.08752833431522	-1.16605847020069\\
-3.02953722802248	-1.04935888480526\\
-2.9492923192467	-0.910731054901148\\
-2.83866985696159	-0.744886013906675\\
-2.68680635565555	-0.545843632021727\\
-2.48001009197639	-0.307772295691794\\
-2.20297704173767	-0.0269519369913502\\
-1.84266886104903	0.294767101014322\\
-1.39569223068177	0.645751374552216\\
-0.876809173496224	1.0027034275221\\
-0.321315333824451	1.33532954261279\\
0.224920657710854	1.61726471950061\\
0.722602501490466	1.83555657428617\\
1.15023955592394	1.99171058627205\\
1.50385186179304	2.09599160249212\\
1.79026056897879	2.16104741003003\\
2.02047904511769	2.19817110296771\\
2.20572333459932	2.21607809894601\\
2.35571328277463	2.22100511820195\\
};
\addplot [color=mycolor1, forget plot]
  table[row sep=crcr]{%
2.37758606528632	2.28167835318404\\
2.49177779236238	2.27816889033819\\
2.58643654129522	2.26921889562557\\
2.66577691988729	2.25659271656946\\
2.73299918861372	2.24146446362388\\
2.79054164262947	2.22461204330309\\
2.84027340016404	2.20654634251511\\
2.88363789642987	2.18759675296512\\
2.92175850758457	2.16796776122089\\
2.95551605130682	2.14777643322451\\
2.98560570694247	2.12707723384355\\
3.01257893511641	2.10587836965161\\
3.03687443210908	2.08415237025142\\
3.05884100197166	2.06184266324354\\
3.07875438884017	2.03886726768229\\
3.09682950155217	2.01512031012179\\
3.11322901475573	1.99047177773649\\
3.12806899273829	1.96476571310981\\
3.14142191386205	1.93781689048177\\
3.1533172425677	1.90940586879712\\
3.16373947432695	1.87927217334283\\
3.17262333881433	1.8471051976523\\
3.17984555601306	1.81253222273927\\
3.18521215787343	1.77510270119484\\
3.18843985746846	1.73426762499693\\
3.18912918620102	1.68935235999834\\
3.18672600829574	1.63952075883262\\
3.18046639082213	1.58372764217373\\
3.16929742868451	1.52065589925095\\
3.15176323195933	1.4486336729205\\
3.12584069311508	1.36552687954528\\
3.08870415640836	1.26860400463144\\
3.03639360410928	1.1543768731755\\
2.96336381683274	1.01843987857661\\
2.8619195262561	0.855374897665057\\
2.72163445749676	0.658883198778094\\
2.52908689796336	0.422472560497084\\
2.26870515950629	0.141233696761046\\
1.92608821230028	-0.184774505377792\\
1.49499092845071	-0.545931274374462\\
0.986418384452357	-0.920048933956963\\
0.432982829304749	-1.27571827247748\\
-0.119136128434278	-1.58321574433414\\
-0.627655027613959	-1.82575970727758\\
-1.06757582968432	-2.00240478044851\\
-1.43255385347586	-2.12275069775533\\
-1.72842728644552	-2.19990845867801\\
-1.9661116709102	-2.24602123664282\\
-2.15710471968667	-2.27062600113633\\
-2.31149491765359	-2.28061196117877\\
-2.43743478465075	-2.28074408647642\\
-2.5412582089816	-2.27424702197866\\
-2.62780100444458	-2.26328076567257\\
-2.70073509484606	-2.24928521530659\\
-2.76285212574483	-2.23321675902357\\
-2.81628506334122	-2.21570678829857\\
-2.86267486836922	-2.19716683987995\\
-2.9032937035981	-2.17785815738066\\
-2.93913541648268	-2.15793774301906\\
-2.97098193864479	-2.13748886740034\\
-2.99945211478397	-2.11654123389789\\
-3.02503771560578	-2.09508417081845\\
-3.04813004952554	-2.07307503562399\\
-3.06903960224466	-2.05044423866413\\
-3.08801041732097	-2.02709778038385\\
-3.10523040918521	-2.00291784861494\\
-3.12083841282735	-1.97776177798815\\
-3.13492847566302	-1.95145948973794\\
-3.14755165113287	-1.92380937821806\\
-3.15871533073761	-1.89457246868706\\
-3.16837992399641	-1.86346452125125\\
-3.17645243491076	-1.83014558084612\\
-3.1827761531404	-1.79420625383446\\
-3.18711523016356	-1.755149705987\\
-3.18913227693457	-1.71236799799602\\
-3.18835620094771	-1.66511087430589\\
-3.18413615430636	-1.61244447464527\\
-3.17557549147682	-1.55319664750823\\
-3.16143678281318	-1.48588469655782\\
-3.14000494688445	-1.40862079042619\\
-3.10889042539018	-1.31899075571488\\
-3.06474890541543	-1.21390562553849\\
-3.00289199147913	-1.0894368677975\\
-2.91677516560724	-0.940675663175271\\
-2.79740245163396	-0.761722570024749\\
-2.63283920452629	-0.546042725412211\\
-2.40836566454116	-0.287620868514534\\
-2.10836455303723	0.0164973432651121\\
-1.72141396712113	0.362048522724962\\
-1.24883324577633	0.733216621124839\\
-0.712552708207109	1.10224434366042\\
-0.153813549914169	1.43692681283272\\
0.38088733213369	1.71300064169735\\
0.856965914982096	1.92187713206091\\
1.25926017399376	2.0688092073437\\
1.58849820036728	2.16591618255055\\
1.85380508806444	2.22618318092815\\
2.06677018214786	2.26052458407289\\
2.23831992589926	2.27710592776164\\
2.37758606528632	2.28167835318404\\
};
\addplot [color=mycolor1, forget plot]
  table[row sep=crcr]{%
2.39521560661132	2.33808947020594\\
2.50165301737442	2.33481631441034\\
2.59021226763556	2.32644121742552\\
2.66473014948301	2.31458089640988\\
2.72811656402173	2.30031450520744\\
2.78258937535318	2.28435986778607\\
2.8298509249221	2.26719041706212\\
2.87121826114054	2.24911257660912\\
2.90771855167311	2.23031702064168\\
2.94015893115056	2.21091268772948\\
2.96917775336511	2.19094933533025\\
2.9952823265977	2.17043239030596\\
3.01887677215071	2.14933252757017\\
3.04028259129508	2.12759154668238\\
3.05975376304627	2.10512554827704\\
3.07748764274719	2.08182603017976\\
3.0936325248088	2.05755925643596\\
3.1082924230745	2.03216405481763\\
3.12152937147492	2.0054480379043\\
3.13336332535661	1.97718209618812\\
3.14376952253335	1.94709285958934\\
3.15267291408491	1.91485264802381\\
3.15993896395509	1.88006621244632\\
3.16535969787814	1.84225328166619\\
3.16863329139493	1.8008255486039\\
3.16933462925266	1.75505621888867\\
3.16687300610201	1.7040395707407\\
3.16043127369357	1.646637119965\\
3.14887800857505	1.5814059914449\\
3.13064037732106	1.50650418960118\\
3.1035201403668	1.41956730718509\\
3.06442914594481	1.31755356243158\\
3.00901637102516	1.19656307143056\\
2.93116469893273	1.05166211781725\\
2.82237535363197	0.876802042322309\\
2.67118404936331	0.665044383789976\\
2.46305877837921	0.409511599850905\\
2.18178892819626	0.105702096678543\\
1.81392666502375	-0.244362060607632\\
1.35711711704518	-0.62712930014674\\
0.829177392341424	-1.01559675794281\\
0.269344016277399	-1.37549260102969\\
-0.274259788466607	-1.67834639642003\\
-0.763154680849728	-1.91159759274296\\
-1.17858291271912	-2.07844526629714\\
-1.51926750260519	-2.19079797487089\\
-1.7937312306146	-2.26237864675983\\
-2.013711434124	-2.3050575482378\\
-2.19053980187294	-2.32783626535857\\
-2.33376756044192	-2.33709825588292\\
-2.4509528573853	-2.33721912976606\\
-2.54790269211076	-2.3311502847474\\
-2.62902585113351	-2.32086907669468\\
-2.69766227901308	-2.30769673096066\\
-2.7563504132456	-2.2925139611881\\
-2.80703147496583	-2.2759046309315\\
-2.85120130171226	-2.25825088157889\\
-2.89002186142181	-2.2397960981004\\
-2.92440287539562	-2.22068665899547\\
-2.95506162620203	-2.20099964421522\\
-2.98256691656624	-2.18076116422536\\
-3.00737148513045	-2.1599583329753\\
-3.02983594952401	-2.13854684018862\\
-3.05024644954002	-2.11645537990111\\
-3.06882751475763	-2.09358772798619\\
-3.08575120843401	-2.06982294411827\\
-3.10114324655063	-2.0450139461929\\
-3.11508651472267	-2.01898452948588\\
-3.12762217267833	-1.99152475185492\\
-3.13874831762105	-1.96238445921168\\
-3.14841594642258	-1.9312645639131\\
-3.15652168143442	-1.89780549389601\\
-3.16289636663906	-1.86157198093439\\
-3.16728814587444	-1.82203302644447\\
-3.16933792492491	-1.77853544149122\\
-3.16854407998	-1.73026876914547\\
-3.16421174030815	-1.67621863279578\\
-3.15537971184138	-1.6151046176509\\
-3.14071483019245	-1.54529779542672\\
-3.11835896528835	-1.46471234062786\\
-3.0857080945164	-1.37066645943963\\
-3.03909709476125	-1.25971278195924\\
-2.97336311152426	-1.12745369940347\\
-2.88127896804591	-0.968395991395105\\
-2.7529218658074	-0.775985408972052\\
-2.57524500381626	-0.543126418552202\\
-2.33255204446239	-0.263728412551782\\
-2.00920815844371	0.0640732522834683\\
-1.59607758629773	0.433054732063467\\
-1.10000687731179	0.822759794085273\\
-0.550220947233858	1.20119336628135\\
0.00730350945160745	1.53525815237335\\
0.527202673577963	1.80377330441763\\
0.980457052624716	2.00268705038353\\
1.35787365899489	2.14055887331388\\
1.66406241824938	2.23087830678547\\
1.90978990059699	2.28670063179813\\
2.10687494963863	2.31848086352427\\
2.26583641719627	2.33384372724978\\
2.39521560661132	2.33808947020594\\
};
\addplot [color=mycolor1, forget plot]
  table[row sep=crcr]{%
2.40888077172283	2.3902330200096\\
2.50813233444798	2.38717908606175\\
2.59100005011747	2.37934066202057\\
2.66098213322652	2.3682008737186\\
2.72072924962287	2.35475236287878\\
2.77226248661235	2.33965762874679\\
2.81713501474985	2.32335511691346\\
2.85655047962942	2.30612938999659\\
2.8914493925071	2.28815765007762\\
2.92257224517012	2.26954064736369\\
2.95050578023133	2.25032319062776\\
2.97571704516746	2.23050763640493\\
2.99857852155886	2.21006254104203\\
3.01938665427382	2.18892788205039\\
3.03837541110589	2.16701774112359\\
3.05572600135215	2.14422099246346\\
3.07157351078336	2.12040029297962\\
3.0860109244689	2.09538948320704\\
3.09909077261541	2.06898934996484\\
3.110824418657	2.04096155145672\\
3.1211787858935	2.01102034331287\\
3.13007005924853	1.97882155055495\\
3.13735356516316	1.94394798350517\\
3.1428085744848	1.90589016811499\\
3.14611611703187	1.86402081910777\\
3.14682693374144	1.81756088702465\\
3.14431526271719	1.76553421586626\\
3.1377120291608	1.70670683383351\\
3.12580787897313	1.63950571972651\\
3.10691202114415	1.56191082000341\\
3.07864686425438	1.47131399451154\\
3.03765166922578	1.36434169898715\\
2.97916448488124	1.236649989726\\
2.89646197940779	1.08273284380471\\
2.78019192232968	0.895861341069844\\
2.61780086671964	0.668427872973931\\
2.39365140476653	0.393224344394331\\
2.09108675215881	0.066400864885246\\
1.69815533443762	-0.30756101600652\\
1.21720249138408	-0.710637308388417\\
0.673242697314047	-1.11100197657112\\
0.111271203340768	-1.47238612026827\\
-0.420354927699846	-1.76865799316422\\
-0.888053846144788	-1.99185475738594\\
-1.2791714796078	-2.14896926102479\\
-1.59673873350113	-2.253711661735\\
-1.85127279553356	-2.3200990957414\\
-2.05494113784792	-2.35961363973481\\
-2.21875836531109	-2.3807149952148\\
-2.35171732991723	-2.38931115041\\
-2.4608135960206	-2.38942186162935\\
-2.55137267379465	-2.38375139117903\\
-2.62741912650439	-2.37411210124081\\
-2.69199612519804	-2.36171752052997\\
-2.7474161315611	-2.34737908193473\\
-2.79544906409573	-2.33163659063348\\
-2.83746076631175	-2.31484448727144\\
-2.87451414789845	-2.29722896426605\\
-2.90744300093537	-2.27892588091402\\
-2.93690601309933	-2.260005954406\\
-2.96342644474103	-2.24049142405698\\
-2.98742137697517	-2.22036690515763\\
-3.00922329903649	-2.19958618681875\\
-3.02909598395469	-2.17807609740442\\
-3.04724601225948	-2.15573813949756\\
-3.06383087348644	-2.13244830482443\\
-3.0789642517126	-2.10805526648643\\
-3.09271884414577	-2.08237697620772\\
-3.10512683912712	-2.05519554276558\\
-3.11617796394417	-2.02625011386991\\
-3.12581477556484	-1.99522730843816\\
-3.13392457607716	-1.96174852900089\\
-3.1403269462786	-1.92535320050831\\
-3.14475534472522	-1.88547660169896\\
-3.14683042644459	-1.84142044089567\\
-3.14602156332116	-1.79231363703443\\
-3.14159130388004	-1.73705986251555\\
-3.13251492454502	-1.67426729172221\\
-3.11736346176195	-1.60215481634377\\
-3.09413337798508	-1.51842823891945\\
-3.05999944396679	-1.42012104981175\\
-3.01096128934895	-1.30340086206869\\
-2.94135499957542	-1.16336257543353\\
-2.8432282197083	-0.993879982512098\\
-2.70567713390113	-0.787699302400684\\
-2.51450836283949	-0.537165999597471\\
-2.2531264418093	-0.236249327948188\\
-1.90623194822527	0.115453118936052\\
-1.46769549096299	0.507185376642161\\
-0.950611710694782	0.913494425757623\\
-0.391314880114672	1.29859008381683\\
0.16092568827981	1.62959380211794\\
0.663450670397169	1.88921030518518\\
1.09326257240482	2.07787932923935\\
1.44658202591583	2.2069691762133\\
1.73111770281516	2.2909095372093\\
1.95873315187099	2.34261929249117\\
2.14121952233708	2.37204479129414\\
2.28861434227596	2.38628816417568\\
2.40888077172283	2.3902330200096\\
};
\addplot [color=mycolor1, forget plot]
  table[row sep=crcr]{%
2.41888318380138	2.43818635730555\\
2.51147824027556	2.43533570150822\\
2.58904001753741	2.42799778514815\\
2.654762915214	2.41753476294698\\
2.71106571560561	2.40486046826886\\
2.75979350888519	2.39058653892867\\
2.80236582063492	2.37511886450643\\
2.83988441701825	2.35872135292565\\
2.87321169884619	2.34155823262177\\
2.9030278718053	2.32372217986472\\
2.92987282283087	2.3052529845246\\
2.95417692552194	2.28614979942586\\
2.97628375721661	2.26637893920662\\
2.99646682350968	2.24587849060945\\
3.01494175254589	2.22456052875055\\
3.0318749635532	2.20231141395228\\
3.04738947390946	2.17899041324983\\
3.06156824349506	2.15442671091284\\
3.07445523077159	2.12841471565484\\
3.08605412360892	2.10070741684961\\
3.09632448182186	2.07100736819664\\
3.10517475647273	2.03895466388302\\
3.11245129298451	2.00411099441553\\
3.11792192439974	1.96593849533151\\
3.12125203444865	1.92377159147383\\
3.12196989283308	1.87677934295999\\
3.11941645171259	1.82391486427375\\
3.11267237480603	1.76384718481904\\
3.100451490343	1.69486951063386\\
3.08094472458089	1.61477657455123\\
3.05159173440928	1.52070370295161\\
3.008749919873	1.40892425665714\\
2.94722700922035	1.27461720696809\\
2.85965903539448	1.11165831545422\\
2.73578992327268	0.912586932136786\\
2.56192833078652	0.66909961063489\\
2.32135179263591	0.373729959642227\\
1.99719390871752	0.0235653536649638\\
1.57960139546763	-0.373916177171587\\
1.07639734702074	-0.79572889688355\\
0.519932676754131	-1.20541169193261\\
-0.0401808144590513	-1.56571184119036\\
-0.556982479305206	-1.85380500958858\\
-1.00251770771059	-2.06647459713139\\
-1.36985885057439	-2.21406294236225\\
-1.66558777908723	-2.31161292294382\\
-1.90163229697547	-2.37318111115254\\
-2.0902909759994	-2.40978353887551\\
-2.24216045712678	-2.4293446369676\\
-2.36567306671119	-2.43732844588558\\
-2.46729667711223	-2.4374299717228\\
-2.55191766208865	-2.43212985980174\\
-2.62321463845191	-2.42309129638358\\
-2.68396502687506	-2.41143003860427\\
-2.7362792241262	-2.39789414199754\\
-2.78177383166558	-2.3829826772989\\
-2.82169808008363	-2.36702412299338\\
-2.85702574275742	-2.35022828069243\\
-2.88852204894739	-2.33272076557404\\
-2.9167925847934	-2.3145659349788\\
-2.94231919346127	-2.29578204363779\\
-2.9654864259931	-2.27635107291352\\
-2.98660104464018	-2.25622481115633\\
-3.00590633144478	-2.23532819032211\\
-3.02359241755068	-2.213560499177\\
-3.03980345588516	-2.19079482401433\\
-3.05464216166413	-2.16687586668619\\
-3.06817200400236	-2.14161612451635\\
-3.08041711743815	-2.11479026317304\\
-3.09135978700224	-2.0861273514492\\
-3.10093511580077	-2.05530043606938\\
-3.10902217497754	-2.02191269248365\\
-3.11543051501338	-1.98547906615454\\
-3.11988031622032	-1.94540188226001\\
-3.1219735731995	-1.90093830482429\\
-3.12115239057052	-1.85115671672689\\
-3.11663849111376	-1.79487802364494\\
-3.10734508823608	-1.73059656028233\\
-3.09174796201318	-1.65637386358516\\
-3.06769657194691	-1.56969770035753\\
-3.03213858413043	-1.4673001803593\\
-2.98072466317665	-1.3449370790949\\
-2.90726347981168	-1.19715618329567\\
-2.80303457019989	-1.01714771007273\\
-2.65609915783797	-0.796913211476596\\
-2.45108995970766	-0.528249316603974\\
-2.17061792601485	-0.205349676514746\\
-1.80012956956224	0.170304936508782\\
-1.33725525923846	0.583846334044822\\
-0.801923235957124	1.0045985704694\\
-0.237080966282234	1.39362898194891\\
0.306283202348292	1.71941108888218\\
0.789517259102474	1.96912730676105\\
1.19575951981373	2.14748633998509\\
1.52597909755489	2.26815294435095\\
1.79027540467186	2.3461286629649\\
2.00117230186604	2.39404152151433\\
2.17024730915772	2.42130366179175\\
2.3070153558866	2.43451861795335\\
2.41888318380138	2.43818635730555\\
};
\addplot [color=mycolor1, forget plot]
  table[row sep=crcr]{%
2.42553419358814	2.48209316550803\\
2.51196378081259	2.47943096870429\\
2.58458213605233	2.47255952928049\\
2.64631060549358	2.46273135324056\\
2.69936012467214	2.45078846621064\\
2.74541804078164	2.43729579639707\\
2.78578391756268	2.42262902919096\\
2.82146785269512	2.4070326697568\\
2.85326176109995	2.39065857672067\\
2.88179127793504	2.37359159422191\\
2.9075537462819	2.35586655222134\\
2.93094614675948	2.33747938774235\\
2.95228567701671	2.31839416027534\\
2.97182487470824	2.29854709497804\\
2.98976259824536	2.27784836062498\\
3.00625176062514	2.25618199419038\\
3.02140439839821	2.23340416756501\\
3.03529440988456	2.20933981837323\\
3.04795808237633	2.18377751003194\\
3.0593923196453	2.15646222458848\\
3.06955025068456	2.1270856048844\\
3.07833361533312	2.09527292729243\\
3.08558093838671	2.06056577433141\\
3.09104995792431	2.02239895096344\\
3.09439197162826	1.98006960102621\\
3.09511456410917	1.93269567074746\\
3.09252736404776	1.87915976927787\\
3.08566273996497	1.81803305009449\\
3.07315925787811	1.7474720507549\\
3.05308983755072	1.66507989221115\\
3.02270871379862	1.56772318318652\\
2.97808286959396	1.45130101909892\\
2.91357071143426	1.31048152477025\\
2.82113288193912	1.13847444146357\\
2.68955830356001	0.927033947156868\\
2.50397138595715	0.667137086569504\\
2.24660171292782	0.351150802508159\\
1.9006557586107	-0.022570977245387\\
1.45903192426012	-0.442983784762838\\
0.935743711773808	-0.881729012214833\\
0.370352399760586	-1.29810141004348\\
-0.184260873698683	-1.65496980145545\\
-0.683992062919049	-1.93362144158048\\
-1.10690190382878	-2.13553351675148\\
-1.45125304056741	-2.27390535791362\\
-1.72646580436798	-2.36469590442473\\
-1.94539735940974	-2.42180269160389\\
-2.12025416477838	-2.45572707555127\\
-2.26115164240814	-2.47387374299914\\
-2.37597227077536	-2.48129425404923\\
-2.47069213137682	-2.48138746964015\\
-2.54979752218387	-2.47643153711324\\
-2.61665524824535	-2.46795461928887\\
-2.67380430738313	-2.45698365254657\\
-2.7231740280917	-2.44420873454906\\
-2.76624351461178	-2.43009134652392\\
-2.804157207019	-2.41493573690072\\
-2.83780856311567	-2.39893619990212\\
-2.86790084645478	-2.38220850145124\\
-2.89499150134421	-2.36481077198873\\
-2.91952471037678	-2.34675729424175\\
-2.94185536793424	-2.32802739607041\\
-2.96226673544778	-2.30857086859801\\
-2.98098335819641	-2.28831080921709\\
-2.99818033221118	-2.26714443625605\\
-3.01398964954104	-2.24494217170217\\
-3.02850407393302	-2.22154509715157\\
-3.04177877142498	-2.19676072585496\\
-3.0538307123217	-2.17035687706612\\
-3.06463564530312	-2.14205326725037\\
-3.0741221912185	-2.11151022469609\\
-3.08216227591155	-2.07831366445504\\
-3.08855666595789	-2.04195509708647\\
-3.09301371161753	-2.00180494507621\\
-3.09511842166653	-1.95707675054247\\
-3.09428751965502	-1.90677891236173\\
-3.08970390002161	-1.8496493307119\\
-3.08022054787565	-1.78406675893942\\
-3.06421905138112	-1.70793096236617\\
-3.03940094597951	-1.61850272426761\\
-3.00248164581652	-1.51219654758801\\
-2.9487497377427	-1.38432932031328\\
-2.87146019074656	-1.22886076983121\\
-2.76108079899122	-1.03824471047934\\
-2.60458386264906	-0.803692396134978\\
-2.38540845463613	-0.516471258475225\\
-2.08550773718212	-0.171195817604975\\
-1.69154178596567	0.228301500494895\\
-1.20566764318991	0.662469355804623\\
-0.655061178092835	1.09533772275853\\
-0.0884979704197673	1.48566918100562\\
0.442917165571432	1.80437867094387\\
0.905533049896758	2.04349773065613\\
1.28846169219662	2.21165095136388\\
1.59671409466504	2.32430362517328\\
1.84216252941488	2.39672320743641\\
2.0376485110943	2.44113560505875\\
2.19440565187908	2.46641069964569\\
2.3214083155111	2.47868070833737\\
2.42553419358814	2.48209316550803\\
};
\addplot [color=mycolor1, forget plot]
  table[row sep=crcr]{%
2.42914281428483	2.52214630241406\\
2.5098606425793	2.51965884751003\\
2.57787455974418	2.51322203854237\\
2.63586005304591	2.50398887156166\\
2.68584133577255	2.4927359032191\\
2.7293640482531	2.47998518591892\\
2.76761984560266	2.46608444165716\\
2.80153619318871	2.45126002906961\\
2.83184130663175	2.4356521034578\\
2.8591113714304	2.41933800609414\\
2.88380507716818	2.40234775957729\\
2.90628899315419	2.38467416302029\\
2.92685624829824	2.36627908944838\\
2.94574022954792	2.3470970047594\\
2.96312448231146	2.32703633653899\\
2.97914961171824	2.30597904757141\\
2.99391769395348	2.2837785644528\\
3.00749447417415	2.2602560429447\\
3.01990942140661	2.23519479358935\\
3.03115350419836	2.20833252217004\\
3.04117431513899	2.17935083817664\\
3.04986787277575	2.14786122530522\\
3.05706601807852	2.11338631890824\\
3.06251772928118	2.07533485311084\\
3.06586179724353	2.03296796742609\\
3.06658697040766	1.9853536256505\\
3.06397364630225	1.93130461639889\\
3.05700809149553	1.86929391704424\\
3.04425552116582	1.79733918115981\\
3.02367162803375	1.71284623573629\\
2.99232317303891	1.61240136761138\\
2.94597874879659	1.49150834188721\\
2.87852863649959	1.34428980571425\\
2.78122268186052	1.16323950414677\\
2.64184244700154	0.939270709619785\\
2.44428403194152	0.662620627599521\\
2.16978442306006	0.325602725991836\\
1.80195354986152	-0.071787254720533\\
1.33713714180375	-0.514344623845644\\
0.796155506874123	-0.968029684742976\\
0.225385408830048	-1.38848241834113\\
-0.320495137767467	-1.7398323501747\\
-0.801466793838679	-2.00808984704342\\
-1.20169669685893	-2.19920949262367\\
-1.52401280356067	-2.3287433088123\\
-1.7800298904326	-2.41320755995004\\
-1.9831462530216	-2.46619055938935\\
-2.14531363799268	-2.49765257942019\\
-2.27613067877928	-2.51449980177372\\
-2.38294898519408	-2.52140184213171\\
-2.47128834382054	-2.52148752949664\\
-2.54527068598905	-2.51685142827531\\
-2.60798137799079	-2.50889931954025\\
-2.66174506026186	-2.49857737059843\\
-2.70832843285511	-2.4865226709097\\
-2.74908701514419	-2.47316205670539\\
-2.78507089292095	-2.45877723410144\\
-2.81710103606706	-2.44354791224032\\
-2.845824633339	-2.427580480807\\
-2.87175544628144	-2.41092706846434\\
-2.89530339855213	-2.39359809108922\\
-2.91679634871061	-2.37557028981545\\
-2.936496102415	-2.35679153970488\\
-2.95461009070748	-2.33718323428111\\
-2.97129969082941	-2.31664072628661\\
-2.9866858343625	-2.29503207100744\\
-3.00085229047382	-2.27219513542462\\
-3.01384679610838	-2.24793297581879\\
-3.02568000197646	-2.22200722554269\\
-3.03632198587308	-2.19412905234739\\
-3.04569582223605	-2.1639470178792\\
-3.05366734848216	-2.13103087246186\\
-3.06002977695509	-2.09484990883986\\
-3.0644810800858	-2.0547439292298\\
-3.06659099391686	-2.00988408513625\\
-3.06575284006289	-1.95921974918372\\
-3.06111285637371	-1.90140609570132\\
-3.05146592213324	-1.83470518767195\\
-3.03510092726093	-1.75685131364158\\
-3.00957113235041	-1.66487000420575\\
-2.97135518288351	-1.5548423264199\\
-2.91536692282082	-1.42161891556544\\
-2.83428115319192	-1.25852896894489\\
-2.7177092002887	-1.05723449872739\\
-2.55148008890396	-0.808110195610234\\
-2.31782762169535	-0.501924537373715\\
-1.99821561543265	-0.133944263918715\\
-1.58104134546577	0.289132020717469\\
-1.07374775719885	0.742526958041193\\
-0.510972336618491	1.18507853541027\\
0.0537157539135096	1.57423152002543\\
0.570635840131045	1.88433173513163\\
1.01181445364394	2.11241994787035\\
1.37197100770443	2.27059809947887\\
1.65945988956961	2.37567301381497\\
1.88740169647059	2.44293063520625\\
2.06869210106127	2.48411823489821\\
2.21413317989098	2.5075678533355\\
2.3321570981413	2.51896913700286\\
2.42914281428483	2.52214630241406\\
};
\addplot [color=mycolor1, forget plot]
  table[row sep=crcr]{%
2.43000560883911	2.55857157411362\\
2.50542930976853	2.5562462145367\\
2.5691541465995	2.55021438887263\\
2.62363377252617	2.54153864695314\\
2.67072424682261	2.53093579785419\\
2.71184359297447	2.51888853629303\\
2.74808617215659	2.50571873005342\\
2.78030480632662	2.49163582113813\\
2.80917004144808	2.47676895742148\\
2.83521318831917	2.46118836197196\\
2.85885777907958	2.44491946682238\\
2.88044266719758	2.42795207482972\\
2.90023901419008	2.41024599991185\\
2.91846271780422	2.39173410286394\\
2.93528334893509	2.37232328020922\\
2.9508303105995	2.35189370892216\\
2.9651966635197	2.33029645555811\\
2.97844084339309	2.30734939306867\\
2.99058629571449	2.28283120819943\\
3.00161884823388	2.25647310514837\\
3.01148139949471	2.22794759411449\\
3.02006518730663	2.19685346859266\\
3.02719646128919	2.16269568598247\\
3.032616740877	2.12485832201185\\
3.03595387466936	2.08256800212379\\
3.03667964313811	2.03484413297623\\
3.03404737943109	1.98043076123442\\
3.02699959897154	1.91770289468106\\
3.01403034841165	1.84453769696113\\
2.99297927404867	1.75813866559983\\
2.9607241064281	1.65480066598533\\
2.91272749734146	1.52961106448614\\
2.84239275664357	1.37611132381162\\
2.74022248164007	1.18602667489593\\
2.59293678693402	0.949371056469283\\
2.38316159844013	0.655625245963671\\
2.09121669429894	0.297186618404667\\
1.70149648802604	-0.123883606133013\\
1.21452152414961	-0.58761333599355\\
0.6584099902644	-1.05409912183383\\
0.0857008954183244	-1.47609947001611\\
-0.448648923464263	-1.82012192354314\\
-0.909667247310918	-2.07730900092981\\
-1.28747756439644	-2.25775275935131\\
-1.58881445245579	-2.3788685303394\\
-1.82692255391993	-2.45742904132522\\
-2.01543400573807	-2.50660329190737\\
-2.16593074767542	-2.53580052536536\\
-2.287479025774	-2.5514529918328\\
-2.38692357646327	-2.55787740509025\\
-2.46936206415589	-2.55795626182771\\
-2.53858447918773	-2.55361744838998\\
-2.59742172857421	-2.54615564023323\\
-2.64800528589124	-2.53644346222473\\
-2.69195542105061	-2.52506949611656\\
-2.73051630339294	-2.51242866609342\\
-2.76465287164565	-2.49878175250296\\
-2.79512054459071	-2.48429480801279\\
-2.82251568425663	-2.46906536428564\\
-2.84731236472344	-2.45313983471818\\
-2.8698893167037	-2.43652493831251\\
-2.89054973919312	-2.41919495783558\\
-2.909535846854	-2.40109598879854\\
-2.92703944376479	-2.38214789987525\\
-2.94320940052873	-2.36224442497792\\
-2.95815660549424	-2.34125158732077\\
-2.97195672065852	-2.31900447936387\\
-2.98465086686821	-2.29530226248473\\
-2.99624416382948	-2.26990108413472\\
-3.00670183073618	-2.24250441619777\\
-3.01594228058818	-2.21275007129988\\
-3.02382627194575	-2.18019282211358\\
-3.03014065252787	-2.14428108934219\\
-3.03457444334637	-2.10432551822\\
-3.03668382089796	-2.05945635209584\\
-3.03584072775242	-2.00856523662128\\
-3.03115702909942	-1.95022534945272\\
-3.02137183160458	-1.88258151604598\\
-3.00468315655043	-1.80319948472831\\
-2.97849609747813	-1.70886187224915\\
-2.93904850659789	-1.59530079115891\\
-2.88086707514858	-1.45687279619227\\
-2.79601948321354	-1.28623227530518\\
-2.6732142885432	-1.07419103369826\\
-2.49708205345216	-0.810240129508162\\
-2.24864845706541	-0.484691325075486\\
-1.90909185335305	-0.0937332482755964\\
-1.46912479590529	0.352510784013221\\
-0.942205516224551	0.823542318810933\\
-0.370427105395691	1.27329407034341\\
0.18907995510647	1.65898618259533\\
0.689464296947835	1.9592427279986\\
1.10880856622092	2.17608537037877\\
1.44693610665946	2.32460882647534\\
1.71488697109807	2.42255091117002\\
1.92659462130799	2.48502084216652\\
2.09481030058443	2.52323798408973\\
2.22984903760658	2.54500951727224\\
2.33961026878607	2.55561145439992\\
2.43000560883911	2.55857157411362\\
};
\addplot [color=mycolor1, forget plot]
  table[row sep=crcr]{%
2.42839893897157	2.59161347785827\\
2.49891160776086	2.58943859617106\\
2.5586393815682	2.58378428441539\\
2.60983529655238	2.57563073807164\\
2.65420358518009	2.56564016929899\\
2.69304703887209	2.55425912691462\\
2.72737207287741	2.54178559052053\\
2.75796392474883	2.52841326757055\\
2.78544082091949	2.51426099890381\\
2.8102932758938	2.4993923045474\\
2.83291279628798	2.48382828371773\\
2.85361294631788	2.46755592724537\\
2.87264482072889	2.45053315679859\\
2.89020833727843	2.43269141648883\\
2.90646031256835	2.41393631034023\\
2.92151995843505	2.39414654095953\\
2.93547218584112	2.37317121902336\\
2.94836889534608	2.35082545040057\\
2.96022823981116	2.32688394425288\\
2.97103163931986	2.30107219914192\\
2.98071808016543	2.27305459043335\\
2.98917489968384	2.24241837019282\\
2.99622378987899	2.20865215841507\\
3.00160005897145	2.17111689337597\\
3.00492213694687	2.12900633796492\\
3.00564668835006	2.08129299979105\\
3.00300217589035	2.02665358622318\\
2.99588981025141	1.96336577112433\\
2.98273484323523	1.88916514773764\\
2.9612623557401	1.80104840378502\\
2.92815985732262	1.69500827984193\\
2.8785767525192	1.56569451323293\\
2.80541021102086	1.40603032094314\\
2.69837761964472	1.20691688874364\\
2.54308121316076	0.957407214341849\\
2.32083702813122	0.646213341099079\\
2.01114531096811	0.265981362292742\\
1.59961914501329	-0.178687219208589\\
1.09170162030625	-0.662444078680426\\
0.523148560624259	-1.13948493470032\\
-0.0482280155187187	-1.56062127744243\\
-0.568683111535514	-1.89578522694911\\
-1.00897944928141	-2.1414632250918\\
-1.36486403376478	-2.31146022447935\\
-1.64632583661454	-2.42459818521446\\
-1.86775609182295	-2.49765963341006\\
-2.04278178152133	-2.54331658749513\\
-2.18253654625458	-2.57042917760987\\
-2.29555273001758	-2.58498190658312\\
-2.38819486713318	-2.59096580666851\\
-2.46517079139088	-2.59103845424658\\
-2.52996784564671	-2.58697614108782\\
-2.58518641381692	-2.5799724839866\\
-2.6327834639797	-2.57083303956825\\
-2.67424707129207	-2.56010195883131\\
-2.71072081715863	-2.54814477250849\\
-2.74309261413974	-2.53520286296155\\
-2.77205847163025	-2.52142953749779\\
-2.79816859225998	-2.50691400797144\\
-2.82186093299831	-2.49169729619974\\
-2.84348578504061	-2.47578263684834\\
-2.86332383214565	-2.45914202411449\\
-2.88159938833896	-2.4417199475055\\
-2.89848998425722	-2.42343496121357\\
-2.91413309060001	-2.40417945271752\\
-2.92863048354321	-2.3838177685781\\
-2.94205053165504	-2.36218268433478\\
-2.95442848652006	-2.33907004504167\\
-2.96576466316621	-2.31423123091869\\
-2.97602017368924	-2.28736289594039\\
-2.98510959416056	-2.25809315870114\\
-2.99288955404653	-2.2259630587995\\
-2.99914166929046	-2.19040157886496\\
-3.00354738752565	-2.15069180315162\\
-3.00565100845932	-2.10592474384398\\
-3.00480512126117	-2.05493589499166\\
-3.00008956030611	-1.99621754249474\\
-2.99019013318851	-1.92779720802456\\
-2.97321606849211	-1.8470695906231\\
-2.9464247454543	-1.75056724231749\\
-2.90580954052106	-1.63365798584864\\
-2.84549761729985	-1.49017579735687\\
-2.75692173228313	-1.31205379608571\\
-2.62783923073872	-1.08919163341809\\
-2.44162569282153	-0.810148681924428\\
-2.17810548285373	-0.464835980349436\\
-1.81841420228852	-0.0506761365086647\\
-1.35621023012268	0.418182881435659\\
-0.811643913041554	0.905094544320476\\
-0.234026790647956	1.35956208676017\\
0.31732084940483	1.73973430997547\\
0.799593610230153	2.02919135077307\\
1.19704506377141	2.23474970256805\\
1.51401904710699	2.37399798823262\\
1.76364336609869	2.46524767908681\\
1.96030951561495	2.52328093884892\\
2.11647783080989	2.55876082706519\\
2.24194455351926	2.57898823216638\\
2.3440931018766	2.58885379549762\\
2.42839893897157	2.59161347785827\\
};
\addplot [color=mycolor1, forget plot]
  table[row sep=crcr]{%
2.42457355287415	2.621523428073\\
2.49052555348901	2.61948838573613\\
2.54652557549044	2.61418624100554\\
2.5946447751745	2.60652205238196\\
2.63644992345651	2.59710806248537\\
2.67313947238414	2.58635759423051\\
2.70564012646988	2.57454655749056\\
2.73467577519072	2.56185404543694\\
2.76081706891956	2.54838927883922\\
2.78451735616315	2.53420950163965\\
2.80613891728896	2.51933176099744\\
2.82597220488806	2.50374044594455\\
2.84424995897383	2.48739177711343\\
2.86115748122019	2.47021599066643\\
2.8768399395112	2.45211765197708\\
2.89140727205672	2.43297431092441\\
2.90493702657521	2.41263353221843\\
2.91747527242207	2.39090817293977\\
2.92903553501016	2.36756961221875\\
2.93959549574904	2.34233844187454\\
2.94909094576697	2.31487187547121\\
2.95740613582958	2.28474679228606\\
2.96435916638781	2.2514368545102\\
2.96968031544688	2.21428145292397\\
2.97298005798564	2.17244325254973\\
2.97370175105131	2.12484969488447\\
2.97105116935237	2.0701118059777\\
2.96389070753553	2.00641091054438\\
2.95057931564856	1.93134038439232\\
2.92872919451826	1.84168607285294\\
2.8948366770195	1.73312814808155\\
2.8437306653131	1.59985712136739\\
2.76778238029053	1.43413919446595\\
2.65588398862367	1.22599238599821\\
2.49246037204388	0.963443405009668\\
2.25748017035919	0.634428293595332\\
1.92974695414895	0.232038141126992\\
1.49658300369732	-0.236056243872965\\
0.969109087334938	-0.738532999794152\\
0.390884592398836	-1.22381266260643\\
-0.17609913651268	-1.64182639793885\\
-0.680710189455144	-1.96686693340394\\
-1.0998697153845	-2.20079527770734\\
-1.43448717000876	-2.36065433874548\\
-1.69718709536818	-2.46625907080926\\
-1.90310149597363	-2.53420367432117\\
-2.06566949584141	-2.57661116539494\\
-2.19552571443911	-2.60180275093068\\
-2.30067674173315	-2.61534176757871\\
-2.3870346266155	-2.62091880392253\\
-2.45894748078343	-2.62098579549941\\
-2.51962636315109	-2.61718088248714\\
-2.57146235558115	-2.61060556819087\\
-2.61625436119793	-2.60200413307862\\
-2.65537077903324	-2.59187997978379\\
-2.68986407397265	-2.5805715534327\\
-2.72055229588059	-2.56830226729342\\
-2.74807748000692	-2.5552135604492\\
-2.77294781574204	-2.54138686457017\\
-2.79556832752743	-2.52685815230927\\
-2.81626333304886	-2.51162741180158\\
-2.83529292886278	-2.49566454406377\\
-2.85286505359682	-2.47891262867918\\
-2.86914418934422	-2.46128913358444\\
-2.88425741063562	-2.44268538492361\\
-2.89829822701907	-2.42296441587734\\
-2.91132845317397	-2.4019571466252\\
-2.92337815065671	-2.37945668613773\\
-2.93444349154553	-2.35521036779604\\
-2.94448216812327	-2.32890891078919\\
-2.95340567856407	-2.30017180805798\\
-2.96106740583477	-2.26852763901446\\
-2.96724479935986	-2.23338743466187\\
-2.97161304719748	-2.19400840336833\\
-2.97370620162521	-2.1494441447997\\
-2.97285949413804	-2.09847578960964\\
-2.96812308252839	-2.03951613824585\\
-2.95813202714109	-1.97047573735495\\
-2.9409090029799	-1.88857618308768\\
-2.91356436109427	-1.79009320308102\\
-2.87184350497758	-1.67001501497011\\
-2.8094614865568	-1.52162358128125\\
-2.71718707040197	-1.33608164949733\\
-2.58177514845009	-1.10231058322602\\
-2.38528794053165	-0.807888873787678\\
-2.10636620236644	-0.442398846979578\\
-1.72638849802968	-0.00485656194973809\\
-1.24263964739572	0.48592730484768\\
-0.682563345208584	0.986820029965961\\
-0.102218521560992	1.44355819563319\\
0.438335201205228	1.8163867024402\\
0.901333779174422	2.09433643676666\\
1.27709655064445	2.28870855265296\\
1.57386992374557	2.41909611378573\\
1.80634015035226	2.50408009849308\\
1.98907227577506	2.55800280219291\\
2.1341303634353	2.59095821724414\\
2.25077741915636	2.60976288276926\\
2.34590198860912	2.6189491003489\\
2.42457355287415	2.621523428073\\
};
\addplot [color=mycolor1, forget plot]
  table[row sep=crcr]{%
2.41875121389676	2.64855056580978\\
2.4804621851032	2.64664564603231\\
2.53298198384958	2.64167236070519\\
2.57821643935736	2.63446706530605\\
2.61760751089612	2.62559618406988\\
2.65225890000336	2.61544246022642\\
2.68302485996611	2.60426140570847\\
2.71057344301215	2.59221854971585\\
2.73543192786561	2.57941415423297\\
2.75801972560871	2.56589961230934\\
2.77867238846366	2.55168820942244\\
2.79765920720138	2.53676195645932\\
2.81519610462298	2.52107557701103\\
2.83145499419961	2.50455831766021\\
2.84657039211788	2.48711396420712\\
2.86064379112477	2.46861923583169\\
2.87374608562724	2.44892055691462\\
2.88591814898051	2.42782904582382\\
2.89716947953687	2.40511338836097\\
2.90747462505364	2.38049005708748\\
2.91676683306505	2.35361006794347\\
2.92492801306316	2.32404109433784\\
2.93177356786976	2.29124323227207\\
2.93702985183273	2.25453595001245\\
2.940300775593	2.21305264996538\\
2.94101813331877	2.16567766229958\\
2.93836715219469	2.11095817986936\\
2.93117389733477	2.04698043291975\\
2.91773356609498	1.971195270361\\
2.89554727910843	1.88017399131407\\
2.86091934145589	1.76927385505018\\
2.80835074464214	1.63220393324588\\
2.72966595866595	1.4605325021307\\
2.61288926072011	1.24333100600682\\
2.44120489286289	0.967530144238057\\
2.19319903262895	0.620288794588344\\
1.84713034824417	0.195375852040556\\
1.39258091998237	-0.295881881399157\\
0.847097432884703	-0.815618147791448\\
0.262016003075326	-1.30678087031817\\
-0.297752906266213	-1.71958679315429\\
-0.784953348055309	-2.0334850455994\\
-1.18284747663061	-2.25558368389165\\
-1.49696516448511	-2.40566648821301\\
-1.74199732638048	-2.50417547685707\\
-1.93348132882357	-2.56736047948983\\
-2.08453129304809	-2.60676337977591\\
-2.20525285892648	-2.63018218870162\\
-2.303141400249	-2.64278522805763\\
-2.38368413694762	-2.64798585573839\\
-2.45089726047376	-2.64804768345522\\
-2.50773920845619	-2.64448267291651\\
-2.55641057010319	-2.6383081754167\\
-2.59856667747303	-2.63021237239539\\
-2.63546727246695	-2.62066123673404\\
-2.66808204557137	-2.60996823327106\\
-2.69716550575917	-2.59834013178333\\
-2.7233105241102	-2.58590733562874\\
-2.74698695784711	-2.57274402416935\\
-2.76856973595611	-2.55888146843808\\
-2.78835940702573	-2.54431666008423\\
-2.80659720941267	-2.52901761584986\\
-2.82347607789838	-2.51292621290439\\
-2.83914854982753	-2.49595906868554\\
-2.85373220922532	-2.47800673592301\\
-2.8673130622666	-2.45893129569143\\
-2.87994703717352	-2.43856226799976\\
-2.89165961838025	-2.41669059623414\\
-2.902443432702	-2.39306027588406\\
-2.91225337540026	-2.36735696387606\\
-2.92099855885655	-2.33919258990395\\
-2.92852993172743	-2.30808455024994\\
-2.9346217685964	-2.27342743290474\\
-2.93894423737863	-2.23445430645854\\
-2.94102270266134	-2.19018327078028\\
-2.94017697916224	-2.13934303530154\\
-2.93542987776555	-2.08026855077139\\
-2.92536828840323	-2.01075402551339\\
-2.90793065937658	-1.92784624848952\\
-2.88008112556126	-1.82755761831534\\
-2.83731364442982	-1.70448125563007\\
-2.77291809674371	-1.55131641033625\\
-2.67696845008103	-1.35840314840475\\
-2.53516236288159	-1.11361346343406\\
-2.32818792810975	-0.803494524702402\\
-2.0335326165553	-0.417390918102354\\
-1.63315184890276	0.0436749996658289\\
-1.12868460204515	0.55555786694248\\
-0.555370186023146	1.06841079251652\\
0.0246852414975189	1.52504568575042\\
0.552154009179338	1.88894204508582\\
0.995072749419049	2.15489125427073\\
1.34954746142368	2.33827777055983\\
1.62710857431764	2.46023536325952\\
1.84354179456123	2.53936050375954\\
2.01336098445757	2.58947345133322\\
2.1481605412785	2.62009781566314\\
2.25666811573677	2.63758949473537\\
2.34530095960704	2.64614792059426\\
2.41875121389676	2.64855056580978\\
};
\addplot [color=mycolor1, forget plot]
  table[row sep=crcr]{%
2.41112293345451	2.67293497202217\\
2.46888392391395	2.67115131797152\\
2.51815038737103	2.6664855191518\\
2.56067745902692	2.65971096292733\\
2.59779343839778	2.65135197568606\\
2.6305157273557	2.64176311344075\\
2.65963253350677	2.63118102002161\\
2.68576094455292	2.61975863731399\\
2.70938859486226	2.60758789630244\\
2.73090382849039	2.59471475314211\\
2.75061769692551	2.58114902195038\\
2.76878007240405	2.56687056330675\\
2.78559143844458	2.55183281196918\\
2.8012114227226	2.53596424504605\\
2.81576478577869	2.51916812567583\\
2.82934531927976	2.50132065759491\\
2.84201790198104	2.48226751893041\\
2.85381878116153	2.46181858333906\\
2.8647539665374	2.43974046017921\\
2.87479541554687	2.41574626808028\\
2.8838744197952	2.3894817673208\\
2.89187122478307	2.36050657363737\\
2.89859935644222	2.32826859854975\\
2.90378227390621	2.29206901942597\\
2.90701863342752	2.2510138451484\\
2.90773033309651	2.20394632395823\\
2.90508412791903	2.14935179353612\\
2.89787220142401	2.08522283788636\\
2.88432854807521	2.00886770995555\\
2.86184503952644	1.916639659751\\
2.82653307728443	1.80356260538734\\
2.77255796402825	1.66284105249546\\
2.69117522918125	1.4853017815288\\
2.56949523498214	1.25900114520614\\
2.38939363438971	0.969699040009248\\
2.12804201406743	0.60378361818778\\
1.76333966100236	0.155977290279563\\
1.28774343536765	-0.358089056184037\\
0.72595113667567	-0.893477459637415\\
0.136840561367889	-1.38815401247061\\
-0.413143683273551	-1.79385084762324\\
-0.881710561305588	-2.09580917159059\\
-1.25843603279283	-2.30612464980497\\
-1.55288592047271	-2.44682454430528\\
-1.7813060775171	-2.53866033056596\\
-1.9593657775584	-2.59741700772015\\
-2.09975332575718	-2.63403834464104\\
-2.21203071942399	-2.65581837430199\\
-2.3032006562605	-2.66755558712192\\
-2.37835240081262	-2.67240733578158\\
-2.44119571607529	-2.67246443748957\\
-2.49445761919681	-2.66912333754563\\
-2.54016488193194	-2.66332432155596\\
-2.57984205540937	-2.6557040973601\\
-2.61464993480629	-2.64669419424165\\
-2.6454827915952	-2.63658501085421\\
-2.67303717734763	-2.62556789562519\\
-2.69786105735113	-2.61376299788296\\
-2.72038922461991	-2.60123775195561\\
-2.74096903869299	-2.58801907026067\\
-2.75987924735955	-2.57410120073672\\
-2.77734377856985	-2.55945048935761\\
-2.793541793656	-2.54400782159856\\
-2.80861487682785	-2.52768920006705\\
-2.82267193569293	-2.51038468758181\\
-2.83579215890868	-2.491955765212\\
-2.84802618740108	-2.47223099410424\\
-2.85939547818274	-2.45099970448246\\
-2.86988964901186	-2.42800324168044\\
-2.87946135839992	-2.40292305034405\\
-2.8880179592727	-2.37536453831611\\
-2.89540870763175	-2.34483518069009\\
-2.90140561910968	-2.31071462828254\\
-2.90567500103259	-2.27221356472263\\
-2.90773500983351	-2.22831655519435\\
-2.90689191136059	-2.17770193064862\\
-2.90214344552749	-2.11862858948262\\
-2.8920308887691	-2.04877526035613\\
-2.87441080959479	-1.96501245936017\\
-2.84610196161767	-1.86308285568593\\
-2.8023432419264	-1.73716857524884\\
-2.7359855348757	-1.5793538010083\\
-2.63637493654125	-1.37909972675918\\
-2.48809271214243	-1.12315205734959\\
-2.27038916682889	-0.796974956580139\\
-1.95964380346866	-0.389789045959459\\
-1.53877736296124	0.0949035646598881\\
-1.01455397500912	0.626922435424875\\
-0.430388009563244	1.14961064156151\\
0.146485075591149	1.60386358275269\\
0.65890842805611	1.95746634107526\\
1.08124245096797	2.21110290073615\\
1.41497076391917	2.38377830939657\\
1.67431219460888	2.49773918811789\\
1.87576040675745	2.57138888298126\\
2.03360307134768	2.61796802131199\\
2.15891606343668	2.64643668053227\\
2.25989814780917	2.66271445002484\\
2.34251989232733	2.67069163243849\\
2.41112293345451	2.67293497202217\\
};
\addplot [color=mycolor1, forget plot]
  table[row sep=crcr]{%
2.40184832977892	2.69490294654388\\
2.45592399283202	2.69323249711691\\
2.50214465959359	2.68885462745153\\
2.54212772027489	2.68248487007762\\
2.57709765951613	2.67460878682083\\
2.60799303796255	2.66555490609426\\
2.63554167666137	2.65554239876409\\
2.6603140114805	2.64471252194086\\
2.68276133876676	2.6331494654426\\
2.7032434880069	2.62089414511776\\
2.72204899549531	2.60795318726871\\
2.73940987242113	2.59430452761714\\
2.75551239638478	2.57990051837748\\
2.77050489726047	2.56466908339177\\
2.78450318336705	2.54851321266359\\
2.79759401240609	2.53130889797802\\
2.80983681818068	2.51290144869255\\
2.82126373107969	2.49309996625585\\
2.83187775268152	2.47166957438063\\
2.84164873539826	2.44832077323747\\
2.85050654188269	2.42269497733214\\
2.85833036546549	2.39434486122415\\
2.86493260415113	2.36270750643164\\
2.87003477162677	2.32706741444136\\
2.87323149515886	2.28650507241896\\
2.87393635669052	2.23982470919166\\
2.87129963211028	2.18545186014563\\
2.86408200330457	2.12128703196035\\
2.85045876424552	2.04449595538954\\
2.82771432238764	1.95121039336567\\
2.79176614553139	1.8361102048244\\
2.73643546713606	1.69187094277892\\
2.65238485586856	1.5085310432453\\
2.52576058108595	1.27305716438514\\
2.33705616610302	0.969957779644263\\
2.06200028345344	0.584866453040137\\
1.67835833154431	0.113785735228833\\
1.18214610812648	-0.42263604568819\\
0.605896103323672	-0.971925405430728\\
0.0155723827383446	-1.46775406562183\\
-0.522312519362957	-1.86462709186122\\
-0.971324398289734	-2.15404224631969\\
-1.32715043845893	-2.35271822870257\\
-1.60279530089054	-2.48444397602867\\
-1.81560852479424	-2.57000911025981\\
-1.98117106386415	-2.62464285417094\\
-2.1116732147888	-2.65868520750944\\
-2.21612975662973	-2.67894743218132\\
-2.30107153790832	-2.68988207556836\\
-2.37121551086357	-2.69440981171719\\
-2.42998826643122	-2.69446257712194\\
-2.4799043774965	-2.69133079768861\\
-2.52283159053019	-2.68588400343966\\
-2.56017497653528	-2.67871156117301\\
-2.59300495343547	-2.67021324110272\\
-2.62214686783574	-2.66065811175114\\
-2.64824425085508	-2.65022322168541\\
-2.6718039350366	-2.63901919428186\\
-2.69322855158082	-2.6271071999645\\
-2.7128401395691	-2.61451012441759\\
-2.73089740259439	-2.60121972067263\\
-2.74760834240366	-2.58720087567082\\
-2.76313944898201	-2.57239369124244\\
-2.77762224240739	-2.55671378533672\\
-2.79115768371606	-2.54005100487191\\
-2.80381875835273	-2.52226656886728\\
-2.81565135572206	-2.50318850185678\\
-2.82667339617656	-2.48260504946375\\
-2.83687196698051	-2.46025556649031\\
-2.84619799109553	-2.43581810407097\\
-2.85455762548908	-2.40889255730016\\
-2.86179910668912	-2.37897771193176\\
-2.86769303203279	-2.3454397646135\\
-2.87190292593754	-2.30746876063673\\
-2.8739411298695	-2.26401771203514\\
-2.87310214305253	-2.2137166667641\\
-2.86836083396254	-2.15475036650014\\
-2.85821536473773	-2.08468305677749\\
-2.84044272991812	-2.00020763943901\\
-2.81171705992515	-1.8967905929009\\
-2.76701826400222	-1.7681864806928\\
-2.69874331942785	-1.60582994541959\\
-2.59547450038752	-1.39824243423878\\
-2.44061218668422	-1.13095957820658\\
-2.21190189403025	-0.78830978415099\\
-1.88467872020738	-0.359531309080791\\
-1.44327960281451	0.148851239629981\\
-0.900402958899709	0.699900932966638\\
-0.307870210739911	1.23020995198984\\
0.263072675579177	1.67991415655052\\
0.758799419972088	2.02207454581275\\
1.16029175073245	2.26323582897871\\
1.47391126142261	2.42552507639461\\
1.71600758027358	2.53191511984887\\
1.90345289959078	2.60044748272965\\
2.050174427334	2.64374495208587\\
2.16669927035574	2.67021656755708\\
2.26070957776861	2.6853697803666\\
2.33775391788163	2.6928077177244\\
2.40184832977892	2.69490294654388\\
};
\addplot [color=mycolor1, forget plot]
  table[row sep=crcr]{%
2.39105565046959	2.71466394891098\\
2.44168643913923	2.7130993739671\\
2.48505087718409	2.70899156370742\\
2.52264008371797	2.70300275790928\\
2.5555834337588	2.69558274045283\\
2.58474724027934	2.68703595981662\\
2.61080394512603	2.67756538313932\\
2.63428115432691	2.66730141297161\\
2.65559676206356	2.6563210478661\\
2.67508435447998	2.64466053875074\\
2.69301172311061	2.63232359642322\\
2.70959440943117	2.61928645031054\\
2.72500558850878	2.60550057022014\\
2.73938317702514	2.59089353517344\\
2.75283475043425	2.57536830065925\\
2.76544062920564	2.55880093494172\\
2.7772553117192	2.54103673622695\\
2.78830726502361	2.5218844812342\\
2.79859690981943	2.50110836859014\\
2.80809242516848	2.47841698035865\\
2.8167227152342	2.4534482560457\\
2.82436647131996	2.42574900413702\\
2.83083564371333	2.39474678989435\\
2.83585067333367	2.35971101846401\\
2.8390032983272	2.31969850350355\\
2.83970027217576	2.27347651354509\\
2.83707728986047	2.21941285894894\\
2.82986581942565	2.15531758873091\\
2.81618486650372	2.07821402943217\\
2.79321303813971	1.98400893839282\\
2.75667255046015	1.86702687505075\\
2.70003133071598	1.71938838577005\\
2.61333263100642	1.53029269807504\\
2.48170337858554	1.28553493342318\\
2.28417482600422	0.968284901730114\\
1.99500961002305	0.563450340431757\\
1.592112701967	0.0687015669963673\\
1.07581729000886	-0.489513382033007\\
0.487110705977186	-1.05080876082787\\
-0.101641479202603	-1.54545169258212\\
-0.625362480657865	-1.93196938522036\\
-1.05415748912718	-2.20840572355328\\
-1.38948135606153	-2.39565815264789\\
-1.64718968517319	-2.51882184377782\\
-1.84534326727985	-2.59849595956324\\
-1.99925943132172	-2.64928709852718\\
-2.12058058446563	-2.68093414564767\\
-2.21777859831219	-2.69978770858757\\
-2.29693437442552	-2.70997680820576\\
-2.36241671214485	-2.71420298673928\\
-2.4173901737337	-2.71425176296527\\
-2.46417386728752	-2.71131600828767\\
-2.50448964864668	-2.70620012054386\\
-2.53963310878231	-2.69944981929434\\
-2.57059186364337	-2.69143552820743\\
-2.59812807799416	-2.68240655774621\\
-2.62283663401238	-2.67252668224414\\
-2.64518657839968	-2.66189767317864\\
-2.6655509604152	-2.65057488898442\\
-2.68422850198075	-2.6385775051426\\
-2.70145943030626	-2.62589502007647\\
-2.71743705844548	-2.61249106726922\\
-2.73231619149175	-2.59830516623178\\
-2.74621908146117	-2.58325277124674\\
-2.7592393960627	-2.5672237744592\\
-2.77144446655925	-2.55007945339229\\
-2.7828759086264	-2.5316476957986\\
-2.79354854264693	-2.51171616361689\\
-2.80344735085607	-2.49002284799106\\
-2.81252196710058	-2.46624318803999\\
-2.82067785705647	-2.43997253467787\\
-2.82776284595137	-2.4107021746446\\
-2.83354688066196	-2.37778629462227\\
-2.83769169898016	-2.34039601769445\\
-2.83970513134109	-2.29745476968859\\
-2.83887159838726	-2.2475464208599\\
-2.83414520261156	-2.18878349072564\\
-2.82398340099586	-2.11861679335381\\
-2.80608582235899	-2.03356028041366\\
-2.77698255603545	-1.92879753547014\\
-2.73139010013438	-1.79763801733551\\
-2.6612351704066	-1.63082968596641\\
-2.55429669422483	-1.41588772712857\\
-2.39272325480453	-1.13704585751673\\
-2.15268490071564	-0.77744335145647\\
-1.8085585557974	-0.326512109778409\\
-1.34662019034505	0.205578402133016\\
-0.786342626004456	0.774402472667871\\
-0.188013121231037	1.31003964582746\\
0.374409853793793	1.75315073077823\\
0.852071731470989	2.08291484474334\\
1.23266559621533	2.3115591569358\\
1.52687413548001	2.46381909057387\\
1.75266676341099	2.56305005688437\\
1.92702017200067	2.62679740425486\\
2.06339979492001	2.66704294905067\\
2.17176767297531	2.69166092518201\\
2.2593053602677	2.70577013195945\\
2.33116355135832	2.71270670876987\\
2.39105565046959	2.71466394891098\\
};
\addplot [color=mycolor1, forget plot]
  table[row sep=crcr]{%
2.37884203763194	2.73240879223995\\
2.4262463534695	2.73094342879247\\
2.46692757621165	2.72708936391518\\
2.50226073696543	2.72145961913605\\
2.53328781311229	2.71447087784343\\
2.56080871004439	2.70640526457967\\
2.5854449290535	2.69745069568468\\
2.60768463890829	2.68772748032498\\
2.62791494206404	2.67730593553107\\
2.6464452044794	2.6662179994832\\
2.66352405277528	2.65446472577958\\
2.6793518021746	2.64202084628609\\
2.69408951242514	2.62883713957879\\
2.70786547887196	2.61484103799892\\
2.72077968780334	2.59993568796996\\
2.73290655569953	2.58399750554662\\
2.74429609976839	2.5668721135433\\
2.75497352690937	2.5483683842569\\
2.76493705589793	2.52825011894172\\
2.77415357525633	2.50622464329364\\
2.78255144923767	2.48192724851407\\
2.79000936000601	2.4548999034889\\
2.79633942553089	2.42456191960397\\
2.80126181244299	2.39016913416001\\
2.80436642492361	2.35075648925176\\
2.80505458049549	2.30505631604419\\
2.80244918709069	2.25138075418868\\
2.79525467387766	2.18745099670203\\
2.78153604010664	2.11014802606856\\
2.758367560554	2.01514983613783\\
2.7212744512416	1.89641365784961\\
2.66336095320443	1.74547682738926\\
2.57402172341646	1.55064360100373\\
2.43730295388958	1.29644711496952\\
2.23068579536726	0.964623854548406\\
1.92695107239325	0.53940119782852\\
1.50447501405737	0.0205785582624191\\
0.968746023487389	-0.558742255112305\\
0.369736970208452	-1.13000185512183\\
-0.214705012903247	-1.62115746921311\\
-0.722436941866099	-1.9959639182891\\
-1.13057305149817	-2.25912794505457\\
-1.44588359012524	-2.43522464577601\\
-1.68651158363303	-2.55023300337369\\
-1.87089179250685	-2.62437145274093\\
-2.01394001635376	-2.67157654255079\\
-2.1267181216797	-2.70099465656543\\
-2.21716486116874	-2.71853801731161\\
-2.29093332892937	-2.72803299245666\\
-2.35206672882154	-2.7319778944991\\
-2.40348677360069	-2.73202299109313\\
-2.44733230308416	-2.72927115191948\\
-2.48519096075237	-2.72446665978726\\
-2.51825772237007	-2.71811489139163\\
-2.54744411116073	-2.71055909230139\\
-2.57345419832581	-2.70203023728123\\
-2.59683809457397	-2.69267976219929\\
-2.61803003408079	-2.68260120792063\\
-2.63737578015784	-2.67184454230672\\
-2.65515252366643	-2.66042552932652\\
-2.67158341559993	-2.64833164167971\\
-2.68684818699216	-2.63552545590418\\
-2.70109084093572	-2.62194610110999\\
-2.71442507394136	-2.60750907732493\\
-2.7269378446073	-2.59210456801815\\
-2.73869132010812	-2.57559421036609\\
-2.74972326774459	-2.55780613071517\\
-2.76004579562609	-2.53852787825735\\
-2.76964215815257	-2.51749667161289\\
-2.7784610964171	-2.49438607796518\\
-2.78640783518803	-2.46878782609997\\
-2.79333033579016	-2.44018684377489\\
-2.79899859316409	-2.40792670023395\\
-2.80307347538253	-2.37116126318891\\
-2.80505951574642	-2.32878629733392\\
-2.80423264437882	-2.27934157092956\\
-2.79952819480964	-2.22086929786549\\
-2.78936520895674	-2.15070788412002\\
-2.7713680059692	-2.06519087440544\\
-2.74192293759335	-1.95921180708588\\
-2.69547797082562	-1.82561616358676\\
-2.62347135048132	-1.65442475362061\\
-2.51283474043797	-1.43207320182174\\
-2.34438634992853	-1.1413920454077\\
-2.09264625946526	-0.76427828495814\\
-1.73114810967095	-0.2905765432957\\
-1.24871320700027	0.265184724864449\\
-0.672449713736175	0.850361898452582\\
-0.0709690600813887	1.38896482709428\\
0.480509122770547	1.82356632238975\\
0.938992152198613	2.14015562258572\\
1.29878931037204	2.35633720929923\\
1.57431739364831	2.4989422377873\\
1.78470493929849	2.59140743991512\\
1.94680749926314	2.65067669043398\\
2.0735538240495	2.68807927237218\\
2.1743349153968	2.71097316510059\\
2.2558500138841	2.72411098961202\\
2.3228751080575	2.73058038763443\\
2.37884203763194	2.73240879223995\\
};
\addplot [color=mycolor1, forget plot]
  table[row sep=crcr]{%
2.36527365438272	2.74830870908072\\
2.40964991746339	2.7469364995168\\
2.44780579915534	2.74332128955086\\
2.48100929698089	2.73803052651152\\
2.51022183713887	2.73145019589874\\
2.53618210125969	2.72384168192079\\
2.55946459271266	2.71537889659003\\
2.58052106015224	2.70617275202405\\
2.59971013934627	2.69628735428491\\
2.61731878089818	2.68575065701308\\
2.63357785800937	2.6745612993535\\
2.64867357201913	2.66269271269721\\
2.6627557512408	2.65009516481937\\
2.67594377880519	2.63669612760843\\
2.68833062790302	2.62239914930745\\
2.69998528743098	2.60708124673624\\
2.71095369814491	2.59058868007056\\
2.7212581648169	2.57273080906794\\
2.73089503983224	2.55327153086106\\
2.73983025987883	2.5319175355474\\
2.74799202070095	2.50830224501375\\
2.75525943556605	2.48196376072738\\
2.76144534500368	2.45231434242239\\
2.7662703691099	2.4185977235074\\
2.76932355026417	2.37982870970839\\
2.77000206564684	2.33470665069354\\
2.76741771764363	2.28149000033681\\
2.76024993931903	2.2178126088649\\
2.74651183528815	2.14041300741655\\
2.72317454125283	2.04473625024061\\
2.68556393474381	1.92435911002144\\
2.62640871451058	1.77020478075194\\
2.53442204404956	1.56962081682207\\
2.39250057485056	1.30577768731714\\
2.17647867610017	0.958875721384523\\
1.8576501387814	0.512529816055839\\
1.41526549058398	-0.030780494230754\\
0.860889959227299	-0.630372554180624\\
0.253891702488302	-1.20940152139326\\
-0.323558108606942	-1.69481349550972\\
-0.813700808611469	-2.05671810266489\\
-1.20091960518074	-2.30643522281392\\
-1.4967680434039	-2.47167953487244\\
-1.72114720592667	-2.57892790820529\\
-1.89257879775753	-2.64786151485634\\
-2.02546999690482	-2.69171490991658\\
-2.13028266636831	-2.71905474556955\\
-2.21443591196754	-2.73537676422754\\
-2.28317683298629	-2.74422401082905\\
-2.3402439663865	-2.74790596622875\\
-2.38833355093365	-2.7479476596335\\
-2.42941776898398	-2.74536870731669\\
-2.46496045199247	-2.74085775992939\\
-2.49606383485234	-2.73488281086318\\
-2.52356930303391	-2.72776187791145\\
-2.54812735187924	-2.71970888677516\\
-2.57024676673503	-2.7108637874166\\
-2.59032961545124	-2.70131246095496\\
-2.60869642198736	-2.69109987555867\\
-2.6256044407356	-2.68023866329648\\
-2.6412609982819	-2.66871448687934\\
-2.65583323412694	-2.65648905152365\\
-2.66945514000038	-2.64350127674358\\
-2.68223249488475	-2.6296669046439\\
-2.69424607072992	-2.61487663978795\\
-2.70555330792793	-2.59899275958483\\
-2.71618850375176	-2.58184397872029\\
-2.72616139775262	-2.56321817320308\\
-2.73545385022647	-2.54285234259458\\
-2.74401406053032	-2.52041887789477\\
-2.75174741329837	-2.49550675676421\\
-2.75850249706639	-2.46759563065094\\
-2.76404998802014	-2.43601978130414\\
-2.76805072479322	-2.39991742195724\\
-2.77000706768743	-2.35815851420924\\
-2.76918793999981	-2.30924073259587\\
-2.76451178202781	-2.25113782679321\\
-2.75436136354981	-2.18107670228624\\
-2.73628754151374	-2.09520878021411\\
-2.70653284298508	-1.98812972299983\\
-2.65927065768527	-1.85220040887384\\
-2.58543021080492	-1.67666991121794\\
-2.47104662203931	-1.44681282868073\\
-2.29552004164257	-1.14394425989429\\
-2.03164242041711	-0.748667517327314\\
-1.65225577728419	-0.251513538736312\\
-1.14943023701283	0.327810235700239\\
-0.558776498007111	0.927735895787799\\
0.0431409482925311	1.46687835816249\\
0.581416088700065	1.89118337284762\\
1.01983159465553	2.19397491574896\\
1.35905688462949	2.39782266432595\\
1.61664699951616	2.53115395667853\\
1.8124797275961	2.61722576532144\\
1.96310543470836	2.67229944803121\\
2.08086225396267	2.70704894646139\\
2.17457171154062	2.72833581772816\\
2.2504702101494	2.74056777650213\\
2.31298100828986	2.74660085820289\\
2.36527365438272	2.74830870908072\\
};
\addplot [color=mycolor1, forget plot]
  table[row sep=crcr]{%
2.3503853119908	2.76251494643398\\
2.39191393217561	2.76123037943232\\
2.42768859660251	2.75784042750851\\
2.45887833723077	2.75287022845848\\
2.48637012136388	2.74667722880223\\
2.510846019112	2.73950350164073\\
2.53283704608621	2.73150990440398\\
2.55276121997183	2.72279858834094\\
2.57095078490094	2.71342788164366\\
2.58767188916159	2.70342205775786\\
2.60313891570007	2.69277756784621\\
2.61752494851065	2.68146672721784\\
2.63096937682453	2.66943946143598\\
2.64358330724213	2.656623455247\\
2.65545321580629	2.64292285417333\\
2.66664308967188	2.62821550969925\\
2.67719515377777	2.61234860840996\\
2.68712912855599	2.59513236017421\\
2.69643979677686	2.57633121563916\\
2.70509244243733	2.55565180726134\\
2.71301542150625	2.53272641584916\\
2.72008867039546	2.50709018833066\\
2.7261262507295	2.47814946676669\\
2.73084989653595	2.44513726724934\\
2.73384867932385	2.40704990747715\\
2.73451683301892	2.36255561318433\\
2.73195661543024	2.30986101940349\\
2.72482435363388	2.2465139818327\\
2.711083155337	2.16911018188178\\
2.68760184537471	2.07285693063183\\
2.64950384721276	1.95093593613899\\
2.58912858477636	1.79362188109626\\
2.49447036384301	1.58723660965141\\
2.34719855894177	1.31347506200585\\
2.12139403076715	0.950889813008775\\
1.78687361851418	0.482581582351276\\
1.32425335231122	-0.0856291170117338\\
0.752183398073065	-0.704480587244766\\
0.139677587891934	-1.28892186282411\\
-0.42816231067355	-1.76638553844743\\
-0.899324293288053	-2.11435121474952\\
-1.26551887600693	-2.35054509822281\\
-1.54249592627137	-2.50526302268155\\
-1.7514250245256	-2.60513147346486\\
-1.91067266168712	-2.66916706266552\\
-2.03405548013089	-2.70988262961109\\
-2.13142588241301	-2.73528065477131\\
-2.20969916046123	-2.75046160356716\\
-2.27373753915316	-2.75870303328772\\
-2.3269942216892	-2.76213862713827\\
-2.37195570736334	-2.7621771646753\\
-2.41043972602469	-2.75976104871229\\
-2.44379557720665	-2.75552731042341\\
-2.47303976527881	-2.74990921479259\\
-2.49894883600904	-2.74320130790478\\
-2.52212372944933	-2.73560163068733\\
-2.54303497524366	-2.72723942343221\\
-2.5620548349764	-2.71819343096336\\
-2.57948042108719	-2.70850398289026\\
-2.59555047760972	-2.69818083952949\\
-2.61045762762849	-2.68720805526125\\
-2.62435730680135	-2.67554663761915\\
-2.63737420432198	-2.66313546515162\\
-2.6496067533685	-2.64989070435837\\
-2.66113000678477	-2.63570379354718\\
-2.67199706851311	-2.62043790966508\\
-2.68223910229461	-2.60392267912295\\
-2.69186378362787	-2.5859467119911\\
-2.70085187355923	-2.56624730312205\\
-2.70915133990608	-2.54449631638781\\
-2.71666808282094	-2.52028079436336\\
-2.72325175723032	-2.49307613112037\\
-2.72867429229312	-2.46220857806186\\
-2.73259726361307	-2.42680221280929\\
-2.73452189320905	-2.38570295979474\\
-2.73371147248218	-2.33736830171394\\
-2.72906928957888	-2.27970522948764\\
-2.71894380765585	-2.20982984310806\\
-2.70081399747552	-2.12370930394393\\
-2.67077795434588	-2.01563260904588\\
-2.62272724530581	-1.87745314027105\\
-2.54705860167537	-1.69759855679207\\
-2.42885477702257	-1.46009112163033\\
-2.24599939558024	-1.14460545563963\\
-1.96947512171323	-0.730403961985268\\
-1.57163178709569	-0.209047206830658\\
-1.04860508517012	0.3936365298367\\
-0.445360833644853	1.00649873650033\\
0.154214806613381	1.54369452870495\\
0.67719354757618	1.95604462684509\\
1.09485028267557	2.24455198983996\\
1.41382211716279	2.43625168009219\\
1.65421359806668	2.5606892612365\\
1.8362909434323	2.6407179570803\\
1.97615061287271	2.69185560424471\\
2.08550273385151	2.7241245251592\\
2.17260633020613	2.74391022237171\\
2.24325488576203	2.7552954844211\\
2.30153959553947	2.76092014780299\\
2.3503853119908	2.76251494643398\\
};
\addplot [color=mycolor1, forget plot]
  table[row sep=crcr]{%
2.33417922251919	2.77515857935272\\
2.37302446345421	2.77395663381085\\
2.40654963067401	2.77077951041994\\
2.43583199686454	2.76611296825101\\
2.46168950562335	2.76028785819516\\
2.48475173000904	2.7535282337604\\
2.50550933109755	2.74598276018145\\
2.5243489983959	2.73774540873181\\
2.54157844074649	2.72886912755511\\
2.55744444898051	2.71937479102455\\
2.57214604857724	2.70925687233551\\
2.58584410069049	2.6984867418507\\
2.59866826704092	2.68701413897711\\
2.61072194848004	2.67476712102847\\
2.62208558667726	2.66165061021449\\
2.63281854814552	2.64754350692396\\
2.6429596636201	2.6322941888767\\
2.65252635132696	2.61571404867556\\
2.6615120867452	2.59756851130044\\
2.66988176493976	2.57756468474824\\
2.6775641922802	2.55533438297514\\
2.68444047616295	2.53041064659606\\
2.69032634517252	2.50219495706871\\
2.69494524328088	2.46991090852706\\
2.69788708180155	2.4325378702118\\
2.69854424714761	2.38871466854988\\
2.69601088276543	2.33659781368121\\
2.68892192270199	2.27365026140483\\
2.67519210905055	2.19632400672814\\
2.65158830987743	2.09958293128851\\
2.61302737146264	1.9761971289107\\
2.55144333649516	1.81575407886478\\
2.45406877280891	1.60347200034296\\
2.30125726814113	1.31944292483288\\
2.06521822926748	0.94045104662107\\
1.71432392256784	0.44922297385281\\
1.23115674064735	-0.144283789810344\\
0.64254576172766	-0.781166412816022\\
0.0271944808310615	-1.36848884553553\\
-0.528487228620339	-1.83585571054603\\
-0.979468638449927	-2.16898652663144\\
-1.32465584313549	-2.39166122610503\\
-1.58337414476832	-2.5361915531421\\
-1.77761447908935	-2.62904253610715\\
-1.92538542429413	-2.68846398527979\\
-2.039851616804	-2.72623686500676\\
-2.13025406291663	-2.74981681312022\\
-2.20302147092732	-2.76392931196305\\
-2.26265139697322	-2.77160284798201\\
-2.31232951967749	-2.77480711141059\\
-2.35434685090081	-2.77484271481674\\
-2.39037762959059	-2.77258026463459\\
-2.42166492224548	-2.76860877200978\\
-2.44914575953804	-2.76332915972775\\
-2.47353657346962	-2.75701408575001\\
-2.495392336303	-2.74984676076529\\
-2.51514806328149	-2.74194642214541\\
-2.5331483197224	-2.73338515779818\\
-2.54966844276935	-2.72419899145912\\
-2.56492994429187	-2.71439505253689\\
-2.57911174959065	-2.7039559742482\\
-2.59235838752929	-2.69284222744608\\
-2.60478588133654	-2.68099280546934\\
-2.61648583152902	-2.66832446685929\\
-2.62752799079809	-2.65472957859868\\
-2.63796147539942	-2.64007245461603\\
-2.64781461497922	-2.62418392938969\\
-2.65709329070221	-2.60685372101366\\
-2.66577742457177	-2.58781989333687\\
-2.67381502638283	-2.56675438293266\\
-2.68111282640781	-2.54324305418456\\
-2.68752193708849	-2.51675799216027\\
-2.69281605417463	-2.48661859130609\\
-2.69665818485541	-2.45193621304128\\
-2.69854935756574	-2.41153439090227\\
-2.6977484920616	-2.36383216765948\\
-2.69314531457329	-2.30667127533311\\
-2.68305573365365	-2.23705737709181\\
-2.66488806280295	-2.15077062614394\\
-2.63459468046404	-2.04178326316977\\
-2.58577654841186	-1.90141537153563\\
-2.50827077513128	-1.7172172108897\\
-2.38614393387481	-1.47185548805787\\
-2.19565192404784	-1.1432245218687\\
-1.90588545328172	-0.709206749592182\\
-1.48896435522745	-0.162825728991575\\
-0.946038384383019	0.462888162769359\\
-0.332236635993454	1.0866376214684\\
0.262157127043303	1.6193428477606\\
0.767906914478278	2.01820505971951\\
1.16428518567672	2.2920606042203\\
1.46339148439475	2.47184040961561\\
1.68730986127867	2.58775757832882\\
1.85638013965914	2.66207117916182\\
1.98612588689922	2.70951094253041\\
2.08760480900204	2.73945608556077\\
2.16852420162778	2.75783643569324\\
2.23425447519047	2.76842852203312\\
2.28857408052101	2.77367002737662\\
2.33417922251919	2.77515857935272\\
};
\addplot [color=mycolor1, forget plot]
  table[row sep=crcr]{%
2.31662243599076	2.78635024525841\\
2.35293417322862	2.78522633722784\\
2.38433045880942	2.78225065824871\\
2.41180325991813	2.7778722261458\\
2.43610635920612	2.77239704930667\\
2.45782051375344	2.76603233073719\\
2.47739883494592	2.75891532736314\\
2.49519883713148	2.75113236092503\\
2.51150535913619	2.74273136444641\\
2.52654713200155	2.73373007205922\\
2.54050884222562	2.72412117266028\\
2.55353993303382	2.71387524921832\\
2.56576097919128	2.70294199797262\\
2.57726818947959	2.69124999428929\\
2.58813638716678	2.67870509974901\\
2.59842065983668	2.66518745748042\\
2.60815673126147	2.65054687583518\\
2.61735996812363	2.63459623152095\\
2.62602277037307	2.61710230605039\\
2.63410987622555	2.59777316859469\\
2.64155079762947	2.57624078212906\\
2.64822811992764	2.55203685768825\\
2.65395963466464	2.52455898553249\\
2.6584710289845	2.4930225250869\\
2.66135378373019	2.45639130137023\\
2.66199942680694	2.41327629119163\\
2.65949527222369	2.36178533355917\\
2.6524563644985	2.29929721566371\\
2.63875037745638	2.22211878641674\\
2.61504196243896	2.12496360453309\\
2.57603596220367	2.00017105922659\\
2.51324192059867	1.83659727800389\\
2.41308093908502	1.61826797727075\\
2.25448927532998	1.32352756237873\\
2.00767471046191	0.927262586440358\\
1.6396299310979	0.412023553211461\\
1.13564164668971	-0.207132141700543\\
0.531891017170401	-0.860550597649819\\
-0.0834486858662258	-1.44803462631231\\
-0.624497145372981	-1.90321556429516\\
-1.05427299723325	-2.22074456387053\\
-1.37856985036945	-2.4299693354227\\
-1.61965084392854	-2.56465624946155\\
-1.79992400599472	-2.65083349015314\\
-1.93687163112228	-2.70590311341063\\
-2.04296139720988	-2.74091147029444\\
-2.12682657384936	-2.76278570212073\\
-2.19442722084214	-2.77589557972898\\
-2.24991539535961	-2.78303561167106\\
-2.2962256307893	-2.78602219784378\\
-2.33546637180161	-2.78605506640721\\
-2.36917823068472	-2.78393789758166\\
-2.39850548094883	-2.78021491926245\\
-2.42431127936886	-2.77525686174071\\
-2.44725617194405	-2.7693159257264\\
-2.46785237352234	-2.76256144815528\\
-2.48650184003541	-2.75510330734979\\
-2.50352333224779	-2.74700737306404\\
-2.51917188068512	-2.7383056689549\\
-2.53365291367801	-2.72900291552327\\
-2.54713256339255	-2.7190804973541\\
-2.55974516930548	-2.7084984965352\\
-2.57159866170076	-2.69719616348986\\
-2.58277827002371	-2.68509100112774\\
-2.59334882299732	-2.67207648161534\\
-2.60335576135157	-2.65801827063435\\
-2.6128248474056	-2.64274867898989\\
-2.62176040693981	-2.62605887200621\\
-2.6301417522632	-2.60768811336371\\
-2.6379171756221	-2.58730895947424\\
-2.64499451442092	-2.56450678882722\\
-2.65122668480172	-2.53875124671864\\
-2.65638960695603	-2.50935594678835\\
-2.66014834317394	-2.47542083303262\\
-2.66200458033747	-2.43574854134014\\
-2.66121400291597	-2.38872122347622\\
-2.65665419266781	-2.33211656121882\\
-2.64660999519241	-2.26282968154675\\
-2.62841985146878	-2.17645012552654\\
-2.59788825718166	-2.06662154701748\\
-2.54831481543425	-1.92410119775107\\
-2.46894529719174	-1.73549809709094\\
-2.34275646897583	-1.48200569342412\\
-2.14425031032493	-1.1395812012755\\
-1.8405441962249	-0.68470252093548\\
-1.40387341614752	-0.112406976598289\\
-0.841502519951627	0.535834141894993\\
-0.219445282773875	1.16814747973172\\
0.366866749718658	1.69376187769669\\
0.853610391505406	2.07772463285486\\
1.22833872432155	2.33666348088705\\
1.50801765790865	2.50478235141778\\
1.71616753137343	2.61254193101756\\
1.87292919968403	2.68144670794859\\
1.99315921357654	2.72540708647408\\
2.08724855622801	2.75317114021806\\
2.16236616371458	2.770233056107\\
2.22347880009503	2.78008048176229\\
2.27407015977558	2.78496175200151\\
2.31662243599076	2.78635024525841\\
};
\addplot [color=mycolor1, forget plot]
  table[row sep=crcr]{%
2.29764238025503	2.79617948043294\\
2.33155776382023	2.7951294127764\\
2.36093593396223	2.79234472148207\\
2.38668934938194	2.78824006338051\\
2.40951199610384	2.78309819003935\\
2.4299391191885	2.77711051459377\\
2.44838879937177	2.77040359931626\\
2.46519131130015	2.76305660106635\\
2.48061012244609	2.75511277219028\\
2.49485707323489	2.74658694414486\\
2.50810342978124	2.73747019904336\\
2.5204879439354	2.72773247612557\\
2.53212268191843	2.71732355777748\\
2.54309712414534	2.70617266786296\\
2.55348085049531	2.69418675228871\\
2.56332497685799	2.68124736911181\\
2.57266237715152	2.66720596998822\\
2.58150658947687	2.65187718369467\\
2.58984914288974	2.63502948882822\\
2.5976548223296	2.61637234942624\\
2.60485406837627	2.59553842849492\\
2.61133121281677	2.57205880296976\\
2.61690645784271	2.54532803886171\\
2.62130820407685	2.51455431738734\\
2.62413014725095	2.47868715322246\\
2.62476382931063	2.43631099338295\\
2.62229085265392	2.38548613158552\\
2.61530762106444	2.32350741071506\\
2.60163561168301	2.24653420776192\\
2.57783619728613	2.14902122911547\\
2.53839509382159	2.02285473924832\\
2.47437437686926	1.85610842486001\\
2.37132537767582	1.63151301647124\\
2.20664964662485	1.32549983636793\\
1.94841036224136	0.910921515403966\\
1.5623335377777	0.370431734902438\\
1.03732013187301	-0.274642951463033\\
0.420138859990144	-0.942770060131745\\
-0.192120709557591	-1.52749140993308\\
-0.716137112867406	-1.96845936081391\\
-1.12384148689449	-2.26973704776009\\
-1.42744562508792	-2.46563371112109\\
-1.65151005015505	-2.59082142391287\\
-1.81849752043933	-2.67064968374397\\
-1.94522530502567	-2.72160981845203\\
-2.0434324665042	-2.75401654034908\\
-2.12115231049674	-2.77428731292085\\
-2.18389439956994	-2.78645439868368\\
-2.23548337572101	-2.79309220063738\\
-2.27861768269272	-2.7958735475304\\
-2.31523492728519	-2.79590386087272\\
-2.3467509926555	-2.7939242853969\\
-2.37421804826164	-2.79043718477528\\
-2.39843040416711	-2.78578503878942\\
-2.41999651456956	-2.78020088696098\\
-2.43938871911307	-2.77384106170873\\
-2.45697812031413	-2.76680666955355\\
-2.47305937586617	-2.75915776364104\\
-2.48786853217569	-2.75092264808629\\
-2.50159596948731	-2.74210383819441\\
-2.5143958431194	-2.73268162772698\\
-2.52639295089314	-2.72261584436207\\
-2.53768764755075	-2.71184612372187\\
-2.54835920793859	-2.70029084918318\\
-2.55846787556392	-2.68784475507024\\
-2.5680556957367	-2.67437504951772\\
-2.57714610153282	-2.65971575798996\\
-2.58574207511867	-2.64365979494904\\
-2.5938225208964	-2.62594800823379\\
-2.60133622402374	-2.60625406320411\\
-2.60819237113583	-2.58416347219998\\
-2.61424598537452	-2.55914421883699\\
-2.61927561430565	-2.53050509658149\\
-2.62294892622753	-2.49733578203864\\
-2.62476901977294	-2.45841930989898\\
-2.62398934180504	-2.41210221761324\\
-2.61947654268191	-2.35609894231874\\
-2.6094855718454	-2.28719332139928\\
-2.59128520730172	-2.2007795031393\\
-2.56052874427497	-2.09015840588625\\
-2.51020112791778	-1.94549010324041\\
-2.42891927198263	-1.75236866575036\\
-2.29848437511039	-1.49037886305939\\
-2.09150070389166	-1.13336474617869\\
-1.77303719770791	-0.656399625920843\\
-1.3159015701882	-0.0572398538969062\\
-0.734747581320188	0.612789276987113\\
-0.107048664920237	1.25102495865522\\
0.468223340686164	1.76689291123208\\
0.934333044396128	2.13466154427952\\
1.28716764037607	2.37850745298307\\
1.54789253602088	2.53524599458945\\
1.74095319874678	2.6351980127523\\
1.88605718476763	2.69897948316981\\
1.99732059302863	2.73966108812865\\
2.08446122808108	2.76537413976779\\
2.15412473300267	2.7811966420266\\
2.21089301465229	2.79034350550865\\
2.25797171794367	2.79488540293541\\
2.29764238025503	2.79617948043294\\
};
\addplot [color=mycolor1, forget plot]
  table[row sep=crcr]{%
2.27711966681336	2.80471326690848\\
2.30876470732822	2.80373318111351\\
2.33622690217155	2.80112983301123\\
2.36034442597795	2.79728567370915\\
2.38175539889466	2.7924616367204\\
2.40095253020421	2.786834310215\\
2.41832114071146	2.78052021289298\\
2.43416601174845	2.77359177997818\\
2.44873059324809	2.76608788938589\\
2.46221089244057	2.7580206871475\\
2.47476558676405	2.74937981003638\\
2.48652339508863	2.74013468284281\\
2.49758839947065	2.73023528881658\\
2.50804377223615	2.71961161443807\\
2.5179541893136	2.70817181549812\\
2.52736707217291	2.69579901338413\\
2.53631267572461	2.68234648611736\\
2.54480290812179	2.66763084538734\\
2.5528286079959	2.65142256070542\\
2.56035478463002	2.63343286564032\\
2.56731300020612	2.61329559929467\\
2.57358956422711	2.59054180412344\\
2.57900738873048	2.5645637648759\\
2.5832979924475	2.53456337815224\\
2.58605783945938	2.49947686287451\\
2.58667922539985	2.45786315029811\\
2.58423895852489	2.40773566618065\\
2.57731573035855	2.34630482938462\\
2.56368513748234	2.26957901976898\\
2.53980314901611	2.17174333944803\\
2.49992699240001	2.04420410512122\\
2.43464331397276	1.87419253977717\\
2.32856450729637	1.64302484427267\\
2.1574207074857	1.32502897461142\\
1.88697506485576	0.890884177558295\\
1.48187061299944	0.32374191689811\\
0.935748434048338	-0.347378077649403\\
0.307228859364719	-1.02797244346139\\
-0.298652652289612	-1.60678448308606\\
-0.803317499659814	-2.03157712530496\\
-1.18822916270127	-2.3160609748746\\
-1.47140286407627	-2.49879359146632\\
-1.67906419043248	-2.61482261881218\\
-1.83340828547876	-2.68860811215353\\
-1.95047408647239	-2.73568281866713\\
-2.04125105712786	-2.76563715013135\\
-2.11318329204814	-2.78439779818418\\
-2.17134788716336	-2.79567665281743\\
-2.21925906595674	-2.80184076968151\\
-2.25939302529013	-2.80442825172745\\
-2.29352720274335	-2.80445617227769\\
-2.3229607998811	-2.80260711193358\\
-2.34865991490779	-2.79934421143854\\
-2.37135453980611	-2.7949834601364\\
-2.39160445426495	-2.78973991383537\\
-2.40984471980418	-2.78375769234307\\
-2.42641757489392	-2.77712966684287\\
-2.44159511028663	-2.76991044141953\\
-2.45559558365701	-2.76212485746951\\
-2.46859526417699	-2.75377341099572\\
-2.48073706973861	-2.74483544778696\\
-2.49213684364279	-2.73527066118357\\
-2.50288783414984	-2.72501918483355\\
-2.51306373878632	-2.71400040098098\\
-2.52272052208447	-2.70211044165896\\
-2.53189708615449	-2.68921822164761\\
-2.54061474783449	-2.67515968626879\\
-2.54887533342882	-2.65972975940298\\
-2.55665751642393	-2.64267120488666\\
-2.56391075765847	-2.62365921967494\\
-2.57054580187642	-2.60227998503542\\
-2.57642004026832	-2.57800049216393\\
-2.58131499451859	-2.55012553322605\\
-2.58490141245528	-2.51773547914119\\
-2.58668444774294	-2.4795948022608\\
-2.58591614512812	-2.43401533480081\\
-2.58145318549996	-2.3786485182647\\
-2.57152137197258	-2.31016523795941\\
-2.55331927040942	-2.22375785222391\\
-2.52234413115481	-2.11236728680922\\
-2.4712496105744	-1.96551581309859\\
-2.38797880348207	-1.76769630314829\\
-2.25305643044805	-1.49672758793427\\
-2.03702474111524	-1.1241431238314\\
-1.70284397425902	-0.62365208058744\\
-1.22450190265727	0.00335988167970699\\
-0.625509477875658	0.694114905155601\\
0.00485503797733931	1.3352611742484\\
0.566072085352282	1.83867314563659\\
1.01006365019401	2.18906550931705\\
1.34087070187219	2.41771869486442\\
1.58313849312425	2.56337217786309\\
1.76176167440832	2.65585265256686\\
1.89581443079859	2.7147768970127\\
1.99861613727433	2.75236420962431\\
2.07921101972012	2.77614516660634\\
2.14373753399703	2.79080032912494\\
2.19641075219014	2.79928685625849\\
2.24017376828498	2.80350843912919\\
2.27711966681336	2.80471326690848\\
};
\addplot [color=mycolor1, forget plot]
  table[row sep=crcr]{%
2.25487688810311	2.81199323983976\\
2.28436799553838	2.81107956910574\\
2.3100089430402	2.80864861750409\\
2.33256834724573	2.80505258966712\\
2.35263201677768	2.8005319114714\\
2.37065281728478	2.79524922692905\\
2.38698536527165	2.7893116067201\\
2.40191053147013	2.78278517030559\\
2.41565297676594	2.77570470126259\\
2.42839383599721	2.76807985783397\\
2.4402799545843	2.7598989765926\\
2.45143061802544	2.75113108183596\\
2.4619424020688	2.74172645626618\\
2.47189255390494	2.73161594466065\\
2.48134115440656	2.72070901609231\\
2.49033218214529	2.7088904763179\\
2.49889348116304	2.69601557851705\\
2.50703550697458	2.68190310486602\\
2.51474856656858	2.66632575464071\\
2.52199804710781	2.64899683529037\\
2.52871679629322	2.62955174742179\\
2.53479329508715	2.6075219805043\\
2.54005341352389	2.5822981249259\\
2.54423212086449	2.55307647608831\\
2.54692909744006	2.51878068145456\\
2.54753797263876	2.47794475558929\\
2.54513142675877	2.4285353127635\\
2.53827094999473	2.36767688352728\\
2.52468583058409	2.29122165295407\\
2.5007230709378	2.19307131555196\\
2.46039904329072	2.06411951158295\\
2.3937904251023	1.89068329845897\\
2.28448754298865	1.65252313899048\\
2.10638869534037	1.32164367346809\\
1.82279138568968	0.86641600175977\\
1.39754469963532	0.271049611753693\\
0.830426111489532	-0.426006181532845\\
0.193139376052034	-1.11630813728006\\
-0.402818609233224	-1.68582385182827\\
-0.885895466495825	-2.09254689146715\\
-1.24742522357496	-2.35979211481426\\
-1.51048266855685	-2.52955874150764\\
-1.7023428922059	-2.63676350644936\\
-1.84464870306425	-2.7047947972659\\
-1.95256916176499	-2.74819161705105\\
-2.03633170215196	-2.77583072299165\\
-2.10280408108491	-2.79316676413249\\
-2.15664861365666	-2.80360735984377\\
-2.20108504615934	-2.80932396858313\\
-2.23838006958827	-2.81172803992354\\
-2.27016067988777	-2.81175371512871\\
-2.2976166988921	-2.81002861637584\\
-2.32163361419796	-2.80697906088408\\
-2.34288119505579	-2.8028961489738\\
-2.36187363699789	-2.79797802636934\\
-2.37901107343624	-2.79235732394082\\
-2.39460868067106	-2.78611916859778\\
-2.40891737554244	-2.77931305185168\\
-2.42213871244104	-2.77196058679941\\
-2.43443570141001	-2.76406041841396\\
-2.44594069599043	-2.75559107179853\\
-2.45676111987736	-2.74651221050273\\
-2.46698354232244	-2.73676456192228\\
-2.47667642714893	-2.72626860544654\\
-2.48589173823608	-2.7149219818025\\
-2.49466546269359	-2.70259544602571\\
-2.50301699232055	-2.68912703001946\\
-2.51094716403585	-2.67431387862586\\
-2.51843457479595	-2.65790094139657\\
-2.52542951768997	-2.63956528977856\\
-2.53184447167411	-2.61889420571681\\
-2.53753941350928	-2.59535422173673\\
-2.54229912529256	-2.56824676694474\\
-2.54579782094551	-2.53664362217958\\
-2.54754322323746	-2.49929139094935\\
-2.54678661119326	-2.45446760501918\\
-2.5423753367342	-2.39976018367811\\
-2.53250625450951	-2.3317241259899\\
-2.51430614558058	-2.24534136350177\\
-2.48310931004538	-2.13317135168002\\
-2.43121704898488	-1.98404962205009\\
-2.3458439814805	-1.78126542388781\\
-2.20611831389706	-1.50068724615164\\
-1.98033238609531	-1.11131775015247\\
-1.62930687482662	-0.585608586595307\\
-1.12902246122067	0.0702408160438668\\
-0.513522037910772	0.780218103330241\\
0.116111253157241	1.42083254405841\\
0.660204998816192	1.90902779295882\\
1.08073270905983	2.24097041048473\\
1.38947351393834	2.45439728713025\\
1.61379618081722	2.58927045308193\\
1.77860543104869	2.67460103391161\\
1.90217247786813	2.72891623217254\\
1.99697791304627	2.7635793316525\\
2.07139663550914	2.78553726263968\\
2.13107658113404	2.79909109356255\\
2.17988317892706	2.80695414441097\\
2.22051134717918	2.81087290798355\\
2.25487688810311	2.81199323983976\\
};
\addplot [color=mycolor1, forget plot]
  table[row sep=crcr]{%
2.2306613868761	2.81803070297518\\
2.25810690410877	2.81718012527407\\
2.28201516003601	2.81491320421253\\
2.30308950693452	2.81155368741527\\
2.32186667064398	2.80732269213538\\
2.33876212010734	2.80236972560499\\
2.35410163785533	2.79679295696782\\
2.36814362598294	2.79065256203978\\
2.38109507726808	2.78397948565953\\
2.39312313224099	2.77678107656757\\
2.40436349690098	2.76904449916777\\
2.41492657297905	2.76073847420239\\
2.42490186835632	2.75181366374805\\
2.43436105653859	2.74220184468911\\
2.44335990651309	2.73181387613749\\
2.45193918373476	2.72053633609572\\
2.46012451013591	2.70822655989054\\
2.4679250472588	2.69470563459613\\
2.47533070951499	2.67974865995893\\
2.48230739246368	2.66307123355111\\
2.48878936392124	2.64431058778368\\
2.49466743013696	2.6229989883018\\
2.4997706103052	2.59852571315108\\
2.50383757236253	2.57008185871996\\
2.50647153235448	2.53657882915073\\
2.50706783155106	2.49652574468635\\
2.50469535887311	2.44784158954154\\
2.49789834998327	2.38756214474631\\
2.48435833264454	2.31137583767289\\
2.46030778970481	2.21288289339427\\
2.41950575059641	2.08242347523106\\
2.35147553534986	1.90531321212694\\
2.23868402405164	1.65958770416935\\
2.05300806937287	1.31467312543502\\
1.75510971736689	0.836517617528177\\
1.30849126874066	0.21118988845803\\
0.720798419557306	-0.511317529079947\\
0.0779140443342484	-1.20791854545575\\
-0.504310404303508	-1.76449354912078\\
-0.963651051034922	-2.15132519346601\\
-1.30133101338113	-2.40097689803586\\
-1.54462840311249	-2.55800318024844\\
-1.72127575581048	-2.65671069052081\\
-1.85211382956641	-2.71925994897991\\
-1.95136885095432	-2.75917169432432\\
-2.02850063951073	-2.78462216400938\\
-2.08981496010035	-2.80061234470425\\
-2.13957654655818	-2.81026070860493\\
-2.18072560715238	-2.81555396271566\\
-2.21533107455486	-2.81778429561671\\
-2.2448783990183	-2.81780785993293\\
-2.27045467227012	-2.81620060789198\\
-2.29286973444643	-2.81335422309845\\
-2.31273685207127	-2.80953638074976\\
-2.33052744438039	-2.80492929976484\\
-2.34660885364804	-2.79965478587334\\
-2.36127083449701	-2.7937906722762\\
-2.37474439994278	-2.78738164589066\\
-2.38721539229624	-2.78044630443655\\
-2.3988343416865	-2.77298159220663\\
-2.40972365384161	-2.76496532407901\\
-2.41998282343293	-2.75635722071773\\
-2.42969213301304	-2.74709867891067\\
-2.43891512794588	-2.73711134941806\\
-2.44770002625738	-2.72629446298113\\
-2.45608010789501	-2.71452071131373\\
-2.46407301201541	-2.70163033252864\\
-2.47167873369492	-2.68742284384416\\
-2.4788759266952	-2.67164557285018\\
-2.48561584719577	-2.65397770782051\\
-2.49181285049689	-2.63400793094603\\
-2.49732966899515	-2.61120267356519\\
-2.50195456246426	-2.58486040061168\\
-2.50536549382862	-2.554044685875\\
-2.50707310865982	-2.51748448173663\\
-2.5063283000101	-2.47342271799767\\
-2.50196930299109	-2.41938216717487\\
-2.49216346758393	-2.35179720417731\\
-2.47396280197052	-2.26542754675948\\
-2.4425289071623	-2.15242386901544\\
-2.38978363183492	-2.00087480010687\\
-2.30214558738244	-1.79274231006888\\
-2.15720197537847	-1.50172612091604\\
-1.92078092466796	-1.09405563207464\\
-1.55158685210626	-0.541139515258474\\
-1.02868761619051	0.144455732587054\\
-0.398536047886836	0.871547814567372\\
0.226482120535897	1.50768862618519\\
0.750336570209397	1.97786021388669\\
1.14618923806126	2.29038532705206\\
1.43290798360277	2.48861007795899\\
1.63980650934666	2.61301346167155\\
1.79139786468341	2.69150173946715\\
1.90500767470046	2.74143986091333\\
1.99224745257609	2.77333611988895\\
2.06083057765806	2.79357153066645\\
2.11593135374411	2.80608480485869\\
2.16108190935215	2.81335835477259\\
2.1987423302369	2.81699046255361\\
2.2306613868761	2.81803070297518\\
};
\addplot [color=mycolor1, forget plot]
  table[row sep=crcr]{%
2.20411869675044	2.82279803928925\\
2.22962048977649	2.82200742847636\\
2.25187975884548	2.81989662779306\\
2.27153851214511	2.81676257166763\\
2.28908734150646	2.8128081720023\\
2.30490655602639	2.80817054307244\\
2.31929481642068	2.80293945579783\\
2.33248937844525	2.79716948382655\\
2.34468059668352	2.790887965464\\
2.35602242581111	2.78410009843371\\
2.36664007142398	2.77679198415477\\
2.37663555958338	2.76893211680475\\
2.38609173629602	2.76047159494436\\
2.39507502710029	2.75134317398984\\
2.40363715134788	2.74145914604152\\
2.41181587340279	2.73070790680094\\
2.41963476568831	2.71894892691552\\
2.42710183821288	2.70600566388891\\
2.43420673359399	2.69165569968379\\
2.44091596335171	2.67561702216299\\
2.44716531868489	2.65752881254623\\
2.45284803969333	2.63692423640233\\
2.45779641799448	2.61319136133464\\
2.46175296364436	2.58551609671097\\
2.464324582366	2.55279737375112\\
2.46490843664902	2.51351861804726\\
2.46256950944333	2.46554910456749\\
2.45583393945943	2.40583098262153\\
2.44233259023548	2.32987794822825\\
2.41817504775361	2.23096462491157\\
2.37684071941484	2.09882560974564\\
2.30724399164613	1.91766663513266\\
2.19060259752446	1.66359274901303\\
1.99654643199345	1.30315549704448\\
1.68294141792844	0.799814089190649\\
1.21362999681624	0.142652742032227\\
0.606266086420765	-0.604237935836287\\
-0.0382997540981736	-1.30291832432706\\
-0.602702520520728	-1.84263705063436\\
-1.03625418886858	-2.2078342624618\\
-1.34972907059829	-2.43962109553829\\
-1.573657269817	-2.58415548714345\\
-1.73566550234319	-2.67468489667198\\
-1.8555751156271	-2.73200944254198\\
-1.94661241757109	-2.76861601619204\\
-2.01746950767089	-2.79199533022614\\
-2.07390549224461	-2.80671263664546\\
-2.11980415561342	-2.81561147538022\\
-2.15784017121593	-2.82050384345695\\
-2.18989557116555	-2.82256946670836\\
-2.21732242434057	-2.82259104341717\\
-2.24111117383757	-2.82109587128214\\
-2.26200054487199	-2.81844301039579\\
-2.28055069787866	-2.81487805671626\\
-2.29719284121756	-2.81056820777978\\
-2.31226348081426	-2.80562505579916\\
-2.32602845262016	-2.80011955472088\\
-2.33870003161801	-2.79409186837045\\
-2.35044925968698	-2.78755777079688\\
-2.3614149052239	-2.78051263692365\\
-2.37170999556761	-2.77293366269971\\
-2.38142655029902	-2.76478069164113\\
-2.39063892875089	-2.75599584069135\\
-2.3994060499628	-2.74650197585847\\
-2.40777262176581	-2.73619996140739\\
-2.41576940802391	-2.72496447442836\\
-2.42341245167745	-2.71263801805474\\
-2.43070103652721	-2.69902255510953\\
-2.43761398623936	-2.68386788206692\\
-2.44410362433956	-2.66685541318944\\
-2.45008628715942	-2.64757535338349\\
-2.45542757757212	-2.62549415098482\\
-2.45991936623893	-2.5999073759985\\
-2.46324351677259	-2.56987031334723\\
-2.46491374110498	-2.53409380753901\\
-2.4641805794229	-2.49078487179822\\
-2.45987276887681	-2.4373979208442\\
-2.45012653546256	-2.37023935955886\\
-2.431914115951	-2.28383038390616\\
-2.40021064168641	-2.16987733211045\\
-2.34652301112763	-2.01564625766097\\
-2.25638881346475	-1.8016197458856\\
-2.10567806579636	-1.4990673696351\\
-1.85751247124311	-1.07118544502344\\
-1.46860003919241	-0.488731048675673\\
-0.922577686600991	0.22731151805632\\
-0.280350139184698	0.968585112721327\\
0.335608902264612	1.59573520344585\\
0.83607009051639	2.0450385502171\\
1.20616863030855	2.33728236795813\\
1.47098269446232	2.5203802444838\\
1.66098315442301	2.63462778072726\\
1.7999268326179	2.70656813576981\\
1.90407499279771	2.7523467538357\\
1.98414951579792	2.78162251588829\\
2.0472127716962	2.80022858519715\\
2.09798230247861	2.81175764958274\\
2.13967244255459	2.81847325874164\\
2.1745208493223	2.821833771635\\
2.20411869675044	2.82279803928925\\
};
\addplot [color=mycolor1, forget plot]
  table[row sep=crcr]{%
2.17475110014827	2.82621406094749\\
2.19840629905796	2.82548043297719\\
2.21909692391879	2.82351815726881\\
2.23740723939928	2.82059887586622\\
2.25378441387765	2.81690831762872\\
2.26857565585033	2.81257189340399\\
2.2820540835625	2.80767144137197\\
2.2944370290331	2.802256246321\\
2.30589916343327	2.79635024908732\\
2.31658200720803	2.78995663347652\\
2.3266008614282	2.7830605250051\\
2.33604985169515	2.77563024340975\\
2.34500554288033	2.7676173513012\\
2.35352941858832	2.75895559282036\\
2.36166939494934	2.74955869074923\\
2.36946043370963	2.7393168467454\\
2.37692421752611	2.7280916469946\\
2.38406773332394	2.71570889107058\\
2.39088045524881	2.70194860287711\\
2.39732959413596	2.68653110035243\\
2.40335253200047	2.66909741690414\\
2.40884499632096	2.64918145274385\\
2.41364258794858	2.62616976861731\\
2.41749166427431	2.59924253789691\\
2.42000274952655	2.56728517874866\\
2.42057456170659	2.5287534147332\\
2.41826742025695	2.48146286987346\\
2.41158738197971	2.42225418535761\\
2.39810962268442	2.34645031439725\\
2.37380833079006	2.24696727030339\\
2.3318526433345	2.11286577582151\\
2.26047519304961	1.92710354457398\\
2.13948575923478	1.66360065413531\\
1.93599803799582	1.28569213576058\\
1.6049581204052	0.754385732317796\\
1.1116028431317	0.0634672083856049\\
0.486210425470154	-0.705837383852517\\
-0.155193166771875	-1.40136778907375\\
-0.697401134220052	-1.9200360919257\\
-1.10321661268172	-2.26194328464883\\
-1.39223713720369	-2.47567262411068\\
-1.59721657004419	-2.60798306054165\\
-1.74514557972819	-2.69064598612555\\
-1.85463854625831	-2.74299009034185\\
-1.93787834147517	-2.77646035973368\\
-2.00279362154754	-2.79787835832038\\
-2.05461279398557	-2.81139102620275\\
-2.096854717804	-2.8195803565799\\
-2.13194167177978	-2.82409297020052\\
-2.16157885351556	-2.82600241442489\\
-2.18699247474654	-2.82602211732219\\
-2.20908191952115	-2.82463350550535\\
-2.22851896072046	-2.82216487363621\\
-2.24581377331859	-2.8188409846643\\
-2.26135971457817	-2.81481485408586\\
-2.27546425534414	-2.81018842718826\\
-2.28837070089014	-2.80502616081054\\
-2.30027366745368	-2.79936395201235\\
-2.31133024281672	-2.79321492096289\\
-2.32166810233478	-2.78657298305792\\
-2.33139142673926	-2.77941478337059\\
-2.34058518567037	-2.77170032704136\\
-2.34931815648383	-2.76337246931098\\
-2.35764490647126	-2.75435529488997\\
-2.36560685442574	-2.74455129426235\\
-2.37323242620906	-2.73383711406644\\
-2.38053621180108	-2.72205749859815\\
-2.38751689890873	-2.70901682261767\\
-2.39415357385661	-2.69446730265018\\
-2.40039970249558	-2.67809250334192\\
-2.40617366261997	-2.6594840268271\\
-2.41134397360421	-2.63810811816795\\
-2.41570614112739	-2.61325705037652\\
-2.41894590520874	-2.5839770657284\\
-2.42057989803758	-2.54895945903643\\
-2.41985785096006	-2.5063725201115\\
-2.41559776917939	-2.45359674618959\\
-2.40590158809854	-2.38679936516417\\
-2.38765383558727	-2.30024009281017\\
-2.3556235539057	-2.18513341673795\\
-2.30085516358809	-2.02782512370027\\
-2.2078958747088	-1.80712729616441\\
-2.05068074657191	-1.49156354189462\\
-1.78935724797637	-1.0410346164601\\
-1.37892684098203	-0.426330064342468\\
-0.809611013312882	0.320429150179243\\
-0.158861852805466	1.07182267954521\\
0.442957258278228	1.68480962051844\\
0.916848606964636	2.11037621371517\\
1.26024549942691	2.38157864841056\\
1.50333812085555	2.54967090031518\\
1.67696961654767	2.65407864288348\\
1.80381260079955	2.71975355658681\\
1.89896626186733	2.76157779184113\\
1.97225037057874	2.78837006608586\\
2.03008883770538	2.80543387882379\\
2.0767591327988	2.81603146028465\\
2.11517249949107	2.82221875161342\\
2.14735572388684	2.82532186585778\\
2.17475110014827	2.82621406094749\\
};
\addplot [color=mycolor1, forget plot]
  table[row sep=crcr]{%
2.14185167975353	2.82811886201345\\
2.16375472099596	2.82743931153048\\
2.1829554808675	2.82561810950134\\
2.19998381115138	2.82290302535754\\
2.21524596830833	2.81946356060583\\
2.22905797280445	2.81541406673926\\
2.24166887153974	2.81082888159783\\
2.25327721545887	2.80575228780801\\
2.26404288803612	2.80020501357924\\
2.27409568351791	2.79418834106226\\
2.28354156224498	2.78768647908714\\
2.29246720103243	2.78066759198863\\
2.30094324708977	2.77308369344406\\
2.30902653528962	2.76486947593528\\
2.31676141495498	2.75594002685406\\
2.32418023496131	2.74618726101363\\
2.33130293876362	2.73547475649129\\
2.33813560686435	2.72363049229966\\
2.3446676308791	2.71043671862902\\
2.35086697694649	2.69561579141908\\
2.35667264118912	2.67881018875666\\
2.36198282021328	2.65955395646265\\
2.36663634360813	2.63723126333636\\
2.37038322814755	2.61101516183607\\
2.37283722330557	2.57977529934503\\
2.37339778982183	2.54193586381645\\
2.37111887335268	2.49525205412532\\
2.36448275111581	2.43645056417114\\
2.3510006908997	2.36063993576219\\
2.32649277166447	2.26033118819243\\
2.2837747325805	2.12381882607868\\
2.21029948587487	1.93263172647691\\
2.084263967828	1.65818455267862\\
1.86994484410035	1.26020846482804\\
1.5193384815145	0.697504952025946\\
1.0006983097713	-0.0289546968303891\\
0.360049021024791	-0.817323633554986\\
-0.272206581709307	-1.50322890734363\\
-0.787566691977589	-1.99637801408196\\
-1.16381823972836	-2.31343899538887\\
-1.42823676075027	-2.50899376903536\\
-1.61471437745805	-2.62936567702651\\
-1.74911207095017	-2.70446720841903\\
-1.84867712973555	-2.75206417414387\\
-1.92451708705135	-2.78255795851524\\
-1.98380493175901	-2.80211837265008\\
-2.03125468735656	-2.81449094713054\\
-2.07003569180899	-2.82200876941633\\
-2.10233018110642	-2.82616180320071\\
-2.12967588442384	-2.82792326372033\\
-2.1531801245671	-2.82794119873254\\
-2.17365639415808	-2.82665375450862\\
-2.1917133623488	-2.82436019316507\\
-2.20781410744761	-2.82126560905936\\
-2.22231632455808	-2.81750962028811\\
-2.23550012474565	-2.81318505342203\\
-2.24758757759959	-2.808350221096\\
-2.25875665118115	-2.80303698494289\\
-2.26915127475398	-2.7972559568432\\
-2.27888866191015	-2.79099967643615\\
-2.2880646510665	-2.7842442757679\\
-2.29675756692326	-2.77694992385298\\
-2.30503093126348	-2.76906018715897\\
-2.31293522298438	-2.76050031582731\\
-2.32050878383483	-2.75117434743358\\
-2.32777787103906	-2.74096079067564\\
-2.33475575469844	-2.72970648920066\\
-2.34144062746594	-2.7172180427969\\
-2.34781190937511	-2.70324983788496\\
-2.3538242488797	-2.68748724591913\\
-2.35939806941366	-2.66952277861396\\
-2.3644047611806	-2.64882175930822\\
-2.36864333865978	-2.62467206252469\\
-2.37180314504003	-2.59610912865902\\
-2.37340316803587	-2.56180177567943\\
-2.37269115478618	-2.51987449893684\\
-2.36847187271704	-2.46762474132698\\
-2.35880747173788	-2.40106346667514\\
-2.34048239923269	-2.31415587911096\\
-2.30803121233605	-2.19755920777985\\
-2.25197055349214	-2.03656905299043\\
-2.15571304768967	-1.80808100704446\\
-1.99098591135525	-1.47748835353313\\
-1.71468104540268	-1.00116744224884\\
-1.28068252930592	-0.351115377356678\\
-0.68853915531236	0.425809568332435\\
-0.0341533716274803	1.18172463156856\\
0.547734367825091	1.77464307462038\\
0.991880077971176	2.17360147631448\\
1.30776112775099	2.42310769102853\\
1.52937635180833	2.57635806517277\\
1.6871706168914	2.67124391386155\\
1.80244009974611	2.73092572845202\\
1.88904281906363	2.76899036570571\\
1.95589065974379	2.79342860115457\\
2.00878314321231	2.8090324361412\\
2.05157406424349	2.81874849587472\\
2.08688552050022	2.824435670925\\
2.11654422500564	2.8272949813574\\
2.14185167975353	2.82811886201345\\
};
\addplot [color=mycolor1, forget plot]
  table[row sep=crcr]{%
2.10439665617207	2.8282298624729\\
2.12464188065137	2.82760148176032\\
2.14243234288805	2.82591382277774\\
2.15824659004586	2.82339212332122\\
2.17245231744745	2.82019056110771\\
2.18533615434218	2.8164130354986\\
2.19712446440675	2.81212678727802\\
2.2079981010471	2.80737135710589\\
2.21810301410286	2.80216441784895\\
2.22755795084046	2.79650543077886\\
2.23646007529932	2.79037770929812\\
2.24488905472067	2.78374923438941\\
2.25290997458345	2.77657239889961\\
2.26057530994044	2.76878272888608\\
2.26792607712629	2.76029651585636\\
2.27499219932393	2.75100717441069\\
2.2817920267666	2.74077999580701\\
2.28833084080954	2.7294447746609\\
2.29459801820751	2.7167855079095\\
2.30056230294151	2.70252594691874\\
2.306164269904	2.6863091344892\\
2.31130446643525	2.66766802527548\\
2.31582470136927	2.64598260556978\\
2.31947817809422	2.62041612789413\\
2.32188099444431	2.58981831435422\\
2.32243170733342	2.55257512298983\\
2.32017470859998	2.50637009777338\\
2.31356213260709	2.44779635654283\\
2.30002812104162	2.37171232033471\\
2.2752102358232	2.27015675826242\\
2.23150840732329	2.13052870772527\\
2.15545993513703	1.93268391936682\\
2.02337848732076	1.6451206627233\\
1.79632734061758	1.22354909389806\\
1.42353359793657	0.625220579339345\\
0.878771555207121	-0.138018972691501\\
0.227351529764195	-0.940000491971334\\
-0.388387094450216	-1.60829303205258\\
-0.871991154686601	-2.07120251933972\\
-1.21698954120204	-2.36197680636123\\
-1.45675809677698	-2.53931410504228\\
-1.62520636758421	-2.64804972105209\\
-1.74661180451078	-2.71589014876617\\
-1.83671977780629	-2.75896476121326\\
-1.90554053800263	-2.78663504150652\\
-1.95950195902525	-2.80443718838813\\
-2.00282013971824	-2.8157317148965\\
-2.03832968303853	-2.82261478727292\\
-2.06798441766638	-2.8264279081047\\
-2.09316335521594	-2.82804944289212\\
-2.1148614193132	-2.8280657096879\\
-2.13381102198757	-2.8268740115869\\
-2.15056131628629	-2.82474621291356\\
-2.16553098346993	-2.82186884012359\\
-2.17904410894577	-2.81836885520921\\
-2.19135502121799	-2.81433046201439\\
-2.20266577905565	-2.80980615459494\\
-2.21313866319811	-2.80482396386185\\
-2.22290520522737	-2.79939210980444\\
-2.23207276452311	-2.79350180566567\\
-2.24072932584376	-2.78712866617418\\
-2.24894696417187	-2.7802329739112\\
-2.25678426640306	-2.77275891335017\\
-2.26428788314733	-2.76463276304384\\
-2.27149328874582	-2.75575992192013\\
-2.27842473786471	-2.7460205166516\\
-2.28509430723748	-2.73526317200658\\
-2.2914997822963	-2.7232962958047\\
-2.29762096294645	-2.70987589060062\\
-2.3034136760499	-2.69468838494862\\
-2.30880031805775	-2.67732616054952\\
-2.3136549741375	-2.6572521367095\\
-2.31777982205	-2.633747608672\\
-2.32086716431094	-2.60583389424976\\
-2.32243714590556	-2.57215208683478\\
-2.32173325091376	-2.53077426119804\\
-2.31754253154096	-2.47890003515275\\
-2.30787822319011	-2.41235776020951\\
-2.28940538814651	-2.32476930466821\\
-2.25638205356715	-2.20614149639993\\
-2.19870458744134	-2.04054099553217\\
-2.09845541803663	-1.80262170813905\\
-1.92480803984779	-1.45417862427278\\
-1.63113994870503	-0.947953115606942\\
-1.1713375961867	-0.259161487476996\\
-0.557980698113119	0.545889993495298\\
0.0933622157606264	1.29865044866206\\
0.648758293724474	1.86480006513732\\
1.06001811194664	2.23430726860776\\
1.34770707165212	2.46157149408546\\
1.5481469960902	2.60018432009282\\
1.69063965301853	2.68586874206893\\
1.79484746233692	2.73982193018783\\
1.87332468029321	2.77431381426718\\
1.93407508714821	2.79652186191819\\
1.98228898802755	2.81074462611771\\
2.02141253032432	2.81962733025033\\
2.05379189979266	2.82484177055891\\
2.08106385814126	2.82747058619653\\
2.10439665617207	2.8282298624729\\
};
\addplot [color=mycolor1, forget plot]
  table[row sep=crcr]{%
2.06086420360006	2.82606288482026\\
2.07954946709803	2.82548265069804\\
2.09601333227688	2.82392060654805\\
2.11068597508046	2.82158074459897\\
2.12389875922843	2.81860278558362\\
2.13591063685797	2.81508075559912\\
2.14692662145578	2.81107517523556\\
2.15711091717045	2.8066210743889\\
2.16659637331059	2.80173319287336\\
2.17549135875597	2.79640920788889\\
2.18388478249776	2.79063150275514\\
2.19184974351591	2.78436777594098\\
2.19944612711559	2.77757063694663\\
2.20672234502208	2.77017621530973\\
2.21371632230606	2.76210169905723\\
2.22045574990295	2.75324160072957\\
2.22695753276466	2.74346240291337\\
2.23322625410922	2.73259503550656\\
2.23925132297113	2.72042434612167\\
2.24500223933518	2.70667428353889\\
2.25042103772944	2.69098682330781\\
2.25541034879725	2.67289155617102\\
2.25981445309361	2.65176103973526\\
2.26338882491472	2.62674395617998\\
2.26575027047244	2.59666286660416\\
2.26629346484606	2.55985413589397\\
2.26404769597393	2.51391112540796\\
2.2574242625088	2.45526195276396\\
2.24375881457129	2.37845948752776\\
2.21846221176634	2.274969050451\\
2.17342512124737	2.13110596293197\\
2.09407399513783	1.92471095921482\\
1.95447412087448	1.62083113027563\\
1.71205614108905	1.17076849580878\\
1.31391113289961	0.531696757122038\\
0.743197812322143	-0.268179794547498\\
0.0880735063622423	-1.07514858084357\\
-0.502153092938349	-1.71605812303739\\
-0.948894897869495	-2.14381008954486\\
-1.26111566886754	-2.40699538801196\\
-1.47628697745874	-2.56614725303258\\
-1.62720510986181	-2.66356637669721\\
-1.73615321181851	-2.72444379483013\\
-1.81726335384155	-2.76321533960986\\
-1.87943510189821	-2.78821089182298\\
-1.92836393046319	-2.8043517092476\\
-1.96778443891464	-2.81462919399292\\
-2.00021066507747	-2.82091400872068\\
-2.02737901295415	-2.82440695845339\\
-2.05051798869225	-2.8258967520207\\
-2.07051619978823	-2.82591144571268\\
-2.08802949379759	-2.82480982424699\\
-2.1035508863799	-2.82283791974444\\
-2.11745721549868	-2.82016474710658\\
-2.13004090895075	-2.81690532193666\\
-2.14153202214416	-2.81313569461374\\
-2.15211378353517	-2.80890283901094\\
-2.16193371986965	-2.80423112841719\\
-2.17111171051065	-2.79912646820911\\
-2.17974586158463	-2.79357874503894\\
-2.18791679254925	-2.78756298892149\\
-2.19569072795659	-2.78103946535897\\
-2.20312164729712	-2.77395278139286\\
-2.21025264096103	-2.76622997677119\\
-2.21711653288359	-2.75777745964334\\
-2.22373574576197	-2.7484765167114\\
-2.23012128791669	-2.73817695858281\\
-2.23627061292907	-2.72668822158121\\
-2.24216391578484	-2.71376689048698\\
-2.24775813592976	-2.6990990562707\\
-2.25297745790512	-2.68227505042866\\
-2.2576982894797	-2.66275268114385\\
-2.26172528803559	-2.63980274396403\\
-2.26475249081184	-2.61242658332343\\
-2.26629899592143	-2.57922854142314\\
-2.26559997429051	-2.53821383269632\\
-2.26141709345376	-2.48646024070003\\
-2.25169962860753	-2.41957196711612\\
-2.23296265066169	-2.33075292350808\\
-2.19912348829124	-2.20922159971954\\
-2.13932243627577	-2.03756007410029\\
-2.03403793857676	-1.78773830321613\\
-1.8494498252271	-1.41739349238135\\
-1.53527988504902	-0.875835879740883\\
-1.04748582114543	-0.144956876075536\\
-0.416552666778595	0.683547362854895\\
0.222686672033858	1.42271253383222\\
0.744243343940997	1.95457610884938\\
1.11956379822977	2.29186389120879\\
1.37853086268745	2.49645631370144\\
1.55815545418005	2.62067636550857\\
1.68588914043314	2.69748424140915\\
1.77953750501375	2.74596836911959\\
1.85030312395161	2.77706928414767\\
1.90528603804559	2.7971677366425\\
1.94908325868717	2.81008670307786\\
1.9847488763622	2.81818362897938\\
2.0143657302615	2.82295266501718\\
2.03939014961669	2.82536442418953\\
2.06086420360006	2.82606288482026\\
};
\addplot [color=mycolor1, forget plot]
  table[row sep=crcr]{%
2.00891888614438	2.82078560705859\\
2.02615105351259	2.82025020819442\\
2.04138134146949	2.81880495921137\\
2.05499436364695	2.81663386540661\\
2.06728730449655	2.81386303773614\\
2.07849310905539	2.81057718774165\\
2.0887967526755	2.8068304663362\\
2.09834683370745	2.80265358719083\\
2.1072639405311	2.79805842886958\\
2.11564674666219	2.79304085349802\\
2.12357646721724	2.78758219053757\\
2.13112009777789	2.78164964140228\\
2.13833271062673	2.77519572165534\\
2.14525897670313	2.76815674485985\\
2.15193399614463	2.7604502456637\\
2.15838344164903	2.75197112141917\\
2.1646229334839	2.74258612157438\\
2.1706564564729	2.73212610543142\\
2.17647347428268	2.72037518130714\\
2.18204415717123	2.70705536878776\\
2.18731175146273	2.69180468204638\\
2.19218046700744	2.67414532896778\\
2.19649613010137	2.65343672764823\\
2.2000148408192	2.62880466190764\\
2.20235120160236	2.59903203066476\\
2.2028907851417	2.56238623040231\\
2.20063819958206	2.51633934780944\\
2.19394581432378	2.45710273878048\\
2.18001545400075	2.37883434803851\\
2.15396010897569	2.27226688933048\\
2.10701532869381	2.12234499819615\\
2.02320496199439	1.90439750994316\\
1.87384438654784	1.57932410863569\\
1.61233813108486	1.0938464330092\\
1.18524109310871	0.408176807620506\\
0.590973948330326	-0.425113895273302\\
-0.0569771957156509	-1.22374196019026\\
-0.610890663283743	-1.82551182531642\\
-1.01557587894421	-2.21309612205472\\
-1.29369663941072	-2.44756019768786\\
-1.48442922186893	-2.58863746352841\\
-1.61834726432864	-2.67508007316861\\
-1.71537580525566	-2.72929425484456\\
-1.78794419320833	-2.76398050957151\\
-1.84383522588946	-2.78644933724247\\
-1.8880263018093	-2.80102607714932\\
-1.92378673181909	-2.81034847094563\\
-1.95332352668201	-2.81607261785346\\
-1.97816604215308	-2.81926604942691\\
-1.99940001727583	-2.8206328007354\\
-2.01781348573809	-2.82064601381138\\
-2.03399001271016	-2.81962821404895\\
-2.04836970007472	-2.81780113109942\\
-2.06128999748746	-2.81531730242975\\
-2.07301356805585	-2.81228048793155\\
-2.08374767205679	-2.8087590311147\\
-2.09365787595991	-2.80479465302263\\
-2.10287788718483	-2.80040819927139\\
-2.11151668915681	-2.79560327921136\\
-2.11966375314433	-2.79036837479183\\
-2.1273928436062	-2.78467776236583\\
-2.1347647587155	-2.77849142890533\\
-2.14182922407171	-2.77175404110533\\
-2.1486260635108	-2.76439291855656\\
-2.15518569053768	-2.75631485205166\\
-2.16152888367223	-2.7474014767175\\
-2.16766571440516	-2.73750273431448\\
-2.17359336831123	-2.72642770719064\\
-2.17929240878103	-2.71393172718295\\
-2.18472072980177	-2.69969807263371\\
-2.18980394329585	-2.68331162355013\\
-2.19442009162995	-2.66422030083433\\
-2.19837507527803	-2.64167752756128\\
-2.20136247904574	-2.61465451014093\\
-2.20289646410985	-2.58170334133977\\
-2.20219684149331	-2.54073794130088\\
-2.19798678378918	-2.48867432855378\\
-2.18812642630346	-2.42082483760648\\
-2.16893087296831	-2.32985653195647\\
-2.1338749289035	-2.20398573926515\\
-2.07113493439352	-2.02392816627601\\
-1.95918946251008	-1.75835244731557\\
-1.76067650217052	-1.36011692833705\\
-1.4218821918633	-0.776081944412222\\
-0.904606984461733	-0.000769264299073928\\
-0.26323362215036	0.841939097541932\\
0.351784892089782	1.55350881341773\\
0.831431807622443	2.04281570297915\\
1.1679223194889	2.34525963145284\\
1.39779849369872	2.52687794682429\\
1.55702877496454	2.63699263458927\\
1.67055879911057	2.70525662425323\\
1.7541482407559	2.7485304122958\\
1.8176132090598	2.77642083912154\\
1.86715809602388	2.79453009809739\\
1.90680295862112	2.80622323505773\\
1.93922490440843	2.81358303291265\\
1.96625534626517	2.81793503296059\\
1.98917915602654	2.82014390649056\\
2.00891888614438	2.82078560705859\\
};
\addplot [color=mycolor1, forget plot]
  table[row sep=crcr]{%
1.94484137383948	2.81093474107766\\
1.96074547841375	2.81044028934514\\
1.97485329975312	2.80910128846411\\
1.9875066311284	2.80708302175642\\
1.99897060680077	2.80449883772538\\
2.00945384083686	2.80142467844377\\
2.01912260826149	2.79790864308162\\
2.02811098393352	2.79397726781706\\
2.03652818247299	2.78963955849741\\
2.04446391838361	2.78488941551565\\
2.0519923310893	2.77970683640786\\
2.05917483669373	2.7740581098061\\
2.06606214117594	2.76789508755222\\
2.07269555549188	2.76115351547\\
2.07910767558369	2.7537502987825\\
2.08532241656818	2.74557945784403\\
2.0913543072874	2.73650637307818\\
2.09720684250933	2.72635969607218\\
2.10286953093216	2.71491997250002\\
2.10831302764683	2.7019035098721\\
2.11348133002652	2.68693920708129\\
2.11827931989926	2.66953473105818\\
2.12255271702205	2.64902619986201\\
2.12605531829587	2.6245017198388\\
2.12839434568635	2.59468243768769\\
2.12893701888778	2.55773275922355\\
2.12664639890571	2.51094936877903\\
2.11978435331607	2.45023777811195\\
2.10535805591883	2.36920952937128\\
2.07806245337108	2.25760066386937\\
2.0282387279616	2.09852454951016\\
1.93805368929678	1.8640500671643\\
1.77538712578065	1.51006526183966\\
1.48949970851246	0.979304317245783\\
1.03007431292167	0.241476455505507\\
0.419288702040341	-0.615550762137571\\
-0.205214267148519	-1.3858245453107\\
-0.710233576241256	-1.934731191373\\
-1.06777672003211	-2.27723504612438\\
-1.31073289439563	-2.48206333033099\\
-1.47730305556258	-2.6052645536573\\
-1.59479054014759	-2.68109634887966\\
-1.68045100632792	-2.72895477760171\\
-1.74494270113762	-2.75977779842846\\
-1.79493183413157	-2.77987210230925\\
-1.83469310631565	-2.79298630915439\\
-1.86704631039807	-2.80141951316405\\
-1.8939042740545	-2.80662379841013\\
-1.91659908828351	-2.80954060534081\\
-1.93608103539775	-2.81079415696333\\
-1.95304303474956	-2.81080598165623\\
-1.96800047775809	-2.80986459578593\\
-1.98134368007364	-2.80816896161207\\
-1.9933731150669	-2.80585617766109\\
-2.00432356697461	-2.80301943279653\\
-2.01438099583165	-2.79971978771933\\
-2.02369450740841	-2.79599393431396\\
-2.03238496811591	-2.79185925149242\\
-2.04055127247485	-2.78731697238645\\
-2.04827493068313	-2.78235396204598\\
-2.05562342072536	-2.77694339766353\\
-2.06265259796019	-2.77104449764707\\
-2.06940834678536	-2.76460133187308\\
-2.07592757486766	-2.75754064221577\\
-2.08223857638946	-2.74976849232457\\
-2.08836071411339	-2.74116543002282\\
-2.0943032764969	-2.73157966067934\\
-2.10006323587494	-2.7208174601858\\
-2.10562143584944	-2.70862964564853\\
-2.11093641784446	-2.69469227718988\\
-2.11593456480623	-2.67857872496947\\
-2.12049432251631	-2.65971851899145\\
-2.12442063129081	-2.63733549520008\\
-2.12740273215173	-2.61035271796791\\
-2.12894294327842	-2.57724272324503\\
-2.12823326147808	-2.53578540274212\\
-2.12393536894113	-2.4826658651701\\
-2.11377659503182	-2.41278885724156\\
-2.09378679643449	-2.31808502786372\\
-2.05682850107485	-2.18541886954297\\
-1.98978033523441	-1.99304251906782\\
-1.86853263904077	-1.70545123782526\\
-1.65156467308278	-1.27022149623518\\
-1.28293800086662	-0.634646104230499\\
-0.737052344015008	0.18398021550805\\
-0.098257033791466	1.02392006128289\\
0.476762529669879	1.68961487921372\\
0.905933475822523	2.12757086932723\\
1.20098228351461	2.39279553822506\\
1.40158390766722	2.55128457826433\\
1.54091048947349	2.64762983803274\\
1.64081458187349	2.70769620331022\\
1.71485558795971	2.74602353869501\\
1.77144028197587	2.77088807459421\\
1.81588842973417	2.7871328303387\\
1.85165952558558	2.79768228501973\\
1.88106812247197	2.80435722786817\\
1.90570546513969	2.80832331021653\\
1.92669343243673	2.81034517052612\\
1.94484137383948	2.81093474107766\\
};
\addplot [color=mycolor1, forget plot]
  table[row sep=crcr]{%
1.86245943044403	2.79384728481862\\
1.87719543502277	2.79338878598613\\
1.89032623322453	2.79214220986903\\
1.90215354364192	2.79025543168445\\
1.91291262912871	2.78782990592033\\
1.92278952809222	2.78493333572121\\
1.93193325090473	2.78160803032954\\
1.94046454086296	2.77787638382461\\
1.94848224298122	2.77374436072882\\
1.9560679706476	2.7692035335771\\
1.96328952997607	2.76423199665512\\
1.97020340681034	2.75879432743904\\
1.97685651220648	2.75284065093106\\
1.9832872994753	2.74630476024209\\
1.98952629549268	2.73910114180536\\
1.9955960190075	2.73112062779437\\
2.00151017625338	2.72222422964287\\
2.00727191226183	2.71223446324973\\
2.01287072863135	2.7009231085034\\
2.01827741105682	2.68799377045739\\
2.02343586479672	2.67305668527072\\
2.0282499908962	2.65559169200297\\
2.03256238150448	2.63489272300859\\
2.03611914740736	2.60998272367851\\
2.03851057489655	2.57948003739637\\
2.03906841393883	2.54138299418465\\
2.03668297320485	2.49271294721524\\
2.02946742013352	2.42890633522266\\
2.01412311941155	2.34275423411632\\
1.98471102608089	2.22252873580691\\
1.93026212035532	2.04873132203026\\
1.83035376990038	1.78902783774467\\
1.64857280295573	1.39347046401649\\
1.33103518000575	0.803821593145323\\
0.838526356380404	0.0123651422389971\\
0.227394302186296	-0.845931946906854\\
-0.350442449398308	-1.55919759768021\\
-0.792727001687744	-2.04011003837801\\
-1.09849427935741	-2.33304905062391\\
-1.30556878808597	-2.50762108621555\\
-1.44839577807619	-2.61325125302515\\
-1.5500787806155	-2.67887462553143\\
-1.62495459176856	-2.72070262573729\\
-1.68186320611161	-2.74789805962356\\
-1.72635986223458	-2.76578228194085\\
-1.76203238599039	-2.77754636466723\\
-1.79126603796787	-2.78516531107247\\
-1.81569144812738	-2.7898974168338\\
-1.83645257630568	-2.79256507392194\\
-1.85437113561667	-2.79371753788091\\
-1.87005014771785	-2.7937280692581\\
-1.88394082426028	-2.79285349083442\\
-1.89638679433492	-2.79127159116071\\
-1.9076539962759	-2.78910510451997\\
-1.9179512873667	-2.78643733752859\\
-1.92744491188462	-2.78332245749679\\
-1.93626882061528	-2.77979227183243\\
-1.9445321310142	-2.77586062439388\\
-1.95232457527549	-2.77152610482186\\
-1.9597204993907	-2.76677349436795\\
-1.96678178833739	-2.76157418987653\\
-1.97355996345272	-2.75588571626226\\
-1.98009760410597	-2.74965033109007\\
-1.98642917078006	-2.74279262357416\\
-1.99258123797982	-2.73521589730112\\
-1.99857207095989	-2.72679698194947\\
-2.00441038565776	-2.7173789181426\\
-2.01009299557332	-2.70676066179558\\
-2.01560083891203	-2.69468249590454\\
-2.02089253581957	-2.68080511086198\\
-2.02589404403578	-2.66467913195244\\
-2.03048196669752	-2.64569990157713\\
-2.03445624511432	-2.62303895654064\\
-2.03749460805865	-2.59553774446397\\
-2.03907476422507	-2.56153854272313\\
-2.0383378448221	-2.51860812301941\\
-2.03384155505112	-2.46307344849564\\
-2.02310021273392	-2.38922080245886\\
-2.00170367524217	-2.28788674924078\\
-1.96160130236399	-2.14397601173785\\
-1.88781061634304	-1.93230359683834\\
-1.75275925925105	-1.61201577623333\\
-1.51037323548226	-1.12576982920136\\
-1.10632840669905	-0.428838530197344\\
-0.539072435537281	0.422527341385257\\
0.0748267257916535	1.23045553389087\\
0.590332220910206	1.82760354405671\\
0.960489345185461	2.20542733579409\\
1.21194722364818	2.43147322229943\\
1.38332012285093	2.56686017357665\\
1.50332175285635	2.64983329006234\\
1.59021747160967	2.70207230964653\\
1.65524948912868	2.73573203272077\\
1.70540363675527	2.75776811117796\\
1.74512825602826	2.77228466203469\\
1.77733825522655	2.78178258186608\\
1.80399917194136	2.7878329240668\\
1.82647257532337	2.79144994386784\\
1.84572528341125	2.79330408577433\\
1.86245943044403	2.79384728481862\\
};
\addplot [color=mycolor1, forget plot]
  table[row sep=crcr]{%
1.75108552716988	2.76447222447479\\
1.76488221422702	2.76404250582548\\
1.77724897716815	2.7628680858765\\
1.78845029864056	2.76108084132467\\
1.79869390550384	2.75877123151757\\
1.80814521289382	2.7559992047306\\
1.81693762491815	2.75280140880389\\
1.8251799843703	2.7491959003046\\
1.83296202077276	2.74518509199559\\
1.84035836173541	2.74075739148653\\
1.84743148519643	2.73588779376549\\
1.85423386213361	2.73053755456242\\
1.86080944715795	2.72465296307848\\
1.86719460207298	2.7181631319376\\
1.8734184728415	2.71097661217749\\
1.87950277219999	2.70297650298354\\
1.88546083530285	2.69401353487635\\
1.89129569572855	2.68389632399633\\
1.89699674425462	2.67237756415484\\
1.90253423198378	2.65913424202541\\
1.90785037105443	2.64373885549779\\
1.912844900282	2.62561677736276\\
1.91735139494529	2.60398177517081\\
1.92109767190124	2.57773622969796\\
1.92363808680396	2.54531280800104\\
1.92423467860873	2.50441642898222\\
1.92164237946125	2.45159196649833\\
1.91370900678464	2.38148059663562\\
1.89660921227259	2.28551375431683\\
1.863350321132	2.14961015011902\\
1.80088724098503	1.95028198099332\\
1.68504321311595	1.64918888648009\\
1.47455060025472	1.19108672751144\\
1.1172436115952	0.527194676546508\\
0.600838439980181	-0.303565110777849\\
0.0217110314597016	-1.11787439832514\\
-0.479295792600991	-1.73676848784417\\
-0.845266816030316	-2.13480615233511\\
-1.09566377390525	-2.37469239957717\\
-1.26664301717949	-2.51881458141324\\
-1.38634200994895	-2.60732445286478\\
-1.47293770426302	-2.66320074056089\\
-1.53767767061608	-2.69936009220179\\
-1.58756039597431	-2.72319384635182\\
-1.62704090309671	-2.73905920237903\\
-1.65903569364839	-2.74960854146887\\
-1.68550910797049	-2.7565067516797\\
-1.70782031556884	-2.76082825634916\\
-1.72693340842951	-2.76328338763886\\
-1.74354789771438	-2.76435137514603\\
-1.75818186757464	-2.76436071503665\\
-1.77122627072982	-2.76353901027093\\
-1.78298119838695	-2.76204459183504\\
-1.79368062347662	-2.75998696923435\\
-1.80350960739666	-2.75744024887193\\
-1.81261647449423	-2.75445200543617\\
-1.82112155809316	-2.75104912596252\\
-1.82912356382167	-2.74724156652161\\
-1.83670424214872	-2.74302460222861\\
-1.84393183205323	-2.73837991938817\\
-1.85086358368856	-2.73327573973745\\
-1.85754756019065	-2.72766604744722\\
-1.86402383820955	-2.72148888718672\\
-1.87032515965923	-2.71466359834352\\
-1.87647702221151	-2.70708672901466\\
-1.8824971214059	-2.69862621252646\\
-1.88839395763576	-2.68911315896781\\
-1.89416427351294	-2.67833026766958\\
-1.89978875254881	-2.66599532654949\\
-1.90522502065326	-2.651737399139\\
-1.9103963233033	-2.63506187913752\\
-1.91517306923613	-2.61529820075033\\
-1.91934228271213	-2.59151986780491\\
-1.92255598803098	-2.56241917110889\\
-1.92424181720092	-2.52610575635664\\
-1.92344382818246	-2.47977378495112\\
-1.91853047983872	-2.4191366990536\\
-1.90664281269112	-2.33744360035013\\
-1.88262584056747	-2.2237426721095\\
-1.8369431681246	-2.05985534486197\\
-1.75176729647091	-1.81557078204867\\
-1.5949202394313	-1.44359327983578\\
-1.31697355185953	-0.885815607926106\\
-0.875685625693936	-0.123991058980787\\
-0.309394011606504	0.726956426715789\\
0.24468052449709	1.4568868622189\\
0.678903972736431	1.96012466793852\\
0.982576432505579	2.27011531894379\\
1.18906142341872	2.45571542385598\\
1.33156579852306	2.56827739628308\\
1.43294892172296	2.63836437217429\\
1.50752689674116	2.68319025734382\\
1.5641525322609	2.7124937892268\\
1.60839096210157	2.73192732354029\\
1.64383384795358	2.74487692142254\\
1.67286639677993	2.75343629233736\\
1.6971172922832	2.75893855142175\\
1.71772785572109	2.76225488140204\\
1.73551709087349	2.76396740628574\\
1.75108552716988	2.76447222447479\\
};
\addplot [color=mycolor1, forget plot]
  table[row sep=crcr]{%
1.59153714869271	2.71283556446399\\
1.60476885427873	2.71242284118941\\
1.61672829479412	2.71128658950149\\
1.62764574399131	2.70954418885353\\
1.63770404439344	2.70727595560098\\
1.64705034381964	2.70453435482357\\
1.65580456300521	2.7013500981366\\
1.66406558417514	2.69773608650507\\
1.67191581871882	2.69368979286982\\
1.67942459554526	2.68919444322856\\
1.68665066610256	2.68421919232426\\
1.69364402000437	2.67871836758366\\
1.70044712881062	2.67262974952996\\
1.7070956719489	2.66587175041363\\
1.71361873731923	2.65833922744017\\
1.72003841862168	2.64989750103547\\
1.7263686373693	2.64037391124481\\
1.73261287780803	2.6295458885489\\
1.73876030068066	2.61712396012184\\
1.74477933195565	2.60272722453063\\
1.75060718655548	2.58584737245532\\
1.75613266133152	2.56579488854982\\
1.76116748404913	2.54161687298595\\
1.7653976787231	2.51196854115029\\
1.76829905973407	2.4749071956872\\
1.76898648679696	2.42755320362948\\
1.76593736086363	2.36551782061183\\
1.75647060848263	2.28191688309793\\
1.73574466156984	2.16565555246177\\
1.69482869501793	1.99851591129642\\
1.61718261347504	1.75076654277894\\
1.47352994683164	1.37732437680515\\
1.22051805952349	0.826298262825352\\
0.824104530899889	0.088737425767855\\
0.320869760337405	-0.722231139262628\\
-0.172427576851556	-1.41677043474581\\
-0.564628150823845	-1.90152232511448\\
-0.844164857945204	-2.20555524352634\\
-1.03761732801986	-2.39084560899199\\
-1.17307874967628	-2.50499544609819\\
-1.27057497062747	-2.57706625586182\\
-1.34296497124043	-2.62376301758625\\
-1.39835026011063	-2.6546892307529\\
-1.44189706837267	-2.67549040360413\\
-1.47697734489243	-2.68958397191685\\
-1.50585079022601	-2.69910168381581\\
-1.53007197621501	-2.70541126204121\\
-1.5507371833947	-2.70941263399981\\
-1.56863738283019	-2.71171095290876\\
-1.58435533435496	-2.71272050602903\\
-1.59832868964385	-2.71272876601862\\
-1.61089193830001	-2.71193681655228\\
-1.6223048854811	-2.7104853955052\\
-1.6327723668004	-2.70847195331515\\
-1.64245814281193	-2.705961951801\\
-1.65149485039195	-2.70299636918225\\
-1.65999123121105	-2.6995966271188\\
-1.66803744320138	-2.69576769607146\\
-1.67570899375596	-2.69149984374145\\
-1.6830696565729	-2.68676929722642\\
-1.69017361294467	-2.68153795039362\\
-1.69706697091202	-2.67575213645695\\
-1.70378874713416	-2.6693403819491\\
-1.71037133513253	-2.66220994472263\\
-1.71684041912074	-2.65424179609826\\
-1.72321421250686	-2.64528351033067\\
-1.72950178671067	-2.63513923500997\\
-1.7357000808777	-2.62355547256329\\
-1.74178889769397	-2.61020070304512\\
-1.74772270731284	-2.59463574492772\\
-1.75341723808441	-2.57626987072965\\
-1.75872731967868	-2.55429450216108\\
-1.76340965469847	-2.52758076017022\\
-1.76705891020817	-2.4945172764498\\
-1.7689952370932	-2.45274676514304\\
-1.76806082381912	-2.39872691925673\\
-1.76224153191576	-2.32698079683655\\
-1.74794563915439	-2.22879585130807\\
-1.71861081452158	-2.08997419296628\\
-1.66206523014974	-1.88716184619098\\
-1.55606064235132	-1.58313617346189\\
-1.36362833150289	-1.12657390203955\\
-1.04038019630243	-0.477214984779318\\
-0.579488947169715	0.319746071660618\\
-0.0653357731346624	1.09359382563594\\
0.383438388475592	1.68535391512193\\
0.717113301025199	2.07215344123786\\
0.949730301037335	2.3095776614735\\
1.11114866251162	2.45462986861789\\
1.2256305822835	2.54503016285746\\
1.30931556895403	2.60286539536645\\
1.37240768645625	2.64077716802134\\
1.42136037563239	2.66610339414253\\
1.46033399259921	2.6832198294233\\
1.49207970831041	2.69481567438568\\
1.51846579304674	2.70259274207218\\
1.54079383062352	2.70765720831968\\
1.55999247373734	2.71074520454552\\
1.57673895742309	2.71235644715024\\
1.59153714869271	2.71283556446399\\
};
\addplot [color=mycolor1, forget plot]
  table[row sep=crcr]{%
1.34916822602105	2.61884380257413\\
1.36255348536219	2.61842535800733\\
1.37480683187091	2.61726037653587\\
1.38612752561835	2.61545290337742\\
1.39667669922158	2.61307332288931\\
1.40658638148471	2.61016585503015\\
1.41596613668646	2.60675348673496\\
1.42490800439879	2.60284105382679\\
1.43349020388433	2.598416912692\\
1.44177991815768	2.59345345180439\\
1.44983536863337	2.5879065527139\\
1.45770731417958	2.58171399239607\\
1.46544004576872	2.57479266390242\\
1.47307188948927	2.56703436125339\\
1.48063516614287	2.55829970555158\\
1.48815547206729	2.54840955221251\\
1.49565002401685	2.53713286811873\\
1.50312461938875	2.52416952842272\\
1.5105684469781	2.50912563214587\\
1.51794544380942	2.49147756217942\\
1.52517994674875	2.47051874931118\\
1.53213268434057	2.4452792850012\\
1.53856001621684	2.41440198359606\\
1.5440434126235	2.37594709462455\\
1.54786479332811	2.32707784393597\\
1.54878118856405	2.26354409964061\\
1.54460916042535	2.17882346641676\\
1.53144942360156	2.06269711309166\\
1.50225566229098	1.89898924372039\\
1.44436890588002	1.66251346478269\\
1.33620870262374	1.31721418768455\\
1.14673141186483	0.824026867167908\\
0.849782608285716	0.175946554623806\\
0.461685962801752	-0.548018956645931\\
0.0565516337693074	-1.20228090782282\\
-0.290753796572131	-1.6917502022023\\
-0.554835633179842	-2.01816874854801\\
-0.746358627160081	-2.22639575742822\\
-0.884817909103233	-2.3589451739538\\
-0.986659923035106	-2.44472049770824\\
-1.0634337309332	-2.50144568703854\\
-1.12282433770986	-2.53974002418693\\
-1.16991128279668	-2.5660218412918\\
-1.20809299884232	-2.58425316968421\\
-1.23968783716858	-2.59694161257033\\
-1.26631202944042	-2.60571445312303\\
-1.28911705407786	-2.61165259675128\\
-1.30894048914882	-2.61548905733766\\
-1.32640356331209	-2.617729744945\\
-1.34197546868493	-2.61872870075034\\
-1.35601659862283	-2.61873598932488\\
-1.36880817441296	-2.61792878298912\\
-1.38057291538259	-2.61643186607089\\
-1.39148970600711	-2.61433131778502\\
-1.40170416735516	-2.6116836843858\\
-1.4113363853036	-2.608522081478\\
-1.42048663037264	-2.6048601311055\\
-1.42923963244693	-2.60069429697314\\
-1.43766779293806	-2.5960049543598\\
-1.44583359306231	-2.59075637031308\\
-1.45379136809174	-2.58489564336501\\
-1.46158854894265	-2.57835053799512\\
-1.46926641317011	-2.57102602853142\\
-1.47686032733914	-2.56279922031462\\
-1.48439939058245	-2.55351211746392\\
-1.49190528944329	-2.54296141961962\\
-1.49939002248124	-2.53088409664441\\
-1.50685190825312	-2.51693681519435\\
-1.5142688787915	-2.50066621316221\\
-1.52158734818893	-2.48146525837629\\
-1.5287036798198	-2.4585079950171\\
-1.53543297058567	-2.43064999568616\\
-1.54145557404778	-2.39627321363943\\
-1.5462236004709	-2.35303882637033\\
-1.54879377227572	-2.29748516474355\\
-1.54752207454328	-2.22436239290059\\
-1.53949639900754	-2.12552421645835\\
-1.51947921042305	-1.98811596811773\\
-1.47800070676288	-1.79185510593083\\
-1.39834292507678	-1.50606847102606\\
-1.25373377570828	-1.09096289236972\\
-1.01217443181423	-0.516876345325173\\
-0.663483593147455	0.185309718307808\\
-0.255650441715128	0.892281146234285\\
0.127141962370146	1.46932437295849\\
0.432991150703737	1.87280614464764\\
0.658442536758492	2.13409571476104\\
0.821074581862312	2.30001032816487\\
0.939487523829889	2.40636173951891\\
1.02762555404293	2.47592484154283\\
1.09493604306106	2.52242215305209\\
1.14766234075684	2.55409174160465\\
1.18995036713984	2.57596124909194\\
1.22459961632259	2.59117267921784\\
1.25354049557583	2.60173993143481\\
1.27813357584593	2.60898559194935\\
1.29935845276156	2.61379764430181\\
1.31793477108438	2.61678384887558\\
1.33440123755647	2.61836680110905\\
1.34916822602105	2.61884380257413\\
};
\addplot [color=mycolor1, forget plot]
  table[row sep=crcr]{%
0.967127731322892	2.44416335727229\\
0.982332194010056	2.44368626762445\\
0.996552044814818	2.44233274530933\\
1.00995827727741	2.44019085862995\\
1.02269424379618	2.43731666682272\\
1.03488178584442	2.43373960992044\\
1.04662592338407	2.42946583601168\\
1.05801847480111	2.42447989317166\\
1.06914086623855	2.41874501885892\\
1.08006630690967	2.41220210613007\\
1.09086144272083	2.4047672886263\\
1.10158754476246	2.39632794378619\\
1.1123012330301	2.38673674442344\\
1.12305466889795	2.3758031658969\\
1.13389505829192	2.36328154206065\\
1.1448631692255	2.34885430132771\\
1.15599034564836	2.33210831537199\\
1.16729313058477	2.31250121396592\\
1.17876398162417	2.28931282755925\\
1.19035546509962	2.26157423670833\\
1.20195337503873	2.22796263054996\\
1.21333074848429	2.18664338354513\\
1.22406849048966	2.13503020764076\\
1.23341712189669	2.06941885226787\\
1.24005475817105	1.98443112174681\\
1.24166587486424	1.87219700793212\\
1.23423019400315	1.72125667828972\\
1.21092403527738	1.51545106534366\\
1.1608076636804	1.23401181971476\\
1.06853441869398	0.856195783028522\\
0.918509277596076	0.37568184812346\\
0.706921090474597	-0.177265218634865\\
0.454153574876131	-0.731103204332548\\
0.198053204246737	-1.21018280203609\\
-0.0297170443403976	-1.57844849886536\\
-0.216534441156716	-1.84171786003203\\
-0.364040069802484	-2.02391708406028\\
-0.479373086051614	-2.14919888770883\\
-0.570079113544991	-2.23595700009679\\
-0.642374348028329	-2.29679725082808\\
-0.700942767541163	-2.34003907604897\\
-0.74920207237023	-2.37113492729172\\
-0.789628565488254	-2.39368484478053\\
-0.824025247785614	-2.41009884374503\\
-0.853718838415478	-2.42201651725366\\
-0.879698514691617	-2.43057160046117\\
-0.902712302398931	-2.43655996902277\\
-0.923334050260447	-2.44054766045806\\
-0.942010285885441	-2.44294135282445\\
-0.959093334892685	-2.44403503307794\\
-0.974865011083859	-2.44404132101184\\
-0.989553777211925	-2.44311272620087\\
-1.00334733594278	-2.44135616819522\\
-1.01640198546763	-2.43884288475897\\
-1.02884965678783	-2.43561509341293\\
-1.04080326848977	-2.43169028198276\\
-1.05236084290732	-2.42706367824267\\
-1.06360869408717	-2.42170922170841\\
-1.07462390235819	-2.41557919023216\\
-1.08547621833959	-2.40860249140423\\
-1.0962294803223	-2.40068149139744\\
-1.10694257411915	-2.39168710090024\\
-1.11766990440936	-2.38145164553381\\
-1.12846126953048	-2.36975878532062\\
-1.13936092020188	-2.35632936855015\\
-1.15040540856971	-2.34080153894434\\
-1.161619549011	-2.32270254857145\\
-1.1730093313512	-2.30140838005199\\
-1.1845497978101	-2.27608515356535\\
-1.19616443885553	-2.24560290782785\\
-1.20769006987176	-2.20840694698507\\
-1.21881648820972	-2.16232344046854\\
-1.22898182152812	-2.10426306489148\\
-1.23718963783854	-2.02976886695529\\
-1.24168907404894	-1.93233760331378\\
-1.23942408473097	-1.80245496695398\\
-1.22513423035626	-1.62642196393417\\
-1.19008816520814	-1.38559159763217\\
-1.12101043541084	-1.05813668864252\\
-1.0014442958404	-0.627979956134983\\
-0.81975968717137	-0.104520799938123\\
-0.583468119003452	0.459358633320534\\
-0.324135709273898	0.983425913460733\\
-0.0793924327840767	1.4085297153344\\
0.12836183073586	1.72185853875973\\
0.294802905494589	1.94133196694191\\
0.425234391644833	2.09237330396133\\
0.527378472679993	2.19648606699859\\
0.608201715833219	2.26901494489367\\
0.673135098941547	2.32022387243635\\
0.726187730882803	2.35684603539622\\
0.770268586484437	2.38330546508546\\
0.807488419671494	2.40254198929109\\
0.839391495499743	2.41653933886846\\
0.86712141444169	2.42665821415481\\
0.891536825910838	2.43384682854805\\
0.913291732109646	2.43877538523974\\
0.932891444261152	2.44192316766833\\
0.950731920128378	2.44363578722103\\
0.967127731322892	2.44416335727229\\
};
\addplot [color=mycolor1, forget plot]
  table[row sep=crcr]{%
0.389833257663135	2.13494205399992\\
0.411637012190412	2.13425332784037\\
0.432830583088378	2.13223175394764\\
0.453560693998461	2.12891569696232\\
0.473962470616696	2.1243075922011\\
0.494162472193208	2.11837498054376\\
0.514281289056747	2.11104967794315\\
0.534435780021275	2.10222504132626\\
0.554740988262769	2.09175113467283\\
0.575311732094167	2.07942742190971\\
0.596263810567844	2.06499239801034\\
0.61771468137522	2.04810929394353\\
0.639783341588963	2.02834662784339\\
0.662588940613527	2.00515189276218\\
0.686247331235421	1.9778160401717\\
0.71086424216981	1.94542562462237\\
0.736522914514755	1.90679856576309\\
0.763262708752509	1.86039867194929\\
0.79104312719537	1.80422397231211\\
0.819684692661886	1.73566605619677\\
0.84877425633593	1.65134548513089\\
0.8775187845775	1.54694903865653\\
0.904532894477926	1.41714080153142\\
0.92756298448584	1.25570782784642\\
0.943208032311654	1.0562361419601\\
0.946821913125114	0.813724878526878\\
0.932953901423398	0.527382672453897\\
0.896710338603812	0.203985376296863\\
0.835916667242952	-0.140357545376625\\
0.752908499578301	-0.482727521836322\\
0.654348119758859	-0.800227870027934\\
0.548859926898402	-1.07695991065846\\
0.44418277966519	-1.30673881764637\\
0.345565275630988	-1.49126309862293\\
0.255665541234909	-1.63650765373541\\
0.175238506635367	-1.74971298877467\\
0.103924297554882	-1.83768399565498\\
0.0408371288553757	-1.90612416305119\\
-0.0150784938297504	-1.95954090229694\\
-0.0648626253155773	-2.00138952748404\\
-0.109458271066525	-2.03428043107457\\
-0.149685353120126	-2.06017497680803\\
-0.186241564546573	-2.08054652619508\\
-0.219714564140881	-2.09650461534767\\
-0.250597692592461	-2.10888771948177\\
-0.279305584592668	-2.11833147055275\\
-0.306188200451515	-2.12531846439242\\
-0.331542844513849	-2.13021452029508\\
-0.355624209521529	-2.13329503609864\\
-0.378652673165659	-2.13476408783637\\
-0.400821124569248	-2.13476816612125\\
-0.422300589621874	-2.13340588180217\\
-0.443244891729413	-2.13073456289361\\
-0.463794545707637	-2.12677436071603\\
-0.484080044336877	-2.12151025149346\\
-0.50422466156071	-2.11489213373237\\
-0.524346862829057	-2.10683306033622\\
-0.544562379318148	-2.09720548953563\\
-0.564985964787353	-2.0858352733634\\
-0.585732805647301	-2.07249290875139\\
-0.606919487137589	-2.0568813339019\\
-0.628664316710347	-2.03861923676051\\
-0.651086646328595	-2.01721842383059\\
-0.674304580568565	-1.9920532442971\\
-0.698430046329205	-1.96231935134776\\
-0.723559536426158	-1.92697821841399\\
-0.74975777688409	-1.88468292350284\\
-0.777029900022185	-1.8336801421848\\
-0.805275191803477	-1.7716840361193\\
-0.834211995258691	-1.69572212075661\\
-0.863259374614334	-1.60196627480951\\
-0.891359111601299	-1.48559354762292\\
-0.91672889412078	-1.34078686680571\\
-0.936571286668781	-1.16109958864773\\
-0.946851495651384	-0.940548679227383\\
-0.942415107275109	-0.675822351313804\\
-0.917855840029638	-0.369524113914628\\
-0.869352468419604	-0.0330937167554029\\
-0.79685460095464	0.313315627814919\\
-0.705042783184192	0.64580264069854\\
-0.601938366123092	0.944299305729285\\
-0.496026410028535	1.19775590143363\\
-0.393896012203038	1.4043430387089\\
-0.299443774043	1.56834629821078\\
-0.214272432014153	1.69666805894454\\
-0.138490527366839	1.79646760641875\\
-0.0714171469981265	1.87403677915985\\
-0.0120495608528579	1.93447222249387\\
0.0406752945867412	1.9817307191636\\
0.0877548063802227	2.01881897258306\\
0.130071524556737	2.04800035130604\\
0.168383188990257	2.07097447639075\\
0.203330610054358	2.08901939862155\\
0.235452276844371	2.10309921970234\\
0.265200305703159	2.11394374610395\\
0.292955363290979	2.12210681128536\\
0.319039707043503	2.12800878190824\\
0.34372819552645	2.13196747775331\\
0.3672574253493	2.13422062045018\\
0.389833257663135	2.13494205399992\\
};
\addplot [color=mycolor1, forget plot]
  table[row sep=crcr]{%
-0.306227207130532	1.68738862435859\\
-0.261379969916529	1.68595590112493\\
-0.214795732375831	1.68149627030184\\
-0.166246224137784	1.67371381817293\\
-0.115490049238434	1.66223302401073\\
-0.0622759237369259	1.64658738880165\\
-0.00634848347055719	1.62620670504928\\
0.0525421478533969	1.60040374645278\\
0.114625475832831	1.56836188210587\\
0.180088519800896	1.52912624559407\\
0.249044311164358	1.48160272123015\\
0.321488996788948	1.42457117256621\\
0.397246489756056	1.35672182432064\\
0.475901979302771	1.27672582252266\\
0.556730063408933	1.18335123520745\\
0.638630085725047	1.07563162763586\\
0.720089496514892	0.953082849635422\\
0.799202472199641	0.815943148058473\\
0.873769773125081	0.665385314239871\\
0.941490615215035	0.503628545179529\\
1.00022702608551	0.333879795582411\\
1.04828569470517	0.160073036160582\\
1.08464168652753	-0.0135568843967586\\
1.1090398574796	-0.182963393066682\\
1.12195182993611	-0.344713618929643\\
1.12441644185153	-0.496254799550155\\
1.11782296278774	-0.635992012894199\\
1.10369788255835	-0.763209101950142\\
1.08353585909217	-0.877896160245083\\
1.05868966724269	-0.980546613175455\\
1.03031467339685	-1.07196831526202\\
0.999354059786426	-1.1531310335292\\
0.966549613685775	-1.22505576466716\\
0.93246562408045	-1.28874165082839\\
0.897517326266407	-1.34512240587057\\
0.861998816568914	-1.39504383065693\\
0.826107890702841	-1.43925528372961\\
0.789966863130557	-1.47840971076188\\
0.753639324493781	-1.51306844756036\\
0.717143229840711	-1.54370829105481\\
0.680460870916968	-1.57072926138341\\
0.643546299255664	-1.59446211120164\\
0.606330711723059	-1.61517504828705\\
0.568726230975791	-1.6330793898466\\
0.530628433127489	-1.64833401267997\\
0.491917904745029	-1.66104853925081\\
0.45246105559447	-1.67128523096314\\
0.412110373885495	-1.67905956335708\\
0.370704287784431	-1.68433944479124\\
0.328066791675956	-1.68704301850036\\
0.28400701016003	-1.6870349644935\\
0.238318910849118	-1.68412120003892\\
0.1907814445447	-1.6780418757101\\
0.141159496489401	-1.66846259387974\\
0.0892061853211523	-1.65496386297681\\
0.0346672579765807	-1.63702898237799\\
-0.0227113930501748	-1.61403088740444\\
-0.0831717223145854	-1.58521905359259\\
-0.14692529189719	-1.54970847037251\\
-0.214126735869181	-1.50647406242591\\
-0.284836778083663	-1.45435584111197\\
-0.358973134768415	-1.39208244206276\\
-0.436249208267781	-1.31832313485091\\
-0.516103814772592	-1.23177981597386\\
-0.597630840345963	-1.13132891789912\\
-0.679525486295592	-1.01621571017297\\
-0.760071672073524	-0.886287460348263\\
-0.837198593872501	-0.742227607523675\\
-0.908626835330616	-0.585727435943091\\
-0.972101222376384	-0.419519954114775\\
-1.02567269773647	-0.247220301936271\\
-1.06796069544957	-0.0729730573829008\\
-1.09832162886203	0.0990210510858061\\
-1.11687756207078	0.264977523870333\\
-1.12440879632921	0.421882170576312\\
-1.12215751534027	0.567662524495283\\
-1.11160612860006	0.701180382220813\\
-1.09428272680449	0.822097483980406\\
-1.07162111358625	0.930680848521104\\
-1.04487938471457	1.02760274479467\\
-1.01510672826523	1.11376841378628\\
-0.983143214392898	1.19018455560286\\
-0.949638454263744	1.25786832554039\\
-0.915078603157057	1.31779009367543\\
-0.879814992991123	1.37084140862849\\
-0.844090707442856	1.41782027514089\\
-0.808063458887253	1.45942747136213\\
-0.771824346577996	1.49626934968383\\
-0.73541271568474	1.52886402113299\\
-0.698827615228074	1.55764892337876\\
-0.662036427415693	1.58298854439512\\
-0.624981212737102	1.60518158636642\\
-0.587583244066306	1.62446717822998\\
-0.549746121380202	1.64102993885522\\
-0.511357782671937	1.65500379988569\\
-0.472291663383129	1.66647454828795\\
-0.43240720889068	1.67548106428663\\
-0.391549913177206	1.68201522434357\\
-0.34955104250065	1.68602042054413\\
-0.306227207130532	1.68738862435859\\
};
\addplot [color=mycolor1, forget plot]
  table[row sep=crcr]{%
-0.788141800536172	1.24177643190908\\
-0.666143911461141	1.23782621644052\\
-0.529385546548301	1.22468264945793\\
-0.377598032925018	1.200305816704\\
-0.211503328462869	1.16270309148098\\
-0.033151035064384	1.11025323342796\\
0.153892741506608	1.04211045907784\\
0.344656280172416	0.958584774469102\\
0.533260689535979	0.861346057866915\\
0.713737270154117	0.753323480156497\\
0.880924531852972	0.63828695584622\\
1.03114656753452	0.52024313111183\\
1.16248175759686	0.402853793230899\\
1.27462492668523	0.28904266342515\\
1.36849308754741	0.180844573010999\\
1.44575813645	0.0794519643076454\\
1.50843747841042	-0.0146278478275331\\
1.55859879314411	-0.101382776585321\\
1.59817999649187	-0.181122138434487\\
1.62889940432551	-0.254339608923593\\
1.65222602201337	-0.321615605657848\\
1.66938482224438	-0.383553199496267\\
1.68137941856557	-0.440739234980044\\
1.68902122202907	-0.493722931482854\\
1.69295899129496	-0.543005874624971\\
1.69370577571634	-0.589039019145862\\
1.69166205939839	-0.632223739114816\\
1.68713488522892	-0.672915006659931\\
1.68035320404118	-0.711425502482463\\
1.67147987539793	-0.748029937218246\\
1.66062077750401	-0.782969164058048\\
1.64783144199166	-0.816453845999399\\
1.63312155709335	-0.848667545586881\\
1.61645760174872	-0.879769157777472\\
1.59776379377272	-0.909894624692575\\
1.57692146172094	-0.939157864666103\\
1.55376688433758	-0.967650822563707\\
1.52808758489613	-0.995442505897755\\
1.49961702314798	-1.02257681156117\\
1.46802760103714	-1.0490688693812\\
1.43292190168589	-1.07489952909151\\
1.39382213589897	-1.10000749601614\\
1.3501579134582	-1.12427848162136\\
1.30125274841025	-1.14753059274397\\
1.24631024376023	-1.16949507441256\\
1.18440182338321	-1.18979152402478\\
1.11445938149836	-1.20789696049985\\
1.03527852900531	-1.22310892923822\\
0.945541403358974	-1.23450458424801\\
0.843872161198516	-1.24090101572704\\
0.728942407956638	-1.24082761900118\\
0.599645483779305	-1.23252919451498\\
0.45535282880379	-1.21402729099758\\
0.296245415148313	-1.18327224528407\\
0.12367254535335	-1.13841001001318\\
-0.0595661179947407	-1.07815481856255\\
-0.249164163302196	-1.00219849022497\\
-0.439610195384136	-0.911522452056087\\
-0.624866148744066	-0.808460173525445\\
-0.799261981459283	-0.696429661152239\\
-0.958317522331549	-0.579398221318837\\
-1.099228135799	-0.461263700985717\\
-1.22091815479127	-0.345352723626748\\
-1.32375128971609	-0.23414923729855\\
-1.40907894887731	-0.129252037800879\\
-1.47879034409057	-0.0314877875681934\\
-1.53495731057438	0.0589073177118155\\
-1.57959879465035	0.142101542524296\\
-1.61454971079868	0.218510972211179\\
-1.64140474522503	0.288682572195118\\
-1.6615088143786	0.353214325527291\\
-1.67597278214956	0.412704793477366\\
-1.68570039259146	0.467723849275245\\
-1.69141815567655	0.518797624047021\\
-1.69370383155179	0.5664024541869\\
-1.69301154729384	0.61096420467578\\
-1.68969292068307	0.652860571574267\\
-1.68401425006494	0.692424841635785\\
-1.67617013167615	0.729950176179679\\
-1.66629395933414	0.765693866861409\\
-1.65446574894632	0.799881247186684\\
-1.64071766956537	0.83270908325435\\
-1.62503758428124	0.864348343181529\\
-1.60737082322744	0.894946278472889\\
-1.58762033419848	0.924627755549562\\
-1.56564528659377	0.953495759321603\\
-1.54125814301605	0.981630956651841\\
-1.51422016176231	1.00909015662244\\
-1.4842352568206	1.03590343561629\\
-1.45094212840502	1.06206960633996\\
-1.41390460262001	1.08754959936643\\
-1.37260021186315	1.11225719459006\\
-1.32640725558765	1.13604639638229\\
-1.27459098054248	1.15869461401389\\
-1.21629022788945	1.17988074243297\\
-1.15050707983079	1.19915734777188\\
-1.07610391598764	1.21591665067463\\
-0.991815076452057	1.22935121391861\\
-0.89628409556123	1.23841270101852\\
-0.788141800536172	1.24177643190908\\
};
\addplot [color=mycolor1, forget plot]
  table[row sep=crcr]{%
-0.759481808475352	0.954811303080049\\
-0.464414977518427	0.945239358918495\\
-0.137178419317819	0.913848151332419\\
0.205107567194767	0.859035732518902\\
0.540479022822526	0.783362229568471\\
0.848583269917223	0.693065650564576\\
1.11622371763869	0.595880372562254\\
1.33889721598359	0.498671849537412\\
1.51877538882173	0.406168785154704\\
1.66154340595806	0.320896882719042\\
1.7738798578266	0.243732277033466\\
1.86204322726675	0.174544792552995\\
1.93131125706806	0.1126930881872\\
1.98589505992412	0.0573370457405482\\
2.02905758504342	0.0076098029816707\\
2.06329054280205	-0.0372991895483263\\
2.09048543210269	-0.0781114817153033\\
2.11207672999923	-0.115453032641855\\
2.12915410498925	-0.149859865136441\\
2.14254744934964	-0.181788580002967\\
2.15289020636786	-0.211628032439913\\
2.16066612363249	-0.23971046643039\\
2.16624361766087	-0.266321434193131\\
2.16990095182832	-0.291708335604611\\
2.17184459428808	-0.316087645044589\\
2.17222246794996	-0.339650985413606\\
2.17113330922803	-0.362570232747666\\
2.16863298116275	-0.385001826950307\\
2.16473830582924	-0.407090443167509\\
2.15942876212331	-0.428972152863411\\
2.15264621487007	-0.450777177012557\\
2.14429268050241	-0.472632306258149\\
2.13422597635368	-0.494663032559375\\
2.12225292850496	-0.516995399962178\\
2.10811960970095	-0.539757532349487\\
2.09149782429426	-0.56308072320629\\
2.07196672858698	-0.587099860284975\\
2.04898804725626	-0.611952780355284\\
2.02187279783734	-0.63777786381841\\
1.98973676223325	-0.664708718811725\\
1.95144120150381	-0.692864065538602\\
1.90551470116897	-0.722329760700233\\
1.85005211103009	-0.753128096722297\\
1.78258862099129	-0.785166872057058\\
1.69995396028697	-0.818157275872694\\
1.59812923262143	-0.851486233524169\\
1.47216693255121	-0.884028798398856\\
1.31630617031597	-0.913898206754499\\
1.12452333785629	-0.938173284195572\\
0.891854337270322	-0.952738768752893\\
0.616726685582375	-0.952519250789799\\
0.303922965480029	-0.932459702043087\\
-0.0334113403580325	-0.889312146670426\\
-0.375047850181613	-0.823482720324207\\
-0.698972336787196	-0.739562366176011\\
-0.987934845355978	-0.644862943327322\\
-1.23316467192344	-0.546927933872179\\
-1.43387775149854	-0.451627708476155\\
-1.59438233199738	-0.362543496434898\\
-1.72110483307549	-0.281292466100066\\
-1.82062781970597	-0.208173978793897\\
-1.8987518250651	-0.142752382824162\\
-1.96021453734701	-0.0842579579464069\\
-2.00873229014713	-0.0318223597645517\\
-2.0471598069178	0.0153996296826504\\
-2.07766884025279	0.0581760899462323\\
-2.10190646029094	0.0971806018546626\\
-2.12112253576385	0.132993026489251\\
-2.13626770401609	0.166108231550759\\
-2.14806686295431	0.196947494882579\\
-2.15707362420401	0.225870042184848\\
-2.16371041277271	0.253183617661105\\
-2.16829789455687	0.279153715091212\\
-2.17107649345363	0.304011444539451\\
-2.17222201559771	0.327960160283018\\
-2.17185682732312	0.351181027234208\\
-2.17005760469696	0.373837707814064\\
-2.16686035050174	0.396080335234385\\
-2.16226312799609	0.418048915242808\\
-2.15622676391288	0.439876272143686\\
-2.14867360495218	0.461690627913561\\
-2.13948425462151	0.48361787463078\\
-2.12849205420061	0.505783567443355\\
-2.11547488614215	0.528314622955141\\
-2.10014365189416	0.551340648194742\\
-2.08212648761002	0.574994735392162\\
-2.06094740621443	0.599413416964235\\
-2.0359975682226	0.624735250132764\\
-2.00649677097877	0.651097138197798\\
-1.97144202503041	0.678626911798155\\
-1.92953936980179	0.70742976115612\\
-1.87911471112878	0.737564649699697\\
-1.81800030310009	0.769004637087864\\
-1.7433975081286	0.801571960947746\\
-1.65172775176408	0.834835094188598\\
-1.53850984188323	0.867952605280023\\
-1.39835502910763	0.899453036089182\\
-1.22526238417232	0.926964181094606\\
-1.01351140865888	0.94697152112484\\
-0.759481808475353	0.954811303080049\\
};
\addplot [color=mycolor1, forget plot]
  table[row sep=crcr]{%
-0.324618520603023	0.837628893195232\\
0.141649275301933	0.822760735661914\\
0.596881567596064	0.779453357448525\\
1.0021395912943	0.714932864138764\\
1.3371668551542	0.639646692660512\\
1.60075351461313	0.562611696507694\\
1.80261577585573	0.489443428944657\\
1.95563109944708	0.422718211680479\\
2.07167039486615	0.363082993210658\\
2.16024047217468	0.310200454343011\\
2.2284812634194	0.263332621855819\\
2.28160509550794	0.22164422755014\\
2.32338034966147	0.184339950814669\\
2.35653235040297	0.150715553516891\\
2.38304317205515	0.120168430000446\\
2.404366145207	0.0921909486791049\\
2.42157654482902	0.0663577986355005\\
2.43547688557959	0.0423123706696491\\
2.44667060615226	0.0197542087083929\\
2.45561387912914	-0.0015718189538681\\
2.46265225208374	-0.0218845843751558\\
2.46804668999558	-0.0413741934350738\\
2.47199212297803	-0.060208170850693\\
2.47463060535432	-0.078536598710198\\
2.47606051073955	-0.0964964436946383\\
2.47634271644072	-0.114215275566578\\
2.47550439601706	-0.131814549621168\\
2.47354079088737	-0.149412601717515\\
2.47041513469672	-0.167127486921179\\
2.46605673063414	-0.185079781328189\\
2.46035700892522	-0.203395460421398\\
2.45316319636694	-0.222208964955927\\
2.44426898542543	-0.241666564632794\\
2.4334012623827	-0.26193012664474\\
2.42020149376975	-0.283181382844243\\
2.4041997078324	-0.305626750788811\\
2.38477804171665	-0.329502671013201\\
2.36111941247666	-0.35508121863716\\
2.33213482280654	-0.382675321490496\\
2.29635992462281	-0.412642056840548\\
2.25180763874941	-0.445380800680903\\
2.1957592938514	-0.48131971990023\\
2.1244740268668	-0.520877920134854\\
2.03280193886641	-0.564379475232355\\
1.91371936967003	-0.611877464082452\\
1.75790851186681	-0.662822678866233\\
1.55376102383069	-0.715501866957153\\
1.28867822650361	-0.766243460037781\\
0.953077794663284	-0.808695686544094\\
0.547943309889831	-0.83411341498746\\
0.0929521451383333	-0.833937257179846\\
-0.373411518321853	-0.804349215721416\\
-0.807532662055344	-0.749203359525841\\
-1.17887417675215	-0.678000007762318\\
-1.47743180119758	-0.600914131355315\\
-1.70860296234771	-0.525324568057738\\
-1.88443811822814	-0.455202648837094\\
-2.01761943598438	-0.392020878970487\\
-2.11889173447411	-0.335838989529977\\
-2.19653701794795	-0.286066804736578\\
-2.25666864061581	-0.241892517007049\\
-2.3037204981097	-0.202490471586393\\
-2.34089583370865	-0.167107781766933\\
-2.3705167105481	-0.135091149775208\\
-2.39427849140452	-0.105886833462269\\
-2.41342985701326	-0.0790300757239985\\
-2.4288987866803	-0.0541315781974386\\
-2.44138060147967	-0.030864274957714\\
-2.4513997071776	-0.00895161256239827\\
-2.45935313333664	0.0118423818282623\\
-2.4655414085426	0.0317213486275997\\
-2.4701905387983	0.0508634012982991\\
-2.47346764623651	0.0694267567626212\\
-2.47549200150645	0.0875544409215963\\
-2.4763426177843	0.105378288172085\\
-2.4760631789495	0.123022421945924\\
-2.47466478847366	0.140606376124166\\
-2.47212680681249	0.158247996346675\\
-2.46839586278682	0.176066245797393\\
-2.46338295398581	0.194184031342863\\
-2.45695837006585	0.212731161827081\\
-2.4489439566827	0.231847549143729\\
-2.4391019566509	0.251686761427661\\
-2.42711927798657	0.272420030759712\\
-2.41258548757802	0.29424079441131\\
-2.39496202999984	0.317369787919756\\
-2.37353900245125	0.342060569360835\\
-2.34737411283272	0.368605057766904\\
-2.31520600437833	0.397338061428061\\
-2.27533076853194	0.428638560214717\\
-2.2254262700896	0.462923138093415\\
-2.16230492395649	0.500622438918312\\
-2.08157571644412	0.542123177117839\\
-1.97721206215354	0.587643828606011\\
-1.84108271073518	0.636990557641272\\
-1.66267094055474	0.689118820395049\\
-1.4295781838639	0.74144414246256\\
-1.12998729811571	0.789012731023362\\
-0.758476839361802	0.824122371675962\\
-0.324618520603024	0.837628893195232\\
};
\addplot [color=mycolor1, forget plot]
  table[row sep=crcr]{%
0.191200421204062	0.824265275129285\\
0.71325592275401	0.807963511485502\\
1.16345759404792	0.765425392551587\\
1.52252052148191	0.708448771096694\\
1.79587642946785	0.647122748221874\\
1.9997464806358	0.587587061317099\\
2.15129352137004	0.532674364167705\\
2.26469168011602	0.483229473891889\\
2.35052395916227	0.439117036612124\\
2.41636375999719	0.399802546765864\\
2.46755317204297	0.364641438372434\\
2.50785972415183	0.333007084390018\\
2.5399593089937	0.304338968099849\\
2.56577253858966	0.278154207307022\\
2.58669425502056	0.254043669529567\\
2.60374943665117	0.231662595783897\\
2.61769948976084	0.210720119939289\\
2.62911526822383	0.190969460988905\\
2.63842769866772	0.172199374681048\\
2.64596319880481	0.154226933192088\\
2.6519686414066	0.1368914962194\\
2.65662901872332	0.120049672038544\\
2.66007991001322	0.103571061667474\\
2.66241615420358	0.0873345957210633\\
2.66369765410891	0.071225294808439\\
2.66395290566747	0.0551313030145495\\
2.66318060062165	0.0389410570502846\\
2.66134945605483	0.0225404598091692\\
2.65839625066598	0.00580992571133416\\
2.65422187103344	-0.0113788442080452\\
2.64868496619965	-0.0291665176937357\\
2.64159254520199	-0.0477104051036007\\
2.63268648800917	-0.0671895773652435\\
2.62162441457903	-0.0878111005514818\\
2.60795257653423	-0.109817710668383\\
2.59106725814224	-0.133497293444217\\
2.57015937826012	-0.159194529467751\\
2.54413424393695	-0.187324931307483\\
2.51149425802899	-0.218391047844017\\
2.47016627992243	-0.252999427725799\\
2.41724696483688	-0.291874127123704\\
2.34862987349136	-0.335856254401098\\
2.25847372891593	-0.385865586756277\\
2.13849379012915	-0.442773260758363\\
1.97717055072386	-0.507086259797404\\
1.75931890171518	-0.578280346889523\\
1.46727702516943	-0.653613616569171\\
1.08617723612172	-0.726578100557776\\
0.615460439420353	-0.786230272795211\\
0.0822835338711193	-0.819919980671972\\
-0.458755077233071	-0.82004513526283\\
-0.94927335353285	-0.789253224250595\\
-1.35445595364017	-0.738024537394433\\
-1.66900199975997	-0.677853055182354\\
-1.90537723635566	-0.616892504600364\\
-2.08108780044098	-0.55946756251732\\
-2.21202005649414	-0.507262547905664\\
-2.31051905310019	-0.46053519056928\\
-2.38556530379483	-0.418899791716337\\
-2.44352386353713	-0.38174322250028\\
-2.48887853447455	-0.34842024462932\\
-2.52480076725475	-0.318334243633819\\
-2.55355433630708	-0.290963369918397\\
-2.57677354714022	-0.265862593178887\\
-2.59565238416989	-0.242656271453801\\
-2.61107314174097	-0.221027883359438\\
-2.62369442650236	-0.200709758129085\\
-2.63401188594597	-0.181473866908048\\
-2.64240050775657	-0.163123947349628\\
-2.6491443324095	-0.145488902178262\\
-2.65445744888948	-0.128417291597656\\
-2.65849884811449	-0.111772710978315\\
-2.66138285164333	-0.0954298538359701\\
-2.66318625772025	-0.0792710803233651\\
-2.66395295068996	-0.0631833320099094\\
-2.66369643639807	-0.0470552498199148\\
-2.66240055017646	-0.0307743616965002\\
-2.66001840294916	-0.0142242090219379\\
-2.65646945892056	0.00271872448881402\\
-2.65163445089512	0.0201884314122112\\
-2.64534760931634	0.0383333196120541\\
-2.63738537336875	0.0573208771826185\\
-2.62745031680829	0.0773433114590995\\
-2.61514838243709	0.0986244601731538\\
-2.5999565620817	0.121428321698415\\
-2.58117670599581	0.146069576303246\\
-2.55786892677725	0.172926416392208\\
-2.52875468402557	0.202455744887065\\
-2.49207457729266	0.235210057075738\\
-2.44537863236689	0.271853482986669\\
-2.38521758771343	0.313170239913061\\
-2.30669532153084	0.360049484777172\\
-2.20284719577246	0.413411291637233\\
-2.06386455985811	0.474001591350687\\
-1.87639089359498	0.541924817937139\\
-1.62366965142184	0.615730639568154\\
-1.2884049236082	0.690980647379509\\
-0.861085217365527	0.758884641393307\\
-0.353732957529172	0.807004156430304\\
0.19120042120406	0.824265275129285\\
};
\addplot [color=mycolor1, forget plot]
  table[row sep=crcr]{%
0.613287781213732	0.86010797340398\\
1.11511674731886	0.844659582097954\\
1.51768457820977	0.806751425025845\\
1.82376588317216	0.758240526604769\\
2.0509352373577	0.707296184889737\\
2.21880388620532	0.658276776173363\\
2.34371427611363	0.613012334620282\\
2.43782537863024	0.571971938560831\\
2.50977234567591	0.534990445442697\\
2.56559421684612	0.501653186781069\\
2.6095150415272	0.471480657201323\\
2.64451273544842	0.444009402733239\\
2.67271096584353	0.418822514218511\\
2.69564373909931	0.395556952080649\\
2.71443341082278	0.373900962812272\\
2.72991101586408	0.353587826889258\\
2.74269832317876	0.334388753816769\\
2.75326437201219	0.316106104164191\\
2.7619648292203	0.298567354270926\\
2.7690696320498	0.281619874727851\\
2.77478251664214	0.265126448280041\\
2.7792548183247	0.248961399474088\\
2.78259512976479	0.233007195475165\\
2.78487586620252	0.217151380124197\\
2.78613741627532	0.201283709565552\\
2.786390289109	0.185293361868146\\
2.78561546044436	0.169066091981578\\
2.78376294122543	0.152481195182978\\
2.78074841529247	0.135408125132969\\
2.77644759298665	0.117702584711849\\
2.77068767443536	0.099201866301543\\
2.76323496938589	0.0797191598005111\\
2.75377722007983	0.059036467714995\\
2.74189842845168	0.0368956642359729\\
2.72704285592108	0.0129871103311403\\
2.70846311833792	-0.0130648973866359\\
2.68514459016501	-0.0417206655341929\\
2.65569413444983	-0.0735483955114565\\
2.61817476350412	-0.109252478769324\\
2.56985843389196	-0.149705358213237\\
2.50685675263342	-0.195977005304862\\
2.4235774834724	-0.249345084109551\\
2.31195942416062	-0.311243951870691\\
2.16051209060582	-0.383058764199705\\
1.95347159308953	-0.465579411248454\\
1.67116729490645	-0.557829172076192\\
1.29418386758641	-0.655100904556978\\
0.814779939119185	-0.746989151120519\\
0.253872314130891	-0.818278428838929\\
-0.332556849944329	-0.855612892402552\\
-0.87571841253119	-0.8559992687482\\
-1.32918499612098	-0.827708316243736\\
-1.68184436630816	-0.783209949142495\\
-1.94594384520362	-0.732724453171794\\
-2.14116067344381	-0.682388056643545\\
-2.28577545587929	-0.635124543600925\\
-2.39400860433696	-0.591965379762945\\
-2.47614130760329	-0.552996748802563\\
-2.53940003140643	-0.517896024319721\\
-2.58883226427853	-0.486201189920134\\
-2.62798013867384	-0.457434674716157\\
-2.65935451449369	-0.431154363145532\\
-2.68475737519204	-0.406970189330942\\
-2.70549880674259	-0.384545403180371\\
-2.72254321948034	-0.363591643989305\\
-2.73660859151485	-0.343862040311453\\
-2.74823448933603	-0.325144182371768\\
-2.75782918081735	-0.307253691506714\\
-2.76570258746931	-0.290028595005323\\
-2.77208950790246	-0.273324488302417\\
-2.77716604157746	-0.257010376088385\\
-2.78106115814751	-0.240965055130788\\
-2.78386470419543	-0.225073898559042\\
-2.78563269502162	-0.209225906815504\\
-2.78639042625926	-0.1933108961482\\
-2.78613370669727	-0.177216697306981\\
-2.7848283235279	-0.160826232686187\\
-2.78240767628803	-0.144014327774542\\
-2.77876833081706	-0.126644090570417\\
-2.77376302238152	-0.108562658267982\\
-2.76719034317613	-0.089596061029038\\
-2.75877993484113	-0.069542884450331\\
-2.74817139791829	-0.0481663218772056\\
-2.73488421319197	-0.0251840934520167\\
-2.71827456526713	-0.000255576276756123\\
-2.69747278524308	0.0270346355114722\\
-2.67129175488589	0.0571975662114773\\
-2.63809141127379	0.0908656856475478\\
-2.59557666716171	0.128823353730755\\
-2.54049503654738	0.172038704817913\\
-2.46818719690168	0.221686691139371\\
-2.37193666515798	0.279136317997162\\
-2.24209351508817	0.345838650164078\\
-2.06510027125647	0.422981151964812\\
-1.82303820318996	0.510667469332616\\
-1.49546056309268	0.606340464715187\\
-1.06683713755812	0.702604396053219\\
-0.54172868935217	0.78620186581936\\
0.0405998642924217	0.84168534999174\\
0.61328778121373	0.86010797340398\\
};
\addplot [color=mycolor1, forget plot]
  table[row sep=crcr]{%
0.926393358475887	0.917706271260934\\
1.38607263076893	0.903665969310752\\
1.7422016593669	0.870176702341646\\
2.00870321996272	0.827950270283077\\
2.20597730285032	0.783707671619406\\
2.3525210968257	0.740908734638542\\
2.46258707079647	0.701016603024936\\
2.54644387593162	0.664441849350182\\
2.61130548498574	0.63109723987738\\
2.66221582625019	0.600689035826832\\
2.70272100981106	0.572859548493884\\
2.73534034840694	0.547252309736575\\
2.76188629984635	0.523538805080332\\
2.78367988503753	0.50142682268343\\
2.80169603717409	0.480660429960683\\
2.81666241613587	0.461016464022597\\
2.82912726059294	0.442299855659283\\
2.83950647434844	0.424338836503721\\
2.84811662193418	0.406980457837514\\
2.85519822357702	0.390086551090085\\
2.86093225475325	0.373530120957746\\
2.86545178198856	0.357192100266858\\
2.86885001841987	0.34095836999222\\
2.8711856407957	0.324716937328195\\
2.87248589696127	0.308355158613597\\
2.87274779812196	0.291756886363926\\
2.87193749709468	0.274799407022081\\
2.86998777400157	0.257350015428979\\
2.86679335848258	0.239262040489446\\
2.86220358397008	0.220370090483013\\
2.85601155816496	0.20048422133587\\
2.84793859215063	0.179382641104984\\
2.83761197920958	0.15680244216955\\
2.82453322963001	0.132427693010208\\
2.8080323544497	0.105874023092661\\
2.78720144224183	0.0766686140974801\\
2.76079711622966	0.0442243342015672\\
2.72709580713587	0.00780680891789201\\
2.68367726847776	-0.0335059966993132\\
2.62709978763967	-0.0808694015873919\\
2.55241651917453	-0.135713101802578\\
2.45247556986243	-0.199749541256519\\
2.31698107075966	-0.274879353275059\\
2.13146950088213	-0.362838550985554\\
1.87691333707886	-0.46429770138626\\
1.53194166745117	-0.577051012339622\\
1.08127310409524	-0.693415470461945\\
0.532134511957492	-0.798831354735789\\
-0.0714694655728092	-0.8757747316268\\
-0.659035959560127	-0.913403657671121\\
-1.16938961652807	-0.913918239494097\\
-1.57651842842924	-0.888592637089835\\
-1.88539534489321	-0.849644055527742\\
-2.11474579460562	-0.805803912152524\\
-2.2845970351354	-0.762002705065793\\
-2.41138888813924	-0.720557114614023\\
-2.50728045874624	-0.682312710598663\\
-2.58089194300237	-0.647381474947317\\
-2.63825421303898	-0.615547964013723\\
-2.68359190121149	-0.586474698142311\\
-2.71988950170122	-0.559799491160688\\
-2.74928039355392	-0.535177979701714\\
-2.77330943370495	-0.512299438072088\\
-2.79310978287705	-0.490890090764409\\
-2.80952261211717	-0.470710936834829\\
-2.82317887741413	-0.451553467220147\\
-2.83455576976416	-0.433234848244482\\
-2.84401608759779	-0.415593254807294\\
-2.85183594086373	-0.39848360497849\\
-2.85822435538924	-0.381773742998982\\
-2.86333714558525	-0.365341024272321\\
-2.86728663107626	-0.349069215733003\\
-2.87014823964302	-0.332845608679193\\
-2.87196466935385	-0.316558233809178\\
-2.87274801434064	-0.300093061929035\\
-2.87248004864847	-0.283331064086473\\
-2.87111067944473	-0.266144988608059\\
-2.86855439809411	-0.248395686846582\\
-2.86468434858169	-0.229927781173933\\
-2.85932336565911	-0.210564413826294\\
-2.85223096645201	-0.190100738411087\\
-2.84308474492088	-0.16829571084183\\
-2.83145381965049	-0.144861596391372\\
-2.81676076664323	-0.119450430002067\\
-2.79822658503956	-0.0916364535841698\\
-2.77479031253233	-0.0608933422327805\\
-2.74499035284064	-0.0265649348789936\\
-2.70678760765723	0.0121714882088764\\
-2.65730028311564	0.0563488156370922\\
-2.59240679033264	0.107255467266795\\
-2.50616063962901	0.166465520453224\\
-2.38996807466756	0.235808451596563\\
-2.23156557364936	0.317172385697494\\
-2.01416104841412	0.411922540994109\\
-1.71698590740431	0.519583279163353\\
-1.32012399554165	0.635540125842404\\
-0.817175390612008	0.748615574455404\\
-0.232995168271264	0.841817858572328\\
0.371751247386078	0.899672684197672\\
0.926393358475885	0.917706271260934\\
};
\addplot [color=mycolor1, forget plot]
  table[row sep=crcr]{%
1.15917423619427	0.985452996961435\\
1.57605581690044	0.97276915929865\\
1.89441343642087	0.942843264494399\\
2.13227683472436	0.905150342403449\\
2.30942407846641	0.865413163486818\\
2.44232774805049	0.826589535101777\\
2.54329864806614	0.789986729526301\\
2.62113458251352	0.756032335676635\\
2.6820299188861	0.724722222994033\\
2.73034635185163	0.69585971565669\\
2.76917854378417	0.669176729546374\\
2.8007467931326	0.644392180882967\\
2.82666408408834	0.621238171226077\\
2.84811697000639	0.599470039069359\\
2.86598860922422	0.578868577273424\\
2.88094314711286	0.559238633514245\\
2.8934841634484	0.540406200432067\\
2.90399555917694	0.522215017548692\\
2.91277040371961	0.504523151960812\\
2.92003140075226	0.487199742292122\\
2.92594540855611	0.470121947586638\\
2.93063364090557	0.453172070861024\\
2.93417862736339	0.43623478986836\\
2.93662863152781	0.419194406100511\\
2.93799994881924	0.401932006621231\\
2.93827728833417	0.384322415691087\\
2.93741225391446	0.366230789939307\\
2.93531975068679	0.34750867822912\\
2.93187192760633	0.327989321208702\\
2.92688899133663	0.307481900752102\\
2.92012584763972	0.285764359574226\\
2.91125297782018	0.262574288144739\\
2.89982913907324	0.237597210458558\\
2.88526223146898	0.210451384549692\\
2.86675275384526	0.180667969274259\\
2.84321129383027	0.147665126580317\\
2.81313689528631	0.110714438865427\\
2.77443615632001	0.0688982370962552\\
2.72415278113033	0.0210578615885913\\
2.6580642866756	-0.0342625301334779\\
2.57009133700856	-0.0988597098543036\\
2.45147526364566	-0.174856072119426\\
2.28977247488069	-0.26451353004594\\
2.068051001765	-0.369643323146168\\
1.76555664303862	-0.49022874188461\\
1.3626712298896	-0.62196851668595\\
0.853498448339034	-0.753558732612605\\
0.263067043959383	-0.867084769427221\\
-0.34858168173437	-0.945253034235647\\
-0.911523896811392	-0.981455505235464\\
-1.38070599042454	-0.982007452859509\\
-1.7465684190745	-0.959275748618546\\
-2.02212325732471	-0.924530624568435\\
-2.22731078007852	-0.885302064978562\\
-2.3805439581278	-0.845777644851068\\
-2.49618350570722	-0.807969731568821\\
-2.58467037086583	-0.77267217971083\\
-2.65339185681341	-0.740056351422403\\
-2.70754271270146	-0.71000099655222\\
-2.75079235057624	-0.682263459422644\\
-2.78575807593243	-0.656564373723719\\
-2.8143293235244	-0.632627245573377\\
-2.83788741038465	-0.610195162061931\\
-2.8574545906673	-0.589036205167183\\
-2.87379582783211	-0.568943497701945\\
-2.88748893302291	-0.549732869739\\
-2.89897339195555	-0.531239620165527\\
-2.90858467669404	-0.513315071683582\\
-2.91657853317752	-0.495823221250298\\
-2.92314822799873	-0.478637586915617\\
-2.92843674480804	-0.461638250687693\\
-2.93254525685895	-0.44470904568158\\
-2.9355387481116	-0.427734808183106\\
-2.93744933310586	-0.410598597259897\\
-2.93827758337044	-0.393178768144887\\
-2.93799196850087	-0.375345765687661\\
-2.93652633419291	-0.35695847674322\\
-2.93377514082864	-0.337859941550254\\
-2.92958594539539	-0.317872169395212\\
-2.92374828895184	-0.296789727417868\\
-2.91597769794034	-0.274371665981091\\
-2.90589283948323	-0.250331200815598\\
-2.89298286274102	-0.224322382197802\\
-2.87656041322932	-0.195922740098191\\
-2.85569341655181	-0.164610613804974\\
-2.82910502268571	-0.129735616258361\\
-2.79502541026458	-0.0904806364713247\\
-2.75097066118413	-0.0458144539665412\\
-2.69341216109772	0.00556331091612401\\
-2.61728671989778	0.0652755618461577\\
-2.51529289624149	0.135290740074671\\
-2.37695842808772	0.217842197730919\\
-2.18764946166002	0.315079127957206\\
-1.92825794555101	0.42813693392327\\
-1.57755809935925	0.555223929896933\\
-1.12070581839738	0.688795150854699\\
-0.565356701070268	0.813805801254907\\
0.04470746832057	0.911337695620967\\
0.639826601490393	0.968452871849045\\
1.15917423619427	0.985452996961435\\
};
\addplot [color=mycolor1, forget plot]
  table[row sep=crcr]{%
1.33832785075596	1.05856770426014\\
1.71708873794132	1.04706270790821\\
2.00524308812722	1.01997360948505\\
2.22158293985867	0.98568232028984\\
2.38421685443873	0.949191101343268\\
2.5076145272811	0.91313611260959\\
2.6024663989092	0.87874493050746\\
2.67642270413419	0.846477797541033\\
2.7349083019084	0.816402633286111\\
2.78177963434076	0.78840014646188\\
2.81980118326959	0.762271550607141\\
2.85097669866004	0.737793146276745\\
2.87677601172378	0.714742657350381\\
2.89829006124147	0.692910811688235\\
2.91633737206639	0.672105353139954\\
2.93153778559506	0.652151275517667\\
2.9443639890652	0.632889257135411\\
2.95517784931997	0.614173310232574\\
2.96425621180258	0.595868146084142\\
2.97180927822365	0.577846482438437\\
2.97799364973239	0.559986372597205\\
2.982921432524	0.542168554727318\\
2.98666632910257	0.524273773221959\\
2.9892673015886	0.506179992999476\\
2.99073014005198	0.487759402000392\\
2.99102706084517	0.46887507006363\\
2.99009426746616	0.449377098720827\\
2.9878272023742	0.429098051468598\\
2.98407297275139	0.407847392434451\\
2.97861910870765	0.38540457648988\\
2.97117735555587	0.36151031754284\\
2.96136053270739	0.335855404031068\\
2.94864948746336	0.308066221173719\\
2.93234564125112	0.27768587209012\\
2.91150227341512	0.244149475236976\\
2.88482407104405	0.206751914642895\\
2.8505189770917	0.164606224499872\\
2.80607829448728	0.11659141871253\\
2.74795009456766	0.0612912218850788\\
2.67105933973035	-0.00306711709908142\\
2.56812617464324	-0.0786451112657921\\
2.4287765554169	-0.167922293729583\\
2.23862361750258	-0.273355629826909\\
1.97903645009366	-0.396452773264885\\
1.62946597084443	-0.535843685786365\\
1.17549772978648	-0.684372465629206\\
0.623913899953506	-0.827066784603145\\
0.0156603832950979	-0.944197989256905\\
-0.582713904244943	-1.02082682525928\\
-1.1109120882187	-1.05489053430368\\
-1.54004749962896	-1.05543374616649\\
-1.87136800357853	-1.03485403068786\\
-2.12118690488697	-1.00334708083351\\
-2.30861435903366	-0.967504177395819\\
-2.45007070231361	-0.931008400315427\\
-2.55806939607659	-0.895691241500445\\
-2.64167391994001	-0.862335439946601\\
-2.70732795253916	-0.831170891254812\\
-2.75960186319709	-0.802153688786899\\
-2.80175664512797	-0.775115414285042\\
-2.83614241827917	-0.749840185559272\\
-2.86447275609878	-0.726102868844767\\
-2.88801214554155	-0.703686833934361\\
-2.90770435438771	-0.682391113449945\\
-2.92426093572038	-0.662032191603923\\
-2.9382227944058	-0.642443165421628\\
-2.95000341218912	-0.623471699475948\\
-2.9599194431015	-0.604977491641939\\
-2.96821248644699	-0.58682959233689\\
-2.97506458571853	-0.568903718994405\\
-2.98060916195791	-0.55107959889393\\
-2.98493852032551	-0.533238312675461\\
-2.98810867137945	-0.515259573693341\\
-2.99014191896274	-0.497018851100946\\
-2.99102743998762	-0.47838421890837\\
-2.99071988462755	-0.459212783487904\\
-2.98913583058736	-0.439346503305103\\
-2.98614770441886	-0.418607162070534\\
-2.981574503371	-0.396790184171176\\
-2.97516826859042	-0.373656881793238\\
-2.96659470971629	-0.348924587454162\\
-2.95540556317111	-0.322253943280443\\
-2.94099902838946	-0.293232380082817\\
-2.92256272972903	-0.261352524893812\\
-2.898990733363	-0.22598395477037\\
-2.86876167890882	-0.186336478897755\\
-2.82975838133331	-0.141413299954517\\
-2.77899972415068	-0.08995380351499\\
-2.71224382577774	-0.030370353344629\\
-2.62341239539384	0.0393040537726912\\
-2.50379953054959	0.121410226009802\\
-2.34112351143517	0.218486204153281\\
-2.11880645853817	0.332683514259896\\
-1.81669174283804	0.464386512203611\\
-1.41581099600258	0.609717854256119\\
-0.910199242096203	0.757659551907771\\
-0.32300024499543	0.8900063214995\\
0.289093551297695	0.98803779850949\\
0.858233627317148	1.04278813347991\\
1.33832785075596	1.05856770426014\\
};
\addplot [color=mycolor1, forget plot]
  table[row sep=crcr]{%
1.48195524741105	1.13499600551441\\
1.82771084245232	1.12449792923133\\
2.09121070631347	1.09971889927061\\
2.29056202585284	1.06811011905157\\
2.44200222497211	1.03412112200209\\
2.5582243973705	1.00015506933652\\
2.64857921337322	0.967388428797628\\
2.71979414609436	0.936312684715988\\
2.77668243987642	0.907055225989295\\
2.82270084726161	0.879559355166109\\
2.86035316545087	0.85368207890044\\
2.89147238204279	0.829245848782393\\
2.91741560862578	0.806065029860011\\
2.93919865663493	0.783958663217387\\
2.95758940438435	0.762755871451644\\
2.97317306982624	0.74229736750022\\
2.98639822665132	0.722434940873958\\
2.9976094924628	0.703029924367371\\
3.00707087054564	0.683951164254744\\
3.01498242644814	0.665072750318575\\
3.02149210661462	0.646271611685118\\
3.02670391018261	0.627424996845516\\
3.03068320802233	0.60840780146909\\
3.03345969887125	0.589089668398643\\
3.03502825503851	0.569331749997934\\
3.03534770547678	0.548982986595672\\
3.03433740348466	0.52787571031182\\
3.03187120178483	0.505820325345909\\
3.02776817638863	0.482598737354212\\
3.0217790587727	0.457956097902255\\
3.01356679025977	0.431590285446009\\
3.00267880784298	0.403138350895174\\
2.98850746087563	0.372158903955268\\
2.9702331191046	0.33810910603951\\
2.94674172850081	0.300314597460034\\
2.91650432599898	0.257930436006693\\
2.87739975294452	0.209891288074655\\
2.82645308816859	0.154850528636367\\
2.75945194470782	0.0911125846386886\\
2.67039604942962	0.0165753827145177\\
2.55075154984237	-0.0712701882940495\\
2.38857573601612	-0.175171812629412\\
2.16788371645304	-0.297546184921392\\
1.86937402705675	-0.439126103017382\\
1.47485565200519	-0.596498607285113\\
0.978024766685259	-0.759158540164599\\
0.399280619236233	-0.909031700309866\\
-0.209222691531337	-1.02637010365967\\
-0.782516564403628	-1.09990426092498\\
-1.27340608465234	-1.13162079951875\\
-1.66619789036641	-1.13213477534486\\
-1.96860925157471	-1.11334737001622\\
-2.19781258107627	-1.08443074321809\\
-2.37139232321267	-1.05122597349927\\
-2.50386060594156	-1.01704058497466\\
-2.60616221235153	-0.983579677611965\\
-2.68624082346912	-0.951625326146515\\
-2.74978669526022	-0.921457337484466\\
-2.80087536645179	-0.893094774491843\\
-2.84244511941849	-0.866429077725958\\
-2.87663525618585	-0.841295442698824\\
-2.90502076456061	-0.817510021061471\\
-2.92877422841182	-0.79488848704882\\
-2.94877782711592	-0.77325454830531\\
-2.9657013206889	-0.752443091626493\\
-2.98005679825384	-0.732300512698212\\
-2.99223742913519	-0.712683603755642\\
-3.00254507357401	-0.693457726910913\\
-3.0112100193343	-0.674494643567592\\
-3.01840504582332	-0.655670170619315\\
-3.0242552972786	-0.636861720027347\\
-3.02884494997068	-0.617945709956243\\
-3.0322213045204	-0.598794790290315\\
-3.03439666846907	-0.579274789687264\\
-3.03534817700269	-0.559241256781215\\
-3.03501550055304	-0.538535428402326\\
-3.03329617915488	-0.51697940709317\\
-3.03003807513199	-0.494370262762469\\
-3.02502811051957	-0.47047268187135\\
-3.01797600141845	-0.445009663280249\\
-3.00849104089377	-0.417650592306105\\
-2.99604899719174	-0.387995802807642\\
-2.97994470381437	-0.355556454808895\\
-2.95922364687653	-0.31972822362765\\
-2.93258239896873	-0.279756978818976\\
-2.89822256475082	-0.234694525213657\\
-2.85363542562227	-0.183343111872696\\
-2.79528466319329	-0.124190093700613\\
-2.71814479248552	-0.0553419146260905\\
-2.61505367687108	0.0255137852239566\\
-2.47588198710784	0.121044266512722\\
-2.28669913561442	0.233941461232089\\
-2.02961011797109	0.366015313191922\\
-1.68496239277464	0.516299897716017\\
-1.23873660878186	0.67815187200221\\
-0.696280598734906	0.83700673279516\\
-0.0946098487340973	0.972777609559879\\
0.503930688914962	1.06878127437136\\
1.03995889446949	1.12043375648529\\
1.48195524741105	1.13499600551441\\
};
\addplot [color=mycolor1, forget plot]
  table[row sep=crcr]{%
1.60151165444191	1.21376353859017\\
1.91865392825986	1.20413163687514\\
2.16146055251207	1.18128912214335\\
2.34678220315943	1.15189476983143\\
2.48906901537889	1.11995145575896\\
2.5994810729338	1.08767646335323\\
2.68624928650806	1.05620504317403\\
2.75533692442773	1.02605331630931\\
2.81105064312716	0.997396557024017\\
2.85651480026093	0.970229115480782\\
2.89401520611343	0.944453983022338\\
2.9252412308416	0.919931966326893\\
2.95145466158424	0.896508048638974\\
2.97360744874277	0.874024968197798\\
2.99242420165329	0.852329669320216\\
3.00846039089377	0.831275798408647\\
3.0221437148727	0.810724016880521\\
3.03380368338312	0.790541112544326\\
3.04369284201066	0.770598441972192\\
3.05200195991931	0.750769978441166\\
3.05887075305835	0.730930088589762\\
3.06439519462896	0.710951068268945\\
3.06863209306283	0.690700406598752\\
3.07160133872594	0.670037700497916\\
3.07328599384936	0.648811099381227\\
3.07363019438329	0.626853113692275\\
3.07253461944072	0.603975564926566\\
3.06984903396344	0.579963382149292\\
3.06536108641892	0.554566853084517\\
3.05878009508824	0.527491807468921\\
3.04971390971152	0.498387035823426\\
3.0376359782482	0.466828017325126\\
3.0218383119463	0.432295740259596\\
3.00136387252382	0.394149060626916\\
2.97490864014783	0.351588725528002\\
2.94067879239094	0.303611091883521\\
2.89618159570869	0.248950233714589\\
2.83791994276145	0.186009859425842\\
2.76095254672642	0.112794272297332\\
2.65828483846744	0.0268670495822667\\
2.52010059775551	-0.0745900511703682\\
2.33301050835554	-0.194457881804245\\
2.07994111740238	-0.334801432192292\\
1.74218505871029	-0.495034299008016\\
1.30608372407202	-0.669071004772725\\
0.775416924238896	-0.842931253667855\\
0.183026458477091	-0.996488532947787\\
-0.413353316298409	-1.11162359925376\\
-0.955704250961396	-1.18127373485712\\
-1.41002817879594	-1.21066319585894\\
-1.77044404461756	-1.21114015108714\\
-2.04826567042507	-1.19387320821478\\
-2.26032756649037	-1.16710896417554\\
-2.422531658596	-1.13607078891378\\
-2.54768386419904	-1.10376561082803\\
-2.64540258288297	-1.07179752334559\\
-2.72270144076245	-1.04094758760794\\
-2.78464714192121	-1.01153548207928\\
-2.83490465866034	-0.983631313694125\\
-2.8761431824996	-0.957175611324294\\
-2.91032521757151	-0.932045857202917\\
-2.93890886945483	-0.908092622076673\\
-2.96298879862279	-0.88515861147164\\
-2.98339470092799	-0.86308816155869\\
-3.00076053537135	-0.841731422246322\\
-3.01557354829834	-0.82094559995982\\
-3.02820923357816	-0.800594580985712\\
-3.03895638740859	-0.780547661154982\\
-3.04803507758448	-0.76067776795101\\
-3.05560943928836	-0.740859364190165\\
-3.06179658640004	-0.720966104807109\\
-3.06667248976738	-0.70086824380515\\
-3.07027535439509	-0.68042973592618\\
-3.07260677896861	-0.659504934019606\\
-3.07363076847567	-0.637934739640746\\
-3.0732704647743	-0.61554201418963\\
-3.07140223244709	-0.592125994445668\\
-3.06784645563073	-0.567455372607121\\
-3.06235402300107	-0.541259588542124\\
-3.05458694179213	-0.51321773089914\\
-3.04409073683727	-0.48294424294041\\
-3.03025511957711	-0.449970369159299\\
-3.01225764702602	-0.413719960902326\\
-2.98898242622094	-0.373477916714031\\
-2.95890193537095	-0.328349285909294\\
-2.91990424359213	-0.277207246542429\\
-2.86904005845689	-0.218629624450325\\
-2.80215512061681	-0.150828347646298\\
-2.71336889273852	-0.0715887173907743\\
-2.59437845303026	0.0217350657353069\\
-2.43365766506504	0.132058542911812\\
-2.21590322689587	0.262015160923235\\
-1.92274195279636	0.412645902646693\\
-1.53677203889217	0.581005147098066\\
-1.05126636470144	0.75720443482184\\
-0.483696118163764	0.923554599581091\\
0.118643869731508	1.05962467768482\\
0.694129470594868	1.1520422511544\\
1.19480147923485	1.20034609016361\\
1.60151165444191	1.21376353859017\\
};
\addplot [color=mycolor1, forget plot]
  table[row sep=crcr]{%
1.7041846009319	1.2943141820301\\
1.99626617055785	1.28543761178804\\
2.22124175614415	1.26426293819397\\
2.39453017416156	1.23676782393869\\
2.52896408637316	1.20657960123409\\
2.63438716042362	1.175756675815\\
2.71808169542672	1.14539515887809\\
2.78536305329583	1.11602783814595\\
2.84010543395788	1.08786756517456\\
2.88514673507583	1.06095024777421\\
2.9225829191502	1.03521712036419\\
2.95397691776185	1.01056136374741\\
2.98050567802373	0.986854067154508\\
3.00306368390844	0.963958280238492\\
3.02233615866452	0.941736206473757\\
3.03885115178787	0.920052443127552\\
3.05301683469795	0.898774933595854\\
3.06514832973588	0.87777458064277\\
3.07548702698937	0.85692405068559\\
3.08421440519897	0.836096051577946\\
3.09146172447689	0.81516121575212\\
3.0973165011175	0.793985624376277\\
3.1018263403014	0.772427941024521\\
3.10500044296881	0.750336069553689\\
3.10680888273521	0.727543199906796\\
3.10717953780926	0.703863049386844\\
3.10599233300521	0.679084038331082\\
3.10307016596914	0.652962050479293\\
3.09816551812357	0.625211310957289\\
3.09094122611171	0.595492758529565\\
3.08094312764365	0.563399082666323\\
3.0675611678925	0.528435331338291\\
3.04997386749585	0.489993674758581\\
3.0270685348402	0.447320569827702\\
2.99732588623989	0.399474334660572\\
2.95865242299797	0.34527135316296\\
2.90813688499269	0.283220630985283\\
2.84169952147278	0.211451189241431\\
2.75360015287075	0.127649168935877\\
2.63579015752125	0.0290504103860355\\
2.47718044409962	-0.0874048975346086\\
2.26315440049889	-0.224539491410399\\
1.97624101171958	-0.383676370857818\\
1.59977690800976	-0.562325166838529\\
1.12661068121587	-0.751246369018012\\
0.571426793541186	-0.933269951884109\\
-0.0231945330767768	-1.08754580197928\\
-0.599163055661934	-1.19885002630157\\
-1.10821263274332	-1.26428398020831\\
-1.5280243443977	-1.29146157859367\\
-1.85963020768602	-1.29189976322708\\
-2.11613328811536	-1.27594927750475\\
-2.31347293789909	-1.25103332489436\\
-2.46592186792284	-1.22185313681699\\
-2.58479381981758	-1.19116200190927\\
-2.67857891676389	-1.16047520182953\\
-2.75350344627299	-1.13056847341001\\
-2.81410388947056	-1.10179161552201\\
-2.86369313535435	-1.07425566637257\\
-2.90470726510581	-1.04794158899572\\
-2.93895397442576	-1.02276231742486\\
-2.96778790699597	-0.99859763792049\\
-2.99223398977876	-0.975313379666422\\
-3.01307442280304	-0.952771567434943\\
-3.03091037717764	-0.930835368339748\\
-3.0462060383433	-0.909371033620749\\
-3.05932022560958	-0.888248094174011\\
-3.07052916260407	-0.867338521470358\\
-3.08004284009438	-0.84651524436022\\
-3.08801663398478	-0.82565021998284\\
-3.09455929844392	-0.804612137619662\\
-3.09973806520067	-0.783263755080808\\
-3.10358128812587	-0.761458808292563\\
-3.10607883621961	-0.73903838347614\\
-3.10718022598436	-0.715826588655281\\
-3.10679026744153	-0.691625299739336\\
-3.10476174708494	-0.666207678808018\\
-3.10088434981452	-0.639310060399069\\
-3.09486858146107	-0.610621666132709\\
-3.08632282322089	-0.579771428161253\\
-3.07472072399117	-0.546310967265258\\
-3.0593547589155	-0.509692477113805\\
-3.03926972115841	-0.469239927412545\\
-3.01316684589102	-0.42411168591123\\
-2.97926479270935	-0.373252582288746\\
-2.9350975127546	-0.315334137109362\\
-2.87722141885989	-0.248684455917916\\
-2.80079795942275	-0.171217097031371\\
-2.69902250417858	-0.0803873652608693\\
-2.56241494987765	0.026752980820651\\
-2.37814204960349	0.153247837121803\\
-2.1299429759065	0.30138943179547\\
-1.80001296258586	0.470949407623367\\
-1.37501405570806	0.656404734221779\\
-0.857224232733782	0.844434285581143\\
-0.275494477662446	1.01507514468077\\
0.317042707769568	1.14906003328664\\
0.864103828984793	1.23699824488219\\
1.32963119236916	1.28194972632408\\
1.7041846009319	1.2943141820301\\
};
\addplot [color=mycolor1, forget plot]
  table[row sep=crcr]{%
1.79451333687795	1.37623258172961\\
2.06437894430077	1.36802412314627\\
2.27364278731825	1.34831904966155\\
2.43629400799905	1.32250318949409\\
2.56373188637826	1.29387890390197\\
2.66466706633369	1.26436245928459\\
2.74556856172955	1.2350096720366\\
2.81119281358986	1.20636206852063\\
2.86503696965654	1.17866094921984\\
2.90968598827017	1.15197566340107\\
2.94706591443675	1.12627917729905\\
2.97862492315894	1.10149207155819\\
3.00546185166876	1.07750782499653\\
3.02841748975463	1.05420703582886\\
3.04813968400119	1.03146509239063\\
3.06513002218468	1.00915594522162\\
3.07977748081739	0.987153535081539\\
3.09238274743192	0.965331782097081\\
3.10317577097379	0.943563653214854\\
3.11232829045301	0.921719589106468\\
3.1199625284934	0.89966542345103\\
3.12615683127849	0.877259828969968\\
3.13094873227639	0.854351252443963\\
3.13433567233171	0.830774240267158\\
3.13627339068008	0.806344996398444\\
3.13667178150951	0.780855947410237\\
3.13538775939515	0.754069006761072\\
3.13221435902192	0.725707123922968\\
3.12686486295436	0.695443563909494\\
3.11895013854585	0.662888178353199\\
3.10794647251382	0.627569690518339\\
3.09314987331339	0.588912719714877\\
3.07361085634227	0.546207932699933\\
3.04804084787039	0.498573404835621\\
3.01467720302991	0.444905216476886\\
2.97108820312788	0.38381604428972\\
2.91389271499949	0.313563319504904\\
2.83836419754006	0.231976297684306\\
2.73789479389656	0.136410169507705\\
2.60333854470715	0.0237965139468894\\
2.42239579536337	-0.109060546546738\\
2.17956106428595	-0.264668607987185\\
1.85784309023286	-0.443144421608707\\
1.4441719282026	-0.639516259375913\\
0.939457202380576	-0.841139668930312\\
0.368918685510627	-1.0283302451733\\
-0.218665528781946	-1.18090618141992\\
-0.768973098121028	-1.28733879088908\\
-1.24436510686666	-1.34848886147076\\
-1.63213016362057	-1.3736032385062\\
-1.93796849459237	-1.3740036794835\\
-2.1756594985174	-1.35921433796541\\
-2.36001623058481	-1.33592856434747\\
-2.50381134739832	-1.30839697445668\\
-2.61706185013918	-1.27915087612558\\
-2.70729043857883	-1.24962273688704\\
-2.78004683255238	-1.2205774311595\\
-2.83940783860141	-1.19238591443792\\
-2.88837747174335	-1.16519138917053\\
-2.92918459732128	-1.13900792106641\\
-2.96349738916597	-1.11377818139752\\
-2.99257591894407	-1.08940686552014\\
-3.01738044126433	-1.06577972293003\\
-3.03864843761554	-1.04277407823606\\
-3.05694970805371	-1.02026430306792\\
-3.07272598295598	-0.998124270223542\\
-3.08631952632432	-0.976227977914113\\
-3.09799380993182	-0.954449030816098\\
-3.1079483738202	-0.932659362888273\\
-3.11632931733683	-0.910727400847761\\
-3.12323638902248	-0.888515747455641\\
-3.12872729505156	-0.865878380687456\\
-3.13281957597797	-0.84265729997381\\
-3.13549017384011	-0.818678491650597\\
-3.1366725962201	-0.793747023326093\\
-3.13625135215725	-0.767641003014309\\
-3.13405305519659	-0.740104045476918\\
-3.12983322099334	-0.71083576621519\\
-3.1232572748037	-0.679479662680571\\
-3.11387354646997	-0.645607531759416\\
-3.10107494690942	-0.608699304418124\\
-3.08404441443837	-0.568116856511122\\
-3.06167684265302	-0.523070018181276\\
-3.03246673353144	-0.472572782098417\\
-2.9943459418341	-0.415387948339402\\
-2.94444958092933	-0.349959983556741\\
-2.87878171397394	-0.274340653051243\\
-2.79175109683607	-0.18612422426896\\
-2.67556694180055	-0.0824371959007606\\
-2.51956678073377	0.0399140019352076\\
-2.30978231456736	0.183929908341486\\
-2.02956844260352	0.351203194158337\\
-1.66291465278432	0.539685603279386\\
-1.20226587375373	0.740783582827908\\
-0.659782514063547	0.937902771892843\\
-0.0736961714596904	1.10995578588396\\
0.501412124460094	1.24010882029349\\
1.01738293052278	1.32311336938\\
1.44914772331576	1.36482948316528\\
1.79451333687795	1.37623258172961\\
};
\addplot [color=mycolor1, forget plot]
  table[row sep=crcr]{%
1.875380198175	1.45912264622704\\
2.12532607372983	1.45151275523711\\
2.32049313714941	1.43312663690383\\
2.47351961229483	1.40883077969227\\
2.59454470957599	1.38164056702845\\
2.69129962768452	1.35334146430337\\
2.76954939823841	1.32494670095129\\
2.8335622883477	1.29699922662878\\
2.88650196898471	1.26976072903191\\
2.93072649099663	1.2433268912786\\
2.96800716044344	1.21769670697208\\
2.99968601153698	1.19281378547366\\
3.02678847325747	1.16859071790335\\
3.05010403242527	1.14492320634226\\
3.07024420039637	1.12169798193793\\
3.08768436550023	1.09879692266456\\
3.10279412762091	1.07609881054382\\
3.11585930640243	1.05347958179882\\
3.12709783060887	1.0308115652349\\
3.13667102518602	1.00796198065811\\
3.14469132014911	0.984790824374708\\
3.15122704385755	0.961148168735424\\
3.15630468366305	0.936870825940929\\
3.15990876251869	0.91177825868291\\
3.16197926064529	0.885667551171414\\
3.1624062783217	0.858307174811905\\
3.16102135810059	0.829429184553865\\
3.15758452313218	0.798719355335712\\
3.15176558918009	0.765804602668832\\
3.14311759426987	0.73023681658467\\
3.13103914923143	0.691471966427712\\
3.11472098212371	0.648843010440361\\
3.09306970442142	0.601524809885759\\
3.06459858593356	0.548489034635153\\
3.02727063241863	0.488447303179922\\
2.97827358720255	0.419782361339985\\
2.91370094648205	0.340471870835639\\
2.82811273852315	0.248021431188159\\
2.71396950645002	0.139450861872087\\
2.56100997894323	0.011434072810402\\
2.35585577163837	-0.139207939611686\\
2.08258910796728	-0.314338081602671\\
1.72574780809793	-0.512344302816052\\
1.27738430552161	-0.725264891854567\\
0.747410964471658	-0.937093700635873\\
0.170151684734942	-1.12661859938555\\
-0.403149960273714	-1.27559530863182\\
-0.924739677714599	-1.37654191730408\\
-1.36723218025528	-1.43348961008313\\
-1.72543970187286	-1.45669521773722\\
-2.00807292192314	-1.45706014702998\\
-2.22891235056166	-1.44331104409805\\
-2.40157978026183	-1.42149356437748\\
-2.53749727933407	-1.39546326605403\\
-2.64555530926629	-1.36755240066736\\
-2.7324408719078	-1.33911374280202\\
-2.80311577466288	-1.31089573584426\\
-2.86125293005364	-1.28328246873827\\
-2.90958126713142	-1.25644161240548\\
-2.95014237282275	-1.2304139170949\\
-2.98447637900907	-1.20516677800092\\
-3.01375518981585	-1.18062600865979\\
-3.03887778105092	-1.15669444918053\\
-3.0605385408469	-1.13326260668455\\
-3.07927649620449	-1.11021444400938\\
-3.09551093137147	-1.08743018066274\\
-3.10956722998844	-1.06478721623195\\
-3.12169559672649	-1.04215982907142\\
-3.13208449045607	-1.0194180211406\\
-3.14087001922426	-0.9964257010892\\
-3.14814212743523	-0.97303827856067\\
-3.15394809027255	-0.949099656396012\\
-3.15829357749059	-0.924438536709023\\
-3.16114132532165	-0.898863889641476\\
-3.16240723260219	-0.872159360523098\\
-3.16195344608523	-0.844076303511913\\
-3.15957768583335	-0.814325018612185\\
-3.15499763915335	-0.782563624334002\\
-3.14782865669049	-0.748383809417997\\
-3.13755212396971	-0.711292464246528\\
-3.12347062022884	-0.670687892829954\\
-3.10464412095523	-0.625828968409226\\
-3.07979879451023	-0.575795296373511\\
-3.04719609928311	-0.519436404776792\\
-3.00444475117117	-0.455308738835184\\
-2.9482322176708	-0.381602060475576\\
-2.87394842006821	-0.296064553876908\\
-2.77518113622301	-0.195954326761797\\
-2.64310416802847	-0.0780848431611926\\
-2.46591133519486	0.0608914956915911\\
-2.22877348479082	0.223698787331001\\
-1.91540232194173	0.410797198251262\\
-1.5129282868656	0.617755172649118\\
-1.02102501419702	0.832596173905048\\
-0.461678914241442	1.03596879881983\\
0.120322339013655	1.20694603259918\\
0.672695012185963	1.33204369199023\\
1.15664250869975	1.40994398642384\\
1.5565348565067	1.44859625100981\\
1.875380198175	1.45912264622704\\
};
\addplot [color=mycolor1, forget plot]
  table[row sep=crcr]{%
1.94860967361305	1.54255432833086\\
2.1805183787397	1.53548649136023\\
2.36285721209787	1.5183012557299\\
2.50702155221514	1.49540564898627\\
2.62204553358807	1.46955805053239\\
2.71480973562695	1.44242159207405\\
2.79046529403399	1.41496449889465\\
2.85285038384193	1.387724683507\\
2.90483097030853	1.36097714777595\\
2.94855954345884	1.33483762364216\\
2.98566504672178	1.30932602862144\\
3.01739029515218	1.28440503990512\\
3.04469096502283	1.26000335267682\\
3.06830696683483	1.23602950040386\\
3.08881408202697	1.21237982371928\\
3.10666146617262	1.18894277034559\\
3.12219895498825	1.16560084981681\\
3.1356969214283	1.14223103857391\\
3.14736059175103	1.1187041016261\\
3.15734013078588	1.09488308635246\\
3.16573737353531	1.07062110330677\\
3.17260975468686	1.04575840770177\\
3.17797172733908	1.02011871395454\\
3.18179373411327	0.993504600659352\\
3.18399856897966	0.965691784078038\\
3.18445471760671	0.936421945322298\\
3.18296595418294	0.905393680381121\\
3.17925605963709	0.872250992770669\\
3.17294694888811	0.836568554960588\\
3.16352766531549	0.79783271720215\\
3.15031049243682	0.755416938424182\\
3.13236866825522	0.70854997319778\\
3.10844763558118	0.656274849324365\\
3.07683817380715	0.59739663051999\\
3.03519503220057	0.530417717636257\\
2.98027938081541	0.453462247419244\\
2.90760015624451	0.364198731229946\\
2.81093650936482	0.259788036163911\\
2.6817630125737	0.136922409161699\\
2.50872093805967	-0.00790498392072498\\
2.2775720519001	-0.177646504168353\\
1.97261341510108	-0.373116462736028\\
1.58109419230744	-0.590422057659901\\
1.10155200657516	-0.818241223317062\\
0.553103473504285	-1.03757477755222\\
-0.023057140602249	-1.22686053176458\\
-0.576640773712478	-1.37080621485596\\
-1.06801045513199	-1.46595664204834\\
-1.4789594019965	-1.51886494276565\\
-1.80996604361589	-1.54031035988743\\
-2.07155855448015	-1.54064245584021\\
-2.27711730768827	-1.52783705568473\\
-2.43909171547293	-1.50736341137314\\
-2.56770027554192	-1.48272667724163\\
-2.67085281140652	-1.45607779784481\\
-2.75450984985187	-1.42869176395499\\
-2.82311814696074	-1.4012955399815\\
-2.87999250812671	-1.37427930959536\\
-2.92761450788321	-1.34782843907536\\
-2.96785477081937	-1.32200465604213\\
-3.00213469242636	-1.29679556598416\\
-3.03154308672126	-1.27214464320716\\
-3.05692021372257	-1.24796920448344\\
-3.0789184572603	-1.22417095912209\\
-3.098046311607	-1.20064193399821\\
-3.11470037786735	-1.17726747430447\\
-3.1291886612983	-1.15392734812871\\
-3.14174746141081	-1.13049556681816\\
-3.15255343902769	-1.10683927077373\\
-3.16173193739188	-1.0828168593773\\
-3.16936226115321	-1.05827542579354\\
-3.17548032873782	-1.03304746813961\\
-3.18007887317197	-1.00694677181647\\
-3.18310514345613	-0.979763281858598\\
-3.18445582470331	-0.951256699217872\\
-3.1839686197656	-0.921148431652043\\
-3.18140957999802	-0.889111398528125\\
-3.17645478670739	-0.854757018710966\\
-3.16866429456784	-0.817618491103143\\
-3.15744524822052	-0.777129201102349\\
-3.1419996226369	-0.732594759038347\\
-3.12124991115256	-0.683156840821503\\
-3.09373304327001	-0.627746789871623\\
-3.05744864793884	-0.56502719964194\\
-3.00964261337217	-0.493321254018828\\
-2.94650206075707	-0.410534317308221\\
-2.86273816213399	-0.314084086158319\\
-2.75105243695848	-0.200882268889368\\
-2.60155406129097	-0.0674650913852828\\
-2.40138990312832	0.0895347979608175\\
-2.13526444165352	0.272262905902826\\
-1.78815012598701	0.479549466436689\\
-1.35169578093293	0.704055024761356\\
-0.833802555613541	0.930355778532272\\
-0.26531739322843	1.13717633843827\\
0.305598348497514	1.30500708020502\\
0.831763869877206	1.42424215832089\\
1.28384927376364	1.49704828505071\\
1.65393178900214	1.5328286466653\\
1.94860967361305	1.54255432833086\\
};
\addplot [color=mycolor1, forget plot]
  table[row sep=crcr]{%
2.01533812164163	1.62604405276322\\
2.23078249240264	1.61947135062162\\
2.40132090076523	1.60339148684674\\
2.53722108374555	1.58180236172912\\
2.64654784552852	1.55723001253831\\
2.73543985727512	1.53122219873246\\
2.80850990487305	1.50470009413296\\
2.86921499079895	1.47819106428946\\
2.92015375315593	1.45197729022359\\
2.96329141677968	1.42618900054377\\
3.00012525770858	1.40086244377317\\
3.03180488741777	1.37597573400241\\
3.05921940824373	1.35147086268574\\
3.08306063861848	1.32726703647262\\
3.10386911975224	1.30326853228884\\
3.12206769401675	1.27936903780104\\
3.1379860358974	1.25545368599787\\
3.15187850511821	1.2313995175315\\
3.16393696944925	1.20707480229813\\
3.17429972452134	1.18233745375814\\
3.18305725410281	1.15703263297448\\
3.19025527804017	1.13098953708787\\
3.1958952892501	1.1040172808161\\
3.19993255524039	1.07589969643826\\
3.20227132538617	1.04638878691185\\
3.20275671237797	1.01519645831191\\
3.20116236797684	0.981984021383658\\
3.19717260000278	0.946348776642342\\
3.19035690968367	0.907806772208036\\
3.18013396639077	0.865770541065649\\
3.1657206394711	0.819520290456693\\
3.14605968485133	0.768166672827977\\
3.11971681260298	0.710603047605297\\
3.08473396396719	0.645445388655745\\
3.03842085756421	0.570959515297282\\
2.9770625239435	0.484979939102609\\
2.89552117857411	0.384836153263169\\
2.78672909121182	0.267328088123135\\
2.64113583717848	0.128845047388666\\
2.44635087067706	-0.0341853588117058\\
2.18759873351322	-0.224214143276535\\
1.85017027932879	-0.440534454589518\\
1.42527425881919	-0.676434248589246\\
0.918975998779657	-0.917067119843662\\
0.359018686232294	-1.14112455367898\\
-0.209233416725197	-1.32792310389627\\
-0.739226389230772	-1.46581173965135\\
-1.19996215139496	-1.55507001437202\\
-1.58102344345021	-1.60414390620066\\
-1.88700244748479	-1.62396754081746\\
-2.12940113609882	-1.62426956441044\\
-2.32097043321478	-1.61232871930589\\
-2.4730471142931	-1.59309969429799\\
-2.59478196509427	-1.56977421460537\\
-2.69322879116622	-1.54433650080771\\
-2.77371333840301	-1.51798531534492\\
-2.84022849832474	-1.49142186722937\\
-2.89576968323859	-1.46503638282487\\
-2.94259477266895	-1.43902599538774\\
-2.98241747368191	-1.41346832118405\\
-3.01654851341223	-1.38836707616566\\
-3.04599803371542	-1.36368019711536\\
-3.07154980001532	-1.33933701616698\\
-3.09381511158766	-1.31524854837838\\
-3.11327209412083	-1.29131339930806\\
-3.13029440294031	-1.26742083623812\\
-3.14517216961788	-1.24345196690316\\
-3.15812717037828	-1.2192795917351\\
-3.16932358304173	-1.19476705222343\\
-3.17887525422785	-1.16976623492159\\
-3.18685006373035	-1.14411477395435\\
-3.19327170606398	-1.11763240256517\\
-3.1981189768976	-1.090116321054\\
-3.20132242555929	-1.06133536274487\\
-3.2027579853432	-1.03102264123454\\
-3.20223688804579	-0.998866241078235\\
-3.19949076565381	-0.964497359602382\\
-3.19415028193235	-0.927475108504999\\
-3.1857148364916	-0.887266930464672\\
-3.1735097247936	-0.843223275531408\\
-3.15662545454323	-0.794544834323392\\
-3.1338315005953	-0.740240317604256\\
-3.10345340779718	-0.679072720821543\\
-3.06319775596855	-0.6094927493767\\
-3.00990465381885	-0.529560827287165\\
-2.93920465871892	-0.436866525052448\\
-2.84506407514115	-0.328471731263217\\
-2.71923958708128	-0.200941316919633\\
-2.55077516534008	-0.0505963821734142\\
-2.32594077898868	0.125764016628224\\
-2.02950284685265	0.329331705284922\\
-1.64873641905498	0.556766844342745\\
-1.18109928201364	0.797398721387658\\
-0.643132582441513	1.03258585633054\\
-0.0728119622014619	1.24019417476993\\
0.481442381812748	1.40322281470281\\
0.979351688292782	1.51611243646592\\
1.40041618766564	1.5839478109451\\
1.74275843194902	1.61705145338868\\
2.01533812164163	1.62604405276322\\
};
\addplot [color=mycolor1, forget plot]
  table[row sep=crcr]{%
2.07625084362876	1.70905454066351\\
2.27657171102993	1.7029371442451\\
2.43616868402318	1.68788290929235\\
2.56429398464666	1.66752361618015\\
2.66816102380818	1.64417396855024\\
2.7532589566817	1.61927255932821\\
2.82372647505532	1.59369206321064\\
2.88268142697289	1.56794476004416\\
2.93248158859062	1.54231477478672\\
2.97492112869881	1.51694197169365\\
3.01137531548528	1.49187481670289\\
3.04290611533935	1.46710353913919\\
3.07033908435744	1.44258080620787\\
3.09431943631997	1.41823444553482\\
3.1153530375915	1.39397505235657\\
3.13383644291012	1.36970025017026\\
3.15007888440732	1.34529670157174\\
3.1643182593009	1.3206405388398\\
3.17673253706303	1.29559660685734\\
3.18744755077955	1.27001672492065\\
3.19654179403765	1.24373704136143\\
3.20404857173142	1.21657445123097\\
3.20995561726957	1.18832195568965\\
3.2142020609549	1.15874274941089\\
3.21667238637311	1.12756271826917\\
3.21718671125588	1.09446090317859\\
3.21548633545506	1.05905732599157\\
3.21121295525575	1.02089736817394\\
3.2038791708604	0.979431632163099\\
3.19282679991494	0.933989895601004\\
3.17716789949916	0.88374740859986\\
3.15570109554437	0.827681459205336\\
3.12679262240682	0.764516053245468\\
3.08820733464323	0.692653240751548\\
3.03687042097796	0.610092261916117\\
2.96853800615594	0.51434486165235\\
2.8773614535736	0.402372075235921\\
2.75536443681216	0.27060407493264\\
2.59195470341043	0.115173991350096\\
2.37383613389199	-0.0673944967456265\\
2.08613598372201	-0.278706153942123\\
1.71605520807854	-0.516007374804534\\
1.25998293614149	-0.769296722337668\\
0.732095567027905	-1.02030053062889\\
0.167467140251177	-1.24634563118402\\
-0.387183572051126	-1.42877179658333\\
-0.891025359473241	-1.55991959660899\\
-1.32146597833854	-1.64333876514649\\
-1.6744246737234	-1.68880234054156\\
-1.95736105292541	-1.70713148781447\\
-2.18216337866193	-1.70740611297996\\
-2.36083222890854	-1.69626317837733\\
-2.50366964198596	-1.67819669876535\\
-2.618880768512	-1.6561163679259\\
-2.71276999592862	-1.63185226658598\\
-2.79010612252365	-1.60652857080875\\
-2.8544804346866	-1.58081732677704\\
-2.90860242718634	-1.55510371807489\\
-2.9545270643359	-1.52959151788961\\
-2.99382369479805	-1.50436972387311\\
-3.02769974333328	-1.47945445076631\\
-3.05709081854048	-1.45481514211133\\
-3.08272635217036	-1.43039082628619\\
-3.10517752533419	-1.40610000507192\\
-3.12489235374168	-1.38184641564834\\
-3.14222139692114	-1.35752206036053\\
-3.15743653465407	-1.33300836373815\\
-3.17074451852888	-1.30817597341254\\
-3.18229647416781	-1.28288349586974\\
-3.19219413596961	-1.25697530235906\\
-3.20049329266957	-1.23027842460988\\
-3.20720467138463	-1.20259846415877\\
-3.21229225948476	-1.17371434856197\\
-3.21566882949251	-1.14337167090636\\
-3.21718816250861	-1.11127423516403\\
-3.21663312446418	-1.07707328820045\\
-3.213698289137	-1.0403537381232\\
-3.2079651557215	-1.00061642671769\\
-3.19886708211181	-0.957255233004433\\
-3.18563971506946	-0.909527440378471\\
-3.16725076793578	-0.856515441995987\\
-3.1423002681693	-0.797077617113549\\
-3.1088787136591	-0.729786419488706\\
-3.06436609974594	-0.652853169202736\\
-3.00515074354965	-0.564043495916574\\
-2.92624747803168	-0.460598558864161\\
-2.82081182873313	-0.339202275322706\\
-2.67960829495215	-0.196085812311738\\
-2.49065346318377	-0.027450510797915\\
-2.23959860846959	0.169492024219199\\
-1.91193949263403	0.394534069462105\\
-1.49838501044073	0.641617587095279\\
-1.00327221339828	0.896484499037834\\
-0.451531360927953	1.13781127815106\\
0.113997515571919	1.34378714168818\\
0.647354154723447	1.50075046197496\\
1.11604264788142	1.60706016135398\\
1.50733999281103	1.67011773368875\\
1.82394103035216	1.70073476993218\\
2.07625084362876	1.70905454066351\\
};
\addplot [color=mycolor1, forget plot]
  table[row sep=crcr]{%
2.1317414423095	1.79100753091255\\
2.31810513092557	1.78531088071093\\
2.46750057327107	1.77121361329816\\
2.58826913120071	1.75201867108555\\
2.6868750793552	1.7298478205699\\
2.76823712944396	1.70603634702307\\
2.83607514814713	1.68140766269239\\
2.89320444625466	1.65645536988618\\
2.94176541488176	1.63146116440283\\
2.9833958590435	1.60657035406556\\
3.01935802552462	1.58183998349703\\
3.05063158093467	1.55726939291785\\
3.07798158660956	1.5328195071364\\
3.102008275447	1.50842484691612\\
3.12318358668422	1.48400078265371\\
3.14187800771922	1.45944761539587\\
3.15838023912751	1.43465247446497\\
3.17291145028976	1.40948963655129\\
3.1856353486624	1.3838196170224\\
3.19666488245242	1.35748720900995\\
3.20606608599172	1.33031851643538\\
3.21385932206391	1.30211692135325\\
3.22001794513899	1.27265782792174\\
3.22446417591456	1.24168192229003\\
3.22706171143083	1.20888656827981\\
3.22760426130214	1.17391481141768\\
3.22579875322447	1.1363412764689\\
3.22124132591228	1.09565400392063\\
3.21338333449859	1.0512309698093\\
3.20148330571805	1.00230967401717\\
3.18453892980651	0.947947802132128\\
3.16119056676129	0.886972688037045\\
3.12958422734702	0.817917450365342\\
3.08717771115147	0.738943010388461\\
3.03046968605458	0.647749429602825\\
2.9546318816494	0.541490732974257\\
2.85303999781711	0.416731621646288\\
2.71675510959344	0.269534018621001\\
2.53415623186526	0.0958496398074653\\
2.29124129015289	-0.107486009875524\\
1.97360510470139	-0.340815789433208\\
1.57137861878927	-0.598787897134276\\
1.08721218216131	-0.867771357129065\\
0.543419735910371	-1.1264524054491\\
-0.0194546428704111	-1.35190795116236\\
-0.555956970215128	-1.52845094451582\\
-1.03216247322138	-1.65245572918629\\
-1.4331592886672	-1.73019094646337\\
-1.75983247900571	-1.77227422021084\\
-2.02153685237199	-1.78922555794723\\
-2.23014518586385	-1.7894752978946\\
-2.39685537138576	-1.77907268401652\\
-2.53101872335505	-1.76209832106834\\
-2.6400030291327	-1.74120709171542\\
-2.72945476977381	-1.71808621165136\\
-2.80365248373414	-1.6937872099948\\
-2.86583106514623	-1.66895044366125\\
-2.91844350981957	-1.6439518970657\\
-2.96336049115918	-1.61899762638342\\
-3.00201852468984	-1.59418407199124\\
-3.03552869182068	-1.56953642646061\\
-3.06475613262073	-1.54503293912038\\
-3.0903781973561	-1.52062017474582\\
-3.11292708052982	-1.4962223993832\\
-3.13282113780473	-1.4717470929754\\
-3.150387877346	-1.44708784287421\\
-3.16588073720748	-1.42212539481393\\
-3.17949112278348	-1.39672732661129\\
-3.19135671140971	-1.3707466002794\\
-3.20156667958285	-1.34401909927058\\
-3.21016422967143	-1.31636014226515\\
-3.21714655389592	-1.28755986472617\\
-3.22246214491573	-1.25737726036396\\
-3.22600511642653	-1.22553256485005\\
-3.2276059021702	-1.1916975322311\\
-3.22701731790171	-1.15548298868904\\
-3.22389444409388	-1.11642283618932\\
-3.21776604144455	-1.07395340886645\\
-3.20799413922056	-1.02738675304841\\
-3.19371689091799	-0.975876023855393\\
-3.17376758645825	-0.918370838279727\\
-3.14655965614383	-0.853560298705166\\
-3.10992354808926	-0.7798020024311\\
-3.06087701812257	-0.69503783253758\\
-2.9953078345687	-0.596704182786383\\
-2.9075537768713	-0.481660592789777\\
-2.78989601647332	-0.346195893384541\\
-2.63207603324366	-0.186238557768118\\
-2.42116974669818	0.00199399013596253\\
-2.14256889254755	0.220565435545492\\
-1.78329827876146	0.467362244745593\\
-1.33864880940448	0.733098980637337\\
-0.820555955097994	0.999900587110749\\
-0.261428776333567	1.24457532845698\\
0.293542875111646	1.44680762682395\\
0.802975958022219	1.59679851197821\\
1.24229289017552	1.69648104919221\\
1.60531664101736	1.75499481314277\\
1.89807274077814	1.78330660078127\\
2.1317414423095	1.79100753091255\\
};
\addplot [color=mycolor1, forget plot]
  table[row sep=crcr]{%
2.18202401463123	1.87130548893601\\
2.35546460910354	1.8659989740022\\
2.49531402264379	1.85279783305457\\
2.60909837557366	1.83470888706013\\
2.70262154246927	1.81367741452399\\
2.78029989221005	1.79094111660265\\
2.84548241910839	1.76727406506573\\
2.90071413752772	1.74314848567836\\
2.94793913106009	1.71884008947843\\
2.98865234844797	1.69449608538423\\
3.02401151948419	1.67017893583495\\
3.05491925982564	1.64589442281208\\
3.08208328967226	1.62160953907504\\
3.10606067976851	1.59726372493208\\
3.12729042043629	1.57277568874334\\
3.14611738903238	1.54804722797296\\
3.16280989598737	1.52296493819135\\
3.17757233784711	1.49740035074913\\
3.19055400743643	1.47120880606978\\
3.20185475129721	1.4442272037834\\
3.21152788042525	1.41627064377201\\
3.21958049832703	1.38712786329406\\
3.2259711816471	1.35655526949354\\
3.23060470516282	1.32426925100999\\
3.23332321384119	1.28993631496692\\
3.23389287124117	1.25316042374075\\
3.23198450364748	1.21346668630825\\
3.22714603978494	1.17028027905468\\
3.21876351368187	1.12289912394304\\
3.20600591176761	1.07045845048474\\
3.18774702488831	1.01188497623521\\
3.16245452589533	0.945838247822802\\
3.12803267039657	0.870637166155992\\
3.08160075712392	0.784171977175059\\
3.01918676590626	0.683808451374954\\
2.93532042725459	0.566306538405956\\
2.82253783362597	0.427809633335044\\
2.67089509680695	0.26402657173068\\
2.46779379775515	0.0708365130902191\\
2.19881035353723	-0.1543370372527\\
1.85069554742905	-0.410094596109592\\
1.41757943731687	-0.687947573759316\\
0.909196433566452	-0.970485391769853\\
0.355432238463405	-1.23402580139253\\
-0.199907051267903	-1.45656752040886\\
-0.714824033161	-1.62608083233949\\
-1.16276950598761	-1.74276677810291\\
-1.53551456365519	-1.81504155347651\\
-1.83769667471329	-1.85397291447147\\
-2.07982491225251	-1.86965351824121\\
-2.27348859687057	-1.86988069903074\\
-2.42907384371724	-1.86016742457127\\
-2.55506502775769	-1.84422262288643\\
-2.6580880518937	-1.82447035710515\\
-2.74321038548067	-1.80246535066994\\
-2.81427788442654	-1.77918880243121\\
-2.87420854073542	-1.75524769527952\\
-2.92522499346317	-1.73100553369938\\
-2.96903053613899	-1.70666703771278\\
-3.00693959320492	-1.68233271304135\\
-3.03997360262699	-1.65803391122278\\
-3.06893132085974	-1.63375526384628\\
-3.09444042702253	-1.60944889993664\\
-3.11699547832575	-1.58504325462593\\
-3.13698585647355	-1.56044825124497\\
-3.15471629743739	-1.53555798040615\\
-3.17042183318646	-1.51025157238872\\
-3.18427841674602	-1.48439267580391\\
-3.19641008883668	-1.4578277602352\\
-3.20689322696851	-1.4303833171076\\
-3.21575815862114	-1.40186191715796\\
-3.22298818821358	-1.37203697719307\\
-3.22851585503685	-1.34064597961906\\
-3.23221597727604	-1.30738176323769\\
-3.2338947115149	-1.27188135077356\\
-3.23327342309961	-1.23371158453574\\
-3.2299655588147	-1.19235059325209\\
-3.22344385254215	-1.14716379976709\\
-3.21299395546891	-1.09737280136007\\
-3.1976488032231	-1.04201504444782\\
-3.17609552255039	-0.979891886962476\\
-3.14654328864766	-0.909502698571209\\
-3.10653638634683	-0.828963820567699\\
-3.05269282312502	-0.735915134149284\\
-2.98034873213667	-0.627427148135196\\
-2.88310229766976	-0.499944753247218\\
-2.75230159650874	-0.349351808342394\\
-2.57665665987365	-0.171327690835457\\
-2.34244890996455	0.0377108016772661\\
-2.03527926982411	0.278721927644738\\
-1.64461155087895	0.547139911134911\\
-1.17139168307723	0.830037161564239\\
-0.635415342870461	1.10615942595841\\
-0.0750890768055588	1.35147228458665\\
0.464530598780093	1.54820045262041\\
0.948070919357095	1.69062377979606\\
1.35846538811358	1.78377048989356\\
1.69483670196028	1.83799684704665\\
1.96553148603393	1.86417465145828\\
2.18202401463123	1.87130548893601\\
};
\addplot [color=mycolor1]
  table[row sep=crcr]{%
2.22721472973942	1.94935886896179\\
2.3886648425121	1.94441487511101\\
2.5195634483927	1.93205457572182\\
2.62670779849127	1.91501766628131\\
2.71531853364	1.89508786650117\\
2.78936864949369	1.87341097537522\\
2.85187814440773	1.85071228241933\\
2.90515032165589	1.82744075521499\\
2.9509520177711	1.80386332654994\\
2.99064783647547	1.78012618777659\\
3.02529909977035	1.7562945351756\\
3.0557365623221	1.73237827445755\\
3.08261387510984	1.7083485310185\\
3.10644696879361	1.68414807630787\\
3.12764310033082	1.65969765855822\\
3.14652223914943	1.63489950162782\\
3.16333268711427	1.60963876367151\\
3.17826225469757	1.5837834336332\\
3.19144589188063	1.55718292750329\\
3.20297034763997	1.52966548843736\\
3.2128761683912	1.50103436869378\\
3.2211571124149	1.47106265802835\\
3.22775682617807	1.43948650786522\\
3.23256237104755	1.40599637010381\\
3.23539387180292	1.3702257107414\\
3.23598913837185	1.33173645766061\\
3.23398152983158	1.29000018434938\\
3.22886850103885	1.24437370419161\\
3.21996707976175	1.19406735000614\\
3.20635080827997	1.13810376656035\\
3.18676025898309	1.07526465092859\\
3.15947593702972	1.0040228169715\\
3.12213828317303	0.922457920675385\\
3.07149545967596	0.828157721349833\\
3.00305879148097	0.718116213117772\\
2.91065699973037	0.588662002756959\\
2.785925244978	0.435496623074383\\
2.61788823849866	0.254008232649376\\
2.39306798253518	0.0401523940333445\\
2.09699730361163	-0.20771900183087\\
1.7183819635995	-0.485932899271695\\
1.25639683089422	-0.782386182967809\\
0.728319451607428	-1.07597695395358\\
0.17048677629427	-1.34156752261761\\
-0.372344302750219	-1.55919115860543\\
-0.863264272549092	-1.72086595365401\\
-1.28299919742863	-1.8302347441912\\
-1.62890152595063	-1.89731652670458\\
-1.90833376158261	-1.93331843320936\\
-2.13240654410545	-1.94782687646357\\
-2.3122537087605	-1.94803363323748\\
-2.45746747168844	-1.93896359662991\\
-2.57574556593285	-1.92399109460735\\
-2.67305603610964	-1.90533078490532\\
-2.75395564612754	-1.88441460443591\\
-2.82190758462804	-1.86215611801615\\
-2.87954766726586	-1.83912801979308\\
-2.92889157233296	-1.81567886095776\\
-2.97149068204595	-1.79200910352774\\
-3.00854741762708	-1.76822049003994\\
-3.04100007190525	-1.74434802267285\\
-3.0695851473103	-1.7203805908067\\
-3.09488323143276	-1.69627413213334\\
-3.11735281839452	-1.67195981527391\\
-3.13735524534064	-1.64734883071173\\
-3.15517299859553	-1.62233479306363\\
-3.17102297575804	-1.59679437416259\\
-3.18506579905015	-1.570586527561\\
-3.19741190629232	-1.54355048208521\\
-3.20812485602917	-1.51550254281508\\
-3.21722203851139	-1.48623162017347\\
-3.22467275544594	-1.45549329519237\\
-3.23039339092638	-1.42300210774249\\
-3.23423911335776	-1.38842161161915\\
-3.23599118590349	-1.35135156242766\\
-3.23533847055249	-1.31131137681683\\
-3.23185101774822	-1.2677187105326\\
-3.22494264006069	-1.21986163848029\\
-3.2138179373885	-1.16686249014231\\
-3.19739719631547	-1.10763095340298\\
-3.17420973909784	-1.0408037811618\\
-3.142242561071	-0.964668760862181\\
-3.09872682936675	-0.877072578179989\\
-3.0398417586358	-0.77531812311728\\
-2.96031892266926	-0.656071422287709\\
-2.85295425309884	-0.515330753928937\\
-2.70811161177175	-0.348574764250833\\
-2.51348980394932	-0.151314840167112\\
-2.25479111797749	0.0795960527623154\\
-1.91840618384291	0.343563906305597\\
-1.49721915275415	0.633014609293368\\
-0.998726638487032	0.931115827109969\\
-0.450328181715587	1.21375185121395\\
0.105465468340197	1.45718721810745\\
0.625945116714116	1.64701594620063\\
1.08251666630864	1.78154116777267\\
1.46486936946524	1.86834353319716\\
1.77626331377619	1.91854897142989\\
2.02656829443923	1.94275375393752\\
2.22721472973941	1.94935886896179\\
};
\addlegendentry{Аппроксимации}

\addplot [color=mycolor2]
  table[row sep=crcr]{%
2.12132034355964	2.82842712474619\\
2.18582575000323	2.82640660874952\\
2.2459761483502	2.82067996094826\\
2.30316600770785	2.81153894842004\\
2.35833530419819	2.79908319427389\\
2.4121407651283	2.78328504568152\\
2.46505313939319	2.76402339892432\\
2.51741350921551	2.74110157112695\\
2.56946568278775	2.71425682483196\\
2.62137377470791	2.68316571665979\\
2.67322998558975	2.64744786912504\\
2.72505542741334	2.60667010622203\\
2.77679571750944	2.56035271147659\\
2.82831256449113	2.50797963523731\\
2.87937250576692	2.44901464186941\\
2.92963424895255	2.38292549191834\\
2.97863667364875	2.30921809960124\\
3.02579039162557	2.22748193169334\\
3.0703766721051	2.13744642550811\\
3.11155818595542	2.03904567139316\\
3.14840590189045	1.93248505458784\\
3.1799449834646	1.81829951655162\\
3.20521925970046	1.69738980361013\\
3.2233689191928	1.57102231442397\\
3.23371058284854	1.44078163681543\\
3.23580480244767	1.30847305653245\\
3.22949541157146	1.1759834831\\
3.21490900216393	1.04511946165368\\
3.19240995273442	0.917445713323391\\
3.16251366135309	0.794144173493542\\
3.12576385298257	0.675902054710668\\
3.08257536776515	0.562820516214273\\
3.03302813706062	0.454314775018543\\
2.97656426645917	0.348948817128694\\
2.91146921429886	0.244099783305823\\
2.83385680006128	0.135243682019644\\
2.73548220960228	0.0144130439331602\\
2.59871334341581	-0.133168683665202\\
2.38479638365891	-0.336180712113444\\
2.01057175466146	-0.648371525543042\\
1.34210041814469	-1.13791067535239\\
0.375352321166521	-1.75822148716338\\
-0.527448468220416	-2.26262801632552\\
-1.11365847778129	-2.5437869612038\\
-1.4567327767317	-2.68228003389134\\
-1.6691296256022	-2.75259678373195\\
-1.81428800439327	-2.79054335642282\\
-1.92320462744895	-2.81168897715777\\
-2.01132392805712	-2.82302365363059\\
-2.08680248126988	-2.82787464184343\\
-2.15422943553058	-2.82790871121207\\
-2.21634478307291	-2.8239827357917\\
-2.27487422572476	-2.8165264847195\\
-2.33095693947289	-2.80572465930185\\
-2.3853755403757	-2.79160756168836\\
-2.43868469976231	-2.77409782070174\\
-2.49128507004909	-2.75303495626884\\
-2.54346612600745	-2.7281884019911\\
-2.59543032955127	-2.69926454755406\\
-2.64730535029083	-2.66591103001887\\
-2.69914810041563	-2.62772046423944\\
-2.75094277189846	-2.58423541950896\\
-2.80259428647749	-2.53495641420315\\
-2.85391830194335	-2.47935483422118\\
-2.90462903903618	-2.41689283661657\\
-2.95432664759112	-2.34705230341292\\
-3.00248656709011	-2.26937453210355\\
-3.04845424185856	-2.18351130413452\\
-3.0914493816545	-2.08928598065547\\
-3.13058429431929	-1.98676020448289\\
-3.16490008683964	-1.87629788656397\\
-3.19342217916266	-1.75861429449935\\
-3.21523240499265	-1.63479579561046\\
-3.22954956680171	-1.50627700312196\\
-3.23580523135335	-1.37476797309193\\
-3.23369896357047	-1.2421341494958\\
-3.22321880671614	-1.11024302689198\\
-3.20461860896113	-0.980799507713277\\
-3.17835148049538	-0.85519274756288\\
-3.14496439092874	-0.734369602871579\\
-3.10495870881183	-0.618735254776386\\
-3.05861192744495	-0.508062722468019\\
-3.0057323937692	-0.401369642978784\\
-2.94527025004781	-0.296685462718974\\
-2.87460204254734	-0.190562899149911\\
-2.78805571026974	-0.0770301658026245\\
-2.67361421522119	0.0546866392302208\\
-2.50520482475694	0.225105040795738\\
-2.22541867988909	0.474087715816021\\
-1.7209533613293	0.868461990547897\\
-0.880186083660502	1.44500069250654\\
0.110323684998138	2.03883159526995\\
0.859259206598599	2.42737283484743\\
1.30741523004191	2.62511520505649\\
1.57426953542748	2.72304263504847\\
1.74766615441039	2.77435113512036\\
1.87207484559155	2.80265686669549\\
1.96924085114733	2.81832059141065\\
2.0502990590364	2.82612977760851\\
2.12132034355964	2.82842712474619\\
};
\addlegendentry{Сумма Минковского}

\end{axis}

\begin{axis}[%
width=0.798\linewidth,
height=0.597\linewidth,
at={(-0.104\linewidth,-0.066\linewidth)},
scale only axis,
xmin=0,
xmax=1,
ymin=0,
ymax=1,
axis line style={draw=none},
ticks=none,
axis x line*=bottom,
axis y line*=left,
legend style={legend cell align=left, align=left, draw=white!15!black}
]
\end{axis}
\end{tikzpicture}%
        \caption{Эллипсоидальные аппроксимации для 100 направлений.}
\end{figure}
\begin{figure}[b]

        \centering
        % This file was created by matlab2tikz.
%
%The latest updates can be retrieved from
%  http://www.mathworks.com/matlabcentral/fileexchange/22022-matlab2tikz-matlab2tikz
%where you can also make suggestions and rate matlab2tikz.
%
\definecolor{mycolor1}{rgb}{0.00000,0.44700,0.74100}%
\definecolor{mycolor2}{rgb}{0.85000,0.32500,0.09800}%
%
\begin{tikzpicture}

\begin{axis}[%
width=0.618\linewidth,
height=0.471\linewidth,
at={(0\linewidth,0\linewidth)},
scale only axis,
xmin=-8,
xmax=8,
xlabel style={font=\color{white!15!black}},
xlabel={$x_1$},
ymin=-6,
ymax=6,
ylabel style={font=\color{white!15!black}},
ylabel={$x_2$},
axis background/.style={fill=white},
axis x line*=bottom,
axis y line*=left,
xmajorgrids,
ymajorgrids,
legend style={at={(0.03,0.97)}, anchor=north west, legend cell align=left, align=left, draw=white!15!black}
]
\addplot [color=mycolor1, forget plot]
  table[row sep=crcr]{%
1.1711787112841	3.34230777773576\\
1.34444389618673	3.33686722689733\\
1.50648318534277	3.32144701348819\\
1.65806242150429	3.29723890999844\\
1.79995419924852	3.26523166869947\\
1.93290799118999	3.22622794766654\\
2.05763032437368	3.18086276334402\\
2.17477223041259	3.12962150018857\\
2.28492171716205	3.07285634836238\\
2.3885995032216	3.01080060567632\\
2.48625667387058	2.94358064140543\\
2.57827324651315	2.87122553999274\\
2.66495688065443	2.79367457178694\\
2.74654114431786	2.71078271324095\\
2.82318287015103	2.62232448808267\\
2.8949582146523	2.52799644473262\\
2.96185708693385	2.42741864049226\\
3.02377565347494	2.3201355847161\\
3.0805066685131	2.20561721602438\\
3.13172744617751	2.08326066796132\\
3.17698540702193	1.95239382914097\\
3.21568133490354	1.81228204209558\\
3.24705081945703	1.66213971753183\\
3.27014489780722	1.50114915910447\\
3.28381172002077	1.32848945666784\\
3.28668221628109	1.14337881243048\\
3.2771642749897	0.945133918403077\\
3.25345178932588	0.733249674250213\\
3.21355684151227	0.507501133889645\\
3.15537469570401	0.268066487797808\\
3.07679115926015	0.015664562009573\\
2.97583883725379	-0.248307401701393\\
2.85090136682411	-0.521659161442773\\
2.70095216778915	-0.801347823435408\\
2.52579793500058	-1.08350783899399\\
2.32628156602984	-1.3635980584876\\
2.10439191407115	-1.63667268179468\\
1.86323617293213	-1.89775167737354\\
1.60685736899554	-2.14223401489281\\
1.33991789027605	-2.3662777508081\\
1.06730585933085	-2.56707514274059\\
0.793739425298697	-2.74297851870391\\
0.523437853608312	-2.89347215049445\\
0.259902596754298	-3.0190207516634\\
0.00581903236219554	-3.1208447467499\\
-0.2369372555121	-3.20067372983419\\
-0.467220354507	-3.2605178574641\\
-0.684493265340969	-3.30248032509504\\
-0.888694234930048	-3.32861897022119\\
-1.0801092196493	-3.34085455680139\\
-1.25926106809304	-3.34091776153004\\
-1.42681950330378	-3.33032515794113\\
-1.58353177449541	-3.31037507736533\\
-1.73017158467123	-3.2821558685242\\
-1.86750300244094	-3.24656096722397\\
-1.99625599202063	-3.20430689704777\\
-2.11711054093165	-3.15595168681459\\
-2.23068687011842	-3.10191219445658\\
-2.33753972875556	-3.04247952074268\\
-2.43815523449964	-2.97783215102365\\
-2.5329490934295	-2.90804674702345\\
-2.6222653208492	-2.83310668006404\\
-2.70637479428206	-2.75290849531896\\
-2.78547311726505	-2.66726655590999\\
-2.85967737154948	-2.5759161597878\\
-2.92902140016315	-2.47851546990414\\
-2.99344930844609	-2.37464666482809\\
-3.05280690958282	-2.26381681735669\\
-3.10683089311493	-2.14545915779927\\
-3.15513558252945	-2.01893559215682\\
-3.19719730262603	-1.88354163878489\\
-3.23233664163192	-1.73851533219976\\
-3.25969932318791	-1.58305212075492\\
-3.27823706849968	-1.41632833369867\\
-3.28669080350863	-1.23753634540448\\
-3.28357990805066	-1.04593497683649\\
-3.26720291066679	-0.840918687724783\\
-3.23565696131938	-0.622108318461704\\
-3.18688516768166	-0.389463978954261\\
-3.11876167995196	-0.143416533926881\\
-3.02922302993614	0.11499241610004\\
-2.91644915653916	0.383980836121862\\
-2.77908754849337	0.660934801326638\\
-2.61649915361415	0.942389492277687\\
-2.42898799999709	1.22411526429499\\
-2.21796402213223	1.50132757866017\\
-1.9859883878281	1.76901289011117\\
-1.7366683581316	2.02232922526914\\
-1.47440268220478	2.25701269019761\\
-1.20401775520347	2.46971278707818\\
-0.930363383198716	2.65819606075749\\
-0.657943143023997	2.82139307813531\\
-0.390637102663735	2.95930311865273\\
-0.131543811712393	3.07279931782965\\
0.117062449314586	3.1633871238911\\
0.353684227850852	3.23296276733506\\
0.577496523037587	3.28360333410188\\
0.788216624026908	3.3174035782658\\
0.985971690059736	3.33636162404406\\
1.1711787112841	3.34230777773576\\
};
\addplot [color=mycolor2, forget plot]
  table[row sep=crcr]{%
2.36712178368749	2.74633057020532\\
2.41188305682678	2.74494651398331\\
2.45036041459025	2.74130089565813\\
2.48383536069915	2.73596692388682\\
2.51327948444918	2.72933446739078\\
2.53943919311047	2.72166753585199\\
2.56289492242467	2.71314181506891\\
2.58410302011837	2.70386940753685\\
2.60342571833248	2.69391520541791\\
2.62115280082523	2.68330766515176\\
2.63751738560017	2.67204572812381\\
2.65270745907852	2.66010298488441\\
2.66687427014079	2.64742976011011\\
2.68013832873481	2.63395351039579\\
2.69259349381253	2.61957772004863\\
2.70430943819101	2.6041793136995\\
2.71533261385189	2.58760445132293\\
2.72568568587575	2.56966240776041\\
2.73536523285703	2.55011704085516\\
2.74433729805113	2.52867508999008\\
2.75253007967915	2.50497017880853\\
2.75982261143016	2.47854086090236\\
2.76602760995085	2.44880025137829\\
2.77086559717422	2.41499358451471\\
2.77392567575526	2.37613820122431\\
2.77460549370625	2.33093765236215\\
2.77201821305984	2.27765729690984\\
2.7648464218491	2.21394231571241\\
2.7511098920125	2.13654986655322\\
2.72779317877431	2.04095569483249\\
2.69024836192134	1.92078707468285\\
2.6312539409082	1.7670493095065\\
2.53961413480675	1.56721869756115\\
2.39837999194539	1.30464987086518\\
2.18360687141246	0.959750006540987\\
1.8667826621648	0.516209188824812\\
1.42701784641989	-0.0238781630653391\\
0.87502831245873	-0.620875059217897\\
0.268936641059173	-1.19901922110702\\
-0.309576813214938	-1.68530797756881\\
-0.802100907580029	-2.04896030059078\\
-1.19206161730571	-2.300440537754\\
-1.49041992551985	-2.46708460024485\\
-1.7168759123608	-2.57532489307401\\
-1.88995458136084	-2.64492059461505\\
-2.02413470299114	-2.68919932969544\\
-2.12995725581116	-2.71680265831965\\
-2.21490837909904	-2.73327951527395\\
-2.284286999297	-2.74220891806166\\
-2.34187032729419	-2.74592425079378\\
-2.39038313346388	-2.74596637420131\\
-2.4318186528206	-2.74336542452957\\
-2.46765634360889	-2.73881708355027\\
-2.49901014208278	-2.7327940719517\\
-2.52673028332805	-2.72561759993438\\
-2.551474024202	-2.71750375208859\\
-2.57375536742722	-2.70859393338899\\
-2.5939804436843	-2.69897499500867\\
-2.61247296631128	-2.688692537203\\
-2.6294927100655	-2.67775958603992\\
-2.64524900273341	-2.66616202895009\\
-2.6599105764423	-2.65386167486455\\
-2.67361268902278	-2.64079746083638\\
-2.68646212009346	-2.62688508668123\\
-2.69854042224194	-2.61201517643097\\
-2.70990563030842	-2.59604990865956\\
-2.72059247497691	-2.57881790224895\\
-2.73061098716967	-2.5601069666803\\
-2.73994319179238	-2.53965410007404\\
-2.74853734053432	-2.51713180921804\\
-2.75629877602211	-2.49212938365471\\
-2.76307597883326	-2.46412710489964\\
-2.76863950239953	-2.43246039507853\\
-2.77265014411951	-2.39626942454063\\
-2.7746104875259	-2.35442742359956\\
-2.77379029091263	-2.3054374557388\\
-2.76911009981251	-2.24728211523106\\
-2.75895730729457	-2.17720283029247\\
-2.74089227231279	-2.0913749275097\\
-2.71117633487659	-1.98443346302753\\
-2.66401914162082	-1.84880387542276\\
-2.59041768625865	-1.67383876852635\\
-2.47652724106683	-1.44496938777841\\
-2.30193857203366	-1.14371563581315\\
-2.0396739695219	-0.750852651889909\\
-1.66265564107253	-0.256803878437201\\
-1.16248546087895	0.319450601512933\\
-0.573614265552006	0.917547528939285\\
0.0284025713818862	1.45675943888985\\
0.5685251002902	1.88250662286626\\
1.00960568842464	2.18713123722861\\
1.35150239782658	2.39258093926768\\
1.61139397123953	2.52710219556684\\
1.80908182046774	2.61398894502504\\
1.96116532773652	2.66959557253549\\
2.08006223005889	2.7046815507453\\
2.17466815864962	2.726172144186\\
2.25127885929965	2.73851896753466\\
2.3143624731471	2.74460741055068\\
2.36712178368749	2.74633057020532\\
};
\addplot [color=mycolor1, forget plot]
  table[row sep=crcr]{%
1.27349527731192	3.20471819063596\\
1.44428602357601	3.19936336562619\\
1.60261405164943	3.18430366081914\\
1.7494617470495	3.16085789974953\\
1.88579739700425	3.13011008910517\\
2.0125428514028	3.09293326009606\\
2.13055396048019	3.05001429713906\\
2.24060988431472	3.00187741815919\\
2.34340825178317	2.94890508541948\\
2.43956391894906	2.89135583426738\\
2.5296096946948	2.82937892878458\\
2.61399786648816	2.76302598507574\\
2.69310169254936	2.69225981668208\\
2.76721625454656	2.61696080274643\\
2.83655821210403	2.53693109253768\\
2.90126408806333	2.45189696354759\\
2.96138675922161	2.36150966207012\\
3.01688984642981	2.26534508952985\\
3.06763970522629	2.16290276913649\\
3.11339473035919	2.05360465173142\\
3.15379172653358	1.93679451605467\\
3.18832919463209	1.81173900963237\\
3.21634758276645	1.67763178656044\\
3.23700691962578	1.53360274915386\\
3.24926287231705	1.3787350990062\\
3.25184326398312	1.21209372070441\\
3.24322856924441	1.03276925597542\\
3.22164196864756	0.839942844419659\\
3.18505716005814	0.632976472406207\\
3.13123499683714	0.411532478777436\\
3.05780235675828	0.175722049785833\\
2.96238692637022	-0.0737245209502354\\
2.84281753478955	-0.335283926070913\\
2.69738875896765	-0.606494094576824\\
2.52516943358411	-0.883881067768981\\
2.32630972784737	-1.16300759537571\\
2.10227863372139	-1.43868074143903\\
1.85595607299018	-1.70532290329812\\
1.59152340317857	-1.95746223706772\\
1.31414419128342	-2.19025259681242\\
1.02948815699354	-2.39991296505422\\
0.743198214823109	-2.58399557142257\\
0.460411488715003	-2.74144341261505\\
0.185417358609968	-2.87245752802465\\
-0.0785138856106611	-2.97823672044315\\
-0.329145589188702	-3.06066541941308\\
-0.565175433800149	-3.12201330013307\\
-0.786073059350422	-3.16468601912692\\
-0.991897989417819	-3.1910422753473\\
-1.18312736420307	-3.20327509398365\\
-1.36050921912967	-3.20334606094207\\
-1.52494663407333	-3.19295843529209\\
-1.67741164040707	-3.17355605030546\\
-1.81888459259207	-3.14633753472363\\
-1.95031370082384	-3.11227829447245\\
-2.07258962512226	-3.07215522866845\\
-2.18653078029436	-3.02657109539301\\
-2.29287589627036	-2.97597680757907\\
-2.39228121567003	-2.92069083506565\\
-2.48532040738634	-2.86091544019825\\
-2.57248581455441	-2.79674979051335\\
-2.65419005132678	-2.72820015727833\\
-2.73076724021758	-2.65518748343299\\
-2.80247336639283	-2.57755263051448\\
-2.86948534028096	-2.49505961995269\\
-2.93189842458056	-2.40739718984169\\
-2.98972171238905	-2.31417900956967\\
-3.04287135449774	-2.21494294566854\\
-3.09116124130123	-2.10914986787104\\
-3.13429086713906	-1.99618264208025\\
-3.1718301681864	-1.87534619761917\\
-3.20320126676684	-1.74586990360306\\
-3.2276573295482	-1.60691396750067\\
-3.24425923086295	-1.45758219425911\\
-3.25185150619326	-1.29694420959691\\
-3.24904030415707	-1.12407109656362\\
-3.2341778115443	-0.938089160971005\\
-3.2053599817185	-0.738256903482426\\
-3.16044720340501	-0.524069668561859\\
-3.09712028967762	-0.295394018093037\\
-3.01298571486023	-0.0526286044658788\\
-2.90574246407405	0.203120663078268\\
-2.77341566658946	0.469875931666726\\
-2.61464718527509	0.744671830950842\\
-2.42901069043125	1.02353733575874\\
-2.21729346307338	1.3016207686597\\
-1.98167017289334	1.57348138033746\\
-1.72569877677608	1.83352877918035\\
-1.45410337080575	2.0765414570174\\
-1.17236633997143	2.2981599756885\\
-0.886209684202	2.49524939806107\\
-0.601076144049998	2.66606314379628\\
-0.321711220923973	2.81019956313654\\
-0.0519055926422069	2.92839608031109\\
0.205593759258074	3.02223382531029\\
0.449034858659099	3.09382481911079\\
0.677523760761812	3.14553385255023\\
0.890847711899254	3.1797617627868\\
1.08929520758861	3.1987956202668\\
1.27349527731192	3.20471819063596\\
};
\addplot [color=mycolor2, forget plot]
  table[row sep=crcr]{%
2.33603117668886	2.77383773533797\\
2.37516464725374	2.77262691357743\\
2.40893170213665	2.76942690337039\\
2.43841940062857	2.76472767070704\\
2.46445304757259	2.75886290926017\\
2.48766781562485	2.75205859965667\\
2.50855867157367	2.74446471145197\\
2.52751564883675	2.7361760908235\\
2.54484907620749	2.7272462605596\\
2.56080781241566	2.7176964566711\\
2.57559252411427	2.7075213606192\\
2.58936537895104	2.69669243896537\\
2.60225707842181	2.68515944438462\\
2.61437184672533	2.67285038571794\\
2.62579076958642	2.65967009126052\\
2.63657370548617	2.64549733588026\\
2.64675984467711	2.63018035376686\\
2.65636684635875	2.61353039177146\\
2.66538831824105	2.59531274799315\\
2.67378918633813	2.57523445329204\\
2.68149819424238	2.55292734178153\\
2.68839630448344	2.52792464682231\\
2.69429904156482	2.49962833648105\\
2.69892963366187	2.46726298324861\\
2.70187786222841	2.42980975324494\\
2.70253626684757	2.38591063440619\\
2.69999981971571	2.33372758652888\\
2.69290573546456	2.27073288158737\\
2.67917401784381	2.19339441362953\\
2.65558277526102	2.09670321392352\\
2.61707186745425	1.97347563859052\\
2.55562090465979	1.81337698205752\\
2.45854873336742	1.60175127155448\\
2.30635872094887	1.31887400601422\\
2.07147274626518	0.941730524350173\\
1.72241984531358	0.453080187872498\\
1.24153219533735	-0.137619467154939\\
0.654680997818204	-0.772582252750343\\
0.0395069632936128	-1.35971688279848\\
-0.517637861558398	-1.82830042522886\\
-0.97089736627434	-2.16310655256113\\
-1.31839640890298	-2.38726933266721\\
-1.57909721866367	-2.53290594747214\\
-1.7749207828929	-2.62651336246446\\
-1.92392397638943	-2.6864302500371\\
-2.03934578994682	-2.72451851239282\\
-2.13049454066881	-2.74829318691323\\
-2.20385170818753	-2.76252013130226\\
-2.26395409160786	-2.77025452721126\\
-2.31401603417494	-2.7734836002745\\
-2.35634944249354	-2.7735195169949\\
-2.39264387200647	-2.77124055076024\\
-2.42415372127008	-2.76724084171689\\
-2.45182449666335	-2.76192476872171\\
-2.47637904157075	-2.75556732998468\\
-2.49837723207992	-2.74835332254689\\
-2.51825787341866	-2.74040305387863\\
-2.53636848776742	-2.731789323547\\
-2.5529867379886	-2.72254861451314\\
-2.56833597715801	-2.7126883350026\\
-2.58259659458752	-2.70219126617895\\
-2.59591428500911	-2.69101793056547\\
-2.60840599783681	-2.67910730161753\\
-2.62016406333094	-2.66637606485201\\
-2.63125879931482	-2.65271647587191\\
-2.64173974572939	-2.63799271230598\\
-2.65163553102222	-2.62203546177158\\
-2.66095222192739	-2.60463430292545\\
-2.66966982116945	-2.58552719290596\\
-2.67773632155005	-2.56438603244357\\
-2.68505834758345	-2.54079678062751\\
-2.69148683325652	-2.51423184316855\\
-2.69679525610873	-2.48401131577275\\
-2.70064643363102	-2.44924789564855\\
-2.70254137209289	-2.40876750845558\\
-2.70173941511501	-2.36099335456907\\
-2.69713171116671	-2.30377429501523\\
-2.68703767939672	-2.23412816295588\\
-2.6688733753315	-2.1478558869707\\
-2.63860733006011	-2.03896517257946\\
-2.58987303019969	-1.89883626876177\\
-2.51257074237411	-1.71511913654336\\
-2.39088248288237	-1.47063540976362\\
-2.20124949426826	-1.14348303786958\\
-1.91297510885859	-0.711698128877647\\
-1.49818929428948	-0.168118349712293\\
-0.95743799505463	0.455077656114234\\
-0.344692469317055	1.07773548420951\\
0.250413715583845	1.61106228502936\\
0.758152826604524	2.01148285385328\\
1.1568976469304	2.28696840373081\\
1.45817328343662	2.46805007856471\\
1.68387274034641	2.58488856364122\\
1.85434182773653	2.65981681155943\\
1.98517230036076	2.70765318707294\\
2.0874946534908	2.73784727198925\\
2.1690763903508	2.75637813802873\\
2.23533360006066	2.76705520229934\\
2.29007830562403	2.77233778434643\\
2.33603117668886	2.77383773533797\\
};
\addplot [color=mycolor1, forget plot]
  table[row sep=crcr]{%
1.39922803212736	3.07685806631181\\
1.5674959045967	3.07159193962536\\
1.72184120893004	3.05691962276783\\
1.86354618020044	3.03430258576604\\
1.99383941594411	3.00492434527477\\
2.11386334227955	2.96972519188087\\
2.22465770222629	2.92943638880441\\
2.32715347530744	2.88461124035927\\
2.42217316267472	2.83565189476223\\
2.51043459705609	2.78283160405981\\
2.59255635115887	2.726312621507\\
2.66906346531654	2.66616012709227\\
2.74039265271167	2.60235263950445\\
2.80689642206507	2.53478936498328\\
2.86884572677734	2.46329489204857\\
2.92643083902505	2.38762159185311\\
2.9797601812272	2.30745004343857\\
3.02885684319924	2.22238778377882\\
3.0736524851218	2.13196669573504\\
3.1139782873357	2.03563940659629\\
3.14955257373531	1.9327751939472\\
3.1799647289397	1.82265610911853\\
3.20465508608693	1.70447436470255\\
3.22289063840819	1.57733253330384\\
3.23373681095027	1.44024881653547\\
3.23602624800484	1.29217060644661\\
3.22832680506483	1.13200078307626\\
3.20891290060994	0.958642592487318\\
3.17574729649874	0.771070271643144\\
3.12648431983601	0.568433254718831\\
3.05851021117547	0.350200754759043\\
2.96904057175168	0.116349140091352\\
2.85529634653472	-0.132415162797099\\
2.71477438513296	-0.394420939308942\\
2.54561149927938	-0.666831451081862\\
2.34700920116711	-0.945545149042266\\
2.11964414530881	-1.22527369541402\\
1.86595263790311	-1.4998524733049\\
1.59017201622944	-1.76278083859751\\
1.29806813358121	-2.00790742861739\\
0.996371540126414	-2.23010725848011\\
0.692045715840486	-2.42578400848801\\
0.391565820616375	-2.59308707986694\\
0.100367539191781	-2.73183034725394\\
-0.177450479556643	-2.84318720481823\\
-0.439174975715654	-2.92927763544535\\
-0.683341813504384	-2.99275420405671\\
-0.909499160727534	-3.03645626209875\\
-1.11794272622867	-3.06316020920615\\
-1.30947089185102	-3.07542341942466\\
-1.48518319381155	-3.075503866566\\
-1.64632800142023	-3.06533330557286\\
-1.79419521627932	-3.04652411049664\\
-1.93004558742833	-3.02039455884332\\
-2.05506754143312	-2.98800216298559\\
-2.17035346301742	-2.95017858038136\\
-2.27688899630141	-2.90756246369496\\
-2.37555057125685	-2.86062847315891\\
-2.46710774227685	-2.80971180667064\\
-2.55222799210507	-2.7550282383983\\
-2.63148242817394	-2.69668997476434\\
-2.70535133355913	-2.63471776521598\\
-2.77422888817933	-2.56904972849322\\
-2.83842659644711	-2.49954732641521\\
-2.89817508338303	-2.42599886934325\\
-2.95362398038452	-2.34812089091296\\
-3.00483963484408	-2.26555769827939\\
-3.05180036004648	-2.17787939952491\\
-3.09438890639883	-2.08457874433518\\
-3.13238179607603	-1.98506720320871\\
-3.16543513925834	-1.87867087564694\\
-3.19306656963785	-1.76462708792787\\
-3.21463304463943	-1.64208295384835\\
-3.22930452229022	-1.5100977722518\\
-3.23603405874339	-1.36765196884681\\
-3.23352582137385	-1.21366638503477\\
-3.22020408341029	-1.04703704589849\\
-3.19418868326632	-0.866691949135152\\
-3.15328586629968	-0.671677510030317\\
-3.0950078113619	-0.461282278775401\\
-3.01663883454107	-0.23520307506533\\
-2.91536954396166	0.00624810083846552\\
-2.788518847893	0.261910999084898\\
-2.63385304009768	0.529552178029452\\
-2.44998681340866	0.805705536654112\\
-2.23681291575937	1.0856526413263\\
-1.99586497695243	1.36361286468693\\
-1.7304935342104	1.63317401023785\\
-1.44575426300667	1.88792082565777\\
-1.14798021434768	2.12213786617229\\
-0.844113546482297	2.33141799237157\\
-0.540956133719094	2.51303014802294\\
-0.244517307879121	2.66598261140652\\
0.0404151513087646	2.79081670105124\\
0.310442717751683	2.88923292693133\\
0.563501955589499	2.96366615687849\\
0.798667665060156	3.0168997946765\\
1.01589444674839	3.05176677289582\\
1.21575754923351	3.07094824312334\\
1.39922803212736	3.07685806631181\\
};
\addplot [color=mycolor2, forget plot]
  table[row sep=crcr]{%
2.2994083961655	2.79535024862062\\
2.33353375244637	2.79429370988155\\
2.36308892670794	2.79149226656542\\
2.38899329883575	2.78736357111642\\
2.41194606003088	2.78219240353602\\
2.43248641775227	2.77617155467401\\
2.45103550347488	2.76942852007211\\
2.46792596493415	2.76204308154774\\
2.48342313216485	2.75405889901689\\
2.49774031819916	2.74549105552355\\
2.5110499595793	2.73633077068861\\
2.52349174076519	2.72654803607802\\
2.53517847016565	2.71609262142336\\
2.54620021486251	2.70489368741015\\
2.55662701141105	2.69285807709674\\
2.56651032075193	2.67986721498817\\
2.57588326302111	2.6657723971624\\
2.58475953213234	2.65038808497362\\
2.59313072766097	2.63348259180601\\
2.60096162271741	2.61476524012883\\
2.60818256613162	2.59386860939319\\
2.61467772279225	2.57032380743149\\
2.62026706536589	2.54352563953533\\
2.62467873315634	2.51268289255035\\
2.62750619885164	2.47674632157668\\
2.6281409707708	2.43430271030891\\
2.62566512363811	2.38341658835518\\
2.61867669255461	2.32139034728934\\
2.60500128248795	2.24439673584546\\
2.58120961318637	2.14691411622279\\
2.54180743093121	2.02087048189424\\
2.47789806938556	1.85441211532424\\
2.37511364375075	1.63039068861648\\
2.21099885010276	1.32541676630578\\
1.95381570727021	0.912531393401144\\
1.56939776571841	0.374364860947135\\
1.046278826301	-0.268373525565837\\
0.430224874707573	-0.935264083236797\\
-0.182442693596096	-1.52035883315628\\
-0.708085217542624	-1.96268482913067\\
-1.11780394439882	-2.26544616804894\\
-1.42325717122235	-2.46253403104913\\
-1.64882507222525	-2.58856111738334\\
-1.81697924574442	-2.66894695501801\\
-1.94460168654153	-2.72026685041513\\
-2.04349791307372	-2.7529009999111\\
-2.12175472716191	-2.7733118805893\\
-2.18492142143171	-2.78556135910919\\
-2.23685134067867	-2.79224307449155\\
-2.28026355484299	-2.79504237948199\\
-2.31711056707747	-2.79507291549566\\
-2.34881912484968	-2.79308127701272\\
-2.37644941612019	-2.78957347677697\\
-2.40080174380478	-2.78489445799613\\
-2.42248909688625	-2.7792789317652\\
-2.44198728711015	-2.77288436531976\\
-2.45967010440917	-2.7658126305574\\
-2.47583430670791	-2.75812428765506\\
-2.49071759523835	-2.74984796403495\\
-2.50451166243361	-2.74098636524612\\
-2.51737170827005	-2.73151987698026\\
-2.52942336291049	-2.72140834464276\\
-2.54076764180827	-2.71059136440109\\
-2.55148433878095	-2.69898723541734\\
-2.56163409628376	-2.68649057275093\\
-2.57125925396019	-2.67296843883943\\
-2.58038344507941	-2.65825469619767\\
-2.58900976450084	-2.64214209085254\\
-2.59711714568452	-2.62437131392721\\
-2.60465432163687	-2.60461591281531\\
-2.61153034875796	-2.58246136459062\\
-2.61760004960281	-2.55737577301318\\
-2.6226417206743	-2.52866832876134\\
-2.62632277559825	-2.49542958836365\\
-2.62814615814032	-2.45644430058206\\
-2.6273654794016	-2.41006215886969\\
-2.62284832830332	-2.3540032622465\\
-2.61285230238781	-2.28506149819036\\
-2.59465244750208	-2.19864880205721\\
-2.56391581427681	-2.08809753480014\\
-2.51365623921814	-1.94362297330129\\
-2.43254884392664	-1.75091368452257\\
-2.30250431096509	-1.48970502660703\\
-2.09630378072396	-1.13403862962795\\
-1.77920298094463	-0.659113081286025\\
-1.32393593034075	-0.0624067233417369\\
-0.744435555399311	0.605702954626298\\
-0.117127565699744	1.24352382354984\\
0.459258666404913	1.76037734431491\\
0.927284812249035	2.12965076349379\\
1.28209324409541	2.37485769803888\\
1.54450086032372	2.53260670741068\\
1.73889001480468	2.63324617543375\\
1.88501585025053	2.69747674676842\\
1.99706351776802	2.73844513430852\\
2.0848113928928	2.76433742219006\\
2.15495148237977	2.78026822507319\\
2.21209951830831	2.7894763258861\\
2.25948547345075	2.79404790700794\\
2.2994083961655	2.79535024862062\\
};
\addplot [color=mycolor1, forget plot]
  table[row sep=crcr]{%
1.55639256905055	2.96512930676979\\
1.72194558400121	2.95995959644899\\
1.87187651283173	2.94571684865297\\
2.0078786804364	2.92401863646909\\
2.13152133501261	2.89614746612155\\
2.24422281966148	2.86310233390609\\
2.34724285915032	2.82564636890681\\
2.44168604901987	2.78434809112897\\
2.52851125160194	2.73961565129869\\
2.60854350893253	2.69172436094\\
2.68248638804546	2.64083822956509\\
2.7509335267552	2.58702633781514\\
2.81437867968246	2.53027483711349\\
2.87322387711878	2.4704952622475\\
2.92778547640811	2.40752971955752\\
2.97829795635211	2.34115339425146\\
3.02491531197481	2.27107471816354\\
3.06770987022764	2.19693346115496\\
3.10666827982388	2.1182969614416\\
3.14168433945825	2.03465470072291\\
3.17254822689084	1.94541147237283\\
3.19893158908426	1.8498795065186\\
3.22036787230806	1.74727013771703\\
3.23622724996468	1.63668597826971\\
3.24568561424095	1.51711516213764\\
3.24768745381299	1.38743014141285\\
3.24090323560218	1.24639485411533\\
3.22368343638461	1.09268593009691\\
3.19401403685822	0.924935969093117\\
3.14948258066916	0.741809589413677\\
3.08727020568363	0.542125192126739\\
3.00419323787546	0.325035616898807\\
2.89682647873737	0.0902762099333626\\
2.76174504315679	-0.161524827654143\\
2.59591463370637	-0.428505535065347\\
2.39723087888561	-0.707269802183286\\
2.16514959094201	-0.992739868771152\\
1.90127011825929	-1.27829123417025\\
1.60966980423026	-1.55625824365859\\
1.29679620703209	-1.81878348201875\\
0.970842704091363	-2.05883160002194\\
0.640728210317661	-2.2710859078631\\
0.31496544686145	-2.45247227019177\\
0.000731420951075753	-2.60220433110341\\
-0.296661411711697	-2.72142462868948\\
-0.573868667402983	-2.81262680605443\\
-0.829299485755192	-2.87905047295545\\
-1.06273726530523	-2.92417697995347\\
-1.27492670448002	-2.95137672496134\\
-1.4672079009809	-2.96370222005834\\
-1.64123122171026	-2.96379418017215\\
-1.79875555766596	-2.95386278896524\\
-1.94151727446253	-2.93571227958493\\
-2.07115260235749	-2.91078617505879\\
-2.18915727467812	-2.88021893913876\\
-2.29687043821154	-2.84488604726812\\
-2.39547334929136	-2.8054486049037\\
-2.48599634346008	-2.7623911007221\\
-2.56932981975949	-2.71605221826704\\
-2.64623656817436	-2.66664926811465\\
-2.71736382975639	-2.61429703952844\\
-2.78325415636491	-2.55902189368348\\
-2.84435454835788	-2.50077184161734\\
-2.90102358102023	-2.43942323201905\\
-2.95353634450248	-2.37478455042308\\
-3.00208705769334	-2.30659771959634\\
-3.04678919949292	-2.23453720000039\\
-3.08767294758247	-2.15820712535994\\
-3.12467963559481	-2.07713667842455\\
-3.157652843028	-1.9907739264093\\
-3.18632562825139	-1.89847841126238\\
-3.21030331962046	-1.79951295359335\\
-3.22904122207912	-1.6930354213926\\
-3.24181662900644	-1.57809169429292\\
-3.2476947425595	-1.45361180105303\\
-3.24548865203891	-1.31841232158201\\
-3.23371463787672	-1.17120972867802\\
-3.21054610855636	-1.01065146104482\\
-3.17377289507163	-0.835374085829839\\
-3.12077789444459	-0.644100498574099\\
-3.04855037558751	-0.435789619224154\\
-2.95376394575681	-0.209850277976997\\
-2.83295454904904	0.0335776083465671\\
-2.68283396756642	0.293291226760688\\
-2.50075751710368	0.566692571995787\\
-2.28532087855846	0.849548411045377\\
-2.03698913670611	1.13596864265365\\
-1.75858277962264	1.41872109049825\\
-1.45541039978119	1.68992187208685\\
-1.13490047418897	1.94199817831076\\
-0.805751468807456	2.1686805329637\\
-0.476814596794987	2.36573772500202\\
-0.156028488172868	2.53126267406597\\
0.150323962960791	2.66549690559156\\
0.437924755038436	2.77033638872705\\
0.70434694889485	2.84871913201416\\
0.948738849466358	2.9040595499326\\
1.17141446008507	2.93981790811621\\
1.3734572074647	2.95922355824186\\
1.55639256905055	2.96512930676979\\
};
\addplot [color=mycolor2, forget plot]
  table[row sep=crcr]{%
2.25648148144183	2.81152788377295\\
2.28611854761594	2.81060971185361\\
2.31188301568343	2.80816706740784\\
2.33454818820421	2.80455419539165\\
2.3547033978906	2.80001290555633\\
2.37280419498227	2.79470678315618\\
2.38920726532027	2.78874353664228\\
2.40419509161646	2.78218971157443\\
2.41799360188431	2.77508036725945\\
2.43078493427593	2.76742532867936\\
2.44271673263075	2.75921301833719\\
2.45390891910292	2.75041248499556\\
2.46445857600876	2.74097398775105\\
2.47444335022407	2.73082830913038\\
2.48392363222904	2.71988482420959\\
2.49294363197708	2.70802821854096\\
2.50153135458867	2.69511360418144\\
2.50969735110584	2.68095960758999\\
2.51743196071794	2.66533876684148\\
2.52470053987407	2.64796423727343\\
2.53143584233227	2.62847130085012\\
2.53752619284462	2.60639140337675\\
2.5427972491997	2.58111523783408\\
2.54698373201901	2.5518394717744\\
2.54968508637137	2.51748860956842\\
2.55029483420252	2.47659838720192\\
2.54788592718438	2.42713868356417\\
2.54102104395717	2.36624006572208\\
2.52743272100318	2.28976577528399\\
2.50347567868718	2.19163770921451\\
2.46318348874838	2.06278631195425\\
2.39666951885793	1.88959472276959\\
2.28759809150629	1.6519365163752\\
2.11000044012681	1.3219854402897\\
1.82735074161516	0.868276902077198\\
1.4035409516527	0.274924351724607\\
0.837870821427439	-0.420341133225879\\
0.201098027006392	-1.1100709252071\\
-0.395668872192818	-1.68034674015666\\
-0.880314982894813	-2.08838402879004\\
-1.24348455333275	-2.35683855332146\\
-1.507927557234	-2.52749837896194\\
-1.70086564924503	-2.63530514001686\\
-1.8439884403667	-2.7037269596881\\
-1.95252793963369	-2.74737273901532\\
-2.03676473790973	-2.77516838314625\\
-2.10360615644679	-2.79260071553722\\
-2.15774274589135	-2.80309798095731\\
-2.2024142036615	-2.80884485754463\\
-2.23990137594796	-2.81126134200048\\
-2.27184140715831	-2.81128716861249\\
-2.29943149129235	-2.80955366557274\\
-2.32356254689679	-2.8064896336933\\
-2.3449084001642	-2.80238785252393\\
-2.36398631770982	-2.79744760908678\\
-2.38119878662054	-2.79180230981551\\
-2.39686280734885	-2.78553760429208\\
-2.41123072445566	-2.7787033287183\\
-2.42450521898287	-2.77132131320246\\
-2.43685019558303	-2.76339032616339\\
-2.44839872074248	-2.75488894591073\\
-2.45925878639586	-2.74577683503627\\
-2.46951741246095	-2.73599467701678\\
-2.4792434156444	-2.72546287236615\\
-2.48848902910439	-2.71407895406076\\
-2.49729043540429	-2.70171354578068\\
-2.50566715422489	-2.6882045300937\\
-2.5136200862035	-2.67334889198013\\
-2.52112782906662	-2.65689142202918\\
-2.52814061359786	-2.63850905241165\\
-2.53457079335508	-2.61778897725139\\
-2.54027815951578	-2.59419774705328\\
-2.54504725993472	-2.56703700835965\\
-2.54855205743463	-2.53537912159193\\
-2.55030008292073	-2.49797191743159\\
-2.54954265294916	-2.45309530925004\\
-2.54512775459313	-2.39834166414063\\
-2.53525426020177	-2.33027415548639\\
-2.51705401219814	-2.24388966249411\\
-2.48587290072106	-2.13177492923707\\
-2.43403748503547	-1.98281496038808\\
-2.34881497238625	-1.7803862861264\\
-2.20943487700202	-1.50049942326001\\
-1.98435235053856	-1.11234032997585\\
-1.63453666967953	-0.588442579984248\\
-1.13579855446416	0.0653739113129525\\
-0.521391255039876	0.774076688632134\\
0.108412266451223	1.41484922999526\\
0.65379499857227	1.90419112353122\\
1.07599241105833	2.23744713376225\\
1.38626469515738	2.45193098293142\\
1.61181650075467	2.58754262213015\\
1.77756525843863	2.67335950991766\\
1.90184282954443	2.72798703390261\\
1.99718914272427	2.76284793443485\\
2.07202535032604	2.78492909576245\\
2.13203289943638	2.79855737073537\\
2.18110095632948	2.80646258005474\\
2.2219412271938	2.8104018315038\\
2.25648148144183	2.81152788377295\\
};
\addplot [color=mycolor1, forget plot]
  table[row sep=crcr]{%
1.75789046365152	2.87851795059347\\
1.92027634890559	2.8734607469727\\
2.06510472919776	2.85971412359751\\
2.19462378613701	2.83905983815503\\
2.31083799198744	2.81287128241266\\
2.41549985134058	2.78219036762057\\
2.51012168931292	2.74779378118756\\
2.59599693355803	2.71024730376353\\
2.67422461521597	2.66994888995881\\
2.74573358597395	2.62716200532674\\
2.81130464942246	2.58204086930475\\
2.87158979656993	2.53464911970734\\
2.92712827874576	2.48497317127276\\
2.9783595203119	2.43293127841629\\
3.02563297884943	2.37837907014263\\
3.06921506861588	2.32111211682189\\
3.10929321275395	2.2608659162411\\
3.14597700189229	2.19731354772933\\
3.17929632141098	2.13006113583178\\
3.20919617033611	2.05864118915142\\
3.23552773207612	1.98250384064836\\
3.25803507156029	1.9010060256971\\
3.27633662949837	1.81339871826247\\
3.2899004779828	1.71881254651789\\
3.29801213040874	1.61624249714943\\
3.29973364297103	1.50453310189055\\
3.29385296085882	1.38236664313758\\
3.27882323389149	1.2482587430566\\
3.25269363956878	1.10056848829626\\
3.2130368754626	0.937534243902808\\
3.15688500245051	0.757351543854944\\
3.08069596130842	0.558315211712442\\
2.98038849216385	0.339051842429407\\
2.85150160724088	0.0988657329701925\\
2.68954871829273	-0.161797793059755\\
2.49062888672968	-0.440814094262145\\
2.25230211025796	-0.733886406655619\\
1.97461137213715	-1.03430825388099\\
1.66096254637496	-1.33322930069096\\
1.31845337095567	-1.62057461227626\\
0.95732198834882	-1.88650319769001\\
0.589521285820144	-2.12298216439976\\
0.226850550200004	-2.32492867802353\\
-0.120713759157173	-2.49056304573379\\
-0.446003476157265	-2.62099290234318\\
-0.74485004764125	-2.71934212688711\\
-1.01568637964931	-2.78979839031806\\
-1.25887356310461	-2.83683344653703\\
-1.47600882825676	-2.86468792887994\\
-1.66935421231228	-2.87709929910705\\
-1.8414272424568	-2.87720508913988\\
-1.99473930098236	-2.86755172407349\\
-2.13164670382921	-2.85015597603359\\
-2.25427862994713	-2.82658529012176\\
-2.36451303746788	-2.79803830517911\\
-2.46398008560633	-2.76541678541352\\
-2.5540796809791	-2.72938590743431\\
-2.63600495018158	-2.69042280163042\\
-2.71076690996416	-2.64885455478097\\
-2.77921779159952	-2.60488729808164\\
-2.84207178352668	-2.55862798542887\\
-2.89992269720416	-2.51010026245484\\
-2.9532584501465	-2.45925556676253\\
-3.00247243669148	-2.40598034463359\\
-3.04787190743458	-2.35010004350221\\
-3.08968345364505	-2.29138034936013\\
-3.12805562209542	-2.2295259831704\\
-3.16305858336976	-2.1641772475904\\
-3.19468064919358	-2.0949044234456\\
-3.22282128331157	-2.02120005676339\\
-3.24728007602649	-1.94246916029047\\
-3.26774095669969	-1.8580173964057\\
-3.28375071035319	-1.76703744404503\\
-3.29469066885278	-1.66859403579765\\
-3.29974031987945	-1.56160867076005\\
-3.29783163093767	-1.44484589838781\\
-3.28759333321043	-1.3169045219713\\
-3.26728562584676	-1.17621934205837\\
-3.23472836344095	-1.02108243033109\\
-3.18723070850975	-0.849697569083211\\
-3.12153866748348	-0.660287153423712\\
-3.03382996236572	-0.451276164627668\\
-2.91980308567388	-0.221579177725888\\
-2.77492503123962	0.0289933634714278\\
-2.59490818701817	0.299224837097703\\
-2.37645882199743	0.585956016402782\\
-2.11825076772733	0.883689041791611\\
-1.82192293749608	1.18457256302336\\
-1.49273614167424	1.47898872208505\\
-1.1394873165461	1.75677715837975\\
-0.773492177610049	2.00881906414316\\
-0.406862835189353	2.22845893954485\\
-0.0506604067013423	2.41227436255297\\
0.286471226804994	2.5600189838237\\
0.598883778632005	2.67393043307189\\
0.883783632138136	2.75777669318722\\
1.14065779203266	2.8159688844512\\
1.3705662997484	2.85291106863863\\
1.57550018828592	2.87261358525458\\
1.75789046365152	2.87851795059347\\
};
\addplot [color=mycolor2, forget plot]
  table[row sep=crcr]{%
2.20548089943823	2.82259492056614\\
2.2310756781062	2.82180144067627\\
2.25341388640125	2.81968316600984\\
2.27314045325173	2.81653830900669\\
2.29074816700108	2.81257064955839\\
2.30661901094925	2.80791789296031\\
2.32105293751804	2.80267021067225\\
2.33428820981845	2.79688244343441\\
2.34651597522875	2.7905821012801\\
2.35789081685965	2.78377448399526\\
2.36853844083513	2.77644574287847\\
2.37856127247992	2.76856438176995\\
2.38804247531326	2.76008147691584\\
2.39704872485898	2.75092973518643\\
2.40563193312944	2.74102137806329\\
2.41383000688234	2.7302447118763\\
2.42166661519277	2.71845910236796\\
2.42914982141766	2.70548789061109\\
2.43626927894883	2.69110853666488\\
2.44299146696314	2.67503891108904\\
2.4492521000945	2.65691809977387\\
2.45494429738749	2.63627922524096\\
2.45990018838628	2.6125104174842\\
2.46386209338476	2.58479784748374\\
2.46643673675452	2.55204107421729\\
2.46702119429798	2.51272481779732\\
2.4646806518639	2.46472086352629\\
2.45794217040651	2.40497610469501\\
2.44443919338808	2.32901312638822\\
2.42028758225714	2.23012319567484\\
2.37898108438301	2.0980720291474\\
2.30946546712203	1.91712270122127\\
2.19302308476548	1.663480368929\\
1.99939986372173	1.30384507534387\\
1.68660213849871	0.801798313227344\\
1.21843474900127	0.146242380030686\\
0.612004139575645	-0.599490011561192\\
-0.0325838159621133	-1.29818124981711\\
-0.597959798471228	-1.83882174568265\\
-1.03282192882622	-2.20511881340256\\
-1.34748998865958	-2.43778649253415\\
-1.57235808532948	-2.58292713607252\\
-1.73507076486851	-2.67385011156894\\
-1.85550395067858	-2.73142497897582\\
-1.94693408294812	-2.76818955272608\\
-2.01809082677471	-2.79166777757401\\
-2.07475971740468	-2.80644585476573\\
-2.1208428346447	-2.8153804828952\\
-2.15902755535032	-2.8202920002674\\
-2.19120483273771	-2.82236549524384\\
-2.21873309374681	-2.82238716666129\\
-2.24260738515186	-2.82088663063703\\
-2.26356981838853	-2.81822450201677\\
-2.28218308232773	-2.81464742928703\\
-2.29888029991503	-2.81032332627052\\
-2.31399944553803	-2.80536427229711\\
-2.32780749443154	-2.79984154937507\\
-2.34051761431891	-2.7937955368736\\
-2.35230155238913	-2.78724214311511\\
-2.36329863765224	-2.78017681720398\\
-2.37362234450946	-2.77257678363149\\
-2.38336504885646	-2.76440187873777\\
-2.39260139227276	-2.75559418343563\\
-2.40139051405918	-2.74607650370016\\
-2.40977728886199	-2.73574962341048\\
-2.41779259964182	-2.72448812209074\\
-2.42545256414998	-2.71213439161339\\
-2.43275649824911	-2.69849027468424\\
-2.43968321494595	-2.6833054465566\\
-2.44618498344281	-2.66626121236649\\
-2.45217804115527	-2.64694770291678\\
-2.45752784854098	-2.62483136757586\\
-2.46202609767439	-2.59920792305946\\
-2.46535445985736	-2.56913307163661\\
-2.46702649734364	-2.5333185694399\\
-2.46629277352527	-2.48997324111257\\
-2.46198251161568	-2.43655495878129\\
-2.45223352200701	-2.3693766236903\\
-2.43402210812818	-2.28297159406618\\
-2.40233294227727	-2.169068980355\\
-2.34869425491878	-2.01497722303928\\
-2.25868913883791	-1.80125551068714\\
-2.10827647682834	-1.49929940663201\\
-1.86071669856813	-1.07246009907866\\
-1.47281070198137	-0.491513662073968\\
-0.927929490211162	0.223025382031232\\
-0.286220268576624	0.963687847270177\\
0.330295777802333	1.59139355984857\\
0.83197763016014	2.04178659571054\\
1.20336178905379	2.3350431603245\\
1.4692458554266	2.51888008351382\\
1.66006234090684	2.63361810391247\\
1.79961307762266	2.7058727610355\\
1.90421394291801	2.75185040212565\\
1.98463077787894	2.78125135738429\\
2.04795758667964	2.79993522196825\\
2.09893391063224	2.81151127678533\\
2.14078931321682	2.81825353172206\\
2.17577207927344	2.82162702115639\\
2.20548089943823	2.82259492056614\\
};
\addplot [color=mycolor1, forget plot]
  table[row sep=crcr]{%
2.02528855722988	2.8314627757905\\
2.18364256490326	2.82654712415527\\
2.32232089875638	2.81339711001726\\
2.44429966663803	2.79395569352328\\
2.55211787844742	2.7696676812432\\
2.64791266060073	2.74159314789719\\
2.73346857162937	2.71049808923055\\
2.81026898835822	2.67692434239347\\
2.87954376279983	2.64124227745091\\
2.94231081353296	2.60368983175353\\
2.99941111331336	2.56440098104945\\
3.05153738352734	2.52342612389607\\
3.09925713864025	2.48074627096032\\
3.14303078842218	2.43628243603582\\
3.18322543891858	2.38990122822675\\
3.22012490614952	2.34141733232042\\
3.25393630364273	2.29059331914839\\
3.28479339967968	2.23713703273042\\
3.31275676441611	2.18069664229565\\
3.33781053586055	2.12085331497088\\
3.3598554178422	2.05711135386713\\
3.37869727070575	1.98888555754715\\
3.3940303537022	1.91548550066498\\
3.40541391624517	1.83609643696888\\
3.41224041172014	1.74975663464195\\
3.41369314422302	1.65533126155474\\
3.40869073030275	1.55148360479601\\
3.39581554937713	1.43664570536127\\
3.37322376767736	1.30899285086508\\
3.33853634462256	1.16643045198713\\
3.28871512464255	1.0066085044921\\
3.21993811678531	0.826988980881285\\
3.12750686843875	0.625005188811786\\
3.0058500389319	0.398366670510614\\
2.84873066938348	0.145569184539406\\
2.64980675565619	-0.133354564135314\\
2.40369384305287	-0.435895060253338\\
2.10755788715585	-0.756164807355284\\
1.76295726042551	-1.08448738339765\\
1.37722220051641	-1.40802207807992\\
0.96344170103583	-1.71267650518036\\
0.538553849418496	-1.98584885182753\\
0.120068015783935	-2.21889121823761\\
-0.277199285299691	-2.40824585880885\\
-0.643099822950995	-2.55500109057541\\
-0.972465706932159	-2.66343648969624\\
-1.26423038869403	-2.73937577017955\\
-1.52009738124336	-2.78889636149337\\
-1.74329963282843	-2.81755650507628\\
-1.93768265558867	-2.83005667671636\\
-2.10712845574068	-2.83017869236823\\
-2.25524470362919	-2.82086681791378\\
-2.38522909864705	-2.80436229258904\\
-2.4998363994733	-2.78234341125037\\
-2.60139901110545	-2.75604987509515\\
-2.69187128999656	-2.72638475821845\\
-2.77288092865411	-2.69399440086462\\
-2.84577893581548	-2.65932925322217\\
-2.91168441125529	-2.62268931527768\\
-2.97152283345203	-2.58425754092257\\
-3.02605783251237	-2.54412399449862\\
-3.07591697091745	-2.50230293447381\\
-3.12161223055129	-2.45874445601906\\
-3.16355589123708	-2.41334187899278\\
-3.20207238300434	-2.36593571503202\\
-3.2374065516678	-2.31631477087814\\
-3.26972861743574	-2.26421472691255\\
-3.29913593606771	-2.20931435452912\\
-3.32565148958407	-2.15122939143132\\
-3.34921883136506	-2.089503972563\\
-3.36969297794654	-2.02359941386525\\
-3.38682646430956	-1.95288007154884\\
-3.40024944867019	-1.87659596779902\\
-3.40944235934206	-1.7938619205939\\
-3.41369912774013	-1.70363310855882\\
-3.41207859133372	-1.60467746319032\\
-3.40334129865304	-1.49554621970708\\
-3.3858689805257	-1.37454571988291\\
-3.35756493606156	-1.23971668570371\\
-3.31573660462241	-1.08883244372372\\
-3.25696856650511	-0.919435900458483\\
-3.17700812873283	-0.728947083919059\\
-3.07071029907343	-0.514887822395679\\
-2.93212679730042	-0.275282186335697\\
-2.75486957540281	-0.00928591918375054\\
-2.53290747287493	0.28195292691212\\
-2.26190501443769	0.594330814017856\\
-1.94100156034264	0.92006421124295\\
-1.57453530079942	1.24773500771413\\
-1.17283460989327	1.56356523148982\\
-0.751263100438952	1.85385259421138\\
-0.327490030079096	2.10772970392302\\
0.0819728852720324	2.31905616496465\\
0.464503781835654	2.48673557094366\\
0.81249731627353	2.61366335439495\\
1.12299079725902	2.70508271574215\\
1.39647235888047	2.76707313014238\\
1.63554733779064	2.80551841565643\\
1.8438470703118	2.82556921699742\\
2.02528855722988	2.8314627757905\\
};
\addplot [color=mycolor2, forget plot]
  table[row sep=crcr]{%
2.14285046027138	2.82808756846245\\
2.16480232914021	2.82740651091117\\
2.18404462096453	2.82558137619315\\
2.2011087002191	2.82286059781474\\
2.21640196057306	2.81941412880184\\
2.23024128938241	2.81535662847848\\
2.24287640264083	2.81076264354564\\
2.25450637336062	2.80567659600452\\
2.26529149609173	2.80011930365334\\
2.27536189104698	2.79409210175755\\
2.28482377819513	2.78757922486641\\
2.2937640413391	2.78054884188152\\
2.30225349204603	2.7729529542352\\
2.3103490941841	2.76472622844156\\
2.31809529588709	2.75578371453312\\
2.32552451819258	2.74601728057389\\
2.33265675227913	2.73529045062384\\
2.33949810305412	2.72343114522586\\
2.34603796346328	2.7102215560015\\
2.3522442775425	2.69538398746693\\
2.35805599539808	2.67856088584384\\
2.36337124406967	2.65928630618278\\
2.36802876326672	2.63694450529085\\
2.37177846998737	2.61070876898528\\
2.37423403079971	2.57944924242777\\
2.37479490319824	2.54159109198231\\
2.37251525076091	2.49489137507254\\
2.36587810379294	2.43608027600758\\
2.35239671276938	2.36027289623144\\
2.32789590352291	2.25999265404638\\
2.28520228049655	2.12355755666052\\
2.2117927329946	1.93254039078384\\
2.08591373136419	1.65843271354778\\
1.87192808224271	1.26107452856134\\
1.52191544631949	0.699323977495348\\
1.00401328873093	-0.0260975653771542\\
0.363748071527552	-0.813984659430715\\
-0.2688644928636	-1.50026846347405\\
-0.78505984725054	-1.99421220096836\\
-1.1621814512543	-2.3120046844514\\
-1.42730501323027	-2.50808040891141\\
-1.61430555121665	-2.62878977651985\\
-1.74908287219487	-2.70410346009035\\
-1.84892536871792	-2.75183309069718\\
-1.92497168994207	-2.78240988395908\\
-1.98441631057396	-2.80202204953594\\
-2.03198774293826	-2.81442637276709\\
-2.07086511691551	-2.82196289321932\\
-2.1032373715361	-2.82612594060473\\
-2.1306469031497	-2.82789152306617\\
-2.15420434386284	-2.82790950724923\\
-2.17472557537473	-2.82661924323466\\
-2.19282102135335	-2.8243208006842\\
-2.20895506929689	-2.82121982114766\\
-2.22348640556655	-2.81745629556059\\
-2.23669590171479	-2.81312330436156\\
-2.24880621761687	-2.80827933132791\\
-2.2599957841219	-2.80295635057308\\
-2.27040889628117	-2.79716504396977\\
-2.28016305849344	-2.79089798964273\\
-2.2893543410282	-2.78413133413755\\
-2.29806125319669	-2.77682524217822\\
-2.30634746269321	-2.76892326079756\\
-2.31426356178228	-2.7603506082126\\
-2.32184797734626	-2.75101127969837\\
-2.32912702634906	-2.74078373327658\\
-2.33611401487627	-2.72951475592708\\
-2.34280714846345	-2.717010888234\\
-2.34918583682532	-2.70302646046144\\
-2.35520469439475	-2.68724680041715\\
-2.36078408665524	-2.66926440501115\\
-2.36579532337417	-2.64854464007714\\
-2.37003732217863	-2.62437552995137\\
-2.37319932996421	-2.59579286104035\\
-2.37480028000697	-2.56146615364026\\
-2.3740880014168	-2.51952125498264\\
-2.36986769099099	-2.46725815618681\\
-2.3602027423829	-2.40069260335542\\
-2.34188069376843	-2.31379881220432\\
-2.30944331514121	-2.19725108711924\\
-2.25342347579248	-2.03637727552634\\
-2.15726868574138	-1.8081318719905\\
-1.99277374169129	-1.47800379565238\\
-1.7169275365176	-1.00247264364514\\
-1.2836338369358	-0.353478106408908\\
-0.692128251530688	0.422602884473332\\
-0.0377635087113686	1.17848387603558\\
0.544781901732044	1.77206572627127\\
0.989822588600999	2.17182516324887\\
1.30650097833272	2.42196025609287\\
1.52872677843643	2.57563263041337\\
1.68696675115438	2.67078647683365\\
1.80256019550217	2.73063605644361\\
1.88940158578403	2.76880563330631\\
1.95642880997747	2.79330947859752\\
2.00945904775913	2.80895397684342\\
2.05235799123028	2.81869458288088\\
2.08775582377096	2.82439568419062\\
2.11748483909939	2.82726178475665\\
2.14285046027137	2.82808756846245\\
};
\addplot [color=mycolor1, forget plot]
  table[row sep=crcr]{%
2.39662868959425	2.84951082055756\\
2.54951631809144	2.84478316146543\\
2.68057577202693	2.83236964475803\\
2.79368931059282	2.81435210655368\\
2.89200956708609	2.79221227117425\\
2.97808052474476	2.76699436724974\\
3.05395082474626	2.73942510677382\\
3.12127101874159	2.71000031675759\\
3.18137334210608	2.67904668602785\\
3.23533549826592	2.6467653770816\\
3.28403088381477	2.61326255303785\\
3.32816773512192	2.5785704660063\\
3.36831937711644	2.54266168096823\\
3.40494735390697	2.50545821962719\\
3.43841882493674	2.46683683518139\\
3.46901925630134	2.42663121128447\\
3.49696112960295	2.3846315684237\\
3.52238912121525	2.34058192200471\\
3.54538195800655	2.29417504129906\\
3.56595091331318	2.24504498662911\\
3.58403464861197	2.19275693830407\\
3.599489807854	2.13679386361574\\
3.61207640388388	2.0765393910213\\
3.62143656457363	2.01125607413664\\
3.6270645884653	1.9400580467324\\
3.62826545094727	1.86187693524784\\
3.62409786740233	1.77541990303787\\
3.61329677027294	1.67911905078622\\
3.59416873522277	1.57107248739018\\
3.56445296290204	1.44897996970295\\
3.52114107269472	1.31008145360664\\
3.46025386082096	1.15111750821774\\
3.3765877477028	0.96834975663678\\
3.26347767903156	0.757711152712086\\
3.11269107120892	0.515200371534801\\
2.91468049121388	0.237677533331435\\
2.65956107621816	-0.075792005356255\\
2.33922687728764	-0.42207479588027\\
1.95070870148828	-0.79208503593864\\
1.49992499673157	-1.17005005898236\\
1.00369620923486	-1.53532883829985\\
0.487714276647776	-1.86704741987091\\
-0.0196037104394338	-2.14958815963688\\
-0.494266489177459	-2.37589438155797\\
-0.921094490007363	-2.54715856090294\\
-1.29407185464677	-2.67002059913229\\
-1.61405066260037	-2.7533617882459\\
-1.88582793930159	-2.80600842284946\\
-2.11580931472076	-2.8355752154477\\
-2.31055960343822	-2.8481265843826\\
-2.4760753310711	-2.84826673357076\\
-2.61751342695858	-2.83939071981517\\
-2.73916306057416	-2.82395684149332\\
-2.84452673110119	-2.8037235494515\\
-2.93643778478825	-2.77993637218657\\
-3.01717928551152	-2.7534680827133\\
-3.08858977970994	-2.72492085634415\\
-3.15215174070506	-2.69469950321393\\
-3.20906307246607	-2.66306341554223\\
-3.26029379479034	-2.63016310153615\\
-3.30663043348983	-2.59606561165381\\
-3.3487104699092	-2.56077192816399\\
-3.38704883368716	-2.52422846610237\\
-3.42205801644831	-2.48633416034531\\
-3.45406300707221	-2.44694412473863\\
-3.48331191905296	-2.40587051060485\\
-3.5099828937072	-2.36288092139323\\
-3.53418760693175	-2.31769452613295\\
-3.55597146508519	-2.26997583310268\\
-3.57531032803307	-2.21932591896988\\
-3.59210332225446	-2.1652707442698\\
-3.60616097787423	-2.10724601437021\\
-3.61718750864978	-2.04457786249847\\
-3.62475551469728	-1.97645844436705\\
-3.62827067991085	-1.90191536813603\\
-3.6269231170911	-1.8197738030447\\
-3.61962086504921	-1.72861025449313\\
-3.60489972458324	-1.62669764617954\\
-3.58080240451189	-1.51194306149107\\
-3.5447195921549	-1.38182328893359\\
-3.49318786598444	-1.23333098397222\\
-3.42164818690977	-1.06295867761356\\
-3.32419146548788	-0.866772876631447\\
-3.19336668810003	-0.64066903948441\\
-3.0202170318666	-0.380945389129189\\
-2.79484166778538	-0.0853597202615685\\
-2.50789777204493	0.24524215579012\\
-2.15336789331032	0.60494911172521\\
-1.7323041212614	0.981292060555261\\
-1.25603671291713	1.35564225777066\\
-0.74637262927147	1.70653835040823\\
-0.23129746125784	2.01512023532681\\
0.262194573726511	2.26986342003333\\
0.714231227266017	2.46807926651014\\
1.11438689035123	2.61410444947348\\
1.46044698874666	2.71605977891121\\
1.75558068285544	2.78301068879355\\
2.00562516112094	2.82326121103597\\
2.21719544458287	2.84365845322455\\
2.39662868959425	2.84951082055756\\
};
\addplot [color=mycolor2, forget plot]
  table[row sep=crcr]{%
2.06125160838935	2.82609185066612\\
2.07994925124581	2.82551123450406\\
2.09642363792935	2.82394819411031\\
2.11110532962897	2.82160689086134\\
2.12432597927662	2.81862716062496\\
2.13634475923024	2.81510310819691\\
2.14736685278038	2.81109530783169\\
2.15755659642858	2.80663882557853\\
2.16704694356782	2.80174842493572\\
2.17594634547779	2.79642179774859\\
2.18434377663302	2.79064133531615\\
2.19231238805512	2.78437473909726\\
2.19991210618597	2.77757461782079\\
2.20719137482391	2.77017709745021\\
2.2141881433638	2.76209936047244\\
2.22093012022366	2.7532359127787\\
2.22743422158551	2.74345323023988\\
2.23370503599168	2.73258223742678\\
2.23973197208311	2.72040778022596\\
2.24548452388585	2.70665381282011\\
2.25090471474374	2.69096232906942\\
2.25589515985366	2.67286296058742\\
2.26030012253552	2.65172834480686\\
2.26387506387652	2.62670731102676\\
2.26623679389436	2.59662268469411\\
2.26678003680328	2.55981130214178\\
2.26453423629103	2.51386736871076\\
2.25791108650606	2.45522053002431\\
2.24424716451873	2.37842646554096\\
2.21895543141708	2.27495574519942\\
2.17393123321108	2.13113357503807\\
2.09461115491916	1.92481897622353\\
1.95508074839113	1.62108993957651\\
1.71280218315468	1.17128586036876\\
1.31487939823906	0.532571479500302\\
0.744373465911252	-0.267011737689928\\
0.0892415081571163	-1.0739861548735\\
-0.501235778002658	-1.71516536094982\\
-0.94830109085819	-2.14322609187765\\
-1.26079643502598	-2.40664257753268\\
-1.47616902870883	-2.56594328497232\\
-1.62722885702031	-2.66345388897609\\
-1.73627697297688	-2.72438721319827\\
-1.81745907450743	-2.76319317280894\\
-1.87968387289177	-2.78821006640888\\
-1.92865279428784	-2.80436411932649\\
-1.96810432160467	-2.8146496981553\\
-2.00055506641086	-2.82093927034229\\
-2.02774317329863	-2.82443476449945\\
-2.05089835097381	-2.82592560443978\\
-2.0709100552968	-2.8259403106581\\
-2.08843474420915	-2.82483797460678\\
-2.103965880597	-2.8228648340703\\
-2.11788063559145	-2.82019004338289\\
-2.13047168893757	-2.81692871332125\\
-2.1419692891369	-2.81315695930412\\
-2.15255681437367	-2.80892179935339\\
-2.16238190864915	-2.80424763616303\\
-2.17156454387382	-2.79914039392197\\
-2.18020289961054	-2.79358997034857\\
-2.18837765371054	-2.78757140076561\\
-2.19615507704615	-2.78104495165163\\
-2.2035891855187	-2.77395522776637\\
-2.21072309758051	-2.76622926419841\\
-2.21758965797856	-2.75777346288815\\
-2.22421130370965	-2.74846910370544\\
-2.23059905134031	-2.73816599101705\\
-2.23675035689961	-2.72667355727781\\
-2.2426454121733	-2.7137483885527\\
-2.24824114796816	-2.69907658675329\\
-2.25346173532385	-2.68224851142628\\
-2.25818356520954	-2.66272202846742\\
-2.26221127873464	-2.63976804230512\\
-2.26523890612181	-2.61238809494481\\
-2.26678556693555	-2.57918688165498\\
-2.26608652368133	-2.53817024701707\\
-2.26190369584654	-2.48641710820626\\
-2.25218696779675	-2.41953373467694\\
-2.23345280986463	-2.3307278718439\\
-2.19962169333636	-2.20922521273513\\
-2.13984083508523	-2.03762134861856\\
-2.0346032412399	-1.78791047607263\\
-1.85011538205858	-1.41776640448773\\
-1.53612890561482	-0.87652508965204\\
-1.04857221250265	-0.146003337612458\\
-0.417760327509739	0.682338124572095\\
0.221622946878704	1.42166635464838\\
0.74348915317282	1.95384390101691\\
1.11911647718661	2.2914069222994\\
1.37832089111808	2.49618674388445\\
1.55811461255222	2.62052375633131\\
1.68596713518099	2.69740311002973\\
1.77970009060015	2.74593105205812\\
1.85052730166635	2.77705905146031\\
1.90555619448917	2.79717432215525\\
1.94938858132061	2.81010366980761\\
1.98508170672757	2.8182068466805\\
2.0147205208524	2.8229794234728\\
2.03976279585362	2.82539290711205\\
2.06125160838935	2.82609185066612\\
};
\addplot [color=mycolor1, forget plot]
  table[row sep=crcr]{%
2.94343378702665	2.98150695436743\\
3.08885591882588	2.97703004882834\\
3.21054796459534	2.96551824604259\\
3.31341894940247	2.94914292861528\\
3.40124887065957	2.92937341442907\\
3.47695420897439	2.90719878294888\\
3.54279594379573	2.88327856753722\\
3.60053768793696	2.85804436184619\\
3.65156420497661	2.83176816902748\\
3.69696972781973	2.8046083443135\\
3.73762369200408	2.77664040394839\\
3.77421969304537	2.74787752274462\\
3.80731196955673	2.7182838987095\\
3.83734254235399	2.68778306693339\\
3.86466125736386	2.65626250988181\\
3.88954032096338	2.62357541000337\\
3.91218441887612	2.58954003765049\\
3.93273712261889	2.55393700506867\\
3.95128396880596	2.51650440568945\\
3.96785231077364	2.47693066782098\\
3.98240775591448	2.43484475941707\\
3.99484668103311	2.38980316445684\\
4.0049839205663	2.34127278980445\\
4.0125341945373	2.28860863059909\\
4.01708510995117	2.23102459677709\\
4.01805852587469	2.16755535943817\\
4.01465557245049	2.09700640285185\\
4.00577846405327	2.01788869870269\\
3.98991922134651	1.92833370145539\\
3.96500134333717	1.8259841226427\\
3.92815550330806	1.7078572334151\\
3.87540572228343	1.57018268928028\\
3.80124242044465	1.40823126569638\\
3.69807474295245	1.21618474376124\\
3.55561366713805	0.987168345099138\\
3.36039355718265	0.713697986754594\\
3.0959705528776	0.388984026151238\\
2.74485482730161	0.00965943621581311\\
2.29357605845332	-0.419860896077798\\
1.74123809312363	-0.88273264558884\\
1.10809223296134	-1.3486339523367\\
0.436257105221601	-1.78050448022638\\
-0.222411249521406	-2.14740128269557\\
-0.824456983247826	-2.43456471604976\\
-1.34594972456221	-2.64394799960773\\
-1.78189217938341	-2.78766825269008\\
-2.13923520644798	-2.8808316329203\\
-2.42990512098094	-2.93720386055972\\
-2.66642312116667	-2.96765779361241\\
-2.85990342575475	-2.98016035853\\
-3.01947047104845	-2.980318962509\\
-3.15233141227306	-2.97199814102729\\
-3.26407678003267	-2.95783324946641\\
-3.35901422996584	-2.93961142538769\\
-3.44046250987799	-2.91853917939846\\
-3.51098770154975	-2.89542548863729\\
-3.57258504950268	-2.87080559978002\\
-3.62681601558868	-2.84502437405132\\
-3.67491062596942	-2.81829233073741\\
-3.71784366349641	-2.79072328945494\\
-3.75639138919313	-2.76235953884773\\
-3.7911738055063	-2.73318844920671\\
-3.82268613730342	-2.70315310412262\\
-3.85132218756576	-2.67215862982249\\
-3.87739146216411	-2.64007529563112\\
-3.90113138683763	-2.60673904059147\\
-3.9227155029489	-2.57194977960228\\
-3.94225818044096	-2.53546760974441\\
-3.95981608835485	-2.49700683985336\\
-3.97538638144289	-2.45622757786813\\
-3.98890126247118	-2.41272440886056\\
-4.00021822633899	-2.36601146055877\\
-4.00910483726414	-2.31550286003529\\
-4.01521627070026	-2.26048721032088\\
-4.01806297880681	-2.20009423360216\\
-4.01696458777847	-2.1332511189094\\
-4.01098433732702	-2.05862538227507\\
-3.99883581308979	-1.97455027346157\\
-3.97875018437911	-1.87892820292819\\
-3.94828757872079	-1.76910799411692\\
-3.90407115412907	-1.64173460153032\\
-3.84141925200013	-1.49257885102639\\
-3.75385682688332	-1.31637716254411\\
-3.63252009480728	-1.10676098963577\\
-3.46556723356335	-0.856454205776585\\
-3.23794182258222	-0.55808102507739\\
-2.93227133418999	-0.206114291382088\\
-2.53220302067924	0.199545148698177\\
-2.02933798624782	0.648742309963167\\
-1.43256471133232	1.11760073735716\\
-0.773780891425569	1.57105760630653\\
-0.102173643872332	1.97343260536243\\
0.532508754172318	2.30116044932552\\
1.09595877732084	2.54836533236184\\
1.57436796768815	2.72307496482937\\
1.96969470722219	2.83964895043039\\
2.29210292472253	2.9128645884254\\
2.55419399350715	2.95510989188397\\
2.76792508339582	2.97575476480715\\
2.94343378702665	2.98150695436743\\
};
\addplot [color=mycolor2, forget plot]
  table[row sep=crcr]{%
1.94396230060006	2.81077700198304\\
1.95985108879152	2.81028302218281\\
1.97394601517608	2.80894524160945\\
1.98658836727855	2.80692872305329\\
1.99804290145849	2.80434666455556\\
2.00851794541265	2.80127490447696\\
2.01817955450266	2.79776146990237\\
2.02716163337153	2.7938328465329\\
2.0355732636903	2.7894980045526\\
2.04350405557244	2.78475081876226\\
2.05102806631827	2.77957126772946\\
2.0582066475911	2.77392562507895\\
2.0650904552572	2.76776572935734\\
2.07172076202316	2.761027312666\\
2.07813013562979	2.75362726375721\\
2.08434247168474	2.74545958092923\\
2.09037228714733	2.73638961317069\\
2.09622307156667	2.726245965862\\
2.10188433396748	2.71480911571078\\
2.10732673361643	2.70179526631605\\
2.11249427288296	2.68683315867038\\
2.11729183363615	2.66943021738284\\
2.12156511946277	2.64892218404249\\
2.12506787274033	2.62439657120363\\
2.12740717858999	2.59457357186253\\
2.12794994908125	2.55761602734519\\
2.12565858578613	2.51081799353142\\
2.11879356735678	2.4500804507019\\
2.10435914717095	2.36900692454446\\
2.077044104215	2.25731897225445\\
2.02717662267011	2.09810365785094\\
1.93689704487503	1.86338398381203\\
1.77403815419916	1.50898111967002\\
1.48781236794104	0.977591282821741\\
1.02797794570424	0.239102328779913\\
0.417065751156591	-0.618110712345637\\
-0.207022888965703	-1.38788049639511\\
-0.71136150653178	-1.93605025310214\\
-1.0682992731387	-2.27797487868556\\
-1.31082101086512	-2.48243692785583\\
-1.47709853453944	-2.60542161615692\\
-1.59438921386909	-2.68112630281884\\
-1.67991444787889	-2.72890911966843\\
-1.74431053192608	-2.75968640544852\\
-1.794230072076	-2.77975270835501\\
-1.83393929055117	-2.79284972826202\\
-1.8662525883997	-2.80127251714169\\
-1.89307927889216	-2.80647073286878\\
-1.91574910331315	-2.80938432065387\\
-1.93521073419012	-2.81063655925458\\
-1.95215596271729	-2.81064836758024\\
-1.96709937311284	-2.80970786100419\\
-1.98043069252134	-2.80801373357572\\
-1.99244995795617	-2.80570290192155\\
-2.00339162522548	-2.8028684301293\\
-2.01344140390623	-2.79957129251754\\
-2.02274820662335	-2.79584812070656\\
-2.03143274948098	-2.79171625123768\\
-2.0395938092478	-2.7871768871126\\
-2.04731280353105	-2.78221687147284\\
-2.05465713752028	-2.77680936482239\\
-2.06168260965477	-2.77091357168549\\
-2.06843506041373	-2.76447354870151\\
-2.0749513644328	-2.75741602296175\\
-2.08125979218648	-2.74964703925448\\
-2.08737969086333	-2.74104711921343\\
-2.09332034049265	-2.73146443019397\\
-2.09907871112645	-2.72070519171487\\
-2.10463564888983	-2.70852013631634\\
-2.10994970029888	-2.69458519617113\\
-2.11494725179667	-2.6784735460981\\
-2.11950674301536	-2.65961441457825\\
-2.12343308270449	-2.63723116573116\\
-2.12641542374375	-2.61024611324654\\
-2.12795587734824	-2.57713057502082\\
-2.12724598828813	-2.53566242125648\\
-2.12294648080199	-2.48252332081042\\
-2.112782671042	-2.41261201185111\\
-2.09278017382524	-2.31784838961673\\
-2.05579255081375	-2.18507740519626\\
-1.98867972953625	-1.99251610396963\\
-1.86729573432262	-1.70460212758402\\
-1.65006553490393	-1.26884645769957\\
-1.2810391545868	-0.632579777231207\\
-0.734829643605394	0.186538733238462\\
-0.0961807704141871	1.02629442439069\\
0.478237086983504	1.69129758873056\\
0.906738389298453	2.12857176341927\\
1.20126694458971	2.39332900349002\\
1.4015110305165	2.55153550099923\\
1.54059805830277	2.64771501683872\\
1.64033955717226	2.70768355489814\\
1.7142672731182	2.74595219806201\\
1.77077068523864	2.77078098484228\\
1.81515884049049	2.78700379272891\\
1.85088449176798	2.79753982960281\\
1.88025785289774	2.80420676365557\\
1.90486730397671	2.8081683476314\\
1.92583278470842	2.81018803520323\\
1.94396230060006	2.81077700198304\\
};
\addplot [color=mycolor1, forget plot]
  table[row sep=crcr]{%
3.8069742121015	3.32601021063417\\
3.94310767731051	3.32183896413045\\
4.05420275841036	3.31134312526451\\
4.14616621219568	3.29671363519818\\
4.22331079729383	3.27935618385384\\
4.28882087296	3.26017297175802\\
4.3450761920556	3.23973943178931\\
4.39387674716691	3.2184157825262\\
4.43659966989989	3.19641802187611\\
4.47430959606783	3.17386344125139\\
4.50783701640624	3.15079991969402\\
4.53783440608565	3.12722471731482\\
4.56481674587108	3.10309632370882\\
4.58919091662597	3.07834157938611\\
4.61127701392177	3.05285944670632\\
4.63132365137258	3.02652226330007\\
4.64951864016164	2.99917494528458\\
4.66599594085826	2.9706323448972\\
4.68083940812854	2.94067475863317\\
4.69408353516085	2.90904139284285\\
4.70571110826766	2.87542139564259\\
4.71564736063074	2.83944182921608\\
4.7237498186394	2.80065165304653\\
4.7297924996768	2.75850037441778\\
4.73344235137475	2.71230943916012\\
4.73422467258974	2.66123359920901\\
4.73147249347479	2.60420828388831\\
4.72425214485595	2.53987725159715\\
4.71125292455703	2.46649229210029\\
4.69062195021153	2.38177326153695\\
4.65971462450232	2.28271219372795\\
4.61471493250551	2.16530026432807\\
4.55005689082875	2.0241538357331\\
4.45755195938648	1.85202485053174\\
4.32511660075767	1.63922707388558\\
4.13506999383621	1.3731567878721\\
3.86233660637841	1.03846759065608\\
3.47398655764083	0.619251405989274\\
2.93385906115449	0.105608985940811\\
2.21819414526292	-0.493647269701428\\
1.34285086028235	-1.13739433911321\\
0.381733763959551	-1.75511273586638\\
-0.555945978397205	-2.27758418022185\\
-1.3801338200065	-2.67098549388462\\
-2.05273030094319	-2.94130697961128\\
-2.57903978312717	-3.11502187892367\\
-2.98402766689781	-3.22074360659178\\
-3.29550436360794	-3.28123933156228\\
-3.53714176607316	-3.31240885437513\\
-3.72707871394694	-3.32471889480433\\
-3.87861866314147	-3.32489355374631\\
-4.00137183341736	-3.31722202297675\\
-4.10227530238411	-3.30444276587309\\
-4.18636976282832	-3.28831018307363\\
-4.2573555868939	-3.26995072743412\\
-4.31798139885556	-3.25008592041586\\
-4.37031381264431	-3.22917262122432\\
-4.41592509522117	-3.2074919276244\\
-4.4560246114975	-3.18520597582556\\
-4.4915517058208	-3.16239444919179\\
-4.52324194637074	-3.13907806679447\\
-4.55167477689722	-3.11523355816124\\
-4.57730801706905	-3.09080293236409\\
-4.60050290583896	-3.06569879140361\\
-4.62154219991242	-3.03980676424577\\
-4.64064302591274	-3.01298569489415\\
-4.65796560952048	-2.9850659110852\\
-4.67361857917333	-2.95584566959024\\
-4.68766120322395	-2.92508567906909\\
-4.7001026196294	-2.89250141118904\\
-4.71089781383341	-2.85775269790244\\
-4.71993974885331	-2.82042984810432\\
-4.72704659614644	-2.7800351641179\\
-4.73194237861246	-2.73595824833843\\
-4.73422840048079	-2.68744279296103\\
-4.73334141867878	-2.63354154044063\\
-4.72849231313563	-2.57305464653371\\
-4.71857557025886	-2.50444457872921\\
-4.70203446371177	-2.42571771042901\\
-4.67665827273233	-2.33425874303738\\
-4.63927465991455	-2.22659917351108\\
-4.58528080037595	-2.09809665954561\\
-4.50793119077666	-1.94250354931282\\
-4.39727709648894	-1.75142557911221\\
-4.23867072522465	-1.51375682493256\\
-4.01092918938735	-1.21541927095236\\
-3.68491262374884	-0.840304133414801\\
-3.22495230318177	-0.374306646763107\\
-2.59825348631594	0.185029069108261\\
-1.79721195477659	0.813917342883822\\
-0.866410835864464	1.45435571377579\\
0.0967248201703723	2.03142822392588\\
0.98588338027817	2.4907984854268\\
1.73578374400764	2.82009224229463\\
2.33281856586493	3.03836073769983\\
2.79490690793268	3.17478938223685\\
3.14984561154891	3.25550255453373\\
3.42379321322665	3.29972949659059\\
3.6376413178944	3.32043092197198\\
3.8069742121015	3.32601021063417\\
};
\addplot [color=mycolor2, forget plot]
  table[row sep=crcr]{%
1.74677956590797	2.76321383348015\\
1.76055079162803	2.76278489084878\\
1.77289752304441	2.76161235880782\\
1.78408307816614	2.75982761735725\\
1.79431432821077	2.75752078244666\\
1.80375605241051	2.75475155614031\\
1.81254117813544	2.75155640064587\\
1.82077818881318	2.74795322267997\\
1.82855654270017	2.74394430324449\\
1.83595066345758	2.73951792268229\\
1.84302287744233	2.73464894166046\\
1.84982554550421	2.72929846348412\\
1.85640254543905	2.72341259492248\\
1.86279018923641	2.71692022200251\\
1.86901759477247	2.70972960689701\\
1.87510646340486	2.70172347333093\\
1.881070129905	2.69275205592092\\
1.88691163070849	2.68262330610839\\
1.89262035073271	2.67108901358273\\
1.89816650666999	2.6578249161294\\
1.90349221348034	2.64240175754852\\
1.90849698956012	2.62424240164907\\
1.91301395760529	2.60255695453783\\
1.91677004982509	2.57624233466723\\
1.91931793079299	2.54372286171832\\
1.91991642535571	2.50269036399665\\
1.91731432610301	2.44966862491135\\
1.90934758813476	2.37926393771651\\
1.8921676553373	2.28284881141417\\
1.858736269888	2.14624189106937\\
1.79592121051184	1.94579193806572\\
1.67939844049359	1.64293488187096\\
1.46775176340378	1.18231590857994\\
1.10905206980362	0.515816473456423\\
0.592268408768907	-0.315590085670382\\
0.0149776246806529	-1.12735122067658\\
-0.482966401320024	-1.74247563274645\\
-0.846247138018683	-2.13758925789272\\
-1.09479727169634	-2.37570531565814\\
-1.26460364302433	-2.51883787815067\\
-1.38356177823104	-2.60679926878341\\
-1.46967907361087	-2.66236645582977\\
-1.53410063802864	-2.69834771482707\\
-1.58376484002151	-2.7220768972298\\
-1.62309106660028	-2.73788014820263\\
-1.65497417750504	-2.74839259095045\\
-1.68136499101193	-2.75526922566731\\
-1.70361398362453	-2.75957864139342\\
-1.72267950622259	-2.76202763288094\\
-1.73925718341232	-2.76309323105423\\
-1.75386240879692	-2.76310253391312\\
-1.76688423497511	-2.76228223573638\\
-1.77862138744392	-2.76079006369596\\
-1.78930684247465	-2.75873511590543\\
-1.79912492476823	-2.75619120953072\\
-1.80822340856862	-2.75320570703901\\
-1.8167222134742	-2.749805330294\\
-1.82471973285041	-2.74599989645953\\
-1.83229748174183	-2.74178455266795\\
-1.83952352294432	-2.73714085580965\\
-1.84645497692253	-2.73203688565233\\
-1.85313981422446	-2.72642646053219\\
-1.85961804889195	-2.72024742257549\\
-1.86592238451269	-2.71341885602795\\
-1.87207829967102	-2.70583698036699\\
-1.8781034848286	-2.6973692982315\\
-1.88400644278413	-2.68784634661996\\
-1.88978391652406	-2.67705005108697\\
-1.89541657260374	-2.6646971390544\\
-1.90086197669035	-2.65041519717875\\
-1.90604322535744	-2.63370752631203\\
-1.91083040880429	-2.61390053769103\\
-1.91500991591535	-2.5900632758779\\
-1.91823254593586	-2.5608813002931\\
-1.91992359964181	-2.52445383804068\\
-1.91912269779814	-2.47795849089767\\
-1.91418977930217	-2.41708166155436\\
-1.9022493147976	-2.33502720544078\\
-1.8781137711729	-2.22076643402019\\
-1.83218329233983	-2.05599163654765\\
-1.74651403576271	-1.81029310733755\\
-1.58876039137854	-1.43616386293894\\
-1.30945691793092	-0.875652336430001\\
-0.867067023985944	-0.111895638767759\\
-0.301474533825516	0.73804118811955\\
0.249914151277534	1.46445740076682\\
0.681128085091329	1.96421367923717\\
0.982532897783201	2.27188940194\\
1.18754066861232	2.45616064097104\\
1.32911446155926	2.56798673754701\\
1.42990435001473	2.63766308133445\\
1.50409356324664	2.68225497396183\\
1.56045640237134	2.71142230728431\\
1.60451183081133	2.73077531966039\\
1.63982383437755	2.7436770095946\\
1.66876057492792	2.75220807282934\\
1.69293993199161	2.75769405610699\\
1.71349618514047	2.76100161394126\\
1.73124362836514	2.7627100897579\\
1.74677956590797	2.76321383348015\\
};
\addplot [color=mycolor1, forget plot]
  table[row sep=crcr]{%
5.23359968598068	4.06228569073568\\
5.36158749680671	4.05838065589803\\
5.46374098524715	4.04874039012966\\
5.54679183289556	4.03553600666112\\
5.61543665918267	4.02009609472798\\
5.67301780996828	4.00323838663666\\
5.7219588007418	3.98546437097936\\
5.76404748331935	3.96707561940243\\
5.80062350346106	3.94824452309343\\
5.83270461781342	3.92905791881896\\
5.86107325043917	3.90954423800289\\
5.88633671013261	3.8896904063681\\
5.90896962530503	3.86945219768965\\
5.92934413252931	3.84876026701216\\
5.94775144446764	3.82752320063858\\
5.96441718888765	3.80562836909689\\
5.9795120919587	3.78294101085854\\
5.99315901563786	3.75930172366362\\
6.00543694852082	3.73452234366319\\
6.0163822227875	3.70838001337468\\
6.0259869344175	3.6806090475804\\
6.0341942325797	3.65088997141281\\
6.0408897648076	3.61883478967854\\
6.04588804778265	3.58396709869379\\
6.04891177486158	3.54569499249963\\
6.04956090262756	3.50327371982051\\
6.0472664971646	3.45575351572038\\
6.04122127904133	3.40190563377832\\
6.03027372644046	3.34011579792157\\
6.0127639465138	3.26822818248925\\
5.98626454625468	3.18331316289279\\
5.94716346988753	3.08131622318111\\
5.8899795581516	2.95652060026942\\
5.80622154889659	2.8007204513202\\
5.68247087909964	2.60196159196587\\
5.49719491920497	2.34271072987884\\
5.21573614177751	1.99755522791973\\
4.7837443466677	1.53164430053319\\
4.12360982185974	0.904557796144839\\
3.15088009759361	0.0909888040006891\\
1.84002957784789	-0.872152820396208\\
0.319106348884118	-1.84937470406752\\
-1.15275137205982	-2.66989860688977\\
-2.36798134970148	-3.25059532097248\\
-3.27589286565043	-3.61600344575553\\
-3.92595046499675	-3.83088160761399\\
-4.38901224547138	-3.95194490186727\\
-4.72349231149159	-4.01701000298456\\
-4.97038901434058	-4.04891587652087\\
-5.15700419762603	-4.0610447774668\\
-5.3013489603052	-4.06123206442097\\
-5.41541488601347	-4.0541167851449\\
-5.507322473188	-4.04248567917992\\
-5.58268038608153	-4.02803518021852\\
-5.6454410834196	-4.0118073560535\\
-5.69844424824095	-3.99444337905209\\
-5.74376716383899	-3.97633364667063\\
-5.7829545913717	-3.95770831182488\\
-5.81717231615215	-3.93869275066441\\
-5.84731150137254	-3.91934195400816\\
-5.87406075604608	-3.89966196286193\\
-5.89795661305006	-3.87962314242784\\
-5.91941928808485	-3.85916816308982\\
-5.93877819398976	-3.83821641512527\\
-5.95629015390419	-3.81666588762123\\
-5.97215225555894	-3.79439310081417\\
-5.98651061370172	-3.77125138458311\\
-5.9994658310114	-3.74706757739965\\
-6.01107558656411	-3.72163703610851\\
-6.02135447578981	-3.69471666499943\\
-6.03027092845338	-3.6660154633198\\
-6.03774069318905	-3.63518182071692\\
-6.04361593943317	-3.60178641648343\\
-6.04766840603931	-3.56529903705841\\
-6.04956408873287	-3.5250568181969\\
-6.0488254886897	-3.48022018583099\\
-6.04477507184966	-3.42971085631841\\
-6.03644966793568	-3.37212324025763\\
-6.02246892500576	-3.30559577658956\\
-6.00082957465836	-3.22762096104647\\
-5.96857745114017	-3.13476030577434\\
-5.9212743543892	-3.02221052364306\\
-5.85211579503254	-2.88313695112032\\
-5.75045222363542	-2.70765027157872\\
-5.5993092530868	-2.48127354568202\\
-5.37134478901386	-2.18282749426356\\
-5.02288978461698	-1.78221201710447\\
-4.48777477941243	-1.24061089878179\\
-3.68049222212999	-0.520924958516225\\
-2.53409425373193	0.378121776297124\\
-1.0909212452937	1.37045141300698\\
0.439385935937216	2.287447501653\\
1.79910038993385	2.99051600874801\\
2.85847914257785	3.4563076483258\\
3.62862072756977	3.73827308049239\\
4.17677865185423	3.90035497170982\\
4.56938522686428	3.98976950678617\\
4.85591727395001	4.03610457934639\\
5.06993291680334	4.05686649742183\\
5.23359968598067	4.06228569073568\\
};
\addplot [color=mycolor2, forget plot]
  table[row sep=crcr]{%
1.33301670838737	2.61204888329464\\
1.34644320755614	2.61162908613279\\
1.35874493797112	2.61045944960314\\
1.37011960182753	2.60864331060449\\
1.38072728200983	2.60625048816175\\
1.39069931887731	2.60332468400919\\
1.40014485071414	2.59988834624712\\
1.40915568633019	2.59594569807516\\
1.41780996332873	2.59148436178709\\
1.42617490018531	2.58647582139012\\
1.43430884830736	2.58087482761002\\
1.44226277441712	2.57461773110508\\
1.45008124159532	2.56761961353\\
1.45780289896554	2.55976995270321\\
1.46546042493265	2.55092638520112\\
1.47307978398209	2.54090588631409\\
1.48067853267487	2.52947232623802\\
1.48826271427597	2.5163188064435\\
1.49582155700891	2.50104230408664\\
1.50331863627902	2.4831067372466\\
1.51067718701888	2.46178822934725\\
1.51775549857982	2.43609242260905\\
1.52430509226474	2.40462695806892\\
1.52989828989807	2.36540053415675\\
1.53380007949194	2.31549947077581\\
1.53473644177133	2.25055725852743\\
1.53046737484438	2.16387448001793\\
1.51699215125942	2.0449675132812\\
1.48709098642182	1.87729236425733\\
1.42784761001922	1.63526643156743\\
1.31745996182202	1.28282952524043\\
1.12526661095823	0.78250845831663\\
0.827178623124766	0.131829563845013\\
0.442626569180643	-0.585643728223974\\
0.045598959997802	-1.22687943553216\\
-0.292887299848114	-1.7039329392721\\
-0.550174643240503	-2.02194789246879\\
-0.737231321118399	-2.22531219891865\\
-0.872923550490124	-2.35520786936517\\
-0.973073002559847	-2.43955468949996\\
-1.04880622763155	-2.49550923857833\\
-1.10755169111278	-2.53338648794925\\
-1.15423717537672	-2.55944352147708\\
-1.19217088269001	-2.57755596109554\\
-1.22361657877387	-2.59018418731236\\
-1.25015675706226	-2.59892911557715\\
-1.27292158124762	-2.60485662231768\\
-1.29273488333345	-2.60869099274247\\
-1.31020886281663	-2.61093297797532\\
-1.32580668819217	-2.61193351417705\\
-1.33988468529741	-2.61194075318417\\
-1.35272130697319	-2.61113064541139\\
-1.36453738377374	-2.60962714494715\\
-1.37551051859768	-2.6075157086508\\
-1.38578547830666	-2.60485235082933\\
-1.39548180163462	-2.60166966586222\\
-1.4046994376114	-2.59798070619169\\
-1.41352296466125	-2.59378126805965\\
-1.42202476441441	-2.58905091388597\\
-1.43026740319638	-2.58375290074115\\
-1.43830538703167	-2.57783305831397\\
-1.44618638843291	-2.5712175450445\\
-1.45395198429384	-2.56380928878409\\
-1.46163788389146	-2.55548276837135\\
-1.46927355295561	-2.54607658901317\\
-1.47688103818478	-2.53538300946849\\
-1.48447264150035	-2.52313313308341\\
-1.49204684223526	-2.5089757795822\\
-1.49958144278572	-2.49244694411555\\
-1.50702218051136	-2.47292493719459\\
-1.51426374550244	-2.44956327808078\\
-1.5211177698076	-2.42118828226447\\
-1.52725792808844	-2.38613941931682\\
-1.53212386393177	-2.34201503276749\\
-1.53474935346189	-2.28525898061997\\
-1.5334484297272	-2.21047785433787\\
-1.52523299326183	-2.10930790940293\\
-1.50473158650961	-1.96857741655307\\
-1.4622569408778	-1.7675997104122\\
-1.38081867461681	-1.47540980909169\\
-1.23360993864157	-1.0527993701081\\
-0.989734196601563	-0.473118481173226\\
-0.641946753959142	0.227373334762711\\
-0.240284393996546	0.923745133831825\\
0.133534335238773	1.48729623857675\\
0.431414965234564	1.88026635099027\\
0.651290951575217	2.13508720401846\\
0.810394867871117	2.29739657590908\\
0.926643646894778	2.40180047104697\\
1.01345598548921	2.47031496538093\\
1.07994787151582	2.51624539095708\\
1.13216521285489	2.54760839904196\\
1.17413710720869	2.56931384510598\\
1.2085930466781	2.58444002420209\\
1.23742064298458	2.59496564299668\\
1.26195372251315	2.60219342990986\\
1.28315480571647	2.60699994060218\\
1.30173242034655	2.60998623944193\\
1.31821791114351	2.61157092933178\\
1.33301670838737	2.61204888329464\\
};
\addplot [color=mycolor1, forget plot]
  table[row sep=crcr]{%
7.0392772787345	5.11903163413247\\
7.16626142891188	5.11516704726827\\
7.26629530947662	5.10573296770008\\
7.34678121583159	5.0929404252796\\
7.41275009540204	5.07810515705988\\
7.46770818903169	5.06201734166849\\
7.51415523785521	5.04515049114053\\
7.55390997795022	5.0277825360959\\
7.58831969447633	5.01006760274933\\
7.6183982542733	4.99207933311409\\
7.64491893503354	4.97383734027816\\
7.66847801198178	4.95532339772637\\
7.68953899601874	4.93649119519741\\
7.70846377852075	4.91727191806879\\
7.72553470301668	4.89757698320867\\
7.7409701764394	4.87729870260255\\
7.75493552033362	4.85630928720692\\
7.76755014829301	4.83445835564298\\
7.77889171864666	4.81156891940326\\
7.78899757154672	4.78743163993666\\
7.79786345915379	4.76179696211644\\
7.80543926619556	4.73436449125311\\
7.81162104112765	4.70476865681686\\
7.81623814294102	4.67255923743224\\
7.81903354520929	4.63717461823094\\
7.81963414633169	4.59790456633448\\
7.81750600161046	4.5538375966568\\
7.8118861644611	4.50378523524493\\
7.80167728653956	4.44617092905127\\
7.78528137648615	4.37886367333006\\
7.76033149490167	4.29892322355866\\
7.72324748141428	4.2022006198842\\
7.6684797357689	4.08269666016919\\
7.58718501700106	3.93150794554762\\
7.46484515066429	3.73506430634936\\
7.27690174669529	3.47216907133276\\
6.98076794077052	3.10918441338855\\
6.50205235605686	2.59321050777568\\
5.71605754331805	1.84720902799201\\
4.44720912225841	0.787085083320875\\
2.56880033484852	-0.591757799732749\\
0.257878165560372	-2.07608138743818\\
-1.95873600611616	-3.31246837059941\\
-3.67246583887669	-4.13229019081978\\
-4.84890959120493	-4.60637352971057\\
-5.62924820005475	-4.86462617628409\\
-6.152693041282	-5.00162775078038\\
-6.51414732826121	-5.07201566016901\\
-6.77218075665923	-5.10540017020252\\
-6.96238590458728	-5.11778413520988\\
-7.10672674008617	-5.11798405531009\\
-7.21911589964707	-5.1109811032576\\
-7.3086237166577	-5.09965865272981\\
-7.38133267142339	-5.08571943340421\\
-7.4414301898953	-5.07018253002399\\
-7.4918688504861	-5.05366035915892\\
-7.53477577997466	-5.03651722848336\\
-7.57171304439667	-5.01896232518367\\
-7.60384725922387	-5.00110537118762\\
-7.63206250849633	-4.98299044337303\\
-7.65703701014054	-4.96461668067336\\
-7.67929606109214	-4.9459508960254\\
-7.69924911114715	-4.92693503066006\\
-7.71721597128642	-4.90749018638949\\
-7.73344539439188	-4.88751825480663\\
-7.74812813760957	-4.86690171678298\\
-7.76140587128517	-4.84550189100654\\
-7.77337678567625	-4.82315569550259\\
-7.78409836629924	-4.79967080559342\\
-7.79358749509538	-4.77481891194556\\
-7.80181773494674	-4.74832657256971\\
-7.80871331842344	-4.7198628778369\\
-7.81413892636756	-4.68902276011232\\
-7.81788371978822	-4.65530420766401\\
-7.8196371394281	-4.61807677139272\\
-7.81895247498845	-4.57653739295516\\
-7.81519171634902	-4.52964741199717\\
-7.80744098485871	-4.47604106957693\\
-7.79437851885377	-4.41388992474236\\
-7.77406412241355	-4.34069755793578\\
-7.74359506366882	-4.25298149230635\\
-7.69852849172415	-4.14576846289678\\
-7.63188425699374	-4.01177432194864\\
-7.53237456803583	-3.84004333817358\\
-7.38118488394934	-3.61366203084588\\
-7.14605406356131	-3.3059544239265\\
-6.77063751144112	-2.87457534687001\\
-6.15761509575811	-2.25458614919317\\
-5.15347938851971	-1.36027702764371\\
-3.58354295228443	-0.130374145500047\\
-1.43877767222122	1.34333463692235\\
0.89679834516737	2.74303223250638\\
2.88793902590139	3.77351412635487\\
4.3202076583399	4.40403635626887\\
5.27891129380902	4.75547535999178\\
5.91612850328448	4.94410764971809\\
6.34932856187132	5.04287378380005\\
6.65346014382276	5.09210914897521\\
6.87414979638642	5.11354760941571\\
7.0392772787345	5.11903163413247\\
};
\addplot [color=mycolor2, forget plot]
  table[row sep=crcr]{%
0.340159402357135	2.10600086215548\\
0.3628465262389	2.10528376142592\\
0.384983538536663	2.10317175270087\\
0.406716283993941	2.09969488583627\\
0.428180678011108	2.09484635528048\\
0.449505533367152	2.08858296968783\\
0.470815018388807	2.08082372707867\\
0.492230799808478	2.07144640903074\\
0.513873889130809	2.06028194820058\\
0.535866166731864	2.04710614126222\\
0.558331494769348	2.03162805630885\\
0.581396234946618	2.01347419948199\\
0.605188838922992	1.99216713731932\\
0.629837943593957	1.96709679723808\\
0.655468026444841	1.93748207838122\\
0.682191073509881	1.90231972724602\\
0.710091758583656	1.86031679879876\\
0.739202155707758	1.80980281214685\\
0.769459818573213	1.74861884556833\\
0.80064008913943	1.67398536129912\\
0.832250213425196	1.58236272978444\\
0.863371344598456	1.46934596208096\\
0.892440839855638	1.32968951028285\\
0.916994864544829	1.1576478771209\\
0.933461665095238	0.947922142732608\\
0.93722024091161	0.697507134450695\\
0.923254154813351	0.408373538265392\\
0.887613191091914	0.0899584839014675\\
0.829292619497089	-0.240624003826465\\
0.751350217007357	-0.562241095216022\\
0.660192239960307	-0.855958820248454\\
0.563363666195249	-1.10998962709869\\
0.46734720222843	-1.32075071805171\\
0.376485188537052	-1.49074974756309\\
0.293014529295588	-1.62559329921078\\
0.21763453804227	-1.73168296113876\\
0.150124373920323	-1.8149523813449\\
0.0898064836170812	-1.88038152662915\\
0.0358346873675908	-1.93193627925224\\
-0.0126491668834016	-1.97268813284077\\
-0.0564408523233183	-2.00498324783389\\
-0.0962456055018899	-2.03060375647545\\
-0.132673687998756	-2.05090219860937\\
-0.166246925494325	-2.06690671537743\\
-0.197409467037838	-2.07940074765622\\
-0.226539405413789	-2.08898241554729\\
-0.253959729761681	-2.09610838452086\\
-0.279948024299138	-2.10112612642857\\
-0.304744788313741	-2.10429756290364\\
-0.328560461293696	-2.10581629915087\\
-0.351581317507659	-2.10582004535157\\
-0.373974412350106	-2.10439936011743\\
-0.395891752286059	-2.10160350283493\\
-0.417473838109259	-2.09744391778838\\
-0.438852705211635	-2.09189566598317\\
-0.460154557460578	-2.08489694885098\\
-0.481502063043633	-2.07634671334331\\
-0.503016349197679	-2.06610017415172\\
-0.524818694049379	-2.05396192018484\\
-0.547031861225919	-2.03967607218156\\
-0.569780945882671	-2.02291270740463\\
-0.593193482477847	-2.00324944399919\\
-0.617398378026468	-1.98014665881395\\
-0.64252293665151	-1.95291428015224\\
-0.668686764263143	-1.92066745506512\\
-0.695990582565688	-1.88226770787954\\
-0.724496790253406	-1.83624571219609\\
-0.754196798300473	-1.78070209222836\\
-0.78495757674282	-1.71318514683432\\
-0.816436598886198	-1.63055203632806\\
-0.84795154121966	-1.5288386027832\\
-0.878292379011432	-1.40320237913186\\
-0.905478215418113	-1.24807521708215\\
-0.926507279688136	-1.0577656117023\\
-0.937246943050733	-0.827828665751056\\
-0.932746846728689	-0.557384782911339\\
-0.90829524435065	-0.251913355457592\\
-0.861193639531173	0.0751070859219576\\
-0.792430440350056	0.40386223059073\\
-0.7069485204366	0.713524546104237\\
-0.612037533875673	0.988336489411677\\
-0.514928609605447	1.22072882229166\\
-0.421088281198913	1.41053544394599\\
-0.333753957128362	1.56216496577172\\
-0.254311790829908	1.68184199557282\\
-0.182931519533185	1.77583455664028\\
-0.119117675453817	1.84962629757004\\
-0.062081846382775	1.90768220695569\\
-0.0109584532815586	1.95350093424583\\
0.035085202877312	1.98976990269858\\
0.0768012723721879	2.0185345950784\\
0.114847160032888	2.04134741988353\\
0.149787719570286	2.05938727992309\\
0.182104487988261	2.07355139534386\\
0.212207153711672	2.08452419672195\\
0.240444960310469	2.09282839920086\\
0.267117068662005	2.09886263746276\\
0.292481564777994	2.1029290972206\\
0.316763114824385	2.10525371981033\\
0.340159402357135	2.10600086215548\\
};
\addplot [color=mycolor1, forget plot]
  table[row sep=crcr]{%
6.61613611895751	4.8641942745862\\
6.74267149711494	4.86034166598725\\
6.84257530198247	4.85091880632069\\
6.92309873222533	4.83811961587769\\
6.98919202422371	4.82325590380537\\
7.0443173727696	4.80711879880973\\
7.09095015006891	4.79018426152558\\
7.13089554947986	4.77273282959086\\
7.16549338683827	4.75492090558833\\
7.19575351193484	4.73682393903221\\
7.22244715811728	4.71846288110962\\
7.24616967660651	4.69982042039387\\
7.26738427455025	4.68085079423991\\
7.28645286113455	4.66148541809789\\
7.30365793713208	4.64163566081575\\
7.31921809172996	4.62119353646822\\
7.33329877818253	4.60003072596646\\
7.34601943704234	4.5779960943986\\
7.357457604939	4.55491167731867\\
7.36765031073442	4.53056693316483\\
7.37659276261626	4.5047108687695\\
7.38423401969428	4.47704140898084\\
7.39046896592001	4.44719106021156\\
7.39512539095489	4.41470745448598\\
7.39794422323946	4.37902666722518\\
7.39854977688489	4.33943613597095\\
7.39640496071006	4.29502233073107\\
7.39074321390929	4.24459563420822\\
7.38046349486532	4.18658047707388\\
7.36396512538055	4.11885139407387\\
7.33888219380859	4.03848309344072\\
7.30164578431777	3.94136087081416\\
7.24674328402024	3.82155971649234\\
7.16543172741847	3.67033470293414\\
7.04344941432565	3.47445678193261\\
6.85688671052396	3.21347769087934\\
6.56480894289311	2.85543554581792\\
6.09704081110383	2.35120251270526\\
5.33922949371652	1.631832926147\\
4.13741497712578	0.627506738731737\\
2.39201362139747	-0.653956054596282\\
0.271129584082802	-2.0163128519437\\
-1.76706418105212	-3.15304630900311\\
-3.36477841515265	-3.9171910106998\\
-4.48057736347614	-4.36672322110522\\
-5.23178650862857	-4.61527876569066\\
-5.74140679225126	-4.74863483186194\\
-6.09621456353023	-4.81771508798844\\
-6.35101845621712	-4.85067488062163\\
-6.53967088879576	-4.86295399720598\\
-6.68330764764067	-4.86315077538039\\
-6.79543275686343	-4.85616295793073\\
-6.88490770779658	-4.8448438231133\\
-6.95770489220665	-4.83088712753518\\
-7.01795224457751	-4.81531109806863\\
-7.06856961994611	-4.79873010511294\\
-7.11166597664871	-4.78151108154078\\
-7.14879330431607	-4.76386568823666\\
-7.18111271547939	-4.74590569465811\\
-7.20950540074214	-4.7276767463185\\
-7.23464817377506	-4.7091791018496\\
-7.25706576499147	-4.69038029875894\\
-7.27716750980255	-4.67122266148723\\
-7.29527332439532	-4.65162737865237\\
-7.31163214276176	-4.63149616603733\\
-7.32643488682237	-4.6107110889196\\
-7.33982331218368	-4.58913282362824\\
-7.35189556689479	-4.56659742372425\\
-7.36270892512731	-4.54291147591349\\
-7.37227984687155	-4.51784535141028\\
-7.38058121681288	-4.49112404964878\\
-7.38753628018219	-4.46241485842959\\
-7.39300835961935	-4.43131067100693\\
-7.39678481775374	-4.39730723605238\\
-7.39855278713633	-4.35977175928978\\
-7.39786269021448	-4.31789894229912\\
-7.39407311222847	-4.27064842485589\\
-7.38626644212047	-4.21665415887213\\
-7.37311752223128	-4.15409054931505\\
-7.35268482515135	-4.08047058639661\\
-7.32207055776348	-3.99233467925752\\
-7.27685311900696	-3.88476016134848\\
-7.21011400666836	-3.75057132588405\\
-7.1107263430797	-3.57904450869508\\
-6.96028340185397	-3.35377011677173\\
-6.72755817023565	-3.04918973401285\\
-6.3588483963266	-2.62547601246506\\
-5.76350289791787	-2.02328116456653\\
-4.80347182629616	-1.16809173024485\\
-3.33099395850477	-0.0142908269286488\\
-1.3533002898372	1.34482324438388\\
0.788163082342835	2.6281575200801\\
2.62994410822462	3.58116089787993\\
3.97684522927382	4.1739541276016\\
4.89332317159997	4.50983315037412\\
5.51049081152359	4.69249098100198\\
5.93412383523154	4.78905695763134\\
6.23362438927599	4.83753309979793\\
6.45206700619369	4.8587482316194\\
6.61613611895751	4.8641942745862\\
};
\addplot [color=mycolor2, forget plot]
  table[row sep=crcr]{%
-0.808897193889222	1.2068070685132\\
-0.67471894432654	1.20245833535103\\
-0.523610279594059	1.1879328887286\\
-0.355602286690072	1.16095101337708\\
-0.172113986428172	1.11941457990547\\
0.0236781959832208	1.06184570982191\\
0.226763932915542	0.987874892639806\\
0.430695552019423	0.898606246212317\\
0.628471241032463	0.796666634737447\\
0.813666088851403	0.685849809625889\\
0.981411473905152	0.570458525142341\\
1.12889670178906	0.454591664773143\\
1.25532005519737	0.341614771537379\\
1.3614533580257	0.233920459000961\\
1.44906751721652	0.132944344606522\\
1.52041068524428	0.0393323749214226\\
1.57782741047141	-0.0468418764510821\\
1.62352634881597	-0.125874872234994\\
1.65946499918193	-0.198273711617701\\
1.68731177841586	-0.264643643790927\\
1.70845233617472	-0.325615284853456\\
1.7240172970881	-0.381801300816121\\
1.7349175934188	-0.433773178884676\\
1.74187988803718	-0.482050834467961\\
1.74547854069128	-0.527099946580128\\
1.74616281641526	-0.569333639502137\\
1.74427919808968	-0.609116372085116\\
1.74008919611336	-0.64676873530157\\
1.73378323461243	-0.682572399495952\\
1.72549120574174	-0.716774788177067\\
1.71529021613791	-0.749593254844176\\
1.70320995215191	-0.78121865142323\\
1.68923598730834	-0.811818232943133\\
1.67331125692419	-0.841537862784264\\
1.65533583439673	-0.870503477728018\\
1.63516506164871	-0.898821748127393\\
1.61260601211971	-0.926579827945359\\
1.58741219891264	-0.953844031504873\\
1.55927638618584	-0.980657195786712\\
1.52782132637932	-1.00703438474726\\
1.49258824550724	-1.03295646066862\\
1.45302296379723	-1.05836088380746\\
1.40845972292123	-1.08312890825763\\
1.35810318442154	-1.10706813633384\\
1.30100981290468	-1.12988922487335\\
1.23607118507752	-1.1511755157412\\
1.16200398222345	-1.17034471327827\\
1.0773548933259	-1.18660286237797\\
0.980533637945768	-1.19889346213648\\
0.869893534347268	-1.20584951255666\\
0.743884766397251	-1.2057645440841\\
0.601306128617595	-1.19661014135249\\
0.441667761094257	-1.17613898136997\\
0.265638700923344	-1.14211484190948\\
0.075483669219649	-1.09268866130737\\
-0.124686947267792	-1.02687859490506\\
-0.329061590529294	-0.945022081749565\\
-0.530781194663546	-0.849003795889475\\
-0.722990736443141	-0.742102114593396\\
-0.899945764438998	-0.62845748146599\\
-1.05777997546494	-0.512352747117631\\
-1.1947231245095	-0.397568796071803\\
-1.31082900078583	-0.286996637511487\\
-1.40744077312449	-0.182536253222041\\
-1.48662530858524	-0.0852023743823495\\
-1.55071680461233	0.0046725042194259\\
-1.60201264240033	0.0872227383389218\\
-1.64260458317222	0.162867011670197\\
-1.67430690470771	0.232172845558747\\
-1.69864409443724	0.295765188241698\\
-1.71687011040416	0.354269643004955\\
-1.73000118943305	0.408280244530071\\
-1.73885186708886	0.458343413176045\\
-1.74406894338996	0.504951945121495\\
-1.74616114088827	0.548544855252462\\
-1.74552383846293	0.589510369296379\\
-1.74245906604218	0.628190390074527\\
-1.73719127666227	0.664885439876693\\
-1.72987949587503	0.699859508876439\\
-1.72062641237825	0.733344499759449\\
-1.70948488701277	0.765544109744199\\
-1.6964622550066	0.796637072143682\\
-1.68152269475052	0.82677971561631\\
-1.66458784175479	0.856107805505731\\
-1.64553574026817	0.884737616685108\\
-1.62419814702131	0.912766155007379\\
-1.60035613145138	0.940270395453863\\
-1.57373385601163	0.967305337574049\\
-1.54399037377056	0.993900589216507\\
-1.51070925984532	1.02005507320129\\
-1.47338592058589	1.04572930415321\\
-1.43141254031662	1.0708345031942\\
-1.38406089980276	1.09521761509475\\
-1.33046384917194	1.11864109614215\\
-1.26959722289676	1.14075622649914\\
-1.20026571257888	1.16106882787121\\
-1.12109901138918	1.1788969420808\\
-1.0305687562377	1.19332178033102\\
-0.927042489768694	1.20313689765039\\
-0.808897193889222	1.2068070685132\\
};
\addplot [color=mycolor1, forget plot]
  table[row sep=crcr]{%
4.56917235808472	3.70268895532509\\
4.69993758331261	3.69869279347102\\
4.80517945949634	3.68875692811782\\
4.891315800367	3.67505917995538\\
4.96290040132152	3.65895608509231\\
5.02321828383969	3.64129574673471\\
5.07467792544773	3.62260597381011\\
5.11907237396694	3.60320901432834\\
5.15775539338562	3.58329249739931\\
5.19176199696943	3.56295381135076\\
5.22189209387517	3.54222806388184\\
5.24876929621458	3.52110568085189\\
5.27288272085366	3.49954330350572\\
5.29461693707984	3.4774702130657\\
5.31427347814993	3.45479163779903\\
5.33208619457745	3.43138974763909\\
5.34823195812602	3.40712277935365\\
5.36283768770315	3.38182247942628\\
5.3759842700924	3.3552898503993\\
5.3877076260163	3.32728900317498\\
5.39799687590262	3.29753872307469\\
5.40678924438351	3.26570112173645\\
5.41396095682356	3.23136643455092\\
5.41931285689038	3.19403258541623\\
5.42254870979102	3.15307750500636\\
5.42324298996123	3.10772124497457\\
5.42079311891639	3.05697350605851\\
5.41434817106706	2.9995600221317\\
5.40270123876636	2.93381789265783\\
5.38412463252741	2.85754479201215\\
5.35611365638706	2.767779099135\\
5.31498215350318	2.66047633057464\\
5.25521576712938	2.53003141868109\\
5.16843041292733	2.36857973242306\\
5.04170453134492	2.1650099448242\\
4.85499737117693	1.90370247719085\\
4.57753953354758	1.56336054093121\\
4.16419338875672	1.11739931029442\\
3.55651701769009	0.539884205672537\\
2.70077846049034	-0.176194625640385\\
1.59598179398859	-0.988275705995668\\
0.34581999886164	-1.79163729311536\\
-0.868505547238941	-2.46843827368793\\
-1.89989521026677	-2.96104252617115\\
-2.70104992150966	-3.28328830450124\\
-3.2968512620211	-3.4801094636453\\
-3.73503601365084	-3.59459902752057\\
-4.0596436972177	-3.65770455314172\\
-4.30398704887353	-3.68925815552074\\
-4.49148882414584	-3.7014315428108\\
-4.63824137087147	-3.70161391312862\\
-4.75529535780737	-3.69430712149698\\
-4.85031570845045	-3.68227869809872\\
-4.92869739665469	-3.66724603675176\\
-4.99430017400813	-3.65028169006803\\
-5.04993128854323	-3.63205557469415\\
-5.09766502239726	-3.61298162986896\\
-5.13905677865899	-3.59330788855458\\
-5.17528853037799	-3.57317251755698\\
-5.2072690591085	-3.55263902527036\\
-5.23570398384655	-3.53171846300699\\
-5.26114528157326	-3.51038332169017\\
-5.28402664366843	-3.4885759784651\\
-5.30468886032385	-3.46621343317026\\
-5.32339802304281	-3.44318938443409\\
-5.34035840243682	-3.41937425196426\\
-5.35572121908679	-3.39461345056524\\
-5.3695900665462	-3.36872399804908\\
-5.38202339202143	-3.34148935081348\\
-5.39303413670034	-3.31265217544548\\
-5.40258633778172	-3.28190455342899\\
-5.41058815087062	-3.24887484707502\\
-5.41688030653274	-3.21311008692461\\
-5.42121838611503	-3.17405221510248\\
-5.42324636218192	-3.13100574642952\\
-5.4224573917147	-3.08309325166024\\
-5.4181355313962	-3.02919330887705\\
-5.40926827985627	-2.9678528711087\\
-5.39441364137628	-2.89716183780545\\
-5.37149503593939	-2.81457122256595\\
-5.33747996512858	-2.71662666136708\\
-5.28786926267767	-2.59857519079825\\
-5.2158766204963	-2.45378606231255\\
-5.11110805423888	-2.27291421151429\\
-4.9574722721946	-2.0427622478448\\
-4.730065934337	-1.7449762831817\\
-4.39127956248136	-1.35535458235569\\
-3.88852005730529	-0.846294424062207\\
-3.16171272147315	-0.198036167364282\\
-2.17576608285899	0.575550783655259\\
-0.978405967993526	1.39910336774656\\
0.277135513527748	2.15141677350008\\
1.41235144448438	2.73817910064784\\
2.32873253767492	3.14086783121862\\
3.0217321914352	3.3944321931087\\
3.53266232012548	3.54541338136133\\
3.90919611930924	3.63111514804455\\
4.19018187717937	3.67652388323898\\
4.40369542251996	3.6972200520764\\
4.56917235808472	3.70268895532509\\
};
\addplot [color=mycolor2, forget plot]
  table[row sep=crcr]{%
-0.264450999198444	0.83214045157284\\
0.214248321186214	0.816914604110463\\
0.67404969566841	0.773214467044598\\
1.07668900558516	0.709144913196186\\
1.40495128677245	0.635400964505414\\
1.66060100929812	0.560697628911048\\
1.85510232803173	0.490203198683265\\
2.00198977903837	0.426152500911911\\
2.11319212545656	0.36900377396721\\
2.19803948641445	0.31834386558543\\
2.26344449923751	0.27342323238087\\
2.31441326848521	0.23342546846939\\
2.35454816309746	0.197585432330489\\
2.38644761730663	0.165230879414624\\
2.41199890705233	0.135788848554691\\
2.43258512757489	0.108777538152803\\
2.44922954376228	0.0837934455534409\\
2.46269600712489	0.0604980669807242\\
2.47355903123215	0.0386058415277215\\
2.48225297757613	0.017873829332876\\
2.48910679927453	-0.00190689675315377\\
2.49436870924686	-0.0209183594280127\\
2.49822372585025	-0.0393215490951768\\
2.50080609268576	-0.0572613155803396\\
2.50220791857965	-0.0748705063894331\\
2.50248493397078	-0.092273546530673\\
2.5016599400423	-0.109589626731456\\
2.49972428823439	-0.126935644927175\\
2.49663753563351	-0.144429030651753\\
2.49232524956892	-0.162190573051497\\
2.48667475943331	-0.180347369905173\\
2.47952845167494	-0.199036016081142\\
2.47067394697809	-0.218406153292496\\
2.45983014943958	-0.238624505032164\\
2.44662766294206	-0.259879513653218\\
2.43058135236097	-0.28238666560122\\
2.41105177329823	-0.306394505602213\\
2.38719064187274	-0.33219114112874\\
2.35786324977618	-0.360110606123845\\
2.32153750736643	-0.390537555348389\\
2.27612499249618	-0.423906948327799\\
2.2187544573104	-0.460691807730163\\
2.14545511915088	-0.501365289826523\\
2.05073370658385	-0.546310803193739\\
1.92706761617213	-0.59563317347751\\
1.76445875015479	-0.648796620887027\\
1.55049610602075	-0.704004144224202\\
1.27194805388467	-0.75732112120375\\
0.919473693297603	-0.801912560176133\\
0.49613899942984	-0.828489649186642\\
0.025692080922134	-0.828340341005733\\
-0.449320029613414	-0.798245514008142\\
-0.88412407220191	-0.743051905881648\\
-1.25034969255114	-0.672857552252268\\
-1.54125404579476	-0.597764286432564\\
-1.76463721550565	-0.524729734527018\\
-1.93369422062092	-0.457314658533467\\
-2.06140671142647	-0.396728677678376\\
-2.15842705775399	-0.34290581788556\\
-2.23282103499755	-0.295217495165058\\
-2.29048061161413	-0.252858746887685\\
-2.33565292845559	-0.215030107629663\\
-2.37139548208082	-0.181010551314087\\
-2.39992046853641	-0.150177905945483\\
-2.42284161642236	-0.122006247725035\\
-2.44134718298104	-0.0960546201814722\\
-2.45632033484567	-0.0719536257249904\\
-2.46842298187979	-0.0493926339595112\\
-2.47815443889097	-0.0281085739030853\\
-2.48589273303814	-0.00787648831769808\\
-2.49192386517619	0.0114982926954471\\
-2.49646261608196	0.0301866086436648\\
-2.49966732599551	0.0483408202054516\\
-2.5016502881176	0.0660992978795976\\
-2.50248485714598	0.0835902585406819\\
-2.50220999635652	0.100935130134395\\
-2.50083271248245	0.11825159952906\\
-2.49832861592892	0.135656479949484\\
-2.49464066471791	0.153268522416851\\
-2.48967597965003	0.171211289612278\\
-2.48330043261038	0.189616209620683\\
-2.47533048441306	0.208625929581672\\
-2.46552145099452	0.228398092645316\\
-2.45355096278771	0.249109660476085\\
-2.4389957876638	0.270961887204528\\
-2.42129931899874	0.294185997820034\\
-2.39972575103622	0.319049491914846\\
-2.37329508448751	0.345862697986741\\
-2.34069038956722	0.374984577974526\\
-2.30012498648155	0.40682550239901\\
-2.24915245698217	0.441842163594816\\
-2.18439784997777	0.480514825696315\\
-2.101188611267	0.523287783563198\\
-1.99308212590181	0.570437524813583\\
-1.85135785953935	0.621808199153835\\
-1.66474020557732	0.67632943271341\\
-1.42005374316342	0.731253673206172\\
-1.10516915155054	0.781251166446899\\
-0.715719010491333	0.818066616494719\\
-0.264450999198446	0.83214045157284\\
};
\addplot [color=mycolor1, forget plot]
  table[row sep=crcr]{%
3.17677417039348	3.06220363740278\\
3.31935043399715	3.05782096511845\\
3.43769887712985	3.046630132178\\
3.53706168261554	3.03081663195866\\
3.62140584972235	3.01183421791775\\
3.6937481576491	2.9906465340053\\
3.75639911613872	2.96788699145282\\
3.81114261030455	2.94396424135472\\
3.85936723600453	2.91913183226332\\
3.90216223782819	2.89353430231897\\
3.94038776148057	2.86723767038152\\
3.97472648412622	2.84024947963139\\
4.0057216751557	2.81253172303048\\
4.03380526928449	2.78400879558825\\
4.05931847405358	2.75457184164777\\
4.08252666933673	2.72408034563199\\
4.10362979670892	2.69236145438735\\
4.12276901302913	2.6592072545807\\
4.14003004327764	2.62437001620671\\
4.15544337087205	2.58755522164173\\
4.16898111242984	2.54841200325468\\
4.1805501005934	2.50652038741957\\
4.189980298719	2.46137446315747\\
4.19700713642881	2.41236022919505\\
4.20124560249939	2.35872638667083\\
4.20215284043759	2.29954569137659\\
4.19897438473111	2.23366361187286\\
4.19066679605515	2.15962992618775\\
4.17578595253724	2.07560756621235\\
4.15232521745815	1.97925172399264\\
4.11748082877196	1.86755173387132\\
4.0673135628575	1.73663058035474\\
3.99626893137781	1.58150700167523\\
3.89652289800053	1.39585416370095\\
3.75716409757238	1.17185931374062\\
3.56337212632166	0.900438705505084\\
3.29612665441547	0.572327965997373\\
2.93370576079621	0.180882711667256\\
2.45705472102703	-0.272674545189229\\
1.86050778278718	-0.772482316597883\\
1.16451685236404	-1.2845483777012\\
0.419258436400468	-1.76359479187261\\
-0.310451065573799	-2.17009581352747\\
-0.970512410976555	-2.48499188487467\\
-1.53304219695017	-2.71091402645631\\
-1.99465023348299	-2.86314635273768\\
-2.36617534434433	-2.96004404359145\\
-2.6633874611004	-3.01771046191725\\
-2.90173787310418	-3.04841749658235\\
-3.09431332561097	-3.06087319520195\\
-3.25147897947426	-3.06103735904053\\
-3.38119118486251	-3.05291928331043\\
-3.48947990122615	-3.03919651729106\\
-3.58090349168907	-3.02165201979975\\
-3.65891865332526	-3.00147015489306\\
-3.72616295980993	-2.97943339228997\\
-3.78466451464927	-2.95605216745467\\
-3.83599561684905	-2.9316505741551\\
-3.88138499309718	-2.90642301715518\\
-3.92179985990072	-2.88047171211381\\
-3.95800612186526	-2.85383144118423\\
-3.99061268873467	-2.82648570719649\\
-4.02010416898246	-2.7983769604186\\
-4.04686494837156	-2.76941261446765\\
-4.07119676264246	-2.73946793465286\\
-4.09333122120634	-2.70838645170393\\
-4.11343825448174	-2.67597824779936\\
-4.13163108182871	-2.64201622778269\\
-4.14796798375716	-2.60623028989809\\
-4.16245087258137	-2.56829911936056\\
-4.17502035336242	-2.52783912031028\\
-4.18554661162926	-2.48438975275164\\
-4.19381500561436	-2.43739422323724\\
-4.19950460942144	-2.38617405770586\\
-4.20215704988444	-2.32989552089704\\
-4.20113165723341	-2.26752509195969\\
-4.1955409938594	-2.19777021687624\\
-4.18415793433676	-2.11900032717899\\
-4.16528125285283	-2.02914175339791\\
-4.13654073378355	-1.92553911355845\\
-4.0946150919557	-1.8047762910793\\
-4.03482779602556	-1.66245544209737\\
-3.9505828429046	-1.4929500758959\\
-3.83262165276153	-1.28919420220208\\
-3.66816704733271	-1.04267411268889\\
-3.44026358955362	-0.743995026519903\\
-3.12816604364904	-0.384708680621449\\
-2.71048582054154	0.0387050776214842\\
-2.17322418619597	0.518512363739549\\
-1.52242222011607	1.02971849467443\\
-0.794040529230507	1.53102853372387\\
-0.0485561944193281	1.97767233534558\\
0.651596897764944	2.33925712839912\\
1.26464629340802	2.60828678671132\\
1.77598975688712	2.79508091712544\\
2.19072196670672	2.91742086361568\\
2.52306177366165	2.99292260089135\\
2.78903776233008	3.03581499533663\\
3.00303791269362	3.05649995546078\\
3.17677417039348	3.06220363740278\\
};
\addplot [color=mycolor2, forget plot]
  table[row sep=crcr]{%
0.661273782960425	0.866962186619865\\
1.15796205284672	0.851693038651411\\
1.55385438750588	0.814423660462159\\
1.85381732281985	0.766885959235898\\
2.07617283338148	0.717021594888785\\
2.24051638617859	0.669030895999442\\
2.36292503570404	0.624672111410093\\
2.45527929738384	0.584396976084643\\
2.52599281815815	0.548048739106013\\
2.5809449574233	0.515230268539075\\
2.62424905343567	0.485480899947359\\
2.65880723615788	0.458354204400774\\
2.68669128762984	0.433447573305808\\
2.70939944994863	0.410409558737287\\
2.72802917122473	0.388937630991543\\
2.74339392644928	0.368772335789921\\
2.75610293693549	0.349690567260384\\
2.76661614353836	0.331499105872901\\
2.77528250747898	0.314028834342263\\
2.78236692940358	0.297129709844776\\
2.78806927369543	0.280666427697226\\
2.79253780974802	0.264514657597637\\
2.79587860673463	0.248557718915537\\
2.79816189726337	0.232683562264001\\
2.79942606438194	0.216781929025155\\
2.79967964434784	0.200741562886482\\
2.79890153260681	0.184447344700957\\
2.79703940171752	0.167777212056612\\
2.79400616164061	0.150598705922155\\
2.78967408895369	0.132764956360156\\
2.78386599120974	0.114109874654288\\
2.77634241366005	0.0944422566234181\\
2.76678337573911	0.0735384172667591\\
2.75476234844049	0.0511328667796173\\
2.73970900106703	0.0269064030239178\\
2.72085542073849	0.000470848901667607\\
2.69715767306884	-0.028650448966868\\
2.66718017648142	-0.0610471935020732\\
2.62892365232438	-0.0974520813058362\\
2.57956760595732	-0.138774628197668\\
2.51508546665734	-0.186132564282447\\
2.42967877912178	-0.240862688463524\\
2.31498413577605	-0.304466200882717\\
2.15909101178252	-0.378387607580392\\
1.94572497836975	-0.463428493655018\\
1.65480195610755	-0.558496052466978\\
1.26711459734576	-0.658537049264833\\
0.77657514989749	-0.752577661258321\\
0.207369162339691	-0.824951408390095\\
-0.381574349653424	-0.862479910369312\\
-0.921511686533697	-0.862890828216087\\
-1.36871821476444	-0.835005611462908\\
-1.71481278672137	-0.791341871441737\\
-1.97341590826614	-0.741908687426995\\
-2.16448928144107	-0.692640408697396\\
-2.30612758268735	-0.646348835619765\\
-2.41226322272021	-0.604025166399864\\
-2.49292438342471	-0.565753910039911\\
-2.55514806487159	-0.531226821686693\\
-2.60384842392396	-0.500000681614672\\
-2.64247594030227	-0.471616062174229\\
-2.67347880035625	-0.445646546514242\\
-2.69861595862575	-0.421714983761723\\
-2.71916769755915	-0.399494981073737\\
-2.73607754777726	-0.378706371822785\\
-2.75004866970255	-0.359108715867074\\
-2.76160996555653	-0.340494620357544\\
-2.77116190922859	-0.322683592178605\\
-2.77900862574165	-0.305516634102342\\
-2.78538051207993	-0.288851575556715\\
-2.7904502368164	-0.272559038920184\\
-2.7943440032954	-0.25651891211297\\
-2.79714932739429	-0.240617193347299\\
-2.79892014917985	-0.224743077476029\\
-2.79967979249433	-0.20878615729131\\
-2.79942205736934	-0.192633613239204\\
-2.79811054174722	-0.176167258913038\\
-2.79567611360316	-0.15926029546696\\
-2.79201226690305	-0.141773603712912\\
-2.78696786713369	-0.123551365565637\\
-2.7803364887445	-0.104415753390757\\
-2.7718411169053	-0.0841603527255041\\
-2.7611123526863	-0.0625418867677141\\
-2.74765730453616	-0.0392696879289773\\
-2.73081488156519	-0.0139922179360294\\
-2.70969093033265	0.0137202006216811\\
-2.68306312346527	0.0443973335648985\\
-2.64924005970246	0.0786963249046509\\
-2.60585085623834	0.117433966448259\\
-2.54953005584578	0.161620602371638\\
-2.47544936887848	0.212484709782595\\
-2.37664170939938	0.271459228033626\\
-2.24309728766334	0.34006138581561\\
-2.06078357220008	0.419521432230619\\
-1.81127955202615	0.509903493366639\\
-1.47394379994429	0.608430286702134\\
-1.03407711740967	0.707231481400013\\
-0.498790464781053	0.792473568594955\\
0.0891670364609763	0.848526699392398\\
0.661273782960423	0.866962186619865\\
};
\addplot [color=mycolor1, forget plot]
  table[row sep=crcr]{%
2.35723974553073	2.84416812417343\\
2.51069738272667	2.83942105928315\\
2.64251651332025	2.82693423750504\\
2.75649096143632	2.80877852846073\\
2.85571591958509	2.78643416026401\\
2.94269893019672	2.76094838995348\\
3.0194661373763	2.73305269663057\\
3.08765471098761	2.70324792118058\\
3.14858937420538	2.6718652646589\\
3.20334402719018	2.63910956301037\\
3.25279054010171	2.60508969437066\\
3.29763695043819	2.56983965516583\\
3.33845707939824	2.53333281809185\\
3.37571323580098	2.49549112362842\\
3.40977331694871	2.45619039921715\\
3.44092328605599	2.4152625915511\\
3.46937571450358	2.37249539254628\\
3.49527481797884	2.32762950385245\\
3.51869817541443	2.28035359214749\\
3.53965508182247	2.23029681840946\\
3.55808123108122	2.17701866432022\\
3.57382912947303	2.11999561772275\\
3.58665327772499	2.05860411151641\\
3.5961886953016	1.99209893948848\\
3.60192075740274	1.91958621630789\\
3.60314353362642	1.83998985438197\\
3.59890283311901	1.75201060116323\\
3.5879190029878	1.65407712794404\\
3.56848336662896	1.54428989508587\\
3.53832153606045	1.42036130172444\\
3.49441792939479	1.27956130976565\\
3.43280137970586	1.11868852519635\\
3.34830708785551	0.934105777173234\\
3.23436466491018	0.72190970105947\\
3.08292848391465	0.478344752388961\\
2.88477324192349	0.200607294019178\\
2.63049975041719	-0.111837186200972\\
2.31261787611998	-0.455485174088717\\
1.92874622554254	-0.82108653917307\\
1.48506617658905	-1.1931092075257\\
0.998035770061693	-1.55162561181268\\
0.492319908450856	-1.87674651350774\\
-0.00500095663271922	-2.15371625030846\\
-0.47105533379721	-2.37591157248377\\
-0.891240774070467	-2.54450272708476\\
-1.25958652926944	-2.66583184732752\\
-1.57666258567759	-2.74841093472104\\
-1.84686814876165	-2.80074834040904\\
-2.07622761179615	-2.83023154463703\\
-2.27099612042322	-2.84278135592224\\
-2.43694060189584	-2.84291981785256\\
-2.57905701509396	-2.83399968754863\\
-2.70152578587721	-2.81846070138456\\
-2.80777793317631	-2.79805587627732\\
-2.90060091198471	-2.7740319654112\\
-2.98224904339135	-2.74726589081205\\
-3.05454342556257	-2.71836484841287\\
-3.11895641284093	-2.68773847340791\\
-3.1766804691153	-2.65565027605371\\
-3.22868309348335	-2.62225396515838\\
-3.27575004481665	-2.58761881845915\\
-3.31851901725403	-2.55174708859276\\
-3.35750561464144	-2.51458554782193\\
-3.39312311075937	-2.47603262256475\\
-3.42569713509985	-2.43594209198227\\
-3.45547611349754	-2.39412397304877\\
-3.48263801896802	-2.35034294797793\\
-3.50729374033659	-2.30431447852061\\
-3.52948713952374	-2.25569857292167\\
-3.54919162456181	-2.20409100832365\\
-3.56630279339919	-2.14901165176407\\
-3.58062637817612	-2.08988935876831\\
-3.59186031064004	-2.02604275824673\\
-3.599569200735	-1.95665606520587\\
-3.60314883250387	-1.88074892881337\\
-3.60177739968447	-1.79713929342847\\
-3.59434912274173	-1.70439847331892\\
-3.57938469372077	-1.60079841368833\\
-3.55491199194827	-1.48425299354551\\
-3.51831051117761	-1.35225924191538\\
-3.46611579371748	-1.20185223707173\\
-3.39378973718194	-1.02960194588312\\
-3.295486126987	-0.831704729697735\\
-3.16388956206597	-0.604258631528481\\
-2.99029254310449	-0.343853312370394\\
-2.76519701600118	-0.0486215255346563\\
-2.4798214882969	0.280188756232953\\
-2.12877973569313	0.636373170420278\\
-1.71359192911296	1.00747964391791\\
-1.24556624327728	1.37536326684738\\
-0.745794375128828	1.71945418960585\\
-0.241015860211289	2.0218665981489\\
0.243063883633684	2.27174582229565\\
0.687439762821932	2.46659503015395\\
1.08197874485966	2.6105629683591\\
1.42431832352954	2.71141548526532\\
1.7172655368299	2.77786498936206\\
1.96625697011267	2.81794182510089\\
2.17755897671675	2.83831004944716\\
2.35723974553073	2.84416812417343\\
};
\addplot [color=mycolor2, forget plot]
  table[row sep=crcr]{%
1.18991530942429	0.996510309110826\\
1.60054088847078	0.984021430456015\\
1.91377244156376	0.954577655347213\\
2.14791203027807	0.917473554270012\\
2.32251535465373	0.878305494358859\\
2.45373375878422	0.839972813807476\\
2.55360600181979	0.803767192939426\\
2.63073362704449	0.77012093230025\\
2.69117823317898	0.739041899974995\\
2.73921392535886	0.710346566448502\\
2.77787795324581	0.683778702006049\\
2.80935286990539	0.659067073284671\\
2.83522664770944	0.635951634127522\\
2.85666909055672	0.614193833395493\\
2.874552002102	0.593579139214968\\
2.8895317218807	0.573915920283808\\
2.90210637368735	0.555032767841512\\
2.91265596359902	0.536775278792383\\
2.92147069986961	0.51900277283484\\
2.92877109843297	0.501585135454873\\
2.93472224979762	0.484399835099438\\
2.93944383389818	0.467329089405828\\
2.94301693519272	0.450257116591205\\
2.94548833765155	0.43306738503193\\
2.94687270661257	0.415639756143704\\
2.94715284928651	0.397847396552401\\
2.94627805622044	0.379553310728468\\
2.94416033513793	0.360606310769947\\
2.94066812865059	0.340836191532789\\
2.93561682499492	0.320047811504501\\
2.92875498041601	0.298013685916316\\
2.91974460543797	0.274464570164272\\
2.90813302103366	0.24907733921039\\
2.89331250200592	0.221459244560428\\
2.87446193899167	0.191127356986687\\
2.85046167457741	0.157481714562739\\
2.81976792364993	0.119770512339495\\
2.78022600914904	0.0770459308095315\\
2.72879132806117	0.0281107725380253\\
2.66111397018444	-0.0285389374000364\\
2.57093264307373	-0.0947569876766271\\
2.44923757809347	-0.17272541723286\\
2.28326719813619	-0.264748920737332\\
2.05576903546383	-0.372618964547878\\
1.74588642840482	-0.496154129648031\\
1.33459016541791	-0.630655150650412\\
0.817792878440208	-0.764236171010543\\
0.223210775629237	-0.878588321361184\\
-0.387488105528008	-0.956662577001607\\
-0.94544396218029	-0.992562807753562\\
-1.40823102692931	-0.993115571787897\\
-1.7683138275155	-0.970744942497015\\
-2.03945352735151	-0.936555785635127\\
-2.24154691063925	-0.89791728601457\\
-2.39270406004862	-0.858926887362741\\
-2.50698083273759	-0.821563309124965\\
-2.59458434079788	-0.786617165495715\\
-2.66273967323147	-0.754269284531161\\
-2.71653355994551	-0.724411464036381\\
-2.75956449421103	-0.696813713045393\\
-2.79440326481326	-0.671207546223913\\
-2.8229086170185	-0.64732529779844\\
-2.84644142929244	-0.62491699822928\\
-2.86601018748154	-0.603756084633368\\
-2.88237045822303	-0.58363974573412\\
-2.8960935416649	-0.564386845692692\\
-2.90761432763437	-0.545834891706278\\
-2.91726496392581	-0.527836746921811\\
-2.92529870926235	-0.510257397710516\\
-2.93190687920515	-0.49297088342211\\
-2.93723082678732	-0.475857394184777\\
-2.94137025218947	-0.4588004893601\\
-2.94438869164446	-0.441684360051673\\
-2.94631671937614	-0.424391039555058\\
-2.94715315692405	-0.406797447755458\\
-2.94686438542758	-0.388772134015282\\
-2.94538166956615	-0.370171553943397\\
-2.94259620012493	-0.350835674540475\\
-2.93835131578995	-0.330582644817653\\
-2.93243103516107	-0.309202189085831\\
-2.92454356182793	-0.286447270097373\\
-2.91429773507599	-0.262023419987743\\
-2.90116935647461	-0.235574939359954\\
-2.88445272456195	-0.206666914582644\\
-2.86319023870879	-0.174761714560406\\
-2.83606910795067	-0.13918836809934\\
-2.80126834127463	-0.099103196869256\\
-2.75623050704679	-0.0534408333077383\\
-2.69732087016683	-0.000857646209580513\\
-2.61932360436197	0.0603221566113784\\
-2.51472211999705	0.132126713227889\\
-2.37275701371166	0.21684430587183\\
-2.17846019986387	0.316643567600139\\
-1.9124699695339	0.43258002030732\\
-1.55372938073449	0.562588128013722\\
-1.08855103789696	0.69860911923824\\
-0.526985364418182	0.825043987363956\\
0.0847239315437372	0.922867884974586\\
0.676587249516098	0.979694143100387\\
1.18991530942428	0.996510309110826\\
};
\addplot [color=mycolor1, forget plot]
  table[row sep=crcr]{%
1.84790736958456	2.85539624508498\\
2.00892487226021	2.85038734938052\\
2.15161460587838	2.83684839195735\\
2.27847365068873	2.81662215054955\\
2.39169364626266	2.79111153615171\\
2.49316493440788	2.76136860581943\\
2.58449971813475	2.72816920137418\\
2.66706285803228	2.69207280870998\\
2.7420039377821	2.65346917087495\\
2.81028731684173	2.61261379962829\\
2.87271868648525	2.56965450771895\\
2.92996763362065	2.52465080306163\\
2.98258621771325	2.47758763347676\\
3.03102378366391	2.42838463079329\\
3.07563829063698	2.37690170719065\\
3.1167044047097	2.32294161098691\\
3.15441852079681	2.2662498495781\\
3.18890076607437	2.20651222680153\\
3.22019390087017	2.14335011429239\\
3.24825887456952	2.07631347802289\\
3.27296661059502	2.00487161356855\\
3.29408538192824	1.9284015154322\\
3.311262895117	1.84617383684868\\
3.32400193215177	1.7573365242676\\
3.33162812979233	1.66089649997351\\
3.33324826531379	1.55570032408616\\
3.32769739679072	1.44041576290418\\
3.31347363266625	1.31351787885098\\
3.28866065417024	1.1732859918368\\
3.25084120056975	1.01782206668806\\
3.19701083076644	0.84510710050348\\
3.12351212828304	0.653119751975032\\
3.02602680639661	0.440049138668592\\
2.89968682572986	0.204636619901509\\
2.73938992651037	-0.0533306138224429\\
2.54041231199207	-0.332393904670119\\
2.29936787692003	-0.628772510767444\\
2.01543038031951	-0.935917839041958\\
1.6915119616486	-1.24459598152934\\
1.33488281903928	-1.54376444301386\\
0.956733747481965	-1.8222114167435\\
0.570572069272705	-2.07049238790772\\
0.189937153897144	-2.28244672825619\\
-0.173716559496603	-2.45575876703295\\
-0.512282604500782	-2.5915247044428\\
-0.821236103988375	-2.6932132753532\\
-1.09911022893855	-2.76551271300529\\
-1.34663993521902	-2.81339851505886\\
-1.56590874874089	-2.84153589706066\\
-1.7596689634603	-2.85398154078695\\
-1.93087453770099	-2.85409308307999\\
-2.08239763586714	-2.84455751625081\\
-2.21687874717862	-2.82747429843868\\
-2.33666353451119	-2.80445436217213\\
-2.44379094050302	-2.77671491761417\\
-2.54000866850522	-2.74516153003493\\
-2.62680121605391	-2.71045525623359\\
-2.70542187198806	-2.67306562981485\\
-2.77692404965429	-2.63331144754917\\
-2.84218970045887	-2.59139154037586\\
-2.9019538958854	-2.54740752817672\\
-2.95682537871724	-2.50138022522718\\
-3.00730322378136	-2.45326101069311\\
-3.05378987429397	-2.40293915875533\\
-3.09660082588792	-2.35024585224454\\
-3.13597116970703	-2.2949553819004\\
-3.17205910656379	-2.23678385430369\\
-3.20494641896013	-2.17538558811419\\
-3.23463574062408	-2.11034726537688\\
-3.26104429275771	-2.04117982108518\\
-3.28399355861275	-1.96730800458932\\
-3.30319413986797	-1.88805754448315\\
-3.31822478069885	-1.80263992193353\\
-3.32850427156098	-1.71013495560625\\
-3.33325469357533	-1.6094718081165\\
-3.33145432608481	-1.4994097747024\\
-3.32177869960428	-1.37852151887596\\
-3.30252908668592	-1.24518358106745\\
-3.27154981225758	-1.09758240218248\\
-3.22614016820033	-0.933749189986639\\
-3.16297496042255	-0.751643881709563\\
-3.07806162632474	-0.549316485789908\\
-2.96678257255502	-0.325180258718806\\
-2.82409665235852	-0.0784282751791394\\
-2.64499270725013	0.190400163799164\\
-2.42527520518298	0.478760764606481\\
-2.16267778572755	0.781519542224893\\
-1.85811745768085	1.09072924063741\\
-1.51666503575119	1.39608862323249\\
-1.14768296517624	1.68623139338685\\
-0.763781266444752	1.95059658468864\\
-0.378778991262895	2.18124436867327\\
-0.00539508952684124	2.37393399176962\\
0.346498809368338	2.52815978543668\\
0.670614869027265	2.64635173309969\\
0.964053154865023	2.73272374538172\\
1.22656050661286	2.79220371918969\\
1.45964412621222	2.82966611705173\\
1.66579343430517	2.84949388873804\\
1.84790736958455	2.85539624508498\\
};
\addplot [color=mycolor2, forget plot]
  table[row sep=crcr]{%
1.50442737389697	1.14866712477609\\
1.84486918014159	1.13833029946581\\
2.10448822423007	1.11391457163613\\
2.3011971502059	1.08272290206621\\
2.45090974726946	1.04912002215542\\
2.56603302766523	1.01547382385565\\
2.65570683287098	0.982953137479371\\
2.72651451942016	0.952054315709736\\
2.78317393134374	0.922913951356243\\
2.82907939203983	0.895485069268227\\
2.86669388594656	0.869633380781392\\
2.89782359894184	0.845188565340106\\
2.9238079099883	0.821970740313312\\
2.94565078800753	0.799803392836786\\
2.96411211305387	0.778518992555047\\
2.97977162374811	0.757960692973232\\
2.99307406546933	0.737981977689385\\
3.00436130164132	0.718445251619882\\
3.01389526417741	0.699219902810375\\
3.02187435680076	0.680180095175765\\
3.02844507448732	0.661202401810001\\
3.03371002059801	0.642163299896373\\
3.03773309491289	0.6229364922162\\
3.04054232644606	0.603389979670131\\
3.04213058976255	0.583382773532295\\
3.04245423894509	0.562761098074081\\
3.0414294909429	0.541353887770826\\
3.03892616141411	0.518967322661358\\
3.03475806787604	0.495378063865181\\
3.02866902212632	0.470324740603633\\
3.02031277145188	0.443497090399019\\
3.00922441801632	0.414521954395087\\
2.99477959777149	0.382945070774552\\
2.97613580484955	0.348207292672608\\
2.95214736276928	0.309613518652939\\
2.92124119161162	0.266292391618661\\
2.88123413210468	0.217145053754499\\
2.8290638188195	0.160782847536577\\
2.760394989693	0.0954589923200107\\
2.66905771559782	0.0190128403057332\\
2.54629428349928	-0.0711224631742372\\
2.37989684982899	-0.177729197414635\\
2.15364768360016	-0.303187106141744\\
1.84825198609768	-0.448038536550649\\
1.44615947755651	-0.60844406975839\\
0.942723826257913	-0.773286400684488\\
0.360664582674148	-0.924044527718796\\
-0.246475983542887	-1.04114576216124\\
-0.814639713355748	-1.11403929809116\\
-1.29900475074848	-1.1453421728842\\
-1.6858359904969	-1.14584998396187\\
-1.98365383877806	-1.12734683822528\\
-2.2096341597025	-1.09883498124102\\
-2.38106886874338	-1.06603877517639\\
-2.51215503156031	-1.03220860298567\\
-2.61358846378269	-0.999030512193673\\
-2.69313749782962	-0.967286591441651\\
-2.75637485266642	-0.937264373264917\\
-2.80729884540031	-0.908992680864494\\
-2.8487972968282	-0.882372273200659\\
-2.88297646159004	-0.857246331048154\\
-2.91138955008335	-0.83343748109554\\
-2.93519467329699	-0.810766471372551\\
-2.95526429330422	-0.789060880396075\\
-2.9722615635818	-0.768158466556946\\
-2.98669400668361	-0.747907675110618\\
-2.99895155926237	-0.728166667978184\\
-3.00933370906833	-0.708801604667799\\
-3.01806890596801	-0.68968454834918\\
-3.02532839426926	-0.670691171741342\\
-3.03123591202487	-0.651698322615945\\
-3.0358742176301	-0.632581439134721\\
-3.03928905678174	-0.61321175856261\\
-3.04149092036105	-0.593453225817017\\
-3.04245472768803	-0.573158972216394\\
-3.04211736953351	-0.552167193253061\\
-3.04037283349348	-0.530296201450184\\
-3.0370643805986	-0.507338360176015\\
-3.03197290808334	-0.483052509307378\\
-3.0248001654215	-0.45715436481122\\
-3.01514480943405	-0.429304200989976\\
-3.00246826818674	-0.399090895578792\\
-2.98604584657951	-0.366011128581075\\
-2.96489616756454	-0.329442190017435\\
-2.9376784926671	-0.288606540810655\\
-2.90254216532864	-0.242526202439522\\
-2.85690484271088	-0.189965799366191\\
-2.79712643370924	-0.12936605430645\\
-2.71803656827352	-0.0587780226198686\\
-2.6122765961754	0.0241704844411688\\
-2.4694686152808	0.122197004054777\\
-2.27541753962575	0.238000463103092\\
-2.01207771082531	0.373289058075866\\
-1.66006143102537	0.526794996331922\\
-1.20647397080343	0.69133291546151\\
-0.658778300565647	0.851746047432427\\
-0.0560913168613796	0.987773404311966\\
0.53895165990967	1.08323818102995\\
1.06884545129749	1.13431195086951\\
1.50442737389697	1.14866712477609\\
};
\addplot [color=mycolor1, forget plot]
  table[row sep=crcr]{%
1.51239328471355	2.99187321897874\\
1.67867446084939	2.98667759938608\\
1.82979360645117	2.9723192337742\\
1.96732600132283	2.95037450977582\\
2.09274479981551	2.92210090208251\\
2.20739186151525	2.88848350304977\\
2.31246699395216	2.85027875729118\\
2.40902834647901	2.80805283095023\\
2.49799897887384	2.76221379370481\\
2.58017633617323	2.71303772572031\\
2.65624255967014	2.66068929612474\\
2.72677436488139	2.60523750766455\\
2.79225172868971	2.54666729889171\\
2.85306493714009	2.48488762098636\\
2.909519716944	2.41973650692446\\
2.96184025249874	2.35098355132742\\
3.01016990623873	2.27833013390277\\
3.05456943315731	2.20140765669543\\
3.09501242365782	2.11977403322802\\
3.13137763273854	2.03290867596972\\
3.16343776850579	1.940206292353\\
3.19084423431321	1.84096994158089\\
3.21310727192629	1.73440405875188\\
3.22957098227844	1.61960856793017\\
3.23938288106664	1.49557584701932\\
3.24145810240536	1.36119325479827\\
3.23443928798276	1.21525526659365\\
3.21665487213498	1.05649103857775\\
3.1860812605918	0.883615367048722\\
3.14031868860227	0.69541319788971\\
3.07659653554911	0.490869227322065\\
2.99183113068201	0.269353035253762\\
2.88276573014075	0.0308638592568937\\
2.74622404177374	-0.223675933520155\\
2.57949812677245	-0.492115900945798\\
2.38086022397749	-0.77083354274097\\
2.15013257224817	-1.05465556042521\\
1.88918173053705	-1.33705294216358\\
1.60215899641311	-1.61066869021488\\
1.29533267398154	-1.86812846675396\\
0.976472892525725	-2.10295720056563\\
0.653915809529938	-2.31035347869866\\
0.335560089471391	-2.48761382957053\\
0.0280584572646769	-2.63413417386167\\
-0.263636859153305	-2.75106554317231\\
-0.536360822565975	-2.84078738522603\\
-0.78854748650978	-2.9063621604053\\
-1.01989715152375	-2.95108008829427\\
-1.2310105092283	-2.97813745863803\\
-1.42306013077515	-2.99044419746322\\
-1.59753024291552	-2.9905329869835\\
-1.75602894663267	-2.98053721568007\\
-1.90016322894113	-2.96220964925042\\
-2.03146246118451	-2.93696139900913\\
-2.15133644489144	-2.90590801895207\\
-2.26105652077574	-2.8699151112142\\
-2.36175113824257	-2.82963956649311\\
-2.45440983880741	-2.78556486918132\\
-2.53989160079024	-2.73803019190504\\
-2.61893493296653	-2.68725365585423\\
-2.69216809009164	-2.63335040332824\\
-2.76011842697453	-2.57634618838206\\
-2.82322030867127	-2.51618714518441\\
-2.88182122806917	-2.45274630271093\\
-2.93618590272331	-2.38582731274327\\
-2.9864981671721	-2.31516576443959\\
-3.03286046949205	-2.24042838372721\\
-3.07529073763118	-2.16121036730898\\
-3.11371631342569	-2.07703108769674\\
-3.14796457012237	-1.98732843946967\\
-3.17774974499562	-1.89145219638467\\
-3.20265545172952	-1.7886569420333\\
-3.22211232123201	-1.67809546434025\\
-3.23537031247882	-1.55881402314529\\
-3.24146553402507	-1.42975168383362\\
-3.23918207764659	-1.2897470423625\\
-3.22701062328347	-1.13755722026301\\
-3.20310775495492	-0.971895988183196\\
-3.16526342101049	-0.791500110601493\\
-3.11088911190784	-0.595234941923725\\
-3.03704607498221	-0.382250722140419\\
-2.94054020210392	-0.152197712258686\\
-2.81811512770277	0.09450194458729\\
-2.66677171975208	0.356348286764419\\
-2.48422218721787	0.630477714379279\\
-2.26944337938398	0.912487333112226\\
-2.02322943415994	1.19648116206932\\
-1.74858204526264	1.475429817124\\
-1.45076091491646	1.74185422835084\\
-1.13688708903981	1.98871806337898\\
-0.815141623939223	2.21030456709899\\
-0.49376054385753	2.40283490282641\\
-0.180100480124219	2.56467997680183\\
0.120003514970815	2.69617191608695\\
0.40250210416808	2.79914631394991\\
0.665065380523587	2.87638840670827\\
0.906806359627473	2.93112342909832\\
1.12792007247506	2.96662628190493\\
1.32933028778223	2.98596700459762\\
1.51239328471354	2.99187321897874\\
};
\addplot [color=mycolor2, forget plot]
  table[row sep=crcr]{%
1.72274085757999	1.31026474427909\\
2.01026614301254	1.30152528012605\\
2.23201584329209	1.28065232276943\\
2.40312518212022	1.25350117177955\\
2.5361316963277	1.22363200205133\\
2.6406436810082	1.1930742780253\\
2.72377440734015	1.16291636519221\\
2.79072342258328	1.13369337333612\\
2.84528715314021	1.10562441056408\\
2.89025159926759	1.07875253892653\\
2.9276780567195	1.05302569164154\\
2.95910618091147	1.02834278534591\\
2.98569717105014	1.00457957025386\\
3.00833475323026	0.98160273795666\\
3.02769671139579	0.959277228056095\\
3.04430586543995	0.937469588196718\\
3.05856662245834	0.916049031872386\\
3.07079129744337	0.89488713481783\\
3.081219074456	0.87385669836927\\
3.09002956953075	0.85283006267768\\
3.09735232592094	0.831677002423555\\
3.10327312584447	0.810262240935\\
3.10783767485561	0.788442550430583\\
3.11105295868559	0.766063350952403\\
3.1128863527598	0.742954667906426\\
3.11326235223077	0.718926249874456\\
3.11205655667378	0.693761577131293\\
3.10908625587767	0.667210399324376\\
3.10409657886324	0.638979319139999\\
3.09674062726076	0.608719777182073\\
3.08655122798497	0.576012580850285\\
3.07290077666084	0.540347848747005\\
3.05494390693558	0.501098916829594\\
3.03153513242212	0.457488415542408\\
3.00110980286464	0.408544518113625\\
2.96151132350235	0.353045653896105\\
2.90974057504121	0.289453696806686\\
2.84159626443886	0.215840907962605\\
2.75117357532672	0.129829395360097\\
2.63021169159603	0.0285929215176733\\
2.46737755444351	-0.0909646386826638\\
2.24784969233456	-0.231626738984467\\
1.95417622824219	-0.394518792626553\\
1.57027438833949	-0.576708641133046\\
1.0904661840107	-0.768301080556169\\
0.531564905798336	-0.951569183923089\\
-0.0623418971493884	-1.10568563341469\\
-0.633617989988535	-1.2161021722935\\
-1.13605816033267	-1.28069638907403\\
-1.54939419097584	-1.30745770207334\\
-1.87572926586082	-1.30788835936333\\
-2.12836987817241	-1.29217629577013\\
-2.32304612299803	-1.26759473529956\\
-2.47372557267457	-1.23875160323609\\
-2.591453345156	-1.20835455614213\\
-2.68451842074422	-1.17790230524224\\
-2.75900664524931	-1.14816890818611\\
-2.81935945503074	-1.11950898553585\\
-2.8688261931858	-1.09204052968102\\
-2.90980054357077	-1.06575153233741\\
-2.94406181904487	-1.04056117627549\\
-2.97294553842738	-1.01635444607972\\
-2.99746363021922	-0.993001310143848\\
-3.01838936394926	-0.970366966950246\\
-3.03631768842475	-0.948316912478931\\
-3.05170836803424	-0.926718996332547\\
-3.06491698868893	-0.905443712048879\\
-3.07621730464331	-0.884363429157137\\
-3.08581730005135	-0.863350957130833\\
-3.09387058304999	-0.842277640219219\\
-3.10048420139264	-0.821011062591899\\
-3.10572358823947	-0.799412363166568\\
-3.10961505950969	-0.777333099373584\\
-3.11214605015043	-0.754611546325656\\
-3.11326306418744	-0.731068263346829\\
-3.11286709410248	-0.706500695997612\\
-3.11080600893108	-0.680676501158869\\
-3.10686308067396	-0.653325177135875\\
-3.10074036495277	-0.624127440491527\\
-3.09203500161607	-0.592701605607224\\
-3.08020554612993	-0.558585981669299\\
-3.06452402195011	-0.521216001401317\\
-3.04400726239405	-0.479894455357776\\
-3.01731796332176	-0.433752905048441\\
-2.9826213076251	-0.381702319818754\\
-2.93737677453569	-0.322371818288306\\
-2.87803728329398	-0.254037504262141\\
-2.79962226426849	-0.174551946279693\\
-2.69513849711839	-0.0813055615369774\\
-2.55487350140267	0.0287033971132761\\
-2.36575521837367	0.158525653958557\\
-2.11139719754735	0.310347034826338\\
-1.77424476755579	0.483627088348244\\
-1.34196332706271	0.672275560929426\\
-0.818726179635004	0.862306244954394\\
-0.23541171032341	1.03343880109232\\
0.354231974656627	1.1667926388925\\
0.895347405057608	1.25378950593271\\
1.35412989498003	1.298095646019\\
1.72274085757999	1.31026474427909\\
};
\addplot [color=mycolor1, forget plot]
  table[row sep=crcr]{%
1.28286346141528	3.19376085633426\\
1.45344992176763	3.18841316743345\\
1.61146233751843	3.17338414571857\\
1.75790426515861	3.15000377012599\\
1.89376274945308	3.11936411942584\\
2.01997584878284	3.08234393604552\\
2.13741318854943	3.03963409856881\\
2.24686552217705	2.99176163900756\\
2.34904020189213	2.93911108178861\\
2.44456026355682	2.88194260248964\\
2.53396546817392	2.82040693037988\\
2.61771412103994	2.75455715049177\\
2.69618483097944	2.68435767238165\\
2.76967760447452	2.60969067558258\\
2.83841381879587	2.53036035107147\\
2.90253470641246	2.44609525787348\\
2.96209802805208	2.35654912167942\\
3.01707262919853	2.26130043251898\\
3.06733057921585	2.15985126527585\\
3.11263660015073	2.05162586609647\\
3.15263452503901	1.93596973871599\\
3.18683061376612	1.8121502503604\\
3.21457374287693	1.67936018272564\\
3.23503283868411	1.53672620317594\\
3.24717252973688	1.38332493572243\\
3.24972896934425	1.21821014777914\\
3.24118924870029	1.04045544358652\\
3.21977988880917	0.849217546494955\\
3.1834725580871	0.643825333362289\\
3.13001814199517	0.423898551439897\\
3.05702282840415	0.189496614979834\\
2.96208046408279	-0.0587090367002419\\
2.84297176184823	-0.319256458036893\\
2.69793031738669	-0.589740078626076\\
2.52595624426856	-0.866727932818914\\
2.32713254218933	-1.1458000803693\\
2.10287500662487	-1.42174843885806\\
1.8560370038427	-1.68894570188325\\
1.59080879521382	-1.94184140203882\\
1.31239968144874	-2.17549463251815\\
1.02655471746895	-2.38602990610422\\
0.739008063001648	-2.57092037985953\\
0.454988272845025	-2.72905503920742\\
0.17886313749819	-2.86060861319044\\
-0.0860395711768357	-2.96677798302946\\
-0.33744884501812	-3.0494633517034\\
-0.574048963935673	-3.11096042749872\\
-0.795312543117972	-3.15370479378238\\
-1.00131324321407	-3.18008446073984\\
-1.19254879174875	-3.19231851186315\\
-1.36979058218974	-3.19239018933388\\
-1.53396526611143	-3.18201985703479\\
-1.68606707975293	-3.16266432210517\\
-1.82709638029388	-3.13553173401469\\
-1.95801885894119	-3.1016043038565\\
-2.07974014240425	-3.06166370936887\\
-2.19309129316794	-3.01631605259258\\
-2.29882165882822	-2.96601463753233\\
-2.3975963930293	-2.91107974953655\\
-2.48999669238132	-2.85171517738617\\
-2.57652134994271	-2.78802153705338\\
-2.65758863235173	-2.72000661991507\\
-2.73353777115685	-2.64759306006419\\
-2.80462954671893	-2.5706236378708\\
-2.87104555930817	-2.48886453900615\\
-2.93288584660048	-2.40200688999573\\
-2.99016453620524	-2.30966690859388\\
-3.04280323088502	-2.21138505388806\\
-3.09062182797518	-2.10662465187518\\
-3.13332649168657	-1.99477062473559\\
-3.17049455290271	-1.87512918724549\\
-3.20155624267972	-1.74692971627559\\
-3.22577342679063	-1.60933047455679\\
-3.24221597569766	-1.461430497007\\
-3.24973717941891	-1.30229072451559\\
-3.2469508233127	-1.13096834386344\\
-3.23221430086974	-0.946569114175136\\
-3.20362451552835	-0.748322915770596\\
-3.15903620107256	-0.535687288342782\\
-3.09611518507011	-0.308481479443459\\
-3.01244092568205	-0.0670484250269967\\
-2.90567144297647	0.187566820065364\\
-2.77377693185499	0.453446422817519\\
-2.61533349956532	0.727675555342042\\
-2.42984546794961	1.00631404478284\\
-2.21803827904919	1.28451198369756\\
-1.98204528315884	1.55679599914171\\
-1.72541491064481	1.81751033362336\\
-1.45289895432384	2.06134485117164\\
-1.17004172724966	2.28384340951517\\
-0.882650474864446	2.48178268065626\\
-0.596261176391486	2.65334877084781\\
-0.315705610257031	2.79809996170941\\
-0.0448430163955496	2.91676025414382\\
0.213533748640937	3.01091864183941\\
0.457649300011542	3.08270894153138\\
0.686605126289815	3.13452461130596\\
0.900197669369818	3.16879654796175\\
1.09873331760935	3.18783973202483\\
1.28286346141527	3.19376085633426\\
};
\addplot [color=mycolor2, forget plot]
  table[row sep=crcr]{%
1.89181837155654	1.47713763134783\\
2.13771553571714	1.46964939389994\\
2.33001012910751	1.45153211491801\\
2.48106192932662	1.42754820119153\\
2.60075866889671	1.40065512434169\\
2.69663740741938	1.3726112403047\\
2.77432332374972	1.34442024213581\\
2.83798709737352	1.31662450572021\\
2.89072519217482	1.28948916501309\\
2.93484951396149	1.26311474598279\\
2.97209959965753	1.23750518193163\\
3.00379547743106	1.21260852999437\\
3.03094720294237	1.18834111188663\\
3.05433340731742	1.16460159315882\\
3.07455783053367	1.14127892782587\\
3.0920901946036	1.11825652862231\\
3.10729586029809	1.0954140776154\\
3.12045735698308	1.07262781857304\\
3.13178992463875	1.04976982064082\\
3.141452537821	1.02670648217422\\
3.14955540266705	1.00329639964764\\
3.15616456487969	0.979387626220089\\
3.1613039916535	0.95481426681401\\
3.16495525780851	0.929392287372161\\
3.16705474608268	0.902914344806229\\
3.16748803510655	0.875143362056503\\
3.16608086437666	0.845804470789074\\
3.16258569349412	0.814574812976593\\
3.15666235788526	0.781070521381554\\
3.14785058585457	0.744829977260624\\
3.1355310666242	0.705292164790822\\
3.1188701800061	0.661768613519819\\
3.09674118814482	0.61340709041739\\
3.06761137190888	0.559145019828183\\
3.02938003899483	0.49765095619948\\
2.9791466974858	0.427254225025101\\
2.91288354275194	0.345868141978941\\
2.82498741508678	0.250925399102838\\
2.70770978260807	0.139373802808658\\
2.55054904516899	0.00784046109987624\\
2.33991897174694	-0.146824509493805\\
2.05988981018934	-0.326293957760779\\
1.69548212177068	-0.528509603760869\\
1.24004416179817	-0.744808228464494\\
0.70548881917923	-0.958493798790827\\
0.12783675969007	-1.14817471608847\\
-0.4416032325795	-1.29616971098341\\
-0.956740550574267	-1.39588042462719\\
-1.39228189895034	-1.45193873083072\\
-1.74441246300098	-1.47475135554755\\
-2.02232349506424	-1.47510894977225\\
-2.23973581762482	-1.46157145103386\\
-2.41001416968227	-1.44005415271611\\
-2.54430815062523	-1.41433334477799\\
-2.65128453981933	-1.38670069250552\\
-2.73746462382523	-1.35849200673027\\
-2.80769308911154	-1.33045148904568\\
-2.86556179250319	-1.30296510910786\\
-2.91374397688086	-1.27620490925257\\
-2.95424296342479	-1.25021663838917\\
-2.98857246774732	-1.22497243047476\\
-3.01788602285498	-1.20040220232462\\
-3.04306970540758	-1.17641214094034\\
-3.064808736466	-1.15289534086714\\
-3.08363552486989	-1.12973763674608\\
-3.09996447248869	-1.10682045932657\\
-3.11411724857011	-1.08402180768201\\
-3.12634110550271	-1.06121598213717\\
-3.1368220116024	-1.03827244474618\\
-3.14569381209809	-1.01505399705126\\
-3.15304422044083	-0.99141434594641\\
-3.15891813305506	-0.967195041532572\\
-3.16331851081537	-0.942221698869746\\
-3.16620484757075	-0.916299346538142\\
-3.16748902131662	-0.889206669386819\\
-3.16702806764512	-0.860688821992978\\
-3.16461309274635	-0.830448374032041\\
-3.15995310784245	-0.798133798868254\\
-3.15265195258697	-0.763324721426685\\
-3.14217558593179	-0.725512891445621\\
-3.12780572030924	-0.684077542100717\\
-3.10857386267558	-0.638253454802491\\
-3.08316704752093	-0.587089765742521\\
-3.04979262753941	-0.529397559975725\\
-3.0059843219301	-0.463685201190143\\
-2.94832597450879	-0.388083521184075\\
-2.87206625263878	-0.300271471378458\\
-2.77060661321309	-0.197432821785875\\
-2.63489222674009	-0.076317237101075\\
-2.4528799873791	0.0664401824931821\\
-2.20960200429706	0.233466482055127\\
-1.88896077904493	0.424913341965717\\
-1.47894830289929	0.635762259080663\\
-0.980951413796855	0.853286954305538\\
-0.41898401443042	1.05763974617386\\
0.161163149450842	1.22809747605534\\
0.708107111546855	1.35198310850992\\
1.18512109293991	1.42877601430675\\
1.57838917173825	1.46679056095548\\
1.89181837155654	1.47713763134783\\
};
\addplot [color=mycolor1, forget plot]
  table[row sep=crcr]{%
1.12337085347325	3.41944784996473\\
1.29798732958167	3.41396102716276\\
1.46196601803898	3.39835266926702\\
1.61598180314091	3.37375209585985\\
1.76072271522622	3.34109904968817\\
1.89686168659934	3.30115800815168\\
2.02503714076216	3.25453409588748\\
2.14584007538346	3.20168882359674\\
2.25980570258933	3.14295460284063\\
2.36740810261529	3.07854748545296\\
2.46905668700011	3.00857790509443\\
2.56509354175359	2.93305940788129\\
2.65579093071233	2.85191549030554\\
2.74134839309963	2.76498474793442\\
2.82188897843307	2.67202460225462\\
2.89745423864257	2.57271393484618\\
2.96799765449178	2.46665503396455\\
3.03337622540302	2.35337536315742\\
3.09334001561041	2.23232980883168\\
3.14751954699297	2.10290426749029\\
3.19541108883424	1.96442170598768\\
3.23636015534362	1.81615217714327\\
3.26954393191587	1.65732869446987\\
3.29395396865839	1.48717133646596\\
3.30838136349419	1.30492239406217\\
3.31140784697252	1.10989565819192\\
3.30140765839972	0.901542834665535\\
3.27656672623956	0.679539217503698\\
3.23492707925443	0.443888687781547\\
3.17446495544757	0.195044351376081\\
3.09320973934176	-0.0659645908731955\\
2.98940644267463	-0.337416511784161\\
2.86171601460507	-0.616815932541158\\
2.70943550282497	-0.900876370043321\\
2.53270626723075	-1.18559532876101\\
2.33266789204068	-1.46643785383327\\
2.1115144148106	-1.73862314181527\\
1.87242265834854	-1.99748102758807\\
1.61934908852423	-2.23882142629321\\
1.35672389053374	-2.45925068976833\\
1.08909666054117	-2.65637987051156\\
0.820797073047426	-2.82889748589837\\
0.555663663292206	-2.97651245966763\\
0.296870565317662	-3.09979918436642\\
0.0468560139137498	-3.19998884002555\\
-0.192663551417901	-3.27874919926749\\
-0.420619965366741	-3.33798418929181\\
-0.636487511273445	-3.3796707082659\\
-0.840167085322681	-3.40573820011202\\
-1.03187563778785	-3.41798833978931\\
-1.21204948593243	-3.41804793573803\\
-1.3812649278788	-3.40734686822138\\
-1.54017619937021	-3.38711338672114\\
-1.68946894170933	-3.35838042815215\\
-1.82982655532066	-3.32199816171223\\
-1.96190668838332	-3.27864937833709\\
-2.0863253406858	-3.22886548704013\\
-2.20364644186376	-3.17304173719398\\
-2.31437516862347	-3.11145089064738\\
-2.41895363492475	-3.04425497466228\\
-2.51775789681386	-2.97151501002056\\
-2.61109545485912	-2.89319877443323\\
-2.6992026176303	-2.8091867661675\\
-2.78224121963233	-2.7192766046861\\
-2.86029427845936	-2.62318616576988\\
-2.9333602412369	-2.52055581556418\\
-3.00134552312201	-2.41095019643518\\
-3.0640550961729	-2.29386014169077\\
-3.12118096441606	-2.16870547008987\\
-3.17228848545515	-2.03483964792995\\
-3.21680070377289	-1.8915576168803\\
-3.25398118948045	-1.73810847252857\\
-3.28291638252849	-1.57371512814135\\
-3.30249918624028	-1.39760356519063\\
-3.3114165882906	-1.20904465786337\\
-3.30814542905765	-1.00741168191251\\
-3.29096201643304	-0.792256187138982\\
-3.25797286635204	-0.563403515082209\\
-3.20717493015324	-0.321066389214163\\
-3.13655341623575	-0.0659702764881891\\
-3.04422258654089	0.200522420107314\\
-2.92860855717195	0.476310021275436\\
-2.7886626532819	0.758498504660572\\
-2.62408037933843	1.04342739569919\\
-2.43548813601527	1.32679813915937\\
-2.22455324917765	1.60391155894044\\
-1.99397861951217	1.86999526562393\\
-1.74736374460416	2.12057475325495\\
-1.48894469235357	2.35182434160235\\
-1.22325615775622	2.56083491987714\\
-0.954776947771264	2.74575579109426\\
-0.687619329080636	2.9057998932751\\
-0.425304744052455	3.0411326827088\\
-0.170642417405311	3.15268464004814\\
0.0742964523537774	3.24193210250864\\
0.308131369598009	3.31068389954316\\
0.530079690709056	3.36089818232796\\
0.739843429206208	3.39454053901598\\
0.937494593240998	3.41348430414743\\
1.12337085347325	3.41944784996473\\
};
\addplot [color=mycolor2, forget plot]
  table[row sep=crcr]{%
2.03026270353078	1.64575647587509\\
2.24201361397065	1.63929491819078\\
2.40988976796717	1.62346454151037\\
2.5439086404576	1.60217293817841\\
2.65192348194154	1.57789433711441\\
2.73991142031267	1.55215011587583\\
2.81236758836163	1.52585007471876\\
2.87266513834438	1.49951838258181\\
2.92334343033639	1.47343812077522\\
2.96632580056046	1.44774221252277\\
3.00307980332721	1.42247015209458\\
3.03473381366942	1.39760320969316\\
3.06216162049611	1.37308613323043\\
3.08604387326376	1.34884035107809\\
3.1069128492206	1.32477177949003\\
3.12518516010315	1.30077515426889\\
3.14118566099113	1.27673606856081\\
3.15516484960864	1.25253143538509\\
3.16731134706794	1.22802879750042\\
3.17776054716328	1.2030847121369\\
3.1866001477792	1.17754230256394\\
3.19387298775986	1.15122796591422\\
3.19957736951745	1.12394713932686\\
3.2036648216851	1.09547894128019\\
3.20603501902482	1.06556941121\\
3.2065272979622	1.03392295803928\\
3.20490784773303	1.00019148658358\\
3.20085116825551	0.963960488751789\\
3.19391369462968	0.924731153178388\\
3.18349649194873	0.88189725582094\\
3.16879247861024	0.834715253474649\\
3.14871154984905	0.782265661046414\\
3.12177402504919	0.723403600363448\\
3.08595887985921	0.65669674236961\\
3.03848848218452	0.580350616223418\\
2.97552755034422	0.492126403564453\\
2.89177588676445	0.389269007027596\\
2.77995604944656	0.268491300815408\\
2.63027155695706	0.126116785626424\\
2.43010358708565	-0.0414206862591693\\
2.16460932990471	-0.236405649579915\\
1.81944441184434	-0.457695773160881\\
1.38693674034512	-0.697839094321064\\
0.875038323293743	-0.941158665858655\\
0.313368717274706	-1.1659295814316\\
-0.252135552917663	-1.35184958146322\\
-0.776126199887151	-1.48819301966754\\
-1.22963821375558	-1.57605995670056\\
-1.60388412349005	-1.62425867750996\\
-1.9042365387354	-1.64371748521601\\
-2.1423340163231	-1.64401278743609\\
-2.33075772201044	-1.63226643854284\\
-2.48059210593731	-1.61331948699554\\
-2.60075362405107	-1.59029424094336\\
-2.69810967565792	-1.56513735762139\\
-2.77784760291292	-1.53902978813207\\
-2.84386080707416	-1.51266611354359\\
-2.89907416816592	-1.48643579068407\\
-2.94569574923572	-1.46053795843337\\
-2.98540399805985	-1.43505331283764\\
-3.01948454392525	-1.40998882547804\\
-3.04892951832121	-1.38530541457036\\
-3.07450962932375	-1.36093491127263\\
-3.09682658838307	-1.33679026387485\\
-3.11635136498169	-1.31277142156121\\
-3.13345215510356	-1.28876840528795\\
-3.14841479826139	-1.2646624896102\\
-3.16145755384025	-1.24032605003473\\
-3.17274155601308	-1.21562139138008\\
-3.18237783452297	-1.19039871137346\\
-3.19043146182922	-1.16449323731631\\
-3.19692312450592	-1.13772148055284\\
-3.20182818580579	-1.1098764686405\\
-3.20507307834332	-1.08072172691031\\
-3.20652861203835	-1.04998367925386\\
-3.20599946923685	-1.01734201226768\\
-3.20320874276366	-0.98241738645672\\
-3.19777579327309	-0.944755671769844\\
-3.18918487373995	-0.903807622805553\\
-3.17674076907669	-0.858902590472078\\
-3.15950595960833	-0.809214515661926\\
-3.13621132700539	-0.753718154746509\\
-3.10512897253988	-0.691133486978053\\
-3.06389128737052	-0.619857135842971\\
-3.00923570827716	-0.537882735252563\\
-2.9366524931571	-0.442720368567719\\
-2.83992194445915	-0.331344308054474\\
-2.71056965562503	-0.200238660869918\\
-2.53739135908002	-0.0456861773612729\\
-2.30648368509363	0.135441053243499\\
-2.00272025280522	0.344046356187461\\
-1.61408470738881	0.576195424583096\\
-1.13957746635919	0.820384013776223\\
-0.597773254972512	1.05727631209485\\
-0.0280177497212467	1.26470629064575\\
0.521625471572434	1.42639948348009\\
1.01267121040275	1.53774497708029\\
1.42656549795925	1.60443013244292\\
1.76263786922918	1.63692832306647\\
2.03026270353078	1.64575647587509\\
};
\addplot [color=mycolor1, forget plot]
  table[row sep=crcr]{%
1.01343528097655	3.63585201902066\\
1.19185337592755	3.63023677140222\\
1.36099770341136	3.61412819344363\\
1.52136225084341	3.58850545850642\\
1.67346151505693	3.55418471898578\\
1.81780658280804	3.51182884113409\\
1.95488752504785	3.4619585922676\\
2.08516054724753	3.40496396299583\\
2.20903855445196	3.34111480712293\\
2.32688401468736	3.27057034337028\\
2.43900320906891	3.19338731797567\\
2.54564112956951	3.10952680633696\\
2.64697642417426	3.0188597620073\\
2.74311589813743	2.92117152487511\\
2.83408816681036	2.81616559580879\\
2.9198361300294	2.70346708806976\\
3.00020801283658	2.58262638999081\\
3.07494680802332	2.45312373062445\\
3.14367808259371	2.31437553985711\\
3.20589629815401	2.16574374228541\\
3.26095007601892	2.00654941740021\\
3.30802724837493	1.83609258012509\\
3.3461411152807	1.6536801434531\\
3.37412010291755	1.45866433791852\\
3.39060399223019	1.25049384499662\\
3.39405099878186	1.02877944917842\\
3.38276106660261	0.793374855924018\\
3.35492146274524	0.544471159869154\\
3.30868060031537	0.28270004906654\\
3.24225428319321	0.00923621972990475\\
3.15406455666848	-0.274115814033056\\
3.04290471209723	-0.564870011927675\\
2.90811523997044	-0.859863697273204\\
2.74974649930657	-1.15533745256374\\
2.56867777004751	-1.44709857665007\\
2.36666285770699	-1.73076053886735\\
2.14628189544622	-2.00203209688494\\
1.91079657288267	-2.25701450242647\\
1.66392695151825	-2.49245974972126\\
1.40958524610206	-2.70595014654958\\
1.15160943172697	-2.89597743088331\\
0.893535292189428	-3.06192175638923\\
0.638432314978981	-3.20394967734436\\
0.388812268759267	-3.32286080230751\\
0.146604852015977	-3.41991411899592\\
-0.0868143097595534	-3.49665946835587\\
-0.310561319431775	-3.55479092968387\\
-0.524164133255402	-3.59603017651728\\
-0.727478342126641	-3.62204106531461\\
-0.920607476176883	-3.63437241270124\\
-1.10383281154742	-3.63442383470343\\
-1.2775546979489	-3.62342903307452\\
-1.44224545906707	-3.60245136822155\\
-1.59841278737773	-3.57238744537591\\
-1.74657203121279	-3.53397542842141\\
-1.8872256387729	-3.48780570303788\\
-2.02084811230786	-3.43433226123445\\
-2.14787501917248	-3.3738837591114\\
-2.26869483213253	-3.30667362660043\\
-2.38364258848127	-3.23280891210932\\
-2.49299454691819	-3.15229775837368\\
-2.59696317661036	-3.06505555737761\\
-2.69569193597136	-2.97090994650247\\
-2.7892493954591	-2.86960490543576\\
-2.87762233793308	-2.76080431079362\\
-2.96070754302555	-2.6440954174288\\
-3.03830204253425	-2.51899287473264\\
-3.11009173969587	-2.3849440637389\\
-3.17563843890095	-2.24133676433477\\
-3.23436556243217	-2.08751043345009\\
-3.28554317163611	-1.92277268588853\\
-3.32827339989067	-1.74642289138102\\
-3.36147807928184	-1.55778507495665\\
-3.38389122035015	-1.35625242644068\\
-3.3940600599788	-1.14134551988465\\
-3.39035951910219	-0.912785575902793\\
-3.37102586779912	-0.670582476113253\\
-3.33421575579759	-0.41513447650025\\
-3.27809591248885	-0.147332533287431\\
-3.20096602514468	0.131342905161217\\
-3.10141198482739	0.418749985897776\\
-2.97847885558869	0.712059402219527\\
-2.83184370894702	1.00779493438612\\
-2.6619604392095	1.30195494354937\\
-2.4701454483235	1.5902159480539\\
-2.25857794652784	1.86820131746095\\
-2.03020250578075	2.13178026081226\\
-1.78854156401486	2.37735127744792\\
-1.53744555744897	2.60206509794101\\
-1.28082125932745	2.8039553740791\\
-1.02238045525752	2.98196635995617\\
-0.765441767854556	3.13588812240772\\
-0.512802792854487	3.26622486838006\\
-0.266683626030244	3.37402778513509\\
-0.0287308580328474	3.46072126793759\\
0.199934997210314	3.52794382790228\\
0.418646483445373	3.57741593688789\\
0.627105777240416	3.61083916929073\\
0.825301884117939	3.62982544115539\\
1.01343528097654	3.63585201902066\\
};
\addplot [color=mycolor2, forget plot]
  table[row sep=crcr]{%
2.14516078266191	1.81183437647932\\
2.32810767202528	1.80624082044245\\
2.47498896724813	1.79237950465215\\
2.59392767073726	1.77347427500045\\
2.69121028702518	1.7516000183635\\
2.77162026450685	1.72806638687415\\
2.83877780008583	1.7036840880486\\
2.89542578857523	1.67894145181767\\
2.9436522231263	1.65411894328654\\
2.98505693235282	1.62936267187349\\
3.02087449597627	1.60473135783548\\
3.05206427187012	1.5802262386663\\
3.0793762728029	1.55580999860794\\
3.10339945256859	1.53141858573327\\
3.12459717608243	1.50696836088577\\
3.14333329276745	1.48236011877948\\
3.15989123768703	1.45748094377396\\
3.1744878629247	1.43220448859898\\
3.18728317556303	1.40639001567081\\
3.19838676732086	1.37988036802691\\
3.20786141780366	1.35249890818988\\
3.21572410207699	1.32404535679468\\
3.22194440374892	1.2942903628817\\
3.22644009903405	1.26296853170257\\
3.22906940570025	1.22976951211391\\
3.22961904749807	1.19432659237647\\
3.22778682211214	1.15620205798437\\
3.22315671266391	1.11486831569946\\
3.21516365575605	1.06968347569427\\
3.2030437427563	1.01985971386072\\
3.18576371422725	0.964422352420934\\
3.16192091548773	0.90215733690581\\
3.12960128344385	0.831545008064674\\
3.08617864417194	0.750679609819703\\
3.0280349472822	0.657178707510012\\
2.95018243305724	0.5480985257777\\
2.84578696967122	0.419897682430965\\
2.7056547387299	0.268545246798006\\
2.51790683589687	0.0899623815006545\\
2.26840203226445	-0.118893134541208\\
1.94296185496446	-0.357964425295355\\
1.53264287484255	-0.621143106950835\\
1.04185108331257	-0.893831953901765\\
0.494924609752802	-1.15403342101582\\
-0.066534518525441	-1.37894991663137\\
-0.597772602234644	-1.55378096771476\\
-1.06674192827408	-1.67591294678178\\
-1.46035076606654	-1.75222080015195\\
-1.78055759538858	-1.79347207767742\\
-2.03707648936046	-1.8100868393982\\
-2.24172853257791	-1.81033056486947\\
-2.40550246594102	-1.80010983010687\\
-2.53751880948657	-1.78340589518095\\
-2.64494597426942	-1.76281211984989\\
-2.73327449720316	-1.73998069484092\\
-2.80666650064496	-1.71594483032295\\
-2.8682718360826	-1.69133643129177\\
-2.92048189500717	-1.66652855324256\\
-2.96512276341988	-1.6417272257583\\
-3.00359856873776	-1.61703023471763\\
-3.03699671348769	-1.59246461859214\\
-3.0661648784366	-1.56801048741935\\
-3.09176740614943	-1.54361601670386\\
-3.11432667645284	-1.51920669087131\\
-3.13425352034404	-1.49469073780299\\
-3.15186955473061	-1.4699619740346\\
-3.16742347255512	-1.44490081626718\\
-3.18110270743932	-1.4193739110612\\
-3.19304143990479	-1.39323262886547\\
-3.20332557011713	-1.36631052109911\\
-3.21199500898494	-1.33841972365926\\
-3.21904340259761	-1.30934618874109\\
-3.22441517586908	-1.27884352534193\\
-3.22799953159585	-1.24662511534656\\
-3.22962073876998	-1.21235403508853\\
-3.22902364797926	-1.17563013964662\\
-3.2258528262045	-1.1359734462553\\
-3.21962292958437	-1.09280267290438\\
-3.20967681992079	-1.04540744492495\\
-3.1951263279035	-0.992912296007294\\
-3.17476828543654	-0.934230241279021\\
-3.14696530965509	-0.868003614028773\\
-3.10947680774259	-0.792530587011713\\
-3.05922140853847	-0.705678606652429\\
-2.99195001476793	-0.604793579562332\\
-2.90181625129863	-0.486631565879331\\
-2.78086674353476	-0.347377878770904\\
-2.61857755187232	-0.182890815216818\\
-2.40180375939587	0.0105804915569155\\
-2.11594162626837	0.234854610604258\\
-1.74855246596671	0.487240711876294\\
-1.29628392023436	0.757551460136748\\
-0.773087447884022	1.02700904971299\\
-0.213075259222931	1.27210066912647\\
0.338367057257413	1.47307141584576\\
0.841292482717871	1.62116207087161\\
1.27311059132485	1.71915089955838\\
1.62912389573152	1.77653729265007\\
1.91604217687101	1.80428448539298\\
2.14516078266191	1.81183437647932\\
};
\addplot [color=mycolor1, forget plot]
  table[row sep=crcr]{%
0.940529402382294	3.81671254728581\\
1.12221882568828	3.81098838439057\\
1.29553828688765	3.79447642475729\\
1.46088553715868	3.76805200798769\\
1.61867883079513	3.73244107004308\\
1.76933593247377	3.68822784529482\\
1.91325803670642	3.63586375726737\\
2.05081740222796	3.57567646725338\\
2.18234765030009	3.50787843672007\\
2.30813583155767	3.43257464778903\\
2.42841551097784	3.34976933939613\\
2.5433602463567	3.25937177792064\\
2.65307694136993	3.16120121065812\\
2.75759864377023	3.05499126627416\\
2.85687643993119	2.94039418355026\\
2.95077017923965	2.81698538133578\\
3.03903785950927	2.68426903942937\\
3.12132363527079	2.54168554995017\\
3.19714459664589	2.38862192459141\\
3.26587673431111	2.22442649953105\\
3.32674088588706	2.04842954703031\\
3.37878998011524	1.85997163851086\\
3.4208995765126	1.65844173226941\\
3.45176453222614	1.44332685920611\\
3.46990555407134	1.21427477949425\\
3.4736902636639	0.971169864968868\\
3.46137394331136	0.714220501283853\\
3.43116492062658	0.444053352368563\\
3.38131805752097	0.161805959730513\\
3.31025651689468	-0.130795159074461\\
3.21671664761649	-0.43138744706829\\
3.09990385796391	-0.736974296429673\\
2.95964010967504	-1.0439915601454\\
2.79647850519018	-1.3484462432788\\
2.61176009205239	-1.64612228743937\\
2.40759437810898	-1.93283329431207\\
2.18675786731902	-2.20468916462333\\
1.95252113303128	-2.45833775129871\\
1.70842956190959	-2.69114648965576\\
1.45807115625702	-2.90130167934605\\
1.20486434177597	-3.08782037619996\\
0.951890631527259	-3.25048609763232\\
0.701784809477116	-3.38973034605408\\
0.456683192328822	-3.50648564063056\\
0.218221550805198	-3.6020333187657\\
-0.0124304054884634	-3.67786331299615\\
-0.234511883913556	-3.73555603985411\\
-0.447603321986191	-3.77669032071695\\
-0.65155727176105	-3.8027767759739\\
-0.846433809638309	-3.81521349508384\\
-1.03244372230256	-3.81525964852368\\
-1.20990070786174	-3.80402259260859\\
-1.37918250494462	-3.78245448218111\\
-1.54070009742496	-3.75135511975611\\
-1.69487378229385	-3.71137852399006\\
-1.84211479056816	-3.66304138352832\\
-1.98281120528697	-3.6067321287988\\
-2.11731704942275	-3.54271979797231\\
-2.24594357092373	-3.47116220805626\\
-2.36895190407261	-3.392113189972\\
-2.48654642223561	-3.30552883098262\\
-2.59886821259585	-3.21127281082805\\
-2.70598820032301	-3.10912103854377\\
-2.80789953358164	-2.99876591164773\\
-2.90450892075347	-2.87982064227445\\
-2.99562669928613	-2.75182423785062\\
-3.08095552728283	-2.61424789649424\\
-3.16007774414687	-2.46650378520177\\
-3.23244166981062	-2.30795741139842\\
-3.29734743204461	-2.13794506389383\\
-3.35393335846368	-1.95579805791037\\
-3.40116456981079	-1.7608757129587\\
-3.43782617230095	-1.55260902490024\\
-3.4625243391138	-1.33055671819476\\
-3.47369949481505	-1.09447458398215\\
-3.4696565645729	-0.844397496185794\\
-3.4486174762433	-0.580731050508035\\
-3.40880032255423	-0.304346330489577\\
-3.3485272533785	-0.01666712280979\\
-3.26635885456223	0.280265268716509\\
-3.16124651288204	0.583767278568167\\
-3.03268692450557	0.890545754527654\\
-2.88085636606772	1.1968003018717\\
-2.70669926960498	1.49839709064195\\
-2.511948523168	1.79110154521217\\
-2.29906472086288	2.07084273240482\\
-2.0710966388664	2.33397236601865\\
-1.83148129387468	2.5774802038505\\
-1.58381386251731	2.79913625554401\\
-1.33162172825651	2.99754589709511\\
-1.07817232772824	3.17212143192531\\
-0.826333794313452	3.32298755848107\\
-0.578494768481339	3.45084547052904\\
-0.336538935920287	3.55682069245203\\
-0.101862995942506	3.6423151625029\\
0.124575853936526	3.7088772221073\\
0.342196161687663	3.75809635556959\\
0.550722883442389	3.79152412281218\\
0.750119970262207	3.81061919190203\\
0.940529402382293	3.81671254728581\\
};
\addplot [color=mycolor2, forget plot]
  table[row sep=crcr]{%
2.23886372154498	1.97054596626372\\
2.3971405473277	1.96569800633572\\
2.52565451777639	1.95356181822508\\
2.63101472075458	1.9368076774117\\
2.71829225436106	1.9171769431276\\
2.79134646283389	1.89579091537222\\
2.85311275745035	1.87336152224805\\
2.90583199922238	1.85033104357419\\
2.95122497346438	1.82696357625248\\
2.99062220768506	1.80340459198751\\
3.02505965735058	1.77971963312046\\
3.05534904636493	1.75591938705399\\
3.08212961471371	1.7319758194069\\
3.10590625968392	1.70783237231697\\
3.12707767721752	1.68341015198366\\
3.14595708030931	1.65861132931121\\
3.16278731704588	1.63332051994798\\
3.17775165866286	1.60740460463824\\
3.19098111743425	1.58071123920809\\
3.20255883808743	1.55306614754639\\
3.21252184778417	1.52426916483789\\
3.22086021804516	1.49408888349644\\
3.22751345957729	1.46225563596542\\
3.23236370895152	1.42845241350026\\
3.23522494030351	1.39230315482345\\
3.23582700102603	1.353357628782\\
3.23379266602203	1.31107186583288\\
3.22860504328548	1.26478275150053\\
3.21956142251217	1.21367497880448\\
3.20570787578924	1.15673809634543\\
3.18574640394614	1.09271099770805\\
3.15790302204014	1.02001118671122\\
3.11974100782698	0.93664727675023\\
3.06789961820395	0.840117152408607\\
2.99773845519081	0.727304690745195\\
2.90288113640783	0.594412126820719\\
2.77470249200837	0.437015500958592\\
2.60193895795226	0.250422457776838\\
2.37089293593623	0.0306421785226072\\
2.0671652834548	-0.22364673721209\\
1.68015815458778	-0.50804165113269\\
1.21061935927683	-0.809365687088379\\
0.677943225041865	-1.10554365855548\\
0.119994514024516	-1.37122039316945\\
-0.418603686536259	-1.58717161051174\\
-0.902562091961035	-1.74656914995138\\
-1.31454939143297	-1.85392677460347\\
-1.65327120545997	-1.91961872031168\\
-1.92668956910225	-1.95484601531965\\
-2.14600505565856	-1.96904561234524\\
-2.32220434948555	-1.96924705464741\\
-2.46466491470551	-1.96034786893389\\
-2.58087975605076	-1.94563555192347\\
-2.67664785983346	-1.92727015027268\\
-2.75639513382362	-1.9066511722555\\
-2.82348655693236	-1.88467393820216\\
-2.88048523384333	-1.86190155832319\\
-2.92935322944049	-1.83867809557897\\
-2.97160233631346	-1.81520240074319\\
-3.00840560076461	-1.79157612972125\\
-3.04067936767815	-1.76783490847551\\
-3.06914359976889	-1.74396847328966\\
-3.09436629393814	-1.71993353935428\\
-3.11679624346869	-1.6956618039765\\
-3.1367871984734	-1.67106462072325\\
-3.15461559476158	-1.64603531582131\\
-3.17049337645814	-1.62044974563793\\
-3.18457696299681	-1.59416544138832\\
-3.19697305294373	-1.56701950728727\\
-3.2077416738476	-1.53882530000431\\
-3.2168966457057	-1.50936779884364\\
-3.22440339703895	-1.47839746123413\\
-3.23017382918471	-1.44562223308809\\
-3.23405763531408	-1.4106972351741\\
-3.23582910716819	-1.37321146101976\\
-3.23516795230254	-1.33267058417853\\
-3.2316319243834	-1.28847466843858\\
-3.22461803537015	-1.23988919461343\\
-3.21330762930232	-1.18600737204386\\
-3.1965884725218	-1.12570125453857\\
-3.17294406804385	-1.05755892086571\\
-3.14029656852784	-0.979805398423894\\
-3.09578538824318	-0.890207250560821\\
-3.03546088218197	-0.785967333430171\\
-2.95387728181653	-0.663632354103263\\
-2.84359689196043	-0.5190712343952\\
-2.6947027328728	-0.347651706824467\\
-2.49462112019136	-0.144857159474346\\
-2.22894202869952	0.092288706523193\\
-1.88439266240653	0.362673847196198\\
-1.45498408819297	0.657793680364402\\
-0.950154585058804	0.959712068055632\\
-0.399297508665178	1.24364687671789\\
0.154306540059556	1.48615161318593\\
0.668936484537602	1.67386686455298\\
1.1179353871564	1.80617203399682\\
1.49270534993371	1.89125774695739\\
1.79746627586354	1.9403949720658\\
2.042394554795	1.96407939077264\\
2.23886372154498	1.97054596626372\\
};
\addplot [color=mycolor1, forget plot]
  table[row sep=crcr]{%
0.896441690959812	3.9441923617009\\
1.08051075693992	3.93838963927558\\
1.2567522619063	3.92159579500097\\
1.42551554186611	3.89462204154981\\
1.58716888950602	3.85813663492713\\
1.74208022106361	3.81267167666549\\
1.89060191874816	3.758630911541\\
2.03305881683847	3.69629765064671\\
2.16973841785594	3.62584228168578\\
2.30088255142231	3.54732908093015\\
2.42667980715507	3.46072223444294\\
2.54725817812176	3.36589112949966\\
2.66267744312509	3.26261510845942\\
2.77292089898392	3.15058800222867\\
2.87788613609048	3.02942289234418\\
2.97737464328587	2.8986577007422\\
3.07108014656033	2.75776238280855\\
3.15857574937347	2.60614870633934\\
3.23930017381126	2.44318383344733\\
3.31254372793733	2.26820916924529\\
3.3774350723429	2.08056616661775\\
3.43293044815691	1.87963091917116\\
3.47780776033545	1.66485933575901\\
3.51066874471785	1.4358443267926\\
3.52995327764365	1.19238556079154\\
3.53397050503074	0.934570769130519\\
3.520951542599	0.662865133062121\\
3.48912758734578	0.378201980646154\\
3.43683490925928	0.0820641606591533\\
3.36264404373427	-0.22345818841623\\
3.26550470313232	-0.53564992420845\\
3.14489133868984	-0.851208802312372\\
3.00092865089166	-1.16634927276106\\
2.83447398519056	-1.47697249275014\\
2.64713652401332	-1.77888975621129\\
2.4412221560901	-2.06807363835451\\
2.21960628443296	-2.34090274714581\\
1.98555095319482	-2.59436541308887\\
1.7424931426409	-2.82619562441075\\
1.49383473629253	-3.03492853999866\\
1.24276094183342	-3.21987843739294\\
0.99210489703689	-3.38105444633402\\
0.744265247213911	-3.51903616312019\\
0.501173856815033	-3.63483195176959\\
0.264304426877602	-3.72973898211346\\
0.0347100233669113	-3.80521812041878\\
-0.186922315934679	-3.86279069643553\\
-0.400209775873058	-3.90395918544841\\
-0.605011972928268	-3.93015044620767\\
-0.801373771193119	-3.9426782744144\\
-0.989475129968325	-3.94272132193596\\
-1.16958882528395	-3.93131248815339\\
-1.34204585640571	-3.90933636257011\\
-1.50720775825661	-3.87753193646439\\
-1.66544476555644	-3.83649845119894\\
-1.81711870285525	-3.78670283131044\\
-1.96256952083242	-3.72848763083526\\
-2.10210450369508	-3.66207879984237\\
-2.23598929741258	-3.58759286802539\\
-2.36444003225977	-3.50504336254576\\
-2.48761592541276	-3.41434644820779\\
-2.60561184747482	-3.31532591828493\\
-2.71845042333053	-3.20771779060812\\
-2.82607331901439	-3.09117489037264\\
-2.92833145207355	-2.96527194084203\\
-3.02497396639078	-2.82951184559114\\
-3.11563595065332	-2.68333403757056\\
-3.19982507385023	-2.52612599199289\\
-3.2769075865186	-2.35723924349641\\
-3.34609452020288	-2.17601148989445\\
-3.40642943436212	-1.98179655873834\\
-3.45677972219486	-1.77400407998747\\
-3.49583427838497	-1.55215052784587\\
-3.52211118323165	-1.31592270230808\\
-3.53397981522779	-1.06525351898234\\
-3.52970219987871	-0.800407976357051\\
-3.50749804443063	-0.522074279067044\\
-3.46563632196283	-0.231451448007957\\
-3.4025530281778	0.0696791348611318\\
-3.31698970315709	0.378913893937647\\
-3.20814095193239	0.693235425230204\\
-3.07579282076281	1.00908258119744\\
-2.92042958963605	1.32248811271824\\
-2.74328665053001	1.62927785135783\\
-2.54633316001776	1.9253118886953\\
-2.33217967750365	2.20673703088098\\
-2.10392029761606	2.4702150397221\\
-1.86493159586839	2.71309499905096\\
-1.61865801975323	2.93350962852924\\
-1.36841321339716	3.13039076137595\\
-1.1172199965142	3.30341369522077\\
-0.867701270448185	3.45288993022916\\
-0.622023503989817	3.5796314323213\\
-0.381886304879468	3.68480776815644\\
-0.148547037878136	3.76981232605801\\
0.0771317224625102	3.83614761341639\\
0.294622911237203	3.88533397713145\\
0.503672444507914	3.91884188328714\\
0.704239306475423	3.93804528486142\\
0.896441690959811	3.9441923617009\\
};
\addplot [color=mycolor2, forget plot]
  table[row sep=crcr]{%
2.31219753178786	2.11733153574279\\
2.44922341097804	2.11312775477941\\
2.56158996350191	2.10251021132567\\
2.65470000455284	2.08769859129554\\
2.73267510535881	2.07015547350546\\
2.79865320194544	2.05083686318459\\
2.85503036596741	2.0303609255361\\
2.90364646842595	2.00911982122122\\
2.94592401264342	1.98735336420384\\
2.98297073132288	1.96519745049244\\
3.01565511133232	1.94271587522494\\
3.04466200341872	1.9199211706665\\
3.07053364776049	1.89678811750632\\
3.09369999080958	1.8732622887459\\
3.11450107276432	1.84926514335361\\
3.13320345912556	1.8246966344936\\
3.15001210013918	1.79943592837321\\
3.16507856566243	1.77334057550395\\
3.17850627055307	1.74624429114838\\
3.19035303839031	1.71795335539022\\
3.20063111778286	1.68824151382186\\
3.20930453781797	1.65684313019624\\
3.21628343921874	1.62344419736214\\
3.22141471297521	1.58767063712821\\
3.22446787669025	1.54907309668509\\
3.22511456290194	1.50710716029286\\
3.22289920129302	1.4611075212188\\
3.21719733181251	1.41025418735661\\
3.20715632890983	1.35352823453838\\
3.19161094778317	1.28965404943067\\
3.16896282325786	1.21702465758599\\
3.13700882893513	1.13360723270606\\
3.09269866370287	1.03682870343218\\
3.03179975076724	0.923449784702938\\
2.94845514644759	0.789456527236652\\
2.83465694514752	0.630044135083091\\
2.67976544666188	0.439856708125415\\
2.470453671707	0.213788759349444\\
2.19190030976773	-0.0512092357059949\\
1.83149757056718	-0.353020390318954\\
1.38586064895721	-0.680636258290168\\
0.868979301433518	-1.01254253787914\\
0.314775059181264	-1.32092149657512\\
-0.231992170738203	-1.58148225610957\\
-0.732142184200613	-1.78216604951326\\
-1.16346924919793	-1.92431373598648\\
-1.52110294016983	-2.01754639030242\\
-1.81123723337378	-2.07382755476637\\
-2.04459157333144	-2.10389266620177\\
-2.23231835496362	-2.11604172030406\\
-2.38418957407001	-2.11620865195714\\
-2.50811759304301	-2.10846060918443\\
-2.61026815515519	-2.09552290855396\\
-2.69536337117685	-2.07919915628878\\
-2.76699906822492	-2.06067308728935\\
-2.82791570154822	-2.04071476579902\\
-2.88021121612825	-2.01981812605767\\
-2.92550194722388	-1.99829177037317\\
-2.96504207156239	-1.97631869958058\\
-2.99981164768789	-1.95399557306985\\
-3.03058141070106	-1.9313584754723\\
-3.05796052354452	-1.90839972850473\\
-3.08243183998244	-1.88507868464425\\
-3.10437796560589	-1.86132839610284\\
-3.12410046178291	-1.83705937156592\\
-3.14183384840193	-1.81216118317854\\
-3.15775555529696	-1.786502381649\\
-3.17199259332608	-1.75992896227229\\
-3.18462542050323	-1.73226146215457\\
-3.19568923178027	-1.70329063338424\\
-3.20517267385935	-1.67277150931253\\
-3.21301375125914	-1.64041554563228\\
-3.21909241698553	-1.6058803595666\\
-3.22321899405013	-1.5687563933435\\
-3.22511710383538	-1.52854957457582\\
-3.22439911527329	-1.48465871706912\\
-3.22053117725034	-1.43634598373198\\
-3.21278351860096	-1.38269821389076\\
-3.20015971378189	-1.32257633521977\\
-3.18129580814329	-1.25454957478905\\
-3.15431642558814	-1.17681114515829\\
-3.11663044452128	-1.08707347508828\\
-3.06464488273198	-0.982446092645308\\
-2.99337682270279	-0.859312703859403\\
-2.89596188345167	-0.713255221179613\\
-2.76312292999702	-0.539137347331856\\
-2.5828311924434	-0.331576630228325\\
-2.34073848640261	-0.0861895684435461\\
-2.02245974210641	0.197953775757708\\
-1.61894476666473	0.514714918158474\\
-1.1346272030851	0.847742484176509\\
-0.593813487202299	1.17140343647957\\
-0.037695774543649	1.45827588621514\\
0.489734952359442	1.68949633550006\\
0.95705966773936	1.86007193292024\\
1.35128361218501	1.97629591348342\\
1.67398237851932	2.04958304062953\\
1.93429493771153	2.0915584657179\\
2.14350538852695	2.1117855131171\\
2.31219753178786	2.11733153574279\\
};
\addplot [color=mycolor1, forget plot]
  table[row sep=crcr]{%
0.8754575855778	4.0104177677322\\
1.06078896386434	4.0045735522546\\
1.23854922283576	3.98763331937388\\
1.40906731527161	3.96037745627087\\
1.57269006917172	3.92344595010289\\
1.72976359699493	3.87734483702673\\
1.88061855825207	3.82245355026012\\
2.02555831002059	3.75903236969278\\
2.16484908783362	3.68722948504239\\
2.29871147429479	3.60708742245828\\
2.42731252133389	3.51854876960353\\
2.55075798927877	3.42146128496499\\
2.66908425269244	3.31558261015845\\
2.78224950425876	3.20058493399301\\
2.89012397248969	3.07606009659427\\
2.99247896873398	2.94152578111821\\
3.08897470971137	2.79643362639668\\
3.17914704345333	2.64018030796597\\
3.26239346283231	2.47212287057045\\
3.33795914786553	2.2915998319141\\
3.40492425999905	2.0979597736456\\
3.46219433283419	1.89059921987429\\
3.50849635386469	1.66901146600368\\
3.54238395366623	1.43284750920726\\
3.56225588019801	1.18198916153694\\
3.56639239984929	0.916632614519469\\
3.55301407953746	0.637378062993985\\
3.52036612351583	0.34531757374038\\
3.46682864144034	0.042109632465084\\
3.39104871394099	-0.269974385645804\\
3.29208421282323	-0.588047866480912\\
3.16954310200086	-0.908664677082647\\
3.02369724810034	-1.22794039218397\\
2.85554881414298	-1.5417357521964\\
2.66683175586405	-1.84588617090004\\
2.45994085201191	-2.13644959609351\\
2.23779398117415	-2.40993886934328\\
2.00364623041742	-2.66350649919343\\
1.76088292059494	-2.8950592542108\\
1.51282035604323	-3.10329417726252\\
1.26253818584371	-3.28766205539108\\
1.0127580451136	-3.44827510923211\\
0.765772890696933	-3.58578070169582\\
0.523422924806344	-3.70122243120848\\
0.28710867212055	-3.79590578646691\\
0.0578297533518808	-3.87127974547553\\
-0.16376158403674	-3.92884006165661\\
-0.377301125842935	-3.97005551774127\\
-0.582653973083151	-3.99631547460025\\
-0.779860478794216	-4.00889546709449\\
-0.969088921892505	-4.00893706080468\\
-1.15059559138525	-3.99743830836119\\
-1.32469204290745	-3.97525162052005\\
-1.49171877168303	-3.94308647587504\\
-1.65202430719974	-3.90151500087944\\
-1.80594867567897	-3.85097899069579\\
-1.95381022095887	-3.79179738648444\\
-2.09589487016749	-3.72417357645836\\
-2.23244704428052	-3.64820215943935\\
-2.36366152645962	-3.56387501880103\\
-2.48967570419104	-3.47108672039994\\
-2.61056169286983	-3.36963938799942\\
-2.72631793166803	-3.25924733949355\\
-2.83685992415061	-3.13954190057296\\
-2.94200988646712	-3.0100769607635\\
-3.04148517894587	-2.87033600869235\\
-3.13488555037737	-2.71974158361321\\
-3.22167944008226	-2.5576683066545\\
-3.30118988622717	-2.38346089509554\\
-3.37258100569672	-2.19645878626191\\
-3.43484656180295	-1.99602914905429\\
-3.48680282540288	-1.78161005091959\\
-3.52708873210587	-1.5527652451021\\
-3.5541771522628	-1.3092512763151\\
-3.56640173996894	-1.05109618213473\\
-3.56200401681776	-0.778686835306311\\
-3.53920466986621	-0.492858901596589\\
-3.49630104919356	-0.194979719730555\\
-3.43178919645336	0.112989202526469\\
-3.3445034434412	0.428465808248053\\
-3.2337603623416	0.748272731161792\\
-3.09948811511522	1.06872542460145\\
-2.94231915600336	1.38578570460051\\
-2.76362585630468	1.69527105654625\\
-2.56548592378343	1.99309733314492\\
-2.35057647791853	2.27552318459625\\
-2.12200924627566	2.53936219982267\\
-1.88313046792046	2.78213458674895\\
-1.63731435100178	3.00214257596618\\
-1.38777715533139	3.19846861629568\\
-1.13743151443992	3.37090840357856\\
-0.888790469792121	3.51985876913608\\
-0.643921028663922	3.64618260939846\\
-0.404440041536871	3.75107046116877\\
-0.171541574434479	3.83591307758739\\
0.0539556738293125	3.90219346612614\\
0.271550971989046	3.95140171852103\\
0.481001881327635	3.98497224848479\\
0.682267495359916	4.00424082343988\\
0.8754575855778	4.0104177677322\\
};
\addplot [color=mycolor2, forget plot]
  table[row sep=crcr]{%
2.36607513206161	2.24876922711939\\
2.48479706281508	2.2451218328169\\
2.58299896795016	2.23583800156676\\
2.66512219853627	2.22276998478633\\
2.73454184645255	2.2071481048122\\
2.79382894849366	2.18978550625748\\
2.84495288427685	2.17121476259944\\
2.88943286094029	2.15177833760319\\
2.92844974236369	2.13168839648718\\
2.96292821583002	2.11106637597001\\
2.99359717000535	2.08996915444847\\
3.0210341646257	2.06840627620922\\
3.04569826787349	2.04635111826338\\
3.06795432697702	2.02374786766238\\
3.088090849694	2.00051550832921\\
3.10633302831232	1.97654957172304\\
3.12285196427214	1.9517221016921\\
3.13777079597541	1.92588006614125\\
3.15116815280601	1.89884228016008\\
3.16307912154825	1.87039476134352\\
3.17349368874399	1.84028429845796\\
3.18235238645018	1.80820986219087\\
3.1895385885222	1.77381130435142\\
3.1948665410602	1.73665456055376\\
3.19806371075895	1.69621226889143\\
3.19874532252459	1.651838317919\\
3.1963779229547	1.60273431584277\\
3.19022729282418	1.54790531486466\\
3.17928382873872	1.48610135786744\\
3.16215537235939	1.41574068525492\\
3.13691319543502	1.33481019147983\\
3.1008716536058	1.24074009733246\\
3.0502774481935	1.13025550889201\\
2.97988581083231	0.999223520956103\\
2.88242297065914	0.842552786578786\\
2.74801189806255	0.654283612795641\\
2.56383956356074	0.42815369958251\\
2.31475351698567	0.159119005239815\\
1.98603619888372	-0.153648071589967\\
1.56964637155804	-0.502450673511533\\
1.07307920943061	-0.867685243885396\\
0.525045715892561	-1.21982070756272\\
-0.0300686417610128	-1.52892864853262\\
-0.54848079440757	-1.77614915002699\\
-1.00181111163593	-1.95815382094919\\
-1.38061631647323	-2.08304703639134\\
-1.68892329885739	-2.16344268568353\\
-1.93698816546921	-2.21156852013282\\
-2.13631241828859	-2.23724727348235\\
-2.29726209591901	-2.24765874981433\\
-2.42832874967536	-2.24779758371498\\
-2.53615694693746	-2.24105114662836\\
-2.62583786134551	-2.22968834031408\\
-2.70124308342754	-2.21521955092914\\
-2.7653167740668	-2.19864577489946\\
-2.82030728836154	-2.18062611348689\\
-2.86794271978042	-2.16158898882639\\
-2.90956118911193	-2.14180569951998\\
-2.94620672239702	-2.12143906531823\\
-2.97869967160048	-2.10057561443679\\
-3.00768851348651	-2.07924683740924\\
-3.03368805283886	-2.0574430954663\\
-3.05710765533064	-2.03512250741799\\
-3.07827209651661	-2.01221631374512\\
-3.09743685605133	-1.98863167260534\\
-3.1147991341797	-1.96425247643176\\
-3.13050545896176	-1.93893852243978\\
-3.14465643976465	-1.91252318125205\\
-3.157308968138	-1.88480955464968\\
-3.16847594086442	-1.85556497395101\\
-3.17812335416495	-1.8245135466138\\
-3.18616436406186	-1.79132629348262\\
-3.19244959157946	-1.7556082148057\\
-3.19675252762576	-1.71688135921992\\
-3.19874829759734	-1.67456262237005\\
-3.19798318852437	-1.62793454443151\\
-3.19383109034958	-1.57610678635818\\
-3.18543117438541	-1.51796524405418\\
-3.17159849099963	-1.45210498074569\\
-3.15069547410926	-1.37674258308107\\
-3.12044753174917	-1.28960389793679\\
-3.07768067158824	-1.18778615883497\\
-3.01795644245956	-1.06760331224818\\
-2.93508840278829	-0.924448472815183\\
-2.8205679718414	-0.752764040392522\\
-2.66305620321969	-0.546322006812653\\
-2.44839501627363	-0.299196532781962\\
-2.16110678567916	-0.00797435955711197\\
-1.78879993103978	0.324476345462676\\
-1.33009207596989	0.684705987266555\\
-0.802928758313889	1.04740134737392\\
-0.24540594588675	1.38129334496077\\
0.296099945092465	1.66082990681729\\
0.784276385241216	1.8749810740011\\
1.20050216946353	2.0269837269649\\
1.54301420737431	2.12799812665259\\
1.81975778976501	2.19086075125827\\
2.04205424924148	2.22670699051684\\
2.22100759983092	2.24400500522965\\
2.36607513206161	2.24876922711939\\
};
\addplot [color=mycolor1, forget plot]
  table[row sep=crcr]{%
0.873456719862334	4.01693373145583\\
1.05891327869839	4.01108540720015\\
1.23682310230941	3.9941307624143\\
1.40751329045177	3.96684723559204\\
1.5713287044198	3.92987209111072\\
1.72861344520434	3.88370883688959\\
1.8796961552078	3.8287345317985\\
2.02487818403272	3.76520719212756\\
2.16442376584047	3.69327281387803\\
2.29855146967883	3.61297176422278\\
2.42742629184089	3.52424447990469\\
2.55115185588764	3.42693656085289\\
2.66976227232088	3.32080348051212\\
2.78321329113623	3.20551526489463\\
2.89137246526386	3.08066163265594\\
2.99400814334819	2.94575824861939\\
3.09077724232353	2.80025492986278\\
3.18121193375922	2.64354685821892\\
3.26470563673946	2.47499008868385\\
3.34049907010301	2.29392287872273\\
3.40766760223331	2.09969455651876\\
3.46511176086235	1.89170372440407\\
3.51155351700004	1.66944744605708\\
3.54554177723419	1.4325825396255\\
3.56547127237876	1.18099900892871\\
3.56961947845612	0.914903810388281\\
3.55620599223771	0.634910472204568\\
3.52347746660291	0.342126651254164\\
3.46981837323584	0.0382279755197088\\
3.39388331735628	-0.274496619328825\\
3.29474071424938	-0.593144051856046\\
3.1720114463944	-0.914254534674148\\
3.02598151755879	-1.23393444664973\\
2.85766690030261	-1.54804101814983\\
2.66881335507843	-1.85241233716958\\
2.46182399016865	-2.14311481241083\\
2.23962059159806	-2.4166742889777\\
2.00545749895331	-2.67025899425647\\
1.76271511630104	-2.90179211307992\\
1.5147016951166	-3.10998598831279\\
1.26448700290808	-3.29430426487939\\
1.01478226487975	-3.45486885847563\\
0.767870580179866	-3.59233351054688\\
0.525583595809768	-3.70774515597869\\
0.289314989276039	-3.8024101063091\\
0.0600593545353311	-3.87777627277717\\
-0.161534365867956	-3.9353370560545\\
-0.375103765549384	-3.97655811533715\\
-0.580514536320754	-4.00282531612021\\
-0.777806689788507	-4.01541060842645\\
-0.967147489666844	-4.01545206439023\\
-1.1487917526378	-4.00394443509903\\
-1.32304927507435	-3.9817370633439\\
-1.49025863160819	-3.949536594988\\
-1.65076635603593	-3.90791253566546\\
-1.8049104570209	-3.85730423461502\\
-1.95300726548201	-3.79802831936665\\
-2.09534070557109	-3.73028595429559\\
-2.232153193876	-3.65416956580092\\
-2.36363748332244	-3.56966888501637\\
-2.48992887056574	-3.47667632423253\\
-2.6110972767274	-3.37499184311909\\
-2.72713879426336	-3.26432759100626\\
-2.8379663744662	-3.14431274546656\\
-2.94339942097207	-3.01449911656582\\
-3.04315216863044	-2.87436825898505\\
-3.13682088210797	-2.72334103518177\\
-3.22387012658132	-2.56079079956444\\
-3.30361866903396	-2.38606161298636\\
-3.37522598876934	-2.19849311809292\\
-3.43768093001322	-1.99745385297501\\
-3.489794721229	-1.78238476237182\\
-3.53020138305169	-1.55285435020383\\
-3.55736935648446	-1.30862613309598\\
-3.56962882114418	-1.04973761227755\\
-3.5652193417877	-0.776587727073905\\
-3.54236177337171	-0.490026666871447\\
-3.49935632124874	-0.191438251483621\\
-3.43470496054046	0.117198492103618\\
-3.34725110462667	0.43328428570102\\
-3.23632317040156	0.753626484098471\\
-3.10186302120267	1.07452892770211\\
-2.94451729347739	1.39194696563502\\
-2.76567137060142	1.70169765748539\\
-2.56741318493844	1.99970253001784\\
-2.35242605060152	2.28223116587682\\
-2.12382325714924	2.54611176537777\\
-1.88494811092087	2.78888085186346\\
-1.63916818199947	3.00885671507353\\
-1.38969059392881	3.20513601141079\\
-1.13941768184485	3.37752576543612\\
-0.890852237628319	3.52643083375097\\
-0.646051990474897	3.65271891601133\\
-0.406626059034493	3.75758255335449\\
-0.173762573571543	3.84241229809124\\
0.0517239108538937	3.90868937694286\\
0.269335186361097	3.95790108610335\\
0.478829977292244	3.99147848747363\\
0.680167467369668	4.01075377987807\\
0.873456719862334	4.01693373145583\\
};
\addplot [color=mycolor2, forget plot]
  table[row sep=crcr]{%
2.40210553937661	2.3630934927815\\
2.50510161403136	2.3599252946844\\
2.59093986907906	2.35180674191589\\
2.66329357058539	2.34029018666113\\
2.72494730336123	2.32641316944438\\
2.77802380521816	2.31086697042658\\
2.82415333884121	2.29410829369208\\
2.86459819850988	2.2764331562321\\
2.90034375025032	2.25802585522687\\
2.93216501324467	2.2389914814089\\
2.96067549452266	2.21937748795096\\
2.98636313825409	2.19918788625439\\
3.00961686072481	2.178392380323\\
3.03074612968263	2.15693193032603\\
3.04999531717496	2.13472169426589\\
3.06755402708116	2.11165193088673\\
3.08356420926532	2.08758718990546\\
3.0981245740805	2.06236392291898\\
3.11129257730319	2.03578648931227\\
3.12308402653721	2.00762138329789\\
3.13347013824599	1.9775893515926\\
3.1423716207288	1.94535488741487\\
3.1496490370355	1.91051235464126\\
3.1550882650705	1.87256769100378\\
3.15837925131487	1.83091423017277\\
3.15908534884755	1.78480063269962\\
3.15659919152362	1.73328818764031\\
3.15007907223557	1.67519382034874\\
3.13835688076933	1.60901406613714\\
3.11980449825242	1.53282429113761\\
3.09213997322834	1.44414731633822\\
3.05214840971314	1.33978830226637\\
2.99528834615863	1.21564299799959\\
2.915162391427	1.0665146585686\\
2.80287740920904	0.886041612979106\\
2.64646409737929	0.666974814346141\\
2.43087006477885	0.40227371715175\\
2.13964864643347	0.0877097586390446\\
1.75998295128699	-0.273604994523119\\
1.29162468022708	-0.666084542994651\\
0.755735709369115	-1.06045141329421\\
0.194381811289358	-1.42137835431777\\
-0.343993528887985	-1.72136328187118\\
-0.8230959741329	-1.94997123111311\\
-1.22706646442504	-2.11223255958094\\
-1.55674573595149	-2.22096273838767\\
-1.82167218307478	-2.29005839159569\\
-2.03382776263391	-2.33121944287373\\
-2.20441075259616	-2.35319302273665\\
-2.3427115930779	-2.36213552675717\\
-2.45602015570107	-2.36225150184665\\
-2.54991195846396	-2.35637325432495\\
-2.62861039286575	-2.34639861192798\\
-2.69531198456748	-2.33359694759486\\
-2.75244592225422	-2.3188156891715\\
-2.80187063105124	-2.30261759563426\\
-2.84501915808969	-2.28537159093286\\
-2.88300566451951	-2.26731289744304\\
-2.91670326267435	-2.24858293136569\\
-2.94680101222194	-2.22925579567656\\
-2.97384580138051	-2.20935580740452\\
-2.99827322614566	-2.18886893315524\\
-3.02043039130044	-2.16774999072742\\
-3.04059269726404	-2.14592680917192\\
-3.05897605708338	-2.12330209619639\\
-3.07574553608595	-2.09975345711912\\
-3.09102106812584	-2.07513178920378\\
-3.10488063548612	-2.04925810255761\\
-3.11736107164653	-2.0219186677147\\
-3.12845642910063	-1.99285823992896\\
-3.13811362046281	-1.96177094239724\\
-3.14622475857645	-1.92828818562114\\
-3.15261524936247	-1.89196273519221\\
-3.15702617264633	-1.85224768755824\\
-3.15908873808585	-1.80846863872752\\
-3.15828750318021	-1.75978669611459\\
-3.15390741000263	-1.70514915648678\\
-3.14495728852243	-1.64322365788887\\
-3.13005897651501	-1.57231053418962\\
-3.10728633676627	-1.49022740200106\\
-3.07393230130358	-1.39416092882715\\
-3.02617613796275	-1.28048634955713\\
-2.9586230951157	-1.14457270076291\\
-2.86371092077442	-0.980635813944171\\
-2.73106306753107	-0.781798681809504\\
-2.54709840982175	-0.540702677403615\\
-2.29568726130443	-0.251266722847516\\
-1.96130476006145	0.0877368265485396\\
-1.53613415024095	0.467498290295421\\
-1.02987232282991	0.865253094815708\\
-0.475123546990699	1.24715643189577\\
0.0803998746625715	1.58007232845334\\
0.592422884350118	1.84455614988675\\
1.03471808848556	2.03868190768345\\
1.40071126286749	2.17239124480139\\
1.69656031103102	2.25966486586898\\
1.93360764642683	2.31351636318777\\
2.12368640087881	2.34416651870353\\
2.27709645851434	2.35899206933955\\
2.40210553937661	2.3630934927815\\
};
\addplot [color=mycolor1, forget plot]
  table[row sep=crcr]{%
0.88748083090382	3.97200858962911\\
1.0720778919661	3.96618849790435\\
1.24895705497356	3.94933318180974\\
1.41845873996219	3.92224071305184\\
1.58094188710687	3.88556733999441\\
1.73676494485547	3.83983413351367\\
1.88627089384102	3.78543458741033\\
2.02977530398858	3.72264233455396\\
2.16755653586359	3.65161846206454\\
2.29984731849397	3.57241815496482\\
2.42682705012332	3.48499658733104\\
2.54861426999769	3.38921413209142\\
2.66525883878065	3.28484109258449\\
2.77673344728929	3.17156228599146\\
2.88292415622928	3.04898194384937\\
2.98361976516508	2.91662954878138\\
3.07849993246268	2.77396740708429\\
3.16712213882791	2.62040096697476\\
3.24890782883676	2.45529312748884\\
3.32312840391552	2.27798402600529\\
3.38889220239087	2.08781800617625\\
3.44513420522414	1.88417958710035\\
3.49061094602754	1.66654017533366\\
3.52390393434153	1.43451683673792\\
3.5434357040016	1.18794349033633\\
3.54750315362382	0.926953207658468\\
3.53433281244008	0.652067762950401\\
3.50216159819999	0.364287218810285\\
3.44934408098872	0.0651685085542933\\
3.37448295939045	-0.243121487807968\\
3.27657359760087	-0.557794681664479\\
3.15514702279179	-0.875487236086551\\
3.01039053558596	-1.19237084085232\\
2.84322332138081	-1.50432859235038\\
2.65530806361779	-1.80718120739027\\
2.44898895673042	-2.09693696378953\\
2.22715988255978	-2.37003130859836\\
1.9930801181445	-2.62352252760346\\
1.75016456792847	-2.85521852822322\\
1.50177832216072	-3.06372391945442\\
1.25106108174094	-3.24841164825857\\
1.00079784722329	-3.40933518809728\\
0.753341613742977	-3.54710327268561\\
0.510584667334669	-3.6627393695639\\
0.273969153648827	-3.75754413326148\\
0.0445251531774694	-3.83297319760561\\
-0.177075181045537	-3.89053676919391\\
-0.390456895440626	-3.93172272686753\\
-0.595481787898095	-3.95794172913841\\
-0.792192586781656	-3.97049108501902\\
-0.9807640315538	-3.97053350968948\\
-1.16146162361766	-3.95908697293618\\
-1.3346078374386	-3.93702232266239\\
-1.50055502356405	-3.90506599125295\\
-1.6596639758734	-3.86380572390053\\
-1.81228706908935	-3.81369783019103\\
-1.95875491793273	-3.7550749251003\\
-2.09936560985006	-3.68815349230441\\
-2.23437568316647	-3.61304088447291\\
-2.36399214139935	-3.52974159061903\\
-2.48836490267482	-3.43816276914898\\
-2.60757917840563	-3.338119185261\\
-2.72164736035928	-3.22933781906081\\
-2.83050007654208	-3.11146254031263\\
-2.93397616391571	-2.98405938907472\\
-3.03181141332334	-2.84662316795648\\
-3.12362608650336	-2.69858624707172\\
-3.20891140823143	-2.53933070666704\\
-3.28701552366608	-2.36820518468506\\
-3.35712980862818	-2.18454803109935\\
-3.41827695185381	-1.98771854765969\\
-3.46930290230872	-1.77713812720729\\
-3.50887556937882	-1.55234287588744\\
-3.53549400081329	-1.31304863658416\\
-3.54751247725296	-1.05922803781881\\
-3.54318427306414	-0.791197093373195\\
-3.52072934240834	-0.509705907386554\\
-3.47842843077109	-0.216024376266373\\
-3.41474269343404	0.0879899996174577\\
-3.32845275249314	0.399857903219787\\
-3.2188047962198	0.71649368801053\\
-3.08564520760163	1.03428315833578\\
-2.92952141661873	1.34922808352798\\
-2.7517274258466	1.65714986611883\\
-2.55427904466183	1.95393157086205\\
-2.3398156170525	2.23576713284225\\
-2.11143905370944	2.49938282954807\\
-1.87251307579609	2.74220082153557\\
-1.62645200166476	2.96242645934868\\
-1.37652754064401	3.15905625761151\\
-1.12571497113936	3.33181728725355\\
-0.876589749548109	3.48105775263239\\
-0.631275386710711	3.60761149658494\\
-0.391435775371032	3.71265703174771\\
-0.158301014886573	3.79758650411687\\
0.0672851495639999	3.86389391027349\\
0.284806868002934	3.91308646802211\\
0.494014904291655	3.94661904697569\\
0.694868100710875	3.96584912424045\\
0.887480830903819	3.97200858962911\\
};
\addplot [color=mycolor2, forget plot]
  table[row sep=crcr]{%
2.4225402142989	2.46018632662555\\
2.51205643688159	2.45742975989243\\
2.58715310402205	2.45032443935752\\
2.65088726236523	2.44017747558856\\
2.70557307792621	2.4278666981761\\
2.75297624258242	2.41398036768896\\
2.79445592127621	2.39890929225061\\
2.83106778148485	2.38290771886248\\
2.86363880157304	2.36613373705743\\
2.89282178268163	2.34867615047899\\
2.91913526192633	2.33057230571732\\
2.94299286602794	2.31181977576321\\
2.9647249495008	2.29238376667649\\
2.98459451106734	2.27220144467588\\
3.00280877580309	2.25118393410117\\
3.01952739229601	2.22921642937336\\
3.03486786757133	2.20615664079546\\
3.04890860593242	2.18183161751885\\
3.06168969865664	2.15603283440837\\
3.07321140170315	2.12850927126795\\
3.08343001054312	2.0989580327608\\
3.09225056289839	2.06701183348295\\
3.09951542959146	2.03222237847955\\
3.10498733118484	1.99403827093061\\
3.10832455512295	1.95177553164861\\
3.10904501145278	1.90457806492188\\
3.10647405557169	1.85136439391878\\
3.09966843494282	1.79075568108779\\
3.08730489620387	1.72097851095623\\
3.06751649839952	1.63973451932123\\
3.03765237194324	1.54402890082052\\
2.99392870120542	1.42995433931523\\
2.93093549525602	1.29244386400949\\
2.84098235827299	1.12505312272658\\
2.71335191405367	0.919943702626728\\
2.53377842768651	0.668462503356907\\
2.28502042078931	0.363048770166131\\
1.95020895826046	0.00136628808190746\\
1.52075402939387	-0.407433008979272\\
1.00742697991319	-0.837777316427742\\
0.446142849925326	-1.25106537212028\\
-0.111707244002825	-1.60996241837187\\
-0.620392649297839	-1.89356900100245\\
-1.05487621394852	-2.10098517856431\\
-1.41084644802164	-2.24401564095483\\
-1.69636065488845	-2.33820047087083\\
-1.92385592987469	-2.39753995321719\\
-2.10560952422855	-2.43280258185524\\
-2.25198836060394	-2.45165586010096\\
-2.37115351288484	-2.45935790680001\\
-2.46932748905164	-2.45945525895173\\
-2.55119603512821	-2.45432688174761\\
-2.62028102094775	-2.44556815644518\\
-2.67923988611994	-2.43425027528856\\
-2.73009188392266	-2.42109225869469\\
-2.77438441764409	-2.40657438532166\\
-2.81331400459103	-2.39101305145171\\
-2.84781404185429	-2.37461034956898\\
-2.8786186291562	-2.35748701566959\\
-2.90630918515096	-2.33970433405003\\
-2.93134866004977	-2.32127860593696\\
-2.95410673594858	-2.30219050947069\\
-2.97487839752585	-2.2823908489408\\
-2.99389753846539	-2.26180364522069\\
-3.0113467549276	-2.2403271507598\\
-3.02736410100977	-2.21783311276133\\
-3.0420472940977	-2.19416441213581\\
-3.0554556237838	-2.16913104219236\\
-3.0676096069072	-2.1425042361312\\
-3.07848821597365	-2.11400838576039\\
-3.08802325950052	-2.08331019481765\\
-3.09609017451912	-2.05000425497905\\
-3.10249405386319	-2.01359389111321\\
-3.1069491015482	-1.97346565571904\\
-3.10904877997025	-1.92885521122919\\
-3.10822251951847	-1.87880146499304\\
-3.10367276308135	-1.8220846630784\\
-3.09428297552815	-1.75714270577492\\
-3.07848264028545	-1.68195840127306\\
-3.05404884139697	-1.59390941537601\\
-3.01781608985064	-1.48957429658994\\
-2.96525931255143	-1.36449725176504\\
-2.88991926847882	-1.21294324366504\\
-2.78268341530166	-1.02774855362424\\
-2.63108638333475	-0.800533386004224\\
-2.41917822405782	-0.5228320884181\\
-2.12922877244577	-0.189017812821461\\
-1.7472170033862	0.198338271800076\\
-1.27289457392451	0.622144237570358\\
-0.729699692373591	1.04912960977973\\
-0.16354907825766	1.43911799032485\\
0.374313533081504	1.76164762221984\\
0.847581194730859	2.00624311026009\\
1.24235269675514	2.17958156815997\\
1.56166558886184	2.29626968605072\\
1.816559003061	2.37147371815747\\
2.01974994150058	2.4176362919499\\
2.18266083639468	2.44390408090305\\
2.31454294060768	2.45664623494165\\
2.4225402142989	2.46018632662555\\
};
\addplot [color=mycolor1, forget plot]
  table[row sep=crcr]{%
0.915533679617734	3.88713310961483\\
1.09852957586721	3.8813657567245\\
1.27346356547106	3.86469800603206\\
1.44070502679724	3.83796895114571\\
1.60064309186958	3.80187211215185\\
1.75366660945792	3.75696259991024\\
1.90014859260987	3.70366536152442\\
2.04043405159741	3.64228356855361\\
2.1748302425219	3.57300656400907\\
2.30359849977727	3.49591705144124\\
2.42694694982747	3.4109974107844\\
2.54502351717723	3.31813518200184\\
2.65790873062758	3.21712788786888\\
2.76560792312149	3.10768748793501\\
2.86804249941612	2.9894448810189\\
2.96504003340893	2.86195501551079\\
3.05632306563367	2.72470333476827\\
3.14149661968498	2.57711448481228\\
3.22003466697259	2.41856444310393\\
3.29126606904786	2.24839747987543\\
3.35436094428586	2.06594961004835\\
3.40831896547524	1.87058038140072\\
3.45196180628364	1.66171488462126\\
3.48393279307768	1.43889762943759\\
3.50270769798084	1.20185922764917\\
3.50662134749873	0.950595448240291\\
3.49391501100189	0.685455972113881\\
3.46280893822414	0.407237005046545\\
3.41160242848525	0.117268009256084\\
3.3388000232035	-0.182521169850204\\
3.24325677681965	-0.489569388934429\\
3.12432874896352	-0.800706281557046\\
2.98200847083346	-1.11223998463782\\
2.81702153450226	-1.42011406428428\\
2.63086202041295	-1.72012409561288\\
2.42575242199516	-2.0081703800757\\
2.20452699361871	-2.28051288937255\\
1.97045256789499	-2.53399172221552\\
1.72701316031923	-2.76618270743753\\
1.47769023896395	-2.97547140947847\\
1.22576810427838	-3.16104517817101\\
0.974185079409573	-3.32281700200052\\
0.725439687995681	-3.46130334842394\\
0.481550323893453	-3.57748007529576\\
0.244059467334431	-3.67263724172967\\
0.014069969409949	-3.74824762771832\\
-0.207699290829558	-3.80585726085608\\
-0.420849176249404	-3.84700075283503\\
-0.625235479465544	-3.87314040851484\\
-0.82090867815256	-3.88562588125512\\
-1.00806102171395	-3.8856702655477\\
-1.18698195644625	-3.8743385047252\\
-1.35802174485055	-3.85254446054366\\
-1.52156247283532	-3.82105366033698\\
-1.67799532803701	-3.78048942963511\\
-1.82770294948821	-3.73134074045006\\
-1.97104569765303	-3.67397062098299\\
-2.10835080773941	-3.60862437772012\\
-2.23990352580227	-3.53543718918442\\
-2.36593946219093	-3.45444086222352\\
-2.48663751863826	-3.36556971807446\\
-2.60211285033001	-3.26866571656902\\
-2.71240941474592	-3.16348305042861\\
-2.81749174097168	-3.04969256296791\\
-2.91723563586649	-2.92688647478541\\
-3.01141763956755	-2.79458405898928\\
-3.09970316870044	-2.65223908805077\\
-3.18163346197881	-2.49925009167618\\
-3.25661169457858	-2.3349747094033\\
-3.32388898286324	-2.15874967602557\\
-3.3825514872278	-1.96991820363688\\
-3.43151045682256	-1.76786665163131\\
-3.46949784105365	-1.55207229500711\\
-3.49507096764294	-1.32216355405645\\
-3.50663062600307	-1.07799303572134\\
-3.5024574558465	-0.819721949421519\\
-3.4807714497124	-0.547911752619291\\
-3.43981815001576	-0.263615298723845\\
-3.37798226858461	0.0315442947739052\\
-3.29392471123411	0.335322455088922\\
-3.18673262583612	0.644846778872989\\
-3.05606523351181	0.956670804307634\\
-2.90227288426035	1.26689664545942\\
-2.72646550439925	1.57136383997961\\
-2.53051133047817	1.86588772527822\\
-2.31695772449058	2.1465178955073\\
-2.08888061230601	2.40978030367758\\
-1.84968336974496	2.65286836215824\\
-1.60287524639162	2.87375881821858\\
-1.35186094488384	3.07124383893494\\
-1.0997670197766	3.24488656013633\\
-0.849320137476856	3.39491889635324\\
-0.602780744122213	3.52210551505497\\
-0.361926436947026	3.62759693510459\\
-0.128073852783262	3.71278975837447\\
0.0978738583301783	3.77920553430087\\
0.315365906003484	3.82839362228991\\
0.524138523843202	3.86185871657206\\
0.724151960402994	3.88101071234976\\
0.915533679617733	3.88713310961483\\
};
\addplot [color=mycolor2, forget plot]
  table[row sep=crcr]{%
2.4299081702385	2.54116882702592\\
2.50787494271559	2.53876558816946\\
2.57366262941401	2.53253896978367\\
2.62983071977584	2.52359475287463\\
2.67831546206082	2.51267833276714\\
2.72059573800434	2.50029126138849\\
2.75781232015285	2.48676781509864\\
2.79085365342427	2.47232557052543\\
2.82041782125844	2.45709897920762\\
2.84705757465151	2.44116170019426\\
2.87121325252423	2.42454138381536\\
2.89323696401346	2.40722927966503\\
2.91341038049567	2.389186191406\\
2.93195776796347	2.37034574422184\\
2.94905538221762	2.35061555615369\\
2.96483798082337	2.32987664103611\\
2.97940292712109	2.30798117168998\\
2.9928121358354	2.28474856519831\\
3.00509190739402	2.25995969292994\\
3.01623049197373	2.23334884492343\\
3.02617298565802	2.20459286898918\\
3.03481285392039	2.17329663289134\\
3.04197895192468	2.13897358873028\\
3.04741629281838	2.10101970552392\\
3.05075789123247	2.05867831603404\\
3.05148360533599	2.01099241587134\\
3.04885975085121	1.95673956443917\\
3.04184997123927	1.89434269996271\\
3.02898288471224	1.82174796300119\\
3.00815480790283	1.73625854312748\\
2.97633622607303	1.63431340182052\\
2.9291405632774	1.51120649421977\\
2.86021201401299	1.36076848408523\\
2.76042381261721	1.17510775670447\\
2.61702285655303	0.944686228955867\\
2.41326229334755	0.659356991790123\\
2.12990038377572	0.311457516252769\\
1.75089642057547	-0.0980187258714269\\
1.27463368388592	-0.551510957104492\\
0.725603479611226	-1.01200187265938\\
0.153423684980919	-1.43355393844155\\
-0.386895878096306	-1.781373752186\\
-0.857807998042484	-2.04405169615064\\
-1.24655035727424	-2.22970197490458\\
-1.55802675628121	-2.35488668960367\\
-1.80474892698442	-2.4362870199259\\
-2.00026827854185	-2.48728888963669\\
-2.15635972917608	-2.51757183527422\\
-2.28235123026508	-2.53379703128516\\
-2.3853325708139	-2.54045052172632\\
-2.47060400784085	-2.54053264189128\\
-2.54211426479498	-2.53605092303616\\
-2.60281571015797	-2.52835313116987\\
-2.65493194885777	-2.51834705799954\\
-2.70015307733974	-2.50664450911494\\
-2.7397763721719	-2.49365571111631\\
-2.77480739784914	-2.47965149699101\\
-2.80603286524599	-2.46480449290115\\
-2.83407341333794	-2.44921650261795\\
-2.85942208340816	-2.43293670021977\\
-2.8824725214897	-2.4159735910477\\
-2.90353972275058	-2.39830264290297\\
-2.92287527559821	-2.37987080314453\\
-2.94067846079959	-2.36059866261591\\
-2.95710412985306	-2.34038071534023\\
-2.97226796853908	-2.31908393637231\\
-2.98624950341748	-2.29654472067452\\
-2.99909299838002	-2.27256406567429\\
-3.01080618748486	-2.24690071671462\\
-3.02135657185462	-2.21926180642577\\
-3.03066474070404	-2.18929028214403\\
-3.03859381758708	-2.15654809985281\\
-3.04493362220341	-2.12049372869982\\
-3.0493773843031	-2.08045190262606\\
-3.05148770902473	-2.03557270286934\\
-3.05064675651646	-1.98477586828778\\
-3.04598293737177	-1.92667462167759\\
-3.03626237354422	-1.859471247099\\
-3.01972734727122	-1.78081438888453\\
-2.99385548589297	-1.68760656457908\\
-2.95500305891746	-1.57575272729209\\
-2.89788792633696	-1.43985494402061\\
-2.81487854563123	-1.27290378155379\\
-2.6951315525663	-1.06613497952922\\
-2.52387147871502	-0.809478334678318\\
-2.2827155102493	-0.493463210432633\\
-1.95295399038066	-0.113790440084776\\
-1.52408987804222	0.321166530106526\\
-1.0065467764534	0.783769472801461\\
-0.438781182587107	1.23030599286926\\
0.12365660080417	1.61796478898512\\
0.632364709248575	1.92317878057828\\
1.06242978314545	2.14554424480842\\
1.41124442532834	2.29875214256913\\
1.68861198838555	2.40013228843616\\
1.9081169061554	2.46490185274609\\
2.08260241895416	2.50454347441469\\
2.22262737845324	2.52711937482361\\
2.33635043410676	2.5381045734392\\
2.4299081702385	2.54116882702592\\
};
\addplot [color=mycolor1, forget plot]
  table[row sep=crcr]{%
0.956492163373761	3.77412205831772\\
1.13740076407384	3.76842378481052\\
1.30974206413514	3.75200626229156\\
1.47393318508464	3.72576781510803\\
1.6304120425485	3.69045468628328\\
1.77961572647687	3.64666912332949\\
1.9219640807902	3.59487871776386\\
2.05784721601137	3.53542590681148\\
2.18761584721043	3.46853695485647\\
2.3115735180329	3.39433003513442\\
2.42996992777755	3.31282225379333\\
2.54299471316223	3.22393562312583\\
2.65077114829901	3.12750212021302\\
2.75334931928612	3.02326808008332\\
2.85069841099577	2.91089828469517\\
2.94269782332709	2.78998023422623\\
3.02912692577029	2.66002923679982\\
3.10965337973529	2.52049513581231\\
3.18382012942104	2.37077171518533\\
3.25103141066373	2.21021007956218\\
3.31053848396032	2.03813758337323\\
3.36142629425119	1.85388414361534\\
3.40260292041893	1.65681794814161\\
3.43279450382072	1.44639255212235\\
3.45054929220922	1.22220697421679\\
3.45425537515231	0.984079443041352\\
3.44217736834696	0.732133662143365\\
3.41251732322385	0.466893666768205\\
3.36350395342279	0.189379534222988\\
3.29351130938284	-0.0988082424650076\\
3.20120293909346	-0.395432053992662\\
3.08569057875153	-0.697606752326291\\
2.94668872432899	-1.00185257784537\\
2.78464033521322	-1.30422162394327\\
2.60078732071264	-1.60049568740128\\
2.3971646973498	-1.88643777475042\\
2.17650964895159	-2.15806518450049\\
1.94209347366312	-2.41190418402993\\
1.69750048694484	-2.64518837669219\\
1.44638803650607	-2.85597474469135\\
1.19226280410582	-3.04316905185557\\
0.938301050414134	-3.20646988028905\\
0.687227916835214	-3.34625299598594\\
0.441257847553226	-3.46342270785044\\
0.202088153572837	-3.55925509628672\\
-0.0290677594005788	-3.63525195926824\\
-0.251422224679322	-3.69301689447977\\
-0.464543240306113	-3.73415822607492\\
-0.668282385146043	-3.76021855937426\\
-0.862707241732954	-3.77262779087261\\
-1.04804191430491	-3.77267508472103\\
-1.22461703881608	-3.76149514011564\\
-1.39282923260587	-3.74006453499351\\
-1.55310909471179	-3.70920467702781\\
-1.70589647410159	-3.66958869119952\\
-1.85162161649841	-3.62175029923513\\
-1.99069085949854	-3.56609334832289\\
-2.12347568568543	-3.50290111685884\\
-2.25030411081334	-3.43234487799468\\
-2.37145354827313	-3.35449146089099\\
-2.48714443705732	-3.26930973983096\\
-2.59753404351027	-3.17667612588477\\
-2.70270994881875	-3.07637925481466\\
-2.80268282012914	-2.96812417560813\\
-2.89737814214477	-2.85153646114373\\
-2.98662666980141	-2.72616679860698\\
-3.07015346630092	-2.59149678270892\\
-3.14756553372927	-2.44694683662406\\
-3.21833824985063	-2.29188742545607\\
-3.2818011233653	-2.12565499683434\\
-3.33712380274189	-1.94757435838472\\
-3.38330385000719	-1.75698943172239\\
-3.41915853623843	-1.55330441875053\\
-3.44332381209519	-1.33603724016791\\
-3.45426457288477	-1.10488646486432\\
-3.45030118873616	-0.859811606680412\\
-3.42965768085947	-0.601124392539394\\
-3.3905364155057	-0.329585284462418\\
-3.33122218023402	-0.0464953272455659\\
-3.25021449801282	0.24623106449233\\
-3.14638090878713	0.546030203512032\\
-3.01911638353627	0.849708542167321\\
-2.86848679485848	1.15353175153547\\
-2.69533015530887	1.45338808905218\\
-2.50129095670289	1.74501628182254\\
-2.28877187934261	2.02427253179527\\
-2.0608022087767	2.28739951598753\\
-1.82083942887876	2.53125710237021\\
-1.57253412315359	2.75348176980381\\
-1.31949403672393	2.95255719857406\\
-1.06507961242214	3.12779689068709\\
-0.812252728524991	3.27925516934623\\
-0.563487029195763	3.40759168605377\\
-0.320736368861388	3.51391591728976\\
-0.0854500579454544	3.59963385723916\\
0.141379654705287	3.66631205839245\\
0.359152010999243	3.71556690270327\\
0.567585643337569	3.74898109802236\\
0.766648128212963	3.76804547197061\\
0.95649216337376	3.77412205831772\\
};
\addplot [color=mycolor2, forget plot]
  table[row sep=crcr]{%
2.42663064814023	2.60785811664781\\
2.49468174008971	2.60575870030039\\
2.55239939390668	2.60029427736913\\
2.60193795001227	2.59240432776687\\
2.64492702754717	2.58272400429261\\
2.68261278465861	2.57168187990689\\
2.7159583805664	2.55956400668035\\
2.74571577241066	2.54655620235487\\
2.7724773872352	2.53277211909779\\
2.79671359330112	2.51827189344838\\
2.81880006068362	2.50307443871288\\
2.83903783407902	2.48716533789187\\
2.85766806721563	2.47050158520723\\
2.87488276140725	2.4530139569977\\
2.89083242168568	2.43460747402141\\
2.90563123071895	2.41516018695922\\
2.91936009944686	2.3945203352042\\
2.93206775108925	2.37250176715401\\
2.94376980445504	2.34887734481953\\
2.95444561662831	2.32336986413215\\
2.96403239338554	2.29563977935101\\
2.97241573727883	2.26526869274245\\
2.97941531880948	2.23173711437162\\
2.98476363463382	2.19439434819838\\
2.98807471584772	2.15241743103571\\
2.98879794433029	2.10475472069131\\
2.9861494745245	2.05004785320334\\
2.97900961012992	1.98652323679204\\
2.96576810702092	1.91184105925449\\
2.94408993137816	1.82288661116335\\
2.91056126141298	1.71548807070542\\
2.86016234749081	1.58405422656099\\
2.7855138056449	1.42116462835929\\
2.67590028700184	1.21725945971396\\
2.51630693803723	0.96085555711886\\
2.28733937387612	0.640253754606333\\
1.96811113829447	0.248309634107519\\
1.54507742635003	-0.208816149717553\\
1.02660002081288	-0.702678714632209\\
0.452566596117432	-1.18438280496429\\
-0.116845507547197	-1.60413285410775\\
-0.629111139445965	-1.93406654496179\\
-1.05822398949492	-2.17352895352723\\
-1.40273388675657	-2.33810426546856\\
-1.67410244944011	-2.44718986880552\\
-1.88715984915978	-2.51749020682078\\
-2.05545301269843	-2.56139108298409\\
-2.18985564718869	-2.58746499239997\\
-2.29861963965018	-2.60146966362994\\
-2.3878659616402	-2.60723369569702\\
-2.46210796987436	-2.60730327728648\\
-2.52468330161363	-2.60337981840676\\
-2.57807879986109	-2.59660701049621\\
-2.62416564382979	-2.58775721560065\\
-2.66436694812378	-2.57735252840745\\
-2.69977684240484	-2.56574383339848\\
-2.73124533318201	-2.553162806889\\
-2.75943915518276	-2.53975635217672\\
-2.7848857327831	-2.52560948615007\\
-2.80800517492384	-2.51076050824391\\
-2.82913370106633	-2.49521089982419\\
-2.84854084410645	-2.47893151898856\\
-2.86644204919181	-2.4618660817546\\
-2.88300777859459	-2.44393253682758\\
-2.89836986805199	-2.42502267264894\\
-2.91262560739631	-2.4050000935758\\
-2.92583980018095	-2.3836965333914\\
-2.93804486378162	-2.36090631348358\\
-2.94923883663002	-2.33637857752089\\
-2.95938093476062	-2.30980672078445\\
-2.96838401080529	-2.28081415192535\\
-2.97610286612645	-2.24893513973113\\
-2.98231677778854	-2.21358895431813\\
-2.98670371295673	-2.17404473597005\\
-2.98880233513725	-2.12937341183341\\
-2.9879557777621	-2.07838139629176\\
-2.98322783606726	-2.01951860794754\\
-2.97327707002324	-1.95075043755831\\
-2.95616649574139	-1.86937996465697\\
-2.92907538160684	-1.77180428933348\\
-2.88786595675086	-1.65319172469644\\
-2.8264483041122	-1.50708707923459\\
-2.73590657652846	-1.32502100773348\\
-2.60347279280227	-1.09638260105197\\
-2.41183305744973	-0.809216968254309\\
-2.1401807378859	-0.453250634669411\\
-1.76974184690201	-0.0267037097425988\\
-1.29601116923746	0.453883472638125\\
-0.742939856829561	0.94845834101123\\
-0.163493109629213	1.40443132492638\\
0.382433223343916	1.78091893864817\\
0.854599636609413	2.06434377873827\\
1.24050348687811	2.26394711433531\\
1.54663701409861	2.39844076975961\\
1.78701637795491	2.48631381412283\\
1.976156548392	2.5421269616579\\
2.12631849151321	2.57624209357592\\
2.24701642358105	2.59570022049178\\
2.34536835067793	2.60519857167455\\
2.42663064814023	2.60785811664781\\
};
\addplot [color=mycolor1, forget plot]
  table[row sep=crcr]{%
1.01008260529783	3.64344252016072\\
1.18863581493154	3.63782274867468\\
1.35795691941703	3.62169707538814\\
1.51853505039252	3.596039966336\\
1.67087992678567	3.56166356816446\\
1.81549807204968	3.51922733307875\\
1.95287527571584	3.46924908623552\\
2.08346376072164	3.41211623186198\\
2.20767273034483	3.34809628854992\\
2.32586118968549	3.27734630195263\\
2.43833213811751	3.19992093592348\\
2.54532739923415	3.11577922133519\\
2.64702249179135	3.02479007188445\\
2.74352105303856	2.92673678023746\\
2.83484841210704	2.82132080427773\\
2.92094398579676	2.70816525751865\\
3.0016522446899	2.58681864342859\\
3.07671208976739	2.45675953219305\\
3.14574460866834	2.31740307964802\\
3.20823937192019	2.16811053679873\\
3.26353971445634	2.00820319128623\\
3.31082786321983	1.83698250119168\\
3.34911135512656	1.6537584832036\\
3.37721296910294	1.45788861870122\\
3.39376736983027	1.2488295048429\\
3.39722876537074	1.02620299783167\\
3.38589494341176	0.78987740221178\\
3.35795373804042	0.540062057875569\\
3.31155776316144	0.277410246225679\\
3.24493144135332	0.00312072118151988\\
3.15651027593232	-0.280977070906215\\
3.04510563913755	-0.572373575603527\\
2.9100796338738	-0.86788681152822\\
2.75150571136334	-1.16374511301673\\
2.57028488793736	-1.45575285888894\\
2.36818821165584	-1.73953096665872\\
2.14780582530322	-2.01080537910998\\
1.91240056029732	-2.26570195682995\\
1.66568466521956	-2.50100121158749\\
1.41155505050136	-2.71431398248399\\
1.15382947508208	-2.90415714629118\\
0.896021605662616	-3.0699303069577\\
0.641179677015408	-3.2118128162806\\
0.391797123675682	-3.33061064331799\\
0.149789393445159	-3.42758371932104\\
-0.0834777132379029	-3.50427881045662\\
-0.307125970182486	-3.56238432952652\\
-0.520685487747275	-3.60361492375731\\
-0.724011280286682	-3.62962700154336\\
-0.9172043512485	-3.64196214467776\\
-1.10054218441131	-3.64201332125204\\
-1.27442062080241	-3.63100834590335\\
-1.43930716846449	-3.6100054885638\\
-1.59570468074986	-3.57989701029662\\
-1.7441238231301	-3.54141737984803\\
-1.88506261657972	-3.49515381977946\\
-2.01899143174947	-3.44155757175398\\
-2.14634199782038	-3.38095484324088\\
-2.26749921159166	-3.31355682009303\\
-2.3827947459713	-3.23946843078057\\
-2.49250164354792	-3.15869575998736\\
-2.59682923424688	-3.07115216037838\\
-2.69591783775042	-2.97666322589441\\
-2.78983280734277	-2.87497088806156\\
-2.87855755091879	-2.76573699535562\\
-2.96198523827919	-2.64854684907848\\
-3.03990898554357	-2.52291331009226\\
-3.11201041529539	-2.38828226978606\\
-3.17784664758319	-2.24404050335959\\
-3.23683601067323	-2.08952719546202\\
-3.28824310589124	-1.92405073777965\\
-3.33116435594089	-1.74691271601883\\
-3.36451584528456	-1.5574412687282\\
-3.38702614099515	-1.35503610420793\\
-3.39723783527421	-1.13922723364808\\
-3.39352266483723	-0.909748678869892\\
-3.37411599074839	-0.666626750771519\\
-3.33717673598115	-0.410279694791303\\
-3.28087795806842	-0.141621445498978\\
-3.20353036334593	0.1378428090178\\
-3.1037356896824	0.425946667307661\\
-2.98055904483278	0.71983904773537\\
-2.83370016585106	1.01602761806062\\
-2.66363574932995	1.31050293427032\\
-2.47170207179493	1.59894374716538\\
-2.2600922576455	1.87698593269571\\
-2.03175656787866	2.14051997763506\\
-1.79021401324247	2.38597142907599\\
-1.53930320431057	2.6105200170064\\
-1.28291278747925	2.81222659483883\\
-1.02473300661721	2.99005791525733\\
-0.768060490634631	3.14382020870676\\
-0.515672839772723	3.27402717373248\\
-0.269773766994071	3.3817334889122\\
-0.0319978264617681	3.46836230050207\\
0.196542153408091	3.5355475791765\\
0.415182535501067	3.58500331369573\\
0.623626184417784	3.61842374351327\\
0.821860461073582	3.63741338152024\\
1.01008260529783	3.64344252016072\\
};
\addplot [color=mycolor2, forget plot]
  table[row sep=crcr]{%
2.41476033068263	2.66229893813066\\
2.47426051877808	2.66046188142822\\
2.52496086636225	2.65566055223636\\
2.56868246992092	2.64869594138342\\
2.60680336812849	2.64011084184641\\
2.64037873090607	2.63027218455783\\
2.67022567673547	2.61942490430711\\
2.69698361012018	2.60772747837609\\
2.72115753202921	2.59527550289921\\
2.74314940113939	2.58211732487685\\
2.76328100985426	2.56826428139133\\
2.78181074564567	2.55369717031087\\
2.79894586309751	2.53836997930367\\
2.81485137716571	2.52221150387158\\
2.82965632375841	2.50512521041166\\
2.84345786519026	2.48698749574624\\
2.85632350665693	2.46764432527621\\
2.86829150595093	2.44690607167524\\
2.87936937632061	2.42454020182558\\
2.88953017480314	2.40026124729676\\
2.89870600245254	2.37371721339243\\
2.90677777235215	2.34447119331895\\
2.91355975624061	2.3119763996539\\
2.91877659097508	2.27554202075635\\
2.92202913523956	2.23428613260507\\
2.92274352947618	2.18707017433945\\
2.92009457053138	2.13240700466926\\
2.91288935500511	2.06833106433985\\
2.89938904279321	1.99221462611844\\
2.87703433081253	1.90050926551449\\
2.84202334551324	1.7883898674507\\
2.78867310206417	1.64929041000972\\
2.70849742357398	1.47437617373549\\
2.58902294051886	1.25216810687731\\
2.41272157600675	0.968961378919757\\
2.15740379551078	0.611488822489571\\
1.80110489897901	0.174007585924341\\
1.33491064908747	-0.329868142581518\\
0.780220632640025	-0.858437119498023\\
0.192569509838792	-1.35182905347143\\
-0.362109379256176	-1.76094095913193\\
-0.839173355705796	-2.06834567290477\\
-1.22537900019186	-2.28393895297461\\
-1.52855377354098	-2.42880161679264\\
-1.7643048112528	-2.52358245666273\\
-1.94827238565339	-2.58428805072776\\
-2.09334646337523	-2.62213200702853\\
-2.20933414641352	-2.6446321006958\\
-2.30345686127722	-2.65674983100523\\
-2.38097929309355	-2.66175499013238\\
-2.44574577706084	-2.66181416277486\\
-2.50058475857612	-2.65837441759815\\
-2.54759911312288	-2.65240981638321\\
-2.58837062757489	-2.6445796382357\\
-2.62410339759821	-2.63533052752175\\
-2.65572470420992	-2.62496301407602\\
-2.68395647334647	-2.61367522554978\\
-2.70936633781276	-2.60159182182315\\
-2.73240445392251	-2.58878320443428\\
-2.75343026534433	-2.57527820089079\\
-2.7727320785939	-2.5610722598607\\
-2.79054141304474	-2.54613245103991\\
-2.80704347033075	-2.53040007914688\\
-2.82238463652207	-2.51379139393238\\
-2.83667761962244	-2.49619664361787\\
-2.85000458909476	-2.47747753592046\\
-2.86241848981296	-2.45746300903009\\
-2.8739425229001	-2.43594305053212\\
-2.8845675945778	-2.41266011222196\\
-2.89424730217836	-2.38729742666149\\
-2.90288971521271	-2.35946320289597\\
-2.91034476235518	-2.32866921595343\\
-2.91638536501737	-2.29430163791519\\
-2.92067942563984	-2.25558098569283\\
-2.92274815937726	-2.2115066359581\\
-2.92190368942666	-2.16077928146949\\
-2.91715473495383	-2.10169173654886\\
-2.90706273852554	-2.03197446090965\\
-2.88952074197548	-1.94857728422476\\
-2.86141267148656	-1.84736477030459\\
-2.81809240408791	-1.72270557611584\\
-2.75260969633187	-1.56696416314431\\
-2.65464170389844	-1.37000355082238\\
-2.50927856523459	-1.11908505286392\\
-2.29642420195864	-0.800165548559755\\
-1.99294888695119	-0.402505596279578\\
-1.58128810879217	0.0715688300337349\\
-1.06580695917224	0.594675098125325\\
-0.486211284158695	1.11321633820615\\
0.0924395664717935	1.56881374821226\\
0.611840162879801	1.92719254817372\\
1.0434617323806	2.18638558471255\\
1.38643068898744	2.36383239917408\\
1.65383033355687	2.4813303299899\\
1.86188078388727	2.55739255487381\\
2.02499011504009	2.60552571919054\\
2.1544717399875	2.63494165601582\\
2.25876065635701	2.65175283414999\\
2.34402491715779	2.65998552639311\\
2.41476033068263	2.66229893813066\\
};
\addplot [color=mycolor1, forget plot]
  table[row sep=crcr]{%
1.07692127260879	3.503666468077\\
1.25300918281728	3.4981296553274\\
1.41903449901791	3.48232295184664\\
1.57559066828564	3.45731328419297\\
1.72328858336827	3.42399002181016\\
1.86273010346507	3.38307711742861\\
1.99448928975369	3.33514683294116\\
2.11909938364559	3.28063347005673\\
2.23704386587854	3.21984614712431\\
2.3487502458256	3.15298010112185\\
2.45458550625696	3.0801262906107\\
2.55485235559185	3.00127927236677\\
2.64978561667084	2.91634345581436\\
2.7395482141668	2.82513793286865\\
2.82422632106993	2.72740015821513\\
2.9038232990936	2.62278883421946\\
2.97825213075303	2.51088645081344\\
3.04732610688787	2.3912020576363\\
3.11074762068764	2.26317501621477\\
3.1680950512474	2.12618070521954\\
3.21880792679199	1.97953943919817\\
3.26217087868375	1.82253020947976\\
3.29729737848994	1.65441124709938\\
3.3231149407673	1.47444979335821\\
3.33835441273393	1.28196374684973\\
3.34154716145888	1.07637786772928\\
3.33103533186351	0.857296700760661\\
3.30500166178559	0.624594976529345\\
3.26152616108178	0.378523569503026\\
3.19867657908521	0.119824829489567\\
3.11463706764755	-0.150154690727738\\
3.00787389465825	-0.429372646421058\\
2.87732813737469	-0.715044489263038\\
2.72261396138454	-1.00366786439617\\
2.5441902657782	-1.29113798111643\\
2.34346776726496	-1.57295980618841\\
2.12281774905249	-1.84454133989462\\
1.88546479782922	-2.1015292855393\\
1.63527087347818	-2.3401328052854\\
1.37644393510446	-2.55738011708881\\
1.11322143137332	-2.7512680245816\\
0.84958107671147	-2.92079049449001\\
0.589018967382472	-3.06585923936307\\
0.334414531303806	-3.18714802934969\\
0.0879812010941502	-3.28589915315193\\
-0.148712951538626	-3.36372648548615\\
-0.374677009102123	-3.42243953622834\\
-0.589403634093858	-3.46390144626262\\
-0.792768060091346	-3.48992439196961\\
-0.984931880726181	-3.50219956417983\\
-1.16625854347781	-3.50225568023596\\
-1.33724335614839	-3.49143907338669\\
-1.49845810192408	-3.47090887109396\\
-1.65050882723953	-3.44164188608603\\
-1.7940046862125	-3.40444311702556\\
-1.92953558294803	-3.35995892643898\\
-2.05765650621834	-3.30869092286519\\
-2.17887673587479	-3.25100930325602\\
-2.29365241756244	-3.18716493616498\\
-2.40238129872309	-3.11729982803371\\
-2.50539867091555	-3.04145585700177\\
-2.60297376494856	-2.95958181897214\\
-2.69530599936835	-2.87153894005791\\
-2.7825205972781	-2.77710509254552\\
-2.86466317138413	-2.67597802777069\\
-2.94169294415755	-2.56777802516605\\
-3.01347433245063	-2.45205046662173\\
-3.07976670000611	-2.32826899249109\\
-3.14021218759816	-2.1958400922185\\
-3.19432169577425	-2.05411023834092\\
-3.24145935384523	-1.90237699139921\\
-3.28082620369488	-1.73990587678817\\
-3.31144440682784	-1.5659552316216\\
-3.33214409411165	-1.37981157025044\\
-3.34155604505053	-1.18083819109787\\
-3.33811467655641	-0.968539532429108\\
-3.32007719888221	-0.74264287264097\\
-3.28556593487212	-0.503196977428685\\
-3.23264112331153	-0.250683848211803\\
-3.15941020074127	0.0138653435980458\\
-3.06417561474539	0.28876538350546\\
-2.94561596149961	0.571604778973339\\
-2.80298488238984	0.859231691711481\\
-2.63630059119198	1.14782198589606\\
-2.44648999609513	1.43304350653639\\
-2.23545010493017	1.71031228038472\\
-2.00599954803418	1.97511307778672\\
-1.76171434840376	2.22333637142549\\
-1.50666874172212	2.45157480806186\\
-1.24512443377309	2.65732977749459\\
-0.981221667747762	2.83910042158708\\
-0.718719854451677	2.99635515654093\\
-0.46081804338262	3.12940944708454\\
-0.21006397215261	3.23924643367439\\
0.0316577094427846	3.32731793722812\\
0.263078101244961	3.39535570033236\\
0.483460592181638	3.44521145172643\\
0.692501675998655	3.47873364607262\\
0.890231168036171	3.49768077510133\\
1.07692127260879	3.503666468077\\
};
\addplot [color=mycolor2, forget plot]
  table[row sep=crcr]{%
2.39586003108064	2.70645250921799\\
2.44793921405715	2.7048434128308\\
2.4925056321078	2.7006219481352\\
2.53110320230635	2.69447266868954\\
2.56490106181823	2.68686034672239\\
2.59479573023351	2.67809952494233\\
2.62148280844775	2.66839996721084\\
2.64550781899207	2.65789663089041\\
2.66730263361219	2.64666953175667\\
2.68721182411929	2.63475687548339\\
2.70551186863031	2.6221635898773\\
2.72242520658815	2.6088666094499\\
2.73813050107971	2.59481775820078\\
2.75277002945031	2.57994473802309\\
2.76645481248171	2.56415049078378\\
2.77926786061602	2.54731101771482\\
2.79126572870291	2.52927157937977\\
2.80247840160814	2.50984103858069\\
2.81290735701722	2.48878392383097\\
2.82252144152823	2.46580955589587\\
2.83124991591435	2.44055725967451\\
2.83897162304683	2.41257622901504\\
2.84549862605467	2.38129794978588\\
2.85055172359163	2.34599810652856\\
2.85372375788986	2.30574343486008\\
2.85442423280694	2.25931679290446\\
2.85179486635119	2.20511047637028\\
2.84457937343576	2.140973103094\\
2.83092059535225	2.06398901685406\\
2.80804228925792	1.97016184831735\\
2.77175053613937	1.85396984851581\\
2.71566587753892	1.70777473990774\\
2.63010038706129	1.52114193105368\\
2.50062498549661	1.28037916993507\\
2.30690789344056	0.969241119782617\\
2.02384246864085	0.572942103193158\\
1.62923691139703	0.0883858147587356\\
1.12147624928202	-0.460565639479011\\
0.537853457168564	-1.0169569262796\\
-0.0519105584188836	-1.51239220851406\\
-0.581938954267458	-1.90352395875888\\
-1.01946440750957	-2.1855663139474\\
-1.36351965659421	-2.37768436448737\\
-1.62880917406411	-2.50446722807857\\
-1.83313334167097	-2.5866211214845\\
-1.99194137580707	-2.63902632364079\\
-2.11711539293686	-2.67167858606987\\
-2.21735901291246	-2.69112320755616\\
-2.29894325902303	-2.70162521662839\\
-2.36638328199093	-2.70597802925916\\
-2.42295348268433	-2.70602847514658\\
-2.47105477977271	-2.70301025935254\\
-2.51247004307102	-2.69775504555296\\
-2.5485406354134	-2.69082683301651\\
-2.5802887900039	-2.68260834336394\\
-2.608503090363	-2.6733571722273\\
-2.6337987661433	-2.66324266322134\\
-2.65666067324506	-2.65237030238575\\
-2.67747424204231	-2.64079788561504\\
-2.6965479567738	-2.6285461408226\\
-2.71412978233766	-2.61560550394926\\
-2.73041918415139	-2.60194012103131\\
-2.74557586123964	-2.58748973704813\\
-2.75972594573443	-2.57216985006503\\
-2.77296615566563	-2.555870301796\\
-2.78536618221272	-2.53845230665008\\
-2.7969694176343	-2.51974376357569\\
-2.80779196065837	-2.49953252520825\\
-2.81781964678537	-2.47755709259567\\
-2.82700261099537	-2.453493931178\\
-2.83524655702526	-2.42694022364257\\
-2.84239941590435	-2.397390327888\\
-2.84823132379159	-2.36420340393172\\
-2.85240466733144	-2.32655847699991\\
-2.85442905636686	-2.28339141396671\\
-2.85359303053655	-2.23330561876357\\
-2.84885934088247	-2.17444432624126\\
-2.83870260145156	-2.10430683505434\\
-2.82085533591204	-2.01948395812656\\
-2.79190930965565	-1.91528136423395\\
-2.74669449271135	-1.78520149587765\\
-2.67734125288136	-1.62029135042173\\
-2.57197791015913	-1.40850591963233\\
-2.41329867603964	-1.13464870014876\\
-2.17815769848986	-0.78237514111604\\
-1.84133030932303	-0.341010065890242\\
-1.38819237899538	0.180920934736513\\
-0.835208929213798	0.742291470905776\\
-0.239044017925271	1.27593051799716\\
0.32733697181048	1.72210942956557\\
0.812891499624408	2.05729133431892\\
1.20243775255801	2.29129848572901\\
1.50487719019694	2.44781159112029\\
1.73754441727249	2.55006142206413\\
1.91740163105295	2.61582044512544\\
2.05812274414373	2.65734719705302\\
2.16991322967182	2.68274295031821\\
2.26016704717165	2.69729025276353\\
2.33420250887109	2.70443730605496\\
2.39586003108064	2.70645250921799\\
};
\addplot [color=mycolor1, forget plot]
  table[row sep=crcr]{%
1.15864607231793	3.36165970522797\\
1.33225134403357	3.35620747412833\\
1.49478440075304	3.34073934225922\\
1.64698656841843	3.3164308911847\\
1.78960780066664	3.28425830398293\\
1.9233772092579	3.24501456805803\\
2.04898332032296	3.19932723382943\\
2.16706140026213	3.14767580996821\\
2.27818568495898	3.09040768498839\\
2.38286481120998	3.02775201263799\\
2.48153914528356	2.95983135156412\\
2.57457901793167	2.88667106749531\\
2.662283112096	2.80820663612774\\
2.74487642026642	2.72428906298787\\
2.82250730652801	2.63468868961239\\
2.89524328755055	2.53909770391505\\
2.96306520076115	2.43713173325252\\
3.0258594709254	2.32833098635054\\
3.08340823466972	2.21216153951883\\
3.1353771566508	2.08801754849827\\
3.18130089851176	1.95522542486019\\
3.22056641959446	1.81305135780606\\
3.25239464595539	1.66071399346844\\
3.27582160359392	1.49740459137472\\
3.28968094303244	1.32231751320041\\
3.29259094842626	1.13469435216573\\
3.28295064710913	0.933885175825142\\
3.25895143343967	0.719429892043736\\
3.21861241279621	0.491161172627005\\
3.15984885736914	0.249327100252536\\
3.08058273494598	-0.00527375221765975\\
2.97890085895925	-0.271159770129723\\
2.85325846718984	-0.546060096078085\\
2.70271349409731	-0.826866041703794\\
2.52716107812122	-1.10967315780278\\
2.32752360217438	-1.38993850604152\\
2.10584598766703	-1.66275658210955\\
1.86525579730255	-1.92322686588386\\
1.60977454229429	-2.16685595700952\\
1.34400348706045	-2.38992078133626\\
1.07274033607123	-2.58972555591923\\
0.800598780380851	-2.76471295661684\\
0.531695425490565	-2.91442778511639\\
0.269443497665235	-3.03936427842029\\
0.0164620121028157	-3.14074562674795\\
-0.225415782718549	-3.22028458620438\\
-0.455064137494796	-3.27996256296887\\
-0.671948205727318	-3.32184873132998\\
-0.875995320702246	-3.34796651731514\\
-1.0674721369507	-3.36020494960387\\
-1.24687767699094	-3.36026720629691\\
-1.41485619350969	-3.34964708085678\\
-1.57212977585483	-3.3296246480048\\
-1.71944847164187	-3.30127396546065\\
-1.85755482158101	-3.26547743932403\\
-1.98715961416919	-3.22294310732679\\
-2.10892597843005	-3.17422239821988\\
-2.22345940187905	-3.11972688940498\\
-2.33130174685321	-3.05974325486975\\
-2.43292777207422	-2.99444603774209\\
-2.52874302175775	-2.92390816047626\\
-2.61908221912585	-2.84810925437041\\
-2.70420750336871	-2.76694199026823\\
-2.78430599176549	-2.6802166549814\\
-2.85948624561622	-2.58766426650185\\
-2.92977328352482	-2.48893857372896\\
-2.99510183219263	-2.38361735876521\\
-3.05530754827947	-2.27120356644535\\
-3.11011600320062	-2.1511269414496\\
-3.15912931956567	-2.02274707328312\\
-3.20181051523691	-1.88535904787861\\
-3.23746589169837	-1.73820329135964\\
-3.26522625343941	-1.580481664241\\
-3.28402843150976	-1.41138239503501\\
-3.29259957665535	-1.23011695549039\\
-3.2894480336971	-1.03597232183191\\
-3.27286628917489	-0.828381968701454\\
-3.24095333148943	-0.607017992352561\\
-3.19166535112224	-0.371904411158706\\
-3.12290423878371	-0.123547372835805\\
-3.0326516010854	0.136928612253899\\
-2.91915057597159	0.407657775624189\\
-2.78112754703252	0.685951486808953\\
-2.61803134182142	0.968291107210478\\
-2.43025173714252	1.25042562042498\\
-2.21926821359736	1.52758956887821\\
-1.98768139377993	1.79483020803747\\
-1.73909823593737	2.04740097255614\\
-1.47787533741939	2.28115352612145\\
-1.20876162797556	2.49285493698564\\
-0.936507428543648	2.68037437802348\\
-0.665510906127175	2.84271842506794\\
-0.39955540906315	2.97993113850398\\
-0.141661611193392	3.09290102385607\\
0.105950536635203	3.18312551744094\\
0.341814414927393	3.25247713342869\\
0.565115580825522	3.303000844366\\
0.775566323367475	3.33675669571624\\
0.973278152500676	3.35570945870127\\
1.15864607231793	3.36165970522797\\
};
\addplot [color=mycolor2, forget plot]
  table[row sep=crcr]{%
2.37097245534212	2.7420316362117\\
2.41656189004161	2.74062210000145\\
2.45573036679667	2.73691111095205\\
2.48978854289773	2.73148430336595\\
2.5197299463119	2.72473991858471\\
2.54631766025098	2.71694762527161\\
2.57014492375801	2.70828693151035\\
2.59167800844056	2.69887250480756\\
2.61128690506718	2.68877093192561\\
2.6292675082278	2.67801175474508\\
2.64585777686604	2.66659456961682\\
2.66124954648495	2.65449331573003\\
2.67559712901856	2.64165844829721\\
2.68902346449365	2.62801740138216\\
2.70162432302637	2.61347353472187\\
2.71347085468897	2.59790359057999\\
2.72461061815595	2.58115353271565\\
2.73506706218434	2.56303247644746\\
2.7448372630099	2.54330422245611\\
2.7538875075488	2.52167564799242\\
2.76214601829786	2.49778084702536\\
2.76949168257558	2.47115938582597\\
2.77573698267025	2.4412262613363\\
2.78060227007522	2.40722997517465\\
2.7836768267455	2.36819334751416\\
2.7843593695785	2.32282895919441\\
2.78176603826495	2.26941694331938\\
2.77458623091866	2.20562662620142\\
2.76085399122843	2.12825471098869\\
2.73758242954271	2.03284186876362\\
2.70017911364511	1.91312183239206\\
2.64152756712036	1.76027240469776\\
2.5506233526731	1.56203952020427\\
2.41084116723899	1.30216273183048\\
2.19869982976661	0.961482342718005\\
1.88609986295266	0.523853307612389\\
1.45188103584108	-0.00941184988821951\\
0.905011412639907	-0.600841322450176\\
0.300984103015729	-1.17697437633149\\
-0.279639614986057	-1.66500002627316\\
-0.777141108438632	-2.03230438061452\\
-1.17291841174885	-2.28752439101081\\
-1.4766410052214	-2.45715990617252\\
-1.70755387179492	-2.56752878073292\\
-1.88417489769587	-2.63854841219933\\
-2.02113006664282	-2.68374289252458\\
-2.12912778760137	-2.71191375145814\\
-2.21579746346988	-2.72872411662436\\
-2.28654930912442	-2.73783043742404\\
-2.34524379108104	-2.74161762034185\\
-2.39466718845133	-2.74166067136464\\
-2.43685815704226	-2.73901242010449\\
-2.4733298827699	-2.73438371425431\\
-2.50522156088711	-2.72825746861326\\
-2.53340252940198	-2.7209617754234\\
-2.55854463909447	-2.71271737191783\\
-2.58117315185587	-2.70366879825636\\
-2.60170296423541	-2.69390499614592\\
-2.62046466944398	-2.68347292916575\\
-2.637723477896	-2.67238647518524\\
-2.65369303268425	-2.66063201051763\\
-2.66854549995222	-2.64817157435036\\
-2.68241886767288	-2.63494415067055\\
-2.69542207384251	-2.62086535994409\\
-2.70763835613721	-2.60582566737345\\
-2.71912703456596	-2.58968705650954\\
-2.72992377983627	-2.57227796133157\\
-2.74003925935967	-2.55338607341383\\
-2.74945586473644	-2.5327484173563\\
-2.75812197675783	-2.51003778313961\\
-2.76594286938916	-2.48484416956563\\
-2.77276681926595	-2.4566492548512\\
-2.77836415205913	-2.42479095535871\\
-2.78239562171843	-2.38841368543216\\
-2.78436434544	-2.34639771944883\\
-2.78354193410604	-2.297257676396\\
-2.77885350496149	-2.23899503538068\\
-2.76869638454963	-2.16888211395647\\
-2.75065120711101	-2.08314489972321\\
-2.72101924431481	-1.976501637464\\
-2.67408675857107	-1.8415136263038\\
-2.60099339294028	-1.66775062116819\\
-2.48814654562482	-1.44097153743184\\
-2.31553640753194	-1.14312435524285\\
-2.0566699437989	-0.755346434435791\\
-1.68465037679736	-0.267851466002901\\
-1.19012994655386	0.301873451191688\\
-0.60514456145037	0.895985809499205\\
-0.00307335409498148	1.43520501048875\\
0.540852901816076	1.86392021265926\\
0.987553308803798	2.1724095677462\\
1.3351415266425	2.38127196158624\\
1.59996349938777	2.51834233826423\\
1.80163750250165	2.60698012197524\\
1.95685829931324	2.66373366513709\\
2.07821045594123	2.69954427228426\\
2.17474736069969	2.72147367820844\\
2.25289193509988	2.73406788805195\\
2.31720883710598	2.74027552804658\\
2.37097245534212	2.7420316362117\\
};
\addplot [color=mycolor1, forget plot]
  table[row sep=crcr]{%
1.25819306153321	3.22316713074098\\
1.42932448156702	3.21780042499708\\
1.58817579819832	3.2026898549361\\
1.73569527099036	3.17913584564961\\
1.87282081932095	3.14820898628053\\
2.00044794120658	3.11077272316979\\
2.11940999752583	3.06750715761604\\
2.23046714131272	3.01893166092257\\
2.33430099158643	2.96542509142497\\
2.43151287638883	2.90724308576105\\
2.52262405474737	2.84453231054009\\
2.60807677090703	2.77734179199042\\
2.68823531434906	2.70563155834315\\
2.76338647949024	2.62927888099351\\
2.83373896250298	2.54808241930898\\
2.89942131942143	2.46176458372459\\
2.96047815656954	2.36997244992145\\
3.01686424639106	2.2722775981427\\
3.06843627388874	2.16817533053999\\
3.11494193831615	2.05708385197457\\
3.15600618418592	1.93834420480503\\
3.19111444712788	1.81122204761352\\
3.21959301982294	1.67491278418725\\
3.24058703633076	1.52855210026402\\
3.2530372271061	1.37123465250037\\
3.25565761918672	1.20204443800585\\
3.24691785729109	1.02010113697681\\
3.22503587243533	0.824627215083776\\
3.18798916524724	0.615040344528044\\
3.13355565748483	0.391074048366338\\
3.05939705864913	0.152925467254844\\
2.96319746704448	-0.0985781135962522\\
2.84286528066663	-0.361813147851201\\
2.69679515327729	-0.634226331552334\\
2.52416781458882	-0.912277211147875\\
2.32524184475379	-1.19150302866321\\
2.10157109850985	-1.46673833034004\\
1.85607672581616	-1.73248861197861\\
1.5929240733955	-1.98341096065366\\
1.31720186022244	-2.21481305199938\\
1.0344580862182	-2.42306622844925\\
0.750189036321172	-2.60584971916751\\
0.469385229877587	-2.76219307362953\\
0.196210292324188	-2.89233948726711\\
-0.0661579410990571	-2.99749091688168\\
-0.315538485073338	-3.07950661708148\\
-0.550651292773326	-3.14061457602759\\
-0.770962052237112	-3.18317238106405\\
-0.976508330981463	-3.20949148705748\\
-1.1677347744734	-3.22172275137178\\
-1.34535225314538	-3.22179255714557\\
-1.51022612358149	-3.21137621559538\\
-1.66329270619869	-3.19189623469262\\
-1.8055000413986	-3.16453548237547\\
-1.93776797815316	-3.13025800138574\\
-2.06096279273744	-3.08983262586188\\
-2.17588220806786	-3.0438563962646\\
-2.28324750945659	-2.9927760767148\\
-2.38370023495592	-2.93690694400996\\
-2.47780157576205	-2.87644855549136\\
-2.56603313501092	-2.81149751565592\\
-2.6487980723826	-2.74205742852956\\
-2.72642192920186	-2.6680463020594\\
-2.79915260797772	-2.58930170239171\\
-2.86715909316413	-2.50558396773426\\
-2.93052856475804	-2.41657780352222\\
-2.98926158898672	-2.32189260853916\\
-3.04326508547837	-2.22106193981792\\
-3.09234278368039	-2.11354262766093\\
-3.13618291239866	-1.99871421838422\\
-3.17434294200452	-1.87587967180223\\
-3.20623135758636	-1.74426859563808\\
-3.23108673821157	-1.60304478125932\\
-3.24795492865539	-1.45132042588722\\
-3.2556659137754	-1.28818017046075\\
-3.25281325462513	-1.11271887779823\\
-3.2377407180028	-0.924097747129222\\
-3.2085430476891	-0.721623565866524\\
-3.16309050763878	-0.504855062760248\\
-3.09908931177837	-0.273737620112226\\
-3.01419117581832	-0.0287620629975018\\
-2.90616309346768	0.228865828376097\\
-2.77312069140666	0.497070671546244\\
-2.61381332780608	0.772806139286641\\
-2.42792711446531	1.05205331312857\\
-2.21634862057458	1.32996047962264\\
-1.9813178241369	1.60114259875737\\
-1.72640633608714	1.86011730169536\\
-1.45629215031779	2.10180758900895\\
-1.1763571087301	2.32201043807557\\
-0.892185922484802	2.5177331366009\\
-0.609071712238378	2.68733707063445\\
-0.331621664611324	2.83048477424765\\
-0.0635163824766542	2.94793521683809\\
0.192571151958296	3.04125702685223\\
0.434927199860419	3.11252737579308\\
0.662665959751348	3.16406508076278\\
0.875560713711963	3.19822261541064\\
1.07387188834348	3.21724197812296\\
1.25819306153321	3.22316713074098\\
};
\addplot [color=mycolor2, forget plot]
  table[row sep=crcr]{%
2.34062091623357	2.77043391311226\\
2.38048810855438	2.76920049765774\\
2.41487063113053	2.76594225496986\\
2.44488053164076	2.76115988508774\\
2.47136191107579	2.75519433348111\\
2.49496423907643	2.74827649567593\\
2.51619347442919	2.74055966732415\\
2.53544818577876	2.73214092589803\\
2.55304538806354	2.72307526061283\\
2.56923921638794	2.71338483192463\\
2.58423452433188	2.70306485611385\\
2.59819681231562	2.69208705049761\\
2.61125943422324	2.680401208964\\
2.62352871490474	2.66793522629686\\
2.63508738407729	2.65459370329513\\
2.64599655732029	2.64025510948017\\
2.65629634560386	2.62476733086357\\
2.66600502845203	2.60794126391278\\
2.6751165591453	2.58954190826651\\
2.68359595434742	2.56927612719677\\
2.69137181364272	2.54677583926005\\
2.69832475175538	2.52157480550302\\
2.70426980120371	2.49307627195009\\
2.70892967619744	2.46050733885834\\
2.71189387130658	2.42285377253979\\
2.71255536591029	2.37876560454941\\
2.71001128895158	2.32641859378053\\
2.70290467892945	2.26330850689729\\
2.68916885789367	2.18594318845285\\
2.66561022175146	2.08938165487635\\
2.62722628319647	1.966555716939\\
2.56611036138608	1.80732423281948\\
2.46979588526204	1.59734325436237\\
2.3191579275618	1.31734250648625\\
2.08714632199668	0.944806770699193\\
1.74268782805652	0.462587824528167\\
1.26752298933528	-0.121065315080577\\
0.685174320015884	-0.751117827291476\\
0.0705983045612169	-1.33763520231911\\
-0.490095687381608	-1.80916846000086\\
-0.949033123034393	-2.14814942489748\\
-1.30235718224383	-2.37606131082363\\
-1.56808187960351	-2.52450123376009\\
-1.7679299976156	-2.62003143352293\\
-1.92007065852317	-2.68120975165639\\
-2.03792748199427	-2.72010163125482\\
-2.13097838217791	-2.74437261532731\\
-2.20583863974892	-2.75889124251424\\
-2.26714487892474	-2.76678071574231\\
-2.31818449186105	-2.7700729861553\\
-2.36132268812799	-2.77010970246252\\
-2.3982881179035	-2.7677887039666\\
-2.43036411232187	-2.76371721875338\\
-2.45851781959879	-2.75830844297451\\
-2.48348845348508	-2.75184334400764\\
-2.50584840985986	-2.7445107643716\\
-2.52604616859021	-2.73643374038363\\
-2.54443679848662	-2.72768688696962\\
-2.56130389714171	-2.71830785975126\\
-2.57687551520985	-2.70830478091108\\
-2.59133577674062	-2.69766081344432\\
-2.60483335046394	-2.68633661801923\\
-2.61748754855663	-2.67427112582108\\
-2.62939256346133	-2.66138084688549\\
-2.64062015615187	-2.64755776611475\\
-2.65122095017126	-2.63266572980567\\
-2.66122434073354	-2.61653507058767\\
-2.67063687479661	-2.59895503469838\\
-2.67943877086722	-2.57966333424822\\
-2.68757799217844	-2.55833180968797\\
-2.69496091224232	-2.53454669638471\\
-2.70143803483985	-2.50778125474477\\
-2.70678231334399	-2.47735740461383\\
-2.71065612191777	-2.44239127686218\\
-2.71256045774346	-2.4017149007247\\
-2.71175579375927	-2.35376202792122\\
-2.70713693658239	-2.29639953334494\\
-2.69703222541958	-2.22667588514445\\
-2.67887725147416	-2.14044412457706\\
-2.64868109923095	-2.0318006139283\\
-2.60015857138027	-1.89227550575164\\
-2.52336726231127	-1.70976638353818\\
-2.40277592972888	-1.46747869035947\\
-2.21528573365125	-1.14401469626111\\
-1.93073081033608	-0.717795814247614\\
-1.52128427408323	-0.181220092662584\\
-0.986029825337172	0.435612739739464\\
-0.376064190330708	1.0554004523323\\
0.220679833419737	1.59015249851199\\
0.733329590066657	1.99441882961407\\
1.1380104421674	2.27399244198054\\
1.44477007677968	2.45836496946707\\
1.67499107537872	2.57754226425367\\
1.84901930754608	2.65403439145904\\
1.98261292073439	2.7028810520088\\
2.08708479775557	2.73370957408162\\
2.17035470254002	2.75262407212757\\
2.23795475939865	2.76351769633521\\
2.29378247987082	2.76890493021635\\
2.34062091623356	2.77043391311226\\
};
\addplot [color=mycolor1, forget plot]
  table[row sep=crcr]{%
1.3803158105586	3.09362620152166\\
1.54893729354649	3.08834757461492\\
1.70385067175428	3.0736199824606\\
1.84629109518523	3.05088443207606\\
1.97744700638335	3.02131067122778\\
2.09842739246933	2.98583011929795\\
2.21024459821208	2.94516856333983\\
2.31380737873498	2.89987603999062\\
2.40992028466746	2.85035273838091\\
2.499286622764	2.79687059829109\\
2.58251310245332	2.7395907336281\\
2.6601148985556	2.67857702789487\\
2.73252028326624	2.61380632502745\\
2.8000742552139	2.545175640604\\
2.86304076030295	2.47250678588776\\
2.92160318910353	2.39554875609676\\
2.97586287170976	2.31397820155143\\
3.02583529045299	2.22739828867977\\
3.07144370765178	2.13533627934926\\
3.1125098733779	2.03724022582315\\
3.14874145372062	1.93247531338628\\
3.17971582729964	1.82032060830986\\
3.20485997404229	1.69996731790985\\
3.22342638361572	1.57052018167982\\
3.23446532983979	1.43100432997874\\
3.23679462022262	1.28038090100622\\
3.22896920905005	1.11757588981802\\
3.20925506229208	0.941528012539268\\
3.1756145649968	0.751262515788463\\
3.12571459433532	0.545998238420895\\
3.05697275890688	0.325293773351387\\
2.96666102541377	0.0892337103734397\\
2.85208654475064	-0.161354125139463\\
2.71086306896909	-0.42467583761746\\
2.54126855163561	-0.69778941448721\\
2.34265335777059	-0.976528947597679\\
2.11582407343387	-1.25560544923608\\
1.86329611123416	-1.52893099931784\\
1.58930778114802	-1.79015539430015\\
1.29953650539806	-2.033327807896\\
1.00054660757384	-2.25353593622805\\
0.699089460599829	-2.44736860510669\\
0.401423196147943	-2.61310444429223\\
0.112797733141857	-2.75062047786383\\
-0.162820833709833	-2.86109388236568\\
-0.422792905911817	-2.94660581462791\\
-0.66567587136833	-3.00974651590204\\
-0.891000432494973	-3.05328563427414\\
-1.09902013707136	-3.07993341126121\\
-1.29047876097123	-3.09219047278403\\
-1.46641776589206	-3.09226950456981\\
-1.6280297223777	-3.08206810833061\\
-1.77655415559569	-3.06317411338522\\
-1.91320817811311	-3.03688892090978\\
-2.03914348297347	-3.00425893086493\\
-2.15542213748432	-2.96610879480698\\
-2.26300508755094	-2.92307292285301\\
-2.36274878631352	-2.87562345824462\\
-2.45540665108207	-2.82409403305481\\
-2.54163305914944	-2.76869924496107\\
-2.62198833070582	-2.70955011621246\\
-2.69694366193025	-2.64666593257216\\
-2.76688531448405	-2.57998289280504\\
-2.83211758399588	-2.50935997984026\\
-2.8928641952833	-2.43458242540496\\
-2.94926783270839	-2.35536310126249\\
-3.00138752998868	-2.27134214650044\\
-3.04919363025445	-2.18208514363919\\
-3.09255999762963	-2.08708019949427\\
-3.13125313083228	-1.98573438583329\\
-3.16491781700242	-1.87737017147393\\
-3.19305899997689	-1.76122276026232\\
-3.21501966812754	-1.63643967473304\\
-3.22995486468682	-1.50208453506649\\
-3.23680249553661	-1.35714781644299\\
-3.23425260632015	-1.20056844175191\\
-3.22071841420947	-1.03127133462783\\
-3.19431481332988	-0.848227338451973\\
-3.15285345483355	-0.650542770593227\\
-3.09386769577832	-0.437585506196712\\
-3.01468497997496	-0.209151545640447\\
-2.91256677147678	0.0343312360156933\\
-2.78493380284614	0.291578628037772\\
-2.62968278173373	0.560240488330013\\
-2.44557614400683	0.836762832348514\\
-2.23264986763644	1.11639222886098\\
-1.99254607838426	1.3933852947127\\
-1.72865801087491	1.66144513354078\\
-1.44599765322294	1.91433606894612\\
-1.15076782531539	2.14655445462601\\
-0.849716631807568	2.35389657663218\\
-0.54942625744986	2.53379103644462\\
-0.255700867017199	2.68534239882346\\
0.0268326092836768	2.8091237895267\\
0.294878551857073	2.90681579248229\\
0.546419730517061	2.9808004118877\\
0.780531158675736	3.0337933296798\\
0.997135915631996	3.06855852011688\\
1.19675952955286	3.08771522176552\\
1.3803158105586	3.09362620152166\\
};
\addplot [color=mycolor2, forget plot]
  table[row sep=crcr]{%
2.30478646398007	2.79272230997285\\
2.33956694171741	2.79164558255829\\
2.36967416089367	2.78879189377679\\
2.39604915169522	2.78458826116053\\
2.41940738784067	2.77932580505079\\
2.44030047088818	2.773201622012\\
2.45915907372101	2.76634612415256\\
2.47632326503792	2.75884104672134\\
2.49206419846869	2.75073132446467\\
2.50659979070406	2.74203282882589\\
2.52010613681443	2.73273721255329\\
2.53272583618772	2.72281463555961\\
2.54457401700489	2.71221483445628\\
2.55574258032479	2.70086678072266\\
2.56630299095393	2.68867700627798\\
2.57630779005883	2.67552653084561\\
2.58579087030686	2.66126617950051\\
2.59476641722134	2.64570990850852\\
2.60322625758609	2.62862553629016\\
2.6111351372104	2.60972196762729\\
2.61842313156106	2.58863154881195\\
2.62497390203963	2.56488551457264\\
2.63060672793107	2.53787944840306\\
2.63504896247673	2.50682405511139\\
2.63789341837391	2.47067397545903\\
2.63853153897078	2.4280232661205\\
2.63604690669952	2.37694958087758\\
2.62904264638929	2.31477861271425\\
2.61535713588221	2.23772424957448\\
2.59158987092791	2.14033740317269\\
2.55230801849377	2.01467364838278\\
2.48874098609742	1.8491005504418\\
2.38676701709457	1.626837450619\\
2.22436620901246	1.32503913598231\\
1.97040697847379	0.917321547969514\\
1.59106365529649	0.386261610312481\\
1.07379118803747	-0.249267906989404\\
0.461322633711947	-0.912228345945979\\
-0.152439094359707	-1.49831803336745\\
-0.682985972155761	-1.94473859744379\\
-1.09888941523765	-2.252054648784\\
-1.41006759865393	-2.4528303265095\\
-1.64031122406817	-2.58146771345832\\
-1.81210247322955	-2.66359170219609\\
-1.9425173390933	-2.71603447836118\\
-2.04356549189356	-2.74937887236392\\
-2.12349951943009	-2.77022737719967\\
-2.18799236562903	-2.78273419074224\\
-2.24098691524362	-2.78955303187048\\
-2.28526666845532	-2.79240839789825\\
-2.32283069563633	-2.79243963015844\\
-2.35513972102892	-2.79041036313633\\
-2.38327901438151	-2.78683801838073\\
-2.40806762524164	-2.78207523978057\\
-2.43013272765714	-2.77636196288903\\
-2.44996099328966	-2.76985920138283\\
-2.46793461658927	-2.76267121917085\\
-2.48435692268607	-2.75486016166862\\
-2.49947078649536	-2.74645566828574\\
-2.51347200233559	-2.73746104186232\\
-2.52651903542685	-2.72785695947072\\
-2.53874011744027	-2.71760332751641\\
-2.5502383290549	-2.70663962616922\\
-2.5610950867276	-2.69488390063282\\
-2.57137228112513	-2.68223040464321\\
-2.58111317415563	-2.66854575916828\\
-2.59034202852945	-2.65366333410884\\
-2.59906229695985	-2.63737536881697\\
-2.60725301186582	-2.61942208769188\\
-2.61486275467235	-2.59947669586444\\
-2.62180019037119	-2.5771245895206\\
-2.62791953552688	-2.55183427868937\\
-2.63299832922247	-2.52291622416527\\
-2.63670322317505	-2.48946375149657\\
-2.63853671651824	-2.45026695988795\\
-2.63775297178264	-2.40368534204549\\
-2.63322252863105	-2.34745650548621\\
-2.62321120715757	-2.27840530799946\\
-2.60501339474685	-2.1919983213213\\
-2.57433840211125	-2.08166418718521\\
-2.52428849911235	-1.93778670175437\\
-2.44371639744693	-1.74634206498435\\
-2.31486598594063	-1.48752300588703\\
-2.11105659036845	-1.1359716084557\\
-1.79811766488291	-0.667275034316672\\
-1.34858501413813	-0.0780971683956526\\
-0.774238049795059	0.584032329267382\\
-0.148286129954118	1.22042159931633\\
0.43138878657625	1.74018191686149\\
0.905258255977768	2.11404264728545\\
1.26615647323833	2.36344830444132\\
1.53378736915116	2.52433371153796\\
1.7323137872215	2.62711401014371\\
1.88162743328096	2.69274559445242\\
1.996123787551	2.734609393255\\
2.08576827798602	2.76106149465897\\
2.15739694930724	2.77733056834636\\
2.21573094769686	2.78672991307819\\
2.26407619548579	2.79139417326984\\
2.30478646398007	2.79272230997285\\
};
\addplot [color=mycolor1, forget plot]
  table[row sep=crcr]{%
1.53253678575503	2.97922984659636\\
1.69848220691898	2.97404617197318\\
1.84905338789904	2.95974112755087\\
1.98587942337696	2.93791019857066\\
2.11047743894977	2.90982256838876\\
2.22422441894082	2.87646992265854\\
2.32834778659313	2.83861196188184\\
2.42392717619353	2.79681609187841\\
2.5119022713053	2.75149055169265\\
2.59308338133733	2.70291117802793\\
2.66816267662567	2.65124242894247\\
2.73772482749179	2.59655342193481\\
2.80225631409169	2.53882972272346\\
2.86215298532536	2.4779815333501\\
2.91772561499132	2.41384881780184\\
2.96920327876322	2.34620379492374\\
3.01673438752842	2.2747511352303\\
3.06038518124038	2.19912612837183\\
3.10013542583642	2.11889104851962\\
3.13587097367162	2.0335299450084\\
3.16737275505032	1.94244213940555\\
3.19430167865837	1.84493483995977\\
3.21617885567919	1.74021552379761\\
3.23236056803254	1.62738513522178\\
3.24200754825425	1.50543377229184\\
3.24404854847912	1.37324146874722\\
3.23713904166488	1.22958801733666\\
3.21961750629181	1.07317759260802\\
3.18946448203592	0.902686185291078\\
3.14427388092552	0.716842275389181\\
3.0812521866997	0.514552958629766\\
2.99726887340254	0.295087250864118\\
2.88898890014322	0.0583227105842752\\
2.75312121191095	-0.194952944280957\\
2.5868082044987	-0.462720065356516\\
2.38815062632492	-0.741457191260336\\
2.15680525510023	-1.02603128135291\\
1.89451953418333	-1.30986628440396\\
1.60541388803629	-1.58546199907443\\
1.29584000417588	-1.84522328545909\\
0.97376055486984	-2.08242099143707\\
0.647775025750236	-2.29202113725685\\
0.326058846345277	-2.47115338695656\\
0.0155010949859296	-2.61913164993093\\
-0.278783070476973	-2.73710304913087\\
-0.55354668355016	-2.82749834587453\\
-0.807211255219125	-2.89345983110686\\
-1.03951336708477	-2.93836411768581\\
-1.25111782291838	-2.96548646552457\\
-1.44327236002039	-2.97780172214899\\
-1.61753615327626	-2.97789196897808\\
-1.77558563503421	-2.96792588011471\\
-1.91908666014927	-2.94968000538996\\
-2.04961740702522	-2.92458054500986\\
-2.16862706998544	-2.89375195075539\\
-2.27741818531878	-2.858064557318\\
-2.37714358505926	-2.81817736357363\\
-2.46881171621756	-2.7745744601786\\
-2.55329617472792	-2.7275949154007\\
-2.63134681126098	-2.67745657822587\\
-2.70360078663693	-2.62427451367744\\
-2.77059261446044	-2.56807482875103\\
-2.83276263492102	-2.5088045865118\\
-2.8904635973038	-2.44633840269961\\
-2.94396514665196	-2.38048220756323\\
-2.99345605060642	-2.31097455363744\\
-3.0390439907467	-2.23748576780782\\
-3.08075269491288	-2.15961519038613\\
-3.11851611400772	-2.07688672281537\\
-3.15216925792774	-1.98874293037763\\
-3.18143521176219	-1.89453803467972\\
-3.20590777313985	-1.79353031021597\\
-3.22502911610769	-1.68487471079595\\
-3.23806195164112	-1.56761705291988\\
-3.24405591432524	-1.44069185098044\\
-3.24180851303236	-1.30292702486047\\
-3.22982217800392	-1.15306027114507\\
-3.20626105545628	-0.98977393705402\\
-3.16891466549225	-0.811757630175587\\
-3.11518074780994	-0.617810041760324\\
-3.04208664511558	-0.406992384778554\\
-2.94637653364231	-0.17884323886986\\
-2.82469784762952	0.0663449341509858\\
-2.67391843856611	0.327207594583538\\
-2.4915874111442	0.601000775136923\\
-2.27650873051731	0.883396115860219\\
-2.02932874848718	1.16849676839643\\
-1.75296947956007	1.44917774918503\\
-1.45271532510242	1.71777384233659\\
-1.13582816674967	1.96700459311098\\
-0.810723475202644	2.19090326270296\\
-0.485914798356431	2.38548715216578\\
-0.169021024921071	2.54900209472586\\
0.133920662571018	2.68173940921952\\
0.418738978475628	2.78556175444595\\
0.683059030323465	2.86332311555987\\
0.926008456463667	2.91833410460932\\
1.14783490818052	2.953953549418\\
1.3495335469166	2.97332388388875\\
1.53253678575503	2.97922984659636\\
};
\addplot [color=mycolor2, forget plot]
  table[row sep=crcr]{%
2.26280257423888	2.80960815564716\\
2.29302831305614	2.80867183074462\\
2.31929080247629	2.80618204287693\\
2.3423822844014	2.80250127951935\\
2.36290633502203	2.79787694028509\\
2.3813293543842	2.79247641197041\\
2.39801639079062	2.78640997886444\\
2.41325644043243	2.77974590780946\\
2.42728055074426	2.77252037269508\\
2.44027491297271	2.76474387369191\\
2.45239039602751	2.75640518306304\\
2.46374949358338	2.74747345215825\\
2.47445133409025	2.73789884971652\\
2.48457517910172	2.72761191317346\\
2.49418267040591	2.71652164580413\\
2.5033189539957	2.7045122571966\\
2.51201268803542	2.69143830082525\\
2.52027481314658	2.67711778764795\\
2.52809580404399	2.66132262019722\\
2.53544090084505	2.64376535696691\\
2.54224248849254	2.62408081970194\\
2.54838827505016	2.60180029685615\\
2.55370307958217	2.57631491169239\\
2.55792064149776	2.5468228411808\\
2.56063948195111	2.51225203425696\\
2.56125271337673	2.47114511402165\\
2.55883439128627	2.42148497715885\\
2.55195194722146	2.36042619470558\\
2.53835084075819	2.28387586047843\\
2.5144163927739	2.18583579277575\\
2.47425016722987	2.05738165567431\\
2.40811017730113	1.88515666246544\\
2.29995179407256	1.64947860721804\\
2.12432871806517	1.32318517052677\\
1.84541309187049	0.875462695164259\\
1.42728821133862	0.290077031820641\\
0.867424953745529	-0.3980128423087\\
0.232854072756388	-1.08529691844234\\
-0.366964603365829	-1.65843732329594\\
-0.857779150263658	-2.07163959142021\\
-1.22747996915798	-2.34491019091256\\
-1.4974781299389	-2.51915094733346\\
-1.69475300086089	-2.62937975121338\\
-1.84116903062546	-2.69937576089385\\
-1.95220622288503	-2.74402602797072\\
-2.03835733829138	-2.77245352777192\\
-2.10668852572652	-2.79027457647737\\
-2.16200402841573	-2.80100059504363\\
-2.2076240161277	-2.80686962964739\\
-2.24588648643235	-2.80933620065568\\
-2.27846958141515	-2.80936263900506\\
-2.30660028634828	-2.80759524637279\\
-2.33119146968984	-2.80447285739189\\
-2.35293335492057	-2.80029503494\\
-2.3723556209046	-2.79526567578926\\
-2.38987026671953	-2.78952131869115\\
-2.40580166404906	-2.7831497243389\\
-2.42040792605227	-2.77620212170264\\
-2.43389628536998	-2.76870121870028\\
-2.44643426042814	-2.76064628454099\\
-2.45815779743023	-2.75201611564983\\
-2.46917718349013	-2.74277037511218\\
-2.47958125896757	-2.73284957465974\\
-2.48944026641433	-2.7221738033032\\
-2.49880752772091	-2.71064016748868\\
-2.5077200168184	-2.69811877080593\\
-2.51619777295085	-2.6844469050665\\
-2.52424195866059	-2.66942092406263\\
-2.5318311813046	-2.65278499291078\\
-2.5389154290978	-2.63421549961047\\
-2.54540656145442	-2.61329930282569\\
-2.55116363630205	-2.58950304368931\\
-2.55597027620966	-2.5621292591992\\
-2.55949945480315	-2.53025264788471\\
-2.56125795447763	-2.49262596042441\\
-2.56049726233623	-2.44753862177759\\
-2.55606791828385	-2.39260070946658\\
-2.54617683975173	-2.32440785933653\\
-2.52797596916795	-2.23801615992016\\
-2.49685708013057	-2.12612016964453\\
-2.44524644183476	-1.97779984761483\\
-2.36061832539141	-1.77677482876997\\
-2.22260043664195	-1.4996131001525\\
-2.00028857914076	-1.11622173396557\\
-1.65524651627194	-0.59947096514045\\
-1.16265575560169	0.0462647387570705\\
-0.55270247609633	0.749777113952609\\
0.0775982456260898	1.39099568483034\\
0.627983461948721	1.8847859809947\\
1.05679567880607	2.22324449171281\\
1.37319110253725	2.4419540065758\\
1.60368096486637	2.58053233579598\\
1.77321390764975	2.66830794416033\\
1.90035718559736	2.72419513280603\\
1.99788638089036	2.75985431610817\\
2.07440788424436	2.78243293029504\\
2.13573790410978	2.79636172181024\\
2.18586127967366	2.80443709500679\\
2.22755746886345	2.80845902429587\\
2.26280257423888	2.80960815564716\\
};
\addplot [color=mycolor1, forget plot]
  table[row sep=crcr]{%
1.72691890166584	2.88847723179633\\
1.88977999741713	2.8834031973613\\
2.03536220504261	2.86958333143687\\
2.16582777796012	2.84877669185259\\
2.28311545746725	2.82234504184872\\
2.38892852063427	2.79132564942209\\
2.48474300968047	2.75649464867069\\
2.57182593624988	2.718419395954\\
2.65125727235224	2.67750026980201\\
2.72395218959132	2.63400319819502\\
2.79068166269082	2.58808440644126\\
2.85209053609323	2.53980879057863\\
2.90871270340877	2.4891631136742\\
2.96098333195722	2.43606498605994\\
3.00924818368783	2.38036836691099\\
3.05377010415965	2.32186612963826\\
3.09473271118193	2.26029007125134\\
3.13224123531534	2.19530861546202\\
3.1663203563447	2.12652235981508\\
3.19690874753616	2.05345754998075\\
3.2238498846893	1.97555753587078\\
3.2468785019026	1.89217228867236\\
3.26560188770085	1.80254616152805\\
3.27947503274202	1.70580430389986\\
3.28776850659022	1.60093856189606\\
3.28952794292936	1.48679442299365\\
3.28352431398367	1.36206175158604\\
3.26819506653325	1.22527392271579\\
3.24157815775366	1.07482273819487\\
3.20124481153415	0.909000389538981\\
3.14424341568583	0.726084617565809\\
3.06707742927126	0.524488235265126\\
2.96575475049153	0.302996821707431\\
2.83596244229785	0.0611134075880491\\
2.67343114881857	-0.200492162412919\\
2.4745413993529	-0.479478188643106\\
2.23716580664829	-0.77139298741713\\
1.96162194139772	-1.06950375809132\\
1.65145678651216	-1.36511457787493\\
1.31368992162837	-1.6484886224626\\
0.958235973900378	-1.91024059688417\\
0.596541749872809	-2.14279440193373\\
0.239848108588855	-2.3414111709203\\
-0.102343401094471	-2.50448188054882\\
-0.42317189979988	-2.633118866783\\
-0.718594149747866	-2.73033687201901\\
-0.987022555610887	-2.80016265072688\\
-1.2287068586837	-2.84690338107628\\
-1.44509216148004	-2.87465850432765\\
-1.63828267686112	-2.88705726953517\\
-1.81065220664195	-2.88716101599237\\
-1.96459092140898	-2.87746633983676\\
-2.10235789055556	-2.85995982776086\\
-2.22600690882601	-2.83619235396913\\
-2.33735892084216	-2.80735485227582\\
-2.4380017496072	-2.77434678579574\\
-2.52930430162641	-2.73783403942497\\
-2.61243724478097	-2.69829586975579\\
-2.68839544316936	-2.65606188520638\\
-2.75801954004429	-2.61134049812941\\
-2.82201536321552	-2.56424032138803\\
-2.8809705693721	-2.51478581828564\\
-2.93536834419476	-2.46292828524935\\
-2.98559816563916	-2.40855301349939\\
-3.03196370132581	-2.35148326563642\\
-3.07468789760506	-2.29148152437925\\
-3.11391525625967	-2.22824832451758\\
-3.14971120020528	-2.16141886424223\\
-3.18205830904091	-2.0905575082707\\
-3.21084906159126	-2.01515024616841\\
-3.23587455717597	-1.93459516483031\\
-3.25680850423628	-1.84819105386823\\
-3.27318557608884	-1.7551244211205\\
-3.28437306798325	-1.65445550928868\\
-3.28953470899622	-1.54510446245682\\
-3.28758560690962	-1.42583972519049\\
-3.27713785400533	-1.29527225199449\\
-3.25643767503956	-1.15186139365136\\
-3.2232977623087	-0.993941642129349\\
-3.17503349703523	-0.81978384387707\\
-3.1084201706057	-0.627709629035622\\
-3.01970090069941	-0.41628209193809\\
-2.9046910010042	-0.1845954521154\\
-2.75903948093035	0.0673254023635762\\
-2.57871015733816	0.338037514991398\\
-2.36071268601663	0.624187684126594\\
-2.10402531729576	0.920179220424164\\
-1.81050727508198	1.21822058644483\\
-1.48545896971827	1.50894404467598\\
-1.13747436990128	1.78259850877738\\
-0.777441183093496	2.03053725527135\\
-0.416920991136378	2.24651693100535\\
-0.0664435258140606	2.42737556252555\\
0.265744470379608	2.57294994399036\\
0.574208918267981	2.68541759064244\\
0.856202200836089	2.76840422224328\\
1.11113844155475	2.82615351570793\\
1.33994052889959	2.86291451272814\\
1.54444115644465	2.88257247184433\\
1.72691890166584	2.88847723179633\\
};
\addplot [color=mycolor2, forget plot]
  table[row sep=crcr]{%
2.21306861028515	2.82138468809461\\
2.23919421653483	2.82057483010069\\
2.26198296571115	2.81841389866629\\
2.28209648632592	2.81520741079904\\
2.30004012557273	2.81116410695579\\
2.31620543889458	2.80642506981829\\
2.33089976349982	2.80108275966494\\
2.34436712119354	2.79519354364591\\
2.35680319112508	2.78878591359092\\
2.36836614840949	2.78186575343349\\
2.37918455994482	2.77441950095214\\
2.3893631330347	2.76641571723579\\
2.39898684605211	2.75780535367829\\
2.40812380367975	2.74852084306517\\
2.41682701984414	2.73847400734123\\
2.42513521646558	2.72755264680445\\
2.43307261707491	2.71561553299119\\
2.44064759293219	2.70248534729842\\
2.44784986320374	2.6879388588166\\
2.45464572789424	2.67169327259211\\
2.46097047148896	2.65338713218998\\
2.46671653053907	2.63255331130523\\
2.47171511937966	2.60858028298763\\
2.4757074853344	2.58065567904366\\
2.47829932467659	2.54768257197588\\
2.47888721346337	2.50815292966782\\
2.47653745718111	2.45995258613374\\
2.46978225740399	2.40005495988158\\
2.45626945328734	2.32403227607062\\
2.43214990124387	2.22526860338397\\
2.39099721123487	2.09370287188537\\
2.32193164605888	1.91391676251592\\
2.20659445887431	1.66267101758591\\
2.01537655551612	1.30749113866626\\
1.70707307373937	0.81265146938813\\
1.24532057573136	0.166096340870068\\
0.644241666550335	-0.572997261016063\\
-0.000258562820945922	-1.27152011839544\\
-0.570947604965193	-1.81719050216002\\
-1.01313964274256	-2.18963851724364\\
-1.33455067673994	-2.42728370461455\\
-1.56475863857915	-2.57586851466407\\
-1.73148478174332	-2.6690337084946\\
-1.85490534393418	-2.72803682939888\\
-1.94857853884193	-2.76570357759088\\
-2.02144687313672	-2.78974676871039\\
-2.07944622325466	-2.80487199459892\\
-2.12658297327417	-2.81401105855275\\
-2.16561701663072	-2.81903194698809\\
-2.19849027082329	-2.82115039387458\\
-2.22659750116732	-2.82117260665784\\
-2.25096007245838	-2.8196414537099\\
-2.27233945660591	-2.81692643655401\\
-2.29131279918541	-2.81328021918748\\
-2.30832416989408	-2.80887480881956\\
-2.32371993871405	-2.80382506834013\\
-2.33777359401511	-2.79820415446371\\
-2.35070340680101	-2.79205367855204\\
-2.36268515580077	-2.78539031690628\\
-2.37386137428234	-2.77820994370734\\
-2.38434809181079	-2.77048994828268\\
-2.39423972083171	-2.76219012830864\\
-2.40361251632825	-2.75325236179538\\
-2.41252687705079	-2.74359911533113\\
-2.42102863209388	-2.73313071778383\\
-2.42914934676731	-2.72172119612913\\
-2.43690556889085	-2.70921231194465\\
-2.44429680113426	-2.69540522725624\\
-2.45130180048283	-2.68004892996989\\
-2.45787253219382	-2.66282410550147\\
-2.46392467675033	-2.64332046147067\\
-2.46932289096235	-2.62100444588785\\
-2.47385785781808	-2.59517259170852\\
-2.47721016019479	-2.56488293533537\\
-2.47889250869963	-2.52885233674563\\
-2.47815558144282	-2.48529976084126\\
-2.47383130451668	-2.43170242087853\\
-2.46406632740831	-2.36440952035556\\
-2.44585964082178	-2.27802228655166\\
-2.41424909041632	-2.16439669527417\\
-2.36088161752589	-2.01107690277439\\
-2.27159316119037	-1.79904741422875\\
-2.12283613807592	-1.50040313581385\\
-1.87864430103352	-1.07935997199548\\
-1.49635590393171	-0.506829834323151\\
-0.957923725922759	0.19921414448768\\
-0.31930233989347	0.936241581559927\\
0.300140911852266	1.56686264606333\\
0.8085892828741	2.02329470130406\\
1.18720792635716	2.32224961244474\\
1.45915746964589	2.51027569508556\\
1.65461761218222	2.62780476318544\\
1.79762993314633	2.70185160839854\\
1.90481495801368	2.74896531345433\\
1.98718679399244	2.779081264406\\
2.05201909311113	2.79820951733387\\
2.10417669754535	2.81005399951602\\
2.14697605630563	2.81694845203879\\
2.18272614636253	2.82039605103031\\
2.21306861028515	2.82138468809461\\
};
\addplot [color=mycolor1, forget plot]
  table[row sep=crcr]{%
1.98350190093304	2.83481109413524\\
2.14248056964784	2.82987367299531\\
2.28208118893736	2.81663430426115\\
2.40516899039759	2.79701459731132\\
2.51420403841414	2.7724512279593\\
2.61126839996874	2.7440035972167\\
2.69810891910516	2.71244079897124\\
2.77618358885115	2.67830928457756\\
2.8467054939859	2.64198422970282\\
2.9106817063769	2.60370782662248\\
2.96894633693247	2.56361736515773\\
3.02218784567284	2.5217654316552\\
3.07097109315635	2.47813402637803\\
3.11575472258553	2.43264394117089\\
3.15690442723473	2.38516036500903\\
3.19470255464972	2.33549538741588\\
3.22935436337866	2.2834078343478\\
3.2609910950986	2.22860068335032\\
3.28966985832646	2.1707161524134\\
3.31537013590001	2.10932843153631\\
3.33798651930749	2.0439339238606\\
3.35731702858033	1.97393878711947\\
3.37304608638084	1.89864352854998\\
3.38472087205928	1.81722443533096\\
3.39171938981744	1.72871177170863\\
3.39320817539114	1.63196503927921\\
3.38808722604025	1.52564634298641\\
3.37491967274232	1.40819430322313\\
3.35184434348169	1.27780342807982\\
3.31647150223608	1.132418022608\\
3.26576712772136	0.969756328464007\\
3.19594142625119	0.787390297779356\\
3.10237596899037	0.582918868903819\\
2.9796536297165	0.35428450928417\\
2.82179468067958	0.100284240976438\\
2.62283595679328	-0.178703088485548\\
2.37787712128811	-0.47984112537456\\
2.08459072102887	-0.797045633861037\\
1.74490067134976	-1.12070472977889\\
1.36616010011684	-1.4383845799873\\
0.961011440557227	-1.73669058147411\\
0.545540466365693	-2.00381031164518\\
0.136253891683127	-2.23172729541117\\
-0.252879637056443	-2.41719964416383\\
-0.612215416707421	-2.56131528530312\\
-0.936727653380476	-2.66814611224094\\
-1.22522905296017	-2.74323004872262\\
-1.47916627067631	-2.79237206539434\\
-1.70147892947065	-2.82091381775971\\
-1.89574084990639	-2.83340285743232\\
-2.06560850664664	-2.83352251146187\\
-2.21451442915048	-2.82415886677544\\
-2.34552614361502	-2.80752219717156\\
-2.46130458258079	-2.78527693510509\\
-2.56411606576263	-2.75865895339753\\
-2.65586928113545	-2.72857289609268\\
-2.73816091103867	-2.6956691735887\\
-2.81232128279681	-2.66040306853896\\
-2.87945597318512	-2.62307917768607\\
-2.94048180960857	-2.58388427271685\\
-2.99615700593859	-2.54291118335312\\
-3.04710577173586	-2.50017576111775\\
-3.0938379552933	-2.45562848446283\\
-3.13676430446855	-2.40916184966583\\
-3.17620785403866	-2.3606143575557\\
-3.21241182593122	-2.30977164153326\\
-3.24554428337353	-2.25636507247526\\
-3.27569962030643	-2.20006800733758\\
-3.30289679282638	-2.14048971016388\\
-3.32707400402481	-2.07716686053251\\
-3.34807932820653	-2.00955247449009\\
-3.36565649445862	-1.93700200381204\\
-3.3794247335105	-1.85875637028413\\
-3.38885122290928	-1.77392177183919\\
-3.39321425839128	-1.68144633904356\\
-3.39155488820183	-1.58009424993202\\
-3.38261450833846	-1.46841894084292\\
-3.36475612660729	-1.34473892226158\\
-3.33586825206194	-1.20712293827287\\
-3.29325374438244	-1.05339649906144\\
-3.2335133149152	-0.881189930367348\\
-3.1524474686113	-0.68805933509838\\
-3.04502470951272	-0.4717247658697\\
-2.90549908589009	-0.2304784019155\\
-2.72779972292002	0.0361951825770366\\
-2.50633242227956	0.326800387005326\\
-2.23727259392387	0.636955616243555\\
-1.92022039873202	0.958795871949097\\
-1.55973200992121	1.28113532185731\\
-1.16592980680383	1.59076497852406\\
-0.753506095799707	1.8747576300706\\
-0.33916317715618	2.122984674891\\
0.061546409408701	2.32978928870104\\
0.436689095220135	2.49422395791177\\
0.778975116482342	2.61906324192193\\
1.08543654134701	2.7092890691649\\
1.35635780230378	2.77069358498586\\
1.59405931410446	2.80891338869378\\
1.80188505015079	2.82891482601894\\
1.98350190093304	2.83481109413524\\
};
\addplot [color=mycolor2, forget plot]
  table[row sep=crcr]{%
2.15235170053101	2.82771016179676\\
2.17478115168786	2.8270143638408\\
2.19442963274587	2.82515076697225\\
2.21184331301689	2.82237430294861\\
2.22744071667035	2.81885934274927\\
2.24154721596132	2.81472355735665\\
2.25441907581603	2.81004353652318\\
2.26626047910532	2.80486506527101\\
2.27723574388257	2.79920983718986\\
2.287478180893	2.79307970635794\\
2.2970965514435	2.78645915794991\\
2.30617976561089	2.77931640431305\\
2.31480024420055	2.77160332548791\\
2.32301621447722	2.76325433179324\\
2.33087309285995	2.75418410478504\\
2.33840400822576	2.74428405092516\\
2.34562942082719	2.73341715966002\\
2.3525556768675	2.72141077046633\\
2.35917218516503	2.70804648849882\\
2.36544667654291	2.69304609484333\\
2.37131765367985	2.67605169284216\\
2.37668256454787	2.65659737944596\\
2.38137926771617	2.63406819591809\\
2.38515669445102	2.60763958760633\\
2.38762767356599	2.57618636589477\\
2.38819156745344	2.53814292479036\\
2.3859045295297	2.49128390585528\\
2.37925664472764	2.43237256993924\\
2.36577987128437	2.35658621879794\\
2.341343944372	2.25656583621438\\
2.29887835004151	2.12085217775264\\
2.22608704851716	1.93143401185242\\
2.10168592547686	1.66053185726397\\
1.89085754402466	1.26902361700542\\
1.54649503515728	0.716340999550197\\
1.03571087172451	0.000929336424112065\\
0.399348919740479	-0.782068038128535\\
-0.236420855862073	-1.47169274277777\\
-0.760509941703801	-1.9731442651855\\
-1.14599792097037	-2.29796994256754\\
-1.41795736062666	-2.49909633877957\\
-1.61004980181856	-2.6230915771769\\
-1.74853215088807	-2.7004757486327\\
-1.8510856437997	-2.74950170551074\\
-1.92914976045597	-2.78089014176923\\
-1.99012781130407	-2.80100848790145\\
-2.03888953927628	-2.81372339082726\\
-2.07870967972725	-2.82144283220785\\
-2.11184268359863	-2.82570383909892\\
-2.13987658965057	-2.82750974300448\\
-2.16395440534203	-2.82752820836332\\
-2.1849153903894	-2.82621036536572\\
-2.20338713933759	-2.82386418602627\\
-2.21984685321108	-2.82070066621658\\
-2.23466292048751	-2.81686344476888\\
-2.24812365619123	-2.81244808564602\\
-2.26045749776887	-2.80751474654026\\
-2.27184740571056	-2.80209650138798\\
-2.28244125532504	-2.79620471606388\\
-2.29235939745704	-2.78983234430576\\
-2.30170017193125	-2.78295567340977\\
-2.31054389541644	-2.77553482472919\\
-2.3189556644266	-2.76751315331924\\
-2.32698718181477	-2.75881556255123\\
-2.33467770906455	-2.74934563031601\\
-2.3420541496028	-2.7389813137606\\
-2.34913016391621	-2.72756883803061\\
-2.35590408628134	-2.71491415348089\\
-2.36235522843535	-2.70077102445467\\
-2.36843787494733	-2.68482432663472\\
-2.37407182667224	-2.66666637335596\\
-2.37912760666702	-2.64576288546655\\
-2.38340318009172	-2.62140325599939\\
-2.38658683527468	-2.59262649835452\\
-2.38819693135162	-2.55810873989928\\
-2.38748200409132	-2.51598860125734\\
-2.38325123982084	-2.4635902058119\\
-2.37357973587011	-2.39697463704155\\
-2.35528424240865	-2.31020156485598\\
-2.32297480791804	-2.19410716723213\\
-2.26733796727471	-2.03432483854576\\
-2.17215165550712	-1.8083671327232\\
-2.00985192138508	-1.48263115360921\\
-1.73835691289407	-1.01459181792908\\
-1.31180442026624	-0.375710526844209\\
-0.726543819406856	0.392110669737228\\
-0.0726552518913369	1.1473486329682\\
0.515994265558174	1.74708442268336\\
0.969581768645549	2.15449219604956\\
1.29396436142171	2.41070322055379\\
1.52212487005618	2.56847718214997\\
1.68471060334354	2.66624400461973\\
1.80346802395086	2.72773203965389\\
1.89264247285927	2.76692741400909\\
1.96142401905006	2.7920729024311\\
2.01580175575232	2.80811516354905\\
2.05975745345351	2.81809589920693\\
2.09600028640691	2.82393323818825\\
2.12641710976445	2.8268657627066\\
2.15235170053101	2.82771016179676\\
};
\addplot [color=mycolor1, forget plot]
  table[row sep=crcr]{%
2.33734640818134	2.84173773499851\\
2.49109306502754	2.83698081417114\\
2.62330022544378	2.82445653845541\\
2.73771631127657	2.80622994061618\\
2.83740682018188	2.78378031472703\\
2.92486031715751	2.75815635844025\\
3.00209113109873	2.73009192852672\\
3.07072932765729	2.70009040270244\\
3.13209557814148	2.66848528261135\\
3.18726167237742	2.63548328666928\\
3.23709857129096	2.60119468959842\\
3.28231411075317	2.56565438791066\\
3.32348228736911	2.52883617258919\\
3.36106573814055	2.49066194334983\\
3.39543268532002	2.45100704965754\\
3.42686930034134	2.40970253972035\\
3.45558815745043	2.36653479655119\\
3.48173319375229	2.32124280621342\\
3.50538135569285	2.27351311212424\\
3.52654087650361	2.22297234172502\\
3.54514587585504	2.169177033847\\
3.5610466794569	2.11160033706129\\
3.57399489574042	2.04961498688047\\
3.58362182697132	1.98247180702595\\
3.58940819663856	1.90927283688475\\
3.59064240796475	1.82893811365253\\
3.58636359083918	1.74016524118358\\
3.57528458424016	1.64138137476734\\
3.55568892486679	1.53068855771114\\
3.52529540174499	1.40580622725746\\
3.48108505982604	1.26402050241484\\
3.41909141725667	1.10216073546208\\
3.33417040425861	0.916642753393567\\
3.21980119327106	0.703648025466911\\
3.06803476626328	0.459547051737497\\
2.86981036427367	0.181706580406626\\
2.6159739292958	-0.130208155401798\\
2.29934194311722	-0.472513120888338\\
1.91782040438414	-0.835884415541505\\
1.4777126078581	-1.20491866160028\\
0.995289546876273	-1.56004776980447\\
0.494706089324506	-1.88187020607477\\
0.00238449151309759	-2.1560540933375\\
-0.459355597332624	-2.3761892341717\\
-0.876199631010727	-2.54343586600995\\
-1.24220440088729	-2.66399026487476\\
-1.55780494100006	-2.74618201507608\\
-1.8272057959172	-2.79836113551832\\
-2.05624254163932	-2.82780100467006\\
-2.25101548100488	-2.84034970404085\\
-2.41717577691034	-2.84048729330591\\
-2.5596376113602	-2.83154468445753\\
-2.68252577371285	-2.81595187433117\\
-2.78923445865596	-2.79545890038434\\
-2.88252731183531	-2.77131300475877\\
-2.96464362547795	-2.7443931490121\\
-3.03739528832324	-2.71530905464699\\
-3.1022492448699	-2.68447280866225\\
-3.16039499015794	-2.65215002671948\\
-3.21279858825657	-2.61849606107907\\
-3.26024529251385	-2.58358133702245\\
-3.30337281756184	-2.54740876374752\\
-3.34269704208471	-2.50992529935037\\
-3.37863158278195	-2.47102910840988\\
-3.41150234802259	-2.43057328017037\\
-3.44155787923642	-2.38836672707065\\
-3.46897602068764	-2.34417261896214\\
-3.49386721477555	-2.29770449848986\\
-3.51627448610095	-2.24862004575401\\
-3.53616993572033	-2.19651229908184\\
-3.55344729667083	-2.14089798164329\\
-3.56790977846952	-2.08120242353426\\
-3.57925202243482	-2.01674040514939\\
-3.58703446675794	-1.94669209089562\\
-3.59064774299446	-1.87007310494929\\
-3.58926386356524	-1.78569779700986\\
-3.58176991497527	-1.69213500956189\\
-3.56667883962573	-1.58765649062854\\
-3.54201099524978	-1.470180064693\\
-3.50514035916986	-1.3372137993762\\
-3.45260237602551	-1.18581540912352\\
-3.37987038390187	-1.01259563476494\\
-3.28113136886937	-0.813818484650238\\
-3.14914043914852	-0.585686455627477\\
-2.97531818455351	-0.324937813493696\\
-2.75037108609346	-0.0298939993925478\\
-2.46579913118171	0.297998214986816\\
-2.11652784717784	0.652394548293221\\
-1.70430281127496	1.02086053043047\\
-1.24041247959621	1.38549935232378\\
-0.745589454425837	1.72618573406833\\
-0.245957959688292	2.02551419754205\\
0.233410818537544	2.27295902153481\\
0.673940830231836	2.46611819797747\\
1.06564802589559	2.60904900756689\\
1.40610199972537	2.70934266685266\\
1.69793410542082	2.77553649714771\\
1.9463836188929	2.81552398599753\\
2.15754334599019	2.8358768856879\\
2.33734640818134	2.84173773499851\\
};
\addplot [color=mycolor2, forget plot]
  table[row sep=crcr]{%
2.07403866260917	2.8269543349669\\
2.09315924916392	2.82636066585801\\
2.10999318190359	2.82476358091409\\
2.12498416604991	2.82237301346102\\
2.1384737236779	2.81933272783761\\
2.15072854314413	2.81573951332741\\
2.16195960392293	2.81165577361651\\
2.17233576107662	2.80711780571264\\
2.18199352214924	2.80214117749728\\
2.19104415278104	2.79672407548457\\
2.19957886469313	2.79084915661269\\
2.20767258761167	2.78448421578705\\
2.21538665469627	2.77758182436031\\
2.2227706072954	2.7700779720829\\
2.22986322800465	2.76188963392558\\
2.23669282495772	2.75291106475939\\
2.24327670047541	2.74300847945363\\
2.24961962713554	2.73201257827272\\
2.25571100127946	2.71970809059483\\
2.26152011235544	2.70581907565178\\
2.26698859638195	2.68998804099299\\
2.27201852764331	2.67174585401969\\
2.27645355260126	2.6504676438978\\
2.28004862539231	2.62530691233002\\
2.28242057834992	2.59509496991008\\
2.28296560694686	2.55818389163297\\
2.28071807308664	2.51219529283697\\
2.27410238642777	2.45360860632651\\
2.26048518241586	2.37707127134637\\
2.23534788704536	2.27422554125036\\
2.19073871869562	2.13172032968132\\
2.11242614059813	1.92801461437831\\
1.97516213207316	1.62920486636794\\
1.73745884434234	1.18788523244802\\
1.34691053847526	0.561027491721929\\
0.783496643137142	-0.228530815809744\\
0.128514551779342	-1.03519522066827\\
-0.470022307285884	-1.68503886015666\\
-0.927825138946958	-2.12334570222652\\
-1.24955556419985	-2.39453726327172\\
-1.47174325929099	-2.55887721488673\\
-1.6276173784582	-2.65949588556988\\
-1.74007013041079	-2.72233230901795\\
-1.82370459021303	-2.76231110217522\\
-1.88773858188048	-2.78805580223589\\
-1.93807543546071	-2.80466143548468\\
-1.97858565442676	-2.81522326828103\\
-2.01187345695523	-2.82167525731679\\
-2.0397362836796	-2.82525763629486\\
-2.06344483303272	-2.82678421414579\\
-2.08391748886742	-2.82679934778336\\
-2.10183152825442	-2.82567259474096\\
-2.11769567143009	-2.82365721079403\\
-2.13189845770502	-2.820927108104\\
-2.14474116501868	-2.81760064446228\\
-2.15646063252991	-2.81375615329919\\
-2.16724534973486	-2.80944215691259\\
-2.17724696413185	-2.80468405816212\\
-2.18658860818404	-2.7994884187659\\
-2.1953709700189	-2.79384550823411\\
-2.203676722874	-2.78773053545745\\
-2.2115737211562	-2.78110379048918\\
-2.21911722628461	-2.77390978768314\\
-2.22635131743657	-2.7660753869192\\
-2.2333095527058	-2.75750675718095\\
-2.24001486011258	-2.74808491752273\\
-2.24647854030522	-2.73765942262243\\
-2.25269813468516	-2.72603952351195\\
-2.25865372597533	-2.71298178264141\\
-2.2643019470651	-2.69817258161117\\
-2.26956649912897	-2.68120310461785\\
-2.27432317971486	-2.66153299540644\\
-2.27837603427681	-2.63843659050964\\
-2.28141877576086	-2.61092174376639\\
-2.28297110720314	-2.57760452688373\\
-2.28227113129426	-2.53651120578387\\
-2.27808881038903	-2.48475757987664\\
-2.26839371171135	-2.41801737554461\\
-2.24974780590293	-2.32962364145465\\
-2.21617410986802	-2.209037813379\\
-2.15704627256193	-2.03929834557288\\
-2.05333502049323	-1.79319604265062\\
-1.87212667571019	-1.42961807547341\\
-1.56418794388763	-0.898801961181344\\
-1.08459312721087	-0.180259328262583\\
-0.458142264779495	0.642240804383108\\
0.185645057171374	1.3865479411452\\
0.717664498691321	1.92902099233589\\
1.10355717766559	2.27579000289122\\
1.37077553186421	2.48689678677427\\
1.55630717272349	2.61520166763647\\
1.68820173657598	2.69451217766848\\
1.78481611440694	2.74453254832799\\
1.85774317453226	2.77658392136132\\
1.91434049111072	2.79727288927779\\
1.95937293265292	2.81055649618859\\
1.99600504333981	2.81887305061057\\
2.02639365982054	2.82376652078023\\
2.05204572476894	2.82623889617718\\
2.07403866260917	2.82695433496691\\
};
\addplot [color=mycolor1, forget plot]
  table[row sep=crcr]{%
2.85378684355605	2.95375934774863\\
3.00036271091603	2.94924405278867\\
3.12344126758871	2.93759902399836\\
3.22778799871929	2.92098727686308\\
3.3170997188036	2.90088308953318\\
3.39424650393384	2.87828537605597\\
3.46146503151574	2.85386429021913\\
3.5205077004599	2.82806100926026\\
3.57275564791737	2.80115539161355\\
3.61930372218224	2.77331176433672\\
3.66102418476661	2.74460980903458\\
3.69861443570776	2.71506521735019\\
3.73263274860664	2.6846432222397\\
3.76352495212441	2.6532670549698\\
3.79164418608587	2.62082266241941\\
3.81726524580055	2.58716052734958\\
3.84059455760578	2.55209508554567\\
3.86177645753552	2.51540197328519\\
3.8808961356679	2.4768131280646\\
3.89797932807475	2.43600957663164\\
3.91298855500801	2.39261155440344\\
3.92581538456694	2.34616538903257\\
3.93626780564246	2.29612632827557\\
3.94405127046831	2.24183617786611\\
3.94874124478657	2.18249421862109\\
3.94974408471304	2.11711937884328\\
3.94624161194274	2.04450105336132\\
3.93711271718952	1.96313534372365\\
3.92082251357158	1.87114303712362\\
3.89526591422912	1.76616583920769\\
3.85754835703905	1.64523938113454\\
3.80368328564037	1.50464785205062\\
3.72818855730828	1.33978092653307\\
3.62358394602286	1.14504866135667\\
3.4798554152207	0.91397973008484\\
3.28410704277114	0.639748655323926\\
3.02092802527427	0.316534364168515\\
2.67443814677328	-0.0578283946622168\\
2.23315886346491	-0.477872185184208\\
1.6977294122587	-0.926613010670904\\
1.08809389975987	-1.37524073796466\\
0.443423744619842	-1.789656253505\\
-0.188919701103301	-2.1418787995415\\
-0.769190184836531	-2.41863570498246\\
-1.27496003117096	-2.62168455721952\\
-1.70082019331029	-2.76206254379989\\
-2.05243324705879	-2.85371812142474\\
-2.34036479104677	-2.90954929474476\\
-2.5760484738883	-2.939888827454\\
-2.76983699388432	-2.95240645345205\\
-2.93035907357098	-2.95256259839635\\
-3.06451310398694	-2.94415836419008\\
-3.17770308589254	-2.92980858807872\\
-3.2741269781873	-2.91130015429449\\
-3.35704103530817	-2.88984769035182\\
-3.42897740464765	-2.86627073504579\\
-3.49191444035656	-2.84111477549386\\
-3.5474067123776	-2.81473343561453\\
-3.59668306227496	-2.78734414827436\\
-3.6407201882803	-2.75906578463492\\
-3.680297784754	-2.72994395305715\\
-3.71603984576179	-2.69996777990496\\
-3.74844556194064	-2.66908069854847\\
-3.77791231563064	-2.63718690458173\\
-3.80475257411681	-2.60415454368636\\
-3.82920594423958	-2.5698162864352\\
-3.85144723589105	-2.53396764537538\\
-3.87159104604926	-2.49636315828672\\
-3.88969308395758	-2.45671036490989\\
-3.90574818005476	-2.41466131763696\\
-3.91968462439406	-2.369801168491\\
-3.93135412847148	-2.32163314528318\\
-3.94051625242322	-2.26955894890809\\
-3.94681552699829	-2.2128532506457\\
-3.94974864331101	-2.15063052461213\\
-3.94861786898256	-2.08180190868407\\
-3.94246512699939	-2.00501917332367\\
-3.92997876793897	-1.91860230349286\\
-3.90936183301247	-1.82044697762058\\
-3.87814661617138	-1.70790912168691\\
-3.83293637790775	-1.57766744265031\\
-3.76905389077023	-1.42557504030993\\
-3.68008561837665	-1.24653503414642\\
-3.55734757160428	-1.03448524106805\\
-3.38940075654997	-0.782670607277383\\
-3.16196877730482	-0.484526804717714\\
-2.8589974547303	-0.135636159747718\\
-2.46599260243726	0.262899932561106\\
-1.97647101926526	0.700218704365431\\
-1.40011050779229	1.15307365468656\\
-0.767179368858455	1.58875278686164\\
-0.122883733916398	1.9747623873091\\
0.487411681846658	2.28988111309565\\
1.03206660126965	2.52881822967954\\
1.49770385090727	2.69884339905605\\
1.88530306237435	2.81312253182019\\
2.2036337251602	2.88540035185023\\
2.46405326257749	2.92736787416042\\
2.67759798181907	2.94798892820503\\
2.85378684355605	2.95375934774863\\
};
\addplot [color=mycolor2, forget plot]
  table[row sep=crcr]{%
1.96331297825472	2.81411434692614\\
1.97955568460529	2.81360945989346\\
1.99394894145568	2.81224344495505\\
2.00684563657167	2.81018642729994\\
2.01851919858992	2.80755505797425\\
2.02918453054378	2.80442755317506\\
2.03901273452669	2.80085358909671\\
2.04814162857186	2.79686080122021\\
2.05668335343318	2.79245896553525\\
2.06472992354038	2.78764252681038\\
2.07235728995282	2.78239187592512\\
2.07962829257673	2.77667360100596\\
2.08659474688109	2.7704398071925\\
2.09329881290044	2.76362649201651\\
2.09977371479861	2.75615085852422\\
2.10604380432763	2.7479073287507\\
2.1121238779384	2.73876186553627\\
2.11801754870306	2.72854399284722\\
2.12371431648853	2.71703558071442\\
2.12918473361584	2.70395496073472\\
2.13437266031943	2.68893414383281\\
2.13918292172501	2.67148561889073\\
2.1434614872898	2.6509530546004\\
2.14696315807193	2.62643654424968\\
2.14929780981425	2.59667659095788\\
2.14983877505048	2.55986949608556\\
2.14756240313816	2.51336573228205\\
2.14075880220387	2.45316383993336\\
2.12649489728472	2.37304040554418\\
2.09959243129066	2.26303053274972\\
2.05066101712678	2.10679271522806\\
1.96241744398336	1.87735297786831\\
1.80373134755954	1.5320193803186\\
1.5249370184216	1.01443872907873\\
1.07434285272362	0.290876216853278\\
0.466845707927361	-0.561366935479028\\
-0.165834173365223	-1.34157313650627\\
-0.685139205075571	-1.90593960632484\\
-1.05566763729559	-2.26086599763237\\
-1.30803943883671	-2.4736311669978\\
-1.48098569504271	-2.60155014104102\\
-1.60277403995252	-2.68015961549638\\
-1.69139540992165	-2.72967353157353\\
-1.75798342375903	-2.76149930187783\\
-1.8095001932274	-2.78220824653105\\
-1.8504053136757	-2.79570011885863\\
-1.8836363370713	-2.80436241989564\\
-1.91118293182293	-2.80970035122978\\
-1.93442860552846	-2.81268811575503\\
-1.95435888882385	-2.81397064027515\\
-1.97169141749957	-2.81398282422971\\
-1.98695927691459	-2.81302198532718\\
-2.00056568939737	-2.81129297433242\\
-2.01282070314002	-2.80893688317258\\
-2.0239663129008	-2.80604963840826\\
-2.0341939807038	-2.8026941931011\\
-2.04365705814167	-2.79890855504601\\
-2.0524797183299	-2.79471102260153\\
-2.06076344870776	-2.79010347533507\\
-2.06859180064954	-2.78507323912646\\
-2.0760338591876	-2.77959383119857\\
-2.0831467385059	-2.77362474073185\\
-2.08997729655123	-2.76711028448372\\
-2.09656317542501	-2.7599774726838\\
-2.10293319858159	-2.7521327105695\\
-2.10910707836365	-2.74345702675666\\
-2.11509429371756	-2.73379933765209\\
-2.12089186851221	-2.72296699287375\\
-2.12648058522473	-2.71071244562911\\
-2.13181885532999	-2.69671426376555\\
-2.13683294532342	-2.68054968683557\\
-2.14140135889619	-2.66165426968956\\
-2.14532958657962	-2.63926134362759\\
-2.14830854448596	-2.61230916988259\\
-2.14984462122842	-2.57929506405428\\
-2.14913886684121	-2.53804021148533\\
-2.14487236280542	-2.48530023035206\\
-2.13481351048029	-2.41610339883878\\
-2.11507917243741	-2.322602075395\\
-2.07871661488327	-2.19206490860275\\
-2.01299166199431	-2.00347325007772\\
-1.89455311878832	-1.72253206278974\\
-1.68304146370885	-1.29824361601324\\
-1.32288788987185	-0.677317393127129\\
-0.784256047327998	0.130299107663984\\
-0.143052910286758	0.973226891250416\\
0.444335511621189	1.6531311090341\\
0.887744684682392	2.1055808885141\\
1.19402089518443	2.38089192911604\\
1.40239527567296	2.54552300786836\\
1.54695012512996	2.64548551289316\\
1.65041113146462	2.70769190997727\\
1.72693445679873	2.74730523772255\\
1.78530237107653	2.77295405043041\\
1.83106812686669	2.78968083596682\\
1.8678383551793	2.80052528515005\\
1.89802243479918	2.80737648435111\\
1.92327427133273	2.81144166910494\\
1.94475819050041	2.813511447244\\
1.96331297825472	2.81411434692614\\
};
\addplot [color=mycolor1, forget plot]
  table[row sep=crcr]{%
3.66143984979885	3.26030969750488\\
3.79889567075761	3.25609522771153\\
3.91145095592088	3.24545959918808\\
4.00488405111818	3.23059503687368\\
4.08344387248664	3.21291823573724\\
4.15028586095019	3.1933443182594\\
4.20777948300575	3.1724604713876\\
4.25772412466527	3.15063650206084\\
4.30150107266993	3.12809570627973\\
4.34018118660345	3.10496058427243\\
4.37460178831773	3.08128242180113\\
4.40542201274555	3.05706035845976\\
4.43316292657453	3.0322534630426\\
4.45823672388085	3.00678802538314\\
4.4809679471203	2.98056144301995\\
4.50160874510828	2.95344354072357\\
4.52034952145912	2.92527579532119\\
4.53732584811405	2.89586867450269\\
4.55262214951887	2.86499708846314\\
4.56627235198397	2.8323937631021\\
4.57825739685561	2.79774014511581\\
4.5884991936492	2.76065421558516\\
4.59685019208507	2.72067428857635\\
4.60307721600122	2.67723746494548\\
4.60683743426471	2.62965084430114\\
4.60764320176406	2.5770527932945\\
4.60481076479983	2.51836041969331\\
4.59738513955396	2.45219776821199\\
4.58402929444862	2.37679697110114\\
4.56285927616657	2.28986152667899\\
4.53119697906703	2.18837715999582\\
4.48519762792757	2.06835230804267\\
4.41928951640905	1.92447060357796\\
4.32534401180416	1.74965179478889\\
4.1914966368282	1.5345712889852\\
4.00063747017681	1.26734206152893\\
3.72897141283549	0.933929753391442\\
3.34608977710739	0.520567391559537\\
2.81996545962316	0.0201741504485785\\
2.13164049836243	-0.556265694742713\\
1.29875894103909	-1.16884591759495\\
0.38957392165057	-1.75320489136707\\
-0.498186453022081	-2.24783582138081\\
-1.28370138150545	-2.62273274160032\\
-1.93109871617218	-2.88288433040544\\
-2.44307834405449	-3.05183851426923\\
-2.84090750293457	-3.15567109193353\\
-3.14945119067209	-3.21558432071315\\
-3.39047657724607	-3.24666685292685\\
-3.5810051094907	-3.25901012604828\\
-3.73371646281758	-3.25918281434069\\
-3.85788271398113	-3.2514207524235\\
-3.96026221202168	-3.23845303017948\\
-4.04580440875587	-3.22204163116993\\
-4.11816598291268	-3.2033255661376\\
-4.18007743256084	-3.18303890969571\\
-4.23360073644704	-3.16164924418725\\
-4.28031038105244	-3.13944609773853\\
-4.32142116360366	-3.11659782607227\\
-4.35787908656229	-3.09318838386657\\
-4.39042653213797	-3.06924110152558\\
-4.41964935154084	-3.04473391305451\\
-4.44601108058334	-3.01960882400614\\
-4.4698778459518	-2.99377736739554\\
-4.4915363995599	-2.96712312797632\\
-4.51120693577764	-2.9395019735828\\
-4.52905178805701	-2.91074032479108\\
-4.5451806848625	-2.88063156228785\\
-4.55965291049511	-2.84893047509702\\
-4.57247641804878	-2.81534546188049\\
-4.58360363797199	-2.77952798499219\\
-4.59292337262107	-2.74105851436077\\
-4.60024771005255	-2.69942785082767\\
-4.60529225253164	-2.65401223980981\\
-4.60764702219832	-2.60404001140737\\
-4.60673400038473	-2.54854652175554\\
-4.60174509902098	-2.48631279899729\\
-4.59155101434331	-2.4157813597352\\
-4.5745662013086	-2.33493999874156\\
-4.54854715071333	-2.2411609192126\\
-4.51028900271617	-2.13097878721842\\
-4.45516831287353	-1.99978905489944\\
-4.37645900239336	-1.84145315801323\\
-4.26433536047132	-1.64782626160644\\
-4.10451173349945	-1.40831628899047\\
-3.87667368401192	-1.10982577643923\\
-3.55351055876737	-0.737953295795312\\
-3.10267774522643	-0.281144605569967\\
-2.49611814370087	0.260290711235291\\
-1.7301505261287	0.861715239181722\\
-0.84776523000926	1.46888239758364\\
0.0629472957477424	2.0145391317581\\
0.907080160091894	2.45060873194818\\
1.62516746798835	2.76588732943137\\
2.20290107764375	2.9770626761349\\
2.65467947580856	3.11042204052515\\
3.00487204212609	3.19003972564353\\
3.27722841285825	3.23399961853859\\
3.49116912628829	3.25470361386584\\
3.66143984979885	3.26030969750488\\
};
\addplot [color=mycolor2, forget plot]
  table[row sep=crcr]{%
1.781882940209	2.77322293973926\\
1.79588260529314	2.77278702308228\\
1.80841097659117	2.77159736069738\\
1.81974138446683	2.76978961057895\\
1.83008806950595	2.76745684099947\\
1.83962129977994	2.76466086069903\\
1.8484781286801	2.7614397057913\\
1.8567701602796	2.75781253548015\\
1.86458921862207	2.75378271177296\\
1.87201151627499	2.7493395386105\\
1.87910071970149	2.74445893837388\\
1.88591017453311	2.73910320409754\\
1.89248445759564	2.73321985568351\\
1.89886034778057	2.72673952788851\\
1.90506724197861	2.71957270969144\\
1.91112697391478	2.71160502049679\\
1.91705290988569	2.70269052456319\\
1.92284807805368	2.69264231552977\\
1.92850190848564	2.68121919108762\\
1.93398487063834	2.66810658861676\\
1.93923980603726	2.65288890189017\\
1.94416790475438	2.63500855607246\\
1.94860575705665	2.61370425569063\\
1.95228712340011	2.58791565836191\\
1.95477779378775	2.55613250846652\\
1.95536164810143	2.51614941394148\\
1.95283551157929	2.46465606258949\\
1.94512845155244	2.39653378330288\\
1.92857473641034	2.30362121407213\\
1.89649698060454	2.17253292220582\\
1.83645882975982	1.98093216469277\\
1.72533668269092	1.69210819871046\\
1.52302781961484	1.25184668083432\\
1.17606665822847	0.607304844084989\\
0.663584658507134	-0.216879808826287\\
0.0724679910453545	-1.04778855701782\\
-0.450405259793972	-1.69358161491012\\
-0.83620668374121	-2.11316810285577\\
-1.10040794624059	-2.36628461285566\\
-1.28021043293575	-2.51785112424728\\
-1.40550257695371	-2.61050165599441\\
-1.49571971739991	-2.6687177443404\\
-1.56287639808906	-2.7062287971572\\
-1.61442271385558	-2.73085860808695\\
-1.65508183507587	-2.74719838369212\\
-1.68793327545703	-2.75803071783548\\
-1.71504330725383	-2.76509519539211\\
-1.73783668099559	-2.76951037359158\\
-1.75732083822314	-2.77201338351128\\
-1.77422466624992	-2.77310013824509\\
-1.78908660957274	-2.77310975999451\\
-1.80231201968687	-2.77227676672347\\
-1.81421135362129	-2.77076408710107\\
-1.82502616337916	-2.76868436034322\\
-1.83494712521408	-2.76611388550085\\
-1.84412676764258	-2.76310183373794\\
-1.85268859658263	-2.75967631826468\\
-1.86073372247673	-2.75584830696964\\
-1.8683457190156	-2.75161398652662\\
-1.87559419994801	-2.74695594516419\\
-1.88253743820329	-2.74184337704484\\
-1.88922423869972	-2.73623138924254\\
-1.89569519247438	-2.73005938925365\\
-1.90198337088657	-2.72324842875773\\
-1.90811445297349	-2.71569726071395\\
-1.91410620476902	-2.70727671131428\\
-1.91996713155655	-2.69782174700842\\
-1.92569398016566	-2.68712028512562\\
-1.93126754139454	-2.67489728134176\\
-1.93664582735851	-2.66079180389336\\
-1.94175305655896	-2.64432345531191\\
-1.94646174838001	-2.62484223712471\\
-1.95056317919443	-2.6014520545482\\
-1.9537176319937	-2.57289118093351\\
-1.95536854304165	-2.53734057205639\\
-1.95459018715304	-2.49210796544402\\
-1.94980926975556	-2.43309269516865\\
-1.93828054844667	-2.35385588598758\\
-1.91507345139156	-2.24397836189112\\
-1.87109236299913	-2.08618419617105\\
-1.78932984678071	-1.85168026808045\\
-1.63883898093533	-1.49478489388947\\
-1.37068276603309	-0.956723276792841\\
-0.938084355572819	-0.210099538651129\\
-0.368170161346866	0.646002603077406\\
0.204498185789562	1.40023954540039\\
0.660626177511073	1.92880903266141\\
0.981154247279146	2.2560043438161\\
1.19872262954689	2.45157412015697\\
1.34824136987672	2.56968263421268\\
1.45410936810425	2.64287391127373\\
1.53163409357617	2.68947336578912\\
1.59025726812783	2.71981211549013\\
1.63589119605128	2.73985964858191\\
1.672335905501	2.75317592808163\\
1.7021050431826	2.76195291278141\\
1.72690876815852	2.76758092987106\\
1.74794154747409	2.77096544098006\\
1.76605796463212	2.77270965210016\\
1.781882940209	2.77322293973926\\
};
\addplot [color=mycolor1, forget plot]
  table[row sep=crcr]{%
4.99561704614432	3.93093778001022\\
5.12441715718723	3.92700594018682\\
5.22749359751082	3.91727727481723\\
5.31147431550237	3.90392417940075\\
5.38100855098295	3.88828361280772\\
5.4394192891387	3.87118259644813\\
5.48912451149793	3.853130710441\\
5.53191316501664	3.83443589164418\\
5.56912894315978	3.81527522357547\\
5.60179472704873	3.79573879287683\\
5.63069819801756	3.77585708918583\\
5.65645159570325	3.75561812131712\\
5.67953394460403	3.73497793935292\\
5.70032116110036	3.71386678908289\\
5.71910759992558	3.69219224168065\\
5.73612139531802	3.6698400907359\\
5.75153514953909	3.64667344917968\\
5.7654729662991	3.62253022618498\\
5.77801442013848	3.59721896598886\\
5.78919572749572	3.57051285010664\\
5.79900808959564	3.54214147165365\\
5.80739286480359	3.51177975532787\\
5.81423284683676	3.4790330822595\\
5.81933840613643	3.44341723417179\\
5.82242649126247	3.40433111923476\\
5.82308931925251	3.36101926251882\\
5.82074773116208	3.31251954517394\\
5.81458117655192	3.25758934794285\\
5.80342129046852	3.19459959207491\\
5.7855875771488	3.12138036097956\\
5.75862922647238	3.034992555466\\
5.71891201929251	2.93138554196882\\
5.66094598091564	2.80487890431387\\
5.57627654710481	2.64737706978255\\
5.45164906545852	2.44719972533898\\
5.26602389169874	2.18744287372611\\
4.98604568915978	1.8440724380056\\
4.56051953116549	1.385080343331\\
3.91861389651914	0.775217544816685\\
2.98722249841715	-0.0039109304312002\\
1.75058309585665	-0.912657960913425\\
0.328315571224359	-1.82653552024029\\
-1.04986979985048	-2.59477729602087\\
-2.19910607754332	-3.14384399650946\\
-3.06935198955974	-3.49401988568186\\
-3.7005382255588	-3.70261632449116\\
-4.15499533889883	-3.82140591624197\\
-4.48601544731098	-3.88578484184051\\
-4.7319291316338	-3.91755636433422\\
-4.91871783456797	-3.92969230616816\\
-5.06374868767354	-3.92987792432078\\
-5.1787004733027	-3.92270577353361\\
-5.27154301172934	-3.91095528868635\\
-5.34781428514036	-3.89632892311543\\
-5.4114358091086	-3.87987800734067\\
-5.46523604475778	-3.86225253481791\\
-5.51129068081925	-3.84385014702867\\
-5.55114736942836	-3.82490650236635\\
-5.58597664418855	-3.80555091596297\\
-5.61667493956786	-3.78584100285091\\
-5.64393599650679	-3.76578435055823\\
-5.66830102740274	-3.74535198599743\\
-5.69019434122226	-3.72448650066768\\
-5.70994881233312	-3.70310656504833\\
-5.72782408732873	-3.68110886870713\\
-5.7440194449165	-3.6583680807747\\
-5.75868256018706	-3.63473512752232\\
-5.77191495372025	-3.61003386380834\\
-5.78377454721926	-3.58405602983503\\
-5.79427544279993	-3.556554201649\\
-5.80338474493174	-3.52723223388266\\
-5.81101590439257	-3.49573242376901\\
-5.81701762371723	-3.46161825359495\\
-5.82115673986917	-3.42435103214597\\
-5.82309256192751	-3.38325795858695\\
-5.82233867563938	-3.3374879231185\\
-5.81820587077501	-3.28594949383599\\
-5.80971597542463	-3.22722262281281\\
-5.79546989489136	-3.15943099493215\\
-5.77344210597716	-3.08005461689224\\
-5.74065481114691	-2.98565064564301\\
-5.6926519686067	-2.87143251714126\\
-5.62263692343476	-2.73063152630738\\
-5.52004510004058	-2.55353455835752\\
-5.36819308466233	-2.32608261267201\\
-5.14054916221915	-2.02803325817305\\
-4.79549424832186	-1.63128596818774\\
-4.27157764880956	-1.10094716298679\\
-3.49246762041912	-0.406261979656713\\
-2.40332781996221	0.448023197706848\\
-1.04933145655836	1.37913204945238\\
0.380773929650926	2.23607011864938\\
1.65928387635527	2.89706136864684\\
2.66777140341069	3.34038976153233\\
3.41090951800739	3.61241114041432\\
3.94617389039209	3.77064798880819\\
4.33320444411984	3.85877485260488\\
4.61774349464813	3.90477787832598\\
4.83146612136199	3.92550583087053\\
4.99561704614432	3.93093778001022\\
};
\addplot [color=mycolor2, forget plot]
  table[row sep=crcr]{%
1.41336186657461	2.6452476547433\\
1.42661771092012	2.6448335028727\\
1.438711252913	2.6436839279649\\
1.44984844049261	2.64190594243159\\
1.46019496974537	2.6395722439554\\
1.46988592532474	2.63672910925978\\
1.47903283490991	2.63340160243155\\
1.48772889170151	2.62959686997006\\
1.49605285324855	2.62530599884741\\
1.50407196073714	2.62050471464515\\
1.51184410942881	2.61515305224684\\
1.51941941835957	2.60919401406851\\
1.52684128198338	2.60255111996953\\
1.53414692742136	2.59512463047509\\
1.54136743812141	2.58678607029672\\
1.54852712580497	2.57737046498136\\
1.55564201997255	2.56666538876986\\
1.56271706952095	2.55439544060589\\
1.56974136514771	2.54020000811172\\
1.57668020608856	2.52360095861468\\
1.58346198732936	2.50395488348667\\
1.58995636420672	2.48038113089167\\
1.59593735844811	2.45165102966057\\
1.60101980835639	2.41601348883995\\
1.60454742630167	2.37091401188447\\
1.60539083148236	2.31253186959236\\
1.60157464345816	2.23500407079215\\
1.58957687332405	2.12911610833666\\
1.56301143478508	1.9801432277843\\
1.5102532766994	1.76463817519863\\
1.41081821954095	1.44727205128183\\
1.23288596616825	0.984345860164834\\
0.943081411046606	0.352273078517919\\
0.54428566688769	-0.391140171268018\\
0.107904047733924	-1.09554150842607\\
-0.276044760851313	-1.63657136712392\\
-0.569184196744573	-1.99892328105359\\
-0.77995959606827	-2.22811087339792\\
-0.930346758973861	-2.37209991875548\\
-1.03947457874975	-2.46402444534223\\
-1.12072825956013	-2.52406730683613\\
-1.18290548049581	-2.56416308963738\\
-1.23174035972648	-2.59142346698548\\
-1.27101822274564	-2.61018011835085\\
-1.30329024585911	-2.62314182793299\\
-1.3303159546295	-2.63204789657639\\
-1.3533370722357	-2.63804298620778\\
-1.37324926014167	-2.64189713696169\\
-1.39071186785787	-2.64413816644854\\
-1.40621953027101	-2.64513332474643\\
-1.42014985394465	-2.64514082425715\\
-1.43279581960538	-2.64434303531236\\
-1.44438821946604	-2.64286824669118\\
-1.45511147030128	-2.64080511765368\\
-1.46511494034323	-2.63821233899968\\
-1.47452118274717	-2.63512506370108\\
-1.48343199755177	-2.63155908346106\\
-1.49193294054518	-2.62751335986359\\
-1.50009669731347	-2.62297127778391\\
-1.50798560497077	-2.61790082096134\\
-1.51565350807117	-2.61225374145689\\
-1.52314706266683	-2.6059636828706\\
-1.53050654150842	-2.59894310284286\\
-1.5377661337854	-2.59107870464897\\
-1.54495366368046	-2.5822249073791\\
-1.55208955895927	-2.57219462609524\\
-1.55918476169539	-2.56074624579452\\
-1.56623705092564	-2.54756507140195\\
-1.573224876085	-2.53223657740225\\
-1.58009716105139	-2.51420721748646\\
-1.5867564075857	-2.49272594887438\\
-1.59303037348538	-2.46675518971048\\
-1.59862377892076	-2.43483222422706\\
-1.60303420978513	-2.39484846990763\\
-1.60540220859596	-2.34368979276673\\
-1.60423757597317	-2.27663824466107\\
-1.59690900318884	-2.18636364362759\\
-1.57868173742978	-2.06123237273783\\
-1.54093102719399	-1.88261546525284\\
-1.46810249276303	-1.62137238292075\\
-1.33400160379627	-1.23656406937439\\
-1.10326361017691	-0.688505376792297\\
-0.754348062540462	0.0136382320891431\\
-0.324477087888203	0.758366136414363\\
0.0945266002362541	1.38980765530617\\
0.434093811825414	1.83775491680209\\
0.683522071725052	2.12686100309367\\
0.8613696880509	2.30832381252633\\
0.989104532384684	2.42306408013362\\
1.08294774530413	2.49713991499516\\
1.15378656170739	2.54608051781412\\
1.20871846719522	2.57907857516757\\
1.25239176239966	2.60166685117974\\
1.28790527516386	2.61725928393709\\
1.3173715371126	2.62801946673208\\
1.34226451525509	2.63535427264229\\
1.36363601743509	2.64020015530037\\
1.3822526363932	2.64319329085787\\
1.39868416267362	2.64477324354987\\
1.41336186657461	2.6452476547433\\
};
\addplot [color=mycolor1, forget plot]
  table[row sep=crcr]{%
6.84088886676008	4.99913208029814\\
6.96761917857306	4.99527446465504\\
7.06755331315122	4.98584932276291\\
7.14802261607397	4.97305911339979\\
7.21401975286968	4.95821728261124\\
7.26902978728066	4.94211411572269\\
7.31554052669344	4.92522402903217\\
7.35536390507287	4.90782600632662\\
7.38984331799043	4.89007512858044\\
7.41999042362753	4.87204581526907\\
7.4465772713189	4.85375826900637\\
7.47019948081834	4.83519467923873\\
7.49132023875733	4.81630899908268\\
7.51030129726572	4.79703254482266\\
7.52742495531667	4.77727674831561\\
7.54290961255317	4.75693383296865\\
7.5569205823831	4.73587582594179\\
7.56957724233974	4.71395207172416\\
7.58095716552604	4.69098521945959\\
7.59109753889791	4.66676548027988\\
7.5999938747872	4.64104276040482\\
7.60759571140593	4.61351603914356\\
7.61379862175913	4.58381903830862\\
7.6184313363036	4.55150076352845\\
7.62123602336253	4.51599879927524\\
7.62183858316341	4.47660216315016\\
7.61970388775799	4.43239882863183\\
7.61406769191097	4.38220029407415\\
7.60383144924698	4.32443108425763\\
7.58739662284487	4.25696352991992\\
7.56239769899336	4.1768652550648\\
7.52526100347198	4.08000429228032\\
7.47045574746095	3.96041707443767\\
7.38918722760005	3.80927485345324\\
7.26705835616366	3.61316623865749\\
7.07981294985811	3.35124052435176\\
6.78562883880548	2.99063248318109\\
6.31206390072326	2.4801833081198\\
5.53922240998824	1.74661149592949\\
4.30165628022422	0.712525630456547\\
2.48557351533814	-0.620688304855937\\
0.26402085414709	-2.04765566869058\\
-1.8687359569853	-3.2372039172557\\
-3.52815139697121	-4.03095814698191\\
-4.67630145924028	-4.49358687649038\\
-5.44305045173526	-4.7473157681752\\
-5.96001472258145	-4.8826087065026\\
-6.31831910812099	-4.95237714613542\\
-6.57479182115228	-4.98555661624868\\
-6.76422057514554	-4.99788833910659\\
-6.90818573778395	-4.99808676492652\\
-7.02040997793232	-4.99109349776442\\
-7.10986595001333	-4.97977722865751\\
-7.18258420021252	-4.96583597628711\\
-7.24272375925819	-4.95028803020279\\
-7.29322133832692	-4.93374643429515\\
-7.33619506303057	-4.91657652308681\\
-7.37320185482429	-4.8989885047452\\
-7.4054053934111	-4.88109297144779\\
-7.43368810964642	-4.86293468270704\\
-7.45872731796898	-4.84451327772963\\
-7.48104784848783	-4.82579590645338\\
-7.50105893333677	-4.80672470492933\\
-7.51908030042781	-4.78722084610942\\
-7.53536068244905	-4.76718618350808\\
-7.55009083275759	-4.74650306099718\\
-7.56341240252416	-4.72503256788109\\
-7.57542352380168	-4.70261130382071\\
-7.58618156519308	-4.67904653784841\\
-7.59570321451892	-4.65410946616483\\
-7.60396174412403	-4.62752606413669\\
-7.61088097851963	-4.59896475411069\\
-7.61632504963485	-4.56801972508052\\
-7.62008240431654	-4.53418817181828\\
-7.62184158267811	-4.49683885655334\\
-7.62115478012173	-4.4551680487666\\
-7.61738273052627	-4.40813675180086\\
-7.60961026522076	-4.35437963155908\\
-7.59651464829314	-4.29207025773525\\
-7.57615588237226	-4.21871742268348\\
-7.54563462676158	-4.13085028486\\
-7.50051934440824	-4.02352022392346\\
-7.43386033241791	-3.88949461361287\\
-7.33444657504917	-3.71792630922736\\
-7.18365457895993	-3.4921354354706\\
-6.94970302787466	-3.18596166290699\\
-6.57746887226027	-2.75822063891058\\
-5.97271943598497	-2.14656008491193\\
-4.98915068767605	-1.27049336483544\\
-3.4647769938628	-0.0761671675852215\\
-1.39846942708153	1.34372974891551\\
0.845794119376638	2.6886884672571\\
2.76685927831793	3.68281754683751\\
4.15924238449614	4.29571110156509\\
5.09825681476418	4.63989472459928\\
5.72610420536212	4.82573521306198\\
6.15479889280274	4.92346550355785\\
6.45671715430494	4.97233824045081\\
6.6763049251593	4.99366738199783\\
6.84088886676008	4.99913208029814\\
};
\addplot [color=mycolor2, forget plot]
  table[row sep=crcr]{%
0.523495985188513	2.21095812992688\\
0.543228512121635	2.21033588094888\\
0.562220859464411	2.20852524916379\\
0.58062348639751	2.20558242663241\\
0.59857107624324	2.20152953227212\\
0.616186173211566	2.19635696258384\\
0.6335821876191	2.19002390927179\\
0.650865900163185	2.18245712819836\\
0.668139552271565	2.1735478870409\\
0.685502566909632	2.16314685563153\\
0.703052896168851	2.15105651310548\\
0.720887929723092	2.13702040601345\\
0.739104808628203	2.12070827115511\\
0.757799850973994	2.10169559548784\\
0.777066575245817	2.07943556864067\\
0.796991447022335	2.05322052138817\\
0.817645880480452	2.02212875644123\\
0.839072043534282	1.98495110624118\\
0.861258399072199	1.94008964755294\\
0.884098302558147	1.88541917667427\\
0.907320920341571	1.81810169992184\\
0.930377965797729	1.73434916367601\\
0.952263194491732	1.62914930722371\\
0.971239355789409	1.4960241572878\\
0.984465785527444	1.3270165243409\\
0.987598438751026	1.11333546856789\\
0.974634416241003	0.847385341338023\\
0.938591949871787	0.526853328745189\\
0.873710716112844	0.160179721753477\\
0.778896374237475	-0.230310041134964\\
0.660066987294904	-0.612784015588797\\
0.528630790965447	-0.957467547825973\\
0.396777493547414	-1.24689744957127\\
0.273413498440531	-1.47776602135199\\
0.163031486303281	-1.6561500367302\\
0.0667094469319491	-1.79176889067705\\
-0.0163980118246023	-1.89431828124055\\
-0.0879334239880009	-1.97194566097251\\
-0.149699133735178	-2.0309668534891\\
-0.203365362938089	-2.07609011987583\\
-0.250369196107044	-2.11076539943113\\
-0.291903635828865	-2.13750761633118\\
-0.328942641967419	-2.15815283636643\\
-0.362276439623747	-2.17404814008715\\
-0.392546170745644	-2.18618812408863\\
-0.420273998952583	-2.19531176521621\\
-0.445887865276266	-2.20197089826355\\
-0.469741307298921	-2.20657867192898\\
-0.492129130424326	-2.20944393400595\\
-0.51329974773488	-2.21079569114687\\
-0.533464906173632	-2.21080050167315\\
-0.552807387145015	-2.20957476137215\\
-0.571487146611918	-2.2071932176451\\
-0.589646255366341	-2.20369460979581\\
-0.607412915652489	-2.19908502012294\\
-0.624904762922485	-2.19333928744042\\
-0.642231607068395	-2.18640064970422\\
-0.6594977214769	-2.1781786212345\\
-0.676803745849656	-2.16854495220916\\
-0.69424822429159	-2.15732734415148\\
-0.711928746334305	-2.14430038351473\\
-0.729942584674714	-2.12917287962482\\
-0.748386612912459	-2.11157041820197\\
-0.767356112690995	-2.09101142034797\\
-0.786941798226588	-2.06687426702275\\
-0.807223923936087	-2.0383520342906\\
-0.828261576687304	-2.00439000913586\\
-0.850073992278704	-1.96359939714325\\
-0.872608674221088	-1.91413867617342\\
-0.895687814770749	-1.85355268365995\\
-0.918919604020422	-1.77856110763088\\
-0.941554581125384	-1.6847986646771\\
-0.962261669610689	-1.56654291960342\\
-0.978803465833006	-1.41655046193693\\
-0.987632129793905	-1.22629992368928\\
-0.983559095423338	-0.987224987260853\\
-0.95992427292537	-0.693728548593595\\
-0.9099752585336	-0.348209238467667\\
-0.829844399657837	0.0339699757827997\\
-0.721851236154917	0.424610843801142\\
-0.595120868130539	0.791297011438935\\
-0.462094678343811	1.10958947459289\\
-0.333655905694098	1.3694178283083\\
-0.216470558877756	1.57294120700516\\
-0.113136514165607	1.72867233978889\\
-0.0235974610654696	1.84662458847378\\
0.0534986437817769	1.93581096326671\\
0.119928196544504	2.00345410619619\\
0.177449005002451	2.05502469963258\\
0.22761999764002	2.09455801098953\\
0.271754484623795	2.12499999714405\\
0.310931796261443	2.14849852256881\\
0.346029505257319	2.16662510211742\\
0.377759205625238	2.18053628765935\\
0.4066991226322	2.1910887605286\\
0.433321539256857	2.19892078291678\\
0.458014996438643	2.2045097844662\\
0.481101933251353	2.20821316547787\\
0.502852595133849	2.21029729226656\\
0.523495985188513	2.21095812992688\\
};
\addplot [color=mycolor1, forget plot]
  table[row sep=crcr]{%
6.86250796792293	5.01216327431788\\
6.98926231728329	5.00830501201048\\
7.0892040971588	4.99887920200613\\
7.1696723709199	4.98608919075086\\
7.2356639362686	4.97124863638639\\
7.29066611921641	4.95514778446123\\
7.33716798431924	4.93826093255989\\
7.37698216855827	4.92086693575256\\
7.41145245879218	4.90312076171995\\
7.44159072664735	4.8850967394826\\
7.46816913277388	4.86681500433133\\
7.49178334982416	4.84825769934232\\
7.51289658462578	4.82937874950955\\
7.53187059220708	4.8101094587785\\
7.5489876665659	4.790361260487\\
7.56446620043385	4.7700263919257\\
7.57847150229701	4.74897690563372\\
7.59112294956573	4.72706218251218\\
7.60249812227665	4.70410491905285\\
7.61263422345491	4.67989538484688\\
7.62152679279647	4.65418355600433\\
7.62912540959612	4.62666849341812\\
7.63532570446324	4.5969840120384\\
7.63995648517121	4.56467922102285\\
7.64276002044572	4.5291918154402\\
7.64336233688809	4.4898109229457\\
7.6412284595908	4.44562461062275\\
7.63559431755128	4.39544442207369\\
7.62536153911076	4.33769481597923\\
7.60893170603898	4.27024782131662\\
7.58393922779236	4.19017027676805\\
7.54680982482963	4.09332844236527\\
7.49201078226915	3.97375494172735\\
7.41074223820117	3.82261292741202\\
7.28859398844889	3.62647362156677\\
7.10127677227243	3.364448227552\\
6.80688429362408	3.0035862582056\\
6.33275976369098	2.49253691535988\\
5.55848013628344	1.75760613116161\\
4.31749294271232	0.720672658969268\\
2.49461507568364	-0.617516179954371\\
0.263345568688976	-2.05071979992666\\
-1.8785328447942	-3.245362799024\\
-3.5438740067486	-4.04196088661383\\
-4.69511885692381	-4.50584250495883\\
-5.46335414679059	-4.76006617664413\\
-5.98102449960256	-4.89554530075052\\
-6.33966919335148	-4.96538068726368\\
-6.59630802588342	-4.99858199844547\\
-6.78581731532883	-5.01091915477763\\
-6.92981960632722	-5.01111774157195\\
-7.04205839289073	-5.00412363463622\\
-7.1315169872655	-4.99280707632652\\
-7.20423156477816	-4.97886655640214\\
-7.26436420644563	-4.96332041832582\\
-7.31485330757953	-4.94678161368626\\
-7.35781793374445	-4.92961534825615\\
-7.3948155316173	-4.9120317075125\\
-7.42701007098371	-4.8941411814341\\
-7.45528413804713	-4.87598845082904\\
-7.48031512524895	-4.85757309838982\\
-7.50262789680656	-4.83886223714095\\
-7.52263169492403	-4.81979798318208\\
-7.54064624555851	-4.8003015042472\\
-7.55692027486454	-4.78027466167686\\
-7.57164452987791	-4.75959981905813\\
-7.58496065894079	-4.73813809662397\\
-7.59696679724758	-4.71572613590784\\
-7.60772032464288	-4.69217125881124\\
-7.617237950381	-4.66724472566727\\
-7.62549298070531	-4.64067258756152\\
-7.63240928909447	-4.61212335426371\\
-7.63785107451925	-4.58119131334615\\
-7.64160687227093	-4.54737376726125\\
-7.64336533556924	-4.51003958990812\\
-7.64267879968886	-4.46838515534314\\
-7.63890816347927	-4.42137154178832\\
-7.63113843688252	-4.3676334154589\\
-7.61804704135371	-4.30534418336554\\
-7.59769402623181	-4.23201213696242\\
-7.5671797709553	-4.1441652432129\\
-7.52207161887795	-4.03685227518945\\
-7.45541668020137	-3.90283505324939\\
-7.35599568951142	-3.73125459015708\\
-7.2051643343779	-3.50540534944489\\
-6.97108861704537	-3.19907014224744\\
-6.59851096802524	-2.77093651021558\\
-5.99285900701924	-2.15836741926386\\
-5.00703994044869	-1.28030472689602\\
-3.4776900948767	-0.0820928732758431\\
-1.40284278372905	1.34366140761687\\
0.85134661085044	2.69457031415574\\
2.78004453523949	3.69265884948226\\
4.17678575758048	4.3074785408114\\
5.11795459925128	4.65245605656726\\
5.74682533859377	4.83860148115811\\
6.17600915738681	4.93644425192202\\
6.47816500031136	4.98535593273646\\
6.69786876188027	5.00669659737774\\
6.86250796792293	5.01216327431788\\
};
\addplot [color=mycolor2, forget plot]
  table[row sep=crcr]{%
-0.733575921492504	1.31460374341902\\
-0.632596108039127	1.3113410779402\\
-0.520663702370314	1.3005890110674\\
-0.397263074514887	1.2807737843542\\
-0.262341623792816	1.25022656161158\\
-0.11652851656645	1.20733710372774\\
0.0386574440584269	1.15078286280803\\
0.200679110745031	1.07981345555107\\
0.366013660074528	0.994532769103677\\
0.530406661832258	0.896088323134107\\
0.689338991169707	0.786677146174762\\
0.838605846799504	0.6693274655713\\
0.974845201924293	0.547500688560829\\
1.095867236989	0.424630275227481\\
1.20072115263929	0.303728526329512\\
1.28953654070531	0.187146049008567\\
1.36323727351282	0.076498504484224\\
1.42322986832842	-0.0272778744932675\\
1.47113557370203	-0.12379976936684\\
1.50859537167969	-0.213089685086319\\
1.53714783701748	-0.295440389365299\\
1.5581653382286	-0.371305494710742\\
1.57283062677761	-0.441218943712308\\
1.582138147318	-0.505740022815961\\
1.58690854061136	-0.565418369279737\\
1.58780877458732	-0.620773443737429\\
1.58537339560882	-0.672283856281868\\
1.58002447744766	-0.720383054726347\\
1.57208914862459	-0.765458892081875\\
1.56181433196347	-0.807855384239133\\
1.54937872858848	-0.847875547553941\\
1.53490226248587	-0.88578460612647\\
1.51845326346976	-0.921813123485746\\
1.50005366306014	-0.956159781172203\\
1.47968244310377	-0.988993626619511\\
1.45727753006904	-1.02045566565407\\
1.43273627943341	-1.05065969493691\\
1.40591465056035	-1.0796922657676\\
1.37662513722201	-1.10761164827412\\
1.34463349691367	-1.13444562723217\\
1.30965431967807	-1.16018790954467\\
1.27134550445155	-1.18479286088385\\
1.22930178423368	-1.20816821926737\\
1.18304758612331	-1.23016536542467\\
1.13202976680905	-1.25056668239289\\
1.07561118349157	-1.26906954588727\\
1.01306671797197	-1.28526661767614\\
0.943584354520575	-1.29862247668178\\
0.866275297135439	-1.30844739031735\\
0.780198905747261	-1.31387044835238\\
0.684410242848602	-1.31381664823958\\
0.57803963763987	-1.3069960618567\\
0.460413542584466	-1.29191779453278\\
0.331221736278588	-1.26694601494493\\
0.190724547621685	-1.23041717488425\\
0.0399730394022008	-1.18083190374211\\
-0.119012709766546	-1.11711650257582\\
-0.283185610920674	-1.03891590754923\\
-0.448610699077059	-0.946841825316007\\
-0.610832441756026	-0.842579774836903\\
-0.765415585653945	-0.728782995388376\\
-0.908520369919128	-0.608753247524889\\
-1.03734588454356	-0.485994123255349\\
-1.15032942969175	-0.363769276058191\\
-1.2470907949759	-0.244779334817512\\
-1.32819617988291	-0.131007247923902\\
-1.39484812893842	-0.023716225666711\\
-1.44858972346348	0.0764520407182007\\
-1.49107166189482	0.169335335020153\\
-1.52389492325475	0.255106777306546\\
-1.54852001943411	0.334151396951494\\
-1.56622557067635	0.406971321366467\\
-1.57809896662283	0.47411865229053\\
-1.58504544890288	0.536151200579106\\
-1.58780615864932	0.593605370382616\\
-1.58697923972185	0.646981064609402\\
-1.5830406446606	0.696734560951497\\
-1.57636295534998	0.743276395173477\\
-1.567231530928	0.786972193951703\\
-1.5558578502988	0.828145082622651\\
-1.54239019348571	0.867078777152041\\
-1.52692191983323	0.904020796901718\\
-1.50949762482587	0.939185446898803\\
-1.49011743515793	0.972756349052911\\
-1.46873965915509	1.00488837603656\\
-1.4452819609737	1.03570887660166\\
-1.41962118013877	1.06531808828617\\
-1.39159187784586	1.09378861980928\\
-1.36098366219232	1.12116385507696\\
-1.32753733130181	1.14745508600307\\
-1.2909398841505	1.17263712414442\\
-1.25081849666718	1.19664207428242\\
-1.20673366580052	1.21935088267\\
-1.15817191813318	1.24058221179928\\
-1.1045388085493	1.2600781683996\\
-1.04515346275685	1.27748647074099\\
-0.979246726695719	1.29233887058039\\
-0.905966162524658	1.30402618397681\\
-0.824392727841595	1.31177134385085\\
-0.733575921492505	1.31460374341902\\
};
\addplot [color=mycolor1, forget plot]
  table[row sep=crcr]{%
4.82273228002968	3.83720887290188\\
4.95224699965661	3.83325358762484\\
5.05611778584741	3.82344889459213\\
5.14089177706792	3.80996895800022\\
5.21118152874184	3.79415795626632\\
5.27029535935223	3.77685073784405\\
5.32064745275308	3.75856365681821\\
5.36402814716335	3.73960996602969\\
5.40178480740073	3.72017066332652\\
5.43494478132041	3.70033854357982\\
5.46430026698535	3.68014580834553\\
5.49046771058867	3.65958136247208\\
5.51392987214387	3.63860147624619\\
5.53506587201003	3.61713604058954\\
5.55417272393345	3.59509176286132\\
5.57148068066435	3.57235310028877\\
5.58716392840006	3.54878136750211\\
5.60134761773018	3.52421220091616\\
5.61411181537904	3.49845136332545\\
5.62549263681689	3.47126869055441\\
5.63548052402979	3.4423897885073\\
5.64401531919005	3.41148485356697\\
5.65097740203965	3.3781536756374\\
5.65617363775665	3.34190544085221\\
5.65931612011444	3.30213130495984\\
5.65999052751267	3.2580667421089\\
5.65760906390394	3.20873920256256\\
5.65133996804031	3.15289434124797\\
5.64000063885971	3.0888905337548\\
5.62189313791285	3.01454582992094\\
5.59454673765195	2.92691277479322\\
5.55430807172166	2.82194311606146\\
5.49567845866715	2.69398488798758\\
5.41023066111837	2.5350299931434\\
5.28483775577398	2.33361496287792\\
5.09884485169681	2.07332964712349\\
4.81990247355855	1.73120544840765\\
4.39920671023066	1.27738149278302\\
3.77091717341521	0.680385231150418\\
2.86995021215103	-0.0733906543835908\\
1.68696680769167	-0.942802236466377\\
0.335242594647828	-1.8113829036668\\
-0.975850751363337	-2.54218196592073\\
-2.07722399358059	-3.06831338957807\\
-2.91971033690261	-3.40726568298498\\
-3.53681121128057	-3.61117424376579\\
-3.98482032351319	-3.72825990601219\\
-4.31328463216064	-3.79213151854898\\
-4.55853705626038	-3.82381182226538\\
-4.74555312990996	-3.83595915425882\\
-4.89120350145804	-3.8361435067725\\
-5.00692397817158	-3.82892209014364\\
-5.10056696603678	-3.81706943662609\\
-5.17761549046581	-3.80229342988045\\
-5.24196729344291	-3.7856532643666\\
-5.29644256945701	-3.76780633590028\\
-5.34311629806293	-3.74915634157748\\
-5.38353891570215	-3.72994353477927\\
-5.41888515330459	-3.71030051495013\\
-5.4500560003726	-3.69028708231215\\
-5.4777495889861	-3.66991210828325\\
-5.50251111476878	-3.649147162936\\
-5.52476836016115	-3.62793476012899\\
-5.54485713109196	-3.60619295469908\\
-5.56303946198115	-3.58381733326525\\
-5.5795164823624	-3.56068099779297\\
-5.5944371838535	-3.53663284196546\\
-5.60790385994208	-3.5114941991508\\
-5.61997463426678	-3.48505375427219\\
-5.63066318889856	-3.45706042796672\\
-5.63993550533676	-3.42721373096009\\
-5.64770308990145	-3.39515081727613\\
-5.65381171345032	-3.36042909447345\\
-5.65802406972734	-3.32250271645461\\
-5.65999381797042	-3.28069049628806\\
-5.65922701260224	-3.23413158650851\\
-5.65502458055458	-3.18172344909521\\
-5.64639567479162	-3.12203380256001\\
-5.63192534913398	-3.0531737927659\\
-5.6095692067967	-2.97261265973941\\
-5.57632924476756	-2.87690331670947\\
-5.52773361511053	-2.76127190280889\\
-5.45699012819906	-2.61900185992461\\
-5.35359971688748	-2.44051986601403\\
-5.20110634735667	-2.21209661788149\\
-4.97360823102457	-1.91421992438641\\
-4.63105640072421	-1.52031875631979\\
-4.11552958912839	-0.998417660583009\\
-3.35732869804741	-0.322290494927053\\
-2.30994690716635	0.499343820936391\\
-1.01998239562171	1.38648817003772\\
0.338534070191457	2.20051889899191\\
1.55857214384513	2.83121730922997\\
2.52985081867727	3.25812564683877\\
3.25294571500252	3.52276759202078\\
3.7785533816478	3.67812488620809\\
4.16142077655795	3.76529004425864\\
4.44452431404327	3.81105331235671\\
4.65811579031653	3.83176414287705\\
4.82273228002968	3.83720887290188\\
};
\addplot [color=mycolor2, forget plot]
  table[row sep=crcr]{%
-0.409091368008181	0.847807163764681\\
0.0363872777318144	0.833551248542955\\
0.4817250507643	0.791124827733238\\
0.888249773018939	0.726348923044037\\
1.23188738207708	0.649090012089854\\
1.50689824214324	0.568693910692423\\
1.71996212285585	0.491453974327295\\
1.88260397700221	0.420525765118745\\
2.00639290543533	0.356905955764647\\
2.10100336756632	0.300416597818771\\
2.17388392469238	0.250362608748835\\
2.2305535888983	0.205892398119981\\
2.27503824215836	0.166169537837721\\
2.31026475499303	0.130441927877661\\
2.33836800502053	0.0980607706484964\\
2.36091562510899	0.0684772729283638\\
2.37906807968813	0.0412308829428764\\
2.39369148912165	0.0159354581143135\\
2.40543701739583	-0.00773391989835962\\
2.41479689131733	-0.0300525119747493\\
2.42214411567697	-0.0512556852703417\\
2.42776075688613	-0.071547013686774\\
2.43185813181773	-0.0911049652894258\\
2.43459118001494	-0.110088443142424\\
2.43606856940276	-0.128641425045538\\
2.43635958053514	-0.146896915618108\\
2.43549845661457	-0.164980392485317\\
2.4334866433849	-0.183012901412597\\
2.43029313715	-0.201113934326957\\
2.42585298260436	-0.219404209021938\\
2.4200637916992	-0.238008458984596\\
2.41277996830448	-0.25705833460852\\
2.40380409637995	-0.276695510518368\\
2.39287465074495	-0.297075083466383\\
2.37964877707909	-0.318369323368185\\
2.36367830171839	-0.34077179180409\\
2.34437628718835	-0.364501740005801\\
2.32097022868314	-0.389808489635322\\
2.29243623898783	-0.416975085835033\\
2.25740613658626	-0.446319707981462\\
2.21403617770271	-0.478191786707268\\
2.15982264227314	-0.512956889308037\\
2.09134735282723	-0.550959152568429\\
2.00394095050149	-0.592440809000024\\
1.89127791599731	-0.637383660783571\\
1.74499861443862	-0.685218755545888\\
1.5546537323195	-0.73434281979572\\
1.30865514096727	-0.781435534003174\\
0.99738778007668	-0.820806041810157\\
0.619400593165738	-0.84450103591156\\
0.188952459819122	-0.844293417387755\\
-0.261696494489053	-0.815644863467068\\
-0.69179486850701	-0.760951163274546\\
-1.0686755863563	-0.688639413498638\\
-1.37774703986944	-0.608809307923861\\
-1.6204938645077	-0.529418433597329\\
-1.80683005131328	-0.455100938365687\\
-1.94869725624527	-0.387795309093099\\
-2.05682977985848	-0.327806702640065\\
-2.13977481288561	-0.27463742208824\\
-2.20396352356178	-0.227483314092969\\
-2.25411439088667	-0.18548691140633\\
-2.29365976976601	-0.147849323331971\\
-2.32509766189118	-0.113869568607998\\
-2.35025571162444	-0.0829500057831151\\
-2.37048128221355	-0.0545876561121953\\
-2.38677592984353	-0.0283608978100306\\
-2.39989003031857	-0.00391575783331045\\
-2.41038943186043	0.0190465212089434\\
-2.418702593801	0.0407801420528644\\
-2.42515408608277	0.0615036573149588\\
-2.42998848383307	0.0814073298086471\\
-2.43338741492735	0.100659214206848\\
-2.43548164095655	0.119410218566654\\
-2.43635944668375	0.137798375621783\\
-2.4360721894412	0.15595252101642\\
-2.43463755492039	0.173995546059013\\
-2.43204083528246	0.192047368630618\\
-2.42823435741328	0.210227747942568\\
-2.42313501838075	0.228659056246865\\
-2.4166197095217	0.24746911208251\\
-2.40851820720412	0.266794173188343\\
-2.39860284992443	0.286782179523592\\
-2.38657397207145	0.307596322070028\\
-2.37203957427972	0.329418980482539\\
-2.35448700745897	0.352456002023506\\
-2.33324343183192	0.376941147473959\\
-2.30742034857198	0.403140234953994\\
-2.2758354289806	0.431353934491819\\
-2.23690205457826	0.46191705234338\\
-2.18847352812088	0.495190032571688\\
-2.12762571293684	0.53153448224751\\
-2.05036208875032	0.571257487717239\\
-1.95123794787654	0.614497646681818\\
-1.8229482566059	0.661008445474991\\
-1.65605455441742	0.709778029129612\\
-1.43931539355192	0.758438100532923\\
-1.16155468723619	0.802541670170139\\
-0.816285878889351	0.835160767177662\\
-0.409091368008183	0.84780716376468\\
};
\addplot [color=mycolor1, forget plot]
  table[row sep=crcr]{%
3.32646336156982	3.11950096423544\\
3.4673413537824	3.11517424948534\\
3.58374016942398	3.10417038463928\\
3.68108791118803	3.08867944763077\\
3.76345208436014	3.07014401802894\\
3.83390080001395	3.04951196054361\\
3.89476794811634	3.02740122541234\\
3.94784519226144	3.00420724774397\\
3.994520472178	2.98017315169741\\
4.03587809210986	2.95543579797172\\
4.07277136817759	2.93005601422601\\
4.1058756325731	2.90403833369911\\
4.13572707628554	2.87734364507239\\
4.16275126496685	2.84989692604319\\
4.18728400030374	2.82159143566068\\
4.20958637530507	2.79229021225691\\
4.22985527769723	2.76182536074797\\
4.24823015242909	2.72999534805555\\
4.26479648414853	2.6965603133532\\
4.27958615827755	2.6612352082174\\
4.29257456593334	2.6236803840576\\
4.3036739942792	2.58348901526293\\
4.31272244412135	2.54017045864858\\
4.31946648004315	2.49312826914687\\
4.32353595839411	2.44163107470658\\
4.3244073661675	2.38477380401515\\
4.32135084517486	2.32142579094374\\
4.31335348121521	2.2501609803709\\
4.2990076892852	2.16916380032119\\
4.27634797012147	2.07610236547164\\
4.24261137132037	1.9679591029565\\
4.19388654870193	1.84080937778487\\
4.12460537108111	1.68954583103729\\
4.02682820492829	1.50757102235393\\
3.88930626934119	1.28654804716912\\
3.69644156479011	1.01645458273428\\
3.42765412937966	0.686491934124628\\
3.05850682441724	0.287838803306029\\
2.5661047161497	-0.180634561274986\\
1.94114780808001	-0.704170530007748\\
1.20371230481806	-1.2466740267169\\
0.4093959035994	-1.75723902570244\\
-0.367691869692907	-2.19015566537185\\
-1.06586467664211	-2.52327489742262\\
-1.65471167076941	-2.75980784569078\\
-2.13231136223699	-2.91734668343342\\
-2.51239515758807	-3.01649960533499\\
-2.81340129671657	-3.07491758745073\\
-3.05271283203895	-3.1057586216582\\
-3.24466165286656	-3.11818053282271\\
-3.40036685358521	-3.11834771041054\\
-3.52822452426209	-3.11034882452227\\
-3.63451407501058	-3.09688160317952\\
-3.72393128262089	-3.07972373122979\\
-3.80000535223279	-3.06004519381678\\
-3.86540922520523	-3.03861246415104\\
-3.92218542946281	-3.01592151010757\\
-3.97190915567376	-2.99228458614822\\
-4.01580594520563	-2.96788707091709\\
-4.05483690688043	-2.94282478363867\\
-4.08976073081524	-2.91712844319707\\
-4.12117904297665	-2.8907795265247\\
-4.14956968723582	-2.86372024646818\\
-4.17531113792668	-2.8358593807395\\
-4.19870026860253	-2.80707503683381\\
-4.21996500508263	-2.7772150024272\\
-4.23927288051749	-2.74609502346451\\
-4.25673611981222	-2.71349511828296\\
-4.2724135596746	-2.67915383784852\\
-4.28630941748958	-2.64276019064817\\
-4.29836861884402	-2.60394274067567\\
-4.30846803824527	-2.5622551323156\\
-4.31640254833021	-2.51715696644361\\
-4.32186413763414	-2.46798850942894\\
-4.32441144040951	-2.41393711139262\\
-4.32342566597318	-2.35399237957271\\
-4.31804688193366	-2.28688602589242\\
-4.30708154515252	-2.21101082677981\\
-4.28886759794017	-2.12431132081434\\
-4.26107676054016	-2.02413701363562\\
-4.2204244039374	-1.90704797673904\\
-4.16224618983675	-1.76856567553848\\
-4.07989195483255	-1.60287585146393\\
-3.96389676347797	-1.40253200632372\\
-3.80096093637043	-1.15831208656481\\
-3.57300992369185	-0.859605020688629\\
-3.25719868151819	-0.496092458045106\\
-2.82880022614507	-0.0618785465303164\\
-2.26979422053179	0.437272257365636\\
-1.58377242826251	0.976075726774439\\
-0.809107763453254	1.50920352669317\\
-0.0142167475127106	1.98545362547987\\
0.729332634606697	2.3694845312444\\
1.37460171844811	2.65269650078429\\
1.90677228309312	2.8471362738584\\
2.33340934238541	2.97301567219907\\
2.67163537044435	3.04987357854242\\
2.93979454170461	3.09313057854113\\
3.1538417162047	3.11382835567295\\
3.32646336156982	3.11950096423544\\
};
\addplot [color=mycolor2, forget plot]
  table[row sep=crcr]{%
0.560167015110667	0.853244053900496\\
1.06707393089475	0.837614568300724\\
1.47679824178267	0.799019827265189\\
1.78965952386821	0.749429636354734\\
2.02226091251516	0.697266211987803\\
2.1941524994748	0.647072601752934\\
2.32194042341777	0.600766346427937\\
2.41808474955201	0.558840228174232\\
2.49146704461956	0.521121780938159\\
2.54830649589018	0.487177503668587\\
2.59295308266754	0.456506966732256\\
2.62847138587656	0.428627544501524\\
2.65704463224663	0.403106100218799\\
2.68024800218523	0.379566373857723\\
2.69923259434205	0.357686050860287\\
2.71484974774595	0.337190063938117\\
2.72773576375632	0.317843067151704\\
2.73837023202117	0.299442302209988\\
2.74711659991187	0.281811278308091\\
2.75425064661735	0.264794329660224\\
2.75998059101263	0.24825196618472\\
2.76446130416895	0.232056879954383\\
2.76780426913872	0.216090458887398\\
2.77008437581847	0.200239663848504\\
2.77134425673718	0.184394133581673\\
2.77159659503874	0.168443387858922\\
2.77082462449488	0.152273999962742\\
2.76898086106356	0.135766603317522\\
2.76598393012495	0.118792582172281\\
2.76171315775648	0.101210270872707\\
2.75600034892523	0.0828604480786408\\
2.74861784118104	0.0635608583348252\\
2.73926144240195	0.0430994204588267\\
2.72752614849146	0.0212256878847664\\
2.71287145602787	-0.00235998808074396\\
2.69457142269093	-0.0280202621363476\\
2.67164205333901	-0.0561983212534008\\
2.64273460263223	-0.0874398238642967\\
2.60597729363336	-0.122419446196449\\
2.55873899524786	-0.16197063710764\\
2.49727644824871	-0.207112973876854\\
2.41621472705903	-0.25906134742367\\
2.30781294906669	-0.319178239043526\\
2.16103090376812	-0.388782584104714\\
1.96066602867023	-0.46864379618366\\
1.68754270561763	-0.557892575256645\\
1.32214543016994	-0.652168158065221\\
0.855139672833289	-0.741662557912\\
0.304000364149641	-0.811679733978347\\
-0.278731448373542	-0.848741899301714\\
-0.824668087485252	-0.849099418132795\\
-1.28465122185451	-0.820383506369671\\
-1.6444858391152	-0.77497166528901\\
-1.9147387932445	-0.72330745184811\\
-2.11466040306014	-0.671757944162537\\
-2.26268638490528	-0.623380385091793\\
-2.37333988344336	-0.579257040242056\\
-2.45718036768567	-0.539479019780047\\
-2.52164678111486	-0.503708930612821\\
-2.57193772497335	-0.471464138077679\\
-2.61169988858062	-0.442246757546901\\
-2.64351594898488	-0.415596919174394\\
-2.66923735914705	-0.391109861495715\\
-2.69020855922654	-0.368436997682488\\
-2.70741807953273	-0.347280568397312\\
-2.72160104643071	-0.327386299517498\\
-2.73330938581511	-0.308535984071788\\
-2.74296040698015	-0.290540731779852\\
-2.75087075690752	-0.273235092187402\\
-2.75728033616711	-0.256472024068654\\
-2.76236920991791	-0.240118592446481\\
-2.76626952842005	-0.224052247118856\\
-2.76907379540857	-0.208157535512617\\
-2.77084036426975	-0.192323110178293\\
-2.77159672029497	-0.1764388988405\\
-2.77134086887585	-0.160393308534797\\
-2.77004095713812	-0.144070332773498\\
-2.76763308176857	-0.127346420280503\\
-2.76401705350883	-0.110086943955474\\
-2.75904967228526	-0.092142077288357\\
-2.75253478283033	-0.0733418397797098\\
-2.74420898212786	-0.0534900098683465\\
-2.73372126704024	-0.032356520394106\\
-2.7206040346805	-0.00966784671947868\\
-2.70423150947638	0.014905222825905\\
-2.68375960308018	0.0417631560078521\\
-2.65803800694986	0.0713973388592425\\
-2.62548037540859	0.104414356051744\\
-2.58387101713977	0.141564491099818\\
-2.53007597315801	0.183771420326977\\
-2.45961364608305	0.232153476510912\\
-2.36603228770019	0.288011450466867\\
-2.24006518469893	0.352724342015756\\
-2.06866914692298	0.427428950372729\\
-1.83449282910597	0.512259192042037\\
-1.51738192768083	0.604872089801674\\
-1.10107594364602	0.698358248788268\\
-0.587556093100141	0.780086596003453\\
-0.0122557582588863	0.834865234224473\\
0.560167015110665	0.853244053900496\\
};
\addplot [color=mycolor1, forget plot]
  table[row sep=crcr]{%
2.4470975329268	2.85734050088096\\
2.59925956234538	2.85263748443431\\
2.72936239773121	2.84031623671661\\
2.84139874347278	2.82247155051008\\
2.93859222736187	2.80058642786175\\
3.0235313743472	2.77570091105713\\
3.09829197054803	2.74853551350831\\
3.16454044419819	2.71957967116439\\
3.22361771846211	2.68915438376271\\
3.27660568747186	2.6574562163269\\
3.32437921061262	2.62458795306691\\
3.36764642225098	2.59057968585606\\
3.40697974908862	2.55540298570755\\
3.44283955508847	2.51897998110674\\
3.47559189073129	2.48118857401167\\
3.50552143839477	2.44186459580276\\
3.53284041838165	2.40080138931511\\
3.5576939377613	2.35774706040887\\
3.58016200911911	2.31239944452605\\
3.60025821866181	2.2643986587764\\
3.61792476088008	2.21331694180605\\
3.63302325477194	2.15864530947419\\
3.64532038396865	2.09977636572004\\
3.65446692232179	2.03598240275462\\
3.65996807166613	1.96638771281183\\
3.66114219736428	1.88993384992122\\
3.6570639521016	1.80533650974888\\
3.64648641782961	1.7110329239271\\
3.62773537508888	1.60511957138689\\
3.59856753534355	1.48528233243545\\
3.55598465216144	1.34872632808139\\
3.49599945873118	1.19212300117946\\
3.41336285143547	1.01161127682071\\
3.3012949774795	0.802922570523745\\
3.15133200538093	0.561748137098855\\
2.95352110334314	0.284520029275175\\
2.6973552050403	-0.0302170952383606\\
2.37392335641798	-0.379827544265013\\
1.97946724983572	-0.755471413831684\\
1.51952588658696	-1.14109638579653\\
1.01136727122743	-1.51514517126524\\
0.482032224901594	-1.85544525334606\\
-0.0382838063201061	-2.14522947273086\\
-0.52410154924132	-2.37686303097922\\
-0.959488772986293	-2.55157186200705\\
-1.33839105970505	-2.67639513915615\\
-1.66205132699853	-2.76070307072847\\
-1.93580024564179	-2.81373776109643\\
-2.16654578772536	-2.84340740970685\\
-2.36125505045569	-2.85595956618366\\
-2.52621914730902	-2.8561017997265\\
-2.66679922833549	-2.84728154219588\\
-2.787420973919	-2.83197952277816\\
-2.89167569415667	-2.81196030141443\\
-2.98245314892876	-2.7884673805791\\
-3.06207131943148	-2.76236803217295\\
-3.13238964293222	-2.73425797963794\\
-3.19490246471729	-2.70453592010907\\
-3.25081385812493	-2.67345607439244\\
-3.30109649242935	-2.64116495686063\\
-3.34653745708976	-2.60772685290374\\
-3.38777365579463	-2.57314117569786\\
-3.42531892653787	-2.53735390476143\\
-3.45958457962228	-2.50026460922663\\
-3.49089463019454	-2.46173005540939\\
-3.51949664689044	-2.42156503154378\\
-3.54556883535683	-2.37954074729201\\
-3.56922370906653	-2.33538094838593\\
-3.59050845108607	-2.28875570263873\\
-3.60940181843321	-2.2392726437686\\
-3.6258071618823	-2.18646528955512\\
-3.63954080086355	-2.12977787016306\\
-3.65031457159793	-2.06854590523487\\
-3.65771081504082	-2.00197155739246\\
-3.66114733989249	-1.92909258526504\\
-3.65982893264023	-1.84874357670581\\
-3.65268075561092	-1.75950818944266\\
-3.63825750750274	-1.65966162877518\\
-3.61462074197262	-1.54710408232652\\
-3.57917593635714	-1.41928932570068\\
-3.52846249420908	-1.27316004705987\\
-3.45789767683273	-1.10511567818159\\
-3.36149714823051	-0.911064057090691\\
-3.23164370107576	-0.686649269870069\\
-3.0590694227229	-0.427801906604648\\
-2.83336215057245	-0.131797507300495\\
-2.54445295612327	0.201048934514697\\
-2.18548932838733	0.565232839046605\\
-1.756881214354	0.948298435946715\\
-1.26995300623213	1.33101269062124\\
-0.747435352721523	1.6907508679559\\
-0.218964618192301	2.00735905571867\\
0.286748605912422	2.26841802747823\\
0.748686704214283	2.47098556427373\\
1.15605475094327	2.61965277121247\\
1.50685469940157	2.72301328980851\\
1.80474754969108	2.7905970960028\\
2.05610271647609	2.83106393347599\\
2.26799126297232	2.85149584067307\\
2.4470975329268	2.85734050088096\\
};
\addplot [color=mycolor2, forget plot]
  table[row sep=crcr]{%
1.13400451818224	0.976808589286248\\
1.55591156704315	0.963967632646999\\
1.87844575220844	0.933648692297892\\
2.11936805604402	0.895471980855908\\
2.29861648452528	0.855264674972638\\
2.43291852387781	0.816033640430765\\
2.53480545668349	0.77909964354237\\
2.61323550840971	0.744886767314028\\
2.67451202995887	0.713381203705023\\
2.72306878728849	0.684375563701702\\
2.76204775962919	0.657592066074585\\
2.79370039348197	0.632741553186792\\
2.81966033819558	0.609549681145767\\
2.84112800312704	0.587766765388304\\
2.85899595219203	0.567169748121874\\
2.87393480527273	0.547560569004644\\
2.88645267285243	0.528763065252748\\
2.89693669235387	0.510619424756789\\
2.90568230974701	0.4929866544925\\
2.91291404181462	0.475733242668805\\
2.91880020475392	0.45873605083536\\
2.92346326744079	0.441877401398409\\
2.9269869300111	0.425042290128983\\
2.9294206417286	0.408115632972522\\
2.93078199156806	0.390979441224458\\
2.93105718626686	0.373509802692034\\
2.93019964118205	0.355573524530472\\
2.92812652187264	0.337024262310264\\
2.92471286181861	0.317697915544286\\
2.91978261199468	0.297407007473874\\
2.91309560767229	0.275933680108139\\
2.90432890298117	0.253020816484529\\
2.89305012649987	0.2283606419443\\
2.87867929825578	0.201579947253414\\
2.86043367930716	0.172220819462771\\
2.83724732590173	0.139715489795419\\
2.80765253239014	0.103353715016184\\
2.76960351029923	0.0622412937740589\\
2.72021268256465	0.0152496400797256\\
2.65535696720453	-0.0390393611120289\\
2.56909954942	-0.102377421932274\\
2.45287988771856	-0.176838945573836\\
2.2945089602085	-0.264649236867091\\
2.07732101696771	-0.367628681657688\\
1.78067588590787	-0.48587920660265\\
1.38454964462808	-0.615400748165417\\
0.881673507817823	-0.74534824221461\\
0.294918328976953	-0.858145393319198\\
-0.317126387879956	-0.936341921800144\\
-0.883847296346128	-0.972771788836004\\
-1.35810775058867	-0.973322344140018\\
-1.72865102211688	-0.950297867995204\\
-2.00781965414307	-0.915097530776598\\
-2.21555633660112	-0.875382765679109\\
-2.37050823287822	-0.835416192301556\\
-2.48728119065286	-0.797238697197828\\
-2.57650663409757	-0.761647304777937\\
-2.64570478272576	-0.728805858007367\\
-2.70015917313327	-0.698582512499448\\
-2.74359762199151	-0.6707242684433\\
-2.77867579716957	-0.64494284878883\\
-2.80730847419449	-0.620954518293437\\
-2.83089386719239	-0.598496660609891\\
-2.85046562760151	-0.57733295164336\\
-2.86679650790334	-0.557253161273629\\
-2.88046972309956	-0.538070608957517\\
-2.89192857699452	-0.519618756232101\\
-2.90151130391482	-0.501747631914064\\
-2.9094757140015	-0.484320386239284\\
-2.91601668742886	-0.467210069019795\\
-2.92127854785938	-0.450296626550061\\
-2.92536366807315	-0.433464061900805\\
-2.92833819837386	-0.416597676898128\\
-2.93023548133117	-0.399581297302019\\
-2.93105747144582	-0.382294367470811\\
-2.93077427789105	-0.364608782085508\\
-2.9293217633479	-0.346385296441374\\
-2.9265969355594	-0.3274693196282\\
-2.92245063167818	-0.307685842231661\\
-2.91667668191573	-0.286833176453542\\
-2.90899629612566	-0.264675084706324\\
-2.89903576598859	-0.240930734263808\\
-2.8862945941022	-0.215261731664662\\
-2.87009965741047	-0.187255256459951\\
-2.84953868532166	-0.156402040528551\\
-2.82336272243743	-0.122067683553979\\
-2.78984168857616	-0.0834557348325103\\
-2.74654882533352	-0.0395615824687034\\
-2.69003818636869	0.0108813368567756\\
-2.61536585856166	0.0694543294176981\\
-2.51539944447989	0.138078327170258\\
-2.37989458894041	0.218941624283059\\
-2.19448862572723	0.314173655445411\\
-1.94029548841821	0.424963981921096\\
-1.59600269760179	0.549723930217708\\
-1.14591807084824	0.681304947299176\\
-0.595828970644992	0.805112369640538\\
0.012534643673176	0.902349173444445\\
0.609958617098545	0.959665970740262\\
1.13400451818224	0.976808589286248\\
};
\addplot [color=mycolor1, forget plot]
  table[row sep=crcr]{%
1.90545647109768	2.84473020387397\\
2.06560614281006	2.83975180880271\\
2.20696629698615	2.82634185272677\\
2.3321915951789	2.80637841785645\\
2.44358977384658	2.7812802081649\\
2.54313495484306	2.75210344657922\\
2.63249879955004	2.71962180533369\\
2.71308776287006	2.68438966472592\\
2.78608015084079	2.64679082867193\\
2.85245993353386	2.60707528779914\\
2.91304609217289	2.56538646417426\\
2.96851724174839	2.52178098738162\\
3.01943172859427	2.47624262550686\\
3.06624357723105	2.42869160436426\\
3.10931468151508	2.37899021861514\\
3.14892357416526	2.32694537000167\\
3.18527100428818	2.27230845243878\\
3.21848242258461	2.21477283089051\\
3.24860732457302	2.15396902195747\\
3.27561523244261	2.08945757331174\\
3.29938790103738	2.02071955571186\\
3.31970710623613	1.94714453143744\\
3.33623710895942	1.86801586331425\\
3.34850058536363	1.78249331255182\\
3.35584648861182	1.68959310181112\\
3.35740800761322	1.58816609549383\\
3.3520486262168	1.47687564143641\\
3.33829449887801	1.35417819222828\\
3.3142524026042	1.2183124702892\\
3.27751522297935	1.06730717305853\\
3.22506265000043	0.899023567903594\\
3.15317550326068	0.71125798313182\\
3.05740019708648	0.501939051865713\\
2.9326263925163	0.269461325311117\\
2.77337167138756	0.0131902830470721\\
2.57438512912176	-0.265864250884355\\
2.33164968949332	-0.564299363001459\\
2.04372935410427	-0.875730608718146\\
1.71315341533049	-1.19073305175328\\
1.34725514404524	-1.49766199545704\\
0.957840581159799	-1.7843954061448\\
0.559476469129818	-2.04051966377913\\
0.166909292354912	-2.2591216772568\\
-0.20738281673389	-2.43751067242486\\
-0.554658638939878	-2.57677777831348\\
-0.870174254687224	-2.68063492454041\\
-1.15256142971836	-2.75411665693453\\
-1.40283481017664	-2.80254022426169\\
-1.62342322812576	-2.83085278857385\\
-1.81741597084171	-2.84331816002108\\
-1.98805916965749	-2.84343322803351\\
-2.13846090256022	-2.83397138670082\\
-2.27144346885171	-2.81708109928046\\
-2.38948818663578	-2.79439767208476\\
-2.49473284788786	-2.76714748807323\\
-2.58899587024454	-2.73623658309891\\
-2.67381156646907	-2.70232204470638\\
-2.75046784293595	-2.66586768325229\\
-2.82004187602089	-2.62718644856711\\
-2.88343176937095	-2.58647215117961\\
-2.94138353379713	-2.54382274343107\\
-2.99451340673076	-2.49925699574864\\
-3.04332582379704	-2.45272598968585\\
-3.0882274409002	-2.40412048827574\\
-3.12953757859625	-2.35327494589072\\
-3.16749537455486	-2.29996867912253\\
-3.20226381135857	-2.24392452729567\\
-3.23393064701117	-2.18480517625679\\
-3.26250611642261	-2.12220719462947\\
-3.28791709050908	-2.05565273435872\\
-3.30999716922801	-1.98457877940319\\
-3.32847193922572	-1.90832379869763\\
-3.34293834225436	-1.82611169612514\\
-3.35283678321713	-1.73703309658577\\
-3.35741428508422	-1.64002434273836\\
-3.35567674689123	-1.53384523447966\\
-3.34632834584585	-1.41705773592997\\
-3.32769667673192	-1.28800992744541\\
-3.29764396001492	-1.1448328491852\\
-3.25346864714091	-0.985463117106764\\
-3.19180969536969	-0.807711723896657\\
-3.10857992988814	-0.609408970823442\\
-2.99897728294295	-0.388664619236554\\
-2.85765247301142	-0.144284203070108\\
-2.6791393917441	0.123637165840428\\
-2.4586534840674	0.412984096921417\\
-2.19328667710712	0.718912984736189\\
-1.88343065048257	1.03347769843591\\
-1.53397275642332	1.34597848929993\\
-1.15461344762183	1.64426910082164\\
-0.758827839321173	1.91681242572131\\
-0.361611913312704	2.15477791315486\\
0.0231636257632512	2.35335165025311\\
0.384783706451366	2.51184807879158\\
0.716540098658869	2.63283491704342\\
1.01548918488673	2.7208374714998\\
1.28158359749607	2.78113776271335\\
1.51665497280379	2.81892605629222\\
1.72354052603863	2.83882994316678\\
1.90545647109768	2.84473020387397\\
};
\addplot [color=mycolor2, forget plot]
  table[row sep=crcr]{%
1.46855410137581	1.12707832297397\\
1.81746172861383	1.11648449734565\\
2.0832734524898	1.09148900817058\\
2.28420219676519	1.05963121359182\\
2.43667502329634	1.02541141897586\\
2.55355487106084	0.991253931079652\\
2.64431809237863	0.958339784161561\\
2.71577816800775	0.927157533362092\\
2.77280526976218	0.897829051554277\\
2.81889337956571	0.8702918262286\\
2.85657061328148	0.844397665828629\\
2.88768591502819	0.819964710924385\\
2.91360694503597	0.796803898388846\\
2.93535655415903	0.774731620325453\\
2.95370737147097	0.7535750041178\\
2.96924786678498	0.733173305273046\\
2.98242888329376	0.71337729747401\\
2.99359666860525	0.694047664479373\\
3.00301644870154	0.675052915197177\\
3.01088926630408	0.656267075791459\\
3.01736391707627	0.637567262531493\\
3.02254521232267	0.618831152111924\\
3.02649937460473	0.59993431215182\\
3.02925706552155	0.580747316169379\\
3.03081430616027	0.561132533969967\\
3.03113134592148	0.540940452919435\\
3.03012933592254	0.520005342248276\\
3.02768444097797	0.49814001572594\\
3.02361874677748	0.475129371377977\\
3.01768694324708	0.450722282586946\\
3.00955722904638	0.424621273347811\\
2.99878409207805	0.3964692207245\\
2.98476943299641	0.365832079889668\\
2.96670669226752	0.332176320531016\\
2.9434998840396	0.294839426073746\\
2.91364525751147	0.252991547337343\\
2.87505710623026	0.205586530339498\\
2.82481056570843	0.151301851642755\\
2.7587637319781	0.0884714203679338\\
2.6710140094552	0.015027152988835\\
2.55315732117441	-0.0715059378832148\\
2.39340687238096	-0.173853501639095\\
2.1759204274396	-0.294449287699812\\
1.8814106856495	-0.434129080814116\\
1.49134503928733	-0.589718938010167\\
0.998483596223561	-0.75106799166535\\
0.421858296720502	-0.900377107469046\\
-0.187265258270534	-1.01781985727513\\
-0.763463752921409	-1.09171592863421\\
-1.25816076574939	-1.12367349964627\\
-1.65447551481276	-1.12419096032058\\
-1.9596185164593	-1.10523445409195\\
-2.19074418828945	-1.07607638440074\\
-2.36560531342012	-1.04262754170864\\
-2.49890021152678	-1.00822970036258\\
-2.60172186414097	-0.974599376616324\\
-2.68211856458525	-0.942518623253538\\
-2.74585071655244	-0.91226261074259\\
-2.79703992585854	-0.883844558714942\\
-2.83865455548789	-0.857150338876948\\
-2.87285359067348	-0.832010381280273\\
-2.90122497915828	-0.808236977352645\\
-2.92494991100248	-0.785642778172455\\
-2.94491633958941	-0.764049184780701\\
-2.96179794748695	-0.743289371013107\\
-2.97610953260883	-0.723208506594891\\
-2.98824617924996	-0.703662559393909\\
-2.99851115012501	-0.684516404077932\\
-3.00713581596548	-0.66564160531801\\
-3.01429385668226	-0.646914043782802\\
-3.02011123698435	-0.62821143953346\\
-3.02467295600509	-0.60941075971465\\
-3.02802721267342	-0.59038545286311\\
-3.03018736051096	-0.571002417318382\\
-3.03113180758053	-0.551118577579125\\
-3.03080181858543	-0.530576903738905\\
-3.02909696891869	-0.509201659800999\\
-3.02586775522894	-0.486792600809011\\
-3.02090454690351	-0.463117749314465\\
-3.01392161641998	-0.43790426005694\\
-3.00453433777658	-0.41082671747549\\
-2.99222667527461	-0.381491992925512\\
-2.97630462145942	-0.349419510354504\\
-2.95582901110058	-0.314015440440706\\
-2.92951773672268	-0.274539022594774\\
-2.89560227628824	-0.23005908867203\\
-2.85161602955057	-0.179399434070764\\
-2.79408213382729	-0.121074198907511\\
-2.71805834339307	-0.0532218198776744\\
-2.61649598903557	0.026435103293064\\
-2.47941066892704	0.120533485531534\\
-2.29302953539182	0.231758171538483\\
-2.0395571501463	0.361972183384229\\
-1.69921188392751	0.510376156685171\\
-1.25735371882417	0.670635146156023\\
-0.718111284055277	0.828535287891957\\
-0.117225357787478	0.964113108564206\\
0.483218576030838	1.06040872070266\\
1.02278616318477	1.11239456361227\\
1.4685541013758	1.12707832297397\\
};
\addplot [color=mycolor1, forget plot]
  table[row sep=crcr]{%
1.5511141993765	2.96816970333686\\
1.71675358438934	2.96299691921139\\
1.86682539418795	2.94874046419969\\
2.00300869880171	2.92701307354291\\
2.1268612119055	2.89909435579508\\
2.23979214975311	2.86598173523314\\
2.34305412892067	2.82843762111414\\
2.43774626769991	2.78703031878428\\
2.52482323250256	2.74216802431532\\
2.6051068522611	2.69412618643373\\
2.67929821669458	2.64306893140237\\
2.74798902104091	2.58906536319621\\
2.81167144936764	2.53210151744128\\
2.87074620122304	2.47208864713597\\
2.92552843417927	2.40886839739986\\
2.97625146671294	2.34221530971119\\
3.02306809386465	2.27183699588217\\
3.06604933256557	2.19737224570959\\
3.10518034740554	2.11838728616227\\
3.14035322017697	2.03437040262416\\
3.17135612668665	1.94472517759859\\
3.19785838475057	1.84876272096299\\
3.21939075997375	1.74569349161575\\
3.23532040059592	1.63461969228791\\
3.24481988950191	1.51452982616827\\
3.24683026945662	1.38429792520045\\
3.24001870889995	1.24269129770411\\
3.22273302150037	1.08839248387585\\
3.19295793562874	0.920043451259204\\
3.14828230299302	0.736322673246078\\
3.08589270849307	0.536067880169691\\
3.00261702269694	0.318457344540068\\
2.8950497572107	0.0832576972436309\\
2.75979544183791	-0.168867570731735\\
2.59385881672152	-0.436021360781201\\
2.39518107659023	-0.714779306849007\\
2.16326296217152	-1.0000507108857\\
1.89973724771263	-1.2852211083055\\
1.60869142726366	-1.56266105395898\\
1.29655212742886	-1.82457122668823\\
0.971461435453503	-2.06398451866564\\
0.64226687823649	-2.27564747825153\\
0.317405648969006	-2.45653164756393\\
0.00399029116635881	-2.60587315260145\\
-0.292710800285464	-2.72481554129322\\
-0.569374827774472	-2.81583834360932\\
-0.824413429264136	-2.88215937495653\\
-1.05759927921737	-2.92723658466493\\
-1.2696590252546	-2.954419170429\\
-1.46191201754143	-2.96674238900522\\
-1.63598822105908	-2.96683397124905\\
-1.79362814199161	-2.95689494329898\\
-1.93655247395617	-2.93872345847997\\
-2.06638458222437	-2.91375925764997\\
-2.18460991573799	-2.88313463759206\\
-2.29255955355281	-2.84772397858109\\
-2.39140850588118	-2.80818795523389\\
-2.48218231230374	-2.76501099685318\\
-2.56576770136256	-2.7185318953432\\
-2.64292464567118	-2.66896809950429\\
-2.71429819888329	-2.61643447582328\\
-2.78042917454843	-2.56095734362065\\
-2.84176313728666	-2.50248451756251\\
-2.89865740951947	-2.44089197581842\\
-2.95138591191257	-2.37598765124789\\
-3.00014169247095	-2.30751273337605\\
-3.04503698345508	-2.2351407798849\\
-3.08610057314396	-2.15847487429311\\
-3.12327220164251	-2.0770430384803\\
-3.1563935950482	-1.99029212536997\\
-3.18519565057231	-1.89758049553535\\
-3.20928119314775	-1.79816994866221\\
-3.2281026712299	-1.69121767735085\\
-3.2409341989296	-1.5757694952191\\
-3.24683757518568	-1.45075634298661\\
-3.24462247064948	-1.31499719288961\\
-3.23280210809785	-1.16721305337147\\
-3.20954781988667	-1.00605887830981\\
-3.17264929489326	-0.83018271505137\\
-3.11949258355707	-0.638323940630935\\
-3.04707518900768	-0.429463817566223\\
-2.95208610020282	-0.203039647718152\\
-2.83108570188368	0.0407750039789764\\
-2.68082016772555	0.300741426695085\\
-2.49868774249454	0.574228932152976\\
-2.28333051547243	0.856982607028298\\
-2.03525425987047	1.14311012993192\\
-1.75730260272994	1.42540246962225\\
-1.45477926837949	1.69602394718267\\
-1.13507590084788	1.94746673630532\\
-0.806827848963815	2.17352897098816\\
-0.478810766230246	2.37003509442909\\
-0.158891672379683	2.53511231340366\\
0.146701500614249	2.66901330900869\\
0.433683425640174	2.7736265607983\\
0.699638558134577	2.85187124026548\\
0.943710310761243	2.90713854170722\\
1.166197624607	2.94286609588378\\
1.36816404860046	2.96226391390951\\
1.5511141993765	2.96816970333686\\
};
\addplot [color=mycolor2, forget plot]
  table[row sep=crcr]{%
1.69631441633007	1.28768207007386\\
1.99032662854603	1.27874738696554\\
2.2166701408415	1.2574447719624\\
2.39088211468516	1.22980388087852\\
2.52592030888927	1.19948059308081\\
2.6317285425257	1.16854555802824\\
2.71566106839321	1.13809809930942\\
2.78308250493548	1.10866994684214\\
2.83790007966455	1.08047124194359\\
2.88297368997499	1.0535348197561\\
2.92041396595002	1.02779904998385\\
2.95179366805963	1.00315466733232\\
2.97829637996918	0.979470777011036\\
3.00082110987031	0.956608880037873\\
3.02005620276326	0.934430016464237\\
3.03653189746237	0.912797952928076\\
3.0506579371935	0.891580089193716\\
3.06275061316424	0.870647035321154\\
3.07305223115758	0.849871390220485\\
3.0817450407718	0.829126003760703\\
3.0889610107669	0.808281853879328\\
3.09478837182346	0.787205574173226\\
3.09927551078203	0.765756600727119\\
3.10243253949991	0.743783853697696\\
3.10423064069486	0.721121818891428\\
3.10459908279546	0.697585839253715\\
3.10341956749125	0.672966358648663\\
3.10051729544469	0.647021773042724\\
3.09564776637187	0.619469428583003\\
3.08847781162818	0.589974151989877\\
3.07855860551277	0.558133495200779\\
3.06528728841541	0.523458614364247\\
3.04785217094134	0.485349384665519\\
3.02515399881607	0.44306201098606\\
2.99569207555459	0.395667149125735\\
2.95739875970142	0.341996730740528\\
2.90739882293268	0.280579102362855\\
2.84166243842304	0.209566654956785\\
2.75451725805986	0.126672070571172\\
2.63800260305594	0.0291573292462813\\
2.48113177255788	-0.0860210416188368\\
2.26937740142539	-0.22169921476775\\
1.98527052826225	-0.379277231781463\\
1.61192350068489	-0.55644225831266\\
1.14158841186434	-0.744225259582789\\
0.588057722722855	-0.925696180260258\\
-0.00675744185391754	-1.0800113283142\\
-0.584623958529031	-1.19167406956994\\
-1.09642539040285	-1.25745740352105\\
-1.51896416693009	-1.2848102027044\\
-1.8528008268099	-1.28525153808844\\
-2.11094151451397	-1.26919996062677\\
-2.30941040545633	-1.24414223451508\\
-2.4626089024133	-1.21481927141176\\
-2.58196490385806	-1.18400372538223\\
-2.67605415977655	-1.15321784723463\\
-2.75116272174692	-1.12323800969317\\
-2.81186746537437	-1.09441190091988\\
-2.86150824186208	-1.06684756239571\\
-2.90253908805109	-1.04052294570266\\
-2.93677969678039	-1.01534831669648\\
-2.96559277957666	-0.991201247053671\\
-2.9900087356399	-0.967945805144239\\
-3.01081351667094	-0.945442666432577\\
-3.02861090182525	-0.923554007263144\\
-3.04386692502897	-0.902145396894908\\
-3.05694175382007	-0.881085951917333\\
-3.06811263841632	-0.860247466272762\\
-3.07759040122891	-0.839502907414268\\
-3.08553114872227	-0.818724476404097\\
-3.09204433866097	-0.797781310465758\\
-3.09719794322897	-0.776536827601692\\
-3.10102115455803	-0.754845654470515\\
-3.10350484234504	-0.732550028067738\\
-3.10459976120126	-0.709475509870231\\
-3.10421228968941	-0.685425790570829\\
-3.10219723414302	-0.660176287121552\\
-3.09834691259617	-0.633466133528818\\
-3.09237529912219	-0.604988033344248\\
-3.08389538674237	-0.574375264372061\\
-3.07238701384697	-0.541184894215979\\
-3.05715103904924	-0.504875973594996\\
-3.03724371305329	-0.464781136540198\\
-3.01138206111722	-0.420069719438527\\
-2.97780665437066	-0.369700415319409\\
-2.93408197189791	-0.312362127070219\\
-2.87680689778822	-0.246404329456668\\
-2.80120128949956	-0.16976576083321\\
-2.7005384078293	-0.079928782593947\\
-2.56543493883457	0.0260318956802744\\
-2.3831591312352	0.15115536363008\\
-2.13750906636878	0.297774100114308\\
-1.8105897569183	0.465783528152239\\
-1.38866466012649	0.649891413810413\\
-0.87323150796197	0.837055758569039\\
-0.292272852787305	1.00745919807065\\
0.30138542793791	1.14168765120112\\
0.850896019481423	1.2300133132159\\
1.31925086546401	1.27523515772891\\
1.69631441633007	1.28768207007386\\
};
\addplot [color=mycolor1, forget plot]
  table[row sep=crcr]{%
1.30964269727806	3.1637962565165\\
1.4796617479317	3.15846843373279\\
1.63679061427442	3.14352532559141\\
1.78209449164426	3.12032833944566\\
1.91661469139635	3.08999203623042\\
2.04133587734847	3.05341084022803\\
2.15716704436191	3.01128640334179\\
2.26493187823667	2.96415318583326\\
2.36536518025283	2.91240103174314\\
2.45911293062631	2.85629427339921\\
2.54673426377869	2.79598733673527\\
2.62870414512695	2.73153704934776\\
2.70541590354361	2.66291195707374\\
2.77718301859967	2.58999898766238\\
2.84423971673922	2.51260779861417\\
2.90674001989998	2.4304731350626\\
2.96475493309473	2.34325552009698\\
3.01826746994878	2.25054061827051\\
3.06716521160309	2.1518376669181\\
3.11123009040846	2.04657747494767\\
3.15012510558999	1.93411066423918\\
3.18337774235946	1.81370709861271\\
3.21036002162971	1.68455783764647\\
3.23026541778701	1.54578149644089\\
3.24208343692729	1.39643760860752\\
3.24457356752618	1.23555047198094\\
3.23624174246731	1.06214793587486\\
3.21532452017037	0.875320475251067\\
3.17978895210455	0.674306296938357\\
3.12735936579723	0.458607445326076\\
3.05558538675182	0.228138886152453\\
2.96196699821337	-0.0165939040116348\\
2.84414987595434	-0.274304198788235\\
2.70019456494705	-0.542750437351112\\
2.52890392387698	-0.818625887665045\\
2.33016567717077	-1.09756710070662\\
2.10523835090943	-1.37432977635652\\
1.85689379064363	-1.64314957924953\\
1.58934378464163	-1.89825275383867\\
1.3079278520424	-2.13442513615714\\
1.01860977322162	-2.34751617881986\\
0.727390781441582	-2.53476738101936\\
0.439767930045209	-2.69490895567937\\
0.160339199500988	-2.82803826492504\\
-0.107399608192614	-2.93534672986096\\
-0.361078183550312	-3.0187811774478\\
-0.599343551116539	-3.08071390197956\\
-0.82167924151625	-3.12366813311058\\
-1.02820229942608	-3.15011728731852\\
-1.21947223478175	-3.16235593533549\\
-1.39632974112117	-3.16242964145298\\
-1.55977082001897	-3.15210762402562\\
-1.71085457163499	-3.1328834234634\\
-1.85063940386854	-3.10599186652311\\
-1.98014142579909	-3.07243399096331\\
-2.10030916868881	-3.03300448554837\\
-2.21200973246586	-2.98831837870147\\
-2.31602253244997	-2.93883521303982\\
-2.41303779837131	-2.88487991020164\\
-2.50365777336167	-2.82666011113997\\
-2.58839916531416	-2.76428009887243\\
-2.66769583949962	-2.69775156936529\\
-2.7419010420427	-2.62700157896064\\
-2.81128864048502	-2.5518780086988\\
-2.87605298716733	-2.47215287701608\\
-2.93630707519116	-2.38752382306851\\
-2.99207868262204	-2.29761408868848\\
-3.04330420322828	-2.20197136130826\\
-3.08981985611297	-2.10006591749712\\
-3.13134996953366	-1.99128864432507\\
-3.16749206989299	-1.87494973512688\\
-3.19769861039845	-1.7502791836144\\
-3.22125539743398	-1.61643066493288\\
-3.23725719218883	-1.47249102074309\\
-3.24458168556316	-1.31749836824648\\
-3.24186419916032	-1.15047279834493\\
-3.22747719906101	-0.970464595351572\\
-3.19952112736246	-0.77662561958305\\
-3.15583611401251	-0.568309413072392\\
-3.09404743511042	-0.345203854359003\\
-3.01166010965617	-0.107495624874144\\
-2.90621784361654	0.143942848398775\\
-2.77553580150244	0.407366412040965\\
-2.61800243985426	0.680008613328076\\
-2.432921803668	0.958023744307871\\
-2.22083819760022	1.23657425973828\\
-1.98376120673494	1.51009984708879\\
-1.72520699853086	1.77276127972635\\
-1.45000416550279	2.01899454967052\\
-1.16387551442911	2.24406316113598\\
-0.872876987209391	2.44448553178769\\
-0.582818051995086	2.61825016010458\\
-0.298783922300724	2.76479746668511\\
-0.0248348027380357	2.88481207518776\\
0.236102926428582	2.97990634957964\\
0.482187030483545	3.05227836659459\\
0.712507782295484	3.10440573556148\\
0.926890690971778	3.13880717167437\\
1.12569630509008	3.15787875025747\\
1.30964269727806	3.1637962565165\\
};
\addplot [color=mycolor2, forget plot]
  table[row sep=crcr]{%
1.87066321805417	1.45402756039734\\
2.12177093449327	1.44638275984957\\
2.317761693533	1.42791955610156\\
2.47135347756998	1.40353440201243\\
2.59275799155613	1.37625932425357\\
2.6897624708477	1.34788753229756\\
2.76817229032354	1.319434934294\\
2.83228382782909	1.29144458778211\\
2.88528007826317	1.26417714528765\\
2.92953233347676	1.23772686605507\\
2.96682099594705	1.21209130233767\\
2.99849438692525	1.1872127710152\\
3.025582317517	1.16300278176544\\
3.04887736169328	1.13935617858333\\
3.06899323549112	1.11615904872294\\
3.08640693070161	1.09329282421922\\
3.10148924371942	1.07063602288273\\
3.11452692052714	1.04806448560686\\
3.12573864511145	1.02545060684543\\
3.13528640133144	1.00266183085243\\
3.14328324184147	0.97955854127504\\
3.14979813366696	0.955991371688167\\
3.15485826910997	0.931797888189099\\
3.15844899574587	0.90679852796303\\
3.16051129998121	0.88079160930123\\
3.16093654653781	0.853547150047979\\
3.15955790023812	0.8247991342263\\
3.15613749773549	0.794235741298259\\
3.15034794215717	0.761486888763146\\
3.14174598657415	0.726108225906463\\
3.12973524113668	0.687560446781515\\
3.1135132220467	0.64518246836988\\
3.09199583371357	0.598156684667283\\
3.06370915644131	0.545464286952159\\
3.0266339377191	0.485828872524742\\
2.97798250492584	0.417648064197987\\
2.91388218623718	0.338917487750234\\
2.82893862932173	0.247163201907554\\
2.71567109994163	0.139425480376497\\
2.56388651868539	0.0123921444109677\\
2.36026722654807	-0.137122249369144\\
2.08890423110962	-0.311030939692938\\
1.73420883848114	-0.507843467702896\\
1.28787824980463	-0.719793581163217\\
0.759259933385833	-0.931073702832379\\
0.182177988575539	-1.12053266150978\\
-0.392171517511089	-1.26977533751479\\
-0.915575840684575	-1.37106932892531\\
-1.36004790937376	-1.42827021685356\\
-1.7199956170801	-1.45158832747825\\
-2.00398372858132	-1.45195535619048\\
-2.22580640137757	-1.43814554765677\\
-2.39915840499094	-1.41624205918801\\
-2.53554018619301	-1.39012326080988\\
-2.64390675165217	-1.36213304062674\\
-2.73099292835528	-1.33362899055591\\
-2.80179428767293	-1.30536071197632\\
-2.86000707011612	-1.27771170209651\\
-2.90837620006094	-1.2508483373094\\
-2.94895423287702	-1.22480990474502\\
-2.9832888558787	-1.19956242009131\\
-3.01255721582719	-1.17503050616074\\
-3.03766196113174	-1.15111603355987\\
-3.05930008373523	-1.12770876023663\\
-3.07801248582485	-1.10469210625135\\
-3.09421983150112	-1.08194593790484\\
-3.10824855146559	-1.05934747633484\\
-3.12034968056339	-1.03677098569893\\
-3.13071237649265	-1.01408661289765\\
-3.13947338120028	-0.991158571549518\\
-3.14672326345681	-0.967842743744772\\
-3.15250996389637	-0.9439836870027\\
-3.15683990998776	-0.919410963503353\\
-3.15967674486952	-0.893934642015034\\
-3.16093749189277	-0.867339750538167\\
-3.16048572690583	-0.839379370929042\\
-3.15812101853101	-0.809765956741145\\
-3.1535634778409	-0.77816031235226\\
-3.14643166941203	-0.744157484395111\\
-3.1362112833808	-0.707268575723716\\
-3.12221071816805	-0.666897194254896\\
-3.10349788402036	-0.622308911711488\\
-3.07880985193452	-0.572591804392801\\
-3.04642315041378	-0.516606090574266\\
-3.00396738607712	-0.452921596262271\\
-2.94815890970766	-0.37974451149767\\
-2.87442707678991	-0.294842389227502\\
-2.77641185225325	-0.195494294437964\\
-2.64535151344877	-0.0785320608051692\\
-2.46950778725791	0.0593858575997357\\
-2.23409419163594	0.221008435394047\\
-1.92277618699022	0.406878897255613\\
-1.52245210617972	0.612727365398179\\
-1.03231897555632	0.826789077549283\\
-0.473780599181381	1.02986037347607\\
0.108686675359043	1.20096726029463\\
0.662567557702798	1.32640137592083\\
1.14847974851045	1.40461533226047\\
1.55026489179016	1.44344976208378\\
1.87066321805417	1.45402756039734\\
};
\addplot [color=mycolor1, forget plot]
  table[row sep=crcr]{%
1.14198747225695	3.38832402736194\\
1.31605992222682	3.38285578958292\\
1.47926499672352	3.3673224594284\\
1.63231216590224	3.34287789641818\\
1.77592179996291	3.31048127019128\\
1.91079626334644	3.27091234371535\\
2.03760030921143	3.22478834037688\\
2.15694827157711	3.17258054332779\\
2.26939599959159	3.11462954399408\\
2.37543590802034	3.05115858110924\\
2.47549388692006	2.98228475333172\\
2.56992710848601	2.9080281024882\\
2.65902199295158	2.82831869522321\\
2.74299175804852	2.74300191243686\\
2.82197309033541	2.65184221413475\\
2.89602155468672	2.55452570213947\\
2.96510541366527	2.45066187094384\\
3.02909757560089	2.33978503259681\\
3.08776544561469	2.22135603930962\\
3.14075853869006	2.09476512195938\\
3.18759385630825	1.95933692768926\\
3.22763926585268	1.81433918576067\\
3.26009550419467	1.65899685866019\\
3.28397801391087	1.4925141257177\\
3.29810067759518	1.31410704126207\\
3.30106469525668	1.12305008611112\\
3.29125735525567	0.918739870152637\\
3.26686717093224	0.7007785980871\\
3.22592347601145	0.469078103296655\\
3.16636945965635	0.223981748356033\\
3.08617676437358	-0.0336041103607149\\
2.98350587120135	-0.302084954015504\\
2.85690839958484	-0.579083389716836\\
2.70555499087658	-0.861405365663716\\
2.52945757552058	-1.14509799818222\\
2.32964231475143	-1.42561972089645\\
2.10822613845917	-1.69812186340872\\
1.86836133044444	-1.957811514879\\
1.61403947829286	-2.20033847192607\\
1.34978085508433	-2.42213618717118\\
1.08026481735875	-2.62065529577265\\
0.809969103738292	-2.79445610767395\\
0.542877020464996	-2.94316206126457\\
0.282287201830255	-3.06730577664165\\
0.0307321921675561	-3.16811417651256\\
-0.210010359277831	-3.24727833304284\\
-0.438841755820347	-3.30674243380652\\
-0.655230912051859	-3.34853143793361\\
-0.85909190141726	-3.37462385965159\\
-1.05066706316558	-3.38686711003996\\
-1.23042500857762	-3.38692810863447\\
-1.39897720956761	-3.37627042460475\\
-1.55701316096008	-3.35614973643604\\
-1.70525208644774	-3.32762084659481\\
-1.84440832162054	-3.29155115765214\\
-1.97516739762051	-3.24863703808408\\
-2.09817011972688	-3.1994207324695\\
-2.21400235954319	-3.14430638303466\\
-2.32318872587667	-3.08357436830736\\
-2.42618868176004	-3.01739359026803\\
-2.52339400739865	-2.94583161297398\\
-2.61512676734915	-2.86886272311823\\
-2.70163713203072	-2.78637408546898\\
-2.78310054027091	-2.69817023325833\\
-2.85961378384395	-2.60397618789948\\
-2.93118965997596	-2.50343956182795\\
-2.99774988716902	-2.39613207818106\\
-3.05911602864777	-2.28155105592556\\
-3.11499823485627	-2.15912157337876\\
-3.16498172601717	-2.02820025121288\\
-3.20851112040985	-1.88808190062256\\
-3.24487301706631	-1.73801067006497\\
-3.27317771812207	-1.57719778735318\\
-3.29234169013923	-1.4048484969593\\
-3.30107337735163	-1.22020124874507\\
-3.29786632705124	-1.02258243707351\\
-3.28100522506141	-0.811479737863473\\
-3.2485921674592	-0.586635931031317\\
-3.19860185149968	-0.348162504129153\\
-3.12897453324035	-0.0966678067712452\\
-3.03775338926328	0.166612124818355\\
-2.92326702955321	0.43970012338347\\
-2.78434754524817	0.7198096240344\\
-2.62056039573655	1.00335319673104\\
-2.43240798435225	1.28605503841691\\
-2.22145995280244	1.56317868851008\\
-1.99036681856816	1.82985490562083\\
-1.74273320138674	2.08146498692662\\
-1.48285909721788	2.31401346585324\\
-1.2153915502866	2.52442163704885\\
-0.944951100570337	2.7106925546306\\
-0.675798932141263	2.8719316217413\\
-0.411592829482475	3.00824106676635\\
-0.155252198347039	3.12052944746041\\
0.0910740234581681	3.21028396949869\\
0.325960018222231	3.27934651147373\\
0.548605990898327	3.32972039217949\\
0.758718693172089	3.36342045582833\\
0.956390154658264	3.38236784241906\\
1.14198747225695	3.38832402736193\\
};
\addplot [color=mycolor2, forget plot]
  table[row sep=crcr]{%
2.01273186082157	1.62264188057517\\
2.22882063757858	1.61604978981954\\
2.39982292437263	1.5999264589979\\
2.53605024869709	1.57828560186922\\
2.64560455641858	1.55366230710075\\
2.73465289221244	1.52760891831352\\
2.80782868963228	1.50104856263404\\
2.86860364562702	1.47450913150948\\
2.91958676862407	1.44827262140604\\
2.96275059777069	1.42246876899394\\
2.99959758866234	1.39713323996872\\
3.0312810461154	1.37224358546903\\
3.05869273021849	1.34774130683866\\
3.08252639381533	1.32354521601587\\
3.10332400638802	1.2995592980853\\
3.12150948490404	1.27567705258614\\
3.13741333335669	1.25178352695749\\
3.15129057492953	1.22775577834639\\
3.16333363422859	1.20346219676488\\
3.17368130395596	1.17876092409924\\
3.18242454469853	1.15349746678444\\
3.18960956917868	1.12750149774532\\
3.19523841599993	1.10058275728146\\
3.19926699199864	1.0725258797971\\
3.20160032852801	1.04308388296492\\
3.20208452520021	1.01196994813214\\
3.20049450803311	0.978846985268764\\
3.19651625838312	0.943314301531696\\
3.18972150520314	0.9048904686354\\
3.17953191683756	0.862991203193945\\
3.16516843942157	0.816900741376229\\
3.14557941789191	0.765734845680667\\
3.11933827732454	0.708393357162769\\
3.08449765619282	0.64350043767493\\
3.03838211135335	0.569332132271484\\
2.97729712019097	0.483735409182865\\
2.89613254086358	0.384054181754214\\
2.78785645755072	0.267103344500391\\
2.64296078709071	0.129283819862817\\
2.44909508235137	-0.0329769279253369\\
2.19149803292758	-0.222156640087914\\
1.85540284097846	-0.437620556075269\\
1.43183220207253	-0.672781570954555\\
0.926529254475946	-0.91293714136645\\
0.366905965032531	-1.13685603092568\\
-0.201788153687179	-1.3237955378641\\
-0.732802230074641	-1.46194690177102\\
-1.19478592849004	-1.55144546914079\\
-1.57703266179045	-1.60067155688414\\
-1.88399309881306	-1.62055881991123\\
-2.12714244300369	-1.62086201768036\\
-2.3192602556512	-1.60888726103068\\
-2.47172727407397	-1.5896091371629\\
-2.59373537863074	-1.56623151723462\\
-2.69237111564201	-1.54074516975253\\
-2.77298453280887	-1.51435193789272\\
-2.83958597030085	-1.48775415463792\\
-2.89518319062102	-1.46134215062032\\
-2.94204278026816	-1.43531268450704\\
-2.98188459701558	-1.40974281616709\\
-3.01602373734729	-1.38463567926112\\
-3.04547347217946	-1.35994867978886\\
-3.07101983077187	-1.33561070585615\\
-3.09327577122147	-1.31153242886764\\
-3.11272065473482	-1.28761221449429\\
-3.12972907723809	-1.26373919332839\\
-3.14459190832569	-1.23979443853937\\
-3.15753152786689	-1.21565081849316\\
-3.16871263534396	-1.19117184813658\\
-3.1782495596713	-1.16620969956304\\
-3.18621066100641	-1.14060241545111\\
-3.19262014838458	-1.11417027688287\\
-3.19745740447247	-1.0867111941397\\
-3.20065368243093	-1.05799490378562\\
-3.20208579114637	-1.02775565754519\\
-3.20156608111895	-0.995682968167048\\
-3.19882764192543	-0.961409824006352\\
-3.19350306524589	-0.924497586282473\\
-3.1850943320687	-0.884416530974658\\
-3.17293023047192	-0.840520688268477\\
-3.15610603691663	-0.792015285376631\\
-3.13339778795653	-0.737914789304898\\
-3.10314010978443	-0.676989486867996\\
-3.0630521846915	-0.607699252841271\\
-3.00999156462662	-0.528115846321772\\
-2.93961266425308	-0.435842359447583\\
-2.8459134889737	-0.327955664745927\\
-2.72069030724748	-0.201034638583436\\
-2.55303012238377	-0.0514075447247277\\
-2.32923428621965	0.124137739729845\\
-2.03405456577611	0.326840235647845\\
-1.65465028027992	0.55345949443634\\
-1.18821921518954	0.793467136619876\\
-0.650950165933849	1.02834457263954\\
-0.0805694209150354	1.23597001051807\\
0.474456620282551	1.399221940702\\
0.973544504074251	1.51237661528221\\
1.39585274626457	1.58041146158093\\
1.73928748626154	1.61362058532941\\
2.01273186082157	1.62264188057517\\
};
\addplot [color=mycolor1, forget plot]
  table[row sep=crcr]{%
1.02611651957107	3.60772225884434\\
1.20403531250379	3.60212375025056\\
1.37252272406886	3.58607871275405\\
1.53209147433507	3.5605840611805\\
1.68327442490991	3.52647097361666\\
1.82660014241576	3.48441505698273\\
1.96257504267002	3.43494800021525\\
2.09167048293983	3.37846934578765\\
2.21431340202062	3.31525753296105\\
2.3308793479758	3.24547974201835\\
2.44168695081149	3.16920033169254\\
2.54699307979453	3.08638784461037\\
2.64698807085481	2.99692068599959\\
2.74179052285167	2.9005916824586\\
2.83144125016096	2.79711181942329\\
2.91589605313904	2.68611355417884\\
2.99501703994363	2.56715421982348\\
3.06856231834773	2.4397201868288\\
3.13617399414738	2.30323264312533\\
3.19736458859026	2.15705609778928\\
3.25150225242221	2.00051100716612\\
3.29779554621413	1.83289225205801\\
3.33527911587741	1.65349552358913\\
3.36280235276438	1.4616539294472\\
3.37902409905939	1.2567871829688\\
3.38241759486048	1.03846538671013\\
3.3712910142648	0.806488398293082\\
3.34382979549314	0.560979753679769\\
3.29816702173673	0.302490850827981\\
3.23248664958135	0.0321065160951329\\
3.14516065270606	-0.248462376543206\\
3.03491467713805	-0.536818690225567\\
2.90100793301208	-0.82987341863068\\
2.74340347468625	-1.12391471873249\\
2.56289793654726	-1.41476251714717\\
2.36117911643895	-1.6980036617741\\
2.14078838582012	-1.96928306110288\\
1.90498241059215	-2.22460940016504\\
1.65751055078008	-2.46062665077033\\
1.4023432160411	-2.67480853992983\\
1.14339564859588	-2.86555083124844\\
0.884288382218877	-3.03215927806954\\
0.628172427659431	-3.1747514530428\\
0.377629863914206	-3.29410266310233\\
0.134644947708159	-3.39146838897052\\
-0.0993692244464525	-3.46841037969752\\
-0.323507312636681	-3.52664453446868\\
-0.537289374400536	-3.56791950464855\\
-0.740573585424378	-3.59392766774999\\
-0.933473511872671	-3.60624546586133\\
-1.11628526502292	-3.60629781860137\\
-1.28942670714867	-3.59534075499257\\
-1.45338877731054	-3.5744568633044\\
-1.60869780134301	-3.54455908355186\\
-1.75588709590617	-3.50639940520567\\
-1.89547603925685	-3.46057998697967\\
-2.02795487972215	-3.40756500385319\\
-2.15377376160685	-3.34769213374202\\
-2.2733346897906	-3.28118304140237\\
-2.38698538549539	-3.20815253229395\\
-2.49501418637923	-3.12861626808435\\
-2.5976453079511	-3.04249708878699\\
-2.69503391204524	-2.94963009988326\\
-2.78726052797591	-2.84976677729476\\
-2.87432445198552	-2.74257843623574\\
-2.95613582189578	-2.62765951672197\\
-3.0325061402555	-2.50453127197162\\
-3.10313711807964	-2.37264661756504\\
-3.16760785464923	-2.23139711792507\\
-3.22536058499219	-2.08012335631199\\
-3.27568555028962	-1.91813024917339\\
-3.31770601716222	-1.74470920163683\\
-3.35036512836541	-1.55916930470951\\
-3.37241713470611	-1.36087994730133\\
-3.38242662245726	-1.14932709569401\\
-3.37878052110083	-0.924184843670482\\
-3.35971873119469	-0.685402357949884\\
-3.3233897443763	-0.43330372642119\\
-3.26793702021244	-0.168694272966524\\
-3.19161937754449	0.107038210307673\\
-3.09296357987329	0.391844357044387\\
-2.97093950589242	0.682977531095793\\
-2.82513876334528	0.977023428823052\\
-2.65592880016096	1.27001138997337\\
-2.46455021158412	1.55761110463077\\
-2.25312869633061	1.83540002987176\\
-2.02458646705947	2.09916774465816\\
-1.78245842128309	2.34521061155878\\
-1.53063974954406	2.57056923249898\\
-1.2731063612472	2.77317355736937\\
-1.01365249973888	2.95188184958125\\
-0.755681123047297	3.10642231749431\\
-0.502066530535087	3.2372628547284\\
-0.255091604879366	3.34544136045897\\
-0.0164489460183681	3.43238713735545\\
0.212711718933483	3.49975621592722\\
0.431708626818227	3.54929399415121\\
0.640241689089839	3.58273017202159\\
0.838306582515661	3.60170497518489\\
1.02611651957107	3.60772225884434\\
};
\addplot [color=mycolor2, forget plot]
  table[row sep=crcr]{%
2.13066155564074	1.78934952646239\\
2.31729935703494	1.78364460719312\\
2.46689620124448	1.76952844019794\\
2.58781105164417	1.75031033791726\\
2.68652254186893	1.72811583414305\\
2.76796031070307	1.70428226501363\\
2.83585227040171	1.67963405118568\\
2.89301957020744	1.65466520604145\\
2.94160681957125	1.6296575131894\\
2.98325487595012	1.6047562069412\\
3.01922820624035	1.58001819006828\\
3.05050810394904	1.55544264470556\\
3.07786083388366	1.53099034984978\\
3.10188752944395	1.50659570797689\\
3.12306081329798	1.4821740068146\\
3.14175170109824	1.45762550489228\\
3.15824931183566	1.43283733227823\\
3.17277515750889	1.4076838126804\\
3.18549323891532	1.38202555853736\\
3.19651677012285	1.35570751532808\\
3.20591204311225	1.32855600187292\\
3.21369968872289	1.30037468766754\\
3.21985335966386	1.27093935036681\\
3.22429562795569	1.23999115375056\\
3.22689062345455	1.20722806743009\\
3.22743260712858	1.17229390269361\\
3.22562922655149	1.13476425208792\\
3.22107757786229	1.09412838138774\\
3.21323030795922	1.04976582242581\\
3.20134770676748	1.00091605681835\\
3.18442989421847	0.946639300923022\\
3.16112060377793	0.885766123130951\\
3.12957055354315	0.816833764968077\\
3.08724411929405	0.73800835618114\\
3.03064910810781	0.64699640659248\\
2.95496974575675	0.540959600987511\\
2.85359820694491	0.416470990893481\\
2.71761568164062	0.269599915374646\\
2.53542181057784	0.0963008308050649\\
2.29302638863571	-0.106599668422552\\
1.97600821968529	-0.339474794327379\\
1.5744278205967	-0.597031272628531\\
1.09079846615224	-0.865714438789585\\
0.547271679730867	-1.12426677793354\\
-0.0156987973804935	-1.34975858100649\\
-0.552609514395057	-1.5264343025763\\
-1.02938785202724	-1.65058730346741\\
-1.43097456897238	-1.72843647162033\\
-1.75816612094634	-1.77058643378522\\
-2.02028675676074	-1.78756478900962\\
-2.22921263119039	-1.78781501291504\\
-2.39615822324344	-1.77739781919804\\
-2.53049340461053	-1.76040181520039\\
-2.63960205598709	-1.73948683427999\\
-2.72914325726933	-1.71634290147778\\
-2.80340494859105	-1.69202300667229\\
-2.86562888501474	-1.66716817309768\\
-2.9182730255947	-1.64215460857281\\
-2.96321163045467	-1.6171883613051\\
-3.00188381624089	-1.5923657554321\\
-3.0354025628448	-1.56771182882603\\
-3.06463440778332	-1.54320467631437\\
-3.09025774070329	-1.51879072927095\\
-3.11280553904297	-1.49439415253789\\
-3.13269675775084	-1.46992236302528\\
-3.15025937346884	-1.44526892684105\\
-3.16574720105731	-1.42031461319411\\
-3.17935196092714	-1.39492707139712\\
-3.19121160749489	-1.3689593874057\\
-3.20141557668629	-1.34224762723294\\
-3.21000733136552	-1.3146073592814\\
-3.21698434428276	-1.28582904753639\\
-3.2222954297344	-1.25567210867758\\
-3.22583508947928	-1.22385731664626\\
-3.22743424402313	-1.19005710671384\\
-3.22684633746756	-1.15388316582293\\
-3.22372727881572	-1.11487048454915\\
-3.21760693908985	-1.07245677716519\\
-3.20784885476026	-1.02595584522598\\
-3.19359324689341	-0.974523082798222\\
-3.17367626606242	-0.91711096815095\\
-3.14651532584494	-0.852412257800675\\
-3.10994643917659	-0.778789190780974\\
-3.06099512366542	-0.694189465974422\\
-2.99555986643644	-0.596056555476247\\
-2.90799288715264	-0.48125811924112\\
-2.79059380934311	-0.346091178417871\\
-2.63312516162728	-0.186489932883004\\
-2.42268081912405	0.00133018142641184\\
-2.14465351030925	0.219451212471439\\
-1.78602806334448	0.465803890576448\\
-1.34199072149221	0.731173209142785\\
-0.824317574584908	0.997756503334357\\
-0.265278147565503	1.24239048713434\\
0.28996029303649	1.44471801699729\\
0.799904615435649	1.59485812742332\\
1.23981820678715	1.69467540015643\\
1.60340305424167	1.75327938401769\\
1.8966275701755	1.78163647045096\\
2.13066155564074	1.78934952646239\\
};
\addplot [color=mycolor1]
  table[row sep=crcr]{%
0.948683298050514	3.79473319220206\\
1.12996891013208	3.78902241071535\\
1.3027838103604	3.77255915923418\\
1.46753552344581	3.74623053454947\\
1.62465224482371	3.71077288416293\\
1.77456152903207	3.66677970030574\\
1.91767403561802	3.61471073626391\\
2.05437110063153	3.55490128334521\\
2.18499505421831	3.4875709448019\\
2.30984136685525	3.41283153872394\\
2.42915185734506	3.33069397942336\\
2.54310832592638	3.24107414976038\\
2.65182608463458	3.14379790634541\\
2.75534694820212	3.03860547381077\\
2.8536313296431	2.92515559896292\\
2.9465491655584	2.80302996392973\\
3.0338694906698	2.67173851056894\\
3.11524860652315	2.53072651479627\\
3.19021696761039	2.3793844730251\\
3.25816516606007	2.21706211963439\\
3.31832976399147	2.0430881668871\\
3.36978023105031	1.85679760784185\\
3.41140891541445	1.65756857671014\\
3.44192680709927	1.44487070344488\\
3.45986879009691	1.21832646083672\\
3.46361298678702	0.977785965288651\\
3.45141941180798	0.723413822791541\\
3.4214930614047	0.455783738589644\\
3.37207522993717	0.175972765268488\\
3.30156372141992	-0.114357323644244\\
3.20865740947072	-0.412907981729085\\
3.0925136091427	-0.716739496073036\\
2.95289924526556	-1.02233055985344\\
2.79031115637498	-1.3257107709346\\
2.60603988731933	-1.6226624623078\\
2.40215712665582	-1.90897292279738\\
2.18141953804948	-2.18070442957845\\
1.94709823579504	-2.43444260737754\\
1.70275852491129	-2.66748659951055\\
1.45202370644265	-2.877956852001\\
1.19835703486066	-3.0648138241024\\
0.944888096380122	-3.22779787311511\\
0.694297500839398	-3.36731218002033\\
0.448761194715288	-3.48427490957016\\
0.209946190700892	-3.57996468520442\\
-0.0209555786883461	-3.65587741213202\\
-0.243169602253506	-3.71360522906444\\
-0.456269901983927	-3.7547419037936\\
-0.660108423391693	-3.78081428495334\\
-0.854748940791145	-3.79323662458319\\
-1.04040889805107	-3.79328335955738\\
-1.21741050351997	-3.78207578896275\\
-1.38614100991934	-3.76057854694004\\
-1.54702130890241	-3.72960250007616\\
-1.70048159286015	-3.6898114747465\\
-1.84694273429099	-3.64173092446581\\
-1.98680208954437	-3.58575723170233\\
-2.1204225680783	-3.52216679568137\\
-2.24812396922559	-3.45112440165093\\
-2.37017574651009	-3.37269062063008\\
-2.48679050050333	-3.2868281760587\\
-2.59811762050387	-3.19340735751618\\
-2.70423659456836	-3.09221068144532\\
-2.80514959232286	-2.98293711145668\\
-2.90077300437504	-2.86520627071399\\
-2.99092770791458	-2.73856321838213\\
-3.07532793559953	-2.60248453106578\\
-3.1535687736533	-2.45638663503084\\
-3.22511252958097	-2.29963757630341\\
-3.28927451889835	-2.13157368366421\\
-3.34520925490937	-1.95152284702656\\
-3.39189861348128	-1.7588363466938\\
-3.42814429814894	-1.5529312348605\\
-3.45256782598978	-1.33334504716437\\
-3.46362220116165	-1.0998039137315\\
-3.45962024522194	-0.852303713618117\\
-3.43878487541956	-0.591201557879526\\
-3.3993259814164	-0.317311504170883\\
-3.3395463824176	-0.0319942042762583\\
-3.25797518618734	0.262774139257094\\
-3.15352067565559	0.564371272065384\\
-3.02562738470585	0.869554764908886\\
-2.87441511977517	1.17455760801215\\
-2.70077402394131	1.47525676358082\\
-2.50639202484484	1.767403513355\\
-2.29370038309167	2.04688929885519\\
-2.06573813670639	2.31000991273335\\
-1.8259528645897	2.5536887433386\\
-1.57796799985389	2.77562773795799\\
-1.32535178123783	2.9743703475303\\
-1.07141885643386	3.14927865398809\\
-0.819084909373598	3.30044159903781\\
-0.570781691446472	3.42853925898355\\
-0.328428492520302	3.53468897189284\\
-0.093448740680559	3.62029468577783\\
0.13318251344032	3.68691393725669\\
0.350873927242831	3.73614982594868\\
0.55934713845833	3.76956970364108\\
0.758567824726602	3.78864856747597\\
0.948683298050513	3.79473319220206\\
};
\addlegendentry{Внешние аппроксимации}

\addplot [color=mycolor2, forget plot]
  table[row sep=crcr]{%
2.22721472973942	1.94935886896179\\
2.3886648425121	1.94441487511101\\
2.5195634483927	1.93205457572182\\
2.62670779849127	1.91501766628131\\
2.71531853364	1.89508786650117\\
2.78936864949369	1.87341097537522\\
2.85187814440773	1.85071228241933\\
2.90515032165589	1.82744075521499\\
2.9509520177711	1.80386332654994\\
2.99064783647547	1.78012618777659\\
3.02529909977035	1.7562945351756\\
3.0557365623221	1.73237827445755\\
3.08261387510984	1.7083485310185\\
3.10644696879361	1.68414807630787\\
3.12764310033082	1.65969765855822\\
3.14652223914943	1.63489950162782\\
3.16333268711427	1.60963876367151\\
3.17826225469757	1.5837834336332\\
3.19144589188063	1.55718292750329\\
3.20297034763997	1.52966548843736\\
3.2128761683912	1.50103436869378\\
3.2211571124149	1.47106265802835\\
3.22775682617807	1.43948650786522\\
3.23256237104755	1.40599637010381\\
3.23539387180292	1.3702257107414\\
3.23598913837185	1.33173645766061\\
3.23398152983158	1.29000018434938\\
3.22886850103885	1.24437370419161\\
3.21996707976175	1.19406735000614\\
3.20635080827997	1.13810376656035\\
3.18676025898309	1.07526465092859\\
3.15947593702972	1.0040228169715\\
3.12213828317303	0.922457920675385\\
3.07149545967596	0.828157721349833\\
3.00305879148097	0.718116213117772\\
2.91065699973037	0.588662002756959\\
2.785925244978	0.435496623074383\\
2.61788823849866	0.254008232649376\\
2.39306798253518	0.0401523940333445\\
2.09699730361163	-0.20771900183087\\
1.7183819635995	-0.485932899271695\\
1.25639683089422	-0.782386182967809\\
0.728319451607428	-1.07597695395358\\
0.17048677629427	-1.34156752261761\\
-0.372344302750219	-1.55919115860543\\
-0.863264272549092	-1.72086595365401\\
-1.28299919742863	-1.8302347441912\\
-1.62890152595063	-1.89731652670458\\
-1.90833376158261	-1.93331843320936\\
-2.13240654410545	-1.94782687646357\\
-2.3122537087605	-1.94803363323748\\
-2.45746747168844	-1.93896359662991\\
-2.57574556593285	-1.92399109460735\\
-2.67305603610964	-1.90533078490532\\
-2.75395564612754	-1.88441460443591\\
-2.82190758462804	-1.86215611801615\\
-2.87954766726586	-1.83912801979308\\
-2.92889157233296	-1.81567886095776\\
-2.97149068204595	-1.79200910352774\\
-3.00854741762708	-1.76822049003994\\
-3.04100007190525	-1.74434802267285\\
-3.0695851473103	-1.7203805908067\\
-3.09488323143276	-1.69627413213334\\
-3.11735281839452	-1.67195981527391\\
-3.13735524534064	-1.64734883071173\\
-3.15517299859553	-1.62233479306363\\
-3.17102297575804	-1.59679437416259\\
-3.18506579905015	-1.570586527561\\
-3.19741190629232	-1.54355048208521\\
-3.20812485602917	-1.51550254281508\\
-3.21722203851139	-1.48623162017347\\
-3.22467275544594	-1.45549329519237\\
-3.23039339092638	-1.42300210774249\\
-3.23423911335776	-1.38842161161915\\
-3.23599118590349	-1.35135156242766\\
-3.23533847055249	-1.31131137681683\\
-3.23185101774822	-1.2677187105326\\
-3.22494264006069	-1.21986163848029\\
-3.2138179373885	-1.16686249014231\\
-3.19739719631547	-1.10763095340298\\
-3.17420973909784	-1.0408037811618\\
-3.142242561071	-0.964668760862181\\
-3.09872682936675	-0.877072578179989\\
-3.0398417586358	-0.77531812311728\\
-2.96031892266926	-0.656071422287709\\
-2.85295425309884	-0.515330753928937\\
-2.70811161177175	-0.348574764250833\\
-2.51348980394932	-0.151314840167112\\
-2.25479111797749	0.0795960527623154\\
-1.91840618384291	0.343563906305597\\
-1.49721915275415	0.633014609293368\\
-0.998726638487032	0.931115827109969\\
-0.450328181715587	1.21375185121395\\
0.105465468340197	1.45718721810745\\
0.625945116714116	1.64701594620063\\
1.08251666630864	1.78154116777267\\
1.46486936946524	1.86834353319716\\
1.77626331377619	1.91854897142989\\
2.02656829443923	1.94275375393752\\
2.22721472973941	1.94935886896179\\
};
\addplot [color=mycolor2]
  table[row sep=crcr]{%
2.12132034355964	2.82842712474619\\
2.18582575000323	2.82640660874952\\
2.2459761483502	2.82067996094826\\
2.30316600770785	2.81153894842004\\
2.35833530419819	2.79908319427389\\
2.4121407651283	2.78328504568152\\
2.46505313939319	2.76402339892432\\
2.51741350921551	2.74110157112695\\
2.56946568278775	2.71425682483196\\
2.62137377470791	2.68316571665979\\
2.67322998558975	2.64744786912504\\
2.72505542741334	2.60667010622203\\
2.77679571750944	2.56035271147659\\
2.82831256449113	2.50797963523731\\
2.87937250576692	2.44901464186941\\
2.92963424895255	2.38292549191834\\
2.97863667364875	2.30921809960124\\
3.02579039162557	2.22748193169334\\
3.0703766721051	2.13744642550811\\
3.11155818595542	2.03904567139316\\
3.14840590189045	1.93248505458784\\
3.1799449834646	1.81829951655162\\
3.20521925970046	1.69738980361013\\
3.2233689191928	1.57102231442397\\
3.23371058284854	1.44078163681543\\
3.23580480244767	1.30847305653245\\
3.22949541157146	1.1759834831\\
3.21490900216393	1.04511946165368\\
3.19240995273442	0.917445713323391\\
3.16251366135309	0.794144173493542\\
3.12576385298257	0.675902054710668\\
3.08257536776515	0.562820516214273\\
3.03302813706062	0.454314775018543\\
2.97656426645917	0.348948817128694\\
2.91146921429886	0.244099783305823\\
2.83385680006128	0.135243682019644\\
2.73548220960228	0.0144130439331602\\
2.59871334341581	-0.133168683665202\\
2.38479638365891	-0.336180712113444\\
2.01057175466146	-0.648371525543042\\
1.34210041814469	-1.13791067535239\\
0.375352321166521	-1.75822148716338\\
-0.527448468220416	-2.26262801632552\\
-1.11365847778129	-2.5437869612038\\
-1.4567327767317	-2.68228003389134\\
-1.6691296256022	-2.75259678373195\\
-1.81428800439327	-2.79054335642282\\
-1.92320462744895	-2.81168897715777\\
-2.01132392805712	-2.82302365363059\\
-2.08680248126988	-2.82787464184343\\
-2.15422943553058	-2.82790871121207\\
-2.21634478307291	-2.8239827357917\\
-2.27487422572476	-2.8165264847195\\
-2.33095693947289	-2.80572465930185\\
-2.3853755403757	-2.79160756168836\\
-2.43868469976231	-2.77409782070174\\
-2.49128507004909	-2.75303495626884\\
-2.54346612600745	-2.7281884019911\\
-2.59543032955127	-2.69926454755406\\
-2.64730535029083	-2.66591103001887\\
-2.69914810041563	-2.62772046423944\\
-2.75094277189846	-2.58423541950896\\
-2.80259428647749	-2.53495641420315\\
-2.85391830194335	-2.47935483422118\\
-2.90462903903618	-2.41689283661657\\
-2.95432664759112	-2.34705230341292\\
-3.00248656709011	-2.26937453210355\\
-3.04845424185856	-2.18351130413452\\
-3.0914493816545	-2.08928598065547\\
-3.13058429431929	-1.98676020448289\\
-3.16490008683964	-1.87629788656397\\
-3.19342217916266	-1.75861429449935\\
-3.21523240499265	-1.63479579561046\\
-3.22954956680171	-1.50627700312196\\
-3.23580523135335	-1.37476797309193\\
-3.23369896357047	-1.2421341494958\\
-3.22321880671614	-1.11024302689198\\
-3.20461860896113	-0.980799507713277\\
-3.17835148049538	-0.85519274756288\\
-3.14496439092874	-0.734369602871579\\
-3.10495870881183	-0.618735254776386\\
-3.05861192744495	-0.508062722468019\\
-3.0057323937692	-0.401369642978784\\
-2.94527025004781	-0.296685462718974\\
-2.87460204254734	-0.190562899149911\\
-2.78805571026974	-0.0770301658026245\\
-2.67361421522119	0.0546866392302208\\
-2.50520482475694	0.225105040795738\\
-2.22541867988909	0.474087715816021\\
-1.7209533613293	0.868461990547897\\
-0.880186083660502	1.44500069250654\\
0.110323684998138	2.03883159526995\\
0.859259206598599	2.42737283484743\\
1.30741523004191	2.62511520505649\\
1.57426953542748	2.72304263504847\\
1.74766615441039	2.77435113512036\\
1.87207484559155	2.80265686669549\\
1.96924085114733	2.81832059141065\\
2.0502990590364	2.82612977760851\\
2.12132034355964	2.82842712474619\\
};
\addlegendentry{Внутренние аппроксимации}

\end{axis}

\begin{axis}[%
width=0.798\linewidth,
height=0.578\linewidth,
at={(-0.104\linewidth,-0.064\linewidth)},
scale only axis,
xmin=0,
xmax=1,
ymin=0,
ymax=1,
axis line style={draw=none},
ticks=none,
axis x line*=bottom,
axis y line*=left,
legend style={legend cell align=left, align=left, draw=white!15!black}
]
\end{axis}
\end{tikzpicture}%
        \caption{Эллипсоидальные аппроксимации для 50 направлений.}
\end{figure}
%%%%%%%%%%%%%%%%%%%%%%%%%%%%%%%%%%%%%%%%%%%%%%%%%%%%%%%%%%%%%%%%%%%%%%%%%%%%%%%%


%% Document end


%%%%%%%%%%%%%%%%%%%%%%%%%%%%%%%%%%%%%%%%%%%%%%%%%%%%%%%%%%%%%%%%%%%%%%%%%%%%%%%%
\clearpage
\begin{thebibliography}{9}
% Библиография
\bibitem{kurzh} Kurzhanski A. B., Varaiya P. \textit{Dynamics and Control of Trajectory Tubes.} Birkhauser, 2014.
\end{thebibliography}
\end{document}
\tableofcontents
\clearpage
%%%%%%%%%%%%%%%%%%%%%%%%%%%%%%%%%%%%%%%%%%%%%%%%%%%%%%%%%%%%%%%%%%%%%%%%%%%%%%%%        


%%  Document start


%%%%%%%%%%%%%%%%%%%%%%%%%%%%%%%%%%%%%%%%%%%%%%%%%%%%%%%%%%%%%%%%%%%%%%%%%%%%%%%%

\section{Об эллипсоидах и сумме Минковского}

\begin{definition}
        Назовём \textit{эллипсоидом} множество
$$
        \mathcal{E}(q,\,Q) = \{
x \in \setR^n \,:\, \langle x-q,\,Q^{-1}(x-q) \rangle \leqslant 1
        \},
        \qquad \mbox{где } Q= Q\T > 0.
$$
\end{definition}

\begin{assertion}
        Опорная функция и опорный вектор эллипсоида имеют вид:
$$
        \rho(l\,|\,\Varepsilon(q,\,Q)) =
        \langle l,\,q \rangle + 
        \langle l,\,Ql \rangle^{\nicefrac{1}{2}},
$$
$$
        x(l) = q + \frac{Ql}{\langle l,\,Ql\rangle^{\nicefrac12}}.
$$
\end{assertion}
\begin{proof}

Будем доказывать для случая $q = 0$. Иначе --- аналогично.

Так как по определению
$
        \rho(l\,|\,A) = \sup_{x\in A}\langle l,\,x\rangle,
$
то мы должны решать задачу максимизации скалярного произведения
$
        \langle l,\,x \rangle
$
при условии, что
$
        \langle x,\,Q^{-1}x \rangle = 1.
$
Запишем функцию Лагранжа для этой задачи:
$$
        \mathcal{L}(l,\,x,\,\lambda)
        =
        \langle l,\,x \rangle
        +
        \lambda(\langle x,\,Q^{-1}x \rangle - 1).
$$
Тогда
$$
        \frac{\partial \mathcal{L}}{\partial x}
        =
        l + 2\lambda Q^{-1} x 
        = 0
        \quad
        \Longrightarrow
        \quad
        x(l) = -\frac{1}{2\lambda}Ql.
$$
Подставим получившееся выражение для опорного вектора в условие:
$$
        \left\langle
-\frac{1}{2\lambda}Ql
,\,
-\frac{1}{2\lambda}Q^{-1}Ql
        \right\rangle
        = 1
        \;\Longrightarrow\;
        \lambda
        =
        -\frac12 \langle l,\,Ql \rangle^{\nicefrac12}
        \,\Longrightarrow\,
        x(l) = \frac{Ql}{\langle l,\,Ql\rangle^{\nicefrac12}}.
$$
В таком случае опорная функция в направлении $l \neq 0$ равна
$$
        \rho(l\,|\,\Varepsilon(0,\,Q))
        =
        \left\langle
l
,\,
\frac{Ql}{\langle l,\,Ql\rangle^{\nicefrac12}}
        \right\rangle
        =
        \langle l,\,Ql \rangle^{\nicefrac{1}{2}}.
$$
\end{proof}

\begin{definition}
        \textit{Суммой Минковского} множеств $A$ и $B$ называется множество
$$
        A + B
        =
        \{\,
x = a + b \::\: a \in A,\,b \in B 
        \,\}.
$$
\end{definition}

\begin{assertion}
        Опорная функция суммы Минковского равна сумме опорных функций каждого из множеств, то есть
$$
        \rho\left(l\,\left|\,\sum_{i=1}^n A_i\right.\right) = \sum_{i=1}^n \rho\,(l\,|\,A_i).
$$
\end{assertion}
\clearpage
\begin{figure}[t]
        \centering
        % This file was created by matlab2tikz.
%
%The latest updates can be retrieved from
%  http://www.mathworks.com/matlabcentral/fileexchange/22022-matlab2tikz-matlab2tikz
%where you can also make suggestions and rate matlab2tikz.
%
\definecolor{mycolor1}{rgb}{0.00000,0.44700,0.74100}%
%
\begin{tikzpicture}

\begin{axis}[%
width=0.618\linewidth,
height=0.471\linewidth,
at={(0\linewidth,0\linewidth)},
scale only axis,
xmin=-2,
xmax=4,
xlabel style={font=\color{white!15!black}},
xlabel={$x_1$},
ymin=0.5,
ymax=3.5,
ylabel style={font=\color{white!15!black}},
ylabel={$x_2$},
axis background/.style={fill=white},
axis x line*=bottom,
axis y line*=left,
xmajorgrids,
ymajorgrids,
legend style={legend cell align=left, align=left, draw=white!15!black}
]
\addplot [color=mycolor1]
  table[row sep=crcr]{%
3.12132034355964	3.41421356237309\\
3.14093309737706	3.41361883293874\\
3.15610096472476	3.41218966892045\\
3.16813059909176	3.41027848839136\\
3.17787846987421	3.40808691034947\\
3.18592472730893	3.40573192203561\\
3.19267367435337	3.40328137559688\\
3.19841408644868	3.40077372446074\\
3.20335666517915	3.3982293272274\\
3.20765799352175	3.39565707404995\\
3.21143626106343	3.3930583315506\\
3.2147818220386	3.39042930139934\\
3.217764420063	3.38776240921808\\
3.22043820622192	3.38504707754847\\
3.22284525774146	3.38227008706717\\
3.2250180482594	3.37941564172179\\
3.22698115917429	3.37646519817436\\
3.22875241567428	3.37339708240557\\
3.23034355770318	3.37018588734886\\
3.23176050074785	3.3668016188298\\
3.23300319375062	3.36320852788002\\
3.23406503318587	3.35936353046923\\
3.23493173445265	3.35521406411806\\
3.2355794824484	3.35069515469921\\
3.23597206441911	3.34572534988779\\
3.23605649958035	3.34020099103578\\
3.23575636783317	3.33398799537805\\
3.23496150515886	3.32690982030174\\
3.23351178703941	3.31872942300207\\
3.23117098985666	3.3091215094024\\
3.22758343607311	3.29762858411142\\
3.22219964903172	3.28358902636704\\
3.21414389031876	3.26601495419557\\
3.20196759721768	3.24337600812861\\
3.18316711789886	3.21319839168308\\
3.15318699112585	3.17128288128172\\
3.10323675000489	3.11010066257133\\
3.01525609198129	3.01537616591489\\
2.85016840053085	2.85895352273754\\
2.52458200959445	2.58760093991863\\
1.90440137584336	2.13363920441521\\
0.985501334090361	1.54409660598879\\
0.130059671215813	1.06605503500581\\
-0.409288268560127	0.807214909116197\\
-0.70597634609565	0.687302123652136\\
-0.872419406680929	0.632086042700855\\
-0.97199649313969	0.605972330977308\\
-1.03563236326191	0.593558335040314\\
-1.07869104959228	0.587977176231719\\
-1.10924316645516	0.585982787778395\\
-1.13178875034531	0.58594871840975\\
-1.14897766153775	0.587018094070618\\
-1.1624464899118	0.588720827478583\\
-1.17324845072647	0.59079102809827\\
-1.18208575929697	0.593075264744444\\
-1.18944113039835	0.595484336841729\\
-1.19565527927026	0.597966914051153\\
-1.20097426544368	0.600494649340231\\
-1.20557934247511	0.603053545598106\\
-1.20960630798951	0.605638849748727\\
-1.21315835534862	0.608252001222227\\
-1.2163147887704	0.610898815342025\\
-1.21913703504296	0.613588435376933\\
-1.22167284234597	0.616332784416996\\
-1.22395923010075	0.619146362645509\\
-1.22602455119113	0.622046304964338\\
-1.22788989784863	0.625052658896365\\
-1.2295699951167	0.628188875042512\\
-1.23107366292106	0.631482529497297\\
-1.23240387740984	0.634966324917864\\
-1.23355741534321	0.638679449344891\\
-1.23452401346765	0.642669415179332\\
-1.23528490798758	0.646994563037611\\
-1.23581052306342	0.651727509156087\\
-1.23605692848604	0.656959961411399\\
-1.23596044514103	0.662809563576559\\
-1.23542936997175	0.669429813383256\\
-1.23433108371332	0.677024752794653\\
-1.23247153021665	0.685871266354724\\
-1.22956168278891	0.696353870420598\\
-1.22516102360426	0.709020692660253\\
-1.21857881304302	0.724676739372734\\
-1.2086944178179	0.744545507733446\\
-1.19361473557335	0.770561635854139\\
-1.16998584185419	0.805927250653545\\
-1.13152846224427	0.856225279395141\\
-1.06573985695198	0.931754081541382\\
-0.946173527310501	1.05251473853723\\
-0.715145074514351	1.25784691099334\\
-0.25915963680298	1.61407245297346\\
0.533529697525656	2.15749204989669\\
1.47643270260674	2.72280409766538\\
2.17825862936543	3.08704498818123\\
2.57979469508172	3.26437318172905\\
2.80048557324685	3.34548951140256\\
2.92812298873437	3.38335485119594\\
3.00711025420764	3.40139640666682\\
3.05911603477277	3.40983029938284\\
3.09519171166257	3.41334200179773\\
3.12132034355964	3.41421356237309\\
};

\end{axis}

\begin{axis}[%
width=0.798\linewidth,
height=0.578\linewidth,
at={(-0.104\linewidth,-0.064\linewidth)},
scale only axis,
xmin=0,
xmax=1,
ymin=0,
ymax=1,
axis line style={draw=none},
ticks=none,
axis x line*=bottom,
axis y line*=left,
legend style={legend cell align=left, align=left, draw=white!15!black}
]
\end{axis}
\end{tikzpicture}%
        \caption{Эллипсоид с центром $q = \protect\begin{bmatrix}1\\2\protect\end{bmatrix}$ и матрицей $Q = \protect\begin{bmatrix}5&3\\3&2\protect\end{bmatrix}.$}
\end{figure}
\begin{figure}[b]
        \centering
        % This file was created by matlab2tikz.
%
%The latest updates can be retrieved from
%  http://www.mathworks.com/matlabcentral/fileexchange/22022-matlab2tikz-matlab2tikz
%where you can also make suggestions and rate matlab2tikz.
%
\definecolor{mycolor1}{rgb}{0.00000,0.44700,0.74100}%
\definecolor{mycolor2}{rgb}{0.85000,0.32500,0.09800}%
%
\begin{tikzpicture}

\begin{axis}[%
width=0.618\linewidth,
height=0.487\linewidth,
at={(0\linewidth,0\linewidth)},
scale only axis,
xmin=-4,
xmax=4,
xlabel style={font=\color{white!15!black}},
xlabel={$x_1$},
ymin=-3,
ymax=3,
ylabel style={font=\color{white!15!black}},
ylabel={$x_2$},
axis background/.style={fill=white},
axis x line*=bottom,
axis y line*=left,
xmajorgrids,
ymajorgrids,
legend style={at={(0.03,0.97)}, anchor=north west, legend cell align=left, align=left, draw=white!15!black}
]
\addplot [color=mycolor1]
  table[row sep=crcr]{%
0	1.41421356237309\\
0.0448926526261715	1.41278777581078\\
0.0898751836254369	1.40849029202781\\
0.135035408616089	1.40126046002867\\
0.180456834323975	1.39099628392442\\
0.226216037819367	1.37755312364591\\
0.272379465039817	1.36074202332744\\
0.31899942276683	1.3403278466662\\
0.366109017608597	1.31602749760457\\
0.413715781186158	1.28750864260985\\
0.461793724526318	1.25438953757444\\
0.510273605374737	1.21624080482269\\
0.559031297446442	1.17259030225851\\
0.607874358269208	1.12293255768884\\
0.656527248025466	1.06674455480223\\
0.704616200693154	1.00350985019654\\
0.751655514474458	0.932752901426886\\
0.797037975951299	0.854084849287771\\
0.840033114401921	0.767260538159248\\
0.879797685207568	0.672244052563361\\
0.915402708139832	0.569276526707822\\
0.945879950278728	0.458935986082397\\
0.970287525247814	0.342175739492072\\
0.987789436744396	0.22032715972476\\
0.997738518429436	0.0950562869276414\\
0.999748302867315	-0.0317279345033261\\
0.993739043738288	-0.15800451227805\\
0.979947497005068	-0.281790358648065\\
0.958898165695011	-0.401283709678683\\
0.931342671496431	-0.514977335908855\\
0.898180416909459	-0.62172652940075\\
0.860375718733436	-0.720768510152771\\
0.818884246741863	-0.81170017917703\\
0.774596669241484	-0.894427190999916\\
0.728302096399996	-0.969098608377261\\
0.68066980893543	-1.03603919926208\\
0.632245459597382	-1.09568761863817\\
0.583457251434522	-1.14854484958009\\
0.534627983128064	-1.19513423485098\\
0.485989745067013	-1.23597246546167\\
0.437699042301328	-1.2715498797676\\
0.38985098707616	-1.30231809315216\\
0.342491860563771	-1.32868305133133\\
0.295629790778837	-1.35100187031999\\
0.249243569363955	-1.36958215754347\\
0.203289781078724	-1.3846828264328\\
0.157708488746422	-1.39651568740012\\
0.112427735812955	-1.40524731219809\\
0.0673671215351546	-1.41100082986231\\
0.0224406851852784	-1.41385742962182\\
-0.0224406851852783	-1.41385742962182\\
-0.0673671215351544	-1.41100082986231\\
-0.112427735812955	-1.40524731219809\\
-0.157708488746422	-1.39651568740012\\
-0.203289781078724	-1.3846828264328\\
-0.249243569363954	-1.36958215754347\\
-0.295629790778837	-1.35100187031999\\
-0.342491860563771	-1.32868305133133\\
-0.389850987076159	-1.30231809315216\\
-0.437699042301328	-1.2715498797676\\
-0.485989745067013	-1.23597246546167\\
-0.534627983128064	-1.19513423485098\\
-0.583457251434522	-1.14854484958009\\
-0.632245459597382	-1.09568761863817\\
-0.68066980893543	-1.03603919926208\\
-0.728302096399996	-0.969098608377261\\
-0.774596669241483	-0.894427190999916\\
-0.818884246741864	-0.811700179177029\\
-0.860375718733436	-0.720768510152772\\
-0.898180416909458	-0.621726529400751\\
-0.931342671496431	-0.514977335908856\\
-0.958898165695011	-0.401283709678683\\
-0.979947497005068	-0.281790358648066\\
-0.993739043738288	-0.15800451227805\\
-0.999748302867315	-0.0317279345033253\\
-0.997738518429436	0.0950562869276414\\
-0.987789436744396	0.22032715972476\\
-0.970287525247814	0.34217573949207\\
-0.945879950278728	0.458935986082396\\
-0.915402708139832	0.569276526707822\\
-0.879797685207568	0.672244052563361\\
-0.840033114401921	0.767260538159247\\
-0.797037975951299	0.85408484928777\\
-0.751655514474458	0.932752901426886\\
-0.704616200693154	1.00350985019654\\
-0.656527248025466	1.06674455480223\\
-0.607874358269208	1.12293255768884\\
-0.559031297446443	1.17259030225851\\
-0.510273605374738	1.21624080482269\\
-0.461793724526318	1.25438953757444\\
-0.413715781186158	1.28750864260985\\
-0.366109017608597	1.31602749760457\\
-0.318999422766831	1.3403278466662\\
-0.272379465039818	1.36074202332744\\
-0.226216037819367	1.37755312364591\\
-0.180456834323975	1.39099628392442\\
-0.135035408616089	1.40126046002867\\
-0.0898751836254374	1.40849029202781\\
-0.0448926526261721	1.41278777581078\\
-1.73191211247099e-16	1.41421356237309\\
};
\addlegendentry{Эллипсоиды}

\addplot [color=mycolor1, forget plot]
  table[row sep=crcr]{%
2.12132034355964	1.41421356237309\\
2.14093309737706	1.41361883293874\\
2.15610096472476	1.41218966892045\\
2.16813059909176	1.41027848839136\\
2.17787846987421	1.40808691034947\\
2.18592472730893	1.40573192203561\\
2.19267367435337	1.40328137559688\\
2.19841408644868	1.40077372446074\\
2.20335666517915	1.39822932722739\\
2.20765799352175	1.39565707404995\\
2.21143626106343	1.3930583315506\\
2.2147818220386	1.39042930139934\\
2.217764420063	1.38776240921808\\
2.22043820622192	1.38504707754847\\
2.22284525774146	1.38227008706717\\
2.2250180482594	1.37941564172179\\
2.22698115917429	1.37646519817436\\
2.22875241567428	1.37339708240557\\
2.23034355770318	1.37018588734886\\
2.23176050074785	1.36680161882979\\
2.23300319375062	1.36320852788002\\
2.23406503318587	1.35936353046923\\
2.23493173445265	1.35521406411806\\
2.2355794824484	1.35069515469921\\
2.23597206441911	1.34572534988779\\
2.23605649958035	1.34020099103578\\
2.23575636783317	1.33398799537805\\
2.23496150515886	1.32690982030174\\
2.23351178703941	1.31872942300207\\
2.23117098985666	1.3091215094024\\
2.22758343607311	1.29762858411142\\
2.22219964903172	1.28358902636704\\
2.21414389031876	1.26601495419557\\
2.20196759721768	1.24337600812861\\
2.18316711789886	1.21319839168308\\
2.15318699112585	1.17128288128172\\
2.10323675000489	1.11010066257133\\
2.01525609198129	1.01537616591488\\
1.85016840053085	0.858953522737536\\
1.52458200959445	0.587600939918628\\
0.904401375843364	0.133639204415214\\
-0.0144986659096387	-0.455903394011212\\
-0.869940328784187	-0.933944964994194\\
-1.40928826856013	-1.1927850908838\\
-1.70597634609565	-1.31269787634786\\
-1.87241940668093	-1.36791395729915\\
-1.97199649313969	-1.39402766902269\\
-2.03563236326191	-1.40644166495969\\
-2.07869104959228	-1.41202282376828\\
-2.10924316645516	-1.4140172122216\\
-2.13178875034531	-1.41405128159025\\
-2.14897766153775	-1.41298190592938\\
-2.1624464899118	-1.41127917252142\\
-2.17324845072647	-1.40920897190173\\
-2.18208575929697	-1.40692473525556\\
-2.18944113039835	-1.40451566315827\\
-2.19565527927026	-1.40203308594885\\
-2.20097426544368	-1.39950535065977\\
-2.20557934247511	-1.39694645440189\\
-2.20960630798951	-1.39436115025127\\
-2.21315835534862	-1.39174799877777\\
-2.2163147887704	-1.38910118465798\\
-2.21913703504296	-1.38641156462307\\
-2.22167284234597	-1.383667215583\\
-2.22395923010075	-1.38085363735449\\
-2.22602455119113	-1.37795369503566\\
-2.22788989784863	-1.37494734110363\\
-2.2295699951167	-1.37181112495749\\
-2.23107366292106	-1.3685174705027\\
-2.23240387740984	-1.36503367508214\\
-2.23355741534321	-1.36132055065511\\
-2.23452401346765	-1.35733058482067\\
-2.23528490798758	-1.35300543696239\\
-2.23581052306342	-1.34827249084391\\
-2.23605692848604	-1.3430400385886\\
-2.23596044514103	-1.33719043642344\\
-2.23542936997175	-1.33057018661674\\
-2.23433108371332	-1.32297524720535\\
-2.23247153021665	-1.31412873364528\\
-2.22956168278891	-1.3036461295794\\
-2.22516102360426	-1.29097930733975\\
-2.21857881304302	-1.27532326062727\\
-2.2086944178179	-1.25545449226655\\
-2.19361473557335	-1.22943836414586\\
-2.16998584185419	-1.19407274934646\\
-2.13152846224427	-1.14377472060486\\
-2.06573985695198	-1.06824591845862\\
-1.9461735273105	-0.947485261462767\\
-1.71514507451435	-0.742153089006664\\
-1.25915963680298	-0.385927547026545\\
-0.466470302474344	0.15749204989669\\
0.476432702606736	0.72280409766538\\
1.17825862936543	1.08704498818123\\
1.57979469508172	1.26437318172905\\
1.80048557324685	1.34548951140256\\
1.92812298873437	1.38335485119594\\
2.00711025420764	1.40139640666682\\
2.05911603477277	1.40983029938284\\
2.09519171166257	1.41334200179773\\
2.12132034355964	1.4142135623731\\
};
\addplot [color=mycolor2]
  table[row sep=crcr]{%
2.12132034355964	2.82842712474619\\
2.18582575000323	2.82640660874952\\
2.2459761483502	2.82067996094826\\
2.30316600770785	2.81153894842004\\
2.35833530419819	2.79908319427389\\
2.4121407651283	2.78328504568152\\
2.46505313939319	2.76402339892432\\
2.51741350921551	2.74110157112695\\
2.56946568278775	2.71425682483196\\
2.62137377470791	2.68316571665979\\
2.67322998558975	2.64744786912504\\
2.72505542741334	2.60667010622203\\
2.77679571750944	2.56035271147659\\
2.82831256449113	2.50797963523731\\
2.87937250576692	2.44901464186941\\
2.92963424895255	2.38292549191834\\
2.97863667364875	2.30921809960124\\
3.02579039162557	2.22748193169334\\
3.0703766721051	2.13744642550811\\
3.11155818595542	2.03904567139316\\
3.14840590189045	1.93248505458784\\
3.1799449834646	1.81829951655162\\
3.20521925970046	1.69738980361013\\
3.2233689191928	1.57102231442397\\
3.23371058284854	1.44078163681543\\
3.23580480244767	1.30847305653245\\
3.22949541157146	1.1759834831\\
3.21490900216393	1.04511946165368\\
3.19240995273442	0.917445713323391\\
3.16251366135309	0.794144173493542\\
3.12576385298257	0.675902054710668\\
3.08257536776515	0.562820516214273\\
3.03302813706062	0.454314775018543\\
2.97656426645917	0.348948817128694\\
2.91146921429886	0.244099783305823\\
2.83385680006128	0.135243682019644\\
2.73548220960228	0.0144130439331602\\
2.59871334341581	-0.133168683665202\\
2.38479638365891	-0.336180712113444\\
2.01057175466146	-0.648371525543042\\
1.34210041814469	-1.13791067535239\\
0.375352321166521	-1.75822148716338\\
-0.527448468220416	-2.26262801632552\\
-1.11365847778129	-2.5437869612038\\
-1.4567327767317	-2.68228003389134\\
-1.6691296256022	-2.75259678373195\\
-1.81428800439327	-2.79054335642282\\
-1.92320462744895	-2.81168897715777\\
-2.01132392805712	-2.82302365363059\\
-2.08680248126988	-2.82787464184343\\
-2.15422943553058	-2.82790871121207\\
-2.21634478307291	-2.8239827357917\\
-2.27487422572476	-2.8165264847195\\
-2.33095693947289	-2.80572465930185\\
-2.3853755403757	-2.79160756168836\\
-2.43868469976231	-2.77409782070174\\
-2.49128507004909	-2.75303495626884\\
-2.54346612600745	-2.7281884019911\\
-2.59543032955127	-2.69926454755406\\
-2.64730535029083	-2.66591103001887\\
-2.69914810041563	-2.62772046423944\\
-2.75094277189846	-2.58423541950896\\
-2.80259428647749	-2.53495641420315\\
-2.85391830194335	-2.47935483422118\\
-2.90462903903618	-2.41689283661657\\
-2.95432664759112	-2.34705230341292\\
-3.00248656709011	-2.26937453210355\\
-3.04845424185856	-2.18351130413452\\
-3.0914493816545	-2.08928598065547\\
-3.13058429431929	-1.98676020448289\\
-3.16490008683964	-1.87629788656397\\
-3.19342217916266	-1.75861429449935\\
-3.21523240499265	-1.63479579561046\\
-3.22954956680171	-1.50627700312196\\
-3.23580523135335	-1.37476797309193\\
-3.23369896357047	-1.2421341494958\\
-3.22321880671614	-1.11024302689198\\
-3.20461860896113	-0.980799507713277\\
-3.17835148049538	-0.85519274756288\\
-3.14496439092874	-0.734369602871579\\
-3.10495870881183	-0.618735254776386\\
-3.05861192744495	-0.508062722468019\\
-3.0057323937692	-0.401369642978784\\
-2.94527025004781	-0.296685462718974\\
-2.87460204254734	-0.190562899149911\\
-2.78805571026974	-0.0770301658026245\\
-2.67361421522119	0.0546866392302208\\
-2.50520482475694	0.225105040795738\\
-2.22541867988909	0.474087715816021\\
-1.7209533613293	0.868461990547897\\
-0.880186083660502	1.44500069250654\\
0.110323684998138	2.03883159526995\\
0.859259206598599	2.42737283484743\\
1.30741523004191	2.62511520505649\\
1.57426953542748	2.72304263504847\\
1.74766615441039	2.77435113512036\\
1.87207484559155	2.80265686669549\\
1.96924085114733	2.81832059141065\\
2.0502990590364	2.82612977760851\\
2.12132034355964	2.82842712474619\\
};
\addlegendentry{Сумма Минковского}

\end{axis}

\begin{axis}[%
width=0.798\linewidth,
height=0.597\linewidth,
at={(-0.104\linewidth,-0.066\linewidth)},
scale only axis,
xmin=0,
xmax=1,
ymin=0,
ymax=1,
axis line style={draw=none},
ticks=none,
axis x line*=bottom,
axis y line*=left,
legend style={legend cell align=left, align=left, draw=white!15!black}
]
\end{axis}
\end{tikzpicture}%
        \caption{Сумма двух эллипсоидов.}
\end{figure}

%%%%%%%%%%%%%%%%%%%%%%%%%%%%%%%%%%%%%%%%%%%%%%%%%%%%%%%%%%%%%%%%%%%%%%%%%%%%%%%%
\clearpage
\section{Внешняя оценка суммы эллипсоидов}

\begin{theorem}
        Для суммы Минковского эллипсоидов справедлива следующая внешняя оценка
$$
        \sum\limits_{i=1}^{n} \Varepsilon(q_i,\,Q_i)
        =
        \bigcap\limits_{\| l \| = 1} \Varepsilon(q_+(l),\,Q_+(l)),
$$
где
$$
        \begin{aligned}
q_+(l) &= \sum_{i=1}^{n} q_i,
\\
Q_+(l) &= \sum_{i=1}^n p_i \cdot \sum_{i=1}^{n} \frac{Q_i}{p_i},
\quad
\mbox{где }
p_i = \langle l,\,Q_i l \rangle^{\nicefrac12}.
        \end{aligned}
$$
\end{theorem}

\begin{proof}

Будем доказывать для случая $q_i = 0,$ $i = \overline{1,\,n}$. Случай с произвольными центрами~--- аналогично.

Распишем квадрат опорной функции эллипсоида $\Varepsilon(0,\,Q_+(l))$:
\begin{multline*}
        \rho^2(l\,|\,\Varepsilon(0,\,Q_+(l)))
        =
        \sum_{i=1}^{n}
        \langle
        l,\,Q_il
        \rangle
        +
        \sum_{i < j}
        \left\langle
l,\,\left(
\frac{p_i}{p_j}Q_j + \frac{p_j}{p_i}Q_i
\right)l
        \right\rangle
        \geqslant\\\geqslant
        \left\{
\frac{a+b}{2} \geqslant \sqrt{ab}
        \right\}
        \geqslant
        \sum_{i=1}^{n}\langle l,\,Ql \rangle
        +
        2\sum_{i < j}
        \langle l,\,Q_il \rangle^{\nicefrac12}
        \langle l,\,Q_jl \rangle^{\nicefrac12}
        =\\=
        \left(
\sum_{i=1}^n\langle l,\,Q_il\rangle^{\nicefrac12}
        \right)^2
        =
        \rho^2\left(
l\left|
        \sum_{i=1}^n\Varepsilon(0, Q_i)
\right.
        \right).
\end{multline*}
Таким образом, получили, что для любого $l \neq 0$
$$
        \sum_{i=1}^n \Varepsilon(0,\,Q_i) \subseteq \Varepsilon(0,\,Q_+(l)),
$$
причем, так как равенство опорных функций достигается при
$
        p_i = \langle l,\,Q_i l \rangle^{\nicefrac12},
$
то в направлении $l \neq 0$ эллипсоид $\Varepsilon(0,\,Q_+)$ касается суммы $\sum_{i = 0}^{n} \Varepsilon(0,\,Q_i).$

\end{proof}

\clearpage
\begin{figure}[t]
        \centering
        % This file was created by matlab2tikz.
%
%The latest updates can be retrieved from
%  http://www.mathworks.com/matlabcentral/fileexchange/22022-matlab2tikz-matlab2tikz
%where you can also make suggestions and rate matlab2tikz.
%
\definecolor{mycolor1}{rgb}{0.00000,0.44700,0.74100}%
\definecolor{mycolor2}{rgb}{0.85000,0.32500,0.09800}%
%
\begin{tikzpicture}

\begin{axis}[%
width=0.618\linewidth,
height=0.471\linewidth,
at={(0\linewidth,0\linewidth)},
scale only axis,
xmin=-8,
xmax=8,
xlabel style={font=\color{white!15!black}},
xlabel={$x_1$},
ymin=-5,
ymax=5,
ylabel style={font=\color{white!15!black}},
ylabel={$x_2$},
axis background/.style={fill=white},
axis x line*=bottom,
axis y line*=left,
xmajorgrids,
ymajorgrids,
legend style={at={(0.03,0.97)}, anchor=north west, legend cell align=left, align=left, draw=white!15!black}
]
\addplot [color=mycolor1, forget plot]
  table[row sep=crcr]{%
1.1711787112841	3.34230777773576\\
1.34444389618673	3.33686722689733\\
1.50648318534277	3.32144701348819\\
1.65806242150429	3.29723890999844\\
1.79995419924852	3.26523166869947\\
1.93290799118999	3.22622794766654\\
2.05763032437368	3.18086276334402\\
2.17477223041259	3.12962150018857\\
2.28492171716205	3.07285634836238\\
2.3885995032216	3.01080060567632\\
2.48625667387058	2.94358064140543\\
2.57827324651315	2.87122553999274\\
2.66495688065443	2.79367457178694\\
2.74654114431786	2.71078271324095\\
2.82318287015103	2.62232448808267\\
2.8949582146523	2.52799644473262\\
2.96185708693385	2.42741864049226\\
3.02377565347494	2.3201355847161\\
3.0805066685131	2.20561721602438\\
3.13172744617751	2.08326066796132\\
3.17698540702193	1.95239382914097\\
3.21568133490354	1.81228204209558\\
3.24705081945703	1.66213971753183\\
3.27014489780722	1.50114915910447\\
3.28381172002077	1.32848945666784\\
3.28668221628109	1.14337881243048\\
3.2771642749897	0.945133918403077\\
3.25345178932588	0.733249674250213\\
3.21355684151227	0.507501133889645\\
3.15537469570401	0.268066487797808\\
3.07679115926015	0.015664562009573\\
2.97583883725379	-0.248307401701393\\
2.85090136682411	-0.521659161442773\\
2.70095216778915	-0.801347823435408\\
2.52579793500058	-1.08350783899399\\
2.32628156602984	-1.3635980584876\\
2.10439191407115	-1.63667268179468\\
1.86323617293213	-1.89775167737354\\
1.60685736899554	-2.14223401489281\\
1.33991789027605	-2.3662777508081\\
1.06730585933085	-2.56707514274059\\
0.793739425298697	-2.74297851870391\\
0.523437853608312	-2.89347215049445\\
0.259902596754298	-3.0190207516634\\
0.00581903236219554	-3.1208447467499\\
-0.2369372555121	-3.20067372983419\\
-0.467220354507	-3.2605178574641\\
-0.684493265340969	-3.30248032509504\\
-0.888694234930048	-3.32861897022119\\
-1.0801092196493	-3.34085455680139\\
-1.25926106809304	-3.34091776153004\\
-1.42681950330378	-3.33032515794113\\
-1.58353177449541	-3.31037507736533\\
-1.73017158467123	-3.2821558685242\\
-1.86750300244094	-3.24656096722397\\
-1.99625599202063	-3.20430689704777\\
-2.11711054093165	-3.15595168681459\\
-2.23068687011842	-3.10191219445658\\
-2.33753972875556	-3.04247952074268\\
-2.43815523449964	-2.97783215102365\\
-2.5329490934295	-2.90804674702345\\
-2.6222653208492	-2.83310668006404\\
-2.70637479428206	-2.75290849531896\\
-2.78547311726505	-2.66726655590999\\
-2.85967737154948	-2.5759161597878\\
-2.92902140016315	-2.47851546990414\\
-2.99344930844609	-2.37464666482809\\
-3.05280690958282	-2.26381681735669\\
-3.10683089311493	-2.14545915779927\\
-3.15513558252945	-2.01893559215682\\
-3.19719730262603	-1.88354163878489\\
-3.23233664163192	-1.73851533219976\\
-3.25969932318791	-1.58305212075492\\
-3.27823706849968	-1.41632833369867\\
-3.28669080350863	-1.23753634540448\\
-3.28357990805066	-1.04593497683649\\
-3.26720291066679	-0.840918687724783\\
-3.23565696131938	-0.622108318461704\\
-3.18688516768166	-0.389463978954261\\
-3.11876167995196	-0.143416533926881\\
-3.02922302993614	0.11499241610004\\
-2.91644915653916	0.383980836121862\\
-2.77908754849337	0.660934801326638\\
-2.61649915361415	0.942389492277687\\
-2.42898799999709	1.22411526429499\\
-2.21796402213223	1.50132757866017\\
-1.9859883878281	1.76901289011117\\
-1.7366683581316	2.02232922526914\\
-1.47440268220478	2.25701269019761\\
-1.20401775520347	2.46971278707818\\
-0.930363383198716	2.65819606075749\\
-0.657943143023997	2.82139307813531\\
-0.390637102663735	2.95930311865273\\
-0.131543811712393	3.07279931782965\\
0.117062449314586	3.1633871238911\\
0.353684227850852	3.23296276733506\\
0.577496523037587	3.28360333410188\\
0.788216624026908	3.3174035782658\\
0.985971690059736	3.33636162404406\\
1.1711787112841	3.34230777773576\\
};
\addplot [color=mycolor1, forget plot]
  table[row sep=crcr]{%
2.17430603473841	2.82928793073052\\
2.33044535777815	2.82444897168392\\
2.46594297362775	2.81160678704818\\
2.58415716708103	2.79277030050891\\
2.68789254387122	2.76940595677011\\
2.77946699484521	2.7425714670914\\
2.86078573906653	2.71301902411803\\
2.93341104171975	2.68127265050854\\
2.9986230753596	2.64768508851882\\
3.05747086442469	2.61247907686435\\
3.11081384046105	2.57577692251115\\
3.15935513715665	2.53762134677768\\
3.20366787001532	2.49798979992626\\
3.24421554049886	2.45680381827628\\
3.28136751247959	2.41393452310087\\
3.31541029254594	2.36920500026929\\
3.3465551303234	2.32239002374647\\
3.37494224630124	2.27321336944935\\
3.40064178827692	2.22134278883848\\
3.42365140325324	2.16638255933344\\
3.44389007534897	2.10786339134407\\
3.46118760463784	2.04522934489396\\
3.47526876652081	1.97782129527289\\
3.48573077423729	1.90485640152273\\
3.49201214853071	1.82540300920684\\
3.49335047065297	1.73835053234855\\
3.48872578597209	1.64237424725429\\
3.47678575017332	1.53589584679112\\
3.45574827300754	1.41704249382237\\
3.42327810810487	1.28361072542021\\
3.37633699194203	1.13304806285973\\
3.31101526097114	0.962476172308744\\
3.22237097655495	0.768796474817825\\
3.10433695290387	0.548942043211334\\
2.9498123210435	0.300361978709517\\
2.75112793000477	0.0218247814546395\\
2.50112454582504	-0.285440449235098\\
2.19500744060268	-0.616443644580409\\
1.83279348396173	-0.961489600474761\\
1.42152136958137	-1.3063970008595\\
0.975845648074582	-1.63450693799913\\
0.515964639544959	-1.93017020234628\\
0.0633211215265111	-2.18224450162264\\
-0.363949894265142	-2.38592186002566\\
-0.753796552421497	-2.54230740615597\\
-1.10060884720806	-2.65651208684665\\
-1.40389862561362	-2.73547347665808\\
-1.66643431787696	-2.78630322905355\\
-1.89260401811819	-2.81535915112738\\
-2.08728476369964	-2.82789001873507\\
-2.25518878499693	-2.82801992101541\\
-2.40055085490442	-2.81888822167606\\
-2.52702294235802	-2.80283518923491\\
-2.63767930270744	-2.78157977328078\\
-2.73507199115711	-2.75636932685319\\
-2.82130341222054	-2.72809764441867\\
-2.89809909341037	-2.69739456722559\\
-2.96687326596811	-2.66469244542933\\
-3.02878476984102	-2.63027467163603\\
-3.08478318424195	-2.5943106773748\\
-3.13564609696506	-2.55688082262324\\
-3.18200873955121	-2.51799374438073\\
-3.22438719765583	-2.47759802849779\\
-3.26319624623718	-2.43558952510388\\
-3.29876265010897	-2.39181521440426\\
-3.33133455291616	-2.34607421503113\\
-3.36108736562728	-2.29811628370925\\
-3.38812635908966	-2.24763796026885\\
-3.4124859564092	-2.1942763488638\\
-3.43412549761435	-2.13760038238197\\
-3.45292099547021	-2.0770992851559\\
-3.4686520985225	-2.01216782796055\\
-3.48098310405704	-1.94208786685168\\
-3.48943639793565	-1.86600559716315\\
-3.49335612455861	-1.7829039871982\\
-3.49185921454444	-1.69157008370894\\
-3.4837701826888	-1.59055749156644\\
-3.46753554642006	-1.47814566124369\\
-3.44111377096973	-1.35230025189567\\
-3.40183832761321	-1.21064372066799\\
-3.34625674726662	-1.05045381725873\\
-3.26996105481103	-0.868721532579536\\
-3.16745032441434	-0.662320310385892\\
-3.03211101926641	-0.428362470131523\\
-2.85646733273374	-0.164834179270465\\
-2.63292320715374	0.128425970177643\\
-2.35522203042063	0.448465239159584\\
-2.02065569486654	0.788006644772256\\
-1.63254200363741	1.13497995420302\\
-1.20180828192215	1.47359901929041\\
-0.746311741639337	1.78722909211069\\
-0.287482595077459	2.06211021974623\\
0.15439697822222	2.29018412674959\\
0.56404269287998	2.46977369323667\\
0.932710877487586	2.60426859699022\\
1.2575759170864	2.69994355321456\\
1.54000858568913	2.76398340384577\\
1.78376187044044	2.80319756768445\\
1.99357716381973	2.82340732367622\\
2.17430603473841	2.82928793073052\\
};
\addplot [color=mycolor1, forget plot]
  table[row sep=crcr]{%
6.81839938705513	4.98558555137348\\
6.9451056567848	4.98172857837679\\
7.04503269745632	4.97230404998413\\
7.12550382874078	4.95951351377152\\
7.19150742751882	4.9446702051191\\
7.24652621591938	4.92856445823852\\
7.29304670485787	4.91167081831979\\
7.33288010703728	4.89426840679286\\
7.36736941959477	4.87651242497647\\
7.39752608589151	4.8584773879902\\
7.42412204545158	4.84018356931731\\
7.44775286792895	4.82161320692916\\
7.46888172297768	4.802720283049\\
7.48787036298415	4.78343612628583\\
7.50500109471872	4.76367216612502\\
7.52049232706941	4.74332061039937\\
7.53450938011309	4.72225345843491\\
7.54717163259053	4.70032001509007\\
7.5585566510167	4.67734287814631\\
7.56870160579294	4.65311219540491\\
7.57760198044974	4.62737779738012\\
7.5852072695652	4.59983857492861\\
7.59141298466069	4.57012814868072\\
7.59604777339064	4.53779541139837\\
7.59885369623389	4.5022778262537\\
7.59945651723111	4.46286428877193\\
7.59732094276784	4.41864266581239\\
7.59168253759535	4.36842439739232\\
7.58144256176967	4.31063406338187\\
7.56500233907657	4.243144292049\\
7.5399964148077	4.16302350528493\\
7.50285171749803	4.0661415588715\\
7.44803942928042	3.9465388293111\\
7.36677017908548	3.79539498589725\\
7.24466050473369	3.59931675028097\\
7.0574886748104	3.33749314960869\\
6.76352025145179	2.97714793958209\\
6.29053702849613	2.46732264221061\\
5.51919290384651	1.7351657602429\\
4.28518881238173	0.704044860696217\\
2.47617548846006	-0.62399326502574\\
0.26472488379459	-2.04447711678708\\
-1.85854763216566	-3.22872807539079\\
-3.51179701572037	-4.01952314056496\\
-4.65672460540954	-4.48084729701715\\
-5.42192597149296	-4.73406098849309\\
-5.93815571883776	-4.86916023516302\\
-6.29610684807415	-4.93885917516955\\
-6.55240778900606	-4.97201605639051\\
-6.74175384000817	-4.98434219551015\\
-6.88568139038962	-4.98454045434875\\
-6.99789141017824	-4.97754800336823\\
-7.08734546190676	-4.96623193256517\\
-7.16006824322224	-4.95228978176482\\
-7.22021562299617	-4.93673979314651\\
-7.27072257215752	-4.92019511301591\\
-7.31370624925992	-4.90302121434563\\
-7.3507230387261	-4.88542843600515\\
-7.3829363263936	-4.86752747846997\\
-7.41122838809586	-4.84936318429306\\
-7.43627646256458	-4.8309352520156\\
-7.45860534888868	-4.81221087003346\\
-7.47862427236261	-4.79313219481558\\
-7.49665296586647	-4.77362040406707\\
-7.51294017121452	-4.75357734217863\\
-7.52767665013585	-4.73288533123189\\
-7.54100405805473	-4.71140542676794\\
-7.55302052460794	-4.68897418296435\\
-7.56378340709999	-4.66539881155098\\
-7.57330937097375	-4.64045043926517\\
-7.58157165270455	-4.61385495942758\\
-7.58849402465152	-4.58528069952081\\
-7.5939405470587	-4.55432174128212\\
-7.59769957191291	-4.52047516178699\\
-7.59945951764889	-4.48310960021399\\
-7.59877242863057	-4.44142120888268\\
-7.59499885959066	-4.39437090303387\\
-7.58722344586879	-4.34059333582325\\
-7.5741232741605	-4.27826223117822\\
-7.55375828026879	-4.20488688476054\\
-7.52322939077674	-4.11699767437352\\
-7.4781062033147	-4.00964867067367\\
-7.4114422958755	-3.87561300911386\\
-7.31203520475341	-3.70405586877619\\
-7.16128309678768	-3.47832412658061\\
-6.9274595685231	-3.17231678512378\\
-6.55558188850301	-2.7449831897496\\
-5.95177163129806	-2.13426805552596\\
-4.97054626054604	-1.26027984798751\\
-3.45135195503113	-0.0699980423053237\\
-1.39392514759134	1.34380755585939\\
0.840019649925826	2.68258056595531\\
2.75314581466553	3.67259137928405\\
4.14099226701179	4.28347990938291\\
5.07776315797388	4.62683679623043\\
5.70454533578041	4.81235986820016\\
6.13273164742407	4.90997324701889\\
6.4344036995944	4.95880562065331\\
6.65387188453432	4.98012287766274\\
6.81839938705513	4.98558555137348\\
};
\addplot [color=mycolor1, forget plot]
  table[row sep=crcr]{%
1.42632528730357	3.05419055216814\\
1.59409885697327	3.04894192904819\\
1.74764607607782	3.03434727439337\\
1.88831768585849	3.01189675254528\\
2.01740026430351	2.98279289019926\\
2.13608416558777	2.94798796969752\\
2.24544811665552	2.90822043063386\\
2.34645448710049	2.86404765904994\\
2.43995094101877	2.81587408211711\\
2.52667551525986	2.76397436887494\\
2.60726315074524	2.7085119950182\\
2.68225238936669	2.64955362911531\\
2.75209140658278	2.58707985082535\\
2.81714284020386	2.52099268940941\\
2.8776870472672	2.45112041621127\\
2.93392350905193	2.37721996369354\\
2.98597013463019	2.29897729217642\\
3.03386020403657	2.2160059952028\\
3.07753665694582	2.12784443619997\\
3.11684338412892	2.0339517557063\\
3.15151313143606	1.93370319745457\\
3.18115160055983	1.82638539774865\\
3.20521736059578	1.71119259976722\\
3.22299732316753	1.58722523687895\\
3.23357786670449	1.45349302920453\\
3.23581235158743	1.3089257071312\\
3.2282869316483	1.15239574159971\\
3.20928847715888	0.982758977142026\\
3.176781336107	0.798920617843454\\
3.12840373601196	0.59993508496799\\
3.06149968552777	0.385147822534311\\
2.97320730405123	0.154383494569105\\
2.86062722894035	-0.0918240253086883\\
2.72109091958337	-0.351980889021321\\
2.55253274510375	-0.623406293831111\\
2.3539375047955	-0.902099020098742\\
2.12578916758129	-1.1827809605491\\
1.87040277410249	-1.45918525535049\\
1.59200788692453	-1.72459900904301\\
1.29649516293625	-1.97258135199058\\
0.990837210384203	-2.197696081303\\
0.682309937938043	-2.39607365215453\\
0.377709629271056	-2.56567186636937\\
0.0827467276445912	-2.70621088488075\\
-0.198285197101545	-2.81885870156143\\
-0.462566309044838	-2.90579312858928\\
-0.70860369444456	-2.96975905169869\\
-0.935974346036652	-3.01369849093327\\
-1.14503867672096	-3.04048466100225\\
-1.33667724745256	-3.05275737054036\\
-1.51207595388432	-3.05283983637206\\
-1.67256535682076	-3.04271255194964\\
-1.81950894294363	-3.02402253127754\\
-1.95423070636569	-2.99811154608017\\
-2.07797191745059	-2.96605230356283\\
-2.19186825631256	-2.92868580072754\\
-2.29694038051205	-2.8866561327032\\
-2.3940928301155	-2.84044100242064\\
-2.48411769306366	-2.7903773569584\\
-2.56770060761846	-2.73668222109411\\
-2.64542750422993	-2.67946910987757\\
-2.71779105258215	-2.61876051726236\\
-2.7851961466965	-2.55449698645495\\
-2.84796398665464	-2.48654322508971\\
-2.9063344415271	-2.4146916680026\\
-2.96046643468488	-2.33866383248366\\
-3.01043610126227	-2.25810976873771\\
-3.05623244371903	-2.1726058926298\\
-3.09775016794598	-2.08165151011889\\
-3.13477933218787	-1.98466441800741\\
-3.16699140132554	-1.88097611459704\\
-3.19392129598201	-1.76982740563066\\
-3.21494510178343	-1.65036558457997\\
-3.22925332680441	-1.52164495079917\\
-3.23582007028983	-1.38263325815509\\
-3.2333693489984	-1.23222780416891\\
-3.22034133184322	-1.06928627425662\\
-3.19486361413493	-0.892679032711539\\
-3.15473614964176	-0.701370955594937\\
-3.09744308405396	-0.494541380754748\\
-3.02020998960036	-0.271748964553961\\
-2.92012930110728	-0.0331421889446313\\
-2.79437683989129	0.220296299061972\\
-2.64053313743929	0.486503595447298\\
-2.45699943943893	0.762146269718005\\
-2.24345818717511	1.04256505003492\\
-2.00128002875728	1.3219347994068\\
-1.73374676631364	1.59368386301881\\
-1.44597222977026	1.85114025281027\\
-1.14447704690877	2.08828044420864\\
-0.836487911591588	2.30039820704417\\
-0.529131242108844	2.48452621841593\\
-0.228720064349343	2.63952985558765\\
0.0597204337400604	2.76590334036294\\
0.332641313986094	2.86537701210694\\
0.58791360706013	2.94046425247022\\
0.824614911507097	2.99404850797217\\
1.0427491736426	3.02906396360623\\
1.24296707789219	3.04828205038319\\
1.42632528730357	3.05419055216814\\
};
\addplot [color=mycolor1, forget plot]
  table[row sep=crcr]{%
0.878355311988031	4.00104451440901\\
1.06350692625045	3.99520620111796\\
1.24105208545559	3.97828669634494\\
1.41132245401614	3.95107065481952\\
1.57466773303343	3.91420199851153\\
1.73143696912751	3.86819041177193\\
1.88196376005733	3.81341874894101\\
2.02655438296571	3.750150547048\\
2.16547798030495	3.67853714869157\\
2.29895805486653	3.59862418003358\\
2.42716463522509	3.51035731500992\\
2.55020657118865	3.41358740789436\\
2.668123506243	3.30807520909464\\
2.78087715546415	3.19349600831664\\
2.88834160146377	3.06944468768286\\
2.99029241959652	2.93544182533574\\
3.08639457260405	2.79094167459111\\
3.17618919386929	2.63534305689275\\
3.25907963118477	2.46800444241896\\
3.33431747555457	2.28826473051858\\
3.4009897767428	2.09547144282054\\
3.45800926415817	1.88901813486057\\
3.50411013946209	1.6683927090189\\
3.53785283151403	1.43323782151323\\
3.55764187603864	1.18342353411735\\
3.5617615690935	0.919130581462076\\
3.54843389392288	0.64093999291339\\
3.51590199238786	0.349921399335943\\
3.46253971391698	0.0477085835925056\\
3.38698331236318	-0.263452424287396\\
3.28827546184747	-0.580698823062855\\
3.16600547552711	-0.900604275852195\\
3.02042477853998	-1.21929774783176\\
2.85251553485022	-1.53264516950835\\
2.66399457594665	-1.83647819729149\\
2.45724457303916	-2.12684262395052\\
2.23517770285111	-2.4002325211065\\
2.00105011016354	-2.65377765196434\\
1.7582542402097	-2.88536099227669\\
1.51011809336617	-3.09365738536341\\
1.25973568625162	-3.27809894883646\\
1.00984379818778	-3.43878382182336\\
0.762749729660402	-3.57635010468598\\
0.520306129482393	-3.69183655722441\\
0.283923478606458	-3.78654748590043\\
0.0546086990229475	-3.86193343438586\\
-0.166981171541117	-3.91949359001469\\
-0.380479311033542	-3.96070128664519\\
-0.585749985503425	-3.986950972031\\
-0.782834069573656	-3.99952339235443\\
-0.971901342639129	-3.99956518582877\\
-1.15321025667837	-3.98807919312425\\
-1.32707494440725	-3.96592226831757\\
-1.49383870905093	-3.93380798674879\\
-1.65385299334717	-3.89231225949612\\
-1.80746076448412	-3.84188040852438\\
-1.95498329637778	-3.78283470631474\\
-2.09670942756051	-3.71538173904544\\
-2.23288648811713	-3.63961922624244\\
-2.36371220332813	-3.55554214043776\\
-2.48932698599055	-3.46304813674846\\
-2.60980612184367	-3.36194244219558\\
-2.7251514361164	-3.25194248386318\\
-2.83528211072562	-3.1326826674037\\
-2.9400244113109	-3.00371986447725\\
-3.03910019488404	-2.86454033832207\\
-3.13211422011443	-2.71456903567785\\
-3.21854049496589	-2.55318239915177\\
-3.29770819576308	-2.37972609464072\\
-3.3687881039304	-2.19353927459823\\
-3.43078105286486	-1.99398715573653\\
-3.4825105630904	-1.78050369093681\\
-3.52262264058648	-1.55264583001233\\
-3.54959653315705	-1.31016012225037\\
-3.56177090540339	-1.05306102369323\\
-3.55739011243071	-0.781718072012295\\
-3.53467462163135	-0.496946044577514\\
-3.49191769492277	-0.200088546569051\\
-3.42760684601089	0.106918179217524\\
-3.34056332736365	0.421516953800008\\
-3.23008662410292	0.74055253109619\\
-3.09608510202537	1.06035723112041\\
-2.93917069281702	1.37690232801292\\
-2.76069690826706	1.68600603833541\\
-2.56272661017962	1.98357610732913\\
-2.34792790075655	2.26585543441595\\
-2.11941020973985	2.52963648855715\\
-1.88052401217249	2.77241584827913\\
-1.63465314874106	2.9924724548048\\
-1.38502715711598	3.18886813413911\\
-1.13457364935907	3.3613821295424\\
-0.885820579981077	3.51039961632549\\
-0.640848456119384	3.63677651896027\\
-0.40128537454523	3.7417004733795\\
-0.16833402832988	3.82656253933673\\
0.0571808702239901	3.89284832757713\\
0.274754939025249	3.94205200602846\\
0.48414407094491	3.975612869128\\
0.685307253194799	3.99487187487964\\
0.87835531198803	4.00104451440901\\
};
\addplot [color=mycolor1, forget plot]
  table[row sep=crcr]{%
1.02375040531617	3.61290204042887\\
1.20176097639549	3.60730045239441\\
1.37036957674768	3.59124369131131\\
1.53008538748134	3.56572536946192\\
1.68143780878129	3.53157387649472\\
1.82495211979655	3.48946246279246\\
1.96113166891982	3.43992080417065\\
2.09044497609687	3.38334668835468\\
2.2133163581027	3.32001698205598\\
2.33011892490929	3.25009741061438\\
2.44116901027864	3.17365094361483\\
2.54672128027864	3.09064476183925\\
2.64696390782588	3.0009559112872\\
2.74201331386758	2.90437585191723\\
2.83190806411647	2.8006142012692\\
2.91660158441303	2.68930207223581\\
2.99595343022608	2.56999552386096\\
3.06971893197026	2.44217979639812\\
3.13753715736318	2.30527519718943\\
3.19891731008915	2.15864574879307\\
3.25322395208505	2.00161200452337\\
3.29966183225684	1.83346976493728\\
3.33726166727844	1.65351675400746\\
3.36486898352328	1.46108956040721\\
3.38113910113713	1.25561318766688\\
3.38454247282657	1.03666518752547\\
3.3733857285037	0.804055302209506\\
3.34585461031475	0.55791950056818\\
3.30008499547509	0.29882399643962\\
3.23426669333464	0.027870249673398\\
3.14678092088188	-0.253213474226809\\
3.03636585661394	-0.542013456154021\\
2.90229582008168	-0.835426867441589\\
2.74455015014256	-1.12973284012124\\
2.56394095281318	-1.42074875525535\\
2.36216843685569	-1.70406624141488\\
2.14178131502429	-1.97534198104143\\
1.90603727640003	-2.23060186778484\\
1.65868024595527	-2.46651004688063\\
1.40366973892964	-2.68056058438075\\
1.14490648466575	-2.87116725842797\\
0.885995072210844	-3.03764980487017\\
0.63007116427807	-3.18013500544318\\
0.379703604907356	-3.29940273110009\\
0.136866392078723	-3.3967091096898\\
-0.0970344061660403	-3.47361363503207\\
-0.3210974496125	-3.53182808856116\\
-0.534844239226391	-3.57309603977806\\
-0.738132419461264	-3.5991045027063\\
-0.931073626071126	-3.61142473381268\\
-1.11396114323217	-3.61147691260704\\
-1.28720951056091	-3.60051289518032\\
-1.45130614583367	-3.57961168548513\\
-1.60677385886146	-3.54968318739327\\
-1.75414258331887	-3.51147682884627\\
-1.89392851680262	-3.46559259474307\\
-2.02661895578985	-3.41249278629328\\
-2.15266131796345	-3.35251342660259\\
-2.27245508285067	-3.28587467397293\\
-2.38634561026136	-3.21268991747198\\
-2.49461899453625	-3.13297344723352\\
-2.59749727486108	-3.04664674493076\\
-2.69513344957367	-2.95354355339843\\
-2.78760584167458	-2.85341397942644\\
-2.87491144258746	-2.74592797770453\\
-2.95695793279452	-2.63067867162488\\
-3.0335541551112	-2.5071861012065\\
-3.10439891645093	-2.37490216114041\\
-3.16906813919749	-2.23321771157066\\
-3.22700060200421	-2.0814731142732\\
-3.2774828366435	-1.91897376091784\\
-3.31963422182456	-1.74501249358455\\
-3.35239397487088	-1.55890111606585\\
-3.37451261146532	-1.36001335729562\\
-3.38455150677088	-1.14784151232587\\
-3.38089535402736	-0.922068315058348\\
-3.36178335378977	-0.682654091339456\\
-3.32536546776439	-0.429936596356688\\
-3.26978941824833	-0.164736978685911\\
-3.1933215530012	0.111539791837043\\
-3.09449957081791	0.396827137223658\\
-2.97230730339444	0.688362960794408\\
-2.82635227231545	0.982721291257162\\
-2.65701807848947	1.27592551447855\\
-2.46555953737673	1.5636464007621\\
-2.25411243852087	1.84146980592806\\
-2.02560326847436	2.10520006333872\\
-1.78356466168654	2.35115257139234\\
-1.53188345629107	2.57638853279399\\
-1.27452259365677	2.77885734455166\\
-1.01526081426038	2.95743342231848\\
-0.757485202333998	3.11185660361474\\
-0.504055616196481	3.24260160764969\\
-0.257243124584251	3.35070882318175\\
-0.0187316848425002	3.43760661534103\\
0.210334443627698	3.50494770426823\\
0.429276150050858	3.55447278851852\\
0.637793727556865	3.58790627445286\\
0.835881553665213	3.6068830687125\\
1.02375040531617	3.61290204042887\\
};
\addplot [color=mycolor1, forget plot]
  table[row sep=crcr]{%
1.65642263207936	2.91541373390173\\
1.82037784583777	2.91030086390922\\
1.9677141607887	2.89631048846228\\
2.10039858637834	2.87514661502634\\
2.22021829502882	2.84814149164205\\
2.32876139289674	2.81631933318551\\
2.42741778412894	2.78045311346391\\
2.51739080453339	2.74111237919629\\
2.5997137273383	2.69870202037562\\
2.67526759617419	2.65349283915425\\
2.74479836712207	2.60564506995236\\
2.80893228619176	2.55522600914459\\
2.86818898383772	2.50222278100859\\
2.92299207190501	2.44655108756695\\
2.97367717316179	2.38806060833533\\
3.02049735848801	2.32653755200115\\
3.06362594816039	2.26170472357171\\
3.1031565727605	2.19321936030627\\
3.13910029784126	2.12066890980134\\
3.17137950106921	2.04356487823619\\
3.19981805563009	1.96133487528602\\
3.22412722545868	1.87331304194841\\
3.24388652917448	1.77872919811096\\
3.25851870767061	1.67669733555935\\
3.26725788924465	1.56620458170098\\
3.26911018723029	1.44610257455019\\
3.26280646739801	1.31510446242669\\
3.24674818526298	1.17179264275761\\
3.21894948375011	1.01464503971421\\
3.17698281835873	0.842091215225841\\
3.11794201723986	0.652613547633569\\
3.03844644930004	0.444911865172638\\
2.93472234736788	0.218149481119256\\
2.8028090465519	-0.0277104952854745\\
2.63894062157472	-0.291494298734734\\
2.44013251083137	-0.570393290038137\\
2.20494150359352	-0.859649139467996\\
1.93426251361171	-1.1525223052927\\
1.63191099575297	-1.44070778788785\\
1.30469689836056	-1.71524417896233\\
0.961809545903872	-1.96775111774133\\
0.613598070559529	-2.19163828849998\\
0.270106034005761	-2.3829008079898\\
-0.0601842625425428	-2.54029310737039\\
-0.371081514574716	-2.66493929156151\\
-0.658830666399524	-2.75962282515189\\
-0.921821623365125	-2.82802513033402\\
-1.16008213079777	-2.87409551910905\\
-1.37473982056242	-2.90162188662688\\
-1.56756118680114	-2.91399083853101\\
-1.740606418096	-2.91408981741843\\
-1.89599662931821	-2.90429937494905\\
-2.03577159460937	-2.88653403208306\\
-2.16181254714747	-2.86230368386329\\
-2.27580800711821	-2.83277894819268\\
-2.37924602595903	-2.79885186349478\\
-2.47342135895941	-2.76118829145959\\
-2.55945010266752	-2.72027114215685\\
-2.63828719565115	-2.67643490915497\\
-2.71074408814828	-2.6298925648727\\
-2.77750509407517	-2.58075599703199\\
-2.83914166821686	-2.52905108890512\\
-2.89612426683683	-2.47472838286705\\
-2.94883166501475	-2.41767008273216\\
-2.99755769310528	-2.35769397601212\\
-3.04251536440935	-2.2945547054575\\
-3.08383832442057	-2.22794269467252\\
-3.12157947480033	-2.15748093717595\\
-3.15570652122286	-2.08271979484268\\
-3.18609406848149	-2.00312992664664\\
-3.21251174361749	-1.91809349492053\\
-3.23460767710368	-1.82689389684362\\
-3.25188653209408	-1.72870448128615\\
-3.26368118112669	-1.62257709444266\\
-3.26911716301003	-1.50743193969673\\
-3.2670693455275	-1.38205126127052\\
-3.2561110053532	-1.24508092609879\\
-3.23445719275566	-1.09504625615184\\
-3.19990733869133	-0.930391566459728\\
-3.14979732421527	-0.749556666920957\\
-3.08097940364488	-0.5511073661317\\
-2.98985962964534	-0.333938851596169\\
-2.87253514113668	-0.0975668422736795\\
-2.72508233437513	0.157494529622942\\
-2.54404001527801	0.429303881850831\\
-2.32709234919699	0.714103779021443\\
-2.07387200165371	1.00612432375832\\
-1.78668666913353	1.29775933191409\\
-1.47088122335579	1.58023484821087\\
-1.1345756241589	1.84471730760219\\
-0.787717671493854	2.083588282538\\
-0.440683840723205	2.2914878559795\\
-0.102865668160095	2.46580854778831\\
0.218346993157528	2.60656514302512\\
0.517995089490965	2.71580904103266\\
0.793451069451776	2.79686253352453\\
1.04399255122457	2.85360765026289\\
1.27026126265922	2.88995393049057\\
1.47375493321376	2.90950846059762\\
1.65642263207936	2.91541373390173\\
};
\addplot [color=mycolor1, forget plot]
  table[row sep=crcr]{%
5.18712583895256	4.03643886090506\\
5.31525725204707	4.03252906981332\\
5.41757620024171	4.02287294925828\\
5.50079469346955	4.00964175117464\\
5.5696003253642	3.99416555895309\\
5.62733170440595	3.97726378963646\\
5.67641122072661	3.95943940624426\\
5.71862685754273	3.94099514264964\\
5.75531895681641	3.92210424785202\\
5.78750618287741	3.90285415339594\\
5.81597190070615	3.88327366803006\\
5.84132430607785	3.86334991685115\\
5.8640388204713	3.84303872581938\\
5.88448826356221	3.82227067715433\\
5.90296441651425	3.80095417392495\\
5.91969336093515	3.77897630111696\\
5.9348461628453	3.75620191182305\\
5.94854590917154	3.73247111600344\\
5.96087169458886	3.70759515234812\\
5.97185983006285	3.68135044432882\\
5.98150224897495	3.65347044950186\\
5.98974177522711	3.6236346761894\\
5.99646353807814	3.59145392663468\\
6.00148130126427	3.55645037850838\\
6.00451671496219	3.51803045859987\\
6.00516833049201	3.4754474699895\\
6.00286535774335	3.42774940781586\\
5.99679810904926	3.37370501352264\\
5.98581200843584	3.31169733754317\\
5.96824343302717	3.23956802578995\\
5.94166076599006	3.1543857977506\\
5.90244799924475	3.05209698328066\\
5.84512255933808	2.92699173597258\\
5.76120034149497	2.77088497318124\\
5.63729389255025	2.57187402037809\\
5.45196389052451	2.31254430268633\\
5.170800179951	1.96774488397845\\
4.74005681044081	1.50317000806675\\
4.08343851340225	0.879405714822638\\
3.11873275843801	0.0725212336633277\\
1.82240158153292	-0.879979006481791\\
0.3208772432074	-1.84474570715855\\
-1.13257542859024	-2.65499655952802\\
-2.33491184292571	-3.22951294326548\\
-3.23551674549088	-3.5919659263711\\
-3.8819331864792	-3.80563197939625\\
-4.34333653284971	-3.92625706921822\\
-4.67714451079244	-3.99118894206398\\
-4.92384191511931	-4.02306768443975\\
-5.1104782906461	-4.03519717120396\\
-5.25494229682946	-4.03538413725133\\
-5.36916620850968	-4.02825870448253\\
-5.46124197072875	-4.01660612024233\\
-5.53676482188745	-4.00212386009999\\
-5.59968129721263	-3.98585566318484\\
-5.65282887921241	-3.96844430635513\\
-5.69828447180154	-3.95028150863634\\
-5.73759331009454	-3.93159842884888\\
-5.77192201193149	-3.91252116361816\\
-5.80216268002389	-3.89310518437901\\
-5.82900484455859	-3.8733568160797\\
-5.8529858815526	-3.85324654641236\\
-5.87452674587498	-3.83271703316872\\
-5.89395747634478	-3.81168753720132\\
-5.91153540716701	-3.79005581256549\\
-5.92745802288954	-3.7676980440536\\
-5.94187172094541	-3.74446712558521\\
-5.95487727028482	-3.72018935421026\\
-5.9665323937902	-3.69465943029842\\
-5.97685159715728	-3.66763347237637\\
-5.98580306939653	-3.63881954565161\\
-5.99330214181357	-3.60786493364055\\
-5.99920035424942	-3.57433900902988\\
-6.00326855541121	-3.53771001937137\\
-6.00517152686637	-3.49731329719506\\
-6.00443015115258	-3.45230717604644\\
-6.00036477491604	-3.40161098959502\\
-5.99200950644002	-3.34381653301228\\
-5.9779805973346	-3.27705958700445\\
-5.95627075586493	-3.19883042356599\\
-5.92392157032137	-3.10568985900584\\
-5.87649171664733	-2.99283785316287\\
-5.80717843048081	-2.85345220038719\\
-5.70534840422258	-2.67767671923539\\
-5.5540823371721	-2.45111316334733\\
-5.32619170660676	-2.15275947247519\\
-4.97839825424011	-1.75289683024479\\
-4.44544133267055	-1.21346613054153\\
-3.64360737117788	-0.498615041134729\\
-2.50836858312705	0.391708787828822\\
-1.08269748007528	1.37202389763564\\
0.427899320710524	2.27720604166756\\
1.77169676503285	2.97202628862457\\
2.82116844859093	3.43344452188135\\
3.58608777976602	3.71348668836244\\
4.131761062643	3.87482760873836\\
4.52329054454194	3.96399345419352\\
4.80943067044692	4.01026330629437\\
5.02337853892604	4.03101761132268\\
5.18712583895256	4.03643886090506\\
};
\addplot [color=mycolor1, forget plot]
  table[row sep=crcr]{%
2.11655100682591	2.82843428941261\\
2.27354611471052	2.82356578471402\\
2.41025665966359	2.81060627378123\\
2.52989546066145	2.79154091646479\\
2.63516721135814	2.76782903818278\\
2.72832261743308	2.74053005988702\\
2.81122263331923	2.71040195994312\\
2.88540101761024	2.67797585950846\\
2.95212010046617	2.64361138065595\\
3.01241816078134	2.60753712579265\\
3.06714850265728	2.56987987342384\\
3.11701103132388	2.530685279376\\
3.16257733553198	2.48993216479353\\
3.20431024779221	2.44754189978502\\
3.2425787120241	2.40338394581532\\
3.27766860712889	2.35727827727643\\
3.30978998395011	2.30899513789073\\
3.33908098109372	2.25825237864351\\
3.36560849027125	2.20471045325069\\
3.38936543605966	2.14796500184279\\
3.41026430598958	2.08753682547208\\
3.42812629933144	2.02285894077584\\
3.44266513926722	1.95326031123863\\
3.4534641954334	1.87794579759002\\
3.4599450789137	1.79597189595953\\
3.46132530302978	1.70621801868249\\
3.45656199668418	1.60735356718522\\
3.44427815481841	1.49780211181684\\
3.4226678564112	1.37570607723411\\
3.38937801254489	1.23889915632012\\
3.34136798234704	1.08490031045564\\
3.27475744922555	0.910953996822627\\
3.18469151120958	0.71415720554415\\
3.06528542191618	0.491733786396986\\
2.90976311744834	0.241532573386507\\
2.71096460636707	-0.0371839138929645\\
2.4624261097362	-0.342670722307781\\
2.16014017496843	-0.6695545814815\\
1.80476531683354	-1.00810768327461\\
1.40349383662699	-1.34464591286619\\
0.97038036758317	-1.66351810247334\\
0.524318870350239	-1.95029936328905\\
0.085159888355619	-2.19486019379004\\
-0.330303447862874	-2.39290064579353\\
-0.710776629391997	-2.54551613559592\\
-1.05080767266494	-2.65747804245967\\
-1.34965578360068	-2.7352745046498\\
-1.60964878916396	-2.78560491585113\\
-1.83470724848301	-2.81451245994852\\
-2.02929746385226	-2.82703310171971\\
-2.19780572461383	-2.82716004636509\\
-2.34422376368673	-2.81795933766786\\
-2.47202961915923	-2.80173490909645\\
-2.58417664940125	-2.78019148732509\\
-2.68313477180989	-2.75457447721281\\
-2.77095171368762	-2.72578186937574\\
-2.84931736926041	-2.69445020124993\\
-2.91962332499729	-2.66101894855914\\
-2.98301449583141	-2.625777943375\\
-3.0404322803138	-2.5889018140239\\
-3.0926497659045	-2.55047463225179\\
-3.14029993083631	-2.51050718663972\\
-3.18389785141002	-2.46894866039765\\
-3.22385782311156	-2.42569398502503\\
-3.26050613704066	-2.38058775033286\\
-3.29409006498849	-2.3334252506136\\
-3.32478341468164	-2.28395101236323\\
-3.35268882400787	-2.23185496103844\\
-3.37783676405674	-2.17676622764621\\
-3.4001810051745	-2.11824445995762\\
-3.41959005405378	-2.05576838234722\\
-3.43583377655189	-1.98872124382334\\
-3.44856406235342	-1.91637271697959\\
-3.45728794729811	-1.83785678961846\\
-3.46133108014034	-1.75214528356065\\
-3.45978882279539	-1.65801695123747\\
-3.45146169164846	-1.55402283821256\\
-3.43477150758744	-1.43845010349195\\
-3.40765503359	-1.3092893324724\\
-3.36743408852029	-1.16421545119853\\
-3.31066711786081	-1.00060088864916\\
-3.23300037864963	-0.815592941179428\\
-3.12906223935289	-0.606305635533384\\
-2.99248669002533	-0.370196004944904\\
-2.81621104383475	-0.105701784839631\\
-2.59324586139854	0.186819618571972\\
-2.31809609238649	0.503941380582688\\
-1.98880682986173	0.838150307963187\\
-1.60913736983443	1.17759478204281\\
-1.18981009540952	1.5072611728684\\
-0.747698446788933	1.81168182170006\\
-0.30271621218734	2.07826646569063\\
0.126382115899771	2.29973691404187\\
0.525382985478411	2.47465037794311\\
0.885985247779934	2.60619273420778\\
1.20528688360715	2.70022002649834\\
1.48428693993919	2.7634737488266\\
1.72626932379742	2.80239670394494\\
1.93552996856662	2.8225480577959\\
2.11655100682591	2.82843428941261\\
};
\addplot [color=mycolor1]
  table[row sep=crcr]{%
0.948683298050514	3.79473319220206\\
1.12996891013208	3.78902241071535\\
1.3027838103604	3.77255915923418\\
1.46753552344581	3.74623053454947\\
1.62465224482371	3.71077288416293\\
1.77456152903207	3.66677970030574\\
1.91767403561802	3.61471073626391\\
2.05437110063153	3.55490128334521\\
2.18499505421831	3.4875709448019\\
2.30984136685525	3.41283153872394\\
2.42915185734506	3.33069397942336\\
2.54310832592638	3.24107414976038\\
2.65182608463458	3.14379790634541\\
2.75534694820212	3.03860547381077\\
2.8536313296431	2.92515559896292\\
2.9465491655584	2.80302996392973\\
3.0338694906698	2.67173851056894\\
3.11524860652315	2.53072651479627\\
3.19021696761039	2.3793844730251\\
3.25816516606007	2.21706211963439\\
3.31832976399147	2.0430881668871\\
3.36978023105031	1.85679760784185\\
3.41140891541445	1.65756857671014\\
3.44192680709927	1.44487070344488\\
3.45986879009691	1.21832646083672\\
3.46361298678702	0.977785965288651\\
3.45141941180798	0.723413822791541\\
3.4214930614047	0.455783738589644\\
3.37207522993717	0.175972765268488\\
3.30156372141992	-0.114357323644244\\
3.20865740947072	-0.412907981729085\\
3.0925136091427	-0.716739496073036\\
2.95289924526556	-1.02233055985344\\
2.79031115637498	-1.3257107709346\\
2.60603988731933	-1.6226624623078\\
2.40215712665582	-1.90897292279738\\
2.18141953804948	-2.18070442957845\\
1.94709823579504	-2.43444260737754\\
1.70275852491129	-2.66748659951055\\
1.45202370644265	-2.877956852001\\
1.19835703486066	-3.0648138241024\\
0.944888096380122	-3.22779787311511\\
0.694297500839398	-3.36731218002033\\
0.448761194715288	-3.48427490957016\\
0.209946190700892	-3.57996468520442\\
-0.0209555786883461	-3.65587741213202\\
-0.243169602253506	-3.71360522906444\\
-0.456269901983927	-3.7547419037936\\
-0.660108423391693	-3.78081428495334\\
-0.854748940791145	-3.79323662458319\\
-1.04040889805107	-3.79328335955738\\
-1.21741050351997	-3.78207578896275\\
-1.38614100991934	-3.76057854694004\\
-1.54702130890241	-3.72960250007616\\
-1.70048159286015	-3.6898114747465\\
-1.84694273429099	-3.64173092446581\\
-1.98680208954437	-3.58575723170233\\
-2.1204225680783	-3.52216679568137\\
-2.24812396922559	-3.45112440165093\\
-2.37017574651009	-3.37269062063008\\
-2.48679050050333	-3.2868281760587\\
-2.59811762050387	-3.19340735751618\\
-2.70423659456836	-3.09221068144532\\
-2.80514959232286	-2.98293711145668\\
-2.90077300437504	-2.86520627071399\\
-2.99092770791458	-2.73856321838213\\
-3.07532793559953	-2.60248453106578\\
-3.1535687736533	-2.45638663503084\\
-3.22511252958097	-2.29963757630341\\
-3.28927451889835	-2.13157368366421\\
-3.34520925490937	-1.95152284702656\\
-3.39189861348128	-1.7588363466938\\
-3.42814429814894	-1.5529312348605\\
-3.45256782598978	-1.33334504716437\\
-3.46362220116165	-1.0998039137315\\
-3.45962024522194	-0.852303713618117\\
-3.43878487541956	-0.591201557879526\\
-3.3993259814164	-0.317311504170883\\
-3.3395463824176	-0.0319942042762583\\
-3.25797518618734	0.262774139257094\\
-3.15352067565559	0.564371272065384\\
-3.02562738470585	0.869554764908886\\
-2.87441511977517	1.17455760801215\\
-2.70077402394131	1.47525676358082\\
-2.50639202484484	1.767403513355\\
-2.29370038309167	2.04688929885519\\
-2.06573813670639	2.31000991273335\\
-1.8259528645897	2.5536887433386\\
-1.57796799985389	2.77562773795799\\
-1.32535178123783	2.9743703475303\\
-1.07141885643386	3.14927865398809\\
-0.819084909373598	3.30044159903781\\
-0.570781691446472	3.42853925898355\\
-0.328428492520302	3.53468897189284\\
-0.093448740680559	3.62029468577783\\
0.13318251344032	3.68691393725669\\
0.350873927242831	3.73614982594868\\
0.55934713845833	3.76956970364108\\
0.758567824726602	3.78864856747597\\
0.948683298050513	3.79473319220206\\
};
\addlegendentry{Аппроксимации}

\addplot [color=mycolor2]
  table[row sep=crcr]{%
2.12132034355964	2.82842712474619\\
2.18582575000323	2.82640660874952\\
2.2459761483502	2.82067996094826\\
2.30316600770785	2.81153894842004\\
2.35833530419819	2.79908319427389\\
2.4121407651283	2.78328504568152\\
2.46505313939319	2.76402339892432\\
2.51741350921551	2.74110157112695\\
2.56946568278775	2.71425682483196\\
2.62137377470791	2.68316571665979\\
2.67322998558975	2.64744786912504\\
2.72505542741334	2.60667010622203\\
2.77679571750944	2.56035271147659\\
2.82831256449113	2.50797963523731\\
2.87937250576692	2.44901464186941\\
2.92963424895255	2.38292549191834\\
2.97863667364875	2.30921809960124\\
3.02579039162557	2.22748193169334\\
3.0703766721051	2.13744642550811\\
3.11155818595542	2.03904567139316\\
3.14840590189045	1.93248505458784\\
3.1799449834646	1.81829951655162\\
3.20521925970046	1.69738980361013\\
3.2233689191928	1.57102231442397\\
3.23371058284854	1.44078163681543\\
3.23580480244767	1.30847305653245\\
3.22949541157146	1.1759834831\\
3.21490900216393	1.04511946165368\\
3.19240995273442	0.917445713323391\\
3.16251366135309	0.794144173493542\\
3.12576385298257	0.675902054710668\\
3.08257536776515	0.562820516214273\\
3.03302813706062	0.454314775018543\\
2.97656426645917	0.348948817128694\\
2.91146921429886	0.244099783305823\\
2.83385680006128	0.135243682019644\\
2.73548220960228	0.0144130439331602\\
2.59871334341581	-0.133168683665202\\
2.38479638365891	-0.336180712113444\\
2.01057175466146	-0.648371525543042\\
1.34210041814469	-1.13791067535239\\
0.375352321166521	-1.75822148716338\\
-0.527448468220416	-2.26262801632552\\
-1.11365847778129	-2.5437869612038\\
-1.4567327767317	-2.68228003389134\\
-1.6691296256022	-2.75259678373195\\
-1.81428800439327	-2.79054335642282\\
-1.92320462744895	-2.81168897715777\\
-2.01132392805712	-2.82302365363059\\
-2.08680248126988	-2.82787464184343\\
-2.15422943553058	-2.82790871121207\\
-2.21634478307291	-2.8239827357917\\
-2.27487422572476	-2.8165264847195\\
-2.33095693947289	-2.80572465930185\\
-2.3853755403757	-2.79160756168836\\
-2.43868469976231	-2.77409782070174\\
-2.49128507004909	-2.75303495626884\\
-2.54346612600745	-2.7281884019911\\
-2.59543032955127	-2.69926454755406\\
-2.64730535029083	-2.66591103001887\\
-2.69914810041563	-2.62772046423944\\
-2.75094277189846	-2.58423541950896\\
-2.80259428647749	-2.53495641420315\\
-2.85391830194335	-2.47935483422118\\
-2.90462903903618	-2.41689283661657\\
-2.95432664759112	-2.34705230341292\\
-3.00248656709011	-2.26937453210355\\
-3.04845424185856	-2.18351130413452\\
-3.0914493816545	-2.08928598065547\\
-3.13058429431929	-1.98676020448289\\
-3.16490008683964	-1.87629788656397\\
-3.19342217916266	-1.75861429449935\\
-3.21523240499265	-1.63479579561046\\
-3.22954956680171	-1.50627700312196\\
-3.23580523135335	-1.37476797309193\\
-3.23369896357047	-1.2421341494958\\
-3.22321880671614	-1.11024302689198\\
-3.20461860896113	-0.980799507713277\\
-3.17835148049538	-0.85519274756288\\
-3.14496439092874	-0.734369602871579\\
-3.10495870881183	-0.618735254776386\\
-3.05861192744495	-0.508062722468019\\
-3.0057323937692	-0.401369642978784\\
-2.94527025004781	-0.296685462718974\\
-2.87460204254734	-0.190562899149911\\
-2.78805571026974	-0.0770301658026245\\
-2.67361421522119	0.0546866392302208\\
-2.50520482475694	0.225105040795738\\
-2.22541867988909	0.474087715816021\\
-1.7209533613293	0.868461990547897\\
-0.880186083660502	1.44500069250654\\
0.110323684998138	2.03883159526995\\
0.859259206598599	2.42737283484743\\
1.30741523004191	2.62511520505649\\
1.57426953542748	2.72304263504847\\
1.74766615441039	2.77435113512036\\
1.87207484559155	2.80265686669549\\
1.96924085114733	2.81832059141065\\
2.0502990590364	2.82612977760851\\
2.12132034355964	2.82842712474619\\
};
\addlegendentry{Сумма Минковского}

\end{axis}

\begin{axis}[%
width=0.798\linewidth,
height=0.578\linewidth,
at={(-0.104\linewidth,-0.064\linewidth)},
scale only axis,
xmin=0,
xmax=1,
ymin=0,
ymax=1,
axis line style={draw=none},
ticks=none,
axis x line*=bottom,
axis y line*=left,
legend style={legend cell align=left, align=left, draw=white!15!black}
]
\end{axis}
\end{tikzpicture}%
        \caption{Эллипсоидальные аппроксимации для 10 направлений.}
\end{figure}
\begin{figure}[b]
        \centering
        % This file was created by matlab2tikz.
%
%The latest updates can be retrieved from
%  http://www.mathworks.com/matlabcentral/fileexchange/22022-matlab2tikz-matlab2tikz
%where you can also make suggestions and rate matlab2tikz.
%
\definecolor{mycolor1}{rgb}{0.00000,0.44700,0.74100}%
\definecolor{mycolor2}{rgb}{0.85000,0.32500,0.09800}%
%
\begin{tikzpicture}

\begin{axis}[%
width=0.618\linewidth,
height=0.487\linewidth,
at={(0\linewidth,0\linewidth)},
scale only axis,
xmin=-10,
xmax=10,
xlabel style={font=\color{white!15!black}},
xlabel={$x_1$},
ymin=-6,
ymax=6,
ylabel style={font=\color{white!15!black}},
ylabel={$x_2$},
axis background/.style={fill=white},
axis x line*=bottom,
axis y line*=left,
xmajorgrids,
ymajorgrids,
legend style={at={(0.03,0.97)}, anchor=north west, legend cell align=left, align=left, draw=white!15!black}
]
\addplot [color=mycolor1, forget plot]
  table[row sep=crcr]{%
1.1711787112841	3.34230777773576\\
1.34444389618673	3.33686722689733\\
1.50648318534277	3.32144701348819\\
1.65806242150429	3.29723890999844\\
1.79995419924852	3.26523166869947\\
1.93290799118999	3.22622794766654\\
2.05763032437368	3.18086276334402\\
2.17477223041259	3.12962150018857\\
2.28492171716205	3.07285634836238\\
2.3885995032216	3.01080060567632\\
2.48625667387058	2.94358064140543\\
2.57827324651315	2.87122553999274\\
2.66495688065443	2.79367457178694\\
2.74654114431786	2.71078271324095\\
2.82318287015103	2.62232448808267\\
2.8949582146523	2.52799644473262\\
2.96185708693385	2.42741864049226\\
3.02377565347494	2.3201355847161\\
3.0805066685131	2.20561721602438\\
3.13172744617751	2.08326066796132\\
3.17698540702193	1.95239382914097\\
3.21568133490354	1.81228204209558\\
3.24705081945703	1.66213971753183\\
3.27014489780722	1.50114915910447\\
3.28381172002077	1.32848945666784\\
3.28668221628109	1.14337881243048\\
3.2771642749897	0.945133918403077\\
3.25345178932588	0.733249674250213\\
3.21355684151227	0.507501133889645\\
3.15537469570401	0.268066487797808\\
3.07679115926015	0.015664562009573\\
2.97583883725379	-0.248307401701393\\
2.85090136682411	-0.521659161442773\\
2.70095216778915	-0.801347823435408\\
2.52579793500058	-1.08350783899399\\
2.32628156602984	-1.3635980584876\\
2.10439191407115	-1.63667268179468\\
1.86323617293213	-1.89775167737354\\
1.60685736899554	-2.14223401489281\\
1.33991789027605	-2.3662777508081\\
1.06730585933085	-2.56707514274059\\
0.793739425298697	-2.74297851870391\\
0.523437853608312	-2.89347215049445\\
0.259902596754298	-3.0190207516634\\
0.00581903236219554	-3.1208447467499\\
-0.2369372555121	-3.20067372983419\\
-0.467220354507	-3.2605178574641\\
-0.684493265340969	-3.30248032509504\\
-0.888694234930048	-3.32861897022119\\
-1.0801092196493	-3.34085455680139\\
-1.25926106809304	-3.34091776153004\\
-1.42681950330378	-3.33032515794113\\
-1.58353177449541	-3.31037507736533\\
-1.73017158467123	-3.2821558685242\\
-1.86750300244094	-3.24656096722397\\
-1.99625599202063	-3.20430689704777\\
-2.11711054093165	-3.15595168681459\\
-2.23068687011842	-3.10191219445658\\
-2.33753972875556	-3.04247952074268\\
-2.43815523449964	-2.97783215102365\\
-2.5329490934295	-2.90804674702345\\
-2.6222653208492	-2.83310668006404\\
-2.70637479428206	-2.75290849531896\\
-2.78547311726505	-2.66726655590999\\
-2.85967737154948	-2.5759161597878\\
-2.92902140016315	-2.47851546990414\\
-2.99344930844609	-2.37464666482809\\
-3.05280690958282	-2.26381681735669\\
-3.10683089311493	-2.14545915779927\\
-3.15513558252945	-2.01893559215682\\
-3.19719730262603	-1.88354163878489\\
-3.23233664163192	-1.73851533219976\\
-3.25969932318791	-1.58305212075492\\
-3.27823706849968	-1.41632833369867\\
-3.28669080350863	-1.23753634540448\\
-3.28357990805066	-1.04593497683649\\
-3.26720291066679	-0.840918687724783\\
-3.23565696131938	-0.622108318461704\\
-3.18688516768166	-0.389463978954261\\
-3.11876167995196	-0.143416533926881\\
-3.02922302993614	0.11499241610004\\
-2.91644915653916	0.383980836121862\\
-2.77908754849337	0.660934801326638\\
-2.61649915361415	0.942389492277687\\
-2.42898799999709	1.22411526429499\\
-2.21796402213223	1.50132757866017\\
-1.9859883878281	1.76901289011117\\
-1.7366683581316	2.02232922526914\\
-1.47440268220478	2.25701269019761\\
-1.20401775520347	2.46971278707818\\
-0.930363383198716	2.65819606075749\\
-0.657943143023997	2.82139307813531\\
-0.390637102663735	2.95930311865273\\
-0.131543811712393	3.07279931782965\\
0.117062449314586	3.1633871238911\\
0.353684227850852	3.23296276733506\\
0.577496523037587	3.28360333410188\\
0.788216624026908	3.3174035782658\\
0.985971690059736	3.33636162404406\\
1.1711787112841	3.34230777773576\\
};
\addplot [color=mycolor1, forget plot]
  table[row sep=crcr]{%
1.21926508344108	3.27334196367036\\
1.39130630412839	3.26794365173349\\
1.55153598090393	3.25269915059364\\
1.70081822737307	3.22886110922975\\
1.84001535490823	3.19746467488806\\
1.96995663540462	3.15934744514424\\
2.09141838442226	3.11517076291109\\
2.20511208961792	3.06544020144585\\
2.31167798429415	3.01052405216791\\
2.41168207975524	2.95066926128589\\
2.50561517604179	2.88601465321008\\
2.59389276130835	2.81660150697173\\
2.67685499694353	2.74238167588435\\
2.75476618649088	2.66322350375262\\
2.82781326027155	2.57891582390079\\
2.89610289157071	2.489170352485\\
2.95965690978143	2.39362282222959\\
3.0184057055742	2.29183326184168\\
3.07217934834312	2.18328592468469\\
3.12069617535093	2.0673895238837\\
3.16354868972977	1.94347865783223\\
3.20018675510097	1.81081762911891\\
3.22989834610506	1.66860828812922\\
3.25178857180767	1.51600407734397\\
3.26475841566826	1.35213309692555\\
3.26748572490109	1.17613369082119\\
3.25841251405898	0.987206612265718\\
3.23574463765734	0.784687977211719\\
3.19747218836968	0.568146474359487\\
3.14142113656216	0.337505988781863\\
3.06534780932005	0.0931901220106009\\
2.96708629627689	-0.163722532576174\\
2.84475284042693	-0.431353836986617\\
2.69699912472874	-0.706924876445341\\
2.52328828199352	-0.986738253877743\\
2.32414719016614	-1.26628199746165\\
2.1013342676447	-1.54047573935053\\
1.85786382933463	-1.80404664367384\\
1.59785273337257	-2.05198218219746\\
1.32619884624731	-2.27997573381088\\
1.04814809460687	-2.48477539672332\\
0.768837071389945	-2.66437165107218\\
0.492898365330516	-2.81800515148501\\
0.224188427726118	-2.94602190329025\\
-0.0343421486742361	-3.04963199342895\\
-0.280651418753673	-3.13063387286929\\
-0.513517918101939	-3.19115408703827\\
-0.732402138426053	-3.23343245666966\\
-0.937291976094646	-3.25966380617091\\
-1.12855566644943	-3.27189398375029\\
-1.30681506667752	-3.27196083427941\\
-1.47284399601866	-3.26146857518149\\
-1.62749116596371	-3.24178474838236\\
-1.77162454720074	-3.21405096074922\\
-1.9060930438003	-3.1792009422838\\
-2.03170136752072	-3.13798151469425\\
-2.14919451031851	-3.09097368125037\\
-2.25924887975171	-3.03861221534132\\
-2.36246781585791	-2.9812029127911\\
-2.45937977071431	-2.91893717481279\\
-2.55043787954686	-2.85190389003105\\
-2.63601998899356	-2.78009875378493\\
-2.71642844964106	-2.70343125192797\\
-2.79188914508071	-2.62172958112562\\
-2.86254933659929	-2.53474380432833\\
-2.92847396752118	-2.44214756807608\\
-2.98964010899711	-2.34353875337005\\
-3.04592925441029	-2.23843950866587\\
-3.09711719913452	-2.12629623737968\\
-3.14286129724365	-2.00648030014085\\
-3.18268499588943	-1.87829046258472\\
-3.21595975211415	-1.74095849136247\\
-3.24188479216075	-1.59365978768918\\
-3.25946575421678	-1.43553154684198\\
-3.26749415178163	-1.26570160623257\\
-3.26453089777461	-1.0833317902911\\
-3.24889889583809	-0.887679962191258\\
-3.21869188468844	-0.678184770435691\\
-3.17180903470924	-0.454575648787598\\
-3.10602656951161	-0.217007231912446\\
-3.01911769523831	0.0337887596613096\\
-2.90902859638195	0.296350206020716\\
-2.77410929927924	0.568356995772149\\
-2.6133828822338	0.846566357802756\\
-2.42681658880744	1.12685188938682\\
-2.21553984909621	1.4043777730154\\
-1.98194673587055	1.67391391702028\\
-1.72963332752344	1.93025930170482\\
-1.4631559553519	2.16870275608031\\
-1.18764436113277	2.38543046078226\\
-0.908344887791558	2.57779963933086\\
-0.630184750931516	2.74443551343009\\
-0.357433580063127	2.88515703813491\\
-0.0935024920494292	3.00077594658033\\
0.15911854673327	3.09283095590222\\
0.398812914783312	3.163314718129\\
0.624719128954104	3.21443380598977\\
0.836580651685356	3.24842176843127\\
1.03459285132093	3.26740886591368\\
1.21926508344108	3.27334196367036\\
};
\addplot [color=mycolor1, forget plot]
  table[row sep=crcr]{%
1.27235524758866	3.20606889606907\\
1.44317107064595	3.20071319568392\\
1.60153774787983	3.18564973372891\\
1.74843510276585	3.16219597058339\\
1.88482914833813	3.13143492326719\\
2.01163976581522	3.09423891990115\\
2.12972112934266	3.05129435069619\\
2.23985098586104	3.00312508327947\\
2.34272577688468	2.95011332159164\\
2.43895935773047	2.89251739452003\\
2.52908368509872	2.8304863814774\\
2.61355030737859	2.76407171387989\\
2.69273182444275	2.69323600549816\\
2.76692271100404	2.61785941119026\\
2.83633904449915	2.53774382694731\\
2.9011167660551	2.45261524824953\\
2.96130814896022	2.36212461590775\\
3.01687616843364	2.26584751343373\\
3.06768647413494	2.16328315185123\\
3.11349667956523	2.05385320276079\\
3.15394272225835	1.93690123748885\\
3.18852214664256	1.81169382176359\\
3.21657436300479	1.67742472593303\\
3.23725830595004	1.53322426152362\\
3.24952854266518	1.37817645273807\\
3.25211187660099	1.21134756681076\\
3.24348797643739	1.03183035661999\\
3.22187962263265	0.838808978297762\\
3.18526077526704	0.631649497785332\\
3.13139352539076	0.410019487872334\\
3.05790730118758	0.174036479049343\\
2.96243394314779	-0.0755620455059427\\
2.84280816668981	-0.337245310360568\\
2.69733198476799	-0.608544401674519\\
2.52508258389448	-0.885980316488405\\
2.32621826608086	-1.16511378557182\\
2.10221443243537	-1.44075380334538\\
1.85595416779634	-1.70732887684435\\
1.59161769905144	-1.95937674660199\\
1.31436302269743	-2.19206276802241\\
1.02985073295835	-2.40161735788381\\
0.743712717332933	-2.58560229945124\\
0.46107506075	-2.74296710880307\\
0.186217675282633	-2.87391600087286\\
-0.0775960985361585	-2.97964801275228\\
-0.328133772499326	-3.06204567327999\\
-0.564094671509743	-3.12337551626293\\
-0.784948102500155	-3.16603958225219\\
-0.990751904935546	-3.19239302335658\\
-1.18198074976573	-3.20462570620811\\
-1.35937986129559	-3.20469658668081\\
-1.52384947555703	-3.19430684199975\\
-1.67635893937152	-3.17489872235242\\
-1.81788618426142	-3.14766969123728\\
-1.94937729908481	-3.11359431862355\\
-2.07172112440674	-3.07344891393212\\
-2.18573453476242	-3.02783582028248\\
-2.29215496469956	-2.9772056520116\\
-2.39163756765905	-2.92187665091016\\
-2.48475509068474	-2.86205088696231\\
-2.57199908552444	-2.79782734529923\\
-2.65378147146904	-2.72921210646848\\
-2.73043574186991	-2.65612590224109\\
-2.80221729049535	-2.57840935563713\\
-2.86930244872764	-2.49582622010879\\
-2.93178588931728	-2.40806493900825\\
-2.98967608320614	-2.31473886823903\\
-3.04288850753444	-2.21538555650091\\
-3.09123631078617	-2.10946557375074\\
-3.13441816403804	-1.99636153685547\\
-3.17200309148303	-1.87537822273494\\
-3.20341221640447	-1.74574500735886\\
-3.22789763501203	-1.60662234752231\\
-3.24451911637134	-1.45711464736294\\
-3.25212012271858	-1.29629261534778\\
-3.24930586973196	-1.12322905938835\\
-3.23442791317944	-0.937052827376804\\
-3.20558210147849	-0.737025951103928\\
-3.16062953150577	-0.522648426972396\\
-3.09725286895748	-0.293792623174874\\
-3.01306191087989	-0.0508640093726965\\
-2.90576066218136	0.205024084380883\\
-2.77338096163089	0.471886516761707\\
-2.61457267711254	0.746751816909297\\
-2.42891789191419	1.0256452840926\\
-2.21721135786237	1.30371511604229\\
-1.98163270338355	1.57552461067301\\
-1.72574098366183	1.83549137118179\\
-1.45425689642249	2.0784046669254\\
-1.17265530960846	2.29991677896501\\
-0.886647911507271	2.49690345800275\\
-0.601666181623873	2.66762625184561\\
-0.322445205446415	2.81168834633155\\
-0.0527673783806232	2.92982878700886\\
0.204625871021466	3.02362775785022\\
0.447985415526008	3.09519465143329\\
0.676417888047168	3.14689079713794\\
0.889709449657395	3.1811133975024\\
1.08814645535659	3.20014614291162\\
1.27235524758866	3.20606889606907\\
};
\addplot [color=mycolor1, forget plot]
  table[row sep=crcr]{%
1.331122526514	3.1411519504109\\
1.50070265021491	3.13583953954657\\
1.65714137817264	3.12096355391375\\
1.801554491142	3.09791011124431\\
1.93502746499673	3.06781117661229\\
2.05858258751093	3.03157307872426\\
2.1731603825899	2.98990544455629\\
2.27961069804984	2.94334806244517\\
2.37868996212446	2.89229446064749\\
2.47106207995261	2.83701177144793\\
2.55730118991524	2.7776568956931\\
2.63789504769877	2.71428921024775\\
2.71324818860979	2.64688015759524\\
2.78368427326967	2.57532008083767\\
2.849447181013	2.49942265679744\\
2.9107005052992	2.41892725977025\\
2.96752514659502	2.33349957604737\\
3.01991470673442	2.24273079836124\\
3.06776837901516	2.14613577319474\\
3.11088101529588	2.04315056769814\\
3.14893005419316	1.93313008597277\\
3.18145904053263	1.81534662115847\\
3.20785759670683	1.68899060993897\\
3.22733798348076	1.5531753925286\\
3.23890890115111	1.40694850199169\\
3.24134805532164	1.24931291790325\\
3.23317639837383	1.07926277119951\\
3.2126390158541	0.895839016149664\\
3.177700448064	0.69821122718357\\
3.1260656987857	0.48579126058317\\
3.05524169695212	0.258381986555632\\
2.96265615170331	0.0163582778932624\\
2.84584911120226	-0.239133300220675\\
2.70274374574876	-0.505985173879308\\
2.53198394173171	-0.780996403060881\\
2.33329738953371	-1.05985624741079\\
2.10781090040156	-1.33729898614381\\
1.85822480693181	-1.60745595567134\\
1.58876376753239	-1.86437606022391\\
1.30487103163859	-2.10262351755484\\
1.01268938097015	-2.31782176529302\\
0.718441030482152	-2.50702035799616\\
0.427846180930232	-2.66881735136476\\
0.145694042649061	-2.80324567414949\\
-0.124380790790276	-2.91149239657131\\
-0.379927228454498	-2.99554341560088\\
-0.619563635809872	-3.05783481070723\\
-0.842784978801955	-3.10096237588935\\
-1.04974635506385	-3.12746975963674\\
-1.24106000163502	-3.13971312950987\\
-1.41762489959441	-3.13978845934957\\
-1.58049474935395	-3.12950408936179\\
-1.73078211578909	-3.11038263397954\\
-1.86959284659487	-3.0836797413193\\
-1.99798391664057	-3.0504108901298\\
-2.11693835508858	-3.01138053033187\\
-2.22735200674953	-2.96721019972459\\
-2.33002807564449	-2.9183638378905\\
-2.42567646575971	-2.86516952714163\\
-2.51491579139968	-2.80783748683269\\
-2.59827657406231	-2.74647447073001\\
-2.67620460331178	-2.68109487045736\\
-2.74906375329484	-2.61162888258028\\
-2.81713774976441	-2.53792809983284\\
-2.88063050418508	-2.45976886910422\\
-2.93966469482067	-2.37685374065141\\
-2.99427829770052	-2.28881132968064\\
-3.04441876799606	-2.19519493611852\\
-3.08993455892186	-2.09548033507462\\
-3.13056365740489	-1.98906327620029\\
-3.1659188360687	-1.87525743658402\\
-3.19546940308262	-1.75329388623641\\
-3.21851942578628	-1.6223235794633\\
-3.23418278584247	-1.48142501135315\\
-3.24135609946731	-1.32961999626986\\
-3.23869164592522	-1.16590151836407\\
-3.22457415197589	-0.989278675191041\\
-3.19710771688668	-0.798844633232678\\
-3.15412234119223	-0.593873733391723\\
-3.09321312554347	-0.373952564769022\\
-3.01182828769036	-0.139145727453714\\
-2.90742279527531	0.109811261632613\\
-2.77768964509709	0.371312723440716\\
-2.62086721620327	0.642715147198648\\
-2.43609684116487	0.920255137140879\\
-2.22377291482778	1.19911285221477\\
-1.9857996409858	1.4736651303669\\
-1.72566175604194	1.73792940542773\\
-1.44824671829608	1.98613763849639\\
-1.15942195970072	2.2133243081796\\
-0.865448289650424	2.41579463270279\\
-0.572362115157663	2.59137293020573\\
-0.285459358621382	2.73940143022293\\
-0.0089668275124522	2.8605320594217\\
0.254079449410444	2.95639689740252\\
0.50178376295928	3.02924769275413\\
0.73322946004012	3.08163193886222\\
0.948268430835706	3.11614082010529\\
1.14730794379744	3.13523684905876\\
1.331122526514	3.1411519504109\\
};
\addplot [color=mycolor1, forget plot]
  table[row sep=crcr]{%
1.3964102868975	3.07930591740507\\
1.56473037248086	3.07403794379094\\
1.71915971789631	3.05935744921679\\
1.86097350375265	3.03672287758242\\
1.99139432743895	3.00731572039262\\
2.1115596313916	2.9720749730111\\
2.22250508572583	2.93173110760911\\
2.32515837474101	2.88683696403915\\
2.42033934512417	2.83779441843637\\
2.50876368745401	2.78487654478706\\
2.59104822992295	2.72824544210295\\
2.66771656628559	2.66796611120257\\
2.73920417543268	2.60401683410806\\
2.8058624705604	2.53629650265557\\
2.86796138470457	2.46462930281241\\
2.92569018902551	2.38876711316769\\
2.97915627449035	2.30838993674501\\
3.0283816240354	2.2231046670328\\
3.07329667483836	2.13244250358555\\
3.11373123222949	2.03585539346854\\
3.14940206395745	1.93271200046933\\
3.17989679896267	1.82229391931966\\
3.20465381435363	1.70379319033865\\
3.22293797458151	1.57631267238219\\
3.2338124753162	1.4388715448039\\
3.23610777024232	1.29041917122538\\
3.22838979925308	1.12986177408304\\
3.20893170831586	0.956107756545554\\
3.17569616377437	0.768138804989481\\
3.12633929237158	0.565114530656926\\
3.05825190772653	0.346517306048897\\
2.96865788840053	0.112339502298826\\
2.85479090633259	-0.136694414172483\\
2.71416515318386	-0.398894887802234\\
2.54493847995357	-0.671409304753301\\
2.34633470428385	-0.950126228446477\\
2.11905007755647	-1.22975688596338\\
1.86553297853278	-1.50414781259021\\
1.59002111748132	-1.76682066303184\\
1.29826723729962	-2.01165403512721\\
0.996976978743987	-2.23355486779611\\
0.693081911192092	-2.42895471339356\\
0.39302449262342	-2.59602245879899\\
0.10221242031689	-2.73458150354873\\
-0.175275587441964	-2.84580580466324\\
-0.436737284630285	-2.93180948107383\\
-0.680711669681773	-2.99523570280546\\
-0.90674417217064	-3.03891333463364\\
-1.11512406441372	-3.0656088453726\\
-1.30664144946907	-3.07787111261669\\
-1.48238709584989	-3.07795134920049\\
-1.64360097670177	-3.06777622825643\\
-1.79156543558501	-3.04895448590184\\
-1.92753469255854	-3.02280191024239\\
-2.05269169243189	-2.99037438471845\\
-2.16812430727995	-2.95250254880701\\
-2.27481451443136	-2.90982444674297\\
-2.37363578578764	-2.862814384836\\
-2.46535529269494	-2.81180734627586\\
-2.5506385879597	-2.75701894637208\\
-2.63005519501394	-2.69856122981593\\
-2.70408406639507	-2.6364547413632\\
-2.77311822563175	-2.57063732609075\\
-2.83746812660903	-2.50097008809051\\
-2.89736339022059	-2.42724088988245\\
-2.95295263753188	-2.34916572943973\\
-3.00430115212401	-2.26638830150595\\
-3.05138608715343	-2.17847804645976\\
-3.09408889814955	-2.08492702572269\\
-3.13218464480151	-1.98514605333694\\
-3.16532778277964	-1.87846068021922\\
-3.19303408855744	-1.76410789964738\\
-3.21465847136983	-1.64123485723529\\
-3.22936869755953	-1.50890145057942\\
-3.23611559058606	-1.36608953760508\\
-3.23360122780567	-1.21172256498737\\
-3.22024823272812	-1.04470074878479\\
-3.19417568159924	-0.863958330476281\\
-3.15319057074906	-0.668550491360247\\
-3.09480814848275	-0.457777436782653\\
-3.01631904392544	-0.231350619837756\\
-2.91492429939803	0.0104007841530797\\
-2.78795789248472	0.266298052535966\\
-2.63320550966211	0.534090230068579\\
-2.44930392305496	0.81029787181832\\
-2.23616740772478	1.0901969999926\\
-1.99534603198641	1.36801221596138\\
-1.73019696643741	1.63734829722721\\
-1.44576953465053	1.89181673044\\
-1.14837734553415	2.12573379650668\\
-0.844933408739004	2.33472294072841\\
-0.542206513864935	2.51607716502315\\
-0.246175073832186	2.66881927116423\\
0.0383973476672991	2.79349537142056\\
0.308127665128874	2.89180296092848\\
0.560959304573922	2.96616893137416\\
0.795966987499965	3.01936648760927\\
1.01310046780864	3.05421819894996\\
1.21292743436782	3.07339593495123\\
1.39641028689749	3.07930591740507\\
};
\addplot [color=mycolor1, forget plot]
  table[row sep=crcr]{%
1.46927735143059	3.02133831480419\\
1.63629310035011	3.01611658260144\\
1.7886098202728	3.00164167558635\\
1.9276891077562	2.97944772464369\\
2.05491029732378	2.95076569259784\\
2.17153957212107	2.91656520547484\\
2.27871666322874	2.8775945552278\\
2.37745251949422	2.8344162564902\\
2.46863330682164	2.78743718747514\\
2.55302761411155	2.73693325892181\\
2.63129483361326	2.68306900537496\\
2.70399342855486	2.62591267004484\\
2.77158828747314	2.56544738150251\\
2.83445666632126	2.5015789733937\\
2.89289239235212	2.43414092191058\\
2.94710808822543	2.36289679566997\\
2.99723519856857	2.28754054406744\\
3.04332158332015	2.2076949033788\\
3.08532639615547	2.12290818421282\\
3.1231119034702	2.03264973086929\\
3.15643183233097	1.93630442907388\\
3.18491578251768	1.83316680739732\\
3.20804922766571	1.72243556373827\\
3.22514871272181	1.60320979852764\\
3.23533211017806	1.47448891063472\\
3.23748435336648	1.33517907761217\\
3.23022011209626	1.18411055554444\\
3.21184667676275	1.02007169815298\\
3.18033318732	0.841867471807365\\
3.13329656201969	0.648411901057839\\
3.06802005852571	0.43886436819431\\
2.9815266205418	0.212817298608521\\
2.87073386701779	-0.0294650388365179\\
2.73271642082416	-0.286772864596821\\
2.56508773964061	-0.556683859649172\\
2.36648135624935	-0.835374737329632\\
2.13706031610444	-1.11760600143702\\
1.87892800807281	-1.39696779352388\\
1.59628548819868	-1.66641963371706\\
1.29521584343495	-1.91905712943754\\
0.98308441078536	-2.14893516111894\\
0.667683177910749	-2.35173150036812\\
0.356342600548117	-2.52508420102095\\
0.0552298920758284	-2.6685567979865\\
-0.231040238055092	-2.78330877698785\\
-0.499476806935808	-2.87161508239866\\
-0.748546728386371	-2.93637438496716\\
-0.977882288979001	-2.98069824932831\\
-1.18795708828636	-3.00761818760242\\
-1.37979150940508	-3.01990728770175\\
-1.55471582553872	-3.01999292769046\\
-1.71419607931498	-3.00993230317797\\
-1.85971553954364	-2.99142602962007\\
-1.99269995310453	-2.96585147839687\\
-2.11447465948285	-2.93430374220686\\
-2.22624346296074	-2.89763701989019\\
-2.32908151045466	-2.85650259944695\\
-2.42393660051112	-2.81138175736593\\
-2.51163509895577	-2.76261313414686\\
-2.59288993246338	-2.71041479938398\\
-2.66830903791877	-2.65490151430437\\
-2.73840325102495	-2.59609778961128\\
-2.80359300360421	-2.53394731925019\\
-2.86421343036985	-2.46831930479252\\
-2.92051761046401	-2.39901210446869\\
-2.9726777203323	-2.32575456506696\\
-3.02078387548822	-2.24820533607854\\
-3.06484040527515	-2.16595043299995\\
-3.10475924890094	-2.07849932066814\\
-3.14035009427886	-1.98527984148405\\
-3.17130681823211	-1.88563243717909\\
-3.19718975024394	-1.77880433519281\\
-3.21740331018235	-1.66394473219286\\
-3.23116872653173	-1.54010256139314\\
-3.23749192767323	-1.40622924192809\\
-3.23512747185839	-1.26118994304902\\
-3.22254076830271	-1.10378839066771\\
-3.19787313688991	-0.932812045608432\\
-3.1589177725706	-0.747106330769343\\
-3.10311960877118	-0.545687819133067\\
-3.02761813373429	-0.327905611498936\\
-2.92935806841843	-0.0936553921639545\\
-2.80529528362486	0.156360933207452\\
-2.6527188844722	0.420357817462845\\
-2.46968808889038	0.695227527519055\\
-2.25554013643435	0.976425937428937\\
-2.01136913889677	1.25807902966462\\
-1.74032911995783	1.53337706969685\\
-1.44761450798646	1.7952432948003\\
-1.14004500386212	2.03715506205079\\
-0.825313975647545	2.25391343942017\\
-0.511086370392609	2.44215796957082\\
-0.204180704567995	2.6005151780265\\
0.089984434113084	2.72940082914642\\
0.367614630809603	2.83059573675472\\
0.626479062175471	2.90674459359535\\
0.865667727030579	2.96089679720825\\
1.08527323667551	2.99615293675265\\
1.28607612949814	3.0154312519322\\
1.46927735143059	3.02133831480419\\
};
\addplot [color=mycolor1, forget plot]
  table[row sep=crcr]{%
1.55106012395173	2.96820108237492\\
1.71670039509252	2.96302826671365\\
1.866773650437	2.94877167105386\\
2.00295881409983	2.92704398092938\\
2.12681348158948	2.89912477506173\\
2.23974677558376	2.86601146149039\\
2.34301123972137	2.82846644198549\\
2.43770593604595	2.78705801960205\\
2.52478548784315	2.74219439080212\\
2.60507169099129	2.69415100565756\\
2.67926560970047	2.64309199157726\\
2.74795891918622	2.5890864527319\\
2.81164378731278	2.53212042352181\\
2.87072089988808	2.47210515402108\\
2.92550540208963	2.40888228458754\\
2.97623060040506	2.34222635006831\\
3.02304927747868	2.27184495383945\\
3.06603243671452	2.19737687562467\\
3.10516522738603	2.11838833098142\\
3.14033971357704	2.03436759302091\\
3.17134405045729	1.94471823184124\\
3.19784753184586	1.8487513458686\\
3.21938089574431	1.74567738509342\\
3.23531125925981	1.63459854816662\\
3.24481117124365	1.5145033422472\\
3.24682163905332	1.38426581574507\\
3.2400097972802	1.24265331190513\\
3.22272343233275	1.0883484320068\\
3.19294725989111	0.919993240239861\\
3.14827014535604	0.736266352737939\\
3.0858787298984	0.536005698720147\\
3.00260100298442	0.318389812049431\\
2.89503167884312	0.0831856446549773\\
2.75977558920631	-0.168942950911744\\
2.59383787432449	-0.436098516778841\\
2.39516019521871	-0.714856398931109\\
2.16324375154413	-1.00012576884616\\
1.89972165804939	-1.28529226677395\\
1.60868151094332	-1.56272681971187\\
1.29654972154919	-1.82463070149011\\
0.971467852367241	-2.06403750195299\\
0.642282701040046	-2.27569441529627\\
0.317430689649746	-2.4565734498626\\
0.00402370353224809	-2.60591096109272\\
-0.292670312621192	-2.72485050727559\\
-0.569328782256929	-2.81587147438732\\
-0.824363369645803	-2.88219145532937\\
-1.05754664046762	-2.92726816036025\\
-1.26960505851823	-2.95445057042329\\
-1.46185776232636	-2.96677376570594\\
-1.63593450851817	-2.96686534407497\\
-1.79357561533683	-2.95692623777213\\
-1.93650161604717	-2.9387545377024\\
-2.06633574417822	-2.9137899458279\\
-2.18456334326995	-2.88316473659085\\
-2.29251540962804	-2.84775327892353\\
-2.3913668893472	-2.80821624292687\\
-2.48214327273579	-2.76503805721257\\
-2.56573125059349	-2.7185575147283\\
-2.6428907665372	-2.66899206563575\\
-2.71426685168592	-2.61645657716661\\
-2.78040030166351	-2.5609773681715\\
-2.84173666627119	-2.50250225126656\\
-2.89863325499027	-2.44090720078849\\
-2.95136397640314	-2.3760001438983\\
-3.00012186641745	-2.30752226257338\\
-3.04501914441794	-2.23514710521278\\
-3.08608458434066	-2.15847774454589\\
-3.12325790984976	-2.07704219053697\\
-3.15638082793169	-1.99028728365994\\
-3.18518421352558	-1.89757137239653\\
-3.20927086577889	-1.79815624601589\\
-3.22809320374143	-1.69119909028635\\
-3.24092530881614	-1.57574571829855\\
-3.24682894495659	-1.45072708034643\\
-3.24461374770692	-1.31496217351149\\
-3.23279290849798	-1.16717205316684\\
-3.20953773849238	-1.0060117507856\\
-3.17263792532288	-0.830129432068425\\
-3.11947955211359	-0.638264643741304\\
-3.04706020690436	-0.429398878067323\\
-2.95206903691371	-0.202969729290941\\
-2.83106667794031	0.0408488926212685\\
-2.68079965641057	0.300817909394284\\
-2.49866665674673	0.574306299043266\\
-2.2833102427322	0.857058927982916\\
-2.03523660271512	1.14318345421905\\
-1.75728959879587	1.42547108532208\\
-1.45477290398947	1.69608663675106\\
-1.13507778488407	1.94752294706826\\
-0.806838945162423	2.17357884107625\\
-0.478831266741026	2.37007933016134\\
-0.158921038277483	2.53515197102043\\
0.14666436956898	2.66904955862832\\
0.433639963808184	2.77366049616377\\
0.699590316977137	2.85190376286616\\
0.943658792568718	2.90717031589308\\
1.16614417848768	2.94289755467673\\
1.36810982104501	2.9622952925113\\
1.55106012395173	2.96820108237492\\
};
\addplot [color=mycolor1, forget plot]
  table[row sep=crcr]{%
1.64345852363055	2.92105776339129\\
1.80761732204454	2.9159376568551\\
1.95528246259752	2.90191530665662\\
2.08838519009964	2.88068407505593\\
2.20868456958566	2.85357029991398\\
2.31774709165881	2.82159539265171\\
2.41694629827407	2.78553142969586\\
2.50747327914833	2.74594812184625\\
2.59035220798831	2.70325101153669\\
2.66645738332475	2.65771166454735\\
2.7365297390966	2.60949094897271\\
2.80119172506743	2.55865651553041\\
2.86096001123987	2.50519547473524\\
2.91625577678492	2.44902309758856\\
2.96741249260937	2.38998819246268\\
3.01468115556098	2.32787565263928\\
3.05823291728477	2.26240653502024\\
3.09815899310691	2.19323592423288\\
3.13446764802029	2.11994876026648\\
3.16707794468208	2.04205376673256\\
3.19580980735125	1.95897562065497\\
3.22036981287264	1.87004557129875\\
3.24033197908955	1.77449087509657\\
3.25511271117103	1.67142371408152\\
3.26393904345144	1.55983077873778\\
3.26580948222163	1.4385655265798\\
3.25944729491222	1.30634641319336\\
3.24324730307265	1.16176629450234\\
3.21521958208466	1.00332085639153\\
3.1729375897116	0.829467334288683\\
3.11350486983964	0.638728528193248\\
3.03356407943737	0.429859907970045\\
2.92938400748147	0.202096599130512\\
2.79707107094353	-0.0445127188891665\\
2.63295314192722	-0.308703031259661\\
2.43416129871963	-0.58758430600421\\
2.19937376746793	-0.876349005913475\\
1.92958437026846	-1.1682643309048\\
1.62865018731794	-1.45510280891482\\
1.30333820181883	-1.72804614070483\\
0.962706283169578	-1.97889373554604\\
0.616909840592004	-2.20122852165362\\
0.275783131818425	-2.39117340478182\\
-0.0523698660120398	-2.54754598280501\\
-0.361476459509195	-2.67147265521413\\
-0.647834840576828	-2.76569685549902\\
-0.909834199478792	-2.8338396057494\\
-1.14746606281625	-2.8797869376726\\
-1.36180400226045	-2.90727098253691\\
-1.55455572035072	-2.91963433147983\\
-1.72772588709667	-2.91973241682908\\
-1.88338728987833	-2.90992407129012\\
-2.02353978634332	-2.89211005318921\\
-2.15003280356409	-2.86779221172766\\
-2.26453015630305	-2.83813698352557\\
-2.36850105812257	-2.80403468242865\\
-2.46322609254644	-2.76615089014963\\
-2.54981079323019	-2.72496898281352\\
-2.62920226484629	-2.68082419945641\\
-2.7022061438885	-2.63393023468613\\
-2.76950238990311	-2.58439948452057\\
-2.83165912241808	-2.53225801075129\\
-2.88914413537482	-2.47745613762484\\
-2.94233393935166	-2.41987541944139\\
-2.991520274945	-2.35933254995108\\
-3.03691405420067	-2.29558063773091\\
-3.07864664870852	-2.22830815131749\\
-3.11676836891342	-2.15713574628113\\
-3.15124387830048	-2.08161112701245\\
-3.18194416388457	-2.00120207572529\\
-3.20863454636729	-1.915287813493\\
-3.23095806911326	-1.82314896648223\\
-3.24841347485075	-1.72395663315592\\
-3.26032690319308	-1.61676144428311\\
-3.2658164975705	-1.50048416440397\\
-3.2637494350632	-1.37391042204906\\
-3.25269172194939	-1.23569372924949\\
-3.23085280738982	-1.08437321816084\\
-3.19603021020376	-0.918415572441364\\
-3.14556463730224	-0.736294298285524\\
-3.0763241683432	-0.536622987018219\\
-2.984747049609	-0.31836059341864\\
-2.86698469390163	-0.0811021282462965\\
-2.71919399936644	0.174548415407669\\
-2.53801971706134	0.446560906457461\\
-2.32126736186821	0.731109536982981\\
-2.06868395448563	1.02240047830783\\
-1.78265173563995	1.31286886198096\\
-1.4685171815477	1.5938532425014\\
-1.13431162016923	1.85668637703656\\
-0.789815291758204	2.09393194184063\\
-0.445196810880552	2.3003844084033\\
-0.109648904470953	2.47353264858748\\
0.209589152140003	2.61342251208492\\
0.507643648353963	2.72208375164278\\
0.781911203897705	2.80278585825734\\
1.03164857637783	2.85934727107597\\
1.25745038849592	2.89561713910808\\
1.46075663711784	2.91515243254314\\
1.64345852363055	2.92105776339129\\
};
\addplot [color=mycolor1, forget plot]
  table[row sep=crcr]{%
1.74865638965985	2.88137384176368\\
1.9111836710517	2.87631163289344\\
2.05623578848918	2.8625432733052\\
2.18603530742935	2.84184384607306\\
2.30256716606395	2.81558335823543\\
2.40756928437339	2.78480240242788\\
2.50254327140952	2.75027755034177\\
2.58877477830147	2.71257508773336\\
2.66735724515472	2.67209371664647\\
2.73921552628005	2.62909765426629\\
2.80512756595767	2.58374173106424\\
2.86574328648835	2.53608997072186\\
2.92160039629944	2.48612890195269\\
2.97313709926905	2.43377659779317\\
3.0207017959453	2.37888820127053\\
3.06455987921545	2.32125849205571\\
3.10489767976282	2.26062187935964\\
3.14182353134469	2.19665007013035\\
3.17536581272197	2.12894755657744\\
3.20546768582882	2.05704499373496\\
3.23197808934203	1.98039050162045\\
3.25463836423213	1.89833894078851\\
3.27306368856977	1.81013929990314\\
3.28671829938547	1.71492054271635\\
3.29488331912033	1.61167665999203\\
3.29661596543324	1.49925236882212\\
3.29069916429219	1.37633205880754\\
3.27558139366417	1.24143642297809\\
3.24930844435586	1.09293399751025\\
3.20945245739726	0.929078802553234\\
3.15305014452637	0.748090410689812\\
3.07657268871799	0.548298317047832\\
2.97596499191847	0.328376066452439\\
2.84680977757094	0.0876869626864989\\
2.68468495329439	-0.17325657939976\\
2.48577362462204	-0.452264497626377\\
2.24772972631639	-0.744992585231754\\
1.97067943599865	-1.04472501879821\\
1.65807220993474	-1.34265634595364\\
1.31698343105269	-1.62881221959049\\
0.957554514883364	-1.89348839583847\\
0.591585930754306	-2.12878970482027\\
0.230708359684339	-2.32973727022693\\
-0.115245097273801	-2.49460300242127\\
-0.439198726834851	-2.62449590283836\\
-0.73702122592129	-2.72250681712574\\
-1.0071386753611	-2.7927748429065\\
-1.24987806314248	-2.83972219698163\\
-1.46679045121205	-2.86754714735139\\
-1.66009022809612	-2.87995479744129\\
-1.83225170660669	-2.88005998154867\\
-1.98574995545417	-2.87039434462988\\
-2.12291225381591	-2.85296575291794\\
-2.24584534304923	-2.82933679992632\\
-2.35641026393456	-2.80070390154217\\
-2.45622465215348	-2.76796819319801\\
-2.54667927319293	-2.7317951045581\\
-2.62896065345727	-2.69266242953424\\
-2.70407508028692	-2.6508980274472\\
-2.77287140616133	-2.60670872452633\\
-2.8360613932267	-2.56020198090403\\
-2.89423707595181	-2.51140169625282\\
-2.9478850124543	-2.46025927635858\\
-2.99739747596487	-2.40666083432898\\
-3.04308069233051	-2.3504311788191\\
-3.08516020825859	-2.29133505490838\\
-3.12378340691881	-2.22907595079856\\
-3.15901908747461	-2.16329266303052\\
-3.19085389967043	-2.0935537234282\\
-3.21918527572757	-2.01934973515528\\
-3.24381033001434	-1.94008365202977\\
-3.26441000484119	-1.85505908316613\\
-3.28052753819491	-1.76346684744097\\
-3.29154014235314	-1.66437029480382\\
-3.29662266873162	-1.55669044235931\\
-3.29470210917163	-1.4391928762441\\
-3.28440226177932	-1.31047983674861\\
-3.26397914640617	-1.16899318217089\\
-3.23125040615276	-1.01303728294905\\
-3.18352689318838	-0.840835481318406\\
-3.11756307229887	-0.650639262158281\\
-3.02955578265117	-0.440914290992723\\
-2.91523790632757	-0.210628333221629\\
-2.77013032758312	0.0403443027574127\\
-2.59002029256205	0.310719099018591\\
-2.37170493501425	0.597277998955034\\
-2.11394990123552	0.894492241963369\\
-1.81846128678003	1.19452690062365\\
-1.49051303930177	1.48783794592996\\
-1.13884197202703	1.76438739211616\\
-0.774635244305867	2.01519847780868\\
-0.409838890092488	2.23374014663527\\
-0.055353551683624	2.4166687307997\\
0.280296632940744	2.5637629907927\\
0.591527576299578	2.67724232125039\\
0.875558910164217	2.76083170511072\\
1.13185505571184	2.8188917889414\\
1.36143427945912	2.85578004516305\\
1.56623969593983	2.87546934518756\\
1.74865638965985	2.88137384176368\\
};
\addplot [color=mycolor1, forget plot]
  table[row sep=crcr]{%
1.86949259182206	2.851041894416\\
2.03018408192628	2.84604446976994\\
2.17237182230191	2.83255422748916\\
2.29861218587825	2.81242751649097\\
2.41114071012953	2.78707343834094\\
2.51187956915807	2.75754580885987\\
2.60246366961384	2.72461978989047\\
2.68427383880935	2.68885303783657\\
2.75847075137008	2.65063311987221\\
2.82602639165477	2.61021350747583\\
2.88775166141764	2.56774038583839\\
2.94431972289547	2.52327219864455\\
2.99628515378765	2.47679346786944\\
3.0440991929524	2.42822406951643\\
3.0881213996135	2.37742483755954\\
3.12862800618561	2.32420011403568\\
3.16581715426406	2.2682976576057\\
3.19981108386009	2.20940615768815\\
3.2306552047691	2.14715046917842\\
3.2583138162204	2.08108457955932\\
3.28266205296936	2.01068224645031\\
3.30347341769176	1.9353252070536\\
3.32040200762152	1.85428888022718\\
3.3329582615423	1.766725592993\\
3.3404767620074	1.67164562940943\\
3.34207438265767	1.56789692668809\\
3.33659699577093	1.45414520093235\\
3.32255330074806	1.3288579322199\\
3.29803556898633	1.19029834444202\\
3.26063004462054	1.0365397390164\\
3.20732570918268	0.865516700806141\\
3.13444095158236	0.675137754701349\\
3.03760531141148	0.463492569463746\\
2.91185824557558	0.229191149003763\\
2.751953616705	-0.0281375813465989\\
2.55296987173708	-0.307201442265731\\
2.31128645067285	-0.604357222621918\\
2.02585325470708	-0.913112149382156\\
1.69944506937969	-1.22415553322009\\
1.33936031707961	-1.52621725105204\\
0.957020192028156	-1.8077470468533\\
0.566323354814297	-2.05894306961403\\
0.181253089060065	-2.27336840108622\\
-0.186359555833534	-2.44856971763358\\
-0.528173197105508	-2.58564108209243\\
-0.839579847232978	-2.68814030184989\\
-1.11914634816221	-2.76088306328742\\
-1.36770876357337	-2.80897125440314\\
-1.58747699763472	-2.83717491568213\\
-1.78132765176004	-2.84962817416705\\
-1.95232315093529	-2.84974105301399\\
-2.10342373972285	-2.84023327533267\\
-2.23733814474857	-2.8232230272202\\
-2.35646319745511	-2.80033068410671\\
-2.46287530736805	-2.77277712795278\\
-2.55834911181294	-2.7414682639622\\
-2.64438817914622	-2.70706376792262\\
-2.72225912478861	-2.67003109368689\\
-2.79302456833141	-2.63068688472884\\
-2.85757276629238	-2.58922811280272\\
-2.91664309970992	-2.54575503873085\\
-2.97084729611412	-2.50028772566202\\
-3.02068658959441	-2.45277745981182\\
-3.0665651341151	-2.40311409839735\\
-3.10879997935108	-2.35113008330869\\
-3.14762784831883	-2.29660163007258\\
-3.18320884957155	-2.23924741727532\\
-3.21562712602563	-2.17872495371844\\
-3.24488829079953	-2.11462468320623\\
-3.27091332569106	-2.04646179793107\\
-3.29352841538703	-1.97366567471089\\
-3.31244995552533	-1.89556683644361\\
-3.32726370464709	-1.8113814004094\\
-3.33739675935027	-1.72019315351482\\
-3.34208075361804	-1.62093377446207\\
-3.3403045017562	-1.5123624388513\\
-3.33075439751556	-1.39304730665817\\
-3.31174159508735	-1.2613535150509\\
-3.28111695661172	-1.11544570406821\\
-3.23617900766813	-0.953318255767754\\
-3.17358828797476	-0.772873598414232\\
-3.08931551102783	-0.572077543716153\\
-2.97867232411474	-0.349227927358609\\
-2.83650045946101	-0.103371682617185\\
-2.6576173160146	0.165117691616635\\
-2.43760740047999	0.453853738244781\\
-2.17396722216338	0.757806234807604\\
-1.86742258058554	1.06902256126122\\
-1.52298134004091	1.37704825677867\\
-1.15013513350289	1.67022508059427\\
-0.761815116812467	1.93763085392071\\
-0.372273819119186	2.17099814937115\\
0.00534641066062597	2.36587592682938\\
0.360863004955486	2.52169235164974\\
0.687831398489772	2.64092765973912\\
0.983332883262683	2.72791009701897\\
1.24718773674731	2.78769820053344\\
1.48102129027608	2.82528353001849\\
1.68745078808525	2.84514024371596\\
1.86949259182206	2.851041894416\\
};
\addplot [color=mycolor1, forget plot]
  table[row sep=crcr]{%
2.00970785227068	2.83255951582902\\
2.16829459506053	2.82763576051588\\
2.30731523094934	2.81445258780785\\
2.4297046947207	2.79494515119588\\
2.53797278103981	2.77055534078189\\
2.6342363110595	2.74234305753359\\
2.72026598535531	2.71107549953793\\
2.79753589144043	2.67729624992662\\
2.86726977473392	2.641377482169\\
2.93048162747608	2.60355872384335\\
2.98800995892579	2.56397518676149\\
3.04054597826645	2.52267808539351\\
3.0886562731758	2.47964880147187\\
3.13280064745708	2.43480827137261\\
3.1733457264768	2.38802258405018\\
3.21057482112467	2.33910547032397\\
3.24469439457135	2.2878181226942\\
3.27583731546928	2.23386659248464\\
3.30406290887543	2.17689685470301\\
3.3293536276346	2.11648750120062\\
3.35160795356739	2.05213991488855\\
3.37062888832095	1.98326569362213\\
3.38610709619862	1.90917104296972\\
3.39759740635128	1.8290378685784\\
3.40448696981579	1.74190142248968\\
3.40595292314895	1.64662468673838\\
3.40090701464311	1.54187037394823\\
3.3879244932748	1.42607275936865\\
3.36515505297591	1.29741396377849\\
3.33021556567402	1.15381342249475\\
3.28006917739702	0.992945935825488\\
3.21090547265683	0.812313688080073\\
3.11805519292571	0.609410871367224\\
2.9960036943785	0.382033114912436\\
2.83860917803556	0.128788191766161\\
2.63967063205985	-0.150161553980915\\
2.39398493840761	-0.452182928665707\\
2.09890926629675	-0.7713121847596\\
1.75614168558561	-1.09789400392915\\
1.37302381558774	-1.41923798734738\\
0.962478354853678	-1.72151318483291\\
0.541122414584741	-1.99241543448791\\
0.126086512136677	-2.22353565354694\\
-0.268133107704887	-2.41143559577134\\
-0.631577552867782	-2.55720324292457\\
-0.959131116260233	-2.66503948191669\\
-1.24968021080543	-2.74066012508232\\
-1.50483046760844	-2.79004009696308\\
-1.72770386342792	-2.81865646246515\\
-1.92204364433203	-2.83115261151491\\
-2.09164728934969	-2.83127375429125\\
-2.24005708169665	-2.82194264110306\\
-2.37042229740948	-2.80538913525812\\
-2.48546291352305	-2.78328649808138\\
-2.58748692100441	-2.75687309892156\\
-2.67843186444601	-2.72705265847209\\
-2.75991406419349	-2.69447306987477\\
-2.8332769777984	-2.65958660284275\\
-2.8996347939691	-2.62269497831406\\
-2.95990987145525	-2.58398257526998\\
-3.01486390614539	-2.54354048910545\\
-3.06512327954946	-2.50138357319334\\
-3.11119923529598	-2.45746206919823\\
-3.15350353100543	-2.41166899718123\\
-3.19236012037668	-2.36384413054485\\
-3.22801328528932	-2.3137751086922\\
-3.26063248329872	-2.26119602514444\\
-3.29031400960702	-2.20578365590005\\
-3.31707939300565	-2.1471513506101\\
-3.34087024559752	-2.08484049055914\\
-3.36153905617021	-2.01830932075474\\
-3.37883514501698	-1.94691889450658\\
-3.39238467255464	-1.86991584538964\\
-3.40166320955607	-1.78641176052254\\
-3.40595894338724	-1.69535914022167\\
-3.40432415968643	-1.59552441581267\\
-3.39551232876152	-1.48545947032062\\
-3.37789822385948	-1.36347491041274\\
-3.34937958137217	-1.22762150438802\\
-3.30726196979925	-1.07569147765284\\
-3.24813565478921	-0.905259609032158\\
-3.16776724190762	-0.7137958141245\\
-3.06105331900569	-0.498894981689488\\
-2.92212022887542	-0.258680560012784\\
-2.74469760922797	0.00756902553045162\\
-2.5229174330319	0.298574963947673\\
-2.25263627569165	0.610127618005103\\
-1.93316830403618	0.934409971105092\\
-1.5689363762796	1.26008813656112\\
-1.17019380909004	1.57359617920111\\
-0.752052372940286	1.86152328722302\\
-0.331816374923651	2.11328116118183\\
0.0743649903603743	2.32291239975837\\
0.454129450740461	2.48937682241393\\
0.799989615652643	2.61552393552942\\
1.10897926826103	2.70649812614107\\
1.38150841585277	2.76827057497105\\
1.62007426223507	2.80663226408797\\
1.82819971892363	2.82666489950488\\
2.00970785227068	2.83255951582902\\
};
\addplot [color=mycolor1, forget plot]
  table[row sep=crcr]{%
2.17430603473841	2.82928793073052\\
2.33044535777815	2.82444897168392\\
2.46594297362775	2.81160678704818\\
2.58415716708103	2.79277030050891\\
2.68789254387122	2.76940595677011\\
2.77946699484521	2.7425714670914\\
2.86078573906653	2.71301902411803\\
2.93341104171975	2.68127265050854\\
2.9986230753596	2.64768508851882\\
3.05747086442469	2.61247907686435\\
3.11081384046105	2.57577692251115\\
3.15935513715665	2.53762134677768\\
3.20366787001532	2.49798979992626\\
3.24421554049886	2.45680381827628\\
3.28136751247959	2.41393452310087\\
3.31541029254594	2.36920500026929\\
3.3465551303234	2.32239002374647\\
3.37494224630124	2.27321336944935\\
3.40064178827692	2.22134278883848\\
3.42365140325324	2.16638255933344\\
3.44389007534897	2.10786339134407\\
3.46118760463784	2.04522934489396\\
3.47526876652081	1.97782129527289\\
3.48573077423729	1.90485640152273\\
3.49201214853071	1.82540300920684\\
3.49335047065297	1.73835053234855\\
3.48872578597209	1.64237424725429\\
3.47678575017332	1.53589584679112\\
3.45574827300754	1.41704249382237\\
3.42327810810487	1.28361072542021\\
3.37633699194203	1.13304806285973\\
3.31101526097114	0.962476172308744\\
3.22237097655495	0.768796474817825\\
3.10433695290387	0.548942043211334\\
2.9498123210435	0.300361978709517\\
2.75112793000477	0.0218247814546395\\
2.50112454582504	-0.285440449235098\\
2.19500744060268	-0.616443644580409\\
1.83279348396173	-0.961489600474761\\
1.42152136958137	-1.3063970008595\\
0.975845648074582	-1.63450693799913\\
0.515964639544959	-1.93017020234628\\
0.0633211215265111	-2.18224450162264\\
-0.363949894265142	-2.38592186002566\\
-0.753796552421497	-2.54230740615597\\
-1.10060884720806	-2.65651208684665\\
-1.40389862561362	-2.73547347665808\\
-1.66643431787696	-2.78630322905355\\
-1.89260401811819	-2.81535915112738\\
-2.08728476369964	-2.82789001873507\\
-2.25518878499693	-2.82801992101541\\
-2.40055085490442	-2.81888822167606\\
-2.52702294235802	-2.80283518923491\\
-2.63767930270744	-2.78157977328078\\
-2.73507199115711	-2.75636932685319\\
-2.82130341222054	-2.72809764441867\\
-2.89809909341037	-2.69739456722559\\
-2.96687326596811	-2.66469244542933\\
-3.02878476984102	-2.63027467163603\\
-3.08478318424195	-2.5943106773748\\
-3.13564609696506	-2.55688082262324\\
-3.18200873955121	-2.51799374438073\\
-3.22438719765583	-2.47759802849779\\
-3.26319624623718	-2.43558952510388\\
-3.29876265010897	-2.39181521440426\\
-3.33133455291616	-2.34607421503113\\
-3.36108736562728	-2.29811628370925\\
-3.38812635908966	-2.24763796026885\\
-3.4124859564092	-2.1942763488638\\
-3.43412549761435	-2.13760038238197\\
-3.45292099547021	-2.0770992851559\\
-3.4686520985225	-2.01216782796055\\
-3.48098310405704	-1.94208786685168\\
-3.48943639793565	-1.86600559716315\\
-3.49335612455861	-1.7829039871982\\
-3.49185921454444	-1.69157008370894\\
-3.4837701826888	-1.59055749156644\\
-3.46753554642006	-1.47814566124369\\
-3.44111377096973	-1.35230025189567\\
-3.40183832761321	-1.21064372066799\\
-3.34625674726662	-1.05045381725873\\
-3.26996105481103	-0.868721532579536\\
-3.16745032441434	-0.662320310385892\\
-3.03211101926641	-0.428362470131523\\
-2.85646733273374	-0.164834179270465\\
-2.63292320715374	0.128425970177643\\
-2.35522203042063	0.448465239159584\\
-2.02065569486654	0.788006644772256\\
-1.63254200363741	1.13497995420302\\
-1.20180828192215	1.47359901929041\\
-0.746311741639337	1.78722909211069\\
-0.287482595077459	2.06211021974623\\
0.15439697822222	2.29018412674959\\
0.56404269287998	2.46977369323667\\
0.932710877487586	2.60426859699022\\
1.2575759170864	2.69994355321456\\
1.54000858568913	2.76398340384577\\
1.78376187044044	2.80319756768445\\
1.99357716381973	2.82340732367622\\
2.17430603473841	2.82928793073052\\
};
\addplot [color=mycolor1, forget plot]
  table[row sep=crcr]{%
2.37008966809432	2.84583467340354\\
2.52336101304862	2.84109395545007\\
2.65493105339172	2.82863117278086\\
2.76862267778487	2.81052085769044\\
2.86755001510834	2.7882437769117\\
2.95423266046411	2.76284622438936\\
3.03070422684652	2.73505813212284\\
3.09860634829873	2.70537870361567\\
3.15926627203745	2.6741376613985\\
3.2137591926997	2.64153863503644\\
3.26295751751183	2.60768960954319\\
3.30756937735905	2.57262400672931\\
3.34816845266683	2.53631493276348\\
3.38521681902747	2.49868435450491\\
3.41908214635702	2.45960840435614\\
3.45005024766046	2.41891960171226\\
3.47833367691025	2.37640647248396\\
3.50407681296056	2.33181081137422\\
3.5273576240679	2.28482263812596\\
3.54818606830286	2.23507272898644\\
3.56649882899361	2.18212244331108\\
3.58214978799726	2.12545040199865\\
3.59489527503244	2.06443540369702\\
3.60437266468315	1.99833478863803\\
3.6100702845173	1.92625729534473\\
3.6112858071673	1.84712934765145\\
3.60706929830707	1.7596537595386\\
3.59614590411288	1.66226026082249\\
3.57681194832122	1.55304843364005\\
3.54679746572038	1.42972636836512\\
3.50308914992433	1.28955395370645\\
3.44171303509782	1.1293104537239\\
3.35749134537993	0.945325142230009\\
3.24382230805877	0.733640633043858\\
3.09259867497051	0.490420643488118\\
2.89448951307269	0.212751666019624\\
2.63993722127874	-0.100030692797326\\
2.321252018958	-0.44454182182079\\
1.93586348388899	-0.811582534465944\\
1.48987037994912	-1.18554016734986\\
0.999850167481938	-1.54625462069504\\
0.490800302083166	-1.87351819056015\\
-0.00976760886881658	-2.15229736504357\\
-0.47862074310768	-2.3758291948706\\
-0.900969397609134	-2.54529082015793\\
-1.27082671096515	-2.66712019821288\\
-1.58885251183779	-2.74994862101959\\
-1.85957398047222	-2.80238751830318\\
-2.08913881445424	-2.83189831408148\\
-2.28390274429081	-2.84444872664284\\
-2.44970755124551	-2.84458774491333\\
-2.59160185077621	-2.83568206418269\\
-2.71380175301886	-2.82017758069955\\
-2.81976180515058	-2.79982915091515\\
-2.91228461517198	-2.77588316498957\\
-2.99363402781255	-2.74921520687656\\
-3.06563694687861	-2.72043083707034\\
-3.12976911307429	-2.68993811207581\\
-3.18722482786806	-2.65799919115049\\
-3.2389724585397	-2.62476673107416\\
-3.28579804751296	-2.59030927637789\\
-3.32833924397085	-2.55462866090615\\
-3.36711145022197	-2.51767153878045\\
-3.40252769945223	-2.47933650369906\\
-3.43491342463749	-2.43947777474164\\
-3.46451696149501	-2.39790607275902\\
-3.49151635019761	-2.35438704353505\\
-3.51602275007773	-2.30863737162493\\
-3.53808054304273	-2.2603185491965\\
-3.55766395638719	-2.20902810013721\\
-3.57466976262089	-2.15428789845941\\
-3.58890528732447	-2.09552905336005\\
-3.60007054506754	-2.03207265890987\\
-3.60773279122859	-1.96310553255902\\
-3.61129108299517	-1.8876499221355\\
-3.60992754867974	-1.80452611441937\\
-3.60254096123919	-1.71230707561059\\
-3.58765697617668	-1.60926498692926\\
-3.56330832071048	-1.49331136579171\\
-3.52687810256052	-1.36193640716485\\
-3.47490208222548	-1.21216100573125\\
-3.40283508129398	-1.04052939130352\\
-3.30480996090834	-0.843194970645823\\
-3.17346649956951	-0.61618907910191\\
-3.00001523271263	-0.356005884048084\\
-2.77482637545278	-0.060655994624172\\
-2.48893587499026	0.26874259104347\\
-2.13675360062321	0.626078821425667\\
-1.71965020354387	0.998892456188959\\
-1.24894356722181	1.36887963300529\\
-0.745958132188205	1.71518129242515\\
-0.237835878231387	2.01959717603921\\
0.249302172258792	2.27105681113961\\
0.696170609995157	2.46700138361849\\
1.09254058306598	2.61163994282778\\
1.43609585729356	2.71285279072919\\
1.72975951920885	2.77946657892918\\
1.97909740102934	2.81960054669284\\
2.19048887725582	2.83997842985783\\
2.37008966809432	2.84583467340354\\
};
\addplot [color=mycolor1, forget plot]
  table[row sep=crcr]{%
2.6064589828494	2.88862616640401\\
2.75637109414822	2.883999158983\\
2.88357674785248	2.87195708876679\\
2.99239608382493	2.85462841786627\\
3.08625831948619	2.83349617714752\\
3.16787772828815	2.80958545644448\\
3.23940444300088	2.7835968756565\\
3.30254662301602	2.75600014215947\\
3.35866654942507	2.72709908273117\\
3.40885502240302	2.69707659985484\\
3.4539884661676	2.66602556057904\\
3.4947725378337	2.63396979088962\\
3.53177528488427	2.60087803167415\\
3.56545219557076	2.56667278660369\\
3.59616489562369	2.53123534419147\\
3.62419476499703	2.49440779820407\\
3.64975236089754	2.4559925584995\\
3.67298321301724	2.41574959242231\\
3.69397027804979	2.37339143202941\\
3.71273307794191	2.32857580000093\\
3.72922327387838	2.28089552774105\\
3.74331611674247	2.22986524679513\\
3.7547968298526	2.17490411565226\\
3.76334047684362	2.11531358795446\\
3.76848318927127	2.05024893201245\\
3.76958170216264	1.9786828892658\\
3.76575688652169	1.89935956348535\\
3.7558152982188	1.81073649949425\\
3.73814067326371	1.71091325142572\\
3.71054500789473	1.59754622004815\\
3.67006719259592	1.46775350112306\\
3.61270839144689	1.31802251130379\\
3.53310287780543	1.14415182510341\\
3.42415232901896	0.941294042417582\\
3.27672169828017	0.704226331451283\\
3.07963497410675	0.428057684036333\\
2.82043165537474	0.109644908416964\\
2.48755709893795	-0.250105954072952\\
2.07450524984844	-0.643388213434865\\
1.58529271977403	-1.05349246283719\\
1.03845463107716	-1.455972350082\\
0.465526666727522	-1.82428685022561\\
-0.0971822994778088	-2.13769698101945\\
-0.619092708293362	-2.38657010343355\\
-1.0818079135853	-2.57227964755975\\
-1.47930913785865	-2.70326146416607\\
-1.81429707356905	-2.79054540480154\\
-2.09396044839245	-2.84474516538712\\
-2.32688763075065	-2.87470949979353\\
-2.52134887775033	-2.88725599524887\\
-2.68456326331782	-2.88740426406207\\
-2.82251915336173	-2.8787542171408\\
-2.94005166132089	-2.8638482597678\\
-3.04101234893106	-2.84446473367481\\
-3.12845234049856	-2.82183799344284\\
-3.20478679703347	-2.79681700764569\\
-3.27193131730941	-2.76997725891164\\
-3.3314105736454	-2.74169880644514\\
-3.38444298534991	-2.71222039102835\\
-3.43200595019206	-2.68167673487756\\
-3.47488577361502	-2.65012405459818\\
-3.51371571737217	-2.61755724492019\\
-3.54900485010229	-2.58392108518757\\
-3.58115973412663	-2.54911704627594\\
-3.61050044905449	-2.513006731903\\
-3.63727202218022	-2.47541259993116\\
-3.6616519849773	-2.43611632205393\\
-3.68375447858502	-2.39485491537756\\
-3.70363106361525	-2.35131458846271\\
-3.72126812544317	-2.3051220656886\\
-3.73658047820497	-2.25583296979466\\
-3.74940042748902	-2.20291663828086\\
-3.75946111404373	-2.14573651277383\\
-3.76637237832512	-2.08352496403139\\
-3.76958659341788	-2.01535110261679\\
-3.76835083198556	-1.94007980586714\\
-3.76164027613148	-1.85631995225746\\
-3.74806589202947	-1.76235990295578\\
-3.72574714829292	-1.65608906519836\\
-3.6921384105946	-1.53490686388163\\
-3.64379699476674	-1.3956265130413\\
-3.57608526073647	-1.23439409674593\\
-3.48281657465312	-1.04666947244277\\
-3.35590135639156	-0.827362343558006\\
-3.18515089500215	-0.571289645864595\\
-2.95857988607527	-0.274202847881349\\
-2.6637901042153	0.0653566355405698\\
-2.2911130842731	0.443383637526595\\
-1.83861873047926	0.847729220696273\\
-1.31734186160238	1.25738786218882\\
-0.752965148538522	1.64591896811454\\
-0.180742070492837	1.98874283474401\\
0.364702770050638	2.27033687843499\\
0.858482795179477	2.48690166765121\\
1.28869281512832	2.64393862627122\\
1.6542297268519	2.75166978874569\\
1.96051424731082	2.82117963527452\\
2.21572948658655	2.86228446899281\\
2.42844674742143	2.88280838763335\\
2.6064589828494	2.88862616640401\\
};
\addplot [color=mycolor1, forget plot]
  table[row sep=crcr]{%
2.89659432980286	2.96676193805005\\
3.04261530234076	2.96226512107439\\
3.16502460886113	2.95068440839662\\
3.26865741175649	2.93418704719984\\
3.35725109593054	2.91424503832653\\
3.43369849912562	2.89185260649569\\
3.50024831591357	2.86767479814833\\
3.55865857421896	2.84214816144364\\
3.61031232001395	2.81554874916724\\
3.65630421859564	2.78803798830021\\
3.69750524874502	2.7596935331198\\
3.7346110320036	2.73052984690876\\
3.76817793593031	2.70051165407789\\
3.7986499818713	2.66956232868485\\
3.82637874357441	2.63756856028971\\
3.85163778682918	2.60438214155117\\
3.87463271664979	2.56981937111724\\
3.89550751961209	2.53365830398773\\
3.91434757504091	2.49563387047219\\
3.93117942554675	2.45543069532303\\
3.94596711272854	2.41267325741374\\
3.95860456359512	2.36691281659495\\
3.96890311675038	2.31761027725621\\
3.97657275161087	2.26411383557913\\
3.98119485614645	2.20562984672729\\
3.98218333748952	2.14118483016385\\
3.97872940632414	2.06957590343883\\
3.96972327094597	1.9893062437335\\
3.95364306233142	1.89850159302278\\
3.9283974564574	1.79480380868164\\
3.89110391335995	1.675239124805\\
3.83778061354591	1.53606460590245\\
3.76293148666367	1.37261143308965\\
3.6590218454198	1.17917817429086\\
3.51590365944118	0.949097715740115\\
3.32040545205894	0.675227100275456\\
3.05662035949933	0.351281984348361\\
2.70790143678588	-0.0254718849548195\\
2.26183065135216	-0.45005683315673\\
1.71833479132094	-0.905539269966142\\
1.09751946508794	-1.36238128704534\\
0.439952295888921	-1.78508392229778\\
-0.204892158233907	-2.14427478985174\\
-0.795522309411559	-2.42598247372705\\
-1.30879486149514	-2.63205385457764\\
-1.73948386246382	-2.7740324841352\\
-2.09385275020469	-2.86641317063372\\
-2.38310831218506	-2.9225058993971\\
-2.61920035292473	-2.95290136813909\\
-2.81284514872881	-2.96541205374497\\
-2.97290932641795	-2.96556939715673\\
-3.10644015794655	-2.95720537493408\\
-3.2189319875517	-2.94294495808261\\
-3.31463614585934	-2.92457530690133\\
-3.39683944813556	-2.90330721565073\\
-3.46809074661824	-2.87995516111223\\
-3.53037679839086	-2.85505969363724\\
-3.58525568559304	-2.82897019776391\\
-3.63395695620088	-2.80190075278642\\
-3.67745648029622	-2.77396777398898\\
-3.71653236360908	-2.74521525198787\\
-3.75180672247609	-2.71563145415037\\
-3.78377686886483	-2.68515963854271\\
-3.81283848422717	-2.65370444876066\\
-3.83930262805466	-2.62113505968558\\
-3.86340787361186	-2.58728572894365\\
-3.88532843746716	-2.55195410850564\\
-3.90517882758428	-2.51489743876753\\
-3.92301524024991	-2.47582655031098\\
-3.93883365621706	-2.43439741085335\\
-3.95256428852306	-2.39019975498695\\
-3.96406168175765	-2.34274210155407\\
-3.97308930908288	-2.29143217652475\\
-3.9792968971866	-2.23555139539052\\
-3.98218784472244	-2.17422159649788\\
-3.98107286747296	-2.1063616421403\\
-3.97500424384076	-2.0306308336959\\
-3.96268255309857	-1.9453554158036\\
-3.94232441658273	-1.84843405923733\\
-3.91147547523812	-1.73721883569902\\
-3.86674834272132	-1.6083714994132\\
-3.80346313929779	-1.45770448476151\\
-3.71517576571336	-1.28003921712422\\
-3.59311423455181	-1.06916436795909\\
-3.42564401367706	-0.81807283846141\\
-3.1981127735367	-0.519810584526548\\
-2.89383572578272	-0.169431726839988\\
-2.49743742084217	0.232526788235511\\
-2.00153691664825	0.675524368527635\\
-1.41545564256883	1.1360006086817\\
-0.770243446554317	1.58012503409236\\
-0.11298263180075	1.9739034807845\\
0.508902205780387	2.29501428786976\\
1.06251140744219	2.53789012609105\\
1.53425404867556	2.71015448202869\\
1.92556178584946	2.82553485172154\\
2.24585827135187	2.89826474559884\\
2.50708931975046	2.9403670821331\\
2.72072963620371	2.9610001758028\\
2.89659432980286	2.96676193805005\\
};
\addplot [color=mycolor1, forget plot]
  table[row sep=crcr]{%
3.25912552532344	3.09324266531364\\
3.40075479465786	3.08889121951157\\
3.51801223395664	3.07780502924075\\
3.61624480213526	3.06217246485952\\
3.69947648861053	3.04344120313106\\
3.77075341827375	3.02256613476829\\
3.83239965728367	3.00017203474404\\
3.88620378798641	2.97666014161672\\
3.93355428317857	2.95227813953397\\
3.97553779282755	2.92716623650056\\
4.01301075784132	2.90138751585974\\
4.04665182531094	2.87494781248438\\
4.07700035875947	2.84780848553431\\
4.1044847677202	2.8198942489043\\
4.12944326369463	2.79109743186516\\
4.1521388515986	2.76127951769643\\
4.17276978659718	2.73027044613358\\
4.19147629157254	2.69786590048875\\
4.20834398499086	2.66382258803285\\
4.22340416900177	2.62785133053263\\
4.23663083508319	2.58960758460892\\
4.24793392093488	2.5486787841831\\
4.25714795241982	2.50456761286711\\
4.26401466840201	2.45666994036201\\
4.26815746931866	2.40424565290605\\
4.2690444268092	2.34637992245714\\
4.26593495471885	2.28193153358199\\
4.2578027961164	2.20946366815217\\
4.2432243386947	2.12715103461714\\
4.22021593992867	2.03265559398637\\
4.18599646687352	1.92296203525389\\
4.13664176158023	1.79416546599789\\
4.06658864013277	1.64121227545014\\
3.96794662741513	1.45762193816308\\
3.82961331960112	1.23528635760583\\
3.63633210909192	0.96459705871368\\
3.36821531474038	0.63543950469671\\
3.002045454138	0.239976599825101\\
2.51667707823438	-0.221836445673655\\
1.9044993812272	-0.734699889847041\\
1.18580916303899	-1.2634372751351\\
0.413743826998447	-1.7597067026252\\
-0.341863310285406	-2.18064657921402\\
-1.02281270170815	-2.50552986704684\\
-1.59982705916651	-2.73729161093054\\
-2.07028016133366	-2.89245873154349\\
-2.4465644340315	-2.99061032245346\\
-2.74589996190596	-3.04869727856406\\
-2.98479716449934	-3.07948042799385\\
-3.17703046646929	-3.09191777405463\\
-3.33338405234074	-3.09208364407099\\
-3.46206006354589	-3.08403218295096\\
-3.56922864873687	-3.07045261226997\\
-3.65952583873925	-3.05312518023861\\
-3.73644959712913	-3.03322632582248\\
-3.802657834974	-3.01152960911262\\
-3.86018713294046	-2.98853736603995\\
-3.91061169893487	-2.96456703941671\\
-3.95515868121247	-2.93980794958082\\
-3.99479201962409	-2.91435870000864\\
-4.03027367842802	-2.8882517715398\\
-4.06220855779067	-2.8614695133002\\
-4.09107752686464	-2.83395423114886\\
-4.11726169670518	-2.80561409923711\\
-4.14106010818103	-2.77632597920563\\
-4.16270233220879	-2.74593579823306\\
-4.18235698062868	-2.71425683029612\\
-4.20013674200496	-2.68106599098774\\
-4.21610023890133	-2.64609805777553\\
-4.23025071160572	-2.60903753620112\\
-4.24253123028556	-2.569507683344\\
-4.25281578214716	-2.52705594759409\\
-4.2608951209982	-2.48113475918942\\
-4.26645563286624	-2.43107617280046\\
-4.2690485600191	-2.37605827577271\\
-4.26804558369017	-2.31506047765258\\
-4.26257476548153	-2.24680372935623\\
-4.25142785944008	-2.16967034780763\\
-4.23292558549316	-2.08159650663929\\
-4.20472109273816	-1.97992895305211\\
-4.16351325898368	-1.86123725761094\\
-4.1046316041567	-1.72107692250579\\
-4.02144842120085	-1.55371432541402\\
-3.90458758851443	-1.35186824245691\\
-3.74097753180596	-1.106627367739\\
-3.51303771445789	-0.807919823449194\\
-3.19886052999548	-0.446266543352548\\
-2.77522862261228	-0.0168551754644445\\
-2.22596970505169	0.473625417843501\\
-1.55583841724168	0.99997826760293\\
-0.802144912027912	1.51868967653404\\
-0.0296675510206517	1.98150862124943\\
0.694232425793066	2.35537571235375\\
1.32497545153508	2.63219319355454\\
1.84780947670892	2.82320517706321\\
2.26914351855745	2.94750772659703\\
2.60476612527971	3.02376569855963\\
2.87197017953517	3.06686310061952\\
3.08600600689612	3.08755613156197\\
3.25912552532344	3.09324266531364\\
};
\addplot [color=mycolor1, forget plot]
  table[row sep=crcr]{%
3.72011336764885	3.28650261533118\\
3.85702413168351	3.28230597003519\\
3.96897610105512	3.27172810514178\\
4.06180092705263	3.25696084177994\\
4.13977420248845	3.23941639939732\\
4.20606357291101	3.22000459090589\\
4.26304295070732	3.19930775068244\\
4.31251211391209	3.17769171474465\\
4.35585073130903	3.15537674917136\\
4.39412715861489	3.13248318681217\\
4.42817594083798	3.10906089241272\\
4.45865349172613	3.08510821856777\\
4.4860783837653	3.06058398783418\\
4.51086062960552	3.03541471430641\\
4.5333229443513	3.00949844243116\\
4.55371602420306	2.98270603911308\\
4.57222920919513	2.95488040947898\\
4.58899741369647	2.92583384332133\\
4.60410483656924	2.89534348991946\\
4.61758565067858	2.86314476919085\\
4.62942157532395	2.82892232882523\\
4.63953591305895	2.79229792284285\\
4.6477832358126	2.75281428543453\\
4.65393336975607	2.70991366424717\\
4.65764755998012	2.66290910319778\\
4.65844355059981	2.61094574676467\\
4.65564456679094	2.55294826337533\\
4.64830447353098	2.48754880315085\\
4.63509714703282	2.412987529111\\
4.61415146693176	2.32697452337006\\
4.58280309607122	2.2264978124928\\
4.53721893257655	2.10755820649293\\
4.47182920549984	1.96481080674113\\
4.37847981232864	1.79110508686486\\
4.24521490284242	1.57696626547966\\
4.05468868211603	1.31021198654074\\
3.78258157057486	0.976271917354255\\
3.39746211800831	0.560513847144141\\
2.86564603019587	0.0547349303129109\\
2.16628095956492	-0.530918998666906\\
1.3163458188637	-1.15601715113801\\
0.386364271571009	-1.75373532161103\\
-0.521395033430425	-2.25951955673995\\
-1.32245507455195	-2.64185412289414\\
-1.98003247030003	-2.90611385282926\\
-2.49783304165879	-3.0770017070695\\
-2.89858578292377	-3.18160571774392\\
-3.20833386187208	-3.24175809995871\\
-3.44961337829407	-3.27287671472173\\
-3.63990028487811	-3.28520643597026\\
-3.79212961182057	-3.28537994521807\\
-3.9157126484899	-3.27765525513644\\
-4.01748172736179	-3.26476547925274\\
-4.10242429214688	-3.24846956642585\\
-4.17421535716773	-3.22990138708828\\
-4.23559313594463	-3.20978984403993\\
-4.28862168962215	-3.18859808518074\\
-4.3348747062288	-3.16661214278255\\
-4.37556482043913	-3.14399778752436\\
-4.41163533937276	-3.12083719473372\\
-4.44382586715106	-3.09715260383444\\
-4.4727196324573	-3.07292144245103\\
-4.49877782605937	-3.04808571249083\\
-4.52236456680204	-3.02255738720676\\
-4.54376496468835	-2.99622089810873\\
-4.56319795420717	-2.96893334830397\\
-4.5808250056787	-2.94052278162212\\
-4.59675540197106	-2.91078460550761\\
-4.61104843177575	-2.87947606989276\\
-4.62371255163821	-2.84630851359366\\
-4.63470126528994	-2.81093687707772\\
-4.64390511629923	-2.77294571693644\\
-4.6511387335686	-2.73183060765709\\
-4.65612123157329	-2.68697333269246\\
-4.65844733256287	-2.63760858249132\\
-4.65754516584234	-2.58277889731557\\
-4.65261452453361	-2.52127318585437\\
-4.64253597186815	-2.45154214564333\\
-4.62573588562742	-2.37158111951525\\
-4.59998426947178	-2.2787672414254\\
-4.56208956747593	-2.16963347584623\\
-4.5074365468492	-2.03955904725652\\
-4.4292905327985	-1.88235946304682\\
-4.31777414606879	-1.6897858888492\\
-4.15845144678625	-1.4510336699434\\
-3.93065071450564	-1.15260296148236\\
-3.60631549513373	-0.779398394422836\\
-3.15176018832689	-0.318841953790946\\
-2.53704200326684	0.229845489251054\\
-1.75695286026187	0.84232819540724\\
-0.855158528695751	1.46283364501566\\
0.0765427825210307	2.02106832642988\\
0.938738827138494	2.46648514783927\\
1.66964433031892	2.78741035581351\\
2.25519674352221	3.00145883703757\\
2.71117451008825	3.13606823051264\\
3.06331051275912	3.21613443636322\\
3.33632262894028	3.26020434931857\\
3.55022760940807	3.28090751767623\\
3.72011336764885	3.28650261533118\\
};
\addplot [color=mycolor1, forget plot]
  table[row sep=crcr]{%
4.31376385210612	3.5712909289112\\
4.44604796470272	3.56724523047818\\
4.55294229853712	3.55715130579748\\
4.64071758107381	3.54319152926674\\
4.71386006446686	3.52673700116767\\
4.77562771826157	3.50865147131496\\
4.82842228145861	3.48947632770434\\
4.8740398886074	3.46954452840045\\
4.91384175266373	3.44905163095831\\
4.94887186762898	3.42810053785964\\
4.97993921391307	3.40672986044454\\
5.00767585842429	3.38493186964506\\
5.03257843421039	3.36266366895516\\
5.05503796211655	3.33985381874632\\
5.07536132941632	3.31640577504021\\
5.09378664609597	3.29219895681292\\
5.11049395508259	3.26708789225794\\
5.12561224768768	3.24089963630617\\
5.13922334338563	3.21342944799483\\
5.15136287276803	3.18443453110304\\
5.16201830606558	3.15362544550125\\
5.17112365225563	3.12065456076874\\
5.17855006487048	3.08510061308643\\
5.18409106277195	3.04644799415585\\
5.1874403073849	3.00405877844646\\
5.18815871459962	2.95713457915007\\
5.18562586284048	2.90466395578317\\
5.17896776172049	2.84534903574844\\
5.16694835218675	2.77750189542916\\
5.14780441920348	2.69889655486705\\
5.11899093564329	2.6065555205818\\
5.07678312764305	2.4964401358647\\
5.01564854053493	2.36300223925064\\
4.92725373028048	2.1985462606597\\
4.79891330831509	1.99236703555086\\
4.61127879316853	1.72973582654781\\
4.33531671753969	1.39118557929181\\
3.92978196480634	0.953580933677038\\
3.3437375136301	0.396514247226397\\
2.53427804952791	-0.280980476147483\\
1.50745677047191	-1.03587163166222\\
0.357068633485824	-1.7751575057098\\
-0.761980801568648	-2.39880021333496\\
-1.72327572086905	-2.85783412550436\\
-2.48205251320246	-3.16295654865162\\
-3.05554184320146	-3.35235574201659\\
-3.48329478537669	-3.46408909594419\\
-3.80382171578637	-3.52638354906937\\
-4.04728934136811	-3.55781365822627\\
-4.23545614256618	-3.57002395842887\\
-4.38356160756193	-3.57020410987645\\
-4.50222697748644	-3.56279421365319\\
-4.59890521232152	-3.55055423665639\\
-4.67889059104945	-3.53521283874192\\
-4.74599897837135	-3.51785831702342\\
-4.80302249625893	-3.49917539843617\\
-4.85203452262181	-3.47959019054209\\
-4.89459620783079	-3.45936000860568\\
-4.93189798638166	-3.4386296892856\\
-4.9648577892612	-3.41746720159537\\
-4.99419005699258	-3.39588623180777\\
-5.02045477689393	-3.37386039493518\\
-5.04409263240669	-3.35133191832602\\
-5.06545031740138	-3.32821654269527\\
-5.08479872897419	-3.30440569935721\\
-5.1023458529469	-3.27976657861271\\
-5.11824553461315	-3.25414040114044\\
-5.1326028775606	-3.22733897851624\\
-5.14547666374989	-3.19913945845492\\
-5.1568788852583	-3.16927696331457\\
-5.16677117692064	-3.13743461832273\\
-5.17505759304233	-3.1032301976982\\
-5.18157272299476	-3.06619825240833\\
-5.18606350855219	-3.02576606630659\\
-5.18816218546643	-2.98122103397904\\
-5.1873463224102	-2.93166593613698\\
-5.18287964143246	-2.87595691119277\\
-5.17372362286784	-2.81261638748794\\
-5.15840389644149	-2.73970941162479\\
-5.13480555360966	-2.65466608790162\\
-5.09985528731257	-2.55402457940664\\
-5.04902196155985	-2.43305815523723\\
-4.97552653095822	-2.28523843610815\\
-4.86909697091442	-2.1014864433282\\
-4.71405748155353	-1.86921149932266\\
-4.48661682319537	-1.57134726483217\\
-4.1518034654525	-1.18623914751836\\
-3.66256272570685	-0.690777535072004\\
-2.96831088898074	-0.0714274511682935\\
-2.04438402146404	0.653640890152905\\
-0.938508424831904	1.41435936861618\\
0.21598448538672	2.10611182803394\\
1.26708430107625	2.64931391731234\\
2.1278446066936	3.02747233197286\\
2.78963433717024	3.26955278862576\\
3.28506386419243	3.41591365444406\\
3.65486208112682	3.50005904778578\\
3.93365070187683	3.54509915839146\\
4.14720426646698	3.56579116524099\\
4.31376385210611	3.5712909289112\\
};
\addplot [color=mycolor1, forget plot]
  table[row sep=crcr]{%
5.07461298996009	3.9742534137051\\
5.20312201465524	3.97033115483588\\
5.30587092362685	3.96063385017361\\
5.38952296477647	3.94733331450053\\
5.45874344806197	3.93176352930561\\
5.51686185932693	3.91474824761516\\
5.56629796059966	3.89679421172553\\
5.60884022191911	3.8782071291593\\
5.64583086106727	3.85916244128577\\
5.67829092663376	3.83974909822166\\
5.70700623307129	3.81999687063234\\
5.7325872750885	3.79989339005687\\
5.75551152455045	3.77939461209223\\
5.77615356449002	3.75843092844573\\
5.79480664213944	3.73691026877469\\
5.81169800874334	3.71471898236091\\
5.82699960588411	3.69172093049256\\
5.84083510008146	3.66775496851056\\
5.85328385958774	3.6426307988632\\
5.86438214150385	3.61612299648009\\
5.87412146180624	3.58796281533438\\
5.88244380882355	3.55782714999972\\
5.88923298014295	3.52532371137192\\
5.89430080465441	3.48997102991825\\
5.89736625162972	3.45117124520797\\
5.89802426051472	3.40817265543968\\
5.8956992693481	3.36001749023562\\
5.88957539716316	3.30546801732987\\
5.87849020807931	3.24290038135406\\
5.86077046633799	3.17014965928821\\
5.8339736340321	3.08428016763865\\
5.79447338218908	2.98124009511028\\
5.73678308902399	2.85533667481716\\
5.65243596636619	2.69843657867155\\
5.52812160933873	2.49876569105719\\
5.34263239614403	2.23920506114815\\
5.0621707267555	1.89525219867204\\
4.63447728325225	1.4339410702098\\
3.98645789817976	0.818301459391054\\
3.04127562108193	0.0276824728203648\\
1.78004297196272	-0.899091935467874\\
0.325218964179175	-1.83387368454933\\
-1.08389811375681	-2.61937894138237\\
-2.2550301337491	-3.17893916732926\\
-3.13784810473174	-3.53419885722625\\
-3.77536201908882	-3.74490148296817\\
-4.23270744586419	-3.86445431115576\\
-4.56488181034869	-3.92906223056567\\
-4.81111173084384	-3.96087712458478\\
-4.99782490940931	-3.97300961597196\\
-5.14260703237204	-3.97319579578444\\
-5.25724330095103	-3.96604388570524\\
-5.34975483269134	-3.95433565851878\\
-5.42570363473993	-3.93977138106006\\
-5.48902167942809	-3.92339911305248\\
-5.54254114935149	-3.90586575046038\\
-5.58833817113287	-3.88756639621381\\
-5.62795931121304	-3.86873478129887\\
-5.66257339361963	-3.84949884281418\\
-5.69307497569185	-3.82991527850158\\
-5.72015597962227	-3.8099911362125\\
-5.74435596365189	-3.78969721327136\\
-5.76609779204375	-3.76897613123127\\
-5.78571311862739	-3.74774681542691\\
-5.80346059509227	-3.72590641372493\\
-5.81953872845096	-3.70333024722295\\
-5.83409464438102	-3.6798700882352\\
-5.84722954031131	-3.65535084147328\\
-5.85900125249416	-3.62956551949584\\
-5.86942405656107	-3.60226822087717\\
-5.87846552319394	-3.57316460983039\\
-5.88603991148949	-3.54189912649133\\
-5.89199714349145	-3.50803778459862\\
-5.89610578039421	-3.47104487502588\\
-5.89802748328929	-3.43025109272021\\
-5.89727897404888	-3.38480938741684\\
-5.89317515025065	-3.33363295680044\\
-5.88474311936696	-3.27530684970659\\
-5.87059038716489	-3.20795996555619\\
-5.84869928000546	-3.1290767595092\\
-5.81610037050964	-3.03521604508351\\
-5.76834404664527	-2.92158566067602\\
-5.69863131208933	-2.78139436731751\\
-5.59636883937504	-2.60486868121386\\
-5.44477406956813	-2.37780660366194\\
-5.21703948257889	-2.07964643193155\\
-4.87085181653827	-1.68161072853984\\
-4.34317610222974	-1.1474913730259\\
-3.55463859109555	-0.444439617948892\\
-2.44646178577775	0.424728851617969\\
-1.06299018604471	1.37607382544733\\
0.400171324256592	2.25282459691835\\
1.70555026237845	2.92773748712853\\
2.73097404485681	3.37854084526397\\
3.48314877041734	3.65388972130317\\
4.02274125320431	3.81341724852021\\
4.41164031661734	3.90197566701296\\
4.69683755618252	3.94808845837674\\
4.91064267037425	3.96882631493131\\
5.07461298996009	3.9742534137051\\
};
\addplot [color=mycolor1, forget plot]
  table[row sep=crcr]{%
5.99541349123009	4.49732482894193\\
6.12198373456767	4.49346810979022\\
6.22232276431917	4.48400228275453\\
6.30345828935967	4.47110453976137\\
6.37022710499521	4.45608804678951\\
6.4260340040194	4.43975081575307\\
6.47332606041965	4.42257641347924\\
6.51389546554197	4.40485202361184\\
6.54907708213389	4.38673928808308\\
6.57987997837295	4.36831750635314\\
6.60707667311733	4.34961025530518\\
6.63126471753926	4.33060181272328\\
6.65290979885519	4.31124713321961\\
6.67237623661826	4.29147760548794\\
6.68994867800905	4.27120391812361\\
6.70584748329541	4.25031680783956\\
6.72023943094676	4.2286861073336\\
6.73324478583477	4.20615826222468\\
6.74494135221224	4.18255229293414\\
6.7553658016758	4.15765400075927\\
6.7645122701438	4.13120802717579\\
6.77232790850611	4.10290714059211\\
6.77870469585667	4.07237780676337\\
6.7834663114413	4.03916064288895\\
6.78634810483478	4.0026836765805\\
6.7869670294593	3.96222529387516\\
6.7847765162843	3.9168621418823\\
6.77899814289203	3.86539467577519\\
6.76851666415351	3.80623886471428\\
6.7517158001861	3.73726568768493\\
6.72621592257996	3.65555853467875\\
6.68844539018337	3.55703918497539\\
6.63292334288572	3.43588027230273\\
6.55103235852934	3.28356844963774\\
6.42888013525645	3.0874023791601\\
6.24355374627446	2.82812556715442\\
5.95672477318787	2.47646662632\\
5.50486682757952	1.98928488869036\\
4.78954650692568	1.31006109479641\\
3.68844777342875	0.389582967520768\\
2.13844260427236	-0.748782620037423\\
0.291673312658474	-1.93519244771774\\
-1.48840841691493	-2.92778854675826\\
-2.91487958197734	-3.60977739845915\\
-3.93942991406759	-4.02237595996972\\
-4.64665575647416	-4.2562875645141\\
-5.13582834199775	-4.38424753063528\\
-5.48132614055454	-4.45149222266906\\
-5.73208249096043	-4.48391637079128\\
-5.91920627616184	-4.49608929107534\\
-6.06253170237649	-4.49628172862736\\
-6.17493002609912	-4.4892744801395\\
-6.26494796344672	-4.47788510802943\\
-6.33839878081503	-4.46380206027429\\
-6.39932959310621	-4.44804860908601\\
-6.45061973895545	-4.43124671036897\\
-6.4943587403697	-4.41377052971911\\
-6.53209025416744	-4.3958376866492\\
-6.56497287364746	-4.37756447947498\\
-6.5938882314731	-4.35899976779666\\
-6.61951498853594	-4.34014589512689\\
-6.64238027283283	-4.32097153382069\\
-6.6628958968002	-4.30141933996902\\
-6.68138407285814	-4.28141013959875\\
-6.69809570437103	-4.26084466461084\\
-6.71322326879524	-4.23960341595104\\
-6.72690960312022	-4.21754493771128\\
-6.73925340887719	-4.19450257061826\\
-6.7503119251	-4.17027957233026\\
-6.76010090999522	-4.14464231222923\\
-6.76859177539645	-4.11731104005872\\
-6.77570538291145	-4.08794745695008\\
-6.78130157695893	-4.05613793861392\\
-6.7851629118082	-4.0213707060733\\
-6.7869700919149	-3.98300440271407\\
-6.78626516134385	-3.94022424412106\\
-6.78239605797812	-3.89197987008238\\
-6.77443209693362	-3.83689575601213\\
-6.76103300288552	-3.77313969088716\\
-6.74024193420233	-3.69822593956794\\
-6.7091511345993	-3.60871475364321\\
-6.66334908680843	-3.49974462133492\\
-6.59598477240038	-3.36429144842522\\
-6.49615044266738	-3.19198199938406\\
-6.34604973540744	-2.96719982612318\\
-6.11607156316345	-2.66617766747122\\
-5.7566940389327	-2.25311741435295\\
-5.18768700672088	-1.67742587824769\\
-4.2942798350915	-0.881332086703665\\
-2.966737669024	0.159264763485881\\
-1.23176837138047	1.35185575181467\\
0.630117465771918	2.46760048475806\\
2.25388472598717	3.30754348402831\\
3.47340789742028	3.84405640500079\\
4.32607619065736	4.15642304228316\\
4.91317670196399	4.33011777856789\\
5.32296814448572	4.42349646802813\\
5.61627328687972	4.47095329859258\\
5.83215503416641	4.49191078227295\\
5.99541349123009	4.49732482894193\\
};
\addplot [color=mycolor1, forget plot]
  table[row sep=crcr]{%
6.89736886606275	5.03319437558776\\
7.02416387175193	5.02933501139698\\
7.12411966905462	5.01990796379786\\
7.20458777221079	5.0071180347061\\
7.27057166197707	4.9922792439007\\
7.32556233563215	4.97618178754688\\
7.37205090732628	4.9592997824019\\
7.4118511675274	4.94191188332185\\
7.44630755102136	4.924172880178\\
7.47643228783712	4.90615695916026\\
7.50299772990644	4.88788414863349\\
7.52659964487822	4.86933651746032\\
7.54770128007571	4.85046794499134\\
7.56666440163349	4.8312097142689\\
7.58377130067985	4.81147325938556\\
7.59924036284171	4.79115083779699\\
7.61323689171352	4.77011454015385\\
7.6258802666304	4.74821380263283\\
7.63724808003991	4.72527139393899\\
7.64737756120027	4.70107767299109\\
7.65626429327812	4.67538272270993\\
7.66385792005916	4.64788572851158\\
7.67005416198436	4.61822064710494\\
7.67468194685035	4.58593674444558\\
7.67748369861844	4.55047188167056\\
7.67808563840317	4.51111534899351\\
7.67595302539693	4.46695534627469\\
7.6703220523474	4.41680346793679\\
7.66009460612644	4.35908403918405\\
7.64367242880821	4.2916685701503\\
7.61868976481584	4.21162259560055\\
7.58157130548474	4.11480947629323\\
7.52678116568926	3.99525565555512\\
7.44551109794936	3.84411120881695\\
7.32332970899458	3.64791936884771\\
7.13589448163364	3.38573012706076\\
6.84116389135927	3.02445608149658\\
6.36613618695305	2.51243804049203\\
5.58954013189169	1.77531831863544\\
4.34304282563732	0.73379839849817\\
2.50920940090667	-0.612411235566368\\
0.262259739171652	-2.05567829798883\\
-1.89433615075502	-3.25854162383184\\
-3.56922904333947	-4.05972382111383\\
-4.72545819139513	-4.5256232886552\\
-5.49608724978678	-4.7806438240618\\
-6.01489616881318	-4.91642316290733\\
-6.37409111441823	-4.98636678638294\\
-6.6309998647971	-5.01960357672328\\
-6.82064113847222	-5.03194962969862\\
-6.96470528987139	-5.03214847701271\\
-7.07696932594897	-5.0251529030354\\
-7.16643373595251	-5.01383567684412\\
-7.23914378704781	-4.99989606987022\\
-7.29926650130696	-4.98435252965248\\
-7.34974301291436	-4.96781787150778\\
-7.39269392434695	-4.95065710246054\\
-7.42967754762933	-4.93308011617047\\
-7.46185833587123	-4.91519724170708\\
-7.49011913873529	-4.89705303525015\\
-7.51513748514114	-4.87864698942531\\
-7.5374383030822	-4.85994615784343\\
-7.55743085815116	-4.84089262376764\\
-7.57543487892197	-4.82140754519059\\
-7.59169908559255	-4.80139279414265\\
-7.60641421853607	-4.78073076348686\\
-7.6197219241111	-4.75928262020958\\
-7.63172034425345	-4.73688506958777\\
-7.64246687771582	-4.71334551425335\\
-7.65197826816	-4.68843531256141\\
-7.66022787530608	-4.66188063126278\\
-7.66713964921773	-4.63335011340417\\
-7.67257789316159	-4.60243819627687\\
-7.67633127945071	-4.568642344819\\
-7.67808863581077	-4.53133159948159\\
-7.67740251220767	-4.48970248650128\\
-7.67363405872692	-4.44271618478672\\
-7.66586855333436	-4.38900733663324\\
-7.65278364435701	-4.32674905720569\\
-7.6324394206162	-4.25344879525755\\
-7.60193576333479	-4.16563255672918\\
-7.55683815140872	-4.05834487081539\\
-7.49018848824136	-3.92433857229832\\
-7.3907541462704	-3.75273558519102\\
-7.23985724583102	-3.52678910568387\\
-7.00557905309622	-3.22019061556202\\
-6.63244583047343	-2.79142193565118\\
-6.02533913185054	-2.17738838513611\\
-5.03589622488247	-1.29611144069466\\
-3.49852803761304	-0.0916384912261362\\
-1.4099049355907	1.34356442438593\\
0.860303057242737	2.70407592263365\\
2.80131087132802	3.70855023314097\\
4.20507350080996	4.32647318144255\\
5.14971166906231	4.6727291307529\\
5.78023129590174	4.85936587721496\\
6.21020475115491	4.95739022037701\\
6.51274555585935	5.00636499346055\\
6.7326384758569	5.02772444162467\\
6.89736886606275	5.03319437558776\\
};
\addplot [color=mycolor1, forget plot]
  table[row sep=crcr]{%
7.29591158957958	5.27513025289535\\
7.42332987221212	5.27125335301717\\
7.52358643507382	5.26179883012366\\
7.60417597996328	5.2489901777062\\
7.67018025597365	5.23414719531595\\
7.72513429683464	5.21806073965346\\
7.77155458382699	5.20120373405925\\
7.81126978887302	5.1838531464475\\
7.8456331896464	5.16616213164784\\
7.8756623093767	5.14820348838183\\
7.90213267362373	5.12999615354589\\
7.9256419455298	5.11152139069478\\
7.94665449572159	5.09273253083439\\
7.96553275207919	5.07356053360328\\
7.98255940086198	5.05391670628169\\
7.99795308180025	5.03369335345425\\
8.01187929539751	5.01276276994595\\
8.02445761963065	4.99097474117082\\
8.03576589188746	4.96815252170294\\
8.04584166962416	4.9440870861721\\
8.05468098115947	4.91852925510848\\
8.06223406538229	4.8911790599331\\
8.06839742085364	4.86167138534677\\
8.07300096778685	4.8295564552579\\
8.07578835965867	4.79427301852752\\
8.07638728205741	4.75511099372161\\
8.07426462963591	4.71115859635049\\
8.06865819541017	4.66122616456551\\
8.05847090822105	4.60373425702524\\
8.04210376970811	4.5365457484206\\
8.01718571494063	4.4567080860975\\
7.98012522319577	4.3600479551804\\
7.92534468945407	4.24051775681551\\
7.84393410430332	4.08911620168565\\
7.72121595557031	3.89206964324699\\
7.53224428080095	3.62774426898471\\
7.23346459228597	3.26153245470666\\
6.74802758508566	2.73834698777122\\
5.94514374297721	1.97638429548909\\
4.63617708355683	0.882869012273487\\
2.67725454219386	-0.554912272002079\\
0.250106303551813	-2.11382512863947\\
-2.07544543435575	-3.41106365837185\\
-3.85922221824205	-4.26450734222697\\
-5.07193654273235	-4.7532753562224\\
-5.8697099806853	-5.01733122425154\\
-6.40155209947301	-5.15654607646189\\
-6.76717349051142	-5.22775289084373\\
-7.02734471168533	-5.26141775845063\\
-7.21867447336467	-5.2738769789914\\
-7.36361282690222	-5.27407889076835\\
-7.47631503422351	-5.26705713646384\\
-7.56597768604665	-5.25571554687432\\
-7.63875149643691	-5.24176419093408\\
-7.6988620035609	-5.22622413513902\\
-7.74928367714008	-5.2097076762349\\
-7.7921565112804	-5.19257827760396\\
-7.82905027501155	-5.1750441323721\\
-7.86113625917744	-5.15721404607183\\
-7.88930141267163	-5.13913133431034\\
-7.91422571902979	-5.12079454432989\\
-7.93643556700714	-5.10217005697309\\
-7.95634108450513	-5.08319952348469\\
-7.97426250931867	-5.06380387995354\\
-7.99044887366695	-5.0438849610277\\
-8.00509113481198	-5.02332528700023\\
-8.01833113061329	-5.00198630280175\\
-8.03026721982775	-4.97970513216102\\
-8.04095708344062	-4.9562897293411\\
-8.05041784766114	-4.93151213056845\\
-8.05862338820194	-4.90509929667592\\
-8.065498337632	-4.87672076222637\\
-8.07090788086842	-4.84597191630921\\
-8.07464179934598	-4.81235116353054\\
-8.07639027069993	-4.77522833395339\\
-8.07570740910741	-4.73380033467176\\
-8.07195602234769	-4.68702783425895\\
-8.06422280598895	-4.63354317114868\\
-8.05118578101769	-4.57151365771521\\
-8.03090251436506	-4.49843416256824\\
-8.00046327374921	-4.410804886061\\
-7.95540723953655	-4.30361828179464\\
-7.88871095532438	-4.16952157079232\\
-7.78898364202658	-3.99741843465297\\
-7.63716302068276	-3.77009842557475\\
-7.40037751408439	-3.46023676319357\\
-7.02074134537897	-3.02403193890216\\
-6.39703409109933	-2.39328418225762\\
-5.36654591193065	-1.47560069962743\\
-3.73799675272195	-0.199935111082237\\
-1.49146927378901	1.3435635047952\\
0.962926008148577	2.81449051151017\\
3.04480000395319	3.8920465842516\\
4.52832215317503	4.54522389310579\\
5.5122497122189	4.90595764984086\\
6.16153292543111	5.09818439883339\\
6.6006229495706	5.19830404295129\\
6.90772781078717	5.24802599683915\\
7.12996437498889	5.26961747647854\\
7.29591158957958	5.27513025289535\\
};
\addplot [color=mycolor1, forget plot]
  table[row sep=crcr]{%
6.81839938705513	4.98558555137348\\
6.9451056567848	4.98172857837679\\
7.04503269745632	4.97230404998413\\
7.12550382874078	4.95951351377152\\
7.19150742751882	4.9446702051191\\
7.24652621591938	4.92856445823852\\
7.29304670485787	4.91167081831979\\
7.33288010703728	4.89426840679286\\
7.36736941959477	4.87651242497647\\
7.39752608589151	4.8584773879902\\
7.42412204545158	4.84018356931731\\
7.44775286792895	4.82161320692916\\
7.46888172297768	4.802720283049\\
7.48787036298415	4.78343612628583\\
7.50500109471872	4.76367216612502\\
7.52049232706941	4.74332061039937\\
7.53450938011309	4.72225345843491\\
7.54717163259053	4.70032001509007\\
7.5585566510167	4.67734287814631\\
7.56870160579294	4.65311219540491\\
7.57760198044974	4.62737779738012\\
7.5852072695652	4.59983857492861\\
7.59141298466069	4.57012814868072\\
7.59604777339064	4.53779541139837\\
7.59885369623389	4.5022778262537\\
7.59945651723111	4.46286428877193\\
7.59732094276784	4.41864266581239\\
7.59168253759535	4.36842439739232\\
7.58144256176967	4.31063406338187\\
7.56500233907657	4.243144292049\\
7.5399964148077	4.16302350528493\\
7.50285171749803	4.0661415588715\\
7.44803942928042	3.9465388293111\\
7.36677017908548	3.79539498589725\\
7.24466050473369	3.59931675028097\\
7.0574886748104	3.33749314960869\\
6.76352025145179	2.97714793958209\\
6.29053702849613	2.46732264221061\\
5.51919290384651	1.7351657602429\\
4.28518881238173	0.704044860696217\\
2.47617548846006	-0.62399326502574\\
0.26472488379459	-2.04447711678708\\
-1.85854763216566	-3.22872807539079\\
-3.51179701572037	-4.01952314056496\\
-4.65672460540954	-4.48084729701715\\
-5.42192597149296	-4.73406098849309\\
-5.93815571883776	-4.86916023516302\\
-6.29610684807415	-4.93885917516955\\
-6.55240778900606	-4.97201605639051\\
-6.74175384000817	-4.98434219551015\\
-6.88568139038962	-4.98454045434875\\
-6.99789141017824	-4.97754800336823\\
-7.08734546190676	-4.96623193256517\\
-7.16006824322224	-4.95228978176482\\
-7.22021562299617	-4.93673979314651\\
-7.27072257215752	-4.92019511301591\\
-7.31370624925992	-4.90302121434563\\
-7.3507230387261	-4.88542843600515\\
-7.3829363263936	-4.86752747846997\\
-7.41122838809586	-4.84936318429306\\
-7.43627646256458	-4.8309352520156\\
-7.45860534888868	-4.81221087003346\\
-7.47862427236261	-4.79313219481558\\
-7.49665296586647	-4.77362040406707\\
-7.51294017121452	-4.75357734217863\\
-7.52767665013585	-4.73288533123189\\
-7.54100405805473	-4.71140542676794\\
-7.55302052460794	-4.68897418296435\\
-7.56378340709999	-4.66539881155098\\
-7.57330937097375	-4.64045043926517\\
-7.58157165270455	-4.61385495942758\\
-7.58849402465152	-4.58528069952081\\
-7.5939405470587	-4.55432174128212\\
-7.59769957191291	-4.52047516178699\\
-7.59945951764889	-4.48310960021399\\
-7.59877242863057	-4.44142120888268\\
-7.59499885959066	-4.39437090303387\\
-7.58722344586879	-4.34059333582325\\
-7.5741232741605	-4.27826223117822\\
-7.55375828026879	-4.20488688476054\\
-7.52322939077674	-4.11699767437352\\
-7.4781062033147	-4.00964867067367\\
-7.4114422958755	-3.87561300911386\\
-7.31203520475341	-3.70405586877619\\
-7.16128309678768	-3.47832412658061\\
-6.9274595685231	-3.17231678512378\\
-6.55558188850301	-2.7449831897496\\
-5.95177163129806	-2.13426805552596\\
-4.97054626054604	-1.26027984798751\\
-3.45135195503113	-0.0699980423053237\\
-1.39392514759134	1.34380755585939\\
0.840019649925826	2.68258056595531\\
2.75314581466553	3.67259137928405\\
4.14099226701179	4.28347990938291\\
5.07776315797388	4.62683679623043\\
5.70454533578041	4.81235986820016\\
6.13273164742407	4.90997324701889\\
6.4344036995944	4.95880562065331\\
6.65387188453432	4.98012287766274\\
6.81839938705513	4.98558555137348\\
};
\addplot [color=mycolor1, forget plot]
  table[row sep=crcr]{%
5.81106869281575	4.39030195368767\\
5.9378306742482	4.38643831889657\\
6.03846583405338	4.37694387713713\\
6.1199335771122	4.36399287535022\\
6.1870374509301	4.34890071949983\\
6.24316659297427	4.33246893432805\\
6.29076134204276	4.31518444620006\\
6.33161167052361	4.29733720124044\\
6.36705246181677	4.27909093792948\\
6.39809385319555	4.26052644680076\\
6.42550985478205	4.24166828271736\\
6.44989960323473	4.22250127616143\\
6.47173029631398	4.20298058067738\\
6.49136760427196	4.18303748327818\\
6.50909732396471	4.16258230635481\\
6.52514074436214	4.14150517702381\\
6.53966534011105	4.11967508309843\\
6.55279182890572	4.09693738654155\\
6.56459820933533	4.07310977119143\\
6.57512106561025	4.04797642449512\\
6.58435412975198	4.02128006254521\\
6.59224378227706	3.99271117302261\\
6.59868079617576	3.96189353339659\\
6.60348711570957	3.92836460753721\\
6.60639570497698	3.89154874944605\\
6.60702032865372	3.85072011495395\\
6.60481024568603	3.80495058240551\\
6.59898169346763	3.75303544587264\\
6.58841279772551	3.69338554863195\\
6.57147948699918	3.62386780739903\\
6.54579401358252	3.54156491815036\\
6.50777897262767	3.44240636935973\\
6.45195745624496	3.32059186998236\\
6.36974484521994	3.16767858486497\\
6.24735944954143	2.97113266238818\\
6.06220161960239	2.7120821831487\\
5.77677395539875	2.3621237276036\\
5.32965500631877	1.88001801110771\\
4.62732032063472	1.21306234745563\\
3.55676427257078	0.318016299685843\\
2.06476136417372	-0.777861068382086\\
0.298062509953041	-1.91287086600401\\
-1.40643183116124	-2.86326345521889\\
-2.78185360802044	-3.52076771016897\\
-3.77869138961688	-3.9221517335401\\
-4.47246549395942	-4.15158450527908\\
-4.95546408019367	-4.27791426206472\\
-5.2982738775063	-4.34462797714021\\
-5.54798757103047	-4.37691313728978\\
-5.73484389100408	-4.38906632495553\\
-5.87826329687165	-4.38925751520342\\
-5.99091736913433	-4.38223347418573\\
-6.08125526720893	-4.37080307140861\\
-6.15504250345002	-4.35665515142711\\
-6.21630322685912	-4.34081614461725\\
-6.26790632881065	-4.32391153902328\\
-6.31193723841499	-4.3063185848153\\
-6.34993871357164	-4.28825732814437\\
-6.38306999766929	-4.26984585027003\\
-6.41221405884161	-4.25113423359834\\
-6.43805112511992	-4.23212557690578\\
-6.46110988715076	-4.21278892058008\\
-6.48180359518745	-4.19306696380742\\
-6.5004557152727	-4.17288029514566\\
-6.51731819170392	-4.15212915680827\\
-6.53258431563611	-4.13069332223909\\
-6.54639750008312	-4.10843037237793\\
-6.55885677247197	-4.08517244024534\\
-6.5700194288572	-4.06072131195114\\
-6.5799009870112	-4.03484159215411\\
-6.58847227898074	-4.00725143358615\\
-6.59565318817315	-3.97761005983719\\
-6.60130210178703	-3.94550093310744\\
-6.60519953099088	-3.91040886721914\\
-6.60702341415294	-3.87168855594162\\
-6.60631213863073	-3.82852070749267\\
-6.60240890778542	-3.77984996581519\\
-6.5943770580994	-3.72429558219703\\
-6.58086905971169	-3.66002056526266\\
-6.55991993495031	-3.58453639081319\\
-6.52861445599441	-3.49440592415262\\
-6.48253879024475	-3.38478309582889\\
-6.41485667012296	-3.24868830269853\\
-6.31472279447289	-3.07585768979424\\
-6.16452915602238	-2.8509292437166\\
-5.93517781718945	-2.55071472514717\\
-5.57847827374839	-2.14070828646252\\
-5.01746429854773	-1.57305770883352\\
-4.14439668688545	-0.795006619829821\\
-2.86036399139633	0.211598326002595\\
-1.19674002935348	1.3552324973526\\
0.583584041481237	2.42208766959095\\
2.14300486862341	3.22866929909522\\
3.3241693684594	3.74823849458138\\
4.15735160444676	4.05342574908404\\
4.73528772043855	4.22438788118071\\
5.14097069607059	4.31681952704536\\
5.43256333691566	4.3639936027741\\
5.6478623113483	4.38489141310636\\
5.81106869281575	4.39030195368767\\
};
\addplot [color=mycolor1, forget plot]
  table[row sep=crcr]{%
4.79211818889952	3.82077342471787\\
4.92177086224927	3.81681362251261\\
5.02579382855331	3.80699436771221\\
5.11071934761554	3.79349020396659\\
5.18115322605272	3.77764668970029\\
5.24040110153725	3.76030015906999\\
5.29087648834215	3.74196825006895\\
5.334370015191	3.72296522291877\\
5.37222975797709	3.70347281736717\\
5.40548392042828	3.68358434168994\\
5.43492556274751	3.66333232196307\\
5.46117193263937	3.64270583274064\\
5.48470650120408	3.62166118565911\\
5.50590899835184	3.60012820367493\\
5.52507694347959	3.57801342936336\\
5.54244099201972	3.55520106522052\\
5.55817563103771	3.53155208318496\\
5.57240620970301	3.50690168662155\\
5.58521288765786	3.481055108402\\
5.59663176028572	3.453781546994\\
5.60665312406216	3.42480584883557\\
5.61521653142038	3.39379730983344\\
5.62220190131706	3.36035465530512\\
5.62741543027179	3.32398581588658\\
5.63056828640992	3.28408047206688\\
5.63124490194165	3.23987237622389\\
5.62885583532081	3.19038699505496\\
5.62256719023864	3.13436775439726\\
5.61119365519057	3.07017064517139\\
5.59303397095464	2.99561142824242\\
5.56561361540334	2.90774105071726\\
5.5252755607558	2.80251167617108\\
5.46651940192237	2.67427662899367\\
5.38092287164628	2.51504408972986\\
5.25538259842559	2.31339080669481\\
5.06931449840661	2.05299768168736\\
4.7905529760893	1.71109084870631\\
4.37072655265967	1.25819696264862\\
3.74488371386983	0.663512848732174\\
2.84933918050736	-0.0857440839287663\\
1.67583031501588	-0.948209188546892\\
0.336492051431472	-1.80883640449244\\
-0.962811281776992	-2.53305604756679\\
-2.05571937770858	-3.05513154500464\\
-2.89325465498746	-3.39208229568967\\
-3.50782571836996	-3.59514896413235\\
-3.95467324606576	-3.71192766817774\\
-4.28267967718778	-3.77570835930012\\
-4.52781894824707	-3.80737297697914\\
-4.71488412494648	-3.81952287066889\\
-4.86065518138394	-3.81970699278071\\
-4.97652334114615	-3.81247611664875\\
-5.07031940568377	-3.80060392687166\\
-5.14751628170736	-3.78579935958603\\
-5.2120073467488	-3.76912310516208\\
-5.26661129858299	-3.7512339630798\\
-5.31340301878347	-3.73253677799184\\
-5.35393349155295	-3.71327267409269\\
-5.38937825513544	-3.69357487338636\\
-5.42063916980341	-3.67350359035917\\
-5.44841520493515	-3.65306793974812\\
-5.47325231470367	-3.63223959432685\\
-5.49557894254391	-3.61096105269788\\
-5.51573145198247	-3.58915025213298\\
-5.53397233142163	-3.56670256895459\\
-5.55050306209345	-3.54349080645135\\
-5.5654728855542	-3.51936347099641\\
-5.57898424168559	-3.4941414155384\\
-5.59109529174929	-3.46761274293533\\
-5.60181963693514	-3.43952567750913\\
-5.61112304392667	-3.40957890264328\\
-5.61891664770849	-3.37740859296992\\
-5.6250456597327	-3.34257099954288\\
-5.62927198358847	-3.30451891450158\\
-5.63124820150209	-3.26256955533503\\
-5.63047893279206	-3.21586022250144\\
-5.62626322571677	-3.16328626637663\\
-5.6176078196875	-3.10341308010106\\
-5.60309475075204	-3.03434942572295\\
-5.58067602824568	-2.95356249238552\\
-5.54734979529474	-2.85760436722309\\
-5.49864116120367	-2.741703537316\\
-5.42775866109708	-2.59915315257045\\
-5.32421527535131	-2.42040586979525\\
-5.17159688773638	-2.19179337220612\\
-4.94411730786096	-1.89393757040694\\
-4.60201321517407	-1.50054537619279\\
-4.08799717243774	-0.980163705403888\\
-3.333539748464	-0.307359363168784\\
-2.29356464353244	0.508483010097138\\
-1.01486676146027	1.38789111961271\\
0.331085521457898	2.1943914998068\\
1.54082044438049	2.81975256886592\\
2.50549061640488	3.24374496810197\\
3.22499695397131	3.50706584023681\\
3.74886645174885	3.66190485532824\\
4.13098470470159	3.74889693772828\\
4.41383400597597	3.79461769413917\\
4.62740908872181	3.81532611953117\\
4.79211818889952	3.82077342471787\\
};
\addplot [color=mycolor1, forget plot]
  table[row sep=crcr]{%
3.95999323486845	3.39757254170735\\
4.09484250813319	3.39344321228675\\
4.2045297265259	3.38308210600781\\
4.29508387125841	3.36867800128559\\
4.37087714532624	3.35162544588445\\
4.43511993062576	3.33281395952028\\
4.49020046207001	3.31280761045635\\
4.53791827504927	3.29195743937878\\
4.57964576454014	3.27047251437929\\
4.61644106500604	3.24846521707147\\
4.64912774304262	3.22598024130648\\
4.67835162694857	3.20301311246613\\
4.70462167883089	3.17952181318685\\
4.72833955405471	3.15543373939353\\
4.74982098673809	3.13064935989219\\
4.76931112259044	3.10504340685299\\
4.78699521765181	3.07846405916893\\
4.80300561850361	3.05073031928915\\
4.81742555817475	3.02162757703464\\
4.83028998579573	2.99090116593392\\
4.84158335147927	2.95824752002809\\
4.85123394759627	2.92330230365497\\
4.8591040140495	2.88562458039237\\
4.86497428238551	2.84467566638736\\
4.86852086469722	2.79979071572883\\
4.86928123764474	2.75014022072918\\
4.86660428864334	2.69467734480257\\
4.85957658975848	2.63206514756602\\
4.84691261115477	2.56057504602007\\
4.82678946546552	2.47794396053811\\
4.79659542946469	2.38117226334323\\
4.75254376940376	2.26623814365281\\
4.68907718339365	2.12769852720028\\
4.59795460683918	1.95815040051454\\
4.46688732676326	1.74756400797157\\
4.27764254367676	1.48263691142847\\
4.00387072308215	1.14670604183995\\
3.61005153759665	0.721636574198892\\
3.05560765177234	0.194451774755332\\
2.31137241334251	-0.428642285968568\\
1.39083488530141	-1.10555625327907\\
0.373895154446771	-1.75913136062439\\
-0.617367064691563	-2.31148893173218\\
-1.48259799127861	-2.7245327663766\\
-2.18148886039909	-3.00546910573645\\
-2.72247859519357	-3.18406267255354\\
-3.1346724468484	-3.29168669210835\\
-3.44905165555013	-3.35275913373484\\
-3.69127710003747	-3.38401246048074\\
-3.88062555859937	-3.3962893159387\\
-4.03102237946802	-3.39646583053762\\
-4.15240860493939	-3.38888182308754\\
-4.25189304055679	-3.37628371252528\\
-4.33460233967622	-3.36041786217831\\
-4.40427697195088	-3.34239825578999\\
-4.46368138722626	-3.32293419830987\\
-4.51488543859349	-3.30247223687471\\
-4.55945837766816	-3.2812854333876\\
-4.59860372563644	-3.25953004009571\\
-4.63325399543124	-3.23728172962425\\
-4.66413790900644	-3.2145587924539\\
-4.69182854788939	-3.19133686150793\\
-4.71677809651419	-3.16755798725861\\
-4.7393429950025	-3.14313581339344\\
-4.75980208289272	-3.11795792443445\\
-4.7783694726755	-3.09188599300709\\
-4.7952033007698	-3.06475404861714\\
-4.81041106952656	-3.03636496063061\\
-4.8240519512485	-3.00648503446068\\
-4.83613612420751	-2.97483643062173\\
-4.84662090779453	-2.94108690350011\\
-4.85540311393893	-2.90483609045282\\
-4.8623065785354	-2.86559722460702\\
-4.86706320060209	-2.82277264472673\\
-4.86928487803478	-2.77562075794744\\
-4.86842229710538	-2.72321106614295\\
-4.86370430180861	-2.6643623334864\\
-4.85404804085803	-2.59755672352673\\
-4.83792446013161	-2.52081946986195\\
-4.81315470961862	-2.43154904828298\\
-4.77659879836072	-2.32627679894911\\
-4.72367605630488	-2.20032845352253\\
-4.64762643990108	-2.04735745318607\\
-4.53838863733007	-1.85873574998009\\
-4.38096984835562	-1.62286298531766\\
-4.1533355323116	-1.32469184340526\\
-3.8244975574566	-0.946371254917345\\
-3.35530815035159	-0.47108481288752\\
-2.7077897400668	0.106751713865016\\
-1.86973007851861	0.764620138831268\\
-0.887073673343907	1.44068914574646\\
0.132452543813392	2.05155563779595\\
1.06971589193757	2.5358237813157\\
1.85310797649221	2.87987586763252\\
2.47009835668971	3.10548017401548\\
2.94265469363239	3.24502618750772\\
3.30233078271589	3.32683325735637\\
3.57783719557537	3.37132200537939\\
3.79158168965869	3.39201965933028\\
3.95999323486845	3.39757254170735\\
};
\addplot [color=mycolor1, forget plot]
  table[row sep=crcr]{%
3.31964373338877	3.11680725665986\\
3.46059685762262	3.11247807026117\\
3.57708127662116	3.10146599881294\\
3.67451705414661	3.08596097142581\\
3.75696740880465	3.06740608789205\\
3.82749830257685	3.04674991857918\\
3.888442678605	3.02461109511954\\
3.94159191694989	3.001385629976\\
3.98833402720211	2.97731709974738\\
4.02975355906996	2.95254269648813\\
4.06670414816545	2.92712347062241\\
4.09986146259606	2.90106408349867\\
4.12976201336457	2.87432546960569\\
4.15683165226114	2.84683257984022\\
4.1814064231642	2.81847858065534\\
4.20374761128889	2.7891263560364\\
4.22405224219013	2.75860779629214\\
4.24245983997726	2.72672109262395\\
4.25905590451528	2.69322604436327\\
4.27387226537608	2.65783719413068\\
4.28688417700432	2.62021440850668\\
4.29800369589844	2.5799502930124\\
4.30706848081821	2.53655354267434\\
4.31382462055043	2.48942694947274\\
4.31790133406429	2.43783827225292\\
4.31877427644065	2.38088146767593\\
4.31571252742472	2.3174248153883\\
4.30770184951786	2.24604117926284\\
4.29333306429863	2.16491400132116\\
4.27063886204853	2.07171075141813\\
4.23685446381317	1.96341402908087\\
4.1880672114508	1.83610108349573\\
4.11870939430507	1.68466979149208\\
4.02084615921213	1.5025342124434\\
3.88324316513264	1.28138011224131\\
3.69033637688843	1.01122646403587\\
3.4216152440466	0.681343386361434\\
3.05276596181384	0.283009552357022\\
2.56107231208726	-0.184793005303621\\
1.9374093620001	-0.707248384379817\\
1.20187957948651	-1.24835241324659\\
0.40982998738374	-1.75746107812609\\
-0.365070108068133	-2.18915793181554\\
-1.06149298312255	-2.52144033937862\\
-1.64914225121043	-2.75749030492595\\
-2.12602172492128	-2.91479010698421\\
-2.50572442956788	-3.01384258513672\\
-2.80656397105716	-3.07222755005006\\
-3.04583480008297	-3.10306288904879\\
-3.2378125671698	-3.11548637709263\\
-3.39358276868888	-3.11565342567214\\
-3.5215221365268	-3.1076492923641\\
-3.62789922876217	-3.094170883302\\
-3.71740391909101	-3.07699615540657\\
-3.79356234576415	-3.05729574550448\\
-3.85904598931447	-3.03583683613959\\
-3.91589681262369	-3.01311602974166\\
-3.96568992832198	-2.98944609620787\\
-4.00965105193101	-2.96501280481785\\
-4.04874158413868	-2.93991225009947\\
-4.08372054805974	-2.91417532469041\\
-4.11518990057731	-2.88778359109191\\
-4.14362779027714	-2.86067927062269\\
-4.16941295659567	-2.83277108062239\\
-4.19284249089347	-2.80393700477406\\
-4.21414448452044	-2.77402464624158\\
-4.23348657969749	-2.74284950605136\\
-4.25098104927468	-2.71019129526367\\
-4.26668671068013	-2.67578819118601\\
-4.28060768645165	-2.63932875635692\\
-4.2926887203984	-2.60044102900901\\
-4.30280640314065	-2.55867803931154\\
-4.31075520152593	-2.51349867667858\\
-4.31622655132413	-2.46424239171588\\
-4.31877835652204	-2.4100956127698\\
-4.3177908838693	-2.35004692972493\\
-4.31240301138506	-2.28282697679007\\
-4.30141973612445	-2.20682747567492\\
-4.28317728538454	-2.11999210763607\\
-4.25534552110689	-2.01967006216038\\
-4.21463814624494	-1.90242229035895\\
-4.15639015745231	-1.76377354814919\\
-4.07395353425142	-1.59791747429286\\
-3.95787205311072	-1.39742388831325\\
-3.79486863777062	-1.15310163350004\\
-3.56691807683934	-0.854393647214405\\
-3.25126970548047	-0.491066383999079\\
-2.82334993058864	-0.057334736614125\\
-2.26532838218795	0.440940424271642\\
-1.58091915229243	0.978480466449083\\
-0.808389310351763	1.51014074160763\\
-0.0157820009785098	1.98502234731836\\
0.725768143862515	2.36801909836561\\
1.36956393520291	2.65058247727238\\
1.90079146402238	2.84467600208762\\
2.32689533909663	2.97039684409577\\
2.66486093238406	3.04719472764852\\
2.93292528781976	3.09043587806671\\
3.14697196975999	3.11113324325111\\
3.31964373338877	3.11680725665986\\
};
\addplot [color=mycolor1, forget plot]
  table[row sep=crcr]{%
2.8295327054462	2.9466008779749\\
2.97642610855213	2.94207499671692\\
3.09988970543158	2.93039295654103\\
3.20464856418559	2.91371517181489\\
3.29437580196887	2.89351712903362\\
3.37192808911391	2.87080038913761\\
3.43953494245618	2.84623802679657\\
3.4989453355168	2.82027388436162\\
3.55153916974506	2.79319002120546\\
3.59841129928026	2.76515244759292\\
3.64043465239194	2.73624202568855\\
3.67830760138359	2.70647516533049\\
3.71258947966441	2.6758173996928\\
3.74372712841474	2.64419188113415\\
3.77207456717094	2.61148412790016\\
3.79790728059861	2.5775438631922\\
3.82143215068827	2.54218444063185\\
3.84279369705492	2.50518009028548\\
3.86207698134672	2.46626100924553\\
3.87930725265861	2.42510613230112\\
3.89444612840362	2.38133322903582\\
3.90738378625799	2.3344857638394\\
3.91792624803269	2.28401570541228\\
3.92577631430403	2.22926116270718\\
3.93050598965004	2.16941733623002\\
3.93151722677788	2.10349879512476\\
3.92798638587754	2.03029053102864\\
3.91878579530399	1.94828466613264\\
3.90237305398243	1.855599309194\\
3.87663518959904	1.74987637235859\\
3.83867086386778	1.62815736097154\\
3.78449111775298	1.48674276649606\\
3.70862241492482	1.32105686703013\\
3.60361677607863	1.12557497438373\\
3.45953832854664	0.893939279584528\\
3.26365014721632	0.619506570978022\\
3.00082463472069	0.29671869953972\\
2.65561419860998	-0.0762714533761386\\
2.21706286502895	-0.493729627568006\\
1.68619671892557	-0.938656377115806\\
1.08285674531277	-1.38265831675536\\
0.445432578270779	-1.79241778806565\\
-0.179885110508808	-2.14072410941843\\
-0.754317926290556	-2.41469139112317\\
-1.25584181744111	-2.61602985214543\\
-1.67895470657398	-2.75549727534181\\
-2.02899033676413	-2.84673786287883\\
-2.31615813868755	-2.9024182014246\\
-2.55160205772157	-2.93272494003078\\
-2.74546895585324	-2.9452462777605\\
-2.90625184398168	-2.94540172336058\\
-3.0407635978437	-2.9369743970697\\
-3.15435598513096	-2.92257310831705\\
-3.25119590115092	-2.90398444924689\\
-3.33452168008777	-2.88242517854157\\
-3.40685557962428	-2.85871771507956\\
-3.47017090376185	-2.83341037990418\\
-3.52602004104249	-2.80685924361002\\
-3.57563130948749	-2.77928368215287\\
-3.6199817995631	-2.75080399443141\\
-3.65985205939119	-2.72146673256919\\
-3.69586712014066	-2.69126152679475\\
-3.72852722176749	-2.6601319168179\\
-3.7582307012757	-2.6279818414424\\
-3.78529081653201	-2.59467885057009\\
-3.80994775181099	-2.56005469350327\\
-3.83237664138479	-2.52390363935476\\
-3.85269211510605	-2.48597865435748\\
-3.87094958091437	-2.44598536461657\\
-3.88714318232659	-2.40357354660641\\
-3.9012000727569	-2.35832569061562\\
-3.91297029725293	-2.30974195490434\\
-3.92221112123775	-2.25722055102696\\
-3.92856403549743	-2.20003225407342\\
-3.93152181524005	-2.13728729900023\\
-3.9303818073913	-2.06789240167837\\
-3.92417992018739	-1.99049506249897\\
-3.91159742794709	-1.90341179254178\\
-3.89082955793276	-1.8045367757561\\
-3.85940100403848	-1.69122852929405\\
-3.81390986646538	-1.56017608957687\\
-3.74968089091811	-1.40725677726019\\
-3.66031887909048	-1.22742176886083\\
-3.53719150856458	-1.01469576302685\\
-3.3689732107208	-0.76246914035007\\
-3.14160337218006	-0.464400090228629\\
-2.83938580218212	-0.116368668956703\\
-2.44832025295945	0.280211398732743\\
-1.96241815215748	0.71430772072177\\
-1.39154278411818	1.16286197831489\\
-0.765517221975054	1.59379223854703\\
-0.1285047127005	1.97543774602584\\
0.475270323667959	2.28718535710219\\
1.01486948515288	2.52389873550162\\
1.47704271994541	2.69265356832579\\
1.86252585744045	2.80630441153345\\
2.17972734440079	2.87832260557126\\
2.43967596952195	2.92021193495773\\
2.65316087503536	2.9408255990419\\
2.8295327054462	2.9466008779749\\
};
\addplot [color=mycolor1, forget plot]
  table[row sep=crcr]{%
2.44984468888118	2.85779721664325\\
2.60196738533444	2.8530955341335\\
2.73201868937063	2.84077925552878\\
2.84399722427826	2.82294384498009\\
2.941130399199	2.80107235469133\\
3.026009095818	2.77620459012577\\
3.10071052533583	2.7490607249949\\
3.1669019408754	2.72012984881909\\
3.22592470970097	2.68973265485586\\
3.27886093326873	2.65806546150423\\
3.32658553570199	2.62523087231385\\
3.36980663514464	2.59125886400063\\
3.40909660163786	2.55612095481477\\
3.44491572902982	2.51973927917976\\
3.47763000266974	2.48199180044322\\
3.50752405799971	2.44271446448007\\
3.53481009681523	2.40170078042435\\
3.55963324491399	2.35869907193627\\
3.58207357934416	2.31340744425876\\
3.6021448055327	2.26546633725287\\
3.61978930212512	2.21444836608594\\
3.63486894898146	2.15984497665486\\
3.6471507808511	2.10104925358721\\
3.65628602800719	2.03733401243435\\
3.66178046940091	1.96782409416737\\
3.66295318110665	1.89146159371687\\
3.65887966452276	1.80696267972253\\
3.64831397236724	1.71276488449446\\
3.62958291927769	1.60696463932235\\
3.60044417174799	1.48724713914983\\
3.55790006148047	1.35081571892447\\
3.49796295032051	1.19433822014812\\
3.41538138482406	1.01394710682261\\
3.3033694664129	0.805363083825396\\
3.15345104039838	0.564260916986537\\
2.95565145013395	0.287049461849504\\
2.69942978217378	-0.0277551991169861\\
2.37583077534834	-0.377545202656914\\
1.98105202249717	-0.75349516448623\\
1.52061048003519	-1.13953849261922\\
1.01179793653135	-1.51406798747553\\
0.481729687576415	-1.85483926486565\\
-0.039299878752971	-2.14502112512604\\
-0.525728760766699	-2.37694655936211\\
-0.961583347612355	-2.55184349143257\\
-1.34080776834139	-2.67677340867116\\
-1.66466717479823	-2.76113363997588\\
-1.93852205206722	-2.81418918953832\\
-2.16930815024689	-2.84386430177315\\
-2.36401462859855	-2.85641646336314\\
-2.52894863991	-2.8565588079033\\
-2.66948238030248	-2.8477415602283\\
-2.7900488682189	-2.83244662824804\\
-2.89424415876269	-2.81243887826061\\
-2.98496100689268	-2.78896168880488\\
-3.06451923449325	-2.76288202714237\\
-3.13477937039516	-2.7347952654056\\
-3.19723637238164	-2.70509977050788\\
-3.25309462510342	-2.67404948529879\\
-3.30332692445318	-2.64179071027718\\
-3.34872037797878	-2.60838758321233\\
-3.38991184884979	-2.5738394337785\\
-3.42741510892757	-2.53809221853144\\
-3.46164139897658	-2.50104554062317\\
-3.49291467668763	-2.46255625454029\\
-3.52148247681545	-2.42243928901094\\
-3.54752300405808	-2.3804660457283\\
-3.57114881240569	-2.33636051412372\\
-3.59240717557897	-2.2897930581589\\
-3.61127700091922	-2.24037166112528\\
-3.62766186012688	-2.18763024423159\\
-3.64137837682092	-2.13101349361061\\
-3.65213878904816	-2.06985743237638\\
-3.65952595267066	-2.00336476214815\\
-3.66295831902879	-1.93057379188937\\
-3.66164145467123	-1.85031962616379\\
-3.65450143563783	-1.76118632626276\\
-3.64009397438109	-1.661449251915\\
-3.61648164448542	-1.54900826961948\\
-3.58107074079176	-1.42131599087942\\
-3.53040086531508	-1.27531251851639\\
-3.45988808475131	-1.10739240636343\\
-3.36354413203345	-0.913455098139927\\
-3.23374299194595	-0.689131260121456\\
-3.06120008839267	-0.430331666141548\\
-2.83547558989729	-0.134305568090806\\
-2.54646084629997	0.198661406724477\\
-2.18725706829663	0.563087783868987\\
-1.75823788126407	0.946519661247175\\
-1.2707262246671	1.329691649195\\
-0.747503230270968	1.68991502715918\\
-0.218296874824636	2.0069639827157\\
0.288086702015426	2.26836939515059\\
0.750566434261	2.47117499192935\\
1.15832749650522	2.61998617758171\\
1.50938451881729	2.72342291650602\\
1.80742617457819	2.7910408610157\\
2.05885144943164	2.83151927338772\\
2.27075651633564	2.85195298808745\\
2.44984468888117	2.85779721664325\\
};
\addplot [color=mycolor1, forget plot]
  table[row sep=crcr]{%
2.15083773120232	2.82869718267466\\
2.30732436468486	2.82384624621737\\
2.4433118420125	2.81095668645537\\
2.56209976980254	2.79202803374775\\
2.66645262443518	2.76852402002715\\
2.75866143124573	2.74150316040845\\
2.84061379406736	2.71172005533557\\
2.91386069740201	2.67970163413908\\
2.97967530357787	2.64580343658018\\
3.03910245765468	2.61025057614152\\
3.09299924612863	2.57316716626594\\
3.14206760332443	2.53459711237977\\
3.18688011313304	2.49451841871177\\
3.22790007742325	2.45285255800203\\
3.26549675106933	2.40946998904446\\
3.29995644163108	2.36419255373392\\
3.33148996622944	2.31679321381851\\
3.36023675623487	2.26699337395324\\
3.38626569865824	2.21445786304212\\
3.40957259224826	2.15878749628598\\
3.43007386289986	2.09950900667044\\
3.44759591049805	2.03606201314232\\
3.46185912854562	1.96778258734543\\
3.47245522847033	1.89388290774894\\
3.47881599515294	1.81342648696486\\
3.48017099572729	1.72529860068735\\
3.47549109541112	1.6281719775772\\
3.46341402797019	1.52046878617669\\
3.44214804512423	1.40032192500813\\
3.40935054063824	1.26554232467367\\
3.3619819544403	1.11360553644566\\
3.29614389481357	0.941681805088982\\
3.2069287312535	0.746750454613378\\
3.08834196896464	0.525861154298206\\
2.93341318106459	0.276624430611918\\
2.7346792123074	-0.00199005653599627\\
2.48526439126132	-0.308540843598514\\
2.18069789166611	-0.637876894190381\\
1.82126405509475	-0.980283585224052\\
1.41407042577293	-1.32177794963178\\
0.973529098553298	-1.64611233998187\\
0.519302179515138	-1.93814165635475\\
0.0721749089467226	-2.18714232225843\\
-0.350271619857278	-2.38851645064079\\
-0.736296205869651	-2.54336468492444\\
-1.08035082362415	-2.65665726702385\\
-1.3818401671288	-2.73514641622193\\
-1.64334934794763	-2.7857745520406\\
-1.86907351690406	-2.81477095858355\\
-2.0637212393411	-2.82729792441939\\
-2.23187175557143	-2.82742663992165\\
-2.37766114657535	-2.81826702500692\\
-2.50467106829572	-2.80214488610198\\
-2.61592711340404	-2.78077361439386\\
-2.71394841871074	-2.75539991373785\\
-2.80081557583602	-2.72691936427101\\
-2.87823996979546	-2.6959645656945\\
-2.94762690053014	-2.66297077414461\\
-3.01012976176274	-2.62822399478868\\
-3.06669497313819	-2.59189576090128\\
-3.11809842125759	-2.55406793083113\\
-3.16497452169875	-2.51475000986999\\
-3.20783902965818	-2.47389082687288\\
-3.24710659152563	-2.43138586657657\\
-3.28310383789386	-2.38708115401469\\
-3.31607861296981	-2.340774278353\\
-3.3462057315549	-2.29221290360805\\
-3.37358945379951	-2.2410909216505\\
-3.39826266307694	-2.18704224225232\\
-3.4201825120606	-2.12963207414236\\
-3.43922205140252	-2.06834542350988\\
-3.45515705631335	-2.00257242185593\\
-3.46764689873355	-1.93159000279873\\
-3.4762078570285	-1.85453940282288\\
-3.48017669902106	-1.7703990179856\\
-3.47866172938115	-1.6779524117519\\
-3.47047783053574	-1.5757519309959\\
-3.45406155302938	-1.4620797890869\\
-3.42736251064461	-1.33491119630422\\
-3.3877092317964	-1.19188908423251\\
-3.33165320346531	-1.03032851580001\\
-3.2548076381554	-0.847282536978453\\
-3.15172288291488	-0.639720740842309\\
-3.01588443066694	-0.404894147748275\\
-2.8399829976851	-0.140972001347189\\
-2.61666888523843	0.151994797623094\\
-2.33999732459845	0.470856797629973\\
-2.00757182915667	0.808234985349818\\
-1.62289674730295	1.15214255424824\\
-1.1968205700236	1.48710610175714\\
-0.746798289601564	1.7969696755341\\
-0.293634844758991	2.06845614829498\\
0.143018691654359	2.29382996891854\\
0.548319090065009	2.47151065383526\\
0.913702541845487	2.60480312130623\\
1.23630864336571	2.69980902609968\\
1.51735250612403	2.76353078512541\\
1.76039282780355	2.80262768372586\\
1.9699877541269	2.82281419935843\\
2.15083773120232	2.82869718267466\\
};
\addplot [color=mycolor1, forget plot]
  table[row sep=crcr]{%
1.91158019794265	2.84376903183464\\
2.07163776438853	2.83879386724661\\
2.21285807640053	2.82539747440454\\
2.33791246105122	2.80546152865478\\
2.44912082680197	2.7804062830365\\
2.54846600359532	2.75128830803495\\
2.63762577791934	2.71888098047477\\
2.71801084845853	2.68373809671895\\
2.79080242003554	2.64624280460785\\
2.85698641658264	2.60664449345616\\
2.91738312264264	2.5650861096821\\
2.9726720213868	2.52162396989751\\
3.02341205008208	2.47624170916187\\
3.07005766382338	2.42885960651774\\
3.11297111497144	2.37934019668158\\
3.15243129157618	2.32749080597833\\
3.18863935121423	2.27306343349177\\
3.22172125493937	2.21575222433704\\
3.25172715531842	2.15518864183414\\
3.27862742163927	2.0909343333344\\
3.30230488909748	2.02247159947225\\
3.32254269007925	1.94919132453424\\
3.33900675849467	1.87037822283741\\
3.35122179204887	1.78519333555837\\
3.35853912660402	1.69265393427859\\
3.3600946677081	1.59161145435502\\
3.35475484794567	1.48072896213269\\
3.34104876892245	1.35846122100487\\
3.31708569791958	1.22304305571737\\
3.28045974439537	1.07249594527396\\
3.22814921378437	0.904669156349769\\
3.15642885792177	0.717340480465908\\
3.06083139967709	0.508411707762838\\
2.93622152863079	0.276241139701024\\
2.7770770319997	0.020149459472978\\
2.57809096843142	-0.258902162895629\\
2.3351780468082	-0.557553080740128\\
2.04683497078149	-0.869439194448154\\
1.71554714297952	-1.18511784885378\\
1.34865220465666	-1.49288119946767\\
0.958021807609145	-1.78050890078937\\
0.558338454421943	-2.03748111201489\\
0.164481555358333	-2.25680166315308\\
-0.21095810717441	-2.43573832577295\\
-0.559170187615061	-2.57538181366409\\
-0.875387966569343	-2.67947102389358\\
-1.15825539247116	-2.75307858502423\\
-1.40881879363976	-2.80155900902423\\
-1.62954531860348	-2.82988991803198\\
-1.82356115063245	-2.84235727819136\\
-1.99414405767677	-2.8424727142264\\
-2.14442741054747	-2.8330186505085\\
-2.27725282196893	-2.81614859152382\\
-2.39511589422416	-2.79350028839508\\
-2.50016473838548	-2.76630098678658\\
-2.59422508977545	-2.73545669331649\\
-2.6788363688267	-2.70162402151971\\
-2.75528999541142	-2.66526614123614\\
-2.82466552950899	-2.62669536187234\\
-2.88786267206039	-2.58610494914208\\
-2.94562849673728	-2.54359245794157\\
-2.99857995368833	-2.49917643438195\\
-3.04722197671781	-2.45280791977808\\
-3.09196160662951	-2.404377823958\\
-3.13311851320763	-2.35372093401608\\
-3.17093220955621	-2.30061708200279\\
-3.20556613189311	-2.24478980073838\\
-3.23710861645723	-2.18590264080907\\
-3.26557064483797	-2.12355319619622\\
-3.29088004627111	-2.05726478724901\\
-3.31287163385656	-1.98647567989964\\
-3.33127250427905	-1.91052568967319\\
-3.34568144341957	-1.82864005190303\\
-3.35554105860475	-1.73991058053854\\
-3.3601009295058	-1.64327446592406\\
-3.35836980714755	-1.53749170939389\\
-3.34905485717965	-1.42112337286191\\
-3.33048646848317	-1.29251486138864\\
-3.30052884841376	-1.1497918177212\\
-3.25648057704918	-0.990881452334094\\
-3.19497719768095	-0.813579719624154\\
-3.11192207726117	-0.615694432781554\\
-3.00249429743133	-0.395303860048757\\
-2.8613125677913	-0.151172713601745\\
-2.68286279053364	0.11665152416492\\
-2.46229728022884	0.40610058150193\\
-2.19663792258006	0.712364312934084\\
-1.88621757371078	1.02749962503193\\
-1.53590108495125	1.34076629312927\\
-1.15542361646115	1.63993480818183\\
-0.758353994407265	1.91336173347405\\
-0.359817457823202	2.15211845573736\\
0.0261883777012717	2.35132769387831\\
0.38885595919089	2.51028408905373\\
0.721431866190724	2.63157073461392\\
1.02096906501805	2.71974731916771\\
1.28744399189816	2.7801346433684\\
1.52272459753807	2.81795725135526\\
1.72968641530442	2.83786903536155\\
1.91158019794265	2.84376903183464\\
};
\addplot [color=mycolor1, forget plot]
  table[row sep=crcr]{%
1.71745758679747	2.89173975132996\\
1.8804644346493	2.88666054980203\\
2.02627889269441	2.87281811389132\\
2.15703696678287	2.85196438812595\\
2.27465720289688	2.82545742453582\\
2.38082766665929	2.79433294485042\\
2.47701312392245	2.75936681771373\\
2.56447232708102	2.72112681113683\\
2.64427925055575	2.68001399481002\\
2.71734473025739	2.63629501379885\\
2.78443660076244	2.59012667981921\\
2.84619740298685	2.54157425144665\\
2.90315928776839	2.49062457766663\\
2.95575602681665	2.43719505069377\\
3.00433216544909	2.38113909602147\\
3.04914937549984	2.32224873683445\\
3.09039002973783	2.26025461070945\\
3.12815794227413	2.19482368873312\\
3.16247611357899	2.12555485016604\\
3.19328118862104	2.05197240139194\\
3.22041418435718	1.97351760285347\\
3.24360687108098	1.88953829664735\\
3.26246300883696	1.79927683719634\\
3.27643346532593	1.70185676287416\\
3.28478411983947	1.59626907917125\\
3.28655547768413	1.48135976265904\\
3.28051324744219	1.35582129585754\\
3.26509006168158	1.21819291246397\\
3.23832053313869	1.06687700254827\\
3.1977756659706	0.900182964836113\\
3.14050925870732	0.716414564032025\\
3.06303931277295	0.514021630017732\\
2.96140177152125	0.291839165060381\\
2.831329743771	0.049431359624268\\
2.66862073969067	-0.212463672502666\\
2.46974101058715	-0.491439304059168\\
2.23265724939228	-0.782998939691912\\
1.95776837223382	-1.08040459012058\\
1.64866160836398	-1.37500967383526\\
1.31233089204491	-1.65718103512082\\
0.958592565229468	-1.91767088860956\\
0.598741748159326	-2.14903977073193\\
0.243852707791889	-2.34665123377138\\
-0.0967147415893769	-2.50894703232191\\
-0.416191996672758	-2.63704100182664\\
-0.710574296148952	-2.73391547893574\\
-0.978269118005141	-2.80354919199455\\
-1.2194938979658	-2.85019993873972\\
-1.43564881129259	-2.87792454612846\\
-1.62879099653703	-2.89031939215795\\
-1.80125105253692	-2.8904225079943\\
-1.95538246568011	-2.88071512412685\\
-2.09341468931788	-2.86317442696131\\
-2.21737842377636	-2.839346057336\\
-2.32907704654099	-2.81041845139936\\
-2.43008526104953	-2.77729025756574\\
-2.52176231073087	-2.74062749478147\\
-2.60527181932884	-2.70091001046648\\
-2.68160355032981	-2.65846814186684\\
-2.75159446004058	-2.61351096804578\\
-2.81594769357519	-2.56614758502651\\
-2.87524891462383	-2.51640268507215\\
-2.92997976318596	-2.46422750077732\\
-2.98052842983881	-2.40950694815937\\
-3.02719740245919	-2.35206359744722\\
-3.07020843125574	-2.29165892510769\\
-3.10970469913453	-2.22799215727647\\
-3.14575009217691	-2.16069690234515\\
-3.17832534667139	-2.08933568928717\\
-3.20732070773014	-2.01339248233482\\
-3.23252457198269	-1.93226324211468\\
-3.25360740779271	-1.84524466839372\\
-3.27010006373279	-1.75152142513656\\
-3.28136541998481	-1.65015247165899\\
-3.2865622713529	-1.54005769300379\\
-3.28460047717549	-1.42000696964363\\
-3.27408699466891	-1.28861533321583\\
-3.25326380716356	-1.14435014645906\\
-3.21994156970062	-0.985559539435554\\
-3.17143788509773	-0.81053568915016\\
-3.1045375211571	-0.61763149825431\\
-3.01550430731523	-0.405453196239712\\
-2.90019008781384	-0.173150576686815\\
-2.7543001876026	0.0791858516391993\\
-2.57387541216259	0.350044816530449\\
-2.35601728856905	0.636015792033466\\
-2.09979503696684	0.931474644994217\\
-1.80713233468106	1.22865077950936\\
-1.48333918421578	1.51825424663493\\
-1.13694684041011	1.79065828945332\\
-0.778714076415862	2.03735792720416\\
-0.42003797068775	2.25223275015673\\
-0.0712897935506828	2.43219826408154\\
0.259402493638034	2.57711605372954\\
0.566669600263679	2.6891458577117\\
0.847778827249968	2.77187104282629\\
1.10212364312125	2.82948518216731\\
1.33058671380521	2.86619067025825\\
1.53495371890201	2.88583489506905\\
1.71745758679747	2.89173975132996\\
};
\addplot [color=mycolor1, forget plot]
  table[row sep=crcr]{%
1.55816621582833	2.96411764280911\\
1.72369026381853	2.9589489634012\\
1.87357395094473	2.94471081198243\\
2.00951539727676	2.92302238097717\\
2.13308772686911	2.89516714394286\\
2.24571234639	2.8621446196781\\
2.34865136405567	2.8247181742914\\
2.44301121000513	2.78345639595214\\
2.52975215041391	2.73876741695496\\
2.60970030106496	2.69092650013397\\
2.68356005581673	2.64009761362272\\
2.75192569999174	2.58634982836324\\
2.81529151105137	2.52966933299994\\
2.87405996181646	2.46996775540036\\
2.9285478083254	2.40708735530879\\
2.97898991488656	2.34080353268013\\
3.02554067531358	2.27082499337512\\
3.06827285217283	2.196791835172\\
3.1071735880505	2.11827176852309\\
3.14213725341473	2.03475467638285\\
3.17295469322562	1.94564575898457\\
3.19929833125135	1.85025762396374\\
3.22070250845181	1.74780190281111\\
3.23653840868177	1.63738135042231\\
3.24598303048299	1.51798398475037\\
3.24798201588985	1.38848173911886\\
3.24120693772994	1.24763743617153\\
3.22400916860176	1.09412574273013\\
3.19437511598683	0.926576139898226\\
3.14989189658064	0.74364862494752\\
3.087738838116	0.54415514122812\\
3.00472841444392	0.327240017128452\\
2.89742883428089	0.0926281100414999\\
2.76240535369208	-0.159064235610916\\
2.59661056566265	-0.42598689659216\\
2.39792476504199	-0.704753322834074\\
2.16578860862381	-0.990290191232043\\
1.90179012891402	-1.27596971649145\\
1.61000317878624	-1.55411412857669\\
1.2968823130403	-1.8168464843066\\
0.970638174303505	-2.0571084413693\\
0.640213756305188	-2.26956197274868\\
0.314147274549137	-2.45111751934481\\
-0.000362505488881228	-2.60098113264671\\
-0.297988270820512	-2.720295012052\\
-0.575378380735594	-2.81155756868797\\
-0.830941166451551	-2.87801576920823\\
-1.06446367709236	-2.92315885645935\\
-1.27669672435658	-2.95036437111437\\
-1.46898741728075	-2.96269063235406\\
-1.64299300668765	-2.96278271928243\\
-1.80047857881911	-2.95285388917795\\
-1.94318577414053	-2.93471041232882\\
-2.07275514816701	-2.90979707621904\\
-2.19068588816658	-2.87924906665681\\
-2.29831984755406	-2.843942221781\\
-2.39684037593059	-2.80453778741935\\
-2.48727941435114	-2.76152026863806\\
-2.57052858919337	-2.71522830970021\\
-2.64735163136145	-2.66587917238847\\
-2.71839651149636	-2.61358761803144\\
-2.78420636050438	-2.55838002010735\\
-2.84522865642899	-2.5002044539993\\
-2.90182239101521	-2.43893739131148\\
-2.95426304297833	-2.37438750172038\\
-3.00274522025381	-2.30629695283276\\
-3.04738281614693	-2.23434050698934\\
-3.08820647051653	-2.15812264950087\\
-3.12515804747891	-2.07717295221539\\
-3.15808174399543	-1.990939889959\\
-3.1867113389556	-1.89878340212204\\
-3.2106529959443	-1.79996665418033\\
-3.22936297363835	-1.69364774480912\\
-3.24211962779781	-1.57887258222754\\
-3.24798929893982	-1.45457089871028\\
-3.24578622146153	-1.31955848477936\\
-3.23402770229988	-1.17255030912029\\
-3.21088785082031	-1.01219131125092\\
-3.17415655487047	-0.837114232777981\\
-3.12121566819313	-0.646036467539452\\
-3.04905171671751	-0.437909464786748\\
-2.95433316852084	-0.212132517993081\\
-2.83358778416639	0.0311657486123974\\
-2.68351580771023	0.290794598039761\\
-2.50145812401623	0.564167055530497\\
-2.28599477919338	0.847057253091699\\
-2.03757711193584	1.13357592830069\\
-1.75901774854111	1.41648320142122\\
-1.45562685451183	1.68787901637399\\
-1.13484527909285	1.94016869125852\\
-0.805392760759659	2.16705995139283\\
-0.476146011004946	2.36430279892949\\
-0.155067833387058	2.52997857399797\\
0.151540333209896	2.66432503572871\\
0.439349481079671	2.76924068531352\\
0.705928858253586	2.84766989253353\\
0.950428454900413	2.90303490669397\\
1.17316738454906	2.93880362351463\\
1.37523579774972	2.95821190730456\\
1.55816621582833	2.96411764280911\\
};
\addplot [color=mycolor1, forget plot]
  table[row sep=crcr]{%
1.42632528730357	3.05419055216814\\
1.59409885697326	3.04894192904819\\
1.74764607607782	3.03434727439337\\
1.88831768585849	3.01189675254528\\
2.01740026430351	2.98279289019926\\
2.13608416558777	2.94798796969752\\
2.24544811665552	2.90822043063386\\
2.34645448710049	2.86404765904994\\
2.43995094101876	2.81587408211711\\
2.52667551525986	2.76397436887494\\
2.60726315074524	2.7085119950182\\
2.68225238936669	2.64955362911531\\
2.75209140658278	2.58707985082535\\
2.81714284020386	2.52099268940941\\
2.8776870472672	2.45112041621127\\
2.93392350905193	2.37721996369354\\
2.98597013463019	2.29897729217643\\
3.03386020403657	2.2160059952028\\
3.07753665694582	2.12784443619997\\
3.11684338412892	2.0339517557063\\
3.15151313143606	1.93370319745457\\
3.18115160055983	1.82638539774865\\
3.20521736059578	1.71119259976722\\
3.22299732316753	1.58722523687895\\
3.23357786670449	1.45349302920453\\
3.23581235158743	1.3089257071312\\
3.2282869316483	1.15239574159971\\
3.20928847715888	0.982758977142025\\
3.176781336107	0.798920617843453\\
3.12840373601196	0.599935084967989\\
3.06149968552777	0.38514782253431\\
2.97320730405123	0.154383494569103\\
2.86062722894035	-0.0918240253086897\\
2.72109091958337	-0.351980889021323\\
2.55253274510374	-0.623406293831113\\
2.35393750479549	-0.902099020098744\\
2.12578916758129	-1.1827809605491\\
1.87040277410249	-1.45918525535049\\
1.59200788692452	-1.72459900904301\\
1.29649516293625	-1.97258135199058\\
0.990837210384202	-2.197696081303\\
0.682309937938043	-2.39607365215453\\
0.377709629271056	-2.56567186636938\\
0.0827467276445914	-2.70621088488075\\
-0.198285197101544	-2.81885870156143\\
-0.462566309044838	-2.90579312858928\\
-0.708603694444559	-2.96975905169869\\
-0.935974346036651	-3.01369849093327\\
-1.14503867672095	-3.04048466100225\\
-1.33667724745256	-3.05275737054036\\
-1.51207595388432	-3.05283983637206\\
-1.67256535682076	-3.04271255194964\\
-1.81950894294362	-3.02402253127754\\
-1.95423070636569	-2.99811154608017\\
-2.07797191745059	-2.96605230356283\\
-2.19186825631255	-2.92868580072754\\
-2.29694038051205	-2.8866561327032\\
-2.3940928301155	-2.84044100242064\\
-2.48411769306366	-2.7903773569584\\
-2.56770060761846	-2.73668222109411\\
-2.64542750422993	-2.67946910987757\\
-2.71779105258215	-2.61876051726236\\
-2.7851961466965	-2.55449698645495\\
-2.84796398665464	-2.48654322508971\\
-2.9063344415271	-2.4146916680026\\
-2.96046643468488	-2.33866383248366\\
-3.01043610126227	-2.25810976873771\\
-3.05623244371903	-2.1726058926298\\
-3.09775016794598	-2.08165151011889\\
-3.13477933218787	-1.98466441800741\\
-3.16699140132554	-1.88097611459704\\
-3.19392129598201	-1.76982740563066\\
-3.21494510178343	-1.65036558457997\\
-3.2292533268044	-1.52164495079917\\
-3.23582007028983	-1.38263325815509\\
-3.2333693489984	-1.23222780416891\\
-3.22034133184322	-1.06928627425662\\
-3.19486361413493	-0.892679032711538\\
-3.15473614964176	-0.701370955594936\\
-3.09744308405396	-0.494541380754747\\
-3.02020998960036	-0.271748964553959\\
-2.92012930110728	-0.0331421889446298\\
-2.79437683989129	0.220296299061973\\
-2.64053313743929	0.4865035954473\\
-2.45699943943893	0.762146269718007\\
-2.24345818717511	1.04256505003492\\
-2.00128002875728	1.3219347994068\\
-1.73374676631364	1.59368386301882\\
-1.44597222977026	1.85114025281027\\
-1.14447704690877	2.08828044420864\\
-0.836487911591588	2.30039820704417\\
-0.529131242108844	2.48452621841593\\
-0.228720064349343	2.63952985558765\\
0.05972043374006	2.76590334036294\\
0.332641313986093	2.86537701210694\\
0.587913607060129	2.94046425247022\\
0.824614911507096	2.99404850797217\\
1.0427491736426	3.02906396360623\\
1.24296707789219	3.04828205038319\\
1.42632528730357	3.05419055216814\\
};
\addplot [color=mycolor1, forget plot]
  table[row sep=crcr]{%
1.31655348510211	3.15637921267897\\
1.48642982758109	3.15105639663561\\
1.6433349255677	3.13613505069415\\
1.78835014147596	3.11298458113963\\
1.92253093721614	3.08272520919187\\
2.04687406090608	3.04625525378925\\
2.1622986584819	3.00427899608015\\
2.2696368592467	2.95733267011823\\
2.36963046158222	2.9058073600355\\
2.46293125979115	2.84996834893613\\
2.55010326760306	2.78997090464124\\
2.63162562069553	2.72587271692795\\
2.70789531091329	2.65764330257214\\
2.7792291529804	2.5851707245444\\
2.8458645409221	2.50826596733315\\
2.90795864101281	2.42666529630212\\
2.96558571049066	2.34003092262652\\
3.01873224249287	2.24795031068524\\
3.06728963207688	2.14993451539684\\
3.11104405126304	2.04541603839254\\
3.14966323256332	1.93374686338077\\
3.18267991879898	1.81419759669302\\
3.20947188448626	1.68595902752834\\
3.22923873353934	1.54814796410214\\
3.24097621960527	1.39981992001385\\
3.24344974021805	1.23999211740366\\
3.23517007051758	1.06768127718753\\
3.21437646873718	0.881961600425187\\
3.17903506682906	0.682048823881183\\
3.12686378827857	0.467415571580784\\
3.05539826716717	0.237940380069017\\
2.9621149441898	-0.00591354243759519\\
2.8446252500433	-0.262904611596984\\
2.70094538850893	-0.530834152049075\\
2.52982713787425	-0.806428962524813\\
2.33110705889611	-1.08534184026105\\
2.10600184068736	-1.36232084982776\\
1.8572609218841	-1.63156753781995\\
1.58910066947964	-1.88725091001963\\
1.30689405569773	-2.12408573046757\\
1.01666210925465	-2.33784925911224\\
0.724476500123249	-2.52572184134781\\
0.435905312060046	-2.68639164379551\\
0.155607403891534	-2.819935532752\\
-0.112877160118432	-2.92754352702366\\
-0.367152187414235	-3.01117485959625\\
-0.605855357627645	-3.07322211607211\\
-0.828473614608814	-3.11623165194633\\
-1.03513592590833	-3.14269931239132\\
-1.22641845594262	-3.15493938537975\\
-1.40318038950259	-3.15501361429271\\
-1.56643609265679	-3.14470381180652\\
-1.71726173041519	-3.12551291097201\\
-1.85673089174345	-3.09868249350657\\
-1.98587279537864	-3.06521830682594\\
-2.10564706617362	-3.02591824464424\\
-2.21693006990254	-2.98139949136202\\
-2.32050890884273	-2.93212306146033\\
-2.41708018607659	-2.87841494577717\\
-2.50725146172323	-2.82048366252365\\
-2.59154394175497	-2.75843433336419\\
-2.67039538429476	-2.69227956197409\\
-2.7441625133768	-2.62194745266871\\
-2.81312242894797	-2.54728711579714\\
-2.87747262214401	-2.46807199484482\\
-2.93732926870396	-2.3840013380924\\
-2.99272349836275	-2.29470014050124\\
-3.04359533907467	-2.19971791269805\\
-3.08978502654407	-2.09852670780088\\
-3.13102136894168	-1.99051897056593\\
-3.16690688734356	-1.87500598855174\\
-3.19689954892627	-1.7512180482722\\
-3.22029112401107	-1.61830786078233\\
-3.2361826053834	-1.47535944913705\\
-3.24345783448158	-1.32140549834562\\
-3.24075761963484	-1.15545713000458\\
-3.22645835672939	-0.976551065658248\\
-3.1986615884	-0.783819914013145\\
-3.15520403655315	-0.576591333807192\\
-3.09370104580571	-0.354520223017647\\
-3.01163908232814	-0.117753668347263\\
-2.90653301678303	0.13288011034839\\
-2.77615849394102	0.39568075177871\\
-2.61885563696615	0.667920990033036\\
-2.43387633491964	0.94578097461004\\
-2.22171710630155	1.2244281019515\\
-1.98435422762312	1.49828116328787\\
-1.72529430245151	1.76145443214308\\
-1.4493851392521	2.00831828601761\\
-1.16239594134893	2.23406294420578\\
-0.870448039756891	2.4351388225132\\
-0.579423174651817	2.60948215525256\\
-0.294473622727724	2.7565021318834\\
-0.0197126607919893	2.87687296629366\\
0.241898300129466	2.97221327049281\\
0.488499802108194	3.04473818059727\\
0.719179682800395	3.09694755689515\\
0.933771636258873	3.1313832330886\\
1.13265092126358	3.15046252545726\\
1.31655348510211	3.15637921267897\\
};
\addplot [color=mycolor1, forget plot]
  table[row sep=crcr]{%
1.2248568044801	3.26583710005082\\
1.39676323735686	3.2604434592983\\
1.55679049023156	3.24521862229108\\
1.70581451982452	3.221422185182\\
1.84470834389625	3.19009450224251\\
1.97431066725544	3.15207701567812\\
2.09540597774779	3.10803389929256\\
2.20871277539915	3.05847284285746\\
2.31487729233147	3.00376378484091\\
2.41447069016377	2.94415504221871\\
2.50798823846962	2.87978668139907\\
2.59584937599353	2.81070120318579\\
2.67839784791313	2.73685173789504\\
2.75590131620158	2.65810800822904\\
2.82854997541634	2.57426034850786\\
2.89645379062483	2.48502209186076\\
2.95963802332905	2.3900306692737\\
3.01803673979666	2.28884782091383\\
3.07148401943925	2.18095941565358\\
3.11970261709477	2.06577552528384\\
3.16228990657625	1.94263162365297\\
3.19870107764335	1.81079209738987\\
3.2282298224622	1.66945768180885\\
3.24998719594592	1.51777898202722\\
3.26288005083467	1.35487889119504\\
3.26559152860871	1.17988741327663\\
3.25656761648345	0.991992989171771\\
3.23401578224707	0.790514626554382\\
3.19592403882969	0.57499846910328\\
3.140111030269	0.345340222336652\\
3.06431894751212	0.101930268413435\\
2.96635975428256	-0.154189284546743\\
2.84431936093831	-0.421176800253831\\
2.69681232712494	-0.696285159889052\\
2.52326143687634	-0.97583839918236\\
2.32415570313803	-1.25533021869393\\
2.10122515708008	-1.52966670070289\\
1.85747174494847	-1.79354229005829\\
1.59701994460964	-2.04189681903875\\
1.32479505155844	-2.27036877288047\\
1.04608570702848	-2.47565308395311\\
0.766079005791641	-2.65569655093843\\
0.489457644572723	-2.80971028876742\\
0.220121090286436	-2.9380259158805\\
-0.0389481193804397	-3.04185234037724\\
-0.285688301139277	-3.12299646489722\\
-0.518869419207026	-3.18359900414887\\
-0.737952297696529	-3.22591629846879\\
-0.942931433327034	-3.2521596106048\\
-1.1341854907516	-3.26438966961406\\
-1.31234862482834	-3.26445694460304\\
-1.47820740793431	-3.25397586124015\\
-1.63262284267026	-3.23432191795935\\
-1.77647420393349	-3.20664275108662\\
-1.91062047148805	-3.17187657033859\\
-2.03587515139584	-3.13077349442778\\
-2.15299081170959	-3.08391696630888\\
-2.26265034563407	-3.0317436138914\\
-2.36546264670328	-2.97456072077281\\
-2.46196095618675	-2.9125609786977\\
-2.55260259969955	-2.84583449679776\\
-2.63776917271905	-2.77437821224611\\
-2.71776647992242	-2.69810293469696\\
-2.79282370044644	-2.61683829985855\\
-2.86309135893114	-2.53033593213455\\
-2.92863774711611	-2.4382711419608\\
-2.98944347778396	-2.3402435260626\\
-3.04539387669787	-2.23577691297095\\
-3.0962689453209	-2.12431921703683\\
-3.14173067841532	-2.00524294895247\\
-3.1813076247535	-1.87784739843053\\
-3.214376776523	-1.74136387446161\\
-3.24014321977739	-1.59496587513795\\
-3.25761854918551	-1.43778666219383\\
-3.26559993659443	-1.26894740114535\\
-3.2626530391848	-1.0875996971195\\
-3.24710370237166	-0.892986799456058\\
-3.21704561516177	-0.684527589669214\\
-3.17037344658949	-0.461926123467113\\
-3.1048528720396	-0.225306196851926\\
-3.01823906938497	0.024635631298619\\
-2.90845193438096	0.286474238038438\\
-2.77380746201065	0.557924329770409\\
-2.61328940594513	0.835770476163347\\
-2.42682507743038	1.11590042265069\\
-2.21550988553163	1.39347465647061\\
-1.98171681985069	1.66323968274924\\
-1.72903932359748	1.91995352116603\\
-1.46205156141205	2.15885263362157\\
-1.18591914600036	2.37606806816528\\
-0.905936065855446	2.56890781903211\\
-0.62708081388796	2.73596014692679\\
-0.353670263989849	2.87702212913656\\
-0.0891533241573329	2.99289809453567\\
0.163954213030953	3.0851308914121\\
0.404021800262236	3.15572495189461\\
0.630184158086769	3.2069025595404\\
0.842188542214752	3.24091398383363\\
1.04023911444572	3.25990527538251\\
1.2248568044801	3.26583710005082\\
};
\addplot [color=mycolor1, forget plot]
  table[row sep=crcr]{%
1.14821861877544	3.37822194099113\\
1.32211422966463	3.37275975697398\\
1.48506577373199	3.35725102144531\\
1.63779453771928	3.33285774558015\\
1.78103188774723	3.30054550382355\\
1.91549014928952	3.26109905551922\\
2.04184294580084	3.21513954137717\\
2.16071243679431	3.1631413777036\\
2.27266135883972	3.10544775489936\\
2.37818821564684	3.04228417982872\\
2.47772434226298	2.97376984761565\\
2.57163187060185	2.89992684386039\\
2.66020185224264	2.82068730874675\\
2.74365196007394	2.73589877482412\\
2.82212330574859	2.64532794657016\\
2.89567598835189	2.54866324231546\\
2.96428304456024	2.44551648425036\\
3.02782251609899	2.33542421486215\\
3.08606740302247	2.21784925275971\\
3.13867335214308	2.09218329214815\\
3.18516406667791	1.9577516125178\\
3.22491465336956	1.81382130969792\\
3.25713349629051	1.65961488879327\\
3.28084382325812	1.49433155665004\\
3.29486697834341	1.31717906207993\\
3.29781058888808	1.12741933821346\\
3.28806632819085	0.924431288084098\\
3.2638237265217	0.707793477853093\\
3.22310816963416	0.477387781564748\\
3.16385222464998	0.233521602091018\\
3.08400873558892	-0.0229392581382749\\
2.98171041137278	-0.290442608907457\\
2.85547265898182	-0.566650817741802\\
2.70442391469964	-0.848401476834022\\
2.52853252816493	-1.13175937750553\\
2.32878609395836	-1.41218193840917\\
2.10727494205935	-1.68479879629618\\
1.86714241956969	-1.9447765374198\\
1.6123914789193	-2.18771138208102\\
1.34757265013792	-2.40997843292828\\
1.07740931147432	-2.60897387674382\\
0.806429669689267	-2.78321434775047\\
0.538666472900795	-2.93229411050522\\
0.277460863216762	-3.05673154096928\\
0.0253774981168844	-3.15775215772752\\
-0.215785267024473	-3.23705505106691\\
-0.444918584837936	-3.29659819762263\\
-0.661489845233607	-3.33842296110853\\
-0.865417916930842	-3.36452454422826\\
-1.05695405637369	-3.37676584910671\\
-1.23657808233503	-3.37682731799263\\
-1.40491358433021	-3.36618381742783\\
-1.56266213479399	-3.34610016813901\\
-1.71055440301189	-3.31763840952904\\
-1.84931521840959	-3.28167160219613\\
-1.97963952727928	-3.23890053206647\\
-2.10217647246083	-3.18987093478668\\
-2.21751926609763	-3.1349897903404\\
-2.3261989864738	-3.07453988823723\\
-2.42868084382842	-3.0086922954671\\
-2.52536180090137	-2.93751663369261\\
-2.61656869841968	-2.86098924012431\\
-2.70255623138843	-2.7789993881692\\
-2.78350426090654	-2.69135380942585\\
-2.8595140414808	-2.59777981075905\\
-2.9306030087679	-2.49792733707459\\
-2.996697820998	-2.39137040748995\\
-3.0576253941722	-2.2776084643912\\
-3.11310173490182	-2.15606833597676\\
-3.16271847950798	-2.02610773800262\\
-3.2059272260731	-1.88702154284204\\
-3.24202204076467	-1.73805243153392\\
-3.27012098635517	-1.57840801168737\\
-3.28914822470641	-1.40728699800874\\
-3.29781925092471	-1.22391753054802\\
-3.29463316440711	-1.02761098651421\\
-3.27787753672069	-0.817834448416722\\
-3.24565321045131	-0.594303910063707\\
-3.19592780817642	-0.357097800483401\\
-3.12662703308031	-0.106785954479507\\
-3.0357708082547	0.155437482067335\\
-2.92165557654961	0.427637018465533\\
-2.78307377857452	0.707062532913073\\
-2.61954727411969	0.990151953958752\\
-2.43153652126608	1.27263826333063\\
-2.22057774153498	1.54977366982352\\
-1.98930312676034	1.81665734322728\\
-1.74131842508857	2.06862261920746\\
-1.48094487598371	2.30161695110491\\
-1.21286748687648	2.51250421836134\\
-0.941755014781069	2.69923773288845\\
-0.671919466907365	2.86088622283792\\
-0.4070651975853	2.99753033745599\\
-0.150149193850236	3.11007118332234\\
0.0966532046284112	3.19999973697445\\
0.331900943995047	3.26916921812658\\
0.55478867041782	3.31959838882448\\
0.765025060922061	3.35331887615604\\
0.962709256903624	3.3722680462706\\
1.14821861877544	3.37822194099113\\
};
\addplot [color=mycolor1, forget plot]
  table[row sep=crcr]{%
1.08432098958412	3.48958906878031\\
1.26016263041394	3.48406059583351\\
1.42584987037089	3.46828664202598\\
1.58198865342118	3.44334418104502\\
1.72920180552377	3.41013078867507\\
1.86810226553408	3.36937710519764\\
1.99927420417125	3.32166089092418\\
2.12326000419845	3.26742106314437\\
2.2405513984182	3.20697074047129\\
2.35158338521673	3.14050876766226\\
2.45672982678386	3.06812949587693\\
2.55629986906307	2.98983079230571\\
2.65053450445137	2.90552038458823\\
2.73960273450841	2.81502073768703\\
2.82359688986924	2.71807273618282\\
2.90252673922986	2.6143385213636\\
2.97631208113869	2.50340392539018\\
3.04477357591937	2.38478106818833\\
3.10762165833728	2.25791184958972\\
3.16444349775839	2.12217329129565\\
3.21468817177836	1.97688596873151\\
3.25765053035063	1.82132712217\\
3.29245469720726	1.65475043399235\\
3.31803883456972	1.47641485992724\\
3.33314372828983	1.28562521396671\\
3.33630894286411	1.08178726562651\\
3.32588168137908	0.864479657741579\\
3.30004485434708	0.633543639329572\\
3.25687178373545	0.389189022073703\\
3.19441473769967	0.132110575790638\\
3.1108321596272	-0.136396726777785\\
3.00455406714634	-0.414341923537122\\
2.87447621685277	-0.698985909754783\\
2.72016211379556	-0.986859236316647\\
2.54202061056272	-1.27387130908329\\
2.34142036981858	-1.55551849505863\\
2.12070586008893	-1.82717689529902\\
1.8830952937858	-2.08444177624029\\
1.63246628524926	-2.32345876586698\\
1.37306191215871	-2.5411897987414\\
1.10916823294834	-2.73557159187355\\
0.844817383951105	-2.90555079510593\\
0.583558255283464	-3.05100778424301\\
0.32831573119089	-3.17260093950908\\
0.0813380657245664	-3.27157072675212\\
-0.155783455734006	-3.34953920535941\\
-0.382045527197197	-3.40833035585505\\
-0.596937432007435	-3.44982485664703\\
-0.800337824945961	-3.47585307111208\\
-0.992416369819534	-3.48812343909048\\
-1.17354739216084	-3.48818010731902\\
-1.34423845207748	-3.47738266288519\\
-1.50507393048241	-3.4569013050777\\
-1.6566721384329	-3.42772193538039\\
-1.7996537604388	-3.39065695771374\\
-1.93461930124317	-3.34635878733215\\
-2.06213336918966	-3.29533405368629\\
-2.18271392744576	-3.23795723094195\\
-2.29682497433818	-3.17448296668221\\
-2.40487142121653	-3.10505674818091\\
-2.50719519661724	-3.02972379116863\\
-2.60407181297479	-2.94843619727989\\
-2.69570679019512	-2.8610585351157\\
-2.78223144719621	-2.76737208114765\\
-2.86369765836433	-2.66707803048689\\
-2.94007123848177	-2.55980007038668\\
-3.0112236804157	-2.44508681580146\\
-3.07692204077251	-2.32241474987444\\
-3.1368168701386	-2.1911925053699\\
-3.19042824292162	-2.05076757615838\\
-3.2371301915545	-1.90043686597723\\
-3.27613423396141	-1.73946285848749\\
-3.30647325143702	-1.56709759995265\\
-3.32698777524983	-1.38261705772837\\
-3.33631780438843	-1.18536862902833\\
-3.33290458143599	-0.974834417079117\\
-3.3150081689874	-0.750712059173906\\
-3.28074788496511	-0.513012992199026\\
-3.22817310695987	-0.262174680466433\\
-3.1553708019304	0.00082170216403196\\
-3.06061238059654	0.274343052643648\\
-2.94253533527052	0.556027086251549\\
-2.80034470250325	0.842761973870159\\
-2.6340074807505	1.13074781218675\\
-2.44440357023709	1.41565556019632\\
-2.23339472179801	1.69288079372895\\
-2.00378251961062	1.95786588536709\\
-1.75914769036507	2.20644272094582\\
-1.50359042146601	2.43513784071188\\
-1.2414151827585	2.64138844283209\\
-0.976814721288001	2.82363934648277\\
-0.713602903748065	2.98131945678464\\
-0.455028473577867	3.11472105262322\\
-0.203679564761306	3.22481906642957\\
0.0385301600539593	3.31306897526129\\
0.270314178817421	3.38121429414528\\
0.490928261198941	3.43112311143594\\
0.700069202602977	3.46466199268161\\
0.897772817418163	3.4836072979851\\
1.08432098958412	3.48958906878031\\
};
\addplot [color=mycolor1, forget plot]
  table[row sep=crcr]{%
1.03135213633093	3.59637090839305\\
1.20907004504027	3.59077914425909\\
1.37729149238138	3.57475983814494\\
1.53653710018345	3.54931719978665\\
1.68734744553568	3.51528855270082\\
1.83025841110878	3.47335468467441\\
1.96578322832601	3.42405169926548\\
2.09439954870802	3.36778297747839\\
2.21654011806895	3.30483039105101\\
2.33258587487359	3.2353642907107\\
2.44286051696477	3.15945205909332\\
2.54762576750443	3.07706520209026\\
2.64707671968337	2.98808508293693\\
2.74133675496009	2.89230750409191\\
2.83045161912324	2.78944643230707\\
2.91438231446358	2.67913725852437\\
2.99299653716238	2.56094010047323\\
3.06605847196424	2.43434380459931\\
3.13321687072209	2.29877149590738\\
3.19399151237957	2.1535887667974\\
3.24775840081393	1.99811588966944\\
3.29373444147692	1.83164577081624\\
3.33096288903687	1.65346970017156\\
3.35830161214142	1.46291322151251\\
3.37441719098545	1.25938452546964\\
3.37778900764212	1.0424374575076\\
3.36672866651923	0.811850263579898\\
3.33942099285791	0.567719246035073\\
3.29399299369215	0.310563285224406\\
3.22861581888925	0.0414306205472954\\
3.14164114942691	-0.238006225302083\\
3.03176704475577	-0.525386795849033\\
2.89821937358575	-0.81765279173427\\
2.74092516593776	-1.11111257881753\\
2.56064673823679	-1.40159202394692\\
2.35904427078028	-1.68466770128535\\
2.13864271113069	-1.95595869568967\\
1.90269631859855	-2.21143571861846\\
1.65496642560124	-2.44769807494091\\
1.39944760586285	-2.66217435664607\\
1.14008737146357	-2.85322028953649\\
0.880541753683481	-3.02011052134225\\
0.623995949698562	-3.16294213169387\\
0.37306151489212	-3.2824802789669\\
0.12974553557614	-3.37997902331034\\
-0.104523279302319	-3.45700515846479\\
-0.328830742118102	-3.51528377119382\\
-0.542693766928892	-3.55657483654339\\
-0.745971785698193	-3.58258266842174\\
-0.938782731174552	-3.59489523402033\\
-1.1214290670792	-3.59494797211015\\
-1.29433609777269	-3.58400615651164\\
-1.45800263195444	-3.56316030107577\\
-1.61296283916448	-3.53333004490172\\
-1.75975757063237	-3.49527301601026\\
-1.89891327739221	-3.44959614620652\\
-2.03092676136875	-3.39676771434008\\
-2.15625421059277	-3.33712901411439\\
-2.27530321811131	-3.27090499549261\\
-2.38842672143102	-3.19821354850303\\
-2.49591800484421	-3.11907331953905\\
-2.59800607440349	-3.03341010428896\\
-2.69485084646195	-2.94106197440155\\
-2.78653769193132	-2.84178338833293\\
-2.87307095877536	-2.73524862823342\\
-2.9543661659553	-2.62105500931138\\
-3.03024063679595	-2.498726439141\\
-3.10040243574305	-2.36771807355573\\
-3.16443761172733	-2.22742303238647\\
-3.22179596191571	-2.07718240693662\\
-3.27177584626739	-1.91630010695898\\
-3.31350904610332	-1.74406443604791\\
-3.34594730888101	-1.5597785997573\\
-3.36785308394414	-1.36280254438075\\
-3.37779802106989	-1.15260843741755\\
-3.37417399067421	-0.92885149910718\\
-3.35522247816075	-0.691456475521483\\
-3.31908880564762	-0.440717488068642\\
-3.26390712736859	-0.177405091985249\\
-3.18791975818069	0.0971307505286315\\
-3.08962942159058	0.380878683043629\\
-2.96797524458198	0.671126334464415\\
-2.82251367488618	0.96448540313711\\
-2.65357638041745	1.25699868441593\\
-2.46237236697735	1.54433379386422\\
-2.25100483013732	1.82204987931364\\
-2.02238635691151	2.08590404271753\\
-1.7800567491228	2.33215056177113\\
-1.52792968019001	2.55778436909587\\
-1.27000986510104	2.76069222248298\\
-1.01012604337771	2.93969646714663\\
-0.751716524415859	3.09449943478099\\
-0.49768777443629	3.22555383260815\\
-0.250348979705656	3.33389203355759\\
-0.0114119588916324	3.42094545037425\\
0.217961434923565	3.48837750643975\\
0.437083598214068	3.53794407781902\\
0.645653665675828	3.57138665281924\\
0.843670278731899	3.59035728676248\\
1.03135213633093	3.59637090839305\\
};
\addplot [color=mycolor1, forget plot]
  table[row sep=crcr]{%
0.987871498742845	3.69541322260165\\
1.16735455738493	3.68976238204575\\
1.33788136567027	3.67352014927281\\
1.49990992033289	3.64762963289456\\
1.6539191227879	3.61287609205577\\
1.80038590003472	3.56989587477314\\
1.93976812181382	3.51918672685299\\
2.07249189596827	3.46111825705069\\
2.19894201178193	3.39594180113781\\
2.31945449842671	3.32379926151298\\
2.43431044689757	3.24473073832322\\
2.54373039816002	3.15868094038059\\
2.64786872602072	3.06550449381061\\
2.74680754427043	2.96497037394905\\
2.84054975122996	2.85676578883048\\
2.92901090095079	2.74049995545068\\
3.01200967177911	2.61570834557892\\
3.0892568060592	2.48185814693752\\
3.16034254016137	2.33835589572287\\
3.22472275782997	2.18455848992647\\
3.28170441353351	2.01978908193851\\
3.33043122107849	1.84335964832785\\
3.36987121904565	1.6546022912774\\
3.39880862692961	1.45291144505109\\
3.41584337586375	1.23779899173702\\
3.41940274611399	1.00896362031364\\
3.40777046478862	0.766374328897531\\
3.37913904597866	0.510365512535147\\
3.33169055130225	0.241737460153689\\
3.26370865971377	-0.0381485193979178\\
3.17372040069302	-0.327295984552708\\
3.06065902694148	-0.623039582008411\\
2.92403111682985	-0.922071362753897\\
2.76406319399621	-1.2205420859526\\
2.58179906418554	-1.51424124241489\\
2.37912182003006	-1.79884340078391\\
2.1586854567446	-2.07019150701435\\
1.9237584818392	-2.32457581984514\\
1.67800075380296	-2.5589652969476\\
1.42520869896009	-2.77115794478507\\
1.16906838814842	-2.95983473943839\\
0.912949989095865	-3.12452187550422\\
0.659764068320027	-3.26548192892061\\
0.411885248544655	-3.38356235916136\\
0.171136380764908	-3.48002953470973\\
-0.0611809602427222	-3.55641062829667\\
-0.284226117261211	-3.61435758803426\\
-0.497544137563553	-3.65553962958098\\
-0.700987315142112	-3.68156479488022\\
-0.894641229489225	-3.69392746799862\\
-1.07875959830396	-3.69397702581169\\
-1.25370967242688	-3.68290245745213\\
-1.41992818321254	-3.66172824258476\\
-1.57788686152857	-3.63131759631898\\
-1.72806608400622	-3.59238008450817\\
-1.87093507956281	-3.5454814332458\\
-2.0069372018364	-3.49105403593258\\
-2.13647893981229	-3.42940718912599\\
-2.25992153593609	-3.36073648046014\\
-2.37757427242732	-3.28513203497942\\
-2.48968865497345	-3.20258552877778\\
-2.59645286285159	-3.11299602701529\\
-2.69798594721313	-3.01617481964689\\
-2.79433135048951	-2.91184953141287\\
-2.88544939831458	-2.79966788876174\\
-2.97120849235176	-2.67920164911807\\
-3.05137482238792	-2.54995134904115\\
-3.12560053738497	-2.41135271672331\\
-3.19341049118051	-2.26278582631016\\
-3.25418793804369	-2.10358834462691\\
-3.30715992998904	-1.93307451972076\\
-3.35138369660574	-1.75056184753966\\
-3.38573599681602	-1.55540755584059\\
-3.40890832391732	-1.34705704199601\\
-3.41941187170032	-1.12510601210963\\
-3.41559718901055	-0.889377051198837\\
-3.39569418203983	-0.640009436839153\\
-3.3578781132558	-0.377557973486523\\
-3.30036588172902	-0.103092462559662\\
-3.22154350942313	0.181715443294424\\
-3.12012001649417	0.474535749374782\\
-2.99529505715178	0.772374077051107\\
-2.84691925271312	1.07163416394608\\
-2.67561978752376	1.36825898805647\\
-2.4828628388947	1.65794660849366\\
-2.27093128469621	1.93641959476136\\
-2.0428107979076	2.19971170207586\\
-1.80199636098332	2.44442807993632\\
-1.55224837432126	2.66793932616076\\
-1.29733702121988	2.86848425234937\\
-1.04081254973412	3.04517622282719\\
-0.785829042578658	3.19792659241119\\
-0.535034581828675	3.32731083218641\\
-0.29052660598202	3.43440654295955\\
-0.0538612669716209	3.52062911387546\\
0.173899109718393	3.58758339731566\\
0.392116549741104	3.63694155623125\\
0.600499387624035	3.67035030112556\\
0.799025320798523	3.68936596235326\\
0.987871498742845	3.69541322260165\\
};
\addplot [color=mycolor1, forget plot]
  table[row sep=crcr]{%
0.952714132411607	3.78404114737501\\
1.13380397064677	3.77833685920762\\
1.30637325103211	3.76189732263229\\
1.47083442317104	3.73561543393371\\
1.62762067567458	3.70023265778367\\
1.77716446579211	3.65634701927888\\
1.91988117948408	3.60442233459398\\
2.05615667234629	3.54479760321261\\
2.18633759746216	3.47769588836797\\
2.31072359156383	3.40323231173144\\
2.42956054432884	3.32142100801071\\
2.54303430814367	3.23218104894099\\
2.65126431602905	3.13534147560688\\
2.75429666745447	3.03064569156601\\
2.85209632277312	2.91775558261401\\
2.94453812726338	2.79625585571411\\
3.03139647872044	2.66565924095871\\
3.11233357545849	2.52541338509484\\
3.18688635625902	2.37491048747444\\
3.25445249690282	2.213500986042\\
3.31427619007051	2.04051287582365\\
3.36543493760153	1.85527849771146\\
3.40682924953868	1.65717080131025\\
3.43717797286668	1.44565104805399\\
3.45502291556962	1.22032951214688\\
3.45874735579992	0.9810397392692\\
3.44661367496507	0.72792509839229\\
3.41682531964459	0.461533528993223\\
3.36761704032181	0.182912558011353\\
3.29737431525226	-0.106307761422098\\
3.20477771495196	-0.403860382744142\\
3.08896096889273	-0.706833750006439\\
2.9496639046841	-1.01172793986564\\
2.78735554857161	-1.31458405779946\\
2.60330137337451	-1.61118403748095\\
2.39955419368833	-1.89730252117866\\
2.17886067707563	-2.16897849116513\\
1.94449207431254	-2.42276690311203\\
1.70002351197028	-2.65593308604974\\
1.44909583639802	-2.86656477541579\\
1.19519465816116	-3.05359425813677\\
0.941473578325101	-3.21674038215368\\
0.690636114193532	-3.3563922133266\\
0.444878017363584	-3.47346077781157\\
0.205881893719113	-3.56922337966759\\
-0.0251492633383242	-3.64517894479444\\
-0.24743415390647	-3.70292549841588\\
-0.460543675595968	-3.74406429249515\\
-0.664329558251052	-3.77013028278857\\
-0.858857509148538	-3.78254577658923\\
-1.04434837522275	-3.78259279981239\\
-1.22112868110194	-3.77139956280173\\
-1.38959048251925	-3.74993686590292\\
-1.55015965532614	-3.71902102360354\\
-1.70327135469562	-3.67932067358308\\
-1.84935127458584	-3.63136555260085\\
-1.98880139529637	-3.57555591462119\\
-2.12198904416514	-3.51217173044483\\
-2.24923825863018	-3.44138115671302\\
-2.37082260203892	-3.36324801856241\\
-2.48695872612035	-3.27773823906836\\
-2.59780009525661	-3.18472529276192\\
-2.70343038817182	-3.08399487987252\\
-2.80385617807281	-2.97524912958049\\
-2.89899857143374	-2.85811075903503\\
-2.98868357032124	-2.73212775263374\\
-3.07263102866869	-2.59677929319657\\
-3.15044221867331	-2.45148387999543\\
-3.22158623375341	-2.29561080920482\\
-3.28538575814983	-2.1284964612851\\
-3.34100316177998	-1.9494671113347\\
-3.38742846086942	-1.75787020032102\\
-3.42347143418947	-1.55311608678694\\
-3.44776108074811	-1.33473209991477\\
-3.45875656165306	-1.10243004134154\\
-3.45477459667135	-0.856186900710389\\
-3.4340386540989	-0.596336235620857\\
-3.39475469997954	-0.323664315135675\\
-3.33521618761754	-0.0395009185994836\\
-3.25393688541677	0.254209713866721\\
-3.14980398001913	0.554875578344262\\
-3.02223637374841	0.85927944122605\\
-2.87132601507902	1.16367107638014\\
-2.69793615489473	1.46393315962507\\
-2.50373234616036	1.75581037118509\\
-2.29113115679563	2.03517586494238\\
-2.0631666435215	2.29829796170304\\
-1.82329152448334	2.54206725605282\\
-1.57514323706355	2.76415193623645\\
-1.32231036646699	2.96306465758262\\
-1.06813112802755	3.13814248527204\\
-0.815544970130462	3.28945653769512\\
-0.567005194255521	3.41767637308333\\
-0.324448874733253	3.52391527433429\\
-0.0893127649146833	3.60957823091694\\
0.137419219282995	3.67622740769931\\
0.355150868005853	3.72547273073984\\
0.563602246342536	3.75888945002311\\
0.762739981900453	3.77796071070406\\
0.952714132411606	3.78404114737501\\
};
\addplot [color=mycolor1, forget plot]
  table[row sep=crcr]{%
0.924922571603334	3.86012979383972\\
1.10741517226221	3.85437906114762\\
1.28173030615979	3.83777101375592\\
1.44824772692772	3.81115839519889\\
1.6073672638724	3.77524698969403\\
1.75948843022384	3.73060297968216\\
1.90499467336231	3.67766142563773\\
2.04424113250968	3.61673489540518\\
2.17754490465196	3.5480216365273\\
2.30517696374031	3.47161295997233\\
2.42735501310555	3.38749970936446\\
2.54423666888277	3.29557784790108\\
2.6559124724642	3.19565332509177\\
2.76239831665528	3.08744650423698\\
2.86362695087446	2.97059655376326\\
2.95943831615204	2.84466634353788\\
3.04956856500963	2.70914855109376\\
3.13363776267843	2.56347387892083\\
3.21113646713818	2.40702251364427\\
3.28141167295689	2.23914021248628\\
3.34365300705965	2.05916065760012\\
3.39688060976846	1.8664359259029\\
3.43993683501123	1.66037699706607\\
3.4714847415984	1.44050603636683\\
3.49001724665632	1.20652156428516\\
3.49388160346017	0.958376347328325\\
3.48132425384463	0.696365710316007\\
3.45056065889319	0.42122087860642\\
3.39987291230763	0.13419806142575\\
3.32773433350307	-0.162850132918554\\
3.23295470419749	-0.467437565315586\\
3.11483292281914	-0.776459007045048\\
2.97329712335472	-1.08626991353548\\
2.8090081060877	-1.39283663973601\\
2.62340278296963	-1.69194920344484\\
2.41866172249558	-1.97947427848262\\
2.19759801761047	-2.25161475590067\\
1.96348025985427	-2.50513826980121\\
1.71981555962489	-2.73754254492922\\
1.47012509843908	-2.94713875607942\\
1.21774297344864	-3.13305084521394\\
0.96566055867123	-3.29514366021501\\
0.716426820712345	-3.43390207285486\\
0.472103823814196	-3.55028577351698\\
0.234268654575758	-3.64558147105733\\
0.00404905387826575	-3.7212681828598\\
-0.21782038296065	-3.77890458056491\\
-0.430932716495946	-3.82004159770682\\
-0.635142282220024	-3.84615943395138\\
-0.830502859775551	-3.85862573905873\\
-1.01721333322895	-3.85867078542238\\
-1.19557193450195	-3.84737538802329\\
-1.36593894731248	-3.82566779777615\\
-1.5287070495612	-3.79432647920966\\
-1.68427814337166	-3.75398639801371\\
-1.83304543344168	-3.70514708831453\\
-1.97537956488262	-3.64818130356677\\
-2.11161775084459	-3.58334347413291\\
-2.24205496316544	-3.51077751254264\\
-2.36693640038687	-3.43052374495339\\
-2.48645057430092	-3.34252492663097\\
-2.60072246497577	-3.24663144100706\\
-2.70980628690403	-3.14260590427168\\
-2.81367749143409	-3.03012751626585\\
-2.91222371213947	-2.90879662713068\\
-3.00523445264759	-2.77814013900335\\
-3.09238943676117	-2.63761854151964\\
-3.17324570879828	-2.48663559317992\\
-3.24722381276431	-2.32455190463207\\
-3.31359372064852	-2.15070393914319\\
-3.37146165151542	-1.96443018467537\\
-3.41975954570869	-1.76510640611786\\
-3.45723973290234	-1.5521918499892\\
-3.48247821581105	-1.3252878936088\\
-3.4938908649245	-1.08420970731213\\
-3.48976745481902	-0.829069814385136\\
-3.46832850481223	-0.560369821260565\\
-3.42780882966715	-0.279093054192476\\
-3.36656904634427	0.0132132762966951\\
-3.28323169253416	0.314381602606185\\
-3.17683227421202	0.621610598879662\\
-3.04696849954554	0.931510987921317\\
-2.89392516352604	1.24022063163617\\
-2.71875019385723	1.54358792361473\\
-2.52326142158784	1.83740829534198\\
-2.3099741650074	2.11768516038246\\
-2.08195461616956	2.38087853731976\\
-1.84261899636801	2.62410531352001\\
-1.59550869855381	2.8452649189445\\
-1.34407405027175	3.04307988961685\\
-1.09149384707052	3.21705724794761\\
-0.840547139155294	3.36738904976286\\
-0.593541831202277	3.49481633599458\\
-0.352294833901325	3.6004802503094\\
-0.118152527498448	3.68577925135342\\
0.107961758343429	3.75224470509597\\
0.325485658990213	3.80144075774801\\
0.534150980927119	3.83488943420737\\
0.733919086341772	3.85401872217122\\
0.924922571603333	3.86012979383972\\
};
\addplot [color=mycolor1, forget plot]
  table[row sep=crcr]{%
0.903698258268103	3.92215764008513\\
1.08735132504351	3.91636861736943\\
1.2630878670561	3.89962346304805\\
1.43126467080252	3.8727440060535\\
1.59225780420536	3.83640815616314\\
1.74644301896131	3.79115683907415\\
1.89418044405055	3.73740195984527\\
2.03580251378513	3.67543449883245\\
2.17160419604085	3.60543218458413\\
2.30183471682192	3.52746644617713\\
2.4266900999022	3.44150854364669\\
2.54630594847205	3.34743492958732\\
2.66074998944189	3.2450320258736\\
2.77001398473036	3.13400072270291\\
2.874004695425	3.01396103650767\\
2.97253367537735	2.88445751021317\\
3.06530578539447	2.74496611261371\\
3.15190647658124	2.5949035980394\\
3.231788114712	2.43364052099142\\
3.30425593346166	2.26051934980577\\
3.36845464039736	2.07487935748161\\
3.42335727770903	1.87609012852151\\
3.46775866388929	1.66359551304482\\
3.50027657951629	1.43696954395447\\
3.51936471072105	1.1959850265647\\
3.52334202990972	0.940694008201649\\
3.51044345244203	0.671516969034837\\
3.47889581887818	0.389334317916928\\
3.42702102801173	0.0955698956418503\\
3.35336413039243	-0.207747597431574\\
3.25683844568718	-0.517961781933052\\
3.13687308352426	-0.831820439695171\\
2.99354231586669	-1.14557322156846\\
2.82765341052591	-1.45513677753876\\
2.64077198130094	-1.7563157285933\\
2.43517250879938	-2.04505456852969\\
2.21371504654298	-2.31768641400818\\
1.97966364228776	-2.57114312091857\\
1.73647315857763	-2.80309866965773\\
1.48757553920183	-3.01203162287412\\
1.23619331302362	-3.19720832758851\\
0.985199174442602	-3.3586016427204\\
0.737029308745312	-3.49676734226096\\
0.493648113057673	-3.61270149016879\\
0.256555180764672	-3.70769850472153\\
0.0268223699175176	-3.78322365819072\\
-0.194851198997808	-3.8408075074894\\
-0.408076437012607	-3.88196457610608\\
-0.612711373450221	-3.90813504631275\\
-0.808802834085554	-3.92064622583874\\
-0.996535261072368	-3.92068977991008\\
-1.17618757834346	-3.90931075018121\\
-1.3480979305358	-3.88740485343349\\
-1.5126355067774	-3.85572120270058\\
-1.67017837235513	-3.81486825892942\\
-1.82109615533191	-3.76532141722852\\
-1.96573648222838	-3.7074311251094\\
-2.10441416474785	-3.64143081852443\\
-2.23740226869434	-3.56744425820473\\
-2.36492432410724	-3.48549207330571\\
-2.48714705137577	-3.39549749225424\\
-2.60417307874457	-3.2972913812052\\
-2.71603321453662	-3.19061683572763\\
-2.82267791884308	-3.07513369606113\\
-2.92396770392946	-2.95042349313354\\
-3.01966229315015	-2.81599549175349\\
-3.10940850153282	-2.67129468598716\\
-3.19272698840363	-2.51571282150219\\
-3.2689982985768	-2.3486037636964\\
-3.33744898135135	-2.16930477750443\\
-3.39713908163803	-1.9771654914995\\
-3.44695294995837	-1.77158641000755\\
-3.48559610618886	-1.55206869578192\\
-3.51160175340777	-1.318276409936\\
-3.5233513285425	-1.07011126691083\\
-3.51911393753557	-0.80779804351817\\
-3.49710926858435	-0.531975958161062\\
-3.45559712848183	-0.243787708484773\\
-3.39299365328631	0.0550461005874151\\
-3.30800931705898	0.362182708775455\\
-3.19979749246997	0.674659892579683\\
-3.06809574540893	0.988959834364201\\
-2.91333735231228	1.30114111617248\\
-2.73671011286166	1.60703409649741\\
-2.54014507190236	1.90248118283486\\
-2.32622907275073	2.18359171546618\\
-2.09804955338222	2.44697556903457\\
-1.85899337556969	2.68992265269857\\
-1.61252952337273	2.91050660041041\\
-1.36200597763492	3.10760646104273\\
-1.11048459324002	3.28085520275383\\
-0.860627278989088	3.43053430254579\\
-0.6146358272448	3.5574378624554\\
-0.374239182449423	3.66272820938912\\
-0.140717047647816	3.74779986315749\\
0.0850526535298473	3.81416242323357\\
0.302533810177599	3.86334709699072\\
0.511468552957436	3.89683720004627\\
0.711816190771982	3.91602021218667\\
0.903698258268102	3.92215764008513\\
};
\addplot [color=mycolor1, forget plot]
  table[row sep=crcr]{%
0.888367104169478	3.96922705110388\\
1.07291122390955	3.96340870005969\\
1.24972660934698	3.94655953164109\\
1.41915454831557	3.91947891877675\\
1.58155489367523	3.88282430170261\\
1.73728701994619	3.83711784890623\\
1.88669482939918	3.78275407669708\\
2.03009480218316	3.72000758549893\\
2.16776619914711	3.64904039405054\\
2.2999426476819	3.5699085994557\\
2.42680445556867	3.48256828096067\\
2.54847109980377	3.38688071757307\\
2.6649934270774	3.28261712152967\\
2.77634518475087	3.16946321640947\\
2.88241358406765	3.04702412345997\\
2.98298869260379	2.91483017324689\\
3.07775157595266	2.7723444398493\\
3.16626127875783	2.61897300468039\\
3.24794097593904	2.4540791921464\\
3.32206396272324	2.27700326274185\\
3.38774061280826	2.08708926419992\\
3.44390803562623	1.88372086279525\\
3.4893249028563	1.66636790177474\\
3.52257474521719	1.43464501549463\\
3.54208182626729	1.18838268134413\\
3.546144261553	0.927709422610597\\
3.53298902860703	0.653141346565991\\
3.50085246228754	0.365671846300655\\
3.44808729507601	0.0668504672978017\\
3.37329300996861	-0.241163543734945\\
3.2754604206344	-0.55558933153358\\
3.15411492963611	-0.873069141527941\\
3.0094376299469	-1.18977885737679\\
2.8423415906143	-1.50160330875996\\
2.65448421822659	-1.80436222226359\\
2.4482059418933	-2.0940602942511\\
2.22639883879704	-2.36712731776046\\
1.99232247360235	-2.6206146492307\\
1.74939393345407	-2.85232289414203\\
1.50098193132137	-3.06084981396987\\
1.25023064039342	-3.24556257819414\\
0.999929787427645	-3.40651029606547\\
0.752436848006519	-3.54429883294141\\
0.509647993557539	-3.65995016654044\\
0.273008470238771	-3.7547646023417\\
0.0435506210950787	-3.83019828138013\\
-0.178051930487216	-3.88776249702837\\
-0.391423447866059	-3.92894655792853\\
-0.596425502401252	-3.95516271243238\\
-0.793101019687933	-3.96770989035063\\
-0.981625239981682	-3.96775237649092\\
-1.1622643685979	-3.95630961403404\\
-1.33534170842412	-3.93425381000624\\
-1.50121050516149	-3.90231264186172\\
-1.66023247507985	-3.86107499730855\\
-1.8127609182453	-3.81099824347687\\
-1.95912736560332	-3.75241598788524\\
-2.09963080922159	-3.68554566155298\\
-2.23452868539888	-3.61049553716092\\
-2.36402889969611	-3.52727101104556\\
-2.48828229154454	-3.43578014660196\\
-2.60737503158122	-3.33583861669606\\
-2.72132052997725	-3.22717431025802\\
-2.83005051531157	-3.10943199750812\\
-2.93340503094563	-2.98217859123723\\
-3.03112120282013	-2.84490970765218\\
-3.12282077645936	-2.69705842527188\\
-3.20799662322782	-2.53800736405932\\
-3.28599870174384	-2.36710544939466\\
-3.35602035661871	-2.18369096083397\\
-3.41708636655713	-1.98712264415913\\
-3.46804482675066	-1.7768207039597\\
-3.50756574495878	-1.55231926827857\\
-3.53415006926062	-1.31333125914399\\
-3.54615358389633	-1.05982531792969\\
-3.54183042910619	-0.792112345344705\\
-3.51940052341957	-0.510936253052693\\
-3.4771434253147	-0.217559861614279\\
-3.41351776998375	0.0861669099851317\\
-3.32730027834201	0.397772267523406\\
-3.21773200376542	0.714177318684968\\
-3.08465333840251	1.03177309149074\\
-2.92860545811216	1.34656437398015\\
-2.75087657637592	1.65437294940559\\
-2.5534779083619	1.95107955455206\\
-2.3390459754472	2.23287342098895\\
-2.11068193469523	2.4964744238251\\
-1.87175078567116	2.7392974959319\\
-1.62566981795802	2.95954082059142\\
-1.37571486537172	3.15619453949885\\
-1.12486587587972	3.32898062712203\\
-0.87570296296231	3.4782436754916\\
-0.630353853834017	3.60481537243616\\
-0.390485948581672	3.70987334533868\\
-0.157332027587421	3.79480985536257\\
0.0682623078150467	3.86111972866702\\
0.285780096344525	3.91031146769596\\
0.494971613979965	3.94384147128421\\
0.695795704407208	3.96306883484295\\
0.888367104169477	3.96922705110388\\
};
\addplot [color=mycolor1, forget plot]
  table[row sep=crcr]{%
0.878355311988031	4.00104451440901\\
1.06350692625045	3.99520620111796\\
1.24105208545559	3.97828669634494\\
1.41132245401614	3.95107065481952\\
1.57466773303343	3.91420199851153\\
1.73143696912751	3.86819041177193\\
1.88196376005733	3.81341874894101\\
2.02655438296571	3.750150547048\\
2.16547798030495	3.67853714869157\\
2.29895805486653	3.59862418003358\\
2.42716463522509	3.51035731500992\\
2.55020657118865	3.41358740789436\\
2.668123506243	3.30807520909464\\
2.78087715546415	3.19349600831664\\
2.88834160146377	3.06944468768286\\
2.99029241959652	2.93544182533574\\
3.08639457260405	2.79094167459111\\
3.17618919386929	2.63534305689275\\
3.25907963118477	2.46800444241896\\
3.33431747555457	2.28826473051858\\
3.4009897767428	2.09547144282054\\
3.45800926415817	1.88901813486057\\
3.50411013946209	1.6683927090189\\
3.53785283151403	1.43323782151323\\
3.55764187603864	1.18342353411735\\
3.5617615690935	0.919130581462076\\
3.54843389392288	0.64093999291339\\
3.51590199238786	0.349921399335943\\
3.46253971391698	0.0477085835925056\\
3.38698331236318	-0.263452424287396\\
3.28827546184747	-0.580698823062855\\
3.16600547552711	-0.900604275852195\\
3.02042477853998	-1.21929774783176\\
2.85251553485022	-1.53264516950835\\
2.66399457594665	-1.83647819729149\\
2.45724457303916	-2.12684262395052\\
2.23517770285111	-2.4002325211065\\
2.00105011016354	-2.65377765196434\\
1.7582542402097	-2.88536099227669\\
1.51011809336617	-3.09365738536341\\
1.25973568625162	-3.27809894883646\\
1.00984379818778	-3.43878382182336\\
0.762749729660402	-3.57635010468598\\
0.520306129482393	-3.69183655722441\\
0.283923478606458	-3.78654748590043\\
0.0546086990229475	-3.86193343438586\\
-0.166981171541117	-3.91949359001469\\
-0.380479311033542	-3.96070128664519\\
-0.585749985503425	-3.986950972031\\
-0.782834069573656	-3.99952339235443\\
-0.971901342639129	-3.99956518582877\\
-1.15321025667837	-3.98807919312425\\
-1.32707494440725	-3.96592226831757\\
-1.49383870905093	-3.93380798674879\\
-1.65385299334717	-3.89231225949612\\
-1.80746076448412	-3.84188040852438\\
-1.95498329637778	-3.78283470631474\\
-2.09670942756051	-3.71538173904544\\
-2.23288648811713	-3.63961922624244\\
-2.36371220332813	-3.55554214043776\\
-2.48932698599055	-3.46304813674846\\
-2.60980612184367	-3.36194244219558\\
-2.7251514361164	-3.25194248386318\\
-2.83528211072562	-3.1326826674037\\
-2.9400244113109	-3.00371986447725\\
-3.03910019488404	-2.86454033832207\\
-3.13211422011443	-2.71456903567785\\
-3.21854049496589	-2.55318239915177\\
-3.29770819576308	-2.37972609464072\\
-3.3687881039304	-2.19353927459823\\
-3.43078105286486	-1.99398715573653\\
-3.4825105630904	-1.78050369093681\\
-3.52262264058648	-1.55264583001233\\
-3.54959653315705	-1.31016012225037\\
-3.56177090540339	-1.05306102369323\\
-3.55739011243071	-0.781718072012295\\
-3.53467462163135	-0.496946044577514\\
-3.49191769492277	-0.200088546569051\\
-3.42760684601089	0.106918179217524\\
-3.34056332736365	0.421516953800008\\
-3.23008662410292	0.74055253109619\\
-3.09608510202537	1.06035723112041\\
-2.93917069281702	1.37690232801292\\
-2.76069690826706	1.68600603833541\\
-2.56272661017962	1.98357610732913\\
-2.34792790075655	2.26585543441595\\
-2.11941020973985	2.52963648855715\\
-1.88052401217249	2.77241584827913\\
-1.63465314874106	2.9924724548048\\
-1.38502715711598	3.18886813413911\\
-1.13457364935907	3.3613821295424\\
-0.885820579981077	3.51039961632549\\
-0.640848456119384	3.63677651896027\\
-0.40128537454523	3.7417004733795\\
-0.16833402832988	3.82656253933673\\
0.0571808702239901	3.89284832757713\\
0.274754939025249	3.94205200602846\\
0.48414407094491	3.975612869128\\
0.685307253194799	3.99487187487964\\
0.87835531198803	4.00104451440901\\
};
\addplot [color=mycolor1, forget plot]
  table[row sep=crcr]{%
0.87317260289467	4.01786189580409\\
1.0586470073743	4.01201298588825\\
1.23657813808843	3.99505628803038\\
1.40729283295192	3.96776882189008\\
1.57113567513125	3.93078746483326\\
1.72845048096189	3.88461536516827\\
1.87956560720388	3.8296292438126\\
2.02478211899208	3.7660867946643\\
2.16436396668982	3.69413370111489\\
2.29852943361614	3.61381002254227\\
2.42744322415422	3.52505588894224\\
2.55120865822076	3.42771659234049\\
2.669859524372	3.32154729687035\\
2.78335122505298	3.20621771999007\\
2.8915509323161	3.08131727771277\\
2.99422657288481	2.94636134697276\\
3.09103459362408	2.80079948506522\\
3.18150664226152	2.64402666092272\\
3.26503555735054	2.47539878863308\\
3.34086142196184	2.29425408892726\\
3.40805892143134	2.09994199696792\\
3.46552787019314	1.89186141208578\\
3.51198952458983	1.66950993600597\\
3.54599211843127	1.43254521754385\\
3.565929810729	1.18085842894172\\
3.57007968075661	0.914658061332085\\
3.55666118804975	0.634559542850044\\
3.52392119282383	0.34167275065185\\
3.47024478877763	0.0376757525251807\\
3.39428765284131	-0.275140018359175\\
3.29511970073396	-0.593869140628072\\
3.17236365200774	-0.915049888989435\\
3.02630751934734	-1.23478733986051\\
2.85796923624723	-1.54893823008124\\
2.6690962381088	-1.85334103266034\\
2.46209281396263	-2.14406336221362\\
2.23988130272687	-2.41763291314587\\
2.00571593672854	-2.67122014574578\\
1.76297642204718	-2.90275057507449\\
1.51496986821874	-3.11093871465674\\
1.26476464219	-3.2952500330042\\
1.01507049410291	-3.45580782062897\\
0.76816912641704	-3.59326672357929\\
0.52589097390353	-3.70867415021093\\
0.289628743614559	-3.80333652933828\\
0.0603763166980937	-3.87870162167307\\
-0.161217833407163	-3.93626249501914\\
-0.37479155748923	-3.97748436629825\\
-0.580210632191441	-4.00375260580504\\
-0.777515021874765	-4.01633865542448\\
-0.966871847477634	-4.01638009184131\\
-1.14853571801903	-4.00487119763804\\
-1.32281618140011	-3.98266087997323\\
-1.49005154074905	-3.95045538220979\\
-1.65058804753554	-3.90882383733263\\
-1.80476342481098	-3.85820524655758\\
-1.95289371835076	-3.79891590799786\\
-2.09526256837972	-3.73115666920935\\
-2.23211210715962	-3.65501964694317\\
-2.36363479941683	-3.57049426544381\\
-2.48996564479346	-3.47747262983988\\
-2.61117425248971	-3.37575439108605\\
-2.72725638116091	-3.2650513891372\\
-2.83812461886204	-3.14499249511081\\
-2.94359796877304	-3.01512922244045\\
-3.04339022056795	-2.87494284998519\\
-3.13709714252516	-2.72385400113866\\
-3.22418274777113	-2.56123584983809\\
-3.30396519460744	-2.38643236363178\\
-3.37560330143226	-2.19878321489474\\
-3.43808521149454	-1.99765713755406\\
-3.49022143481789	-1.78249548733109\\
-3.53064529193214	-1.552867446259\\
-3.55782459317131	-1.30853752562824\\
-3.57008902380522	-1.04954457618762\\
-3.56567787070637	-0.776289257127412\\
-3.54281201375396	-0.489623826818346\\
-3.4997920658988	-0.19093445153481\\
-3.43512084778765	0.117797349155185\\
-3.34764306570541	0.43396984883628\\
-3.23668882152542	0.754388232410177\\
-3.10220192627757	1.07535469351025\\
-2.94483103100244	1.39282366721893\\
-2.7659633660714	1.70261215790338\\
-2.56768831396552	2.00064250305003\\
-2.35269006632684	2.28318584141813\\
-2.1240821328952	2.54707244995374\\
-1.88520740306866	2.78984116869868\\
-1.63943250522551	3.00981256628613\\
-1.38996326771129	3.20608531441715\\
-1.13970056624627	3.37846804420796\\
-0.891145740461128	3.52736675030405\\
-0.646355204967131	3.65364981369306\\
-0.406936982742872	3.75851006021307\\
-0.174078362593084	3.8433380143813\\
0.0514066882572753	3.90961465117128\\
0.26902032040407	3.95882687108028\\
0.478521424067281	3.99240526127457\\
0.679869196887858	4.0116815148737\\
0.873172602894669	4.01786189580409\\
};
\addplot [color=mycolor1, forget plot]
  table[row sep=crcr]{%
0.872400712460433	4.02038721270919\\
1.05792368884529	4.01453670886125\\
1.23591279430132	3.99757442485156\\
1.40669415698166	3.97027624228707\\
1.57061160586457	3.93327798661598\\
1.72800818514876	3.88708182838907\\
1.87921147654295	3.83206356936454\\
2.02452177307966	3.76848002634673\\
2.16420225459164	3.69647603203228\\
2.29847042837004	3.61609080803541\\
2.42749020571817	3.52726364931374\\
2.55136408132319	3.42983900962203\\
2.67012496850639	3.32357121094303\\
2.78372732461065	3.20812913063445\\
2.89203728571882	3.08310136072375\\
2.99482163073449	2.94800249437786\\
3.09173552753474	2.80228138172703\\
3.18230919841028	2.64533241228778\\
3.26593390212375	2.47651111686476\\
3.34184799159715	2.29515561666519\\
3.40912429337584	2.10061563872189\\
3.46666068087803	1.89229089164514\\
3.51317646588389	1.66968044266358\\
3.54721805189474	1.43244420253153\\
3.56717804270875	1.1804765244426\\
3.57133243904973	0.913990076760244\\
3.55790032826914	0.633605457730588\\
3.52512913593394	0.340438584799407\\
3.47140564893363	0.0361741613396976\\
3.39538845990468	-0.276889589273369\\
3.29615156509209	-0.595840886473753\\
3.17332267837953	-0.917212741306342\\
3.02719527215502	-1.23710669588607\\
2.8587926074176	-1.55137815237982\\
2.66986666846752	-1.85586663643602\\
2.46282494966837	-2.14664304453676\\
2.24059128960257	-2.42024009883123\\
2.0064196277991	-2.67383432706057\\
1.76368777428001	-2.9053575740163\\
1.51569973809642	-3.1135302481095\\
1.26552008439842	-3.29782276943657\\
1.01585455977801	-3.45836216129842\\
0.768981076019589	-3.59580552596122\\
0.526726777741583	-3.71120155773919\\
0.290481737747762	-3.8058570046132\\
0.061237901848247	-3.88121921953595\\
-0.160357530756129	-3.93878036750128\\
-0.373943110125655	-3.98000446543796\\
-0.579384844552284	-4.00627553969976\\
-0.77672257159037	-4.01886365278914\\
-0.96612302601515	-4.0189050361166\\
-1.14784025442594	-4.00739269981099\\
-1.32218313135468	-3.98517436751489\\
-1.48948922383381	-3.95295518874916\\
-1.65010401763265	-3.91130328345454\\
-1.80436446095217	-3.8606567064451\\
-1.95258582575122	-3.80133085975595\\
-2.09505098149806	-3.73352572889659\\
-2.23200128836394	-3.65733258789161\\
-2.36362842820677	-3.57274002561814\\
-2.4900665935909	-3.47963931098985\\
-2.61138454585985	-3.37782925445319\\
-2.72757713607705	-3.26702085362826\\
-2.83855596441736	-3.14684214525272\\
-2.94413894473072	-3.01684383515928\\
-3.04403865551544	-2.87650645132992\\
-3.13784951437789	-2.72524996645255\\
-3.22503403219608	-2.56244706340121\\
-3.30490871084637	-2.38744145611349\\
-3.37663057017367	-2.19957289845858\\
-3.43918584588259	-1.99821065819708\\
-3.49138309306751	-1.78279721065399\\
-3.53185372744502	-1.55290358461373\\
-3.55906384378357	-1.30829699957148\\
-3.57134178307268	-1.04901997985141\\
-3.56692607720377	-0.775477865301702\\
-3.54403767512824	-0.488528543722764\\
-3.50097829987952	-0.189564564061972\\
-3.4362530729949	0.1194257760861\\
-3.34871021754107	0.435834095364993\\
-3.23768441700587	0.756459683113159\\
-3.10312477660631	1.07760026181148\\
-2.94568542159817	1.39520778860554\\
-2.76675859942821	1.70509912442116\\
-2.56843763007815	2.00319881562071\\
-2.35340908664954	2.2857822333132\\
-2.12478707505283	2.5496852985695\\
-1.88591335131713	2.79245314539585\\
-1.64015198650535	3.01241253190674\\
-1.39070529283191	3.20866760230002\\
-1.1404701847406	3.38103134977479\\
-0.891944060345442	3.5299128589176\\
-0.647179766393561	3.65618236087359\\
-0.407782351929108	3.76103345507587\\
-0.174936821431493	3.84585659144264\\
0.0505444541971551	3.91213206239664\\
0.268164599667046	3.96134569498887\\
0.477682956765757	3.99492678814681\\
0.679058762338607	4.01420566237649\\
0.872400712460432	4.02038721270919\\
};
\addplot [color=mycolor1, forget plot]
  table[row sep=crcr]{%
0.875685557219478	4.00967763805254\\
1.06100272774934	4.00383388897912\\
1.23874599994084	3.98689529297496\\
1.40924453870376	3.95964257308244\\
1.57284539700775	3.92271602638168\\
1.72989491724704	3.87662197661192\\
1.88072398936467	3.82174012688877\\
2.02563619962134	3.75833101425244\\
2.16489801188383	3.68654307749908\\
2.29873023832086	3.60641908780634\\
2.42730016506097	3.51790187687297\\
2.55071379570723	3.42083944803591\\
2.66900776242509	3.3149896888231\\
2.78214053566533	3.20002503333925\\
2.88998264804146	3.07553756233089\\
2.99230574750499	2.94104518790835\\
3.08877042556289	2.7959997556108\\
3.17891294772867	2.6397981105391\\
3.26213126941631	2.47179740992005\\
3.33767107714397	2.29133620127919\\
3.40461307655313	2.09776298196715\\
3.46186336966171	1.89047404078754\\
3.5081495136968	1.66896224611083\\
3.54202567591895	1.43287793509279\\
3.56189106090967	1.18210199051418\\
3.56602625257259	0.916829382170032\\
3.55265192876722	0.637658790924856\\
3.52001312997204	0.345680512379708\\
3.46648947089461	0.0425510835278295\\
3.39072717737164	-0.269460118123823\\
3.29178292249394	-0.587468353417559\\
3.16926320037092	-0.908029045899134\\
3.0234382680544	-1.22725882065251\\
2.85530871517088	-1.54101882183472\\
2.66660715145907	-1.84514416267025\\
2.45972740504391	-2.13569183046976\\
2.23758690778079	-2.4091731874464\\
2.0034408310195	-2.66273895188084\\
1.76067505595249	-2.89429402127679\\
1.51260680563624	-3.10253369378862\\
1.2623168570532	-3.28690728975612\\
1.01252803382881	-3.4475259237388\\
0.765534418367372	-3.58503623101065\\
0.523177189657266	-3.70048142961837\\
0.286857653948686	-3.79516691484637\\
0.0575760054487964	-3.87054178698562\\
-0.164015131686133	-3.92810206847397\\
-0.377551337644567	-3.969316898912\\
-0.582897646867223	-3.99557603830755\\
-0.780094452261493	-4.00815543066842\\
-0.969310149493292	-4.00819704008281\\
-1.15080119641976	-3.99669929558502\\
-1.32487935016732	-3.97451495749124\\
-1.4918853257357	-3.94235382694221\\
-1.65216788081969	-3.90078832860645\\
-1.806067272198	-3.85026053577205\\
-1.95390207358847	-3.79108965251219\\
-2.09595844071301	-3.72347931960282\\
-2.23248102314662	-3.64752438246993\\
-2.36366483543669	-3.56321696870832\\
-2.48964750319551	-3.47045188848123\\
-2.6105013915586	-3.36903151101306\\
-2.72622520662556	-3.25867040014854\\
-2.83673474211532	-3.13900012521297\\
-2.94185253377562	-3.00957481160171\\
-3.04129629697352	-2.86987816734751\\
-3.13466617617788	-2.71933292200084\\
-3.22143105060679	-2.55731384054737\\
-3.3009144433641	-2.38316571498379\\
-3.37228099786505	-2.19622795969058\\
-3.43452503594903	-1.99586758867153\\
-3.48646340108576	-1.78152234342872\\
-3.52673558716778	-1.55275543868523\\
-3.55381496787173	-1.30932262844642\\
-3.56603559239636	-1.05125087648801\\
-3.56163920527467	-0.778925685763544\\
-3.53884647527735	-0.493181072480737\\
-3.49595442884873	-0.195382502407381\\
-3.43145843393958	0.112510508935222\\
-3.34419179023719	0.427917864513764\\
-3.23346972277805	0.747663938626615\\
-3.09921883410767	1.06806550948267\\
-2.94206996319232	1.38508513324789\\
-2.76339399824633	1.69454034792435\\
-2.56526748071004	1.99234636366853\\
-2.35036682262639	2.27476058800659\\
-2.12180357232166	2.53859494654633\\
-1.8829243026072	2.78136778825985\\
-1.63710397863557	3.00137951287317\\
-1.3875599028346	3.19771094877064\\
-1.13720588438874	3.3701564993994\\
-0.888556135040165	3.5191120805132\\
-0.643678721429679	3.64544004071904\\
-0.404191376720517	3.75033068852255\\
-0.171288844332672	3.8351748005113\\
0.0542097043526339	3.90145558792076\\
0.271803250693252	3.95066346198877\\
0.481249224027468	3.98423321927077\\
0.682506708461911	4.00350103458518\\
0.875685557219478	4.00967763805254\\
};
\addplot [color=mycolor1, forget plot]
  table[row sep=crcr]{%
0.882731909221739	3.98702827620123\\
1.06761539375866	3.98119877047761\\
1.24483900571829	3.96431025452675\\
1.41473856780442	3.93715382213604\\
1.57766817606962	3.90037931935314\\
1.73398135577779	3.85450191085997\\
1.88401618029681	3.79990957904154\\
2.02808336540157	3.73687073334937\\
2.16645646266481	3.66554142460836\\
2.29936339425423	3.58597190173672\\
2.42697868338947	3.4981124361362\\
2.54941583457895	3.40181849054501\\
2.66671940613247	3.29685544149948\\
2.77885639919958	3.18290319271153\\
2.88570667111642	3.05956115349769\\
2.98705217798458	2.92635421240694\\
3.08256497769965	2.78274051879899\\
3.17179409970626	2.62812209688069\\
3.25415163517833	2.46185955216491\\
3.32889874738757	2.28329237097139\\
3.39513277197684	2.09176652072532\\
3.45177718708876	1.88667116454702\\
3.49757697745115	1.66748620191229\\
3.53110274393243	1.4338418888766\\
3.55076769715874	1.18559079145782\\
3.55486219390545	0.922890594589315\\
3.54161038162041	0.646293701855839\\
3.50925236618872	0.356836174514016\\
3.45615266872313	0.0561147602165053\\
3.38093134812654	-0.253662597463924\\
3.28260828478366	-0.56966898255585\\
3.16074475307444	-0.888508003772609\\
3.01556137678958	-1.20632901239781\\
2.84801011498882	-1.51900601649008\\
2.6597818717229	-1.82236525663449\\
2.45324093105079	-2.11243438844237\\
2.23129076023667	-2.3856792747371\\
1.99718904460682	-2.63919536900434\\
1.75433899699584	-2.87082967070344\\
1.50608635773878	-3.07922340030803\\
1.25554697065881	-3.26378036902835\\
1.00548064154205	-3.42457735549414\\
0.758216486533735	-3.5622384135832\\
0.515626080765996	-3.67779497954975\\
0.279135032617231	-3.77254959507967\\
0.0497613395445611	-3.84795521492908\\
-0.171830716992709	-3.90551627241799\\
-0.385270399547987	-3.94671303691297\\
-0.590420759709649	-3.97294769465193\\
-0.787323520997589	-3.98550890671645\\
-0.976150797072024	-3.98555100251722\\
-1.15716437571668	-3.97408406911436\\
-1.33068234754739	-3.95197167189453\\
-1.49705231623919	-3.91993356087358\\
-1.65663017597268	-3.87855133874811\\
-1.80976337831088	-3.82827561962295\\
-1.95677765570536	-3.7694336642546\\
-2.09796626742514	-3.70223683836107\\
-2.23358095111042	-3.62678751815175\\
-2.36382387957633	-3.54308528011456\\
-2.48884002864075	-3.45103237950283\\
-2.60870945550432	-3.35043866190228\\
-2.72343907143695	-3.2410261807742\\
-2.83295357390829	-3.12243392489568\\
-2.93708529193436	-2.99422320488968\\
-3.03556280793928	-2.855884416668\\
-3.12799836739509	-2.70684609687665\\
-3.21387429557463	-2.54648741039426\\
-3.29252893414871	-2.37415545137323\\
-3.36314301552118	-2.18918896964803\\
-3.42472793185169	-1.99095030145124\\
-3.47611803587304	-1.77886730131582\\
-3.51596990654864	-1.5524868133086\\
-3.54277234109924	-1.31154051518176\\
-3.55487152426177	-1.05602262530621\\
-3.55051608963588	-0.78627681059057\\
-3.52792622494695	-0.503086622526053\\
-3.48538912189345	-0.207760120922984\\
-3.42137955892997	0.0978043714961962\\
-3.33469919677839	0.411087109991549\\
-3.22462186239684	0.728966240501443\\
-3.09102612672127	1.04779966508264\\
-2.93449296225586	1.363573149435\\
-2.75634736244044	1.67210627095784\\
-2.55862967980305	1.96929477663697\\
-2.34399429304059	2.25135796132634\\
-2.11554707999374	2.51505648762724\\
-1.87664488694667	2.7578512438258\\
-1.63068614503158	2.97798592410831\\
-1.38092054995231	3.17449110084244\\
-1.13029848048179	3.34712106410228\\
-0.881370570904034	3.49624330839168\\
-0.636237859826358	3.62270319198721\\
-0.396545539721535	3.72768397322806\\
-0.163509402534211	3.81257721008895\\
0.0620367292732862	3.87887249966095\\
0.279582967943426	3.9280702282238\\
0.488882722600083	3.96161712127999\\
0.689894874815934	3.98086202666136\\
0.882731909221738	3.98702827620123\\
};
\addplot [color=mycolor1, forget plot]
  table[row sep=crcr]{%
0.893299774619925	3.95386829659369\\
1.07755214373071	3.94805954177936\\
1.25401542773574	3.93124431583162\\
1.42303580063363	3.90422922774552\\
1.5849782444428	3.86767833420769\\
1.74020732941008	3.82211988601337\\
1.88907212038673	3.76795405568808\\
2.03189418792735	3.70546078719016\\
2.16895781927488	3.63480723708031\\
2.3005016488094	3.55605452683267\\
2.42671104466854	3.46916371784449\\
2.54771069215461	3.37400107359669\\
2.66355690548067	3.27034280494737\\
2.77422928199946	3.1578796201603\\
2.8796213954568	3.03622153427032\\
2.97953031852448	2.90490354378534\\
3.07364488506467	2.76339295064852\\
3.16153276848965	2.61109932753198\\
3.24262668758664	2.44738835125698\\
3.31621038180594	2.2716009766731\\
3.38140545071861	2.08307964485391\\
3.43716074643413	1.88120335412893\\
3.48224674236073	1.66543336983428\\
3.5152581350474	1.43537096426486\\
3.53462875686095	1.19082767771975\\
3.53866347378391	0.931906976742413\\
3.52559177989175	0.659093708520822\\
3.49364683462498	0.373344424108135\\
3.44117125418124	0.0761677941582254\\
3.36674676341693	-0.230319245427688\\
3.26933899140187	-0.543376023084219\\
3.14844215251458	-0.859678674650531\\
3.0042028551991	-1.17542656741677\\
2.83750012747688	-1.48651439077919\\
2.64996194821813	-1.78875659376243\\
2.44390769691871	-2.07813814119094\\
2.22221931485283	-2.35105748741888\\
1.98815791197158	-2.60452746429405\\
1.7451527266885	-2.83630800111545\\
1.49659270031397	-3.0449586627256\\
1.24564701169292	-3.22981436239356\\
0.99513183136759	-3.39089983342567\\
0.747429709962517	-3.52880492315626\\
0.504458560161213	-3.64454330167256\\
0.267680966437529	-3.73941334949917\\
0.0381419077695911	-3.81487407197817\\
-0.183476801654758	-3.87244286466293\\
-0.39679508897941	-3.91361704952985\\
-0.601673307903635	-3.93981777331237\\
-0.798155537809111	-3.95235302598206\\
-0.986419868182171	-3.95239585477658\\
-1.16673648453138	-3.94097391618187\\
-1.33943335949599	-3.91896698158723\\
-1.50486877396498	-3.88710964629557\\
-1.66340962350881	-3.8459971348178\\
-1.81541439567	-3.79609266937436\\
-1.96121974953256	-3.7377353433588\\
-2.10112973204507	-3.67114781585288\\
-2.23540678869332	-3.5964434302804\\
-2.36426384812058	-3.51363257887012\\
-2.48785687111275	-3.42262830465481\\
-2.60627735145271	-3.32325127287698\\
-2.71954434210345	-3.21523437045106\\
-2.82759566144855	-3.09822731991584\\
-2.93027802071645	-2.97180183529007\\
-3.02733591852908	-2.83545801111575\\
-3.11839928890021	-2.68863282891876\\
-3.20297008630592	-2.53071188782891\\
-3.28040827084369	-2.36104570920414\\
-3.34991804508014	-2.17897220451209\\
-3.41053571604363	-1.98384708380746\\
-3.46112122223544	-1.77508403838827\\
-3.50035615831711	-1.55220633353923\\
-3.52675197680331	-1.31491082965675\\
-3.53867278879889	-1.06314421601491\\
-3.53437755233595	-0.797189207100136\\
-3.51208603276816	-0.517755532335079\\
-3.47007127328522	-0.226066895661054\\
-3.40677801058063	0.0760687666748171\\
-3.32096139680575	0.386221402916465\\
-3.21183403917211	0.70134969368578\\
-3.07920308030691	1.01787383287239\\
-2.92357492114615	1.3318155924254\\
-2.74620552404545	1.63899910462536\\
-2.5490804545193	1.93529237337846\\
-2.33482042707061	2.21685862135462\\
-2.10652232685255	2.48038217619461\\
-1.86755824066811	2.72323774414946\\
-1.62136202998576	2.94358355286687\\
-1.37123257662162	3.14037417707846\\
-1.1201759479247	3.31330312954958\\
-0.870798319037104	3.46269482591756\\
-0.625251013653767	3.58936892524313\\
-0.385221052769517	3.69449812986298\\
-0.151956201110977	3.7794753728873\\
0.07368758457424	3.8458001464851\\
0.291187166213739	3.89498815865432\\
0.500290138046111	3.92850437284837\\
0.700955402138476	3.94771693646535\\
0.893299774619924	3.95386829659369\\
};
\addplot [color=mycolor1, forget plot]
  table[row sep=crcr]{%
0.907201964607069	3.91167245534824\\
1.09065777465157	3.90588993340582\\
1.266154057183	3.88916794883912\\
1.43405126957733	3.86233344813099\\
1.594729297961	3.82606897971694\\
1.74856773050552	3.78091969528162\\
1.8959304784888	3.72730139516052\\
2.03715367621373	3.66550870945552\\
2.17253591515043	3.59572285092159\\
2.3023300003203	3.51801863638448\\
2.42673554145718	3.43237067112468\\
2.54589180117484	3.33865874564094\\
2.65987031709079	3.23667262488993\\
2.76866689895426	3.12611653258124\\
2.87219268316559	3.00661376124556\\
2.97026401679408	2.87771198423091\\
3.06259105600252	2.73889001751568\\
3.1487651184107	2.58956698232377\\
3.22824504843374	2.42911505251825\\
3.30034316574832	2.2568772212265\\
3.36421179758033	2.07219175911517\\
3.41883196817198	1.87442520570766\\
3.4630065393329	1.66301574203399\\
3.49536093365565	1.43752849981995\\
3.51435543146834	1.1977235870137\\
3.51831372075252	0.943636145187155\\
3.50547257797936	0.675665425934568\\
3.47405682690306	0.394666641041049\\
3.42238156859522	0.102035452677137\\
3.34897972599436	-0.200228877409782\\
3.25274723000599	-0.509498242008303\\
3.13309144873207	-0.822544621750892\\
2.99006239042379	-1.13563484408611\\
2.82444314390421	-1.44469331517534\\
2.63777820796342	-1.74552174078721\\
2.43232677218777	-2.03405132380219\\
2.2109413525581	-2.30659339061513\\
1.97688688500037	-2.56005258739435\\
1.73362686517079	-2.79207387400802\\
1.48460783432562	-3.00110837186224\\
1.23307049154952	-3.18639914912145\\
0.981906815594165	-3.34790143399911\\
0.733571301996436	-3.48615942365957\\
0.490044211354484	-3.60216321962818\\
0.252837747946268	-3.69720593341672\\
0.0230329004807929	-3.77275501842284\\
-0.198665367807622	-3.8303455570256\\
-0.411864969177006	-3.87149796299146\\
-0.616423224962243	-3.89765891705149\\
-0.812387983253727	-3.91016230237642\\
-0.999945933157816	-3.91020610170607\\
-1.17937904953582	-3.89884123580373\\
-1.3510290044646	-3.8769687935728\\
-1.51526875180609	-3.8453427594907\\
-1.67248019647208	-3.80457601721365\\
-1.8230367819316	-3.75514801179825\\
-1.96728987702298	-3.69741295278498\\
-2.10555795271765	-3.63160783367408\\
-2.23811767079117	-3.55785984339636\\
-2.3651961362982	-3.47619297194968\\
-2.4869636832195	-3.38653378625872\\
-2.60352666456006	-3.28871649299009\\
-2.71491980680991	-3.18248752976885\\
-2.821097770212	-3.06751004995626\\
-2.92192564020713	-2.94336880162888\\
-3.01716817460706	-2.8095760590619\\
-3.10647776214747	-2.66557945213046\\
-3.1893812319927	-2.51077275782587\\
-3.2652659155717	-2.34451096228418\\
-3.33336572959656	-2.16613115111861\\
-3.39274854857961	-1.97498099842778\\
-3.44230678277244	-1.77045672708944\\
-3.48075386369101	-1.5520522900466\\
-3.50663020521454	-1.31942101315796\\
-3.51832301358775	-1.07244984603562\\
-3.51410481113174	-0.811344487469956\\
-3.49219533339304	-0.536720862648344\\
-3.45085007673776	-0.249694813293135\\
-3.38847575070051	0.0480421262739279\\
-3.30376801415931	0.354176392328224\\
-3.19586050306031	0.665772375441081\\
-3.06446749681585	0.97933329473128\\
-2.90999768909297	1.29093043938497\\
-2.73361585877175	1.59639665376012\\
-2.53723454146527	1.89156610234076\\
-2.32342899703125	2.17253026180334\\
-2.09528334141111	2.43587407085777\\
-1.85618936035988	2.67885886947717\\
-1.60962792701334	2.89952967674314\\
-1.35896372133843	3.096739924233\\
-1.10727761787898	3.27010201066058\\
-0.857250548598626	3.41988281855851\\
-0.611101534345176	3.54686777559801\\
-0.37057381810315	3.65221571353813\\
-0.13695797183735	3.73732173818946\\
0.0888606376441422	3.8036989372349\\
0.306341543883091	3.85288383501983\\
0.51522509565988	3.88636602159796\\
0.715470826708424	3.90553956693049\\
0.907201964607069	3.91167245534824\\
};
\addplot [color=mycolor1, forget plot]
  table[row sep=crcr]{%
0.924302577268678	3.86189211630072\\
1.10682793588482	3.85614030118489\\
1.28118346465662	3.8395283562989\\
1.44774822158637	3.8129081250885\\
1.6069213205598	3.77698458517328\\
1.75910156319047	3.73232319280708\\
1.90467170212362	3.67935834691924\\
2.04398620368068	3.61840200227287\\
2.1773615125963	3.54965182669744\\
2.30506796545077	3.47319857177777\\
2.42732263390715	3.38903253178664\\
2.54428249642295	3.29704912361095\\
2.65603743712263	3.19705375039243\\
2.76260265707262	3.08876623051121\\
2.86391016387759	2.97182519594288\\
2.95979909106812	2.84579300228258\\
3.05000470335532	2.71016185681574\\
3.13414608564841	2.56436206748747\\
3.21171271537749	2.40777354542344\\
3.28205040596633	2.23974194812529\\
3.34434751338946	2.05960110511928\\
3.39762284392349	1.86670357364382\\
3.44071740249287	1.66046124474937\\
3.47229295909028	1.4403977306503\\
3.49084130871891	1.20621363425014\\
3.49470888810416	0.957864517924979\\
3.482141794043	0.695649247545377\\
3.45135579210955	0.420303290690675\\
3.40063409282872	0.133087650260337\\
3.32845205279382	-0.164139997232879\\
3.23362241769647	-0.468888574874113\\
3.11544783921875	-0.778048547420278\\
2.97386069163747	-1.08797222096061\\
2.80952605372759	-1.39462440672324\\
2.62388462414642	-1.6937954453718\\
2.41911975881142	-1.98135419210385\\
2.19804596743955	-2.25350728117634\\
1.96393173817926	-2.50702712272158\\
1.7202826576272	-2.73941660557328\\
1.47061727341327	-2.94899183374906\\
1.21826637271547	-3.1348809568265\\
0.966217799225929	-3.2969520196137\\
0.717017161305549	-3.43569199190078\\
0.472723616438722	-3.55206163566387\\
0.234911949930583	-3.6473478781585\\
0.00470825077904838	-3.72302931723577\\
-0.217153924593788	-3.78066377945691\\
-0.430268147238923	-3.82180111000609\\
-0.63448884007801	-3.84792031745743\\
-0.829869553541633	-3.86038785584452\\
-1.01660872145817	-3.86043285839573\\
-1.19500398360006	-3.84913508976578\\
-1.36541495081342	-3.82742185085764\\
-1.52823359176441	-3.7960707542068\\
-1.68386109282635	-3.75571600097849\\
-1.83268995444608	-3.70685643347235\\
-1.97509013772311	-3.64986416981211\\
-2.11139819369793	-3.58499304573975\\
-2.24190845035496	-3.512386405709\\
-2.3668654730398	-3.43208402260333\\
-2.48645714046661	-3.34402810450295\\
-2.60080778702254	-3.24806848861867\\
-2.70997095461662	-3.14396724498773\\
-2.81392137979104	-3.03140303150048\\
-2.91254592337998	-2.90997567075147\\
-3.00563324307241	-2.77921156931496\\
-3.09286212990652	-2.63857077978506\\
-3.17378859834742	-2.48745671942205\\
-3.24783206100307	-2.32522980327834\\
-3.31426126159385	-2.15122650860271\\
-3.37218111214272	-1.96478562561834\\
-3.42052220388863	-1.76528360221187\\
-3.45803553631711	-1.55218085095605\\
-3.4832958913014	-1.32508050224315\\
-3.49471815072211	-1.08380015805423\\
-3.49059148637807	-0.828455510171989\\
-3.46913637521871	-0.559552068116832\\
-3.42858832468511	-0.278077702428641\\
-3.36730952046585	0.0144154025009713\\
-3.28392500818716	0.315754618250104\\
-3.1774736798184	0.623133930516661\\
-3.0475572882158	0.933160270630387\\
-2.89446495101418	1.24196910719445\\
-2.71924869608243	1.54540820624748\\
-2.52372969898562	1.83927427087464\\
-2.31042541409597	2.11957373071161\\
-2.08240269310558	2.38277094052677\\
-1.84307691646843	2.62598784012684\\
-1.59598734938198	2.84712897763138\\
-1.34458128952978	3.04492150669661\\
-1.09203405767341	3.21887617504392\\
-0.841121217041329	3.36918768566299\\
-0.594147512821357	3.49659864745824\\
-0.352927239901517	3.60225082355921\\
-0.118804802241516	3.68754254447156\\
0.107297807599128	3.75400451875631\\
0.324818987254881	3.80319990561513\\
0.533490828574192	3.83664957538831\\
0.733274610028959	3.85578031258039\\
0.924302577268677	3.86189211630072\\
};
\addplot [color=mycolor1, forget plot]
  table[row sep=crcr]{%
0.944516283357556	3.80590651146832\\
1.12600694416265	3.80018893313277\\
1.29907839514465	3.78370091403236\\
1.46413313464144	3.75732354672323\\
1.6215942509449	3.72178786797593\\
1.77188427403367	3.67768265364217\\
1.9154090101981	3.62546341875114\\
2.05254514551271	3.56546157697518\\
2.18363055333876	3.49789310579632\\
2.30895639976892	3.42286635562741\\
2.4287602885957	3.34038885649499\\
2.54321981533404	3.25037313791706\\
2.65244600700955	3.15264170713785\\
2.75647621470372	3.04693144590314\\
2.85526610658436	2.93289780187852\\
2.94868049073801	2.81011928079819\\
3.03648279320481	2.67810290045345\\
3.11832314470376	2.5362914557908\\
3.193725211668	2.38407366912169\\
3.26207217013619	2.22079855604496\\
3.3225925950385	2.04579560759144\\
3.37434755227592	1.85840263152925\\
3.41622085719777	1.65800323681731\\
3.44691529552266	1.44407586624896\\
3.46495853482702	1.2162558124227\\
3.46872334307913	0.974410573845452\\
3.45646730708205	0.718726991762843\\
3.42639709238593	0.449805694754777\\
3.37676087139271	0.168754519994192\\
3.30596933613776	-0.122731768557435\\
3.21274043621382	-0.422321960411291\\
3.09625600043273	-0.727047255616864\\
2.95631104282055	-1.03336438385475\\
2.7934311549163	-1.33729128191246\\
2.60893272668596	-1.63461097082836\\
2.40490683411936	-1.9211240646005\\
2.18412034337852	-2.1929170949095\\
1.9498441360456	-2.44660745683292\\
1.70563334591066	-2.67952921263771\\
1.45509320214317	-2.88983651530382\\
1.20166398360216	-3.07651881288057\\
0.948450624488358	-3.23933857772801\\
0.698110229976512	-3.37871350278845\\
0.452798421644794	-3.49556910274425\\
0.21416619587588	-3.59118538035426\\
-0.0166058971438045	-3.66705516544324\\
-0.238750359501091	-3.72476457513914\\
-0.451844476043196	-3.76589970664256\\
-0.655740473954405	-3.79197908652604\\
-0.850500202401142	-3.80440868486485\\
-1.03633766529251	-3.80445512238523\\
-1.21357068491732	-3.79323256638883\\
-1.38258161808189	-3.77169927444621\\
-1.54378626382428	-3.74066046616579\\
-1.69760973338298	-3.70077496828251\\
-1.84446795318847	-3.65256377189301\\
-1.98475352691003	-3.59641921584209\\
-2.11882481403556	-3.53261396042429\\
-2.24699723997037	-3.46130925477345\\
-2.3695360075457	-3.38256225194533\\
-2.48664951816332	-3.29633231161098\\
-2.59848292814463	-3.20248637363108\\
-2.70511136386072	-3.10080360597876\\
-2.80653240361764	-2.99097964416911\\
-2.90265751394226	-2.87263086079165\\
-2.99330221482033	-2.74529924502634\\
-3.07817485806523	-2.60845864283539\\
-3.15686405504991	-2.46152331494399\\
-3.22882500894723	-2.30386001167888\\
-3.29336532118994	-2.13480503046703\\
-3.34963128286441	-1.95368798485662\\
-3.39659625583826	-1.75986421743488\\
-3.43305350587418	-1.55275783808721\\
-3.45761674382456	-1.3319171192781\\
-3.46873256610547	-1.09708323512193\\
-3.46470976061236	-0.848271860648108\\
-3.44377071778722	-0.585864743501096\\
-3.40412947457992	-0.310704947477907\\
-3.34409866432677	-0.0241852744818788\\
-3.26222340618926	0.271684877588597\\
-3.15743393924866	0.574251940885052\\
-3.02920140236346	0.880247522848288\\
-2.87767444134446	1.18588745718983\\
-2.70377096196827	1.4870430624743\\
-2.50920191889443	1.77947271618314\\
-2.29641362803012	2.05908698282915\\
-2.06845016011193	2.32221017542621\\
-1.82875372350786	2.56579959516712\\
-1.58093329845906	2.7875920105064\\
-1.32853619394178	2.98616258070545\\
-1.07485285323216	3.16089911709276\\
-0.822774572692031	3.31190888410948\\
-0.574710991668611	3.43988277640292\\
-0.332563142706253	3.54594231826279\\
-0.0977407564306555	3.6314904143393\\
0.128790253767559	3.69807987412364\\
0.346443592488374	3.74730680680177\\
0.554942570560637	3.78073046282245\\
0.754251950804218	3.79981746875353\\
0.944516283357555	3.80590651146832\\
};
\addplot [color=mycolor1, forget plot]
  table[row sep=crcr]{%
0.967808437787213	3.74499261078458\\
1.14818689311289	3.73931195079239\\
1.31985791757415	3.72295916374077\\
1.48325294871124	3.69684878707962\\
1.63882433054309	3.66174127259313\\
1.78702324306083	3.61825136430489\\
1.92828304146005	3.56685777405959\\
2.06300668643723	3.50791301948369\\
2.19155711540948	3.44165271477233\\
2.31424958279424	3.36820391851422\\
2.43134516225289	3.28759237058299\\
2.54304474522393	3.19974861744842\\
2.6494829867059	3.1045131547088\\
2.75072174486073	3.00164082654933\\
2.84674264284519	2.89080483060273\\
2.9374384594573	2.77160079737071\\
3.02260314284291	2.64355155793964\\
3.1019203554607	2.50611339179397\\
3.17495062009433	2.35868476397706\\
3.24111737238403	2.20061881692842\\
3.29969256622918	2.03124116458242\\
3.34978295744406	1.84987481269934\\
3.39031883591177	1.65587423738532\\
3.42004779543775	1.4486706874312\\
3.43753708839786	1.22783047483062\\
3.44118909537926	0.993127164611371\\
3.4292752135284	0.744626918972085\\
3.39999364017396	0.482783573532793\\
3.35155555677723	0.208536270438194\\
3.28230149779729	-0.0766020530382409\\
3.19084478476633	-0.370481450901351\\
3.07623192928899	-0.670295958881667\\
2.93810193520605	-0.972626900863537\\
2.77681969261661	-1.27356044058473\\
2.5935561610029	-1.56887933595467\\
2.39029241707874	-1.85431300653857\\
2.16973656449054	-2.12581468117063\\
1.935159563803	-2.37982503876469\\
1.69017314560574	-2.61348247365794\\
1.43848440406202	-2.82475130603174\\
1.18366381336087	-3.01245716703307\\
0.928956384295275	-3.17623731692343\\
0.677152922916439	-3.31642727135342\\
0.430524619113761	-3.43391106100985\\
0.190813353526649	-3.52996117362653\\
-0.0407360190398877	-3.60608824569472\\
-0.263316741814673	-3.66391288558087\\
-0.476488298868083	-3.70506493338691\\
-0.680102117264968	-3.73111020158552\\
-0.874231767747263	-3.74350154438241\\
-1.0591115050854	-3.74354965140351\\
-1.23508466119144	-3.73240872042629\\
-1.40256186232585	-3.71107262287069\\
-1.56198815148524	-3.68037794588043\\
-1.71381767872152	-3.64101112654664\\
-1.85849450947297	-3.59351765219552\\
-1.99643816425042	-3.53831192954726\\
-2.12803265189593	-3.47568691584796\\
-2.25361793603441	-3.40582297170541\\
-2.37348294778796	-3.32879566284923\\
-2.48785941152722	-3.24458243257702\\
-2.59691587926138	-3.153068212415\\
-2.70075147467725	-3.05405015647705\\
-2.79938893547174	-2.94724179295157\\
-2.89276662118796	-2.83227699924786\\
-2.9807292348313	-2.70871433859584\\
-3.06301710498207	-2.57604245620105\\
-3.13925400991203	-2.4336874304032\\
-3.20893372022145	-2.28102321172531\\
-3.27140572068634	-2.1173865544282\\
-3.32586097793594	-1.9420981301299\\
-3.37131917984663	-1.75449176545173\\
-3.40661960543466	-1.5539538829359\\
-3.43041868081252	-1.33997511444028\\
-3.44119826803717	-1.11221550915271\\
-3.43728965053286	-0.870583535237989\\
-3.41691871122378	-0.615326928416679\\
-3.37827747551848	-0.347130215287896\\
-3.31962541754045	-0.0672095307517391\\
-3.23942014602516	0.222609279273533\\
-3.13647108052289	0.519847122444371\\
-3.01010204621649	0.821381790707592\\
-2.86030101476354	1.12352752006707\\
-2.6878302144381	1.42219058941629\\
-2.49427053077934	1.71309325270415\\
-2.28198236475224	1.99204203709281\\
-2.05398010648345	2.2552034280972\\
-1.81373520525708	2.49934532639181\\
-1.56493772823386	2.72200884185793\\
-1.31125332476364	2.92159021948065\\
-1.05610981641013	3.09733170608087\\
-0.802537173781546	3.24923679019982\\
-0.553070814292143	3.37793516318734\\
-0.309715500973054	3.48452486394514\\
-0.0739585488149666	3.57041506006226\\
0.153182698290488	3.6371857336636\\
0.371093942046129	3.68647293783938\\
0.579489767029112	3.71988203757484\\
0.778340934540154	3.73892713446705\\
0.967808437787212	3.74499261078458\\
};
\addplot [color=mycolor1, forget plot]
  table[row sep=crcr]{%
0.994196135033692	3.68031066273118\\
1.17340809817122	3.67466886909633\\
1.34358529454558	3.65846042759474\\
1.50519440257395	3.63263740579726\\
1.65872294953164	3.59799278088386\\
1.80465617574299	3.55516956627888\\
1.94345982046772	3.50467133034416\\
2.07556737826532	3.44687286752932\\
2.20137056907467	3.38203024968933\\
2.32121196908053	3.31028982587883\\
2.43537893622874	3.23169598202777\\
2.54409812292944	3.14619764586289\\
2.64752999734826	3.05365365219381\\
2.7457628975869	2.95383719024497\\
2.83880622732637	2.84643965569918\\
2.92658247713054	2.73107434051203\\
3.00891783529258	2.60728052632482\\
3.08553125196808	2.47452871344698\\
3.15602196091464	2.33222792507424\\
3.21985567036081	2.1797362787155\\
3.2763499398936	2.01637630730141\\
3.32465969933645	1.84145681803885\\
3.36376447264931	1.65430334738335\\
3.39245966613	1.45429941414018\\
3.40935525226816	1.24094064278145\\
3.41288624655233	1.01390321439251\\
3.40134033834497	0.773126737678835\\
3.37290854193795	0.518909246585072\\
3.3257642423025	0.252008462058861\\
3.25817385791012	-0.026261159208295\\
3.16863793304269	-0.313951047084765\\
3.05605464214126	-0.608440208278517\\
2.91988919546258	-0.906456213524509\\
2.7603245152051	-1.20417127277554\\
2.57836396307739	-1.4973783244928\\
2.37585911180771	-1.78173589999365\\
2.15544620475775	-2.05305307612738\\
1.92039246005548	-2.30757306327229\\
1.67437275902528	-2.54221124732962\\
1.42121197663355	-2.7547126655596\\
1.16463328497573	-2.94371199518092\\
0.908047180755667	-3.1086997798198\\
0.654402882025848	-3.2499151684208\\
0.406108377881043	-3.36819391757691\\
0.165012567418598	-3.46480052850891\\
-0.0675648393948347	-3.54126761569399\\
-0.290772489374853	-3.59925732074702\\
-0.504151061163646	-3.640451598681\\
-0.70755343146076	-3.66647209103852\\
-0.901069341070793	-3.67882649143326\\
-1.08495903597741	-3.6788765088859\\
-1.25959768192474	-3.66782215616716\\
-1.42543057397612	-3.64669754584442\\
-1.58293813815797	-3.61637421068568\\
-1.73260924409234	-3.57756888264478\\
-1.87492122205869	-3.53085350535999\\
-2.01032505403617	-3.47666595152244\\
-2.13923438109869	-3.41532045672481\\
-2.26201717324251	-3.34701718184587\\
-2.3789891052139	-3.27185060417864\\
-2.49040785532301	-3.18981664267737\\
-2.59646768779841	-3.10081857168896\\
-2.69729379438629	-3.00467189329604\\
-2.79293596339599	-2.90110844015344\\
-2.88336122293346	-2.78978008465047\\
-2.96844518093896	-2.67026255030781\\
-3.04796187215268	-2.54205996954056\\
-3.12157203946577	-2.40461101806195\\
-3.18880994740434	-2.25729768628089\\
-3.24906907751574	-2.09945802108844\\
-3.30158742310312	-1.93040447377617\\
-3.34543361998514	-1.74944978628918\\
-3.37949585047893	-1.55594256956421\\
-3.40247634696937	-1.34931475687185\\
-3.4128953566482	-1.12914277315562\\
-3.40910947746106	-0.895223306830814\\
-3.38935006572105	-0.647662725689275\\
-3.35178750025212	-0.386976208024451\\
-3.29462585142282	-0.114188522962037\\
-3.21622928049041	0.169076531734969\\
-3.11527584916467	0.460535709474934\\
-2.99092658770654	0.757235257041731\\
-2.84298902572669	1.05560799962333\\
-2.67204759944406	1.35160972666908\\
-2.47953180906159	1.64093219927022\\
-2.26769940087452	1.91927262561283\\
-2.03952636078154	2.1826235633948\\
-1.79851474015648	2.42753895198215\\
-1.54844718051117	2.65133526219359\\
-1.29312730776821	2.85220102774083\\
-1.036144754668	3.02920829215682\\
-0.780693638554805	3.1822388174508\\
-0.529458403888326	3.31185067720291\\
-0.284566353757399	3.41911497970214\\
-0.0475957291863381	3.50544923079226\\
0.18037683768094	3.57246640440727\\
0.398706003313474	3.6218503749631\\
0.60709838598585	3.65526119870713\\
0.805534219195617	3.6742687731098\\
0.994196135033692	3.68031066273118\\
};
\addplot [color=mycolor1, forget plot]
  table[row sep=crcr]{%
1.02375040531617	3.61290204042887\\
1.20176097639549	3.60730045239441\\
1.37036957674768	3.59124369131131\\
1.53008538748134	3.56572536946192\\
1.68143780878129	3.53157387649472\\
1.82495211979655	3.48946246279246\\
1.96113166891982	3.43992080417065\\
2.09044497609687	3.38334668835468\\
2.2133163581027	3.32001698205598\\
2.33011892490929	3.25009741061438\\
2.44116901027864	3.17365094361483\\
2.54672128027864	3.09064476183925\\
2.64696390782588	3.0009559112872\\
2.74201331386758	2.90437585191723\\
2.83190806411648	2.8006142012692\\
2.91660158441303	2.68930207223581\\
2.99595343022608	2.56999552386096\\
3.06971893197026	2.44217979639812\\
3.13753715736318	2.30527519718943\\
3.19891731008915	2.15864574879307\\
3.25322395208505	2.00161200452337\\
3.29966183225684	1.83346976493728\\
3.33726166727844	1.65351675400746\\
3.36486898352328	1.46108956040721\\
3.38113910113713	1.25561318766688\\
3.38454247282657	1.03666518752547\\
3.3733857285037	0.804055302209507\\
3.34585461031475	0.557919500568181\\
3.30008499547509	0.298823996439621\\
3.23426669333464	0.0278702496733986\\
3.14678092088188	-0.253213474226809\\
3.03636585661394	-0.54201345615402\\
2.90229582008168	-0.835426867441588\\
2.74455015014256	-1.12973284012124\\
2.56394095281318	-1.42074875525535\\
2.36216843685569	-1.70406624141488\\
2.14178131502429	-1.97534198104143\\
1.90603727640003	-2.23060186778484\\
1.65868024595527	-2.46651004688063\\
1.40366973892964	-2.68056058438075\\
1.14490648466575	-2.87116725842797\\
0.885995072210844	-3.03764980487017\\
0.63007116427807	-3.18013500544318\\
0.379703604907356	-3.29940273110009\\
0.136866392078723	-3.3967091096898\\
-0.0970344061660405	-3.47361363503207\\
-0.321097449612501	-3.53182808856116\\
-0.534844239226391	-3.57309603977806\\
-0.738132419461264	-3.5991045027063\\
-0.931073626071126	-3.61142473381267\\
-1.11396114323217	-3.61147691260704\\
-1.28720951056091	-3.60051289518032\\
-1.45130614583367	-3.57961168548513\\
-1.60677385886146	-3.54968318739327\\
-1.75414258331887	-3.51147682884627\\
-1.89392851680262	-3.46559259474307\\
-2.02661895578985	-3.41249278629328\\
-2.15266131796345	-3.35251342660259\\
-2.27245508285067	-3.28587467397293\\
-2.38634561026136	-3.21268991747198\\
-2.49461899453625	-3.13297344723351\\
-2.59749727486108	-3.04664674493076\\
-2.69513344957367	-2.95354355339843\\
-2.78760584167458	-2.85341397942644\\
-2.87491144258746	-2.74592797770453\\
-2.95695793279452	-2.63067867162487\\
-3.0335541551112	-2.5071861012065\\
-3.10439891645093	-2.37490216114041\\
-3.16906813919749	-2.23321771157066\\
-3.22700060200421	-2.0814731142732\\
-3.2774828366435	-1.91897376091784\\
-3.31963422182456	-1.74501249358455\\
-3.35239397487088	-1.55890111606585\\
-3.37451261146532	-1.36001335729562\\
-3.38455150677088	-1.14784151232587\\
-3.38089535402736	-0.922068315058349\\
-3.36178335378977	-0.682654091339456\\
-3.32536546776439	-0.429936596356689\\
-3.26978941824833	-0.164736978685912\\
-3.1933215530012	0.111539791837042\\
-3.09449957081791	0.396827137223658\\
-2.97230730339444	0.688362960794408\\
-2.82635227231545	0.982721291257161\\
-2.65701807848947	1.27592551447855\\
-2.46555953737673	1.5636464007621\\
-2.25411243852087	1.84146980592806\\
-2.02560326847436	2.10520006333872\\
-1.78356466168654	2.35115257139234\\
-1.53188345629107	2.57638853279399\\
-1.27452259365677	2.77885734455166\\
-1.01526081426038	2.95743342231848\\
-0.757485202333998	3.11185660361474\\
-0.504055616196481	3.24260160764968\\
-0.257243124584251	3.35070882318175\\
-0.0187316848424999	3.43760661534103\\
0.210334443627698	3.50494770426823\\
0.429276150050858	3.55447278851852\\
0.637793727556865	3.58790627445286\\
0.835881553665213	3.6068830687125\\
1.02375040531617	3.61290204042887\\
};
\addplot [color=mycolor1, forget plot]
  table[row sep=crcr]{%
1.05659982423101	3.5436962097753\\
1.23338953524892	3.5381356863494\\
1.40036900034433	3.52223658455641\\
1.55809830241556	3.49703803845282\\
1.70715629008817	3.46340652800132\\
1.84811490673	3.42204718070233\\
1.98152076859	3.37351662858223\\
2.10788216348064	3.31823592487422\\
2.22765991767207	3.25650260540557\\
2.34126085936369	3.18850139185262\\
2.44903285812511	3.11431331704303\\
2.55126062779082	3.03392324367229\\
2.64816164416347	2.94722587889718\\
2.73988165365188	2.85403048404157\\
2.82648934309907	2.7540645618524\\
2.90796981500203	2.64697689049896\\
2.98421657841733	2.53234037859128\\
3.05502183896928	2.4096553521991\\
3.12006497023276	2.27835406514218\\
3.17889919744524	2.1378074573444\\
3.23093675399576	1.98733547741162\\
3.27543311994892	1.82622262905518\\
3.31147146435916	1.65374077137479\\
3.33794913309673	1.46918154206162\\
3.3535689795292	1.27190097051325\\
3.35683951089544	1.06137872310933\\
3.34608911292561	0.837293708330638\\
3.31950076128171	0.599616122860203\\
3.27517415625753	0.348713088675428\\
3.21122140990007	0.0854606072336505\\
3.12589939848037	-0.188651157300746\\
3.01777592707658	-0.471438308117994\\
2.88591787080211	-0.759992854618088\\
2.73007874651697	-1.05072513220936\\
2.550853793361	-1.33949555086233\\
2.34976699558954	-1.62183705326857\\
2.12926057137599	-1.89324869827937\\
1.89257432820553	-2.14952018695187\\
1.64352619912048	-2.38703493485975\\
1.38622833938899	-2.60300137176016\\
1.12478689562998	-2.79557870125465\\
0.863033287022711	-2.96388830141094\\
0.604321973242455	-3.10792613548574\\
0.351410388033882	-3.22840742802162\\
0.106418236504346	-3.32657959408979\\
-0.129149325311288	-3.40403480820078\\
-0.354335519552628	-3.46254393948293\\
-0.568643112585792	-3.50392309778006\\
-0.771939191240562	-3.5299354786536\\
-0.964364609538357	-3.54222559917473\\
-1.14625434955744	-3.54228020380336\\
-1.3180713423304	-3.53140935223569\\
-1.48035384250214	-3.51074166070109\\
-1.63367505571405	-3.48122869904976\\
-1.77861308794007	-3.44365471792408\\
-1.91572914317403	-3.39864895889987\\
-2.04555202384444	-3.34669868759013\\
-2.16856723967643	-3.28816176780067\\
-2.28520931521275	-3.22327808645115\\
-2.39585615477706	-3.15217948163206\\
-2.50082455396636	-3.07489805892091\\
-2.60036613240358	-2.99137293864105\\
-2.69466310590541	-2.90145558780369\\
-2.78382342447307	-2.80491397812694\\
-2.86787488513555	-2.70143589453793\\
-2.94675789701699	-2.59063181274185\\
-3.02031664354309	-2.4720378837501\\
-3.08828847005424	-2.34511972036689\\
-3.15029144544934	-2.20927788655847\\
-3.20581023122645	-2.06385625298164\\
-3.25418067517839	-1.90815470103258\\
-3.29457397191594	-1.74144801953847\\
-3.32598184364106	-1.56301320296016\\
-3.34720503075629	-1.37216764621141\\
-3.35684845374742	-1.16832079568533\\
-3.35332766149975	-0.951041435213579\\
-3.33489244351797	-0.720141650946591\\
-3.29967439518837	-0.475776270276219\\
-3.24576518889597	-0.218552900262679\\
-3.17133050273471	0.0503574420655213\\
-3.07476013326221	0.329125177290405\\
-2.95484728379189	0.615204118239174\\
-2.81097990419437	0.905334800269474\\
-2.6433164241271	1.19563026176681\\
-2.45291113117283	1.48175420485149\\
-2.24175529051304	1.75918285353335\\
-2.01271173396278	2.02352008155767\\
-1.76934182203281	2.27081800201832\\
-1.51564830327697	2.49784970293208\\
-1.25577690828014	2.70229053142342\\
-0.993726460657979	2.88278613974326\\
-0.733110174569144	3.03891122275533\\
-0.476993794902001	3.17104359983949\\
-0.22781652556404	3.28018863279807\\
0.0126152938222836	3.36778859164235\\
0.243080823721761	3.4355438678002\\
0.462864883201927	3.48526240878233\\
0.671664208211566	3.51874399905155\\
0.86949350295471	3.53769889760678\\
1.05659982423101	3.5436962097753\\
};
\addplot [color=mycolor1, forget plot]
  table[row sep=crcr]{%
1.09293589616732	3.47352416739806\\
1.26849674134541	3.46800521628701\\
1.4337963726869	3.45226881742003\\
1.58945551384844	3.42740359121524\\
1.73611105723776	3.39431658087325\\
1.87438896106105	3.35374610094779\\
2.00488523863961	3.30627618560696\\
2.1281529475438	3.25235098875703\\
2.24469342574854	3.19228814442275\\
2.35495035945798	3.12629055407149\\
2.45930556407635	3.05445637538651\\
2.55807560223686	2.97678718834121\\
2.6515085506791	2.89319444595849\\
2.73978036775067	2.80350440791691\\
2.82299041518313	2.70746182798343\\
2.90115576282974	2.60473273944519\\
2.97420396577533	2.49490677185651\\
3.04196406409613	2.37749955163049\\
3.10415563430801	2.25195590161078\\
3.16037584105218	2.11765477296779\\
3.21008462779589	1.97391712583436\\
3.25258848496511	1.82001832512479\\
3.28702369044987	1.65520702208577\\
3.31234058338146	1.47873290964321\\
3.32729135394503	1.28988608325564\\
3.33042502619716	1.08805085086839\\
3.32009472087404	0.872776463495758\\
3.294483716037	0.643866026764014\\
3.25165786063622	0.401482380030563\\
3.18965183998718	0.14626562297628\\
3.10659467762955	-0.120548837136475\\
3.00087467000739	-0.397029733799231\\
2.87133514004594	-0.680491177724248\\
2.71748067453288	-0.967502756385573\\
2.53966159761159	-1.25399141701655\\
2.3391970671875	-1.53544458360951\\
2.11839966185456	-1.80720207873993\\
1.88047959687191	-2.06479972792575\\
1.6293324405293	-2.304309147934\\
1.36924230153405	-2.52261466175929\\
1.10455236597759	-2.71758238549025\\
0.839358910264985	-2.88810324984138\\
0.577273118980962	-3.03402069412792\\
0.321273502343611	-3.15597497606478\\
0.0736493239613713	-3.25520446646513\\
-0.163980901938123	-3.33334094448337\\
-0.39059948088467	-3.39222550860005\\
-0.605691971291168	-3.43375953298264\\
-0.809141323680796	-3.4597947909185\\
-1.00112695072322	-3.47205997306241\\
-1.18203618954599	-3.47211728526765\\
-1.35239117552056	-3.46134177342606\\
-1.51279121505116	-3.44091649946167\\
-1.66386909976731	-3.41183787617206\\
-1.80625908431625	-3.37492682851767\\
-1.94057411061837	-3.33084269798519\\
-2.06739003805033	-3.28009782656068\\
-2.18723495344339	-3.22307152820622\\
-2.30058198023905	-3.1600227071581\\
-2.40784432622723	-3.09110075955144\\
-2.50937157959258	-3.01635464438066\\
-2.60544647762741	-2.93574017215796\\
-2.69628153531545	-2.84912566758599\\
-2.78201504058541	-2.75629624183129\\
-2.86270601000842	-2.65695698090751\\
-2.93832776493263	-2.55073543601379\\
-3.00875984686818	-2.43718390414292\\
-3.07377805825877	-2.31578212656641\\
-3.1330425106563	-2.18594122180271\\
-3.18608371292067	-2.04700991947117\\
-3.23228697158644	-1.89828447842317\\
-3.27087574721622	-1.73902405258735\\
-3.30089516523149	-1.56847368625895\\
-3.32119766893653	-1.38589751586511\\
-3.3304338617257	-1.19062500815314\\
-3.32705290150082	-0.982112972744464\\
-3.30931826690162	-0.760025345617177\\
-3.27534601764221	-0.524330951425817\\
-3.2231732680184	-0.275416197277719\\
-3.15086363930317	-0.0142046695419724\\
-3.05665291922995	0.257730879547194\\
-2.93913116800754	0.538085541447675\\
-2.7974470431082	0.823794593647349\\
-2.6315078179007	1.11108723868343\\
-2.44213824832725	1.3956391351759\\
-2.23115840500554	1.67282304544363\\
-2.00134937353323	1.93803265140908\\
-1.75629691134506	2.18703182923845\\
-1.50013135080415	2.41626989391422\\
-1.2372073006177	2.62310873735567\\
-0.971779315032066	2.80592927048401\\
-0.707725490017997	2.96411384119377\\
-0.448353209970221	3.09792740271686\\
-0.196298235066255	3.20833524870649\\
0.0464916587598717	3.29679723218257\\
0.278709355828535	3.36507081509076\\
0.499601885991418	3.41504341430274\\
0.708866665431189	3.44860294222669\\
0.906546832754389	3.46754676618677\\
1.09293589616732	3.47352416739806\\
};
\addplot [color=mycolor1, forget plot]
  table[row sep=crcr]{%
1.13302068675688	3.40313637942729\\
1.30735216851495	3.39765928538817\\
1.47092677483942	3.38209011309542\\
1.62443718206064	3.3575709367022\\
1.7685880866297	3.32505162330178\\
1.90406761415	3.28530464075765\\
2.03152780257099	3.23894145950706\\
2.15157173658891	3.18642873356713\\
2.26474533572043	3.12810319314109\\
2.37153220990307	3.06418469332344\\
2.47235035129693	2.99478719894024\\
2.56754971585957	2.91992769738027\\
2.6574099652991	2.83953316231655\\
2.74213779835695	2.7534457746188\\
2.82186341190383	2.66142666773136\\
2.89663570968929	2.56315852299072\\
2.96641593285578	2.45824741206679\\
3.03106943574792	2.34622438362293\\
3.09035538994984	2.22654743366613\\
3.14391429031312	2.09860469807455\\
3.19125328747383	1.96171997457725\\
3.23172961998089	1.81516202908499\\
3.26453281477108	1.65815956638465\\
3.28866692652729	1.48992422469463\\
3.30293495665252	1.30968442431625\\
3.30592877774352	1.11673323346959\\
3.29602938262015	0.910493382839516\\
3.27142395306259	0.690601814115529\\
3.23014776659083	0.457014216525408\\
3.17015968197974	0.210126377221273\\
3.08945885590666	-0.0490964927735401\\
2.98624619175463	-0.318998769080705\\
2.85912575571581	-0.597146044523929\\
2.70732901701836	-0.880299346639518\\
2.53093039362746	-1.16448129438572\\
2.33101101524119	-1.44515283591611\\
2.10972539158081	-1.71749742055327\\
1.87023802609507	-1.97678093821641\\
1.61652380359922	-2.21873030170006\\
1.35305954468467	-2.43986251032453\\
1.08446177591869	-2.63770588575249\\
0.815136413441826	-2.81088291357719\\
0.548996485816688	-2.95905851749179\\
0.289280196144789	-3.08278558254861\\
0.0384743594774492	-3.18329306991735\\
-0.201672979692613	-3.26226070706034\\
-0.430077793684999	-3.32161311525602\\
-0.646211432851508	-3.36335192736009\\
-0.849981410527051	-3.38943187493426\\
-1.04161760406947	-3.40167823750354\\
-1.22157284792635	-3.40173856002849\\
-1.39044147947608	-3.39106017274081\\
-1.54889585570102	-3.37088556392582\\
-1.69763890928502	-3.34225904988895\\
-1.83736999643794	-3.30603979435721\\
-1.9687611709434	-3.2629176975686\\
-2.09244126968614	-3.21342986163187\\
-2.20898559685683	-3.1579762239611\\
-2.31890942029672	-3.09683357292348\\
-2.42266387964347	-3.03016757652347\\
-2.52063322625479	-2.95804272260561\\
-2.61313256500743	-2.88043023576772\\
-2.70040545437466	-2.79721413980557\\
-2.78262085456982	-2.7081957039486\\
-2.85986900638624	-2.6130965685346\\
-2.93215588849615	-2.51156090882488\\
-2.99939595191547	-2.40315707968205\\
-3.06140288241337	-2.28737930319375\\
-3.11787821370082	-2.16365013027391\\
-3.16839773091541	-2.03132463968469\\
-3.212395798176	-1.88969764544751\\
-3.24914805925617	-1.73801557117162\\
-3.27775345127651	-1.57549510699851\\
-3.29711719989476	-1.40135125129047\\
-3.30593748850052	-1.2148377624613\\
-3.302699839467	-1.01530323260925\\
-3.28568485651335	-0.802265657690554\\
-3.25299663615765	-0.575507104717066\\
-3.2026203827126	-0.3351873569578\\
-3.13251772553263	-0.0819707903550062\\
-3.04076577657179	0.182845918157791\\
-2.92573984772423	0.457225623874559\\
-2.7863293168359	0.73832973233008\\
-2.62216232277006	1.02253511756338\\
-2.43380124155044	1.30555436200757\\
-2.22286316939781	1.58266832264814\\
-1.99202429442627	1.84905394833601\\
-1.74488709228206	2.10016182470575\\
-1.48572084946671	2.33207840649466\\
-1.21911829403769	2.54180703771746\\
-0.949631265581961	2.72742165398387\\
-0.681448676680737	2.88807983140649\\
-0.41816212230173	3.02391448273191\\
-0.162637548043815	3.1358447873217\\
0.0830137649937929	3.22535264164017\\
0.317387867035559	3.29426385042997\\
0.539693124365956	3.34455979598504\\
0.749633818797704	3.3782314325697\\
0.947292108645063	3.3971767483328\\
1.13302068675687	3.40313637942729\\
};
\addplot [color=mycolor1, forget plot]
  table[row sep=crcr]{%
1.17719733975312	3.33322396506955\\
1.35030249818025	3.32778891865049\\
1.51210809846406	3.31239138794923\\
1.6633919221548	3.28823087405312\\
1.80493755041589	3.25630209514873\\
1.93750430270129	3.21741226946189\\
2.06180738253903	3.17219990803701\\
2.17850539478485	3.12115312410239\\
2.28819294015768	3.06462632145195\\
2.3913965003925	3.00285469798302\\
2.48857225515869	2.93596636593233\\
2.58010480873367	2.86399211193033\\
2.66630605624465	2.7868729486618\\
2.74741359916657	2.7044656838033\\
2.82358824267196	2.61654677904956\\
2.89491018808665	2.52281481350526\\
2.96137358622634	2.42289191838283\\
3.02287915605375	2.31632462878386\\
3.07922461390126	2.20258471810196\\
3.13009272134828	2.08107075675129\\
3.17503687128785	1.95111138561789\\
3.21346432814191	1.81197163079968\\
3.24461756879648	1.66286401890881\\
3.26755469912438	1.50296677519668\\
3.28113072177179	1.33145196182033\\
3.2839825776602	1.14752694432796\\
3.27452241765133	0.950492869956774\\
3.25094543127714	0.739823575020654\\
3.21126052595	0.515267021592148\\
3.15335365277612	0.276968374618114\\
3.07509361635764	0.0256085736204264\\
2.97448735595804	-0.237455475206265\\
2.84988440258159	-0.510072370381124\\
2.70021767939671	-0.789231277761984\\
2.52525123765674	-1.07108606918227\\
2.32578937578246	-1.35109731252547\\
2.1037934386845	-1.62430061525686\\
1.8623602868618	-1.88567821596643\\
1.60554296119306	-2.13057743858749\\
1.33803325211188	-2.35509891946982\\
1.06476312386227	-2.55638059451322\\
0.790501560057985	-2.73273083095373\\
0.519517860924475	-2.88360440588051\\
0.255356471029191	-3.00945165230741\\
0.000735039895534709	-3.11149167834557\\
-0.242450364521467	-3.19146232558214\\
-0.473044209789634	-3.25138778815474\\
-0.690508675550763	-3.29338783178077\\
-0.894787541396454	-3.3195370096192\\
-1.08617634430937	-3.33177145469118\\
-1.2652096649494	-3.33183511510098\\
-1.43256970375565	-3.32125551528476\\
-1.58901598471793	-3.301339723258\\
-1.73533370960581	-3.2731828908662\\
-1.87229737577021	-3.23768367258399\\
-2.00064620563421	-3.19556257903416\\
-2.1210682989065	-3.14738071624703\\
-2.23419094197467	-3.09355738515112\\
-2.34057504222673	-3.03438572111461\\
-2.44071212577514	-2.97004601408423\\
-2.53502271940783	-2.90061663612471\\
-2.6238552305647	-2.82608267277113\\
-2.70748465315566	-2.7463424518198\\
-2.78611057654608	-2.66121222069601\\
-2.85985407502429	-2.57042926552956\\
-2.92875311990893	-2.47365381018884\\
-2.99275620019862	-2.37047009744903\\
-3.05171387521637	-2.26038715189157\\
-3.10536803169253	-2.14283987010183\\
-3.15333870101788	-2.01719129415084\\
-3.19510844104602	-1.88273721512157\\
-3.23000454330044	-1.738714637411\\
-3.25717974709943	-1.58431611467405\\
-3.27559279693591	-1.41871252496896\\
-3.28399114506574	-1.24108742257173\\
-3.28089944041944	-1.05068654847698\\
-3.26461916262123	-0.846886146933993\\
-3.23324672514285	-0.629283014845219\\
-3.18471920149778	-0.397807139638174\\
-3.11689775481201	-0.152853719678765\\
-3.02769764519051	0.104575151663589\\
-2.91526879688354	0.372737614777059\\
-2.7782210069923	0.649055916963807\\
-2.61587298585537	0.930091699417381\\
-2.42848727696386	1.21162640459507\\
-2.21743989233247	1.48886716772943\\
-1.98527245900548	1.75677187288989\\
-1.73559192587236	2.01045298137057\\
-1.47281714099461	2.24559094462114\\
-1.20181194350925	2.4587783360375\\
-0.92747448166092	2.64773182655615\\
-0.654359680035902	2.81134496592553\\
-0.386394751691459	2.94959521024363\\
-0.126716165951527	3.06334822411246\\
0.122375429000432	3.15411339977998\\
0.359367735470225	3.22379856286806\\
0.583430829221697	3.27449645974263\\
0.794284661203453	3.30831872767862\\
0.992064213675017	3.32727966867312\\
1.17719733975312	3.33322396506955\\
};
\addplot [color=mycolor1, forget plot]
  table[row sep=crcr]{%
1.22590434247989	3.26444255553571\\
1.39778568519362	3.2590497847381\\
1.55777518892862	3.24382861499667\\
1.70675103836143	3.22003994081109\\
1.84558826292861	3.18872508758141\\
1.97512733776895	3.15072621278902\\
2.09615428269976	3.10670801538788\\
2.20938891075182	3.05717857651015\\
2.31547857593696	3.00250813821333\\
2.41499540189816	2.94294526954143\\
2.50843549202283	2.87863026509742\\
2.59621902113075	2.80960585033293\\
2.67869040132478	2.73582539079166\\
2.75611791888507	2.65715886373023\\
2.82869237462659	2.5733968811675\\
2.89652434460227	2.48425307600197\\
2.9596397271147	2.38936519468617\\
3.01797327033446	2.28829529598714\\
3.07135979769443	2.1805295503353\\
3.11952288389663	2.06547828426429\\
3.16206080708689	1.94247713761692\\
3.19842974646615	1.81079051711406\\
3.22792445709442	1.66961895656359\\
3.24965710022171	1.51811254070199\\
3.26253562201684	1.35539320285037\\
3.26524415210251	1.18058940520403\\
3.25622942186491	0.992887307773702\\
3.23369920606965	0.791602743226298\\
3.19564113848408	0.576277664196529\\
3.1398725068913	0.346802528699445\\
3.06413287471086	0.103561521425096\\
2.96623008252873	-0.152410070565631\\
2.84424437601905	-0.419277452491931\\
2.69678336750397	-0.694299485893786\\
2.52326227368228	-0.973804263907594\\
2.32416298853439	-1.25328660214706\\
2.10121018814479	-1.52765009372196\\
1.85740346435531	-1.79158308598752\\
1.59686869882091	-2.04001649265165\\
1.32453632558024	-2.26857849595714\\
1.04570301329607	-2.4739540329236\\
0.765565360812775	-2.65408168115743\\
0.488815500179686	-2.80816699289228\\
0.219360997648381	-2.93653887741222\\
-0.0398095987651686	-3.04040602242973\\
-0.286630899201536	-3.12157692434183\\
-0.519871267646057	-3.18219496718836\\
-0.738991605563387	-3.22451960078847\\
-0.943987665821066	-3.25076517916732\\
-1.13524008710841	-3.26299522676806\\
-1.3133853463107	-3.26306258127946\\
-1.47921241980635	-3.25258358098722\\
-1.63358462390113	-3.23293521265411\\
-1.77738335876483	-3.20526623833188\\
-1.9114694962231	-3.17051570223822\\
-2.03665819186413	-3.12943433562294\\
-2.15370343312798	-3.08260603329976\\
-2.26328932688233	-3.03046776553468\\
-2.36602580463572	-2.9733270897908\\
-2.4624470016689	-2.91137693497489\\
-2.55301102484149	-2.84470763450985\\
-2.63810016768621	-2.77331635409199\\
-2.71802087730646	-2.69711414748705\\
-2.79300294516599	-2.61593091635425\\
-2.86319750178541	-2.52951857428171\\
-2.92867346026797	-2.43755274047984\\
-2.98941209046259	-2.33963333071361\\
-3.04529942914146	-2.23528448666135\\
-3.09611625826001	-2.12395440523951\\
-3.14152543401717	-2.00501581362534\\
-3.18105645264084	-1.87776810279971\\
-3.21408733492589	-1.7414425017944\\
-3.23982425672907	-1.59521216135885\\
-3.25727992165527	-1.43820961964884\\
-3.26525255654388	-1.26955481037257\\
-3.26230870533485	-1.08839744737734\\
-3.24677476675413	-0.893978070280039\\
-3.21674442802203	-0.685711889067924\\
-3.17011152831026	-0.463298237743598\\
-3.10463978567584	-0.226855164446353\\
-3.01808102245287	0.0229273383947193\\
-2.9083502301115	0.284631072839721\\
-2.77375704158983	0.555977281428288\\
-2.61327784344003	0.833755689427438\\
-2.42683244890199	1.11385674149264\\
-2.2155098612699	1.3914402852622\\
-1.98167908157629	1.66124848376536\\
-1.72893302723864	1.91803170286154\\
-1.46184919114113	2.15701658680652\\
-1.18559991910093	2.37432382629505\\
-0.905488161095217	2.56725215472164\\
-0.626502063598137	2.73438287412226\\
-0.352967397025358	2.87550894157846\\
-0.0883401830366873	2.99143329249239\\
0.164858930307504	3.08369955520424\\
0.404996785704091	3.1543143820327\\
0.631207403458537	3.20550301505091\\
0.843238767849654	3.23951886961585\\
1.04129670838009	3.25851096417827\\
1.22590434247988	3.26444255553571\\
};
\addplot [color=mycolor1, forget plot]
  table[row sep=crcr]{%
1.27969473451899	3.19743891413949\\
1.45034994159423	3.19208882280089\\
1.608468718383	3.17704946054004\\
1.75504743713589	3.15364704231165\\
1.8910667842996	3.12297092862\\
2.01745932978799	3.08589794332656\\
2.13509003936657	3.04311762731975\\
2.24474574897865	2.99515607355915\\
2.34713052878886	2.9423971209665\\
2.44286465667115	2.88510040147726\\
2.53248555223779	2.82341615911886\\
2.61644949636217	2.75739699170234\\
2.6951332998723	2.68700677794949\\
2.76883531595009	2.61212709684994\\
2.8377753393484	2.53256145660749\\
2.90209302353936	2.44803765162519\\
2.96184449223774	2.35820857506451\\
3.01699683974581	2.2626518460952\\
3.06742021994134	2.16086867923668\\
3.11287723304133	2.05228254415422\\
3.15300935413325	1.93623835707118\\
3.18732023860967	1.81200323241128\\
3.21515593194735	1.67877023059018\\
3.23568236933861	1.53566708793343\\
3.24786116349761	1.38177261724521\\
3.25042565989046	1.21614429781399\\
3.24186071298692	1.03786143510928\\
3.22039170350251	0.846088937327736\\
3.18399096149264	0.640166798123835\\
3.13041270376352	0.41972908786561\\
3.05727006314035	0.184852660212058\\
2.9621682737045	-0.0637711837261371\\
2.84290427252501	-0.324659745035805\\
2.69773225686422	-0.595388289159807\\
2.5256756024321	-0.872510538970974\\
2.32684009585234	-1.15160055593432\\
2.10265963161525	-1.42745523771516\\
1.85599667089474	-1.69446398913854\\
1.59103854112513	-1.94710290900718\\
1.31297908263676	-2.18046319213452\\
1.02753779962426	-2.39070140666667\\
0.740417862665125	-2.57531756857995\\
0.456816757094041	-2.73321902000345\\
0.18107562986538	-2.8645894324229\\
-0.0834972651340051	-2.97062633628157\\
-0.33464256235706	-3.05322456841679\\
-0.571049054307578	-3.11467098787108\\
-0.79218830185694	-3.15739101298888\\
-0.998129124263412	-3.18376270669748\\
-1.18936222632091	-3.19599631795903\\
-1.36665104937862	-3.19606775514251\\
-1.53091423172332	-3.1856915971373\\
-1.68313847021417	-3.16632026979512\\
-1.82431733797418	-3.13915871316964\\
-1.95541060374083	-3.10518684935743\\
-2.07731882889597	-3.06518475312069\\
-2.19086879991828	-3.01975740852929\\
-2.296806278529	-2.96935732324368\\
-2.39579341215939	-2.91430418010914\\
-2.48840886072844	-2.85480126234141\\
-2.57514924628002	-2.79094870596757\\
-2.65643093499139	-2.72275379750876\\
-2.73259144246289	-2.65013860775138\\
-2.80388993994359	-2.57294527616556\\
-2.87050645497965	-2.4909392638613\\
-2.93253942459908	-2.40381089613441\\
-2.99000128898612	-2.31117553424823\\
-3.04281182340003	-2.21257276418519\\
-3.09078891113382	-2.10746508255407\\
-3.13363647922882	-1.99523671409207\\
-3.17092937706462	-1.87519343226531\\
-3.20209511295751	-1.74656459867964\\
-3.22639262956776	-1.60850911330203\\
-3.24288877161496	-1.46012759407702\\
-3.25043388083961	-1.30048387741061\\
-3.24763916545357	-1.12863979540267\\
-3.232860253651	-0.943707989005588\\
-3.20419371017733	-0.744927940467152\\
-3.15949614900126	-0.531769892733092\\
-3.09643841910547	-0.304069017672216\\
-3.01260905935427	-0.062187036124715\\
-2.90567988916881	0.192810438976774\\
-2.77363964444937	0.458985248496227\\
-2.61508666850671	0.733405387358675\\
-2.4295487824329	1.01212011812496\\
-2.21777243410176	1.29027878622215\\
-1.98190506393249	1.56241892651132\\
-1.72549842501278	1.82290682446117\\
-1.45329498813935	2.06646233416912\\
-1.17081808859522	2.28866210291941\\
-0.883846036343302	2.48631281300606\\
-0.597883133052734	2.65762344314827\\
-0.317731837641074	2.8021658944279\\
-0.0472280991005781	2.92066872817214\\
0.210850791495053	3.01471826027117\\
0.454737366582832	3.08644089581483\\
0.683534623061324	3.13822035187157\\
0.89703583690876	3.17247731940366\\
1.09554128372098	3.19151731174063\\
1.27969473451899	3.19743891413949\\
};
\addplot [color=mycolor1, forget plot]
  table[row sep=crcr]{%
1.3392619362044	3.13288106114262\\
1.5086791795128	3.12757437885239\\
1.6648604132956	3.11272344007181\\
1.80894084011687	3.08972360883537\\
1.94202280659676	3.05971330019967\\
2.06514290229219	3.02360320058587\\
2.1792535778672	2.98210580929765\\
2.28521452921726	2.93576279359985\\
2.3837902847826	2.88496894959092\\
2.47565142870222	2.82999235401166\\
2.56137766034903	2.77099073935933\\
2.64146145066806	2.70802435097411\\
2.71631144520293	2.6410656385359\\
2.78625502182879	2.5700061549625\\
2.85153957186433	2.49466102162765\\
2.91233216336754	2.41477129524368\\
2.96871728581953	2.33000455594079\\
3.02069238245767	2.23995404162502\\
3.06816086448863	2.14413669371357\\
3.11092228510539	2.04199056894032\\
3.14865934933596	1.93287223022889\\
3.18092147510529	1.81605498124403\\
3.2071047420893	1.6907291845516\\
3.22642832936629	1.55600643641962\\
3.23790804050797	1.41093009280062\\
3.24032837021464	1.2544955614675\\
3.23221593702453	1.08568485209026\\
3.21181915823305	0.903520955896151\\
3.17710188324506	0.707148356743944\\
3.12576223284096	0.495945692402419\\
3.05529155810392	0.269674225576345\\
2.96309086907791	0.0286599408519614\\
2.84666081340954	-0.226003578863136\\
2.70387285481623	-0.492260043094357\\
2.5333104918282	-0.766949791157519\\
2.33464000483891	-1.0457837492775\\
2.10893700227236	-1.3234898634351\\
1.85887326877483	-1.59416127410092\\
1.58867727353884	-1.85178014035832\\
1.30383136652625	-2.09082616151757\\
1.01054697329997	-2.30683587515119\\
0.715131634073073	-2.49678465263894\\
0.423391948519504	-2.6592193276381\\
0.140191149224581	-2.79414784538079\\
-0.13078263680191	-2.90275563289121\\
-0.387047456258826	-2.98704378778889\\
-0.627211121354136	-3.04947311479465\\
-0.850773576984367	-3.09266743431044\\
-1.05790511290341	-3.11919740795147\\
-1.24923868994443	-3.13144278304647\\
-1.42569600205082	-3.13151872715371\\
-1.58835308692518	-3.12124838514936\\
-1.73834309104347	-3.10216529525421\\
-1.87679002952083	-3.0755328613566\\
-2.00476644935078	-3.04237188331492\\
-2.12326846130743	-3.00349035732629\\
-2.23320275723049	-2.95951214245916\\
-2.33538147355553	-2.91090271021194\\
-2.43052186422163	-2.85799121830943\\
-2.51924862748709	-2.80098875199537\\
-2.60209739095236	-2.74000289964855\\
-2.67951832872522	-2.67504898055312\\
-2.75187920382206	-2.60605829388496\\
-2.81946733449019	-2.53288375738476\\
-2.88249010557909	-2.45530328282913\\
-2.94107370907848	-2.3730212138732\\
-2.9952598198123	-2.28566814510265\\
-3.04499990804278	-2.19279946220562\\
-3.09014687456381	-2.09389300588201\\
-3.13044368203036	-1.98834638334865\\
-3.16550867093336	-1.87547465280863\\
-3.19481732271165	-1.75450941559799\\
-3.21768041602521	-1.62460080063886\\
-3.23321888992378	-1.48482445001928\\
-3.24033638637087	-1.33419643647127\\
-3.23769153638757	-1.17170005268329\\
-3.22367374567674	-0.996329519905232\\
-3.196388676233	-0.807156629135423\\
-3.15366283961751	-0.603426658440192\\
-3.09308042758429	-0.384688747459116\\
-3.01206879249482	-0.150961996233866\\
-2.90804995608205	0.0970695671635621\\
-2.7786711381308	0.357853290762011\\
-2.62211397341698	0.628793108926508\\
-2.43745768365259	0.906158296836139\\
-2.22503879901498	1.1851375243136\\
-1.98672015936148	1.46008544100662\\
-1.72597326286571	1.72496609922759\\
-1.44770719319412	1.97393409200938\\
-1.15784449555035	2.20193615907363\\
-0.862724864969265	2.40519530789025\\
-0.5684724898963	2.58147229590912\\
-0.280464895163095	2.7300712736549\\
-0.0029932945470816	2.85163151150354\\
0.260864143836346	2.94779278933954\\
0.509191499615886	3.02082769660696\\
0.741070000077793	3.07331076844402\\
0.956362264850219	3.10786112278658\\
1.15549403945334	3.12696676719541\\
1.3392619362044	3.13288106114262\\
};
\addplot [color=mycolor1, forget plot]
  table[row sep=crcr]{%
1.4054745831124	3.07149337387774\\
1.57362727100623	3.06623132298505\\
1.72778699106429	3.05157706491179\\
1.8692516647565	3.02899874981702\\
1.99926321162977	2.99968435119239\\
2.11897510688696	2.96457699823885\\
2.22943611038759	2.92440967730864\\
2.33158448622435	2.87973669499959\\
2.42624859489868	2.83096077860544\\
2.51415099035098	2.77835555376001\\
2.59591408402611	2.72208359692615\\
2.67206609429429	2.66221046842641\\
2.74304644182516	2.59871519613369\\
2.80921003520959	2.53149766884962\\
2.87083006084297	2.46038335395687\\
2.92809898034762	2.38512570196443\\
2.98112747194105	2.30540655755974\\
3.02994104686825	2.22083487486481\\
3.07447404219802	2.13094404515007\\
3.1145606499368	2.03518820178653\\
3.14992260504253	1.93293798778814\\
3.1801531438731	1.82347648080454\\
3.20469689494574	1.70599630232067\\
3.22282553140833	1.57959943447058\\
3.23360938613093	1.44330197737494\\
3.23588593502282	1.29604704519222\\
3.22822727176805	1.1367302319388\\
3.20891065145785	0.964243506987811\\
3.17589909545075	0.777544777654567\\
3.12684302710255	0.575761118613226\\
3.05911867262978	0.358332765458058\\
2.96992343272682	0.125200762736179\\
2.85645018416743	-0.122968392686797\\
2.71615742706347	-0.384544218725802\\
2.54713532170004	-0.656725403355817\\
2.34853577548911	-0.935432686045723\\
2.12099170299253	-1.21537911606742\\
1.86691228769943	-1.49037570040727\\
1.59053375100839	-1.75387251044781\\
1.29765092469468	-1.99965166553425\\
0.995049756673487	-2.22251711528282\\
0.689764855128191	-2.41881039084985\\
0.388344336714448	-2.58663740439976\\
0.0962863203254775	-2.72579076860021\\
-0.182266134892198	-2.83744263071393\\
-0.444575305476755	-2.92372607293893\\
-0.689170200651332	-2.98731462456856\\
-0.915605230675081	-3.03107101158091\\
-1.124190568135	-3.05779374085243\\
-1.3157431266952	-3.07005907314908\\
-1.49138202257451	-3.07013998665209\\
-1.65237434032767	-3.05997949315592\\
-1.80002680080641	-3.04119800758308\\
-1.93561465998037	-3.0151192949233\\
-2.06033850655863	-2.98280444559781\\
-2.17530072283779	-2.94508734198385\\
-2.28149506482953	-2.90260795655498\\
-2.37980449691982	-2.85584170843552\\
-2.4710038297239	-2.80512424893919\\
-2.55576479613425	-2.75067168453415\\
-2.6346619863128	-2.69259656245031\\
-2.70817860414535	-2.63092006979291\\
-2.77671136436757	-2.56558091705725\\
-2.84057407142502	-2.4964413451172\\
-2.8999995470633	-2.42329064404795\\
-2.95513963227719	-2.34584652296658\\
-3.00606300125895	-2.26375463621684\\
-3.05275050582436	-2.17658656405072\\
-3.09508773151852	-2.08383657758257\\
-3.13285440499613	-1.9849176037046\\
-3.16571026463317	-1.87915696707727\\
-3.19317702051026	-1.76579275225261\\
-3.21461613019455	-1.64397203747039\\
-3.2292023730239	-1.51275284866754\\
-3.23589372448983	-1.37111251535963\\
-3.23339896779886	-1.217966211236\\
-3.22014603594558	-1.05220081037912\\
-3.19425648770722	-0.872730640045606\\
-3.15353496995275	-0.678582877771494\\
-3.09548696227414	-0.469020438077985\\
-3.01738292443531	-0.243707879537554\\
-2.91639048564559	-0.00291924116312545\\
-2.78979527494392	0.252226179261229\\
-2.63532064441046	0.519533917556674\\
-2.45153225954353	0.795567761858333\\
-2.23827491409184	1.07562197911916\\
-1.99704545399365	1.35390475240838\\
-1.73117936849501	1.62396651620491\\
-1.44574621356274	1.87933276613553\\
-1.14712218591709	2.11421751592803\\
-0.842314357326939	2.32414547803164\\
-0.538198389740975	2.50633192203147\\
-0.240852599292319	2.65975270557451\\
0.0448812356255614	2.78493845106198\\
0.315570224341204	2.88359640994968\\
0.569135749148115	2.9581792577659\\
0.804652925319656	3.01149314457779\\
1.02208728745503	3.04639408810963\\
1.22203103342413	3.06558388836308\\
1.4054745831124	3.07149337387774\\
};
\addplot [color=mycolor1, forget plot]
  table[row sep=crcr]{%
1.47942380987548	3.01409903169868\\
1.64626455039693	3.00888351732135\\
1.79829624689677	2.9944363478465\\
1.93700704846841	2.97230176776496\\
2.06379831826375	2.94371715953913\\
2.1799540989066	2.90965595947843\\
2.28662837119145	2.87086852888969\\
2.38484333686672	2.82791837021459\\
2.4754940038472	2.78121274986893\\
2.55935591460565	2.73102770910772\\
2.63709397460015	2.67752789168202\\
2.70927109620344	2.62078178779733\\
2.77635586641289	2.56077301411612\\
2.83872875041704	2.49740819628071\\
2.89668651596338	2.43052193867583\\
2.95044464676317	2.35987928153766\\
3.0001375352307	2.28517597293728\\
3.04581622485987	2.20603683256147\\
3.08744342398849	2.12201246457777\\
3.12488544666132	2.03257459934331\\
3.15790066486958	1.93711042432436\\
3.18612399696444	1.83491642705397\\
3.20904693798338	1.72519255157135\\
3.22599270701752	1.60703791200326\\
3.23608632413714	1.47944997405672\\
3.23821996159076	1.34133007776124\\
3.23101493250038	1.19149949585151\\
3.21278345253694	1.0287319156817\\
3.1814961642855	0.851810177806806\\
3.13476565527705	0.659616892903152\\
3.06986188544866	0.451269262738913\\
2.9837819155047	0.226306340358796\\
2.87340149427001	-0.0150705104270993\\
2.73573556174386	-0.271718997598551\\
2.56832184339207	-0.54127972878473\\
2.36970959446515	-0.819974718732148\\
2.1399833195045	-1.10257758631277\\
1.88119288598873	-1.38264819930625\\
1.59752959516814	-1.65307042307391\\
1.29512089242035	-1.90682964704465\\
0.981426447742972	-2.13785775176816\\
0.66436393646992	-2.34172200086941\\
0.351394605016278	-2.51598196277505\\
0.0487974547586737	-2.6601626822905\\
-0.23873474837032	-2.77542164655584\\
-0.508170243751502	-2.86405775042834\\
-0.757967507119653	-2.92900735035789\\
-0.98777367530101	-2.9734232879994\\
-1.1980910269562	-3.00037532671187\\
-1.38997469205245	-3.01266848879099\\
-1.56479036970873	-3.01275487347464\\
-1.72403700215334	-3.00270968394413\\
-1.86922667587088	-2.98424595906764\\
-2.00180940524786	-2.95874918876373\\
-2.12313041210999	-2.92731945832026\\
-2.23440947949695	-2.89081381148429\\
-2.33673442944079	-2.84988499437422\\
-2.43106303818853	-2.80501492086142\\
-2.5182295086471	-2.75654245540378\\
-2.59895294951886	-2.70468576332112\\
-2.67384623600738	-2.64955976796782\\
-2.74342424195536	-2.59118933749998\\
-2.80811082315821	-2.52951880014098\\
-2.86824416379404	-2.46441831516597\\
-2.92408022172711	-2.39568754127205\\
-2.97579405823449	-2.32305696396324\\
-3.02347883677504	-2.24618718088355\\
-3.06714223968626	-2.16466640771873\\
-3.10669999303777	-2.07800646707122\\
-3.14196611940757	-1.98563757183635\\
-3.17263947018963	-1.88690233258323\\
-3.19828604522639	-1.78104963392229\\
-3.21831662507747	-1.66722937843925\\
-3.23195938213613	-1.54448964299202\\
-3.2382275020994	-1.41177859804979\\
-3.23588259406218	-1.26795467637459\\
-3.22339602477535	-1.11180998860894\\
-3.198912583081	-0.94211383125201\\
-3.16022439810807	-0.757685076645846\\
-3.10476802094913	-0.557503640031831\\
-3.02966381164673	-0.340870796261394\\
-2.93182298174123	-0.107623705902376\\
-2.80815068243951	0.141601790059955\\
-2.65586777743815	0.405086766923669\\
-2.47295211775567	0.679779419408755\\
-2.25865735059234	0.961166577758489\\
-2.01400900809958	1.24336659938685\\
-1.7421274693308	1.51951626565132\\
-1.44822398926211	1.78244372089397\\
-1.13918912623464	2.02550654974069\\
-0.822829286010876	2.24338606724704\\
-0.506941025352666	2.43262553182077\\
-0.198466686990418	2.59179277046852\\
0.0970790994492219	2.72128431288835\\
0.375842686436026	2.82289348564613\\
0.635569619905188	2.89929725677801\\
0.87535433088247	2.95358556276458\\
1.09531230173884	2.98889935870533\\
1.29625660842899	3.00819221393258\\
1.47942380987548	3.01409903169868\\
};
\addplot [color=mycolor1, forget plot]
  table[row sep=crcr]{%
1.56248802181173	2.9616734001184\\
1.72794160058687	2.95650722867695\\
1.87771037776087	2.94228025725477\\
2.01350416846367	2.9206156108536\\
2.13690554077677	2.89279910679247\\
2.24934335930265	2.85983152504823\\
2.3520855292164	2.8224767991473\\
2.44624293856636	2.781303675091\\
2.5327792614432	2.73672023362352\\
2.61252321755089	2.68900161357266\\
2.68618120369028	2.63831167779904\\
2.75434907197669	2.5847194697673\\
2.81752236361814	2.52821126532338\\
2.87610462006534	2.46869891581417\\
2.93041355954928	2.40602505151122\\
2.98068497658418	2.33996559239673\\
3.02707422747102	2.27022990888279\\
3.06965512668631	2.19645889483955\\
3.10841601017396	2.1182211652924\\
3.14325263091391	2.03500757937128\\
3.17395744815529	1.94622432861669\\
3.20020476631742	1.85118494273159\\
3.22153109376454	1.74910178229258\\
3.23731006382553	1.63907795993805\\
3.24672135907247	1.52010122767342\\
3.24871342237838	1.3910422798955\\
3.24196051612137	1.25066125707407\\
3.22481619765923	1.09762809066295\\
3.19526792680432	0.930564723553199\\
3.15090180614554	0.748119965368382\\
3.08889279489712	0.549090102380487\\
3.00604403733272	0.332598796784509\\
2.89890774126572	0.0983453980010447\\
2.76402519326072	-0.153082636643624\\
2.59831705952781	-0.419864091461348\\
2.39962622519607	-0.698635848076194\\
2.16735629533455	-0.98433558391937\\
1.90306755215678	-1.27032769523082\\
1.6108251747217	-1.54890500107748\\
1.29710068781251	-1.81214296080079\\
0.970146867704863	-2.0529270739582\\
0.638965572146003	-2.26586713629577\\
0.312157381743099	-2.44783584723404\\
-0.00302571524577319	-2.59802067678299\\
-0.301220094054039	-2.71756300043051\\
-0.579056423883941	-2.80897290177942\\
-0.8349411321204	-2.87551534463542\\
-1.06867026275776	-2.92069886383828\\
-1.28100962445009	-2.94791844465975\\
-1.47332349419031	-2.96024657501622\\
-1.6472859563713	-2.96033897061529\\
-1.80467722973043	-2.95041637068623\\
-1.94725182593645	-2.93228999711988\\
-2.07666084446663	-2.90740770495249\\
-2.19441188468987	-2.87690642664304\\
-2.30185338601962	-2.84166287301235\\
-2.40017378689372	-2.80233862274819\\
-2.4904089277545	-2.75941821344266\\
-2.57345341291102	-2.7132401876885\\
-2.6500732538513	-2.66402168351149\\
-2.72091818762069	-2.61187739006263\\
-2.78653274547894	-2.55683370715729\\
-2.84736555949864	-2.49883886357561\\
-2.90377662674676	-2.43776962713736\\
-2.95604236355122	-2.37343511295737\\
-3.00435831666581	-2.30557808202369\\
-3.04883937980577	-2.23387402918455\\
-3.08951730923333	-2.15792829371994\\
-3.12633525131421	-2.07727139352432\\
-3.15913889648398	-1.99135279573921\\
-3.18766376747544	-1.89953340913903\\
-3.21151805059608	-1.80107724325244\\
-3.23016031570991	-1.69514296658646\\
-3.24287149488152	-1.58077657029018\\
-3.24872069156883	-1.45690708487812\\
-3.24652492274259	-1.32234840752889\\
-3.23480399458236	-1.17581188348314\\
-3.21173373099997	-1.01593641696128\\
-3.17510417730993	-0.841345494965563\\
-3.12229467922138	-0.650743181776849\\
-3.05028512889501	-0.443062800463219\\
-2.95573153358211	-0.217680470676936\\
-2.835141774389	0.0253026699225968\\
-2.68518801965275	0.284725337264344\\
-2.50317597893971	0.558027535137331\\
-2.28764751658583	0.841001509445815\\
-2.03902032112432	1.12776023726286\\
-1.76008767943735	1.41104522436188\\
-1.45616345687017	1.6829170560622\\
-1.13471863967304	1.93572767581246\\
-0.804524934801826	2.16312908782645\\
-0.474521414348072	2.36082533570033\\
-0.152730019317042	2.52686942081269\\
0.154502462670652	2.66148988749968\\
0.44282014922353	2.76659143926723\\
0.709783032080508	2.84513401290853\\
0.954545295336056	2.90055901688932\\
1.17743860785017	2.93635299009173\\
1.37956959771753	2.95576769492973\\
1.56248802181173	2.9616734001184\\
};
\addplot [color=mycolor1, forget plot]
  table[row sep=crcr]{%
1.65642263207936	2.91541373390173\\
1.82037784583778	2.91030086390921\\
1.9677141607887	2.89631048846228\\
2.10039858637834	2.87514661502634\\
2.22021829502882	2.84814149164205\\
2.32876139289674	2.81631933318551\\
2.42741778412894	2.78045311346391\\
2.51739080453339	2.74111237919629\\
2.5997137273383	2.69870202037562\\
2.67526759617419	2.65349283915425\\
2.74479836712207	2.60564506995236\\
2.80893228619176	2.55522600914459\\
2.86818898383772	2.50222278100859\\
2.92299207190501	2.44655108756695\\
2.97367717316179	2.38806060833533\\
3.02049735848801	2.32653755200114\\
3.06362594816039	2.26170472357171\\
3.1031565727605	2.19321936030627\\
3.13910029784126	2.12066890980134\\
3.17137950106921	2.04356487823619\\
3.19981805563009	1.96133487528602\\
3.22412722545868	1.87331304194841\\
3.24388652917448	1.77872919811096\\
3.25851870767061	1.67669733555935\\
3.26725788924465	1.56620458170098\\
3.26911018723029	1.44610257455019\\
3.26280646739801	1.3151044624267\\
3.24674818526298	1.17179264275761\\
3.21894948375011	1.01464503971421\\
3.17698281835873	0.842091215225842\\
3.11794201723986	0.652613547633571\\
3.03844644930004	0.444911865172639\\
2.93472234736788	0.218149481119258\\
2.8028090465519	-0.0277104952854724\\
2.63894062157472	-0.291494298734732\\
2.44013251083137	-0.570393290038135\\
2.20494150359352	-0.859649139467993\\
1.93426251361171	-1.1525223052927\\
1.63191099575297	-1.44070778788784\\
1.30469689836056	-1.71524417896233\\
0.961809545903871	-1.96775111774133\\
0.613598070559528	-2.19163828849998\\
0.27010603400576	-2.3829008079898\\
-0.0601842625425439	-2.54029310737039\\
-0.371081514574717	-2.66493929156151\\
-0.658830666399525	-2.75962282515189\\
-0.921821623365127	-2.82802513033402\\
-1.16008213079778	-2.87409551910905\\
-1.37473982056242	-2.90162188662688\\
-1.56756118680114	-2.91399083853101\\
-1.74060641809601	-2.91408981741843\\
-1.89599662931821	-2.90429937494904\\
-2.03577159460937	-2.88653403208306\\
-2.16181254714747	-2.86230368386328\\
-2.27580800711821	-2.83277894819267\\
-2.37924602595903	-2.79885186349478\\
-2.47342135895941	-2.76118829145959\\
-2.55945010266753	-2.72027114215685\\
-2.63828719565115	-2.67643490915497\\
-2.71074408814828	-2.6298925648727\\
-2.77750509407517	-2.58075599703199\\
-2.83914166821686	-2.52905108890512\\
-2.89612426683683	-2.47472838286705\\
-2.94883166501475	-2.41767008273216\\
-2.99755769310528	-2.35769397601212\\
-3.04251536440935	-2.2945547054575\\
-3.08383832442057	-2.22794269467252\\
-3.12157947480033	-2.15748093717595\\
-3.15570652122286	-2.08271979484268\\
-3.18609406848149	-2.00312992664664\\
-3.21251174361749	-1.91809349492053\\
-3.23460767710368	-1.82689389684362\\
-3.25188653209408	-1.72870448128615\\
-3.26368118112669	-1.62257709444266\\
-3.26911716301003	-1.50743193969673\\
-3.2670693455275	-1.38205126127052\\
-3.2561110053532	-1.24508092609879\\
-3.23445719275566	-1.09504625615185\\
-3.19990733869133	-0.930391566459729\\
-3.14979732421527	-0.749556666920958\\
-3.08097940364488	-0.551107366131702\\
-2.98985962964534	-0.333938851596171\\
-2.87253514113668	-0.0975668422736816\\
-2.72508233437513	0.15749452962294\\
-2.54404001527801	0.429303881850829\\
-2.32709234919699	0.714103779021441\\
-2.07387200165371	1.00612432375832\\
-1.78668666913353	1.29775933191409\\
-1.47088122335579	1.58023484821087\\
-1.1345756241589	1.84471730760218\\
-0.787717671493854	2.083588282538\\
-0.440683840723204	2.2914878559795\\
-0.102865668160094	2.46580854778831\\
0.218346993157529	2.60656514302512\\
0.517995089490967	2.71580904103266\\
0.793451069451778	2.79686253352453\\
1.04399255122457	2.85360765026289\\
1.27026126265922	2.88995393049057\\
1.47375493321376	2.90950846059762\\
1.65642263207936	2.91541373390173\\
};
\addplot [color=mycolor1, forget plot]
  table[row sep=crcr]{%
1.76348601971647	2.87683329369742\\
1.92578633299117	2.87177911807686\\
2.07047954639434	2.85804562529419\\
2.19982936085868	2.837418579363\\
2.3158520581634	2.81127338959993\\
2.42030894180586	2.78065273973874\\
2.51471878827486	2.74633336758708\\
2.60037969658333	2.70888073371149\\
2.67839404839091	2.66869233009578\\
2.74969308449871	2.62603116006154\\
2.81505931225171	2.58105106779765\\
2.87514595113622	2.53381545319279\\
2.93049316546575	2.48431065802551\\
2.98154109959006	2.43245504247299\\
3.02863983365929	2.3781045253445\\
3.07205638379692	2.32105515081392\\
3.11197881832259	2.26104307036364\\
3.14851747228343	2.19774218859202\\
3.18170312587335	2.13075961284405\\
3.21148187177012	2.05962896925004\\
3.23770623227883	1.98380160652951\\
3.2601218996854	1.90263571639291\\
3.27834926667941	1.8153834799699\\
3.2918587030151	1.7211765459532\\
3.29993835675765	1.61901052787068\\
3.30165319257391	1.5077298845134\\
3.29579418014286	1.38601568238664\\
3.28081729572738	1.25238055969162\\
3.25477378510873	1.10517799928342\\
3.21523672977476	0.94263703848744\\
3.15923545962907	0.762938836079086\\
3.08322002447145	0.56435741322315\\
2.98309348302936	0.345491101696972\\
2.85436853747411	0.105608547675051\\
2.692519632201	-0.154885635179985\\
2.49359485450611	-0.433906731071035\\
2.25509698278377	-0.727187237009924\\
1.97701804067147	-1.02802696071195\\
1.66273686318081	-1.32754885415733\\
1.31936433370759	-1.61561712920537\\
0.957197314032151	-1.88230756971129\\
0.588281686375872	-2.1195031919381\\
0.224519787448563	-2.32205758013215\\
-0.124024317679309	-2.4881594718107\\
-0.450126094146457	-2.61891568901138\\
-0.749594467833043	-2.71747033472558\\
-1.02086678096851	-2.78804076186645\\
-1.26432516705138	-2.83512893618276\\
-1.48159516891929	-2.86300127331954\\
-1.67496794109156	-2.87541488118393\\
-1.84698734463292	-2.87552103698405\\
-2.00018683612315	-2.86587509023157\\
-2.13694034232468	-2.84849917106608\\
-2.25939062132157	-2.82496362814055\\
-2.36942584813439	-2.79646841722182\\
-2.46868372951739	-2.7639156604951\\
-2.55856966932492	-2.72797036453293\\
-2.64028075426689	-2.68910924597426\\
-2.71483083160932	-2.64765891538533\\
-2.78307414907135	-2.60382507704837\\
-2.84572633842907	-2.55771437365284\\
-2.90338226450221	-2.50935029296217\\
-2.95653064787545	-2.45868428788424\\
-3.00556554356136	-2.40560300220309\\
-3.05079480570215	-2.34993226542908\\
-3.09244564168406	-2.29143832806084\\
-3.13066728644761	-2.22982665188924\\
-3.16553072400175	-2.1647384458182\\
-3.19702525437404	-2.09574504440137\\
-3.22505155192867	-2.02234016598363\\
-3.24941068500344	-1.94393006826753\\
-3.26978836880001	-1.85982165915236\\
-3.28573351194609	-1.76920875235269\\
-3.29662991623451	-1.67115693541895\\
-3.30165985357059	-1.56458803023886\\
-3.29975828568289	-1.44826600751467\\
-3.28955692138947	-1.32078766195834\\
-3.26931849695565	-1.18058362230308\\
-3.23686423711171	-1.02593864805765\\
-3.18950234636825	-0.85504484502446\\
-3.12397381588454	-0.666107183445228\\
-3.03644493809001	-0.457526194514793\\
-2.92259354736716	-0.228184376895331\\
-2.77785413884563	0.0221464900115145\\
-2.59789378652418	0.292291090070833\\
-2.37936353476511	0.579126249471295\\
-2.12088114755301	0.877173423796475\\
-1.8240442075674	1.17857189841392\\
-1.49410500651416	1.47365939389888\\
-1.13989674277931	1.75220125656733\\
-0.772813421464587	2.00499206780641\\
-0.405068595964772	2.22530026810592\\
-0.0478214785723391	2.40965525281262\\
0.290210933590949	2.55779532341061\\
0.603341316210892	2.67196929045974\\
0.888768223076226	2.75597143720336\\
1.14599270169202	2.81424369783125\\
1.37610047475835	2.85121851829582\\
1.58111190506633	2.870929013894\\
1.76348601971647	2.87683329369742\\
};
\addplot [color=mycolor1, forget plot]
  table[row sep=crcr]{%
1.88661887654611	2.84789220388966\\
2.04705214990411	2.8429038537427\\
2.18884443214041	2.82945197986441\\
2.31459898637225	2.80940341620734\\
2.42658588194401	2.78417194600725\\
2.52675220198436	2.75481261074247\\
2.61675052305756	2.72209991644837\\
2.69797402742826	2.68658999318215\\
2.77159191642958	2.64866863606242\\
2.83858199221868	2.60858767046551\\
2.89975909593142	2.56649197280371\\
2.95579906312272	2.52243912845706\\
3.00725833086018	2.47641330663042\\
3.05458952043593	2.42833455819933\\
3.0981533527615	2.37806442389918\\
3.13822720222123	2.32540847912831\\
3.17501049763042	2.27011623126171\\
3.20862705450401	2.21187861648089\\
3.23912427775346	2.15032320768242\\
3.26646900783157	2.08500713814001\\
3.29053959190897	2.0154076671603\\
3.31111353902201	1.94091027094747\\
3.32784985983523	1.86079415202803\\
3.34026489971346	1.77421515866877\\
3.34770016613014	1.68018635367781\\
3.34928037988859	1.57755697444958\\
3.34385986297503	1.46499145273588\\
3.32995565774689	1.34095177579936\\
3.30566691605795	1.20368914956853\\
3.268582925055	1.05125515563213\\
3.21568798943698	0.881548851538667\\
3.14328218917604	0.692424616921975\\
3.04695489446798	0.481894709881565\\
2.92167355656706	0.248465993719188\\
2.7620789756982	-0.00835812346349713\\
2.56309271743613	-0.287419179604642\\
2.32090628856909	-0.585186685965562\\
2.03428782115633	-0.895217074015581\\
1.70589780178838	-1.20814309277256\\
1.34305318204809	-1.51251547659411\\
0.957358301848464	-1.79651289090998\\
0.563027461849464	-2.05004471948193\\
0.17440368490501	-2.2664497757554\\
-0.196376909748047	-2.44316293361382\\
-0.540783844466607	-2.58127675633976\\
-0.854143981342034	-2.68442154501285\\
-1.13505368197173	-2.75751618959368\\
-1.38443246395149	-2.80576439146061\\
-1.6045931848711	-2.83402015795313\\
-1.79851286989534	-2.84647927607051\\
-1.9693409729063	-2.8465932033257\\
-2.1201079802089	-2.8371073567179\\
-2.25357672915568	-2.8201544840852\\
-2.37218440879142	-2.79736219431427\\
-2.47803682426109	-2.76995408295331\\
-2.57292963265337	-2.73883618320411\\
-2.65838120293001	-2.70466698118174\\
-2.73566843032449	-2.66791222186143\\
-2.80586098660883	-2.62888680496751\\
-2.8698519182274	-2.58778620568981\\
-2.92838384723987	-2.54470959153192\\
-2.98207071911846	-2.49967641573398\\
-3.03141535246996	-2.45263787417719\\
-3.07682314525913	-2.40348426509281\\
-3.11861227649762	-2.35204900150335\\
-3.15702066482802	-2.29810979171628\\
-3.19220983321692	-2.24138731469277\\
-3.2242656938994	-2.18154156579224\\
-3.253196112476	-2.11816592758243\\
-3.27892493199109	-2.05077892738059\\
-3.3012819315108	-1.97881358102903\\
-3.31998795355617	-1.90160420286395\\
-3.33463415867616	-1.81837061019379\\
-3.34465406183439	-1.72819981365247\\
-3.34928670602053	-1.63002564346745\\
-3.34752911347362	-1.52260744991088\\
-3.33807619606171	-1.40451024639017\\
-3.31924694264404	-1.27409075461728\\
-3.28889755562535	-1.12949720141945\\
-3.24432634438246	-0.968695913179166\\
-3.18218323873547	-0.789545104072829\\
-3.09841088038382	-0.589945317049548\\
-2.98826611939397	-0.368104162226417\\
-2.84649908086911	-0.122953273533075\\
-2.66779180783707	0.145266116939086\\
-2.4475534338266	0.434295408854172\\
-2.18308925931526	0.739191127217544\\
-1.87496839310639	1.05200132842453\\
-1.52814377342763	1.36215315008865\\
-1.15220751368633	1.65775617088943\\
-0.760348434732937	1.9275974139927\\
-0.367169777737965	2.16314400211131\\
0.0138428598720332	2.3597740592165\\
0.37225558768193	2.51686213953497\\
0.701498660578425	2.63692954475243\\
0.998640491179626	2.72439733159422\\
1.26356280992866	2.78442955309\\
1.4979877098556	2.82211183597127\\
1.70463610269356	2.84199118101627\\
1.88661887654611	2.84789220388966\\
};
\addplot [color=mycolor1, forget plot]
  table[row sep=crcr]{%
2.02970341638092	2.83118406656601\\
2.18799151583129	2.82627070826438\\
2.32657319718337	2.81313005572053\\
2.4484361969554	2.79370724827128\\
2.55612775499061	2.76944789457504\\
2.65179070273404	2.74141210242071\\
2.73721348154227	2.71036551716844\\
2.81388206297824	2.67684947612192\\
2.88302799853583	2.6412338361512\\
2.94567029440224	2.60375608343852\\
3.0026506013816	2.56454984358704\\
3.05466205328004	2.52366528406634\\
3.10227241407705	2.48108331006773\\
3.14594225428358	2.43672495627237\\
3.18603880658966	2.39045697724693\\
3.22284602136534	2.34209432522433\\
3.2565711878713	2.29139995779901\\
3.28734832050702	2.23808222232966\\
3.31523833276256	2.18178990450667\\
3.34022582962197	2.12210489554508\\
3.36221213263002	2.05853232050682\\
3.38100389863481	1.99048788022667\\
3.39629639027895	1.91728210122766\\
3.40765009260943	1.83810118658649\\
3.41445894347145	1.75198426532106\\
3.41590797658147	1.65779713944551\\
3.41091773876497	1.55420328624735\\
3.39807261966851	1.43963415894118\\
3.37553062049951	1.31226317838362\\
3.34091387918683	1.16999188138802\\
3.29118392127937	1.01046337054864\\
3.22251556622729	0.831128393084588\\
3.13020222573058	0.629403191943986\\
3.00865666611958	0.402973082376185\\
2.85161512835336	0.150302156920767\\
2.65269571955597	-0.128613728236073\\
2.40646243195864	-0.431300484550089\\
2.11002654386438	-0.751892842889294\\
1.7649061071668	-1.08070904660359\\
1.37842750844447	-1.40486610785543\\
0.963726805587811	-1.71019729738016\\
0.537833723209402	-1.98401574313531\\
0.118366070380709	-2.21760513799863\\
-0.279767988968165	-2.40737349899909\\
-0.646366232116422	-2.55440928108333\\
-0.976246163067648	-2.66301463115588\\
-1.26835500452852	-2.73904412880406\\
-1.52442446289068	-2.78860444565441\\
-1.74771927502018	-2.81727691382763\\
-1.94211413490662	-2.82977819743403\\
-2.11151509006866	-2.8299004587552\\
-2.2595483536233	-2.82059402181255\\
-2.38942531946137	-2.80410331288318\\
-2.50391055697638	-2.78210802442187\\
-2.60534333966609	-2.75584821494047\\
-2.69568274109955	-2.72622676262518\\
-2.77655963538476	-2.69388956052156\\
-2.84932713683109	-2.65928653977993\\
-2.91510570900307	-2.62271721150604\\
-2.97482169203059	-2.5843641256496\\
-3.02923924618228	-2.54431705456848\\
-3.07898625410271	-2.50259008809168\\
-3.12457489493291	-2.45913327921625\\
-3.16641758628866	-2.4138400312764\\
-3.20483888407323	-2.36655106269949\\
-3.24008378529377	-2.31705550766154\\
-3.27232271764382	-2.26508949195961\\
-3.30165332831405	-2.2103323474067\\
-3.32809900115148	-2.1524004828699\\
-3.35160382843044	-2.09083880791846\\
-3.37202353024845	-2.02510950346722\\
-3.38911153809806	-1.95457785774221\\
-3.40249912689535	-1.87849485180717\\
-3.41166808383927	-1.79597622221363\\
-3.41591394977336	-1.70597791654738\\
-3.41429740138974	-1.60726831190391\\
-3.40558097926308	-1.49839849557875\\
-3.38814838024225	-1.37767365741986\\
-3.35990448915265	-1.24313175674567\\
-3.31815730919366	-1.09254088240715\\
-3.25948987861662	-0.923435062714006\\
-3.17964414953854	-0.733220371008295\\
-3.07346350523359	-0.519398117047068\\
-2.93497869715061	-0.279964343057266\\
-2.75776845829473	-0.0140400219106444\\
-2.53575536604124	0.277264088159463\\
-2.26454919752634	0.5898750285627\\
-1.94323913720542	0.916019429230686\\
-1.5761386446494	1.24425583883913\\
-1.17359697122873	1.56074624024008\\
-0.751049416477331	1.85170522131172\\
-0.326269606485461	2.10618551936316\\
0.0841269569430639	2.31799429981231\\
0.467444213254643	2.48601907070544\\
0.816043340955449	2.61316846626642\\
1.12696299533261	2.70471399004408\\
1.40071392257016	2.76676605369003\\
1.63993250907036	2.80523492133321\\
1.84828108699536	2.82529081769113\\
2.02970341638092	2.83118406656601\\
};
\addplot [color=mycolor1, forget plot]
  table[row sep=crcr]{%
2.19794348574178	2.83020791239487\\
2.35373363816799	2.82538097677279\\
2.48874200304993	2.81258609601171\\
2.60638535099903	2.79384130177606\\
2.70950805314651	2.77061553326634\\
2.80045446322886	2.74396555105998\\
2.88114703858336	2.71464105191456\\
2.9531589992383	2.68316310588763\\
3.01777725847028	2.64988163896509\\
3.07605480264875	2.61501701171813\\
3.1288532349893	2.57868972809118\\
3.17687675042674	2.54094132969035\\
3.22069888491605	2.50174871427288\\
3.26078324797071	2.46103347785424\\
3.29749923425388	2.41866739360158\\
3.33113347916659	2.37447477349825\\
3.36189759804843	2.32823217865962\\
3.38993253319128	2.27966572467875\\
3.41530962187418	2.22844604886074\\
3.4380282810242	2.17418085132003\\
3.45800996505544	2.11640478115467\\
3.47508777476958	2.0545663069574\\
3.48899075557473	1.98801108946729\\
3.49932149906526	1.91596127624999\\
3.50552513068025	1.83749009698622\\
3.50684711571555	1.75149122253605\\
3.50227656741248	1.65664269587682\\
3.49047099763168	1.55136609748902\\
3.46965800078646	1.43378341593225\\
3.437509876863	1.30167761211687\\
3.39099009386595	1.15246929499121\\
3.32617849345752	0.983232960249207\\
3.23809999368771	0.790793683555821\\
3.1206161475555	0.571969256244199\\
2.96649683732306	0.324047654852989\\
2.76786675583152	0.0455943566014359\\
2.51727988560229	-0.262379028283973\\
2.2096098960078	-0.595051692367061\\
1.84459494234405	-0.942756585655447\\
1.42919527126162	-1.29111794491241\\
0.978307570666708	-1.62306038301653\\
0.512678782781185	-1.92241778446513\\
0.0544287312810153	-2.17761606270126\\
-0.377737554718123	-2.38363051745455\\
-0.771451366904411	-2.54157158742407\\
-1.12104353606049	-2.65669577496405\\
-1.42614017964794	-2.73613114161499\\
-1.68970031090223	-2.78716214054464\\
-1.91631039667646	-2.81627692693882\\
-2.11101924854228	-2.82881137685641\\
-2.27867372333326	-2.82894245377662\\
-2.42360785275526	-2.81983869614429\\
-2.54954389456128	-2.80385452952917\\
-2.65960446621492	-2.78271420765073\\
-2.75637414654071	-2.75766555114021\\
-2.84197675809744	-2.72960045336035\\
-2.91815161790832	-2.69914593213989\\
-2.98632159096612	-2.66673140099171\\
-3.04765072082669	-2.63263763015077\\
-3.10309155195078	-2.59703195032721\\
-3.15342321866503	-2.5599932276488\\
-3.19928164481707	-2.52152923413244\\
-3.24118314601982	-2.48158831164077\\
-3.27954254164904	-2.44006666857769\\
-3.31468665714298	-2.39681222612689\\
-3.34686386761021	-2.35162561087716\\
-3.37625011359407	-2.30425864380888\\
-3.40295160773395	-2.25441047833786\\
-3.42700423840378	-2.20172137452279\\
-3.44836945014586	-2.14576394973124\\
-3.46692612433319	-2.08603160999408\\
-3.48245767719466	-2.02192373879323\\
-3.49463321342869	-1.95272710773176\\
-3.50298109834676	-1.87759289802314\\
-3.50685272079441	-1.79550873151772\\
-3.50537351343399	-1.70526530223622\\
-3.49737752988289	-1.605417758054\\
-3.48132122789405	-1.49424324164873\\
-3.45517202082913	-1.3696985434041\\
-3.41626862149497	-1.22938661008532\\
-3.36115520018392	-1.07054914816206\\
-3.28540356604819	-0.890116606399041\\
-3.18346285077177	-0.684867785709255\\
-3.04862189585315	-0.451777235719651\\
-2.87323907188057	-0.188647424477684\\
-2.64947050202814	0.104898760550868\\
-2.37074235958796	0.426112149877728\\
-2.03402508173049	0.76782684360933\\
-1.64243859568701	1.11789622291151\\
-1.20698322528366	1.46022099380214\\
-0.745924559696358	1.77767795717415\\
-0.281335260344847	2.0560103396237\\
0.165852337997601	2.28682675361663\\
0.579901420918831	2.46835078353251\\
0.951887237654244	2.60406024107767\\
1.27902455765975	2.70040825839983\\
1.56284739585687	2.76476655539025\\
1.80730949491724	2.80409733637165\\
2.01733982975625	2.8243298222766\\
2.19794348574178	2.83020791239487\\
};
\addplot [color=mycolor1, forget plot]
  table[row sep=crcr]{%
2.39842872978907	2.84977139256201\\
2.55129038431519	2.84504461674373\\
2.68231539555117	2.83263442291828\\
2.79539001740285	2.81462313009287\\
2.89366946536777	2.79249252018157\\
2.97969935245457	2.76728667816391\\
3.05552932001567	2.73973209652949\\
3.12281051005268	2.71032437359527\\
3.18287548552932	2.67938999343328\\
3.23680211219273	2.64712995242337\\
3.28546384805857	2.61365029122884\\
3.3295689325236	2.57898318346077\\
3.36969066335952	2.54310115785129\\
3.40629054629358	2.50592623984897\\
3.43973570423691	2.46733522432841\\
3.47031157804237	2.42716187302171\\
3.49823064269921	2.3851965200136\\
3.523637592907	2.34118332956168\\
3.5466112048444	2.29481525526136\\
3.56716283849509	2.2457265776743\\
3.58523128640436	2.19348273351238\\
3.60067337611301	2.13756698200104\\
3.61324936578264	2.07736327643932\\
3.62260170039184	2.01213452177731\\
3.62822507741097	1.94099521647983\\
3.62942496089109	1.86287734047253\\
3.62526064607836	1.77648835582449\\
3.61446772309498	1.68026053263946\\
3.59535345920811	1.57229189592498\\
3.56565767806044	1.45028166408724\\
3.5223723437118	1.31146848252927\\
3.46151792285882	1.15259035574255\\
3.37788913343578	0.969904397791377\\
3.26481670291628	0.759336217353002\\
3.11405964046072	0.51687337923087\\
2.91605591056187	0.239360678696166\\
2.66089845199008	-0.074154951409486\\
2.34045290561423	-0.420557366703982\\
1.95172262498189	-0.790768853476641\\
1.50061335174333	-1.16900621117368\\
1.00396182899994	-1.53459576559659\\
0.487507470558486	-1.86661795835332\\
-0.0202704812606559	-2.14941502129083\\
-0.495328724565367	-2.37591010083617\\
-0.922461096163118	-2.54729676726565\\
-1.29564998398071	-2.67022881755567\\
-1.61576083610479	-2.75360467736886\\
-1.88760925020119	-2.80626531035516\\
-2.11761850849402	-2.83583585301951\\
-2.31236767058961	-2.84838727307023\\
-2.47786376359186	-2.848527498087\\
-2.61927104117742	-2.83965348790494\\
-2.74088359664343	-2.82422436669264\\
-2.84620715031017	-2.80399881931045\\
-2.93807710626956	-2.78022231052697\\
-3.01877781024656	-2.75376742051602\\
-3.09014858164517	-2.72523609470473\\
-3.15367233792879	-2.69503292377865\\
-3.21054721822161	-2.6634171134397\\
-3.26174334772701	-2.63053902736006\\
-3.30804728020505	-2.59646561565443\\
-3.35009648208242	-2.56119780348917\\
-3.38840584828985	-2.52468198992928\\
-3.42338783163099	-2.4868171327911\\
-3.45536738907869	-2.44745840594938\\
-3.48459261736701	-2.40641805663432\\
-3.51124166291223	-2.3634638195184\\
-3.53542623466593	-2.3183150301698\\
-3.55719180613415	-2.27063639909317\\
-3.57651434509614	-2.22002924128097\\
-3.59329313427756	-2.16601979160752\\
-3.607338917041	-2.10804406434049\\
-3.61835618706849	-2.04542853193148\\
-3.62591790130348	-1.97736571032573\\
-3.62943018671176	-1.90288357081217\\
-3.62808369083953	-1.82080761576549\\
-3.62078707436777	-1.72971459653246\\
-3.60607682103516	-1.6278774991293\\
-3.58199631503049	-1.51320312714434\\
-3.5459367630778	-1.38316739339852\\
-3.49443481646271	-1.2347610883137\\
-3.42293053187514	-1.0644733062168\\
-3.32551203519009	-0.868364747381815\\
-3.19472224053972	-0.642321746624063\\
-3.02159306783709	-0.38262930064068\\
-2.79620535592164	-0.0870280333253533\\
-2.50919050985094	0.243654883901909\\
-2.1545017961401	0.603522232975445\\
-1.73316922104735	0.980104720301319\\
-1.25652350100294	1.35475173726096\\
-0.746404409624323	1.70596083943897\\
-0.230855365319286	2.01482665889464\\
0.263069460758329	2.26979350025063\\
0.71545763148392	2.46816383184406\\
1.1158702518671	2.61428313674225\\
1.46209994357525	2.71628873977033\\
1.75733287011997	2.78326240843799\\
2.00742483932292	2.82352076593156\\
2.21900696162006	2.84391929265026\\
2.39842872978907	2.84977139256201\\
};
\addplot [color=mycolor1, forget plot]
  table[row sep=crcr]{%
2.64097641089213	2.89659444528762\\
2.79041062611745	2.8919835091847\\
2.91701342264999	2.8799994826167\\
3.02517161377208	2.86277682362462\\
3.11835537351822	2.84179789268256\\
3.1993034241174	2.8180842826645\\
3.27017992279921	2.79233229640008\\
3.33270056385315	2.76500748982383\\
3.38823118982876	2.73641014297704\\
3.43786378858543	2.7067203746822\\
3.48247461290933	2.67602905414433\\
3.52276843328127	2.64435875960573\\
3.55931210575014	2.61167768001073\\
3.59255988695773	2.57790840977447\\
3.62287230601135	2.54293292803773\\
3.65052990345537	2.50659459017085\\
3.67574274773775	2.46869762432826\\
3.69865631155523	2.42900437234479\\
3.71935400691495	2.3872303083034\\
3.73785641225198	2.34303668445357\\
3.75411695064979	2.29602047291274\\
3.76801346544322	2.24570107596631\\
3.77933475253769	2.19150305315141\\
3.78776060210338	2.13273384775694\\
3.79283321636605	2.06855518320083\\
3.79391692835111	1.99794645060503\\
3.79014185654502	1.9196580684176\\
3.78032540163229	1.83215258867449\\
3.76286329027734	1.73353154789707\\
3.73557937374125	1.62144735702884\\
3.69552135443048	1.4930032209677\\
3.63869022202073	1.34465276673472\\
3.5596997724219	1.1721294490384\\
3.45139077633049	0.970471418888199\\
3.3044941123845	0.73426928073975\\
3.10758058775252	0.458352527766974\\
2.84777069524612	0.139206421099586\\
2.51291985474115	-0.222666080392999\\
2.09587767575504	-0.619732372235386\\
1.60026445800111	-1.03518853022456\\
1.04485624976115	-1.44396712379425\\
0.462207844468628	-1.81852813787821\\
-0.109945569927732	-2.13720202313305\\
-0.639837694079034	-2.38988812122926\\
-1.1085225069396	-2.57800121726277\\
-1.51002341235624	-2.71030781968856\\
-1.84740273810277	-2.79822029892015\\
-2.12828543924942	-2.852660444426\\
-2.3616410819168	-2.88268286233461\\
-2.55602732535301	-2.89522665708\\
-2.71886286896576	-2.89537612907735\\
-2.85626797288029	-2.88676174916544\\
-2.9731620129911	-2.87193760572698\\
-3.07344897627211	-2.85268406443714\\
-3.16021175941515	-2.83023305489948\\
-3.23588415401786	-2.80542946599412\\
-3.3023921495358	-2.77884446518797\\
-3.3612657187078	-2.75085422974431\\
-3.41372549818282	-2.72169432633904\\
-3.46074929415268	-2.69149708850984\\
-3.50312281966151	-2.66031711633317\\
-3.54147825685777	-2.62814841171454\\
-3.57632343641523	-2.59493552952458\\
-3.60806373863459	-2.56058033584831\\
-3.63701826208773	-2.52494541321332\\
-3.66343135939488	-2.48785476032272\\
-3.687480279284	-2.44909214452755\\
-3.70927935164872	-2.40839723914093\\
-3.72888088108065	-2.36545948564737\\
-3.74627264792341	-2.31990944066658\\
-3.7613716263694	-2.27130718105428\\
-3.77401318424574	-2.21912713209767\\
-3.7839345887977	-2.16273843990084\\
-3.79075105493889	-2.10137972025599\\
-3.79392176922668	-2.03412668286445\\
-3.79270221906001	-1.95985077568743\\
-3.78607765707419	-1.87716669740717\\
-3.77267056209203	-1.78436658148074\\
-3.75061256297117	-1.67933930236004\\
-3.71736886968593	-1.55947562288976\\
-3.66950212780627	-1.42156566010487\\
-3.60236624409406	-1.26170792856632\\
-3.50973711671127	-1.07527511122444\\
-3.38343266281118	-0.857029400844818\\
-3.21307688276914	-0.601556618540798\\
-2.98635308825682	-0.304279960318684\\
-2.69035381736796	0.0366596101033578\\
-2.31476776430595	0.417622393762846\\
-1.85709586746691	0.826579657652298\\
-1.32825248141111	1.24217289287799\\
-0.754563161249114	1.63710924267664\\
-0.172579790492442	1.98578138140815\\
0.381690366566272	2.27193718950993\\
0.882472466847537	2.49158054055026\\
1.31763497668127	2.65043265612539\\
1.68631313491539	2.75909577113312\\
1.9943501924068	2.82900808155476\\
2.25034735327705	2.87024233480355\\
2.46321131062424	2.89078292609621\\
2.64097641089213	2.89659444528762\\
};
\addplot [color=mycolor1, forget plot]
  table[row sep=crcr]{%
2.93937327811104	2.98020791618488\\
3.0848469729724	2.97572929737168\\
3.20660057920926	2.96421158105103\\
3.30953681559854	2.94782581088244\\
3.39743205735299	2.92804154463718\\
3.47320078612098	2.90584830807318\\
3.53910295079154	2.88190610921701\\
3.59690170272777	2.85664696662268\\
3.64798167114937	2.83034322980455\\
3.69343713544351	2.80315351614966\\
3.73413767026049	2.77515352399524\\
3.77077704972821	2.74635653760725\\
3.80390969856187	2.71672679942007\\
3.83397781176107	2.68618783037745\\
3.8613313854702	2.65462704500914\\
3.88624274436105	2.62189750703973\\
3.9089166545751	2.58781731858184\\
3.92949672483139	2.5521668737934\\
3.94806848002368	2.51468399639427\\
3.9646592060098	2.47505679034342\\
3.97923437833746	2.43291384066253\\
3.99169016663867	2.38781118544366\\
4.00184110903122	2.33921521879601\\
4.00940152307047	2.28648035436313\\
4.01395848707556	2.22881985483185\\
4.0149331832171	2.16526769114884\\
4.01152589598762	2.09462862606547\\
4.00263781427355	2.01541295408357\\
3.98675976919101	1.92575162362794\\
3.96181398478126	1.82328724568355\\
3.92492998739264	1.70503781680153\\
3.87213126655526	1.56723527884344\\
3.79790934549623	1.40515550036124\\
3.69467810189503	1.21299017713548\\
3.55216050847545	0.983882265427233\\
3.35691612825008	0.71037699308149\\
3.09254740660601	0.385728441595871\\
2.74163772771271	0.00662483549291108\\
2.29080897278802	-0.422469058064585\\
1.73923812242842	-0.884699757276557\\
1.10716516270388	-1.34981282010608\\
0.436573322336864	-1.78088447824577\\
-0.220890687156659	-2.14710992718977\\
-0.821943562709433	-2.43379883759398\\
-1.34272331037627	-2.64289484339832\\
-1.7782116219387	-2.78646452955348\\
-2.13529852485433	-2.87956050057985\\
-2.42584722998334	-2.93590877128599\\
-2.6623291416762	-2.96635774621973\\
-2.85582398510105	-2.97886103559161\\
-3.01543395442862	-2.97901953234204\\
-3.14835247008457	-2.97069499888843\\
-3.26016183347867	-2.95652191845138\\
-3.35516490155065	-2.93828744324002\\
-3.43667771452072	-2.9171984584452\\
-3.507264902453	-2.89406441611592\\
-3.56892100214669	-2.86942101856488\\
-3.6232072007064	-2.84361351437841\\
-3.67135349433301	-2.81685272698547\\
-3.71433476709828	-2.78925269730199\\
-3.75292744395917	-2.7608558587592\\
-3.78775171277993	-2.73164965769039\\
-3.81930298044209	-2.70157719211839\\
-3.84797521331808	-2.67054354674246\\
-3.87407805236539	-2.63841889729279\\
-3.89784902335864	-2.60503903927504\\
-3.9194617271411	-2.57020369456018\\
-3.93903054716445	-2.53367271663447\\
-3.95661211382123	-2.49516011774674\\
-3.9722034834123	-2.45432565273035\\
-3.98573669073324	-2.41076349279229\\
-3.99706898085955	-2.36398728674946\\
-4.00596757094343	-2.31341061456523\\
-4.01208717353537	-2.25832146405978\\
-4.01493764077979	-2.19784888121129\\
-4.01383783984167	-2.13091933860427\\
-4.00785007513615	-2.05619964169997\\
-3.99568682003507	-1.97202242746716\\
-3.97557799561631	-1.87628976447428\\
-3.94508247899672	-1.76635072010641\\
-3.90082250269903	-1.63885163641798\\
-3.83811651857233	-1.48956683168124\\
-3.75049206492978	-1.31323991930616\\
-3.62909313531356	-1.10351573998024\\
-3.46209560616133	-0.853141210902829\\
-3.2344777859539	-0.554776905634855\\
-2.92892665923736	-0.202946176203328\\
-2.52917471338455	0.202394248066321\\
-2.02691275660062	0.651054593927904\\
-1.43106859787255	1.11918464768549\\
-0.773466768821263	1.57182874168429\\
-0.103109774127866	1.97345457244795\\
0.530458453168599	2.30060631642212\\
1.0930536622269	2.54743514594387\\
1.57088555125187	2.7219330138421\\
1.96586550132787	2.8384039978474\\
2.2880922624006	2.91157789220269\\
2.55010990667103	2.95381098597267\\
2.76383367615874	2.97445489318156\\
2.93937327811104	2.98020791618488\\
};
\addplot [color=mycolor1, forget plot]
  table[row sep=crcr]{%
3.31310804065758	3.11423285912486\\
3.45413337104492	3.10990129700141\\
3.57070011418171	3.09888133309348\\
3.66822059229077	3.08336274896067\\
3.75075389077549	3.06478914220523\\
3.82136389596575	3.04410976075739\\
3.88238263370365	3.02194389134628\\
3.93560120648659	2.99868810193405\\
3.98240768763913	2.97458640468319\\
4.02388686053869	2.94977631080212\\
4.06089266641142	2.92431908538716\\
4.09410109529664	2.8982195133334\\
4.12404896518883	2.87143857341009\\
4.15116240252091	2.84390119084381\\
4.17577768368916	2.81550044301811\\
4.19815627842688	2.78609906534202\\
4.21849534453176	2.75552874147828\\
4.23693448191354	2.72358739709812\\
4.25355920468701	2.69003450423134\\
4.26840128824111	2.65458421163407\\
4.2814358549939	2.61689591897394\\
4.29257473888013	2.57656168398216\\
4.30165526881013	2.53308956451831\\
4.30842307490455	2.48588161816088\\
4.31250676187696	2.43420476743595\\
4.31338118330189	2.37715203405867\\
4.31031439599432	2.31359068433281\\
4.30229088841038	2.24209254415909\\
4.28789994881676	2.16084011080941\\
4.2651725267102	2.06750024045705\\
4.23134208942419	1.95905570936229\\
4.18249472269963	1.83158559607804\\
4.11306313531554	1.67999283802569\\
4.01511709631591	1.49770258622065\\
3.87743618320911	1.27642243790266\\
3.68448903879646	1.00621107253835\\
3.41583178233321	0.676404659142812\\
3.04726885144551	0.278377622413228\\
2.55625498552266	-0.188781270660875\\
1.93383218683997	-0.710200982381025\\
1.20012725724235	-1.24996491816525\\
0.410247293308147	-1.75767968050705\\
-0.362558776364174	-2.1882089706272\\
-1.05730575048697	-2.51968996339351\\
-1.64380701589454	-2.75527700362356\\
-2.11999550471781	-2.91234747622804\\
-2.49933220685415	-3.01130348963465\\
-2.80001154620597	-3.06965670617956\\
-3.03924310506355	-3.10048655263343\\
-3.23124858898985	-3.11291154929754\\
-3.38708122729526	-3.11307847353079\\
-3.51509913996024	-3.10506929500743\\
-3.62156044024057	-3.09158012403946\\
-3.711149313784	-3.07438917578723\\
-3.78738893952191	-3.05466771209734\\
-3.8529493853219	-3.03318359710622\\
-3.90987206489963	-3.01043404356493\\
-3.95973201171652	-2.98673231744526\\
-4.00375510589819	-2.96226456418639\\
-4.04290302683582	-2.93712714354745\\
-4.07793511653314	-2.91135111566008\\
-4.10945364919197	-2.88491812560413\\
-4.13793706580435	-2.85777040301206\\
-4.1637643602983	-2.82981660720521\\
-4.18723283314462	-2.80093460271564\\
-4.20857073457294	-2.77097181506166\\
-4.22794581138425	-2.73974350941688\\
-4.245470382185	-2.70702910090965\\
-4.26120324543763	-2.67256640697519\\
-4.27514843193874	-2.63604356066886\\
-4.2872505100243	-2.59708809391366\\
-4.29738579648496	-2.55525244546419\\
-4.30534836692443	-2.50999481981321\\
-4.31082912432757	-2.46065388244215\\
-4.31338526900716	-2.40641517487225\\
-4.31239615973142	-2.34626630873118\\
-4.30699952897152	-2.27893688278491\\
-4.29599896864385	-2.20281760686899\\
-4.27772905629921	-2.11585134255531\\
-4.24985786814903	-2.01538698375836\\
-4.20909750788577	-1.89798634141617\\
-4.15078234346563	-1.75917735716389\\
-4.06826643508234	-1.59316129396303\\
-3.95210196693554	-1.39252369712639\\
-3.78903362871618	-1.14810314065818\\
-3.56108363900491	-0.84939444890248\\
-3.24559190290084	-0.486245395053664\\
-2.81813173129827	-0.0529767983154305\\
-2.26105422587032	0.444458656327441\\
-1.578189749031	0.980788387203884\\
-0.807703545356659	1.51104381240574\\
-0.0172819665077225	1.98461563751243\\
0.722354115713504	2.36662229951065\\
1.36473839717828	2.6485642939408\\
1.89506161310841	2.84232579337481\\
2.32065368498424	2.96789444026509\\
2.65836901131155	3.0446346460869\\
2.92634210140821	3.08786054169085\\
3.14038817811298	3.10855749810132\\
3.31310804065758	3.11423285912486\\
};
\addplot [color=mycolor1, forget plot]
  table[row sep=crcr]{%
3.78933340392954	3.31791824327719\\
3.9256218828151	3.31374193339205\\
4.03688754026066	3.30322976742082\\
4.12902223330395	3.28857288936556\\
4.2063313864434	3.27117830469308\\
4.27199608767686	3.25194973580832\\
4.32839497011113	3.23146399032309\\
4.37732801465156	3.21008240302836\\
4.42017287967763	3.18802181828246\\
4.45799495636751	3.16540012959222\\
4.49162555254062	3.1422656083244\\
4.52171793409172	3.1186157297649\\
4.54878780123903	3.09440904826234\\
4.57324266131504	3.06957233862602\\
4.59540313353684	3.04400438056539\\
4.61551824784775	3.01757721978827\\
4.63377612142125	2.9901353736854\\
4.65031090648788	2.96149318670523\\
4.6652065283807	2.93143033182382\\
4.67849741922023	2.89968526525472\\
4.69016615633992	2.86594624339662\\
4.70013759293096	2.82983927634141\\
4.70826867265188	2.79091208910161\\
4.71433258516435	2.74861274838639\\
4.71799515078679	2.70226103120105\\
4.71878017350119	2.6510097787119\\
4.71601874132723	2.59379227609881\\
4.70877471281688	2.52924996225507\\
4.69573432228688	2.45563229247299\\
4.67504105577271	2.37065713885981\\
4.64404636968432	2.27131567053681\\
4.59893080163012	2.15360087231286\\
4.53412752008188	2.01213665651605\\
4.4414546834988	1.83969417544177\\
4.30885398769578	1.62662912569426\\
4.11871132685225	1.36042186851634\\
3.84610274133577	1.0258819329934\\
3.4584010973348	0.607359779437747\\
2.91995060592895	0.0953039023489661\\
2.20759156509503	-0.501193725730767\\
1.33742359745292	-1.14114239634577\\
0.382662983474024	-1.75477769362568\\
-0.548910372727928	-2.27384341746798\\
-1.36839053415821	-2.66499197898418\\
-2.03794194892131	-2.93408438541059\\
-2.56253374214651	-3.10722854796537\\
-2.96667026485512	-3.21272556954055\\
-3.27780092330027	-3.27315255507966\\
-3.51936725246729	-3.30431196037223\\
-3.70937444249712	-3.31662595527043\\
-3.86105204278324	-3.31680038675033\\
-3.98397054689743	-3.30911826665799\\
-4.08504620548983	-3.29631702607449\\
-4.16930917359062	-3.28015199262873\\
-4.2404548367953	-3.26175110641758\\
-4.30122979639096	-3.24183736156445\\
-4.35370019480879	-3.22086886829122\\
-4.3994386079345	-3.1991277038773\\
-4.43965506957823	-3.17677672466477\\
-4.47528972515932	-3.15389610687864\\
-4.50707895446667	-3.13050686958774\\
-4.53560296016067	-3.10658587936471\\
-4.56132023664761	-3.08207514272936\\
-4.58459259871904	-3.05688713619861\\
-4.60570327346796	-3.03090725083102\\
-4.6248697490697	-3.0039939853017\\
-4.6422525006617	-2.97597721465299\\
-4.65796028890226	-2.94665463110278\\
-4.67205238859317	-2.91578625810213\\
-4.68453780508563	-2.8830867485093\\
-4.69537123272274	-2.84821496494332\\
-4.70444515773916	-2.81076007592616\\
-4.71157705238056	-2.77022304923311\\
-4.71648996968417	-2.72599193502021\\
-4.71878391210462	-2.67730863644428\\
-4.71789392855401	-2.6232238652005\\
-4.71302870163677	-2.5625355332369\\
-4.70307995437032	-2.49370375141885\\
-4.6864876004763	-2.41473266923938\\
-4.66103707452359	-2.32300543050104\\
-4.62355218403979	-2.21505373816029\\
-4.56942756657358	-2.08623940876509\\
-4.49191974942988	-1.93032717585299\\
-4.38109402030484	-1.73895152208022\\
-4.22234462466892	-1.50106645059843\\
-3.99459081237621	-1.20270969746124\\
-3.66891074591962	-0.827976910678309\\
-3.21003812908939	-0.363074319302198\\
-2.58576302165782	0.194107345369954\\
-1.78898086231441	0.819660721978775\\
-0.864096666789198	1.45603320419475\\
0.0926204481931285	2.029259224117\\
0.976282252524162	2.48578474467102\\
1.72232346382894	2.81337818018193\\
2.3170353021528	3.03079287264475\\
2.77789272839084	3.16685501765346\\
3.13226917842333	3.24743838858308\\
3.4060300034355	3.29163397229516\\
3.61988992172188	3.31233578974474\\
3.78933340392954	3.31791824327719\\
};
\addplot [color=mycolor1, forget plot]
  table[row sep=crcr]{%
4.40309459243992	3.61673537985275\\
4.53481677921992	3.61270800079255\\
4.64110043478882	3.60267248745111\\
4.72826995708325	3.58880955901003\\
4.80083637203317	3.57248498439558\\
4.86206764080882	3.55455676819874\\
4.91436804378816	3.53556130027745\\
4.95953263533969	3.51582758848991\\
4.99891992945788	3.49554826079772\\
5.03357064447642	3.47482417862802\\
5.06429044154428	3.45369265395223\\
5.09170828781157	3.43214527136778\\
5.11631805711692	3.410138959954\\
5.13850839996918	3.38760254418567\\
5.15858423577884	3.36444013421383\\
5.17678210917628	3.34053216635835\\
5.19328089884202	3.3157345415432\\
5.20820883754311	3.28987605209768\\
5.22164740765234	3.26275408446327\\
5.23363235533374	3.23412840086048\\
5.24415177023397	3.20371260744588\\
5.25314086087471	3.17116268062802\\
5.26047266817884	3.13606161193869\\
5.26594343298205	3.09789879743532\\
5.26925056740572	3.05604217020051\\
5.26996001559929	3.00970014813575\\
5.26745796636491	2.95786908053401\\
5.26087896394693	2.89925977346441\\
5.24899772008713	2.8321934711684\\
5.23006411986865	2.75445280529139\\
5.20154796627757	2.66306595096728\\
5.15973861448595	2.5539918319648\\
5.09911009413821	2.42166098728948\\
5.01131013473353	2.25831539476813\\
4.88356677732647	2.05310112519422\\
4.69627993517117	1.79096594601817\\
4.41979118172258	1.45178521240388\\
4.01147087775361	1.01120055205626\\
3.41775468130055	0.446881263255729\\
2.59202807503267	-0.244177181185543\\
1.53803684977493	-1.01899743473361\\
0.353058949636403	-1.780497840141\\
-0.79904577232246	-2.42258252442959\\
-1.78480006806738	-2.89332952841377\\
-2.55848306879809	-3.20447426270419\\
-3.13987999047351	-3.39650357026675\\
-3.57134976436999	-3.5092188842842\\
-3.89333155653455	-3.57180253210422\\
-4.13710369462435	-3.60327573004656\\
-4.32501861242621	-3.61547196053107\\
-4.47262257592968	-3.61565291919622\\
-4.59069228147971	-3.60828113370718\\
-4.68675783778689	-3.59611933784414\\
-4.76615035916567	-3.58089207619238\\
-4.83270185626321	-3.56368187291071\\
-4.88921007291105	-3.54516800989344\\
-4.93774878031748	-3.52577211107219\\
-4.97987706309158	-3.50574806491598\\
-5.01678227739226	-3.4852382414019\\
-5.04937901093761	-3.46430895694594\\
-5.07837847286748	-3.44297291939761\\
-5.10433771159524	-3.42120332425291\\
-5.12769484256259	-3.39894245124507\\
-5.14879439005106	-3.37610650483996\\
-5.16790548484653	-3.35258775419939\\
-5.18523474790841	-3.32825458453661\\
-5.20093506196988	-3.30294976941533\\
-5.21511097995922	-3.27648704866286\\
-5.22782116803274	-3.24864590680683\\
-5.23907797792926	-3.21916426047402\\
-5.24884394262677	-3.18772855137705\\
-5.2570246441864	-3.15396047294106\\
-5.26345695555513	-3.11739919286055\\
-5.26789102729119	-3.07747741355114\\
-5.26996344992716	-3.03348885180034\\
-5.26915757024225	-2.98454358648732\\
-5.26474463940321	-2.92950601583791\\
-5.25569575852446	-2.86690757081248\\
-5.24054850816678	-2.79482237995296\\
-5.21720209561772	-2.71068811292387\\
-5.1825981888099	-2.61104546079613\\
-5.13221728813398	-2.49115771374786\\
-5.05927746750375	-2.34445847477063\\
-4.95346176740859	-2.16177090388398\\
-4.79894273764576	-1.93028302696452\\
-4.57152676326354	-1.63246322477282\\
-4.23529796064144	-1.24574721441874\\
-3.7412462009014	-0.74544591316852\\
-3.03549773002775	-0.115886240823348\\
-2.08987122394223	0.626159587489338\\
-0.952219886815669	1.40870385076199\\
0.237291686538823	2.12144366640694\\
1.31764533683373	2.67979391491851\\
2.19789026299071	3.06654459157377\\
2.87070142789591	3.31267996601176\\
3.37164122166647	3.46068313324343\\
3.74384219945325	3.54538371592389\\
4.02341056241594	3.5905547199245\\
4.23693830634328	3.61124714534204\\
4.40309459243992	3.61673537985275\\
};
\addplot [color=mycolor1, forget plot]
  table[row sep=crcr]{%
5.18712583895256	4.03643886090506\\
5.31525725204707	4.03252906981332\\
5.41757620024171	4.02287294925828\\
5.50079469346955	4.00964175117464\\
5.5696003253642	3.99416555895309\\
5.62733170440595	3.97726378963646\\
5.67641122072661	3.95943940624426\\
5.71862685754273	3.94099514264964\\
5.75531895681641	3.92210424785202\\
5.78750618287741	3.90285415339594\\
5.81597190070615	3.88327366803006\\
5.84132430607785	3.86334991685115\\
5.8640388204713	3.84303872581938\\
5.88448826356221	3.82227067715433\\
5.90296441651425	3.80095417392495\\
5.91969336093515	3.77897630111696\\
5.9348461628453	3.75620191182305\\
5.94854590917154	3.73247111600344\\
5.96087169458886	3.70759515234812\\
5.97185983006285	3.68135044432882\\
5.98150224897495	3.65347044950186\\
5.98974177522711	3.6236346761894\\
5.99646353807814	3.59145392663468\\
6.00148130126427	3.55645037850838\\
6.00451671496219	3.51803045859987\\
6.00516833049201	3.4754474699895\\
6.00286535774335	3.42774940781586\\
5.99679810904926	3.37370501352264\\
5.98581200843584	3.31169733754317\\
5.96824343302717	3.23956802578995\\
5.94166076599006	3.1543857977506\\
5.90244799924475	3.05209698328066\\
5.84512255933808	2.92699173597258\\
5.76120034149497	2.77088497318124\\
5.63729389255025	2.57187402037809\\
5.45196389052451	2.31254430268633\\
5.170800179951	1.96774488397845\\
4.74005681044081	1.50317000806675\\
4.08343851340225	0.879405714822638\\
3.11873275843801	0.0725212336633277\\
1.82240158153292	-0.879979006481791\\
0.3208772432074	-1.84474570715855\\
-1.13257542859024	-2.65499655952802\\
-2.33491184292571	-3.22951294326548\\
-3.23551674549088	-3.5919659263711\\
-3.8819331864792	-3.80563197939625\\
-4.34333653284971	-3.92625706921822\\
-4.67714451079244	-3.99118894206398\\
-4.92384191511931	-4.02306768443975\\
-5.1104782906461	-4.03519717120396\\
-5.25494229682946	-4.03538413725133\\
-5.36916620850968	-4.02825870448253\\
-5.46124197072875	-4.01660612024233\\
-5.53676482188745	-4.00212386009999\\
-5.59968129721263	-3.98585566318484\\
-5.65282887921241	-3.96844430635513\\
-5.69828447180154	-3.95028150863634\\
-5.73759331009454	-3.93159842884888\\
-5.77192201193149	-3.91252116361816\\
-5.80216268002389	-3.89310518437901\\
-5.82900484455859	-3.8733568160797\\
-5.8529858815526	-3.85324654641236\\
-5.87452674587498	-3.83271703316872\\
-5.89395747634478	-3.81168753720132\\
-5.91153540716701	-3.79005581256549\\
-5.92745802288954	-3.7676980440536\\
-5.94187172094541	-3.74446712558521\\
-5.95487727028482	-3.72018935421026\\
-5.9665323937902	-3.69465943029842\\
-5.97685159715728	-3.66763347237637\\
-5.98580306939653	-3.63881954565161\\
-5.99330214181357	-3.60786493364055\\
-5.99920035424942	-3.57433900902988\\
-6.00326855541121	-3.53771001937137\\
-6.00517152686637	-3.49731329719506\\
-6.00443015115258	-3.45230717604644\\
-6.00036477491604	-3.40161098959502\\
-5.99200950644002	-3.34381653301228\\
-5.9779805973346	-3.27705958700445\\
-5.95627075586493	-3.19883042356599\\
-5.92392157032137	-3.10568985900584\\
-5.87649171664733	-2.99283785316287\\
-5.80717843048081	-2.85345220038719\\
-5.70534840422258	-2.67767671923539\\
-5.5540823371721	-2.45111316334733\\
-5.32619170660676	-2.15275947247519\\
-4.97839825424011	-1.75289683024479\\
-4.44544133267055	-1.21346613054153\\
-3.64360737117788	-0.498615041134729\\
-2.50836858312705	0.391708787828822\\
-1.08269748007528	1.37202389763564\\
0.427899320710524	2.27720604166756\\
1.77169676503285	2.97202628862457\\
2.82116844859093	3.43344452188135\\
3.58608777976602	3.71348668836244\\
4.131761062643	3.87482760873836\\
4.52329054454194	3.96399345419352\\
4.80943067044692	4.01026330629437\\
5.02337853892604	4.03101761132268\\
5.18712583895256	4.03643886090506\\
};
\addplot [color=mycolor1, forget plot]
  table[row sep=crcr]{%
6.12157382844716	4.57111927024128\\
6.24806332456751	4.56726569668653\\
6.34824649775386	4.55781500425311\\
6.4291970916542	4.54494694268638\\
6.49577467523898	4.52997365270056\\
6.55139508238138	4.51369115448788\\
6.59851042630908	4.49658102796028\\
6.63891485485238	4.47892879176142\\
6.67394362166934	4.46089480869968\\
6.70460542067214	4.44255745871458\\
6.7316720572432	4.42393970758168\\
6.75574025693286	4.40502547929216\\
6.77727488947267	4.38576958944698\\
6.79663952794232	4.36610347045196\\
6.814118176515	4.34593801528116\\
6.82993067301352	4.32516431250666\\
6.84424340439968	4.30365268873562\\
6.85717638380716	4.28125022700269\\
6.86880731417405	4.25777673640515\\
6.87917293117044	4.23301897186331\\
6.88826762162818	4.20672271277163\\
6.8960390043297	4.17858207435375\\
6.90237978446501	4.14822510695287\\
6.90711468046038	4.1151942809265\\
6.90998046502719	4.07891977302142\\
6.91059598638067	4.03868242701766\\
6.90841714258277	3.99356163051609\\
6.90266864778835	3.94236074878176\\
6.89223910917539	3.8834985298116\\
6.87551668756016	3.81484790313155\\
6.85012617833902	3.73349184818549\\
6.81249851632434	3.63534605823882\\
6.75714866376768	3.51456422659319\\
6.67543564422854	3.36258547306192\\
6.55339203348151	3.16659726668628\\
6.3678959605814	2.9070891579332\\
6.08006719213609	2.55421584701719\\
5.62497593235754	2.06357009602314\\
4.90087775492559	1.37605332592143\\
3.77905242794992	0.438318091531441\\
2.18933297264066	-0.729136310160695\\
0.287381237730008	-1.95096832528054\\
-1.5447403507541	-2.97262224633514\\
-3.00610436096118	-3.6713485534524\\
-4.04944884188627	-4.09155331359193\\
-4.76576569968722	-4.32849181891376\\
-5.25912890942235	-4.45755806731079\\
-5.60648327241671	-4.52516921820955\\
-5.85799505250904	-4.55769374925049\\
-6.0453526545804	-4.56988338160137\\
-6.18866536129684	-4.57007668099744\\
-6.30093745826316	-4.56307784238045\\
-6.39078107110039	-4.5517108744142\\
-6.46404187547533	-4.53766449240715\\
-6.52478291269304	-4.52196027003734\\
-6.57589106657402	-4.50521810715663\\
-6.6194591016604	-4.48781032519455\\
-6.65703171446965	-4.46995307151583\\
-6.68976743548591	-4.45176155084663\\
-6.71854731555456	-4.43328386418837\\
-6.74404922583356	-4.41452187854762\\
-6.76679946214588	-4.39544402454388\\
-6.78720904944798	-4.37599291397698\\
-6.80559950258139	-4.35608949909373\\
-6.82222114061885	-4.33563479105379\\
-6.83726598293421	-4.31450971391857\\
-6.85087654374139	-4.29257337685617\\
-6.86315134649674	-4.26965983229243\\
-6.87414760938744	-4.24557320696391\\
-6.88388124494256	-4.22008091326308\\
-6.89232401999751	-4.19290443992443\\
-6.89939738720861	-4.16370694997184\\
-6.90496206581973	-4.13207653417616\\
-6.90880183133517	-4.09750341183482\\
-6.91059903522101	-4.05934852983483\\
-6.90989788950252	-4.01679970999594\\
-6.90604912324757	-3.96880944003592\\
-6.89812554682906	-3.91400509544899\\
-6.88479106675031	-3.85055695492978\\
-6.86409340019097	-3.77598032518114\\
-6.8331286479579	-3.6868327962381\\
-6.78748542624817	-3.5782416184092\\
-6.72030123773543	-3.44315227413553\\
-6.62062508317728	-3.27111847402897\\
-6.47053413497078	-3.04635546174779\\
-6.24007622704454	-2.74471365640521\\
-5.87884444936142	-2.32953799591968\\
-5.3044219589401	-1.7483980287049\\
-4.3972539283441	-0.940097828098037\\
-3.04005847943889	0.123663394463265\\
-1.25604321060693	1.34990678306991\\
0.662081669560649	2.49936265935053\\
2.33001000851835	3.36220391793696\\
3.57564188357901	3.91025083801778\\
4.44149212444558	4.22747445667198\\
5.03479210309462	4.4030175728086\\
5.44739037766176	4.49704300301994\\
5.74189957370979	4.5446983443192\\
5.95822788276908	4.56570118070013\\
6.12157382844716	4.57111927024128\\
};
\addplot [color=mycolor1, forget plot]
  table[row sep=crcr]{%
6.9909859082623	5.08978129550532\\
7.11790148972651	5.08591861692123\\
7.21790506732202	5.07648728423415\\
7.29838163561349	5.06369615375623\\
7.36435274701211	5.04886033454228\\
7.41931940096902	5.03276997863885\\
7.46577835194663	5.01589878061026\\
7.50554660831771	4.99852490148967\\
7.53997044729718	4.98080268271892\\
7.57006314704963	4.96280594493406\\
7.59659764406972	4.94455443903197\\
7.62017001937801	4.92603003784696\\
7.6412436766496	4.90718649633422\\
7.66018045043824	4.88795503534393\\
7.6772626550723	4.86824708120501\\
7.69270868132481	4.84795493222628\\
7.70668383791504	4.8269507645884\\
7.71930752208815	4.80508414235582\\
7.73065736702177	4.78217800344967\\
7.74077067438982	4.75802291717117\\
7.74964314023736	4.73236921810162\\
7.75722457110038	4.70491638404746\\
7.76341091045485	4.67529870204613\\
7.7680313806868	4.64306579848476\\
7.77082878287473	4.60765590708681\\
7.77142980517594	4.56835866541107\\
7.7693002597077	4.52426252110813\\
7.76367694468062	4.47417907231269\\
7.75346230241894	4.41653212432005\\
7.73705831926242	4.34919159985885\\
7.71209854955049	4.26921930519493\\
7.67500460179675	4.17247056612995\\
7.62023168766171	4.05295499601583\\
7.53894862904738	3.90178743964219\\
7.41666688165914	3.70543622994144\\
7.22890136428397	3.44278820015857\\
6.93324989831278	3.08039162177707\\
6.4557914223663	2.56576625180694\\
5.67298969892936	1.82278227009393\\
4.41173338381147	0.768977579181146\\
2.54848954519755	-0.598763404959222\\
0.259362341041705	-2.06909945961282\\
-1.93680878161845	-3.29406765961223\\
-3.63733063802839	-4.10755022276248\\
-4.80690862843087	-4.57885393338226\\
-5.58394911087071	-4.8360087212446\\
-6.10581601941451	-4.97259431008842\\
-6.46649792202238	-5.04283035884027\\
-6.72414397385616	-5.07616400887524\\
-6.91415248836824	-5.08853476964782\\
-7.05839474350069	-5.08873432244207\\
-7.17073735575507	-5.08173413086652\\
-7.26022690093588	-5.07041390261868\\
-7.33293315759245	-5.0564751412908\\
-7.39303655546847	-5.04093667685731\\
-7.44348572540194	-5.02441103394238\\
-7.48640552596234	-5.00726273880557\\
-7.52335670437388	-4.98970120591377\\
-7.55550510552885	-4.97183635536423\\
-7.58373436459845	-4.95371242221127\\
-7.60872244132046	-4.93532866348408\\
-7.63099448766155	-4.91665197383152\\
-7.65095987493623	-4.89762434431278\\
-7.6689383733506	-4.87816689879263\\
-7.6851787145651	-4.85818152645098\\
-7.69987164223171	-4.83755068372668\\
-7.71315881284425	-4.81613564451565\\
-7.72513839650761	-4.79377326270991\\
-7.73586784746538	-4.77027113072694\\
-7.74536400090288	-4.74540083794309\\
-7.75360035311974	-4.71888882329563\\
-7.76050104570531	-4.69040404176945\\
-7.76593063930396	-4.65954127749152\\
-7.76967814082003	-4.62579836507921\\
-7.77143279958508	-4.58854471138716\\
-7.77074767729164	-4.54697715285237\\
-7.76698451038783	-4.50005701858685\\
-7.75922917705121	-4.44641874011509\\
-7.74615977391762	-4.38423447145167\\
-7.72583628097031	-4.31100918742543\\
-7.69535695951676	-4.22326338726391\\
-7.65028192517443	-4.11602995471153\\
-7.58363868665366	-3.98203740160641\\
-7.48415836494337	-3.81035641884321\\
-7.33307293563603	-3.58413002197802\\
-7.09823734409178	-3.27680652861004\\
-6.7236017538666	-2.84632035718348\\
-6.11259151482577	-2.22835698922053\\
-5.11344545089394	-1.3384735771218\\
-3.55457909703552	-0.117214279941999\\
-1.4289302365554	1.34338348366825\\
0.884372753194159	2.72972904867204\\
2.85844942495496	3.75135840219081\\
4.28103103102118	4.37759901212331\\
5.23495821793396	4.72727792235517\\
5.86989868459124	4.91523188368736\\
6.30199845709865	5.01374509012517\\
6.60558442554296	5.06289109773317\\
6.82599817175736	5.08430221422049\\
6.9909859082623	5.08978129550532\\
};
\addplot [color=mycolor1, forget plot]
  table[row sep=crcr]{%
7.2832163016226	5.26738346659666\\
7.41061043289889	5.263507259095\\
7.51085362238329	5.25405397104058\\
7.59143597922978	5.24124644397224\\
7.65743670387548	5.22640424861722\\
7.71238936855266	5.21031818762825\\
7.75880959249844	5.19346119896369\\
7.79852552685244	5.17611028823901\\
7.8328901282851	5.15841865182427\\
7.86292071882693	5.14045912617052\\
7.88939269615076	5.12225067952556\\
7.91290364110264	5.10377460001331\\
7.93391787068537	5.0849842368701\\
7.95279777727258	5.06581056233232\\
7.96982602300386	5.0461648913656\\
7.98522123048198	5.02593953196896\\
7.99914888712475	5.00500677865477\\
8.01172855981302	4.98321641326176\\
8.02303807533367	4.96039168395227\\
8.03311497988752	4.93632355656157\\
8.04195528894996	4.91076284102306\\
8.04950922619908	4.88340955722362\\
8.0556732719991	4.85389857879987\\
8.06027732503006	4.82178012145223\\
8.06306501409945	4.78649293281249\\
8.06366399836447	4.74732694439775\\
8.06154114814309	4.70337041156567\\
8.05593424328879	4.65343376196269\\
8.0457462307116	4.59593773479463\\
8.02937820478327	4.5287455533124\\
8.00445936209182	4.4489053281845\\
7.9673988192036	4.35224501095417\\
7.91262045065983	4.23271945852377\\
7.83121765478356	4.08133226464549\\
7.70852088001885	3.88431981373448\\
7.51960327357625	3.62006967991628\\
7.22095771012417	3.25402150257812\\
6.73585491905443	2.73119467365785\\
5.9338037090026	1.97001890949518\\
4.62681297446452	0.878148203435068\\
2.67187007308957	-0.556720013437796\\
0.250486289929001	-2.11193355209638\\
-2.0696655194179	-3.40615562364339\\
-3.8499829292279	-4.25793875217054\\
-5.06091117825139	-4.74598363350573\\
-5.85782546953629	-5.00975354786056\\
-6.38925156712959	-5.14885876736711\\
-6.75466431715758	-5.22002459554351\\
-7.01472669261033	-5.25367520056542\\
-7.20599774179902	-5.26613050162035\\
-7.3509037063389	-5.26633231327831\\
-7.46358791204496	-5.25931164731821\\
-7.5532406990084	-5.24797128446114\\
-7.62600937502927	-5.23402089881646\\
-7.68611755561483	-5.21848143480156\\
-7.73653859168646	-5.20196517777155\\
-7.77941180915661	-5.18483562082599\\
-7.81630656912174	-5.16730099817604\\
-7.84839390877197	-5.14947015553103\\
-7.87656061685162	-5.13138644318063\\
-7.90148657445999	-5.11304843630537\\
-7.92369810403299	-5.09442253706724\\
-7.94360528994299	-5.07545041203258\\
-7.96152834084078	-5.05605300734567\\
-7.97771626876498	-5.03613216313046\\
-7.99236001617379	-5.01557040116862\\
-8.00560140905129	-4.99422916439238\\
-8.0175387954673	-4.97194557148382\\
-8.02822984560691	-4.9485275690411\\
-8.03769167373634	-4.92374718350929\\
-8.04589814162964	-4.89733136452143\\
-8.05277386521658	-4.86894963511716\\
-8.05818400956813	-4.83819737436594\\
-8.06191833293965	-4.80457298148844\\
-8.0636669871316	-4.76744629123877\\
-8.0629840602184	-4.72601423503403\\
-8.05923233959752	-4.67923754237978\\
-8.05149851972958	-4.62574867987064\\
-8.03846066931219	-4.56371521155516\\
-8.01817651567367	-4.49063248713586\\
-7.98773673884338	-4.40300162235347\\
-7.94268149303391	-4.2958168297695\\
-7.875989565814	-4.16172878067136\\
-7.77627537215769	-3.9896481243096\\
-7.62448891865551	-3.76237899002894\\
-7.38778859683814	-3.45262826362175\\
-7.00836393412335	-3.01666537601575\\
-6.38518511700294	-2.38644975042891\\
-5.35599446820208	-1.46991717648909\\
-3.7303363840847	-0.19650874428884\\
-1.48884909752624	1.34353339141959\\
0.959651428398393	2.81092659973965\\
3.03703543232128	3.88615305828664\\
4.51803085418353	4.53821345645394\\
5.50071654952238	4.89848960522985\\
6.14940394979527	5.09053888006187\\
6.58820082916082	5.19059117671614\\
6.89515570020845	5.24028859755106\\
7.11731260873144	5.2618722081855\\
7.2832163016226	5.26738346659666\\
};
\addplot [color=mycolor1, forget plot]
  table[row sep=crcr]{%
6.70347763616719	4.91651399485236\\
6.83007664955833	4.91265981889838\\
6.92998138220201	4.90323710271026\\
7.01047416996457	4.89044293379464\\
7.07652162707673	4.87558963231008\\
7.1315946956581	4.85946790418777\\
7.1781734424047	4.84255304116001\\
7.21806555594753	4.82512492910375\\
7.2526121361442	4.80733942500051\\
7.28282364523882	4.78927155807402\\
7.30947155849757	4.77094197770443\\
7.33315126942091	4.75233317462754\\
7.35432592629523	4.73339927747518\\
7.37335733591397	4.71407166956962\\
7.39052788745482	4.69426175477585\\
7.40605607042567	4.67386164292077\\
7.42010726413257	4.65274316775044\\
7.43280087100144	4.63075540303822\\
7.44421443399252	4.60772064972786\\
7.45438504147412	4.58342869094532\\
7.46330802425303	4.55762892140207\\
7.47093263923078	4.53001972152165\\
7.47715405822289	4.50023412488345\\
7.48180046698014	4.46782036326869\\
7.4846133193766	4.43221517819991\\
7.48521760641887	4.39270671784476\\
7.48307708251151	4.34838215388948\\
7.47742619858199	4.29805244526697\\
7.46716503316036	4.24014223168562\\
7.45069394100136	4.17252539686212\\
7.42564742943805	4.09227413291477\\
7.38845507139868	3.99526728048557\\
7.33359759524606	3.87556506932507\\
7.25231233843863	3.72439006851256\\
7.13028518270596	3.52844199409026\\
6.94347119049367	3.2671147221298\\
6.65058787595554	2.90809150816575\\
6.18057119284037	2.40144748284787\\
5.41690092925881	1.67654181760295\\
4.20115170498602	0.660616150378107\\
2.42827484707059	-0.640964198887718\\
0.268347616781196	-2.02837983043246\\
-1.80653486008804	-3.18560397589839\\
-3.42824772437214	-3.96126484223808\\
-4.55665807253282	-4.41590235047736\\
-5.31392618188489	-4.66647502121664\\
-5.82640166433818	-4.80058418077178\\
-6.18255946006111	-4.8699302108425\\
-6.43799920851102	-4.90297376549035\\
-6.62693996389362	-4.91527247467651\\
-6.77069166466908	-4.91546988765032\\
-6.88284379882309	-4.90848067661172\\
-6.9723011509031	-4.89716395384352\\
-7.04505863936063	-4.88321499271203\\
-7.10525615208148	-4.86765193458418\\
-7.15581995336506	-4.85108855312973\\
-7.19886243218019	-4.83389110241934\\
-7.23593737895186	-4.81627063938727\\
-7.2682068052951	-4.79833845040109\\
-7.29655230128934	-4.78013982151605\\
-7.3216508068323	-4.7616747634758\\
-7.344027034824	-4.74291066225017\\
-7.36409023460512	-4.72378977323213\\
-7.38216021207903	-4.70423328747032\\
-7.398485793522	-4.6841429867534\\
-7.41325781261155	-4.66340106094978\\
-7.4266179677889	-4.64186836716448\\
-7.43866439015068	-4.6193811957718\\
-7.44945438564445	-4.59574642809872\\
-7.45900450396759	-4.57073479109756\\
-7.46728778834022	-4.54407170536698\\
-7.47422772475973	-4.51542494970146\\
-7.47968797536858	-4.48438798100889\\
-7.4834563607787	-4.4504571823829\\
-7.48522061203805	-4.41300045211077\\
-7.48453191140727	-4.37121320755525\\
-7.48074977535588	-4.32405574785452\\
-7.47295767181221	-4.27016245942286\\
-7.45983155827868	-4.20770761137826\\
-7.43943073289554	-4.13420278451995\\
-7.40885710156391	-4.04618426431852\\
-7.36368557622261	-3.9387195459219\\
-7.29698595053399	-3.80461097266097\\
-7.19759899571727	-3.63308679491217\\
-7.04703364595124	-3.40763157658516\\
-6.81384596300016	-3.10245055255353\\
-6.44377654815097	-2.67718339857415\\
-5.84477135816693	-2.07130501058155\\
-4.87556119954063	-1.20797396692885\\
-3.38288024296397	-0.0383960540040172\\
-1.37078772986577	1.3443138563634\\
0.810538102279413	2.65154390175893\\
2.68311428953206	3.62052057186414\\
4.04772863334151	4.22114173444051\\
4.97299565420324	4.56025948872589\\
5.59432313215858	4.74415705549701\\
6.01991840236974	4.84117447389263\\
6.32034721556549	4.88980295022609\\
6.53922157908623	4.91106112284334\\
6.70347763616719	4.91651399485236\\
};
\addplot [color=mycolor1, forget plot]
  table[row sep=crcr]{%
5.6711334783885	4.30975043106863\\
5.7981024301589	4.30587961160684\\
5.89901983552652	4.29635798587537\\
5.98079215866699	4.28335819729862\\
6.04819759579739	4.26819796487796\\
6.1046137173402	4.25168198630243\\
6.15247622562267	4.23430012596883\\
6.19357391517587	4.21634470956528\\
6.22924216395473	4.1979812635934\\
6.26049234894254	4.17929183890547\\
6.28809998345381	4.16030180769172\\
6.31266572321334	4.14099645246868\\
6.33465817685667	4.12133107574606\\
6.35444425796681	4.10123685616594\\
6.37231081268117	4.08062378112781\\
6.38847997395563	4.05938143380288\\
6.40311984915975	4.03737805543945\\
6.41635157079522	4.01445805505438\\
6.42825332307538	3.99043794404617\\
6.43886162772311	3.96510049580183\\
6.44816987671727	3.93818673968355\\
6.45612378990955	3.9093851641175\\
6.46261309868026	3.87831718679015\\
6.4674582429467	3.84451749726722\\
6.47039011205888	3.80740720632041\\
6.47101968873904	3.76625671562929\\
6.46879257848685	3.72013363666001\\
6.46292031781962	3.66782858083868\\
6.45227514912502	3.6077476111253\\
6.43522598444519	3.53775356181477\\
6.40937752556968	3.45492756017388\\
6.37114634207824	3.3552040438814\\
6.31505680508197	3.23280294416779\\
6.23254804360725	3.0793361534654\\
6.10992346565928	2.88240170329279\\
5.92482870183606	2.62343174964573\\
5.64041904469585	2.27470730679447\\
5.19691702437587	1.79647488054428\\
4.50458082029795	1.13896519617542\\
3.45742283844856	0.263403896699317\\
2.00941141207743	-0.800244905551356\\
0.303011623706775	-1.89654499842687\\
-1.34449683242487	-2.81512225579304\\
-2.68112629374548	-3.4540231187615\\
-3.65671878469359	-3.84681066198187\\
-4.34013422475085	-4.07279450366802\\
-4.81839601455809	-4.19787328446349\\
-5.15918042587159	-4.26418658504594\\
-5.4081503849663	-4.29637222498502\\
-5.5948639841194	-4.30851423607433\\
-5.7384171486743	-4.30870448503699\\
-5.85132502829035	-4.30166392516054\\
-5.94196078964505	-4.29019538445848\\
-6.01605318700399	-4.27598864975694\\
-6.07760908488791	-4.26007311282492\\
-6.12948986345481	-4.24307738928111\\
-6.17377834682678	-4.22538140368627\\
-6.21201710773861	-4.20720728139212\\
-6.24536633812097	-4.18867461752844\\
-6.27471041969153	-4.16983452283209\\
-6.30073111301828	-4.15069072193919\\
-6.32395858420191	-4.13121254900976\\
-6.34480741542836	-4.11134271879507\\
-6.36360222110575	-4.09100159519299\\
-6.38059589306671	-4.07008897948351\\
-6.39598246150051	-4.04848399971752\\
-6.40990586422851	-4.02604338814606\\
-6.42246543071431	-4.00259821730328\\
-6.43371852153588	-3.97794898337333\\
-6.44368045765436	-3.9518587450571\\
-6.4523215771323	-3.92404381761165\\
-6.4595609210172	-3.89416125156107\\
-6.46525561527158	-3.86179194902714\\
-6.46918439675131	-3.8264177215145\\
-6.47102279443057	-3.78738976774571\\
-6.47030599987355	-3.74388478137508\\
-6.46637306063012	-3.69484290960486\\
-6.45828203146807	-3.63887861096069\\
-6.44467890473077	-3.57415031720324\\
-6.42359128449403	-3.49816635320554\\
-6.39209674307987	-3.40749056189011\\
-6.34577796197681	-3.29728790107074\\
-6.2778083077063	-3.16061280682044\\
-6.17738955106327	-2.98728708213752\\
-6.02706024461078	-2.76214966984903\\
-5.79812536235465	-2.46246989105206\\
-5.44343817795874	-2.05475714247584\\
-4.88856704559779	-1.49328566815426\\
-4.03113780386764	-0.729107711971661\\
-2.78027450561968	0.251580606084027\\
-1.17052585641563	1.35824257218321\\
0.548410181789146	2.38830122955274\\
2.05915259483019	3.16964756291024\\
3.21103173494561	3.67628253450724\\
4.02923013302172	3.97594957924406\\
4.60011313937785	4.14480849271164\\
5.00266458119172	4.23651798591384\\
5.29299196057574	4.28348280355837\\
5.5079043851364	4.3043405849677\\
5.6711334783885	4.30975043106863\\
};
\addplot [color=mycolor1, forget plot]
  table[row sep=crcr]{%
4.67163867468412	3.75659873546539\\
4.80186866683076	3.75262005310516\\
4.90652567250872	3.74274013679767\\
4.99208151308878	3.72913519677954\\
5.06311456176333	3.71315652110489\\
5.12291958259539	3.69564659311496\\
5.17390741933042	3.67712836345684\\
5.21786995034725	3.65792026237507\\
5.2561582181402	3.63820710234224\\
5.28980398174832	3.61808431913829\\
5.31960387782812	3.59758578423176\\
5.34617847729379	3.57670127592357\\
5.3700141988918	3.55538727713922\\
5.39149329896913	3.53357332830043\\
5.41091539240697	3.51116528725187\\
5.4285128029399	3.48804629794914\\
5.44446126337847	3.46407590816154\\
5.45888694392339	3.43908752149046\\
5.47187038633885	3.41288416844244\\
5.48344759855503	3.3852323987665\\
5.49360826829224	3.35585390306671\\
5.50229073971236	3.32441423610317\\
5.50937301255037	3.29050770125132\\
5.51465850025144	3.25363701578832\\
5.51785452024647	3.21318573661633\\
5.51854032300997	3.16838047263847\\
5.51611962807314	3.11823846564039\\
5.50974966984763	3.06149390496678\\
5.49823388315572	2.99649290883732\\
5.47985722789589	2.92104177132475\\
5.45212943924673	2.83218483792799\\
5.41137828575746	2.72587598020497\\
5.35209610574454	2.5964902658839\\
5.26588076293174	2.43610259371712\\
5.13972473458458	2.23345390462815\\
4.95333122733118	1.97259485714987\\
4.67527629662551	1.63153661214883\\
4.2589173556774	1.18235232003381\\
3.64281353614946	0.596872128640378\\
2.76871086709487	-0.134514116493091\\
1.63239985388465	-0.969704806761522\\
0.341481053554475	-1.79923945066372\\
-0.911702846433598	-2.49772316798219\\
-1.97133552712673	-3.00385668656745\\
-2.78928181566971	-3.33288905651447\\
-3.39378724049444	-3.53260625340405\\
-3.83599982110015	-3.64816015023836\\
-4.1621849514401	-3.71157909184766\\
-4.40688862759836	-3.74318310171582\\
-4.59417261718687	-3.75534464528515\\
-4.74045144297392	-3.75552783829145\\
-4.85693590176527	-3.74825749999184\\
-4.9513691171685	-3.73630399961283\\
-5.0291828692432	-3.72138067376052\\
-5.09425280363193	-3.70455440951017\\
-5.14939157249457	-3.68648981444233\\
-5.19667372054293	-3.66759648240025\\
-5.23765253214277	-3.64811914323348\\
-5.27350690411595	-3.62819359810396\\
-5.30514231084921	-3.60788177858723\\
-5.33326119401197	-3.58719380834022\\
-5.35841265244685	-3.5661017859846\\
-5.3810278694075	-3.54454814645157\\
-5.40144551937356	-3.52245033846133\\
-5.41992997189453	-3.49970286447534\\
-5.43668416493028	-3.47617728657543\\
-5.4518583743395	-3.45172050149471\\
-5.46555564428253	-3.42615136553884\\
-5.47783428839028	-3.39925556254538\\
-5.48870756772893	-3.37077842323692\\
-5.49814035214634	-3.34041519337928\\
-5.50604222919097	-3.30779797902985\\
-5.51225608123179	-3.27247822817927\\
-5.51654052399648	-3.23390307935904\\
-5.51854366039925	-3.19138312850961\\
-5.51776414387268	-3.14404799408307\\
-5.51349321652105	-3.09078427355424\\
-5.50472759415137	-3.0301477285399\\
-5.49003678633989	-2.96023725677942\\
-5.46735789106889	-2.87851157009022\\
-5.43367306040936	-2.78151934276526\\
-5.38449473794708	-2.66449872084913\\
-5.31303423010647	-2.52078270542301\\
-5.20885353055482	-2.34093024024107\\
-5.05570892896308	-2.11152138409498\\
-4.82828202528647	-1.8137207970727\\
-4.4879570021524	-1.4223506632844\\
-3.97996628739933	-0.908028977357546\\
-3.24036481558319	-0.248412187594042\\
-2.22956888661681	0.544610566929457\\
-0.994985668439008	1.39372573796315\\
0.301866163761128	2.17079799240933\\
1.47121494724185	2.77523922020516\\
2.40982402831559	3.18773350054884\\
3.11508888893499	3.44581206508271\\
3.63202941986422	3.59858514417259\\
4.01115966441353	3.68488685491627\\
4.29300553502665	3.73043968840615\\
4.50653510037146	3.75114038016021\\
4.67163867468412	3.75659873546539\\
};
\addplot [color=mycolor1, forget plot]
  table[row sep=crcr]{%
3.86621004882867	3.35342706786038\\
4.00183357207157	3.34927247233028\\
4.1123685284783	3.33883024245586\\
4.20377044648071	3.32429056068696\\
4.28037595763327	3.30705474370343\\
4.34537997702568	3.28801997229603\\
4.40116579060115	3.26775716189122\\
4.44953332480904	3.24662287243617\\
4.49185790821677	3.22483033303829\\
4.52920164601461	3.20249486927488\\
4.56239231197835	3.17966308188443\\
4.59207976211996	3.15633153217353\\
4.61877659959918	3.13245850227588\\
4.64288763783899	3.10797105090717\\
4.66473124550015	3.08276873997757\\
4.68455466314305	3.05672486288503\\
4.7025446915998	3.02968563972196\\
4.71883465603935	3.00146758243057\\
4.73350817184596	2.97185302495721\\
4.74659992366957	2.9405836247656\\
4.7580933725364	2.90735144414468\\
4.76791498485876	2.87178698466784\\
4.77592418240633	2.83344324345249\\
4.7818976783785	2.79177444288712\\
4.78550609576224	2.74610749632804\\
4.78627961188332	2.69560342433257\\
4.78355760174661	2.63920470388718\\
4.77641448341741	2.57556273942764\\
4.7635495930259	2.50293705401208\\
4.74312197712771	2.41905414834294\\
4.71250005368641	2.32090912183603\\
4.66787928390382	2.20448757604381\\
4.60369678663558	2.06438162517247\\
4.51174254996496	1.89328081664084\\
4.37985077778534	1.68136173464953\\
4.19012139625965	1.41574376435977\\
3.91697709373688	1.08056317916047\\
3.52647748988007	0.659046311978013\\
2.98075927797431	0.140115068399494\\
2.25401139944964	-0.46838896206813\\
1.36123583027566	-1.1249298141095\\
0.378653360704582	-1.75643635154976\\
-0.579639605518917	-2.29040469240906\\
-1.41967280366853	-2.69138883814888\\
-2.10247401918624	-2.9658295509358\\
-2.63451282620022	-3.14144828765701\\
-3.04232770706418	-3.24791623306647\\
-3.3549485141355	-3.3086392593448\\
-3.59681921560905	-3.33984201416901\\
-3.78652373158095	-3.35213895732227\\
-3.93761030978875	-3.35231435881987\\
-4.05982011941077	-3.34467762409983\\
-4.16015857488049	-3.33197049824379\\
-4.24370077268982	-3.31594426507581\\
-4.31416325657116	-3.29772045747233\\
-4.37430112835846	-3.27801574968895\\
-4.42618242695195	-3.25728289561473\\
-4.47137831709626	-3.23579978582744\\
-4.51109590643123	-3.21372620312733\\
-4.54627186749189	-3.19114022325021\\
-4.5776390756402	-3.16806158910986\\
-4.60577446408777	-3.14446658671418\\
-4.63113362443092	-3.1202972373356\\
-4.65407589608681	-3.09546655614066\\
-4.67488248461812	-3.06986095178151\\
-4.69376932377597	-3.04334039813289\\
-4.71089581438933	-3.01573670290691\\
-4.72637014416942	-2.98684996782189\\
-4.74025155221982	-2.95644314047749\\
-4.7525496017215	-2.92423436818679\\
-4.76322022105016	-2.88988665116764\\
-4.77215792256101	-2.85299402715322\\
-4.77918315360532	-2.81306316485667\\
-4.78402309752955	-2.76948874955416\\
-4.78628330477081	-2.72152033843684\\
-4.78540610917088	-2.66821734199282\\
-4.78060957357864	-2.60838730064551\\
-4.77079723101516	-2.54050046645771\\
-4.7544233768589	-2.46257061045725\\
-4.72928994097639	-2.37198772160957\\
-4.69223734416034	-2.26528291109474\\
-4.63867132226216	-2.13780063274324\\
-4.56184012363635	-1.98325319303706\\
-4.45174957172697	-1.79315260240838\\
-4.29361423929005	-1.55619623735213\\
-4.06591436106207	-1.25792352950098\\
-3.73878545119878	-0.881544558184724\\
-3.27520921049671	-0.411907453801613\\
-2.64040735696895	0.154629335400547\\
-1.82504978681276	0.794725831716961\\
-0.874286234966954	1.44888122767422\\
0.110528333002808	2.0389454945846\\
1.01821992348271	2.50790754880763\\
1.7810827303388	2.84291284876407\\
2.38588406134361	3.06403611085734\\
2.8520687344207	3.20168452259337\\
3.20887045341322	3.28282783173843\\
3.48343404570952	3.32715821301171\\
3.69724199086767	3.34785822823937\\
3.86621004882867	3.35342706786038\\
};
\addplot [color=mycolor1, forget plot]
  table[row sep=crcr]{%
3.24802072526657	3.08898657669455\\
3.3897758927434	3.08463098294626\\
3.50717780454169	3.07353093809921\\
3.60555965346185	3.05787447765259\\
3.68893799237806	3.0391101092248\\
3.7603551517977	3.01819389413152\\
3.82213347265528	2.99575175340309\\
3.87606096630307	2.97218590476173\\
3.92352615805759	2.9477448044097\\
3.96561606674019	2.92256922878291\\
4.00318764128468	2.89672264545511\\
4.03692007441669	2.87021111198516\\
4.06735325513992	2.84299606912493\\
4.09491606564018	2.81500218748499\\
4.11994711853208	2.78612163966091\\
4.14270973683786	2.75621564574238\\
4.16340240243195	2.72511377835292\\
4.18216546548543	2.69261124842913\\
4.19908456276707	2.65846418064867\\
4.21419089312556	2.62238269572208\\
4.22745820611286	2.58402141962884\\
4.23879603612072	2.54296681275648\\
4.24803831449298	2.4987204280975\\
4.25492595624999	2.45067683506364\\
4.25908126159783	2.39809444370101\\
4.2599708704382	2.34005678287332\\
4.25685237487358	2.27542086782332\\
4.2486972575211	2.20274808738622\\
4.23407919947639	2.1202115526856\\
4.21101150918225	2.02547225932618\\
4.17671002598232	1.91551539532524\\
4.12724851717392	1.78643957729673\\
4.05706479291633	1.63320051076463\\
3.95827691705199	1.44933768874049\\
3.8198074555026	1.2267818260611\\
3.62645740478369	0.955994004666752\\
3.35845471980742	0.626973491313833\\
2.99278339242373	0.232044799412283\\
2.50858313759176	-0.228662052705863\\
1.89851323643394	-0.739765058077636\\
1.18289867440378	-1.26624364600204\\
0.414474409831214	-1.76017377325569\\
-0.337616378172249	-2.17915311760037\\
-1.0157375690553	-2.50268408552043\\
-1.59079943986477	-2.73365860485437\\
-2.06006652098701	-2.88843216535393\\
-2.43571632063607	-2.98641656949835\\
-2.73477077015239	-3.04444784505154\\
-2.97359690442859	-3.07522108427288\\
-3.16587677526447	-3.08766094281896\\
-3.32233866490942	-3.08782658981517\\
-3.45115212667339	-3.07976629512333\\
-3.5584688334254	-3.06616779092057\\
-3.6489146448141	-3.04881172004382\\
-3.7259821618184	-3.02887558897982\\
-3.7923266830378	-3.0071341439782\\
-3.84998372656887	-2.98409079310284\\
-3.90052729064399	-2.96006385616357\\
-3.94518476284576	-2.93524332188682\\
-3.98492053793448	-2.90972826726283\\
-4.02049711639897	-2.88355147382037\\
-4.05251993928572	-2.85669544050908\\
-4.0814703778279	-2.82910249060896\\
-4.10772998172893	-2.80068069715055\\
-4.13159815263183	-2.77130671125858\\
-4.15330473483526	-2.74082614489049\\
-4.17301851823932	-2.70905185267998\\
-4.19085226555785	-2.67576022356852\\
-4.20686455867674	-2.64068539442126\\
-4.22105846772521	-2.60351110647912\\
-4.23337674350934	-2.56385971649354\\
-4.24369287854316	-2.52127762252951\\
-4.25179692277218	-2.47521604069441\\
-4.25737430662315	-2.42500563752001\\
-4.25997501365807	-2.36982293811103\\
-4.2589691055279	-2.30864563778352\\
-4.25348260730029	-2.24019288784219\\
-4.24230478577604	-2.16284527211018\\
-4.22375345868195	-2.0745376090272\\
-4.19547866566049	-1.97261627352369\\
-4.1541765623212	-1.85365258466172\\
-4.0951757525708	-1.71320800031408\\
-4.01185252086737	-1.54556278064905\\
-3.8948460055357	-1.34346386421076\\
-3.73112332047878	-1.09805241778276\\
-3.50318686910388	-0.79934673966639\\
-3.18928479391729	-0.438006532001369\\
-2.7664475480644	-0.00939589426369803\\
-2.2188015897552	0.479649837256682\\
-1.551283722985	1.00395488422053\\
-0.801025192159766	1.52030488372181\\
-0.0322148105001834	1.98092643936069\\
0.688464523498724	2.35312769590979\\
1.3168168305411	2.62889283803365\\
1.83810593029394	2.81933767520858\\
2.2585572322313	2.94337777355454\\
2.59374360323828	3.0195352365236\\
2.86078600862169	3.06260564085484\\
3.07481850245181	3.08329773658049\\
3.24802072526657	3.08898657669455\\
};
\addplot [color=mycolor1, forget plot]
  table[row sep=crcr]{%
2.77439331404181	2.93091302742763\\
2.92201679669116	2.92636277086672\\
3.04637157876297	2.91459504312614\\
3.15208826828026	2.89776376165187\\
3.2427841339122	2.87734691491567\\
3.3212839387362	2.85435203999827\\
3.38979992259227	2.82945891805353\\
3.45007259499269	2.80311756223426\\
3.50347860095776	2.77561515658961\\
3.55111252006413	2.74712164781818\\
3.59384861636267	2.71772067013067\\
3.63238736330016	2.6874303278905\\
3.66729043889089	2.65621687058045\\
3.69900694701284	2.62400327690954\\
3.72789287992069	2.59067406893071\\
3.75422526400935	2.55607719458439\\
3.77821198551289	2.52002347276501\\
3.79999793695132	2.48228383665412\\
3.81966782478975	2.44258440184194\\
3.83724570330605	2.40059919840696\\
3.85269101930178	2.35594021887548\\
3.86589063488885	2.30814422775121\\
3.87664590262258	2.25665553496582\\
3.88465334892666	2.2008036375973\\
3.8894768110636	2.13977426643563\\
3.89050787933548	2.07257193041771\\
3.8869101016783	1.99797155235903\\
3.87754047298414	1.91445631756412\\
3.86083913397739	1.82013864395438\\
3.83467496285018	1.71266178182518\\
3.79613135228652	1.58908219207822\\
3.74121472048687	1.44574010822192\\
3.66447316001772	1.27814261737485\\
3.55853593268374	1.08091920886736\\
3.41365109965799	0.847977226098818\\
3.21745098500263	0.573094105004352\\
2.95545729312942	0.251310050265984\\
2.61320031683666	-0.118511197166621\\
2.18088500636221	-0.530058184913609\\
1.66037168352198	-0.966331273249687\\
1.07123681999009	-1.39989519993898\\
0.45011847127832	-1.79917692173613\\
-0.159383005627218	-2.13866734081843\\
-0.720633271641079	-2.40633549571523\\
-1.21252016464745	-2.60379224476195\\
-1.62935814917913	-2.74118014520894\\
-1.97576460223835	-2.83146612569576\\
-2.26115887990112	-2.8867963145686\\
-2.49603422430993	-2.91702541889677\\
-2.69007033677433	-2.9295545520092\\
-2.85144939310223	-2.92970835120029\\
-2.98678676772916	-2.92122770039677\\
-3.10131263512495	-2.90670689464467\\
-3.19912099480982	-2.88793146665785\\
-3.28340775081399	-2.86612289143199\\
-3.35667147783187	-2.84211015751369\\
-3.42087317426777	-2.81644812355084\\
-3.47755971760584	-2.78949854229469\\
-3.52795787563839	-2.76148532437063\\
-3.57304540203354	-2.73253211436099\\
-3.61360464511784	-2.70268768525832\\
-3.65026290958858	-2.6719428606411\\
-3.6835227709447	-2.64024144169147\\
-3.71378470478819	-2.60748677533737\\
-3.74136374106741	-2.57354502183569\\
-3.76650134914323	-2.53824577448667\\
-3.78937336346313	-2.50138038823889\\
-3.81009443551002	-2.4626981440497\\
-3.82871921349276	-2.42190018056349\\
-3.84524017711625	-2.37863093987127\\
-3.85958176020026	-2.33246667967303\\
-3.87159004412699	-2.28290038161987\\
-3.88101685660539	-2.2293221170647\\
-3.88749650543373	-2.17099360012175\\
-3.89051253781423	-2.10701525198191\\
-3.88935073818355	-2.03628362365249\\
-3.88303293072704	-1.95743652160941\\
-3.87022389757597	-1.8687827950562\\
-3.84910078373677	-1.76821383653104\\
-3.81717092845662	-1.65309525059751\\
-3.77102113456042	-1.52014164967862\\
-3.70598200889864	-1.36528878398867\\
-3.61570295942839	-1.1836020632259\\
-3.49167419272179	-0.969310283789998\\
-3.32283524579673	-0.716141548030639\\
-3.09562346027749	-0.418264181886093\\
-2.79515779611711	-0.0722301614851879\\
-2.40854550978567	0.319857789234741\\
-1.93088423023517	0.746616654536766\\
-1.37241539502882	1.18544283033955\\
-0.761944613841044	1.60567594363182\\
-0.141318421918017	1.97750271825346\\
0.447759648128476	2.28165213535834\\
0.975915141814334	2.51333247222445\\
1.43020225350868	2.67919557517685\\
1.81083524976017	2.79140654950542\\
2.12542789170596	2.8628250344706\\
2.38427515070782	2.90453159423087\\
2.59760801138153	2.92512684507289\\
2.77439331404181	2.93091302742763\\
};
\addplot [color=mycolor1, forget plot]
  table[row sep=crcr]{%
2.40672193448848	2.85099006036898\\
2.55946400860147	2.84626734990196\\
2.69033064347883	2.8338724347951\\
2.80322644467541	2.81588983864324\\
2.90131847641568	2.79380159691169\\
2.98715982076598	2.76865112711042\\
3.06280469611034	2.74116390876034\\
3.1299069409141	2.71183448715012\\
3.18980061012433	2.68098840470885\\
3.24356430475479	2.6488258944964\\
3.29207175671717	2.61545243483868\\
3.33603121277982	2.58089983906641\\
3.37601584075997	2.54514046966064\\
3.41248696586263	2.50809636937505\\
3.44581153967493	2.46964452440797\\
3.47627488404781	2.42961905478184\\
3.50408944080944	2.38781081578398\\
3.52939998615142	2.34396465460018\\
3.55228552002567	2.29777437053471\\
3.57275779753005	2.24887525477032\\
3.59075621010789	2.1968339208358\\
3.60613842512094	2.14113496828313\\
3.61866582373807	2.08116384252486\\
3.62798230340243	2.01618506341848\\
3.63358438976255	1.94531480750321\\
3.63477978748752	1.86748668408962\\
3.63063045232983	1.78140953722261\\
3.61987499455317	1.68551643113255\\
3.60082386156014	1.57790502980948\\
3.57121974908707	1.45627211371012\\
3.52805622590764	1.31785035715285\\
3.46735228172388	1.1593660453844\\
3.38389486753738	0.977055647018671\\
3.27099539636765	0.766811080918345\\
3.1203742964426	0.524568862670887\\
2.92240239086533	0.247103233007234\\
2.66707034700497	-0.0666239051903698\\
2.34611261882098	-0.413576567700192\\
1.9564054111336	-0.78471495934714\\
1.50379513917584	-1.16420787544064\\
1.00519331949073	-1.53123118839722\\
0.486558707776251	-1.86465454960816\\
-0.0233418928124208	-2.14863449799717\\
-0.500224441331953	-2.37600079694935\\
-0.92876000791341	-2.54795215360327\\
-1.30292325563112	-2.67120671237276\\
-1.62364176742573	-2.75474212836996\\
-1.89581714913953	-2.80746710454253\\
-2.12595426693129	-2.8370548462651\\
-2.32069791209789	-2.84960647951803\\
-2.48610350192749	-2.84974705246401\\
-2.62736899749821	-2.84088225899685\\
-2.74881112519193	-2.82547500293986\\
-2.85395039459688	-2.80528503157114\\
-2.94563164671655	-2.78155750762754\\
-3.02614509779226	-2.75516411941453\\
-3.09733360141841	-2.72670575288099\\
-3.16068210115339	-2.69658598894361\\
-3.21738980476314	-2.66506317591553\\
-3.26842731238166	-2.6322870135419\\
-3.314581298808	-2.59832399127148\\
-3.35648915611748	-2.56317477226966\\
-3.39466561513653	-2.52678568243184\\
-3.42952294651184	-2.48905578494707\\
-3.46138595739844	-2.44984052919454\\
-3.49050266464898	-2.40895260246941\\
-3.51705123538792	-2.36616034149742\\
-3.54114352770935	-2.32118384593142\\
-3.5628253207578	-2.27368875417122\\
-3.58207307499217	-2.22327747492459\\
-3.59878678753142	-2.16947750238967\\
-3.61277817756754	-2.11172626935765\\
-3.62375302059351	-2.04935180693829\\
-3.63128590837492	-1.98154828765504\\
-3.63478499888611	-1.90734535493057\\
-3.63344339233583	-1.82557004893318\\
-3.62617260376615	-1.73480026336293\\
-3.61151225487164	-1.63330929028357\\
-3.58750883853339	-1.5190026775697\\
-3.55155596273051	-1.38935235741425\\
-3.50019064181572	-1.2413406084902\\
-3.42884882418522	-1.07144080442315\\
-3.3316059040818	-0.875687035934242\\
-3.20097695900251	-0.649923754453545\\
-3.02794219877558	-0.390375127140654\\
-2.80249808934361	-0.094702634584102\\
-2.51515717546199	0.236352716673745\\
-2.15973732910717	0.596958360645233\\
-1.73716603044673	0.974644700231005\\
-1.25877548178621	1.35066068275713\\
-0.746556765376957	1.70331415643677\\
-0.228820724963049	2.01349036870507\\
0.267101046683619	2.26948919152698\\
0.721110320794781	2.46857197377776\\
1.12270709349533	2.61512503245354\\
1.46971763786224	2.71736211276299\\
1.76540696891737	2.78444046005549\\
2.01571702780733	2.82473480138731\\
2.22735322408389	2.84513919980508\\
2.40672193448848	2.85099006036898\\
};
\addplot [color=mycolor1, forget plot]
  table[row sep=crcr]{%
2.11655100682591	2.82843428941261\\
2.27354611471052	2.82356578471402\\
2.41025665966359	2.81060627378123\\
2.52989546066145	2.79154091646479\\
2.63516721135814	2.76782903818278\\
2.72832261743308	2.74053005988702\\
2.81122263331923	2.71040195994312\\
2.88540101761024	2.67797585950846\\
2.95212010046617	2.64361138065595\\
3.01241816078134	2.60753712579265\\
3.06714850265728	2.56987987342384\\
3.11701103132388	2.530685279376\\
3.16257733553198	2.48993216479353\\
3.20431024779221	2.44754189978502\\
3.2425787120241	2.40338394581532\\
3.27766860712889	2.35727827727643\\
3.30978998395011	2.30899513789073\\
3.33908098109372	2.25825237864351\\
3.36560849027125	2.20471045325069\\
3.38936543605966	2.14796500184279\\
3.41026430598958	2.08753682547208\\
3.42812629933144	2.02285894077584\\
3.44266513926722	1.95326031123863\\
3.4534641954334	1.87794579759002\\
3.4599450789137	1.79597189595953\\
3.46132530302978	1.70621801868249\\
3.45656199668418	1.60735356718522\\
3.44427815481841	1.49780211181684\\
3.4226678564112	1.37570607723411\\
3.38937801254489	1.23889915632012\\
3.34136798234704	1.08490031045564\\
3.27475744922555	0.910953996822627\\
3.18469151120958	0.71415720554415\\
3.06528542191618	0.491733786396986\\
2.90976311744834	0.241532573386507\\
2.71096460636707	-0.0371839138929645\\
2.4624261097362	-0.342670722307781\\
2.16014017496843	-0.6695545814815\\
1.80476531683354	-1.00810768327461\\
1.40349383662699	-1.34464591286619\\
0.97038036758317	-1.66351810247334\\
0.524318870350239	-1.95029936328905\\
0.085159888355619	-2.19486019379004\\
-0.330303447862874	-2.39290064579353\\
-0.710776629391997	-2.54551613559592\\
-1.05080767266494	-2.65747804245967\\
-1.34965578360068	-2.7352745046498\\
-1.60964878916396	-2.78560491585113\\
-1.83470724848301	-2.81451245994852\\
-2.02929746385226	-2.82703310171971\\
-2.19780572461383	-2.82716004636509\\
-2.34422376368673	-2.81795933766786\\
-2.47202961915923	-2.80173490909645\\
-2.58417664940125	-2.78019148732509\\
-2.68313477180989	-2.75457447721281\\
-2.77095171368762	-2.72578186937574\\
-2.84931736926041	-2.69445020124993\\
-2.91962332499729	-2.66101894855914\\
-2.98301449583141	-2.625777943375\\
-3.0404322803138	-2.5889018140239\\
-3.0926497659045	-2.55047463225179\\
-3.14029993083631	-2.51050718663972\\
-3.18389785141002	-2.46894866039765\\
-3.22385782311156	-2.42569398502503\\
-3.26050613704066	-2.38058775033286\\
-3.29409006498849	-2.3334252506136\\
-3.32478341468164	-2.28395101236323\\
-3.35268882400787	-2.23185496103844\\
-3.37783676405674	-2.17676622764621\\
-3.4001810051745	-2.11824445995762\\
-3.41959005405378	-2.05576838234722\\
-3.43583377655189	-1.98872124382334\\
-3.44856406235342	-1.91637271697959\\
-3.45728794729811	-1.83785678961846\\
-3.46133108014034	-1.75214528356065\\
-3.45978882279539	-1.65801695123747\\
-3.45146169164846	-1.55402283821256\\
-3.43477150758744	-1.43845010349195\\
-3.40765503359	-1.3092893324724\\
-3.36743408852029	-1.16421545119853\\
-3.31066711786081	-1.00060088864916\\
-3.23300037864963	-0.815592941179428\\
-3.12906223935289	-0.606305635533384\\
-2.99248669002533	-0.370196004944904\\
-2.81621104383475	-0.105701784839631\\
-2.59324586139854	0.186819618571972\\
-2.31809609238649	0.503941380582688\\
-1.98880682986173	0.838150307963187\\
-1.60913736983443	1.17759478204281\\
-1.18981009540952	1.5072611728684\\
-0.747698446788933	1.81168182170006\\
-0.30271621218734	2.07826646569063\\
0.126382115899771	2.29973691404187\\
0.525382985478411	2.47465037794311\\
0.885985247779934	2.60619273420778\\
1.20528688360715	2.70022002649834\\
1.48428693993919	2.7634737488266\\
1.72626932379742	2.80239670394494\\
1.93552996856662	2.8225480577959\\
2.11655100682591	2.82843428941261\\
};
\addplot [color=mycolor1, forget plot]
  table[row sep=crcr]{%
1.8839109593124	2.84837258818049\\
2.04438503565292	2.84338280475722\\
2.18623968056034	2.82992488077221\\
2.31207075292011	2.80986400883025\\
2.42414288002402	2.78461324524576\\
2.524399227141	2.75522744745394\\
2.6144895971911	2.72248123179932\\
2.69580523727765	2.68693097387905\\
2.76951401643886	2.64896275152078\\
2.83659283040392	2.60882865184404\\
2.89785590868346	2.56667375911347\\
2.95397867354868	2.52255579514719\\
3.00551727602874	2.47645898572703\\
3.05292412571317	2.42830335512052\\
3.09655976598816	2.37795033370079\\
3.13670139641996	2.32520530362994\\
3.17354824791263	2.26981749794226\\
3.20722389261628	2.21147750004641\\
3.23777542608711	2.14981245574515\\
3.26516929363055	2.08437900355741\\
3.28928334186076	2.01465385142123\\
3.30989445454165	1.94002188583742\\
3.32666087447042	1.85976171107942\\
3.33909802279223	1.77302861621705\\
3.34654632203126	1.67883521851811\\
3.3481292617274	1.57603053813278\\
3.34269983548856	1.46327919049815\\
3.32877377018421	1.33904400156975\\
3.30444912628243	1.20157803439204\\
3.26731469495041	1.04893624597627\\
3.21435548933653	0.879023236256821\\
3.14187443310187	0.689701858699571\\
3.04546717651756	0.478996523364139\\
2.92011247449501	0.245430336841415\\
2.7604689387904	-0.0114734525998377\\
2.5614829365389	-0.290535148120325\\
2.31937579911317	-0.588206229077695\\
2.03294464193934	-0.898035058199291\\
1.70486833862432	-1.21066306833375\\
1.34246114266892	-1.51466922851812\\
0.957298419245449	-1.7982752131383\\
0.563544258574439	-2.05143637377369\\
0.17548435714616	-2.26752728383446\\
-0.194793772703191	-2.44400064822539\\
-0.538789710898925	-2.58194925525039\\
-0.851840571636949	-2.68499183533562\\
-1.13253788137548	-2.75803083744598\\
-1.38178777086944	-2.80625377472262\\
-1.60188666155828	-2.83450133180103\\
-1.79579559551911	-2.84695953436325\\
-1.96665021039329	-2.84707329656296\\
-2.11746986643662	-2.83758398914435\\
-2.25100884799222	-2.82062207494757\\
-2.3696979777122	-2.79781403418103\\
-2.47563843265873	-2.77038304485132\\
-2.57062256708164	-2.73923512802462\\
-2.65616642914844	-2.70502896323499\\
-2.73354530585091	-2.66823056893121\\
-2.80382776856561	-2.62915512247521\\
-2.86790611868862	-2.58799833682481\\
-2.92652247610629	-2.5448595531361\\
-2.98029044423412	-2.49975832182142\\
-3.02971259851557	-2.45264585394876\\
-3.07519414669671	-2.40341237922406\\
-3.11705309513639	-2.35189115869908\\
-3.15552717907817	-2.29785966662167\\
-3.19077770350021	-2.24103826799102\\
-3.22289030677915	-2.18108656759356\\
-3.25187250472187	-2.11759748601923\\
-3.27764769496605	-2.05008902577164\\
-3.3000450960285	-1.97799362927374\\
-3.31878485592477	-1.90064501222353\\
-3.33345729045611	-1.81726240582165\\
-3.34349490965776	-1.72693230686436\\
-3.34813559491003	-1.62858819671687\\
-3.34637507943646	-1.52098938281574\\
-3.33690693424297	-1.40270135184946\\
-3.3180489099609	-1.27208212003948\\
-3.28765635604558	-1.12728245865763\\
-3.24302759373264	-0.966273063059091\\
-3.18081418964464	-0.786919056262009\\
-3.09696316231815	-0.587131211462139\\
-2.98673994755229	-0.365131324720503\\
-2.84490907694386	-0.119869220079494\\
-2.66617395415224	0.148392918344402\\
-2.44597148905526	0.437376123390119\\
-2.18163739288019	0.742122945956752\\
-1.87376584258502	1.05468104394214\\
-1.52731875729217	1.36449608541033\\
-1.15187248617585	1.65971438859468\\
-0.760574942281166	1.92916920291734\\
-0.367973512747125	2.16436994990962\\
0.0125009114487025	2.36072200208788\\
0.370454450605222	2.51760845619305\\
0.69933719935473	2.637544049508\\
0.996219485266728	2.7249350397717\\
1.2609731326353	2.7849286865175\\
1.49530475588198	2.82259567527393\\
1.70191874368275	2.84247146195521\\
1.8839109593124	2.84837258818049\\
};
\addplot [color=mycolor1, forget plot]
  table[row sep=crcr]{%
1.69485018192866	2.89996807890675\\
1.85820655715354	2.89487647765564\\
2.00457984397636	2.8809797270943\\
2.13604336602734	2.86001243183153\\
2.25446721546826	2.83332346706828\\
2.36150279435773	2.80194460779108\\
2.45858756385122	2.76665089554655\\
2.54696018123358	2.72801094037392\\
2.62767995343108	2.68642736149034\\
2.70164705308327	2.64216844335124\\
2.76962154304058	2.59539234248843\\
2.83224022351806	2.54616513680408\\
2.89003087130571	2.49447383688251\\
2.94342373414113	2.44023526897727\\
2.99276027509175	2.38330153487777\\
3.03829919395036	2.32346257290518\\
3.08021972265549	2.26044619262228\\
3.11862212089304	2.19391583437811\\
3.15352519751962	2.1234662139144\\
3.18486055866431	2.04861695467206\\
3.21246313728203	1.96880429383634\\
3.23605739555129	1.88337098822631\\
3.25523842050502	1.79155467072672\\
3.2694469773572	1.69247516304914\\
3.27793749214014	1.58512170826644\\
3.27973799910182	1.46834185451507\\
3.27360148014203	1.34083495111\\
3.25794903920057	1.20115510414749\\
3.23080747248973	1.04773118604595\\
3.18974772417294	0.878915217825194\\
3.13183736099263	0.693074924434428\\
3.05363038060174	0.488750457842159\\
2.95123130069541	0.264896523752804\\
2.82048480477904	0.0212242347237356\\
2.65734908593136	-0.241365893412844\\
2.45849463889622	-0.520314844299026\\
2.22211017793365	-0.811023372373281\\
1.94878415697773	-1.10674648214036\\
1.6421946299421	-1.39895948782619\\
1.30927104075118	-1.67827753077494\\
0.959596009127865	-1.9357781217093\\
0.604107705613491	-2.1643428143617\\
0.25348980666344	-2.35957492855925\\
-0.0832302884218608	-2.5200350160566\\
-0.399501438849518	-2.64684060282257\\
-0.691411116616583	-2.74289832883057\\
-0.957357474203116	-2.81207428785222\\
-1.19748394163742	-2.85850998165818\\
-1.41308626186256	-2.88616140914034\\
-1.60611127361413	-2.89854677647915\\
-1.77878771692437	-2.89864837370952\\
-1.93338211754963	-2.88891044962224\\
-2.07205337146284	-2.87128738842466\\
-2.19677689735299	-2.84731199527596\\
-2.30931382793077	-2.81816645646392\\
-2.41120717242651	-2.784747251869\\
-2.50379272061424	-2.74772055487814\\
-2.58821691441871	-2.70756750439545\\
-2.66545700910847	-2.66462009255071\\
-2.73634086543492	-2.61908892742578\\
-2.80156496636929	-2.57108420877264\\
-2.86170999100684	-2.52063113127606\\
-2.9172536873189	-2.46768073119723\\
-2.96858098836634	-2.41211698161585\\
-3.01599139254191	-2.35376074751745\\
-3.05970362599114	-2.29237104537152\\
-3.09985755302369	-2.2276439152803\\
-3.13651321368978	-2.1592091074652\\
-3.16964675459914	-2.08662471002761\\
-3.19914288327941	-2.00936980644851\\
-3.22478332091583	-1.92683526041964\\
-3.24623055931555	-1.83831280347719\\
-3.26300606042059	-1.7429827835832\\
-3.27446190405073	-1.63990127816714\\
-3.27974485941134	-1.52798787187569\\
-3.27775205581099	-1.40601637633116\\
-3.26707808478902	-1.27261230077595\\
-3.24595485914421	-1.12626317540111\\
-3.21218847434543	-0.965351063463567\\
-3.16310248510393	-0.788220765195193\\
-3.09550534797973	-0.593301764250026\\
-3.0057118108052	-0.379305157815972\\
-2.88966262082075	-0.145514809122397\\
-2.7431989937066	0.10782186103107\\
-2.5625459681075	0.379032043483445\\
-2.34502296408943	0.664572021693395\\
-2.08991266545217	0.958757334829479\\
-1.79928738607017	1.25387241352455\\
-1.47847542184976	1.54081565734261\\
-1.13585696179675	1.81025596046772\\
-0.781886175584397	2.05402234197167\\
-0.427573956485562	2.26628262034397\\
-0.0829201636283366	2.44413355176673\\
0.244226294998288	2.58749481622467\\
0.548649664549752	2.69848476061002\\
0.827654445755815	2.78058761635831\\
1.0805877294045	2.83787921004732\\
1.3082392155596	2.87445182827387\\
1.51228482589137	2.89406303166017\\
1.69485018192866	2.89996807890675\\
};
\addplot [color=mycolor1, forget plot]
  table[row sep=crcr]{%
1.5395140061444	2.97500957994766\\
1.70534410082583	2.96983000906519\\
1.85572707401358	2.95554327789407\\
1.9923107530755	2.93375139112282\\
2.11662750506519	2.90572748918844\\
2.23006643917711	2.87246545096575\\
2.33386447233891	2.83472602671762\\
2.42910859408973	2.79307698683249\\
2.51674415445317	2.74792657593276\\
2.59758582791122	2.69955050273766\\
2.67232917150208	2.64811311550725\\
2.74156152853298	2.59368353952608\\
2.80577155393159	2.53624752869216\\
2.8653569492562	2.47571569084652\\
2.92063016456976	2.41192863217861\\
2.97182189850544	2.34465945447924\\
3.01908223835145	2.27361394319377\\
3.06247924904098	2.19842871193207\\
3.10199475663549	2.11866752710578\\
3.13751698776756	2.03381603362686\\
3.16882963099094	1.94327515304639\\
3.19559679248079	1.84635355110563\\
3.21734325007464	1.74225980594756\\
3.23342940734364	1.63009530021964\\
3.2430204850141	1.50884947785317\\
3.24504988117975	1.37740003626802\\
3.2381774770256	1.23452196372132\\
3.22074524976044	1.07891115508892\\
3.19073527136989	0.909230628141407\\
3.14573946868521	0.724189852387416\\
3.08295671749822	0.522669626245742\\
2.99924068776572	0.303904659505936\\
2.89122968292588	0.0677307008671348\\
2.75559327064357	-0.185111188979598\\
2.58942210103305	-0.452647179329826\\
2.39075715043921	-0.731391841739698\\
2.15919697674414	-1.01622745388764\\
1.89644649909105	-1.30056297494084\\
1.60661428118054	-1.57684933779702\\
1.29608038339813	-1.83741477251027\\
0.972874658231894	-2.07544116347108\\
0.645689164156102	-2.2858126665623\\
0.32279690444423	-2.46560003908645\\
0.0111702639346649	-2.61408822075381\\
-0.284018382596925	-2.73242298820054\\
-0.559493616072948	-2.82305325505413\\
-0.813673044702863	-2.88914946235601\\
-1.04630650619837	-2.93411858390528\\
-1.25808172423759	-2.96126352480465\\
-1.45027307466865	-2.97358176151219\\
-1.62446613306611	-2.97367251087915\\
-1.78236127963077	-2.96371661956631\\
-1.92564490571912	-2.94549878857511\\
-2.05591214294895	-2.92045034564002\\
-2.17462581182947	-2.88969872908586\\
-2.28309919511411	-2.8541158283054\\
-2.38249348967667	-2.8143613010691\\
-2.47382360087399	-2.77091938932807\\
-2.55796809608302	-2.72412907714267\\
-2.63568066532851	-2.67420807934244\\
-2.70760146949569	-2.62127140011584\\
-2.77426742181663	-2.56534523839753\\
-2.83612085628008	-2.50637695086907\\
-2.89351626999702	-2.44424167588688\\
-2.94672494338748	-2.37874610655234\\
-2.99593728124691	-2.30962979626666\\
-3.04126270453835	-2.23656429522712\\
-3.08272687330215	-2.15915035826483\\
-3.12026594626481	-2.07691344069961\\
-3.15371749158923	-1.98929772057509\\
-3.18280756657785	-1.89565897027643\\
-3.20713339939221	-1.79525677540025\\
-3.22614106377603	-1.68724690463044\\
-3.23909759290579	-1.57067512940798\\
-3.24505722438492	-1.44447455408784\\
-3.24282205851402	-1.30746963988646\\
-3.23089858435342	-1.15839168124217\\
-3.20745362432094	-0.995912562443984\\
-3.17027670028752	-0.818706068731119\\
-3.11676105244137	-0.625548373771311\\
-3.04392265811155	-0.415470422042853\\
-2.94848477357015	-0.18797256088483\\
-2.82706195131493	0.0566974860419996\\
-2.67647622903883	0.317222338979823\\
-2.4942200968804	0.590900251243522\\
-2.27903699278498	0.8734299226913\\
-2.0315209511659	1.15891560518726\\
-1.75456510124819	1.44020027978298\\
-1.45346149726645	1.70955456555951\\
-1.13552077609128	1.95961285942646\\
-0.809240575052896	2.18432063074294\\
-0.48323218558479	2.37962329690684\\
-0.165206978506065	2.54372251001265\\
0.138726712936532	2.67689519113288\\
0.424354947668622	2.78101359892693\\
0.689287467491502	2.85895599889901\\
0.932657573845497	2.91406305971143\\
1.15473186815185	2.94972304820235\\
1.35653090954332	2.96910368814714\\
1.5395140061444	2.97500957994766\\
};
\addplot [color=mycolor1, forget plot]
  table[row sep=crcr]{%
1.41082729830732	3.06696346341668\\
1.57888188824117	3.06170488517268\\
1.73288335955502	3.04706602621272\\
1.87414312688596	3.02452072980028\\
2.00391454723165	2.99526075284613\\
2.12336056424584	2.96023162134034\\
2.2335376505108	2.92016776505852\\
2.33539028955245	2.87562431982674\\
2.42975183496754	2.82700448312965\\
2.51734885600281	2.77458217793819\\
2.59880702224616	2.71852023752744\\
2.67465724389923	2.65888453001565\\
2.74534123038429	2.59565450298443\\
2.81121591551885	2.52873061463361\\
2.87255636770134	2.45793907089664\\
2.92955689257045	2.38303423364092\\
2.98233006803905	2.30369901990077\\
3.0309034452899	2.21954358802503\\
3.07521361811412	2.13010261489511\\
3.1150973197344	2.03483152232268\\
3.15027916623443	1.93310212832522\\
3.1803556508972	1.82419840505262\\
3.20477503873642	1.70731335329514\\
3.22281297056105	1.58154849661164\\
3.23354394743114	1.44591820544066\\
3.23580955861631	1.29936202832266\\
3.22818552015256	1.14076944943835\\
3.20895153448483	0.969022943927854\\
3.17607089684429	0.783066626566984\\
3.12719077837387	0.582008631504728\\
3.05967895816599	0.365264576686387\\
2.9707174033871	0.132745404909382\\
2.85747509860654	-0.114916532506813\\
2.71737778747114	-0.376125734509712\\
2.54847564470256	-0.648111497870774\\
2.34987790098481	-0.92681406401126\\
2.12217966549545	-1.20694813391023\\
1.86776648471642	-1.48230420879851\\
1.59087330425521	-1.74629029409601\\
1.2973196913592	-1.99263140219444\\
0.993939257680404	-2.21607026299848\\
0.687827986807597	-2.41289475473447\\
0.385597020319771	-2.58117320154147\\
0.092798406562557	-2.72067983488085\\
-0.186386490639635	-2.83258573447756\\
-0.449198933908991	-2.91903531731147\\
-0.694162162043553	-2.98272022812905\\
-0.920836124759431	-3.0265233621924\\
-1.12954360328692	-3.05326227094516\\
-1.32111758772385	-3.06552945518219\\
-1.49669412922668	-3.06561076784676\\
-1.65755646084978	-3.05545885686464\\
-1.80502578545109	-3.03670100108363\\
-1.940389825552	-3.01066563361817\\
-2.0648596017813	-2.97841687486326\\
-2.17954605591969	-2.94079047915299\\
-2.28544987622582	-2.8984275184682\\
-2.38345960078487	-2.85180403442792\\
-2.47435451497848	-2.80125604253879\\
-2.55880996272381	-2.74699991262773\\
-2.63740348711924	-2.68914846490745\\
-2.7106207635212	-2.62772324420868\\
-2.77886064751763	-2.56266345208767\\
-2.84243888316447	-2.49383198198069\\
-2.90159014285708	-2.42101894941071\\
-2.95646812835036	-2.34394305783714\\
-3.00714347359018	-2.26225110474146\\
-3.05359916966103	-2.17551592317112\\
-3.09572319326319	-2.08323308317822\\
-3.13329797629251	-1.98481676073562\\
-3.16598632330039	-1.87959533992676\\
-3.19331339329362	-1.76680757652886\\
-3.21464445626887	-1.64560055585467\\
-3.22915838245868	-1.51503127165456\\
-3.23581732988064	-1.37407448527119\\
-3.23333401795761	-1.22164063012467\\
-3.22013951826568	-1.05660889000736\\
-3.19435689680321	-0.877882063447059\\
-3.15378950237545	-0.684471055268767\\
-3.09593718817678	-0.475617033070586\\
-3.01805869148634	-0.250957106870563\\
-2.91730211708451	-0.0107329121178832\\
-2.7909247167741	0.24397134333908\\
-2.63661310096762	0.510994758935142\\
-2.4528908367777	0.786927069310679\\
-2.23956137318445	1.06707387295504\\
-1.9980896606226	1.34563420623355\\
-1.73179791579767	1.61612676971752\\
-1.44576736588601	1.87202631518176\\
-1.14641122706676	2.10748617964671\\
-0.840792519535627	2.31797229868893\\
-0.535850622331257	2.50065357292545\\
-0.237723450808828	2.65447783145559\\
0.0487005982121677	2.77996645491535\\
0.319959039170133	2.87883253552236\\
0.573960279152549	2.95354414673502\\
0.8097798590146	3.00692708029006\\
1.02739289189082	3.04185725774207\\
1.22740630901176	3.06105425092449\\
1.41082729830732	3.06696346341668\\
};
\addplot [color=mycolor1, forget plot]
  table[row sep=crcr]{%
1.30361953236429	3.17036463029885\\
1.47376416606297	3.16503240495085\\
1.63108944317391	3.15007019639534\\
1.7766465540186	3.12683240291195\\
1.91146452429925	3.09642860492775\\
2.03651750336495	3.05974978176111\\
2.1527056347414	3.01749524388881\\
2.26084522591371	2.97019785848818\\
2.36166495027574	2.91824634527061\\
2.45580568463729	2.86190416805311\\
2.54382227054595	2.80132498247682\\
2.62618599578559	2.73656483097968\\
2.70328695170963	2.66759138181632\\
2.77543566424828	2.59429054405119\\
2.84286355022453	2.51647079143163\\
2.90572183974832	2.43386551933816\\
2.96407864894928	2.34613375805254\\
3.01791390089306	2.2528595865683\\
3.06711179070596	2.15355064791606\\
3.11145048955135	2.04763627519511\\
3.15058880159253	1.93446591646719\\
3.18404955764049	1.81330882011404\\
3.21119969208565	1.68335633779026\\
3.23122726951574	1.54372874757138\\
3.24311629403442	1.39348921404227\\
3.2456210665141	1.23166837496241\\
3.23724329235029	1.05730400136355\\
3.21621721229362	0.869501022271874\\
3.18051076818563	0.667517534924947\\
3.12785401697737	0.450881541738476\\
3.05580897864137	0.21954004163846\\
2.96189638125696	-0.0259644596040753\\
2.84379194918649	-0.284305861266947\\
2.69959498153774	-0.553205406163492\\
2.52815280101113	-0.829327537810989\\
2.32939746931219	-1.10829520263138\\
2.10462355751182	-1.38487132898864\\
1.85662197887327	-1.65332173850767\\
1.58960019902782	-1.90792269649361\\
1.30886852536418	-2.14352180023087\\
1.02034111000833	-2.35603101754621\\
0.729958305028817	-2.54274468527235\\
0.443155842501758	-2.70242930019275\\
0.164479140140104	-2.83519990861521\\
-0.10261427537527	-2.94224915593108\\
-0.355776568571634	-3.02551318243085\\
-0.593663001368655	-3.08734677455416\\
-0.815754331765168	-3.13025317655978\\
-1.022157458092	-3.15668638671042\\
-1.21341761347519	-3.16892386956687\\
-1.39035957818846	-3.16899711976838\\
-1.5539635035359	-3.15866437392518\\
-1.7052737242077	-3.13941095678695\\
-1.84533548233189	-3.11246576440986\\
-1.97515349264969	-3.07882568028455\\
-2.09566662361031	-3.03928254929481\\
-2.20773388699658	-2.99444947212902\\
-2.31212797355392	-2.94478466322395\\
-2.4095335250971	-2.89061207068761\\
-2.50054811227694	-2.83213853275303\\
-2.58568448087765	-2.76946756644291\\
-2.66537305925385	-2.70261004411595\\
-2.73996401639047	-2.63149207845245\\
-2.80972835476686	-2.55596045081979\\
-2.87485764102363	-2.47578591157516\\
-2.93546204161393	-2.39066467413014\\
-2.99156635732208	-2.30021843292352\\
-3.04310375458557	-2.20399327251553\\
-3.08990688777381	-2.10145791534584\\
-3.13169611264067	-1.99200189660985\\
-3.16806453130349	-1.87493447766496\\
-3.19845971881004	-1.74948544026469\\
-3.22216221323852	-1.61480937119357\\
-3.23826128142189	-1.46999567551543\\
-3.24562920526706	-1.3140873543616\\
-3.24289650054273	-1.14611251355545\\
-3.22843222084146	-0.965133505438952\\
-3.20033591008952	-0.770319259909192\\
-3.15645078689902	-0.561046193550949\\
-3.09441095916207	-0.337031235685504\\
-3.01173783361475	-0.098495821530246\\
-2.90600046968373	0.153648974685787\\
-2.77504864012543	0.417619031336771\\
-2.6173129607066	0.690614035340361\\
-2.43214277890123	0.968766232625099\\
-2.2201237012431	1.24723435609051\\
-1.98329388368493	1.52047682401108\\
-1.72517741052467	1.78269522605538\\
-1.4505859192436	2.0283827126093\\
-1.16520188757099	2.25286634840677\\
-0.875024671769417	2.45272336342021\\
-0.585801320529994	2.62598737981331\\
-0.302559157014001	2.7721257626261\\
-0.0293125372598301	2.89183212438409\\
0.231042498441196	2.9867134565383\\
0.476678738039672	3.05895312777444\\
0.7066887332391	3.11100953386938\\
0.920891090611341	3.14538139339833\\
1.11963378465583	3.16444637329992\\
1.30361953236428	3.17036463029885\\
};
\addplot [color=mycolor1, forget plot]
  table[row sep=crcr]{%
1.21404429523693	3.28044238815931\\
1.38621268985409	3.27503967250787\\
1.54663278389918	3.25977667517288\\
1.69615770955681	3.23589953360583\\
1.83563983114042	3.20443849817299\\
1.96589963992468	3.16622753638176\\
2.08770582995269	3.12192530571178\\
2.20176333666588	3.07203536120413\\
2.30870677728523	3.01692441575519\\
2.40909733174624	2.9568380968778\\
2.50342159874684	2.89191403251249\\
2.5920913452711	2.82219232605669\\
2.67544335036796	2.74762360551715\\
2.75373874229953	2.66807489620662\\
2.82716136067381	2.5833336011911\\
2.89581475873685	2.49310990092694\\
2.95971751088797	2.39703792042776\\
3.01879652122194	2.29467607384036\\
3.07287805593017	2.18550709727907\\
3.12167626436159	2.06893843718183\\
3.164779035279	1.944303891071\\
3.20163119093747	1.81086771903676\\
3.23151530030509	1.66783287357184\\
3.25353085916163	1.51435553835408\\
3.26657332103525	1.34956880393622\\
3.26931556015483	1.17261897059671\\
3.26019588174115	0.982718497951086\\
3.23741867330031	0.779219721340932\\
3.19897605515409	0.561712644588827\\
3.14270097103519	0.330147720524914\\
3.06636311552953	0.0849797754968957\\
2.96781741455216	-0.172678419947656\\
2.84520857548179	-0.440914677424546\\
2.69722299994229	-0.716920591449242\\
2.52336142265399	-0.996979099102004\\
2.32418586746018	-1.27657335860218\\
2.10148095611961	-1.55063605168546\\
1.85827225065407	-1.81392514924874\\
1.59866930506816	-2.06147264959836\\
1.32754437871465	-2.28902302570026\\
1.05010372376851	-2.49337372784765\\
0.771437131170872	-2.67255571357623\\
0.496130795580913	-2.82583698514912\\
0.228001336931145	-2.95357686792502\\
-0.0300302684969267	-3.05698655331028\\
-0.275940431953769	-3.1378566857644\\
-0.50851573553319	-3.19830070194832\\
-0.727216481893093	-3.24054311834333\\
-0.932024563568576	-3.26676350699635\\
-1.12329864221807	-3.27899388469962\\
-1.30164924092964	-3.27906033897986\\
-1.46783839273164	-3.26855755833863\\
-1.62270342414218	-3.24884563813068\\
-1.76710181361321	-3.22106052479423\\
-1.90187309514551	-3.18613172740119\\
-2.02781378660798	-3.14480294762885\\
-2.14566180775921	-3.0976528684895\\
-2.25608750024786	-3.0451144903595\\
-2.35968899977471	-2.98749218015208\\
-2.45699026122276	-2.92497609629639\\
-2.54844047680928	-2.85765395199859\\
-2.6344139584906	-2.78552024918273\\
-2.71520979391882	-2.70848320565843\\
-2.79105074844663	-2.62636964450667\\
-2.86208099170613	-2.53892814329307\\
-2.92836229218099	-2.4458307708287\\
-2.9898683616798	-2.34667378659936\\
-3.04647705843351	-2.24097775733803\\
-3.09796018939814	-2.12818767179549\\
-3.14397071057272	-2.00767382546431\\
-3.18402723796887	-1.87873452025237\\
-3.21749599223031	-1.74060199786171\\
-3.24357066319546	-1.59245351215947\\
-3.2612512708607	-1.43343004080377\\
-3.26932400463725	-1.2626657993691\\
-3.26634533158282	-1.07933234415153\\
-3.25063542710014	-0.882701412409692\\
-3.22028813827935	-0.672230366860738\\
-3.1732069482427	-0.447672601718421\\
-3.1071780824799	-0.209211783624883\\
-3.01999175754808	0.0423873708727519\\
-2.90961887002595	0.305628153185356\\
-2.77444132743723	0.578158055922668\\
-2.61351892168651	0.856709056758445\\
-2.42685603899915	1.13714194062122\\
-2.21561361070748	1.41462476740585\\
-1.98220507942266	1.6839496369487\\
-1.73022872524769	1.93995393006796\\
-1.46422412897447	2.17797530792202\\
-1.18928764889999	2.39425119110813\\
-0.91062143246285	2.58618445637019\\
-0.633105203016797	2.75243455599225\\
-0.360964850674611	2.89284069661211\\
-0.0975764829442119	3.00822147313603\\
0.154593906607722	3.10011179527205\\
0.393942688293211	3.17049342432048\\
0.619612011186499	3.22155838822615\\
0.831341940748543	3.25552473402513\\
1.02931979977233	3.27450805884036\\
1.21404429523693	3.28044238815931\\
};
\addplot [color=mycolor1, forget plot]
  table[row sep=crcr]{%
1.13918986385114	3.39291030692128\\
1.31334254622258	3.38743932459056\\
1.47666231161076	3.37189486946436\\
1.62985336376191	3.34742713098829\\
1.77363116361104	3.31499238731071\\
1.9086936208948	3.27536813870626\\
2.03570151526558	3.22916982839512\\
2.15526566937106	3.17686731060412\\
2.26793883658075	3.11879998754381\\
2.37421068991487	3.05519005756676\\
2.47450466317591	2.98615365655134\\
2.56917568712316	2.91170988787152\\
2.65850808526946	2.83178786715564\\
2.74271305516713	2.74623199020703\\
2.82192527413863	2.65480569155722\\
2.89619824619334	2.55719401700191\\
2.96549806255904	2.45300540251019\\
3.02969529607241	2.34177314884866\\
3.08855480632043	2.2229572204889\\
3.14172331913164	2.09594719327623\\
3.18871478900932	1.96006744155325\\
3.22889379417839	1.81458600194816\\
3.2614576002984	1.65872897810248\\
3.28541812055037	1.49170283761927\\
3.29958586098608	1.31272743994833\\
3.30255912144019	1.12108299720362\\
3.29272322416275	0.916174187090234\\
3.26826625015252	0.697613960216642\\
3.22721935403153	0.465327738155843\\
3.16753056484358	0.219675153705707\\
3.08718005056154	-0.0384191370860458\\
2.98434084579235	-0.307341574723892\\
2.85758089070568	-0.584696951783404\\
2.70608979312119	-0.867277122662939\\
2.52989900831624	-1.15112144140111\\
2.3300519147241	-1.4316889926073\\
2.10867725524729	-1.70414101309216\\
1.86893120683685	-1.96370287284456\\
1.61480018945704	-2.20604842658807\\
1.35079090761119	-2.42763724664457\\
1.08156304935098	-2.62594429031886\\
0.811571843486864	-2.79954935129738\\
0.544778559945462	-2.94808887382429\\
0.284462885417311	-3.07210182987649\\
0.0331430358728775	-3.17281577583671\\
-0.207412600404753	-3.25191821988336\\
-0.436109895771939	-3.3113472100126\\
-0.652418494402153	-3.35312039216399\\
-0.856250380106214	-3.37920883099783\\
-1.04784395731508	-3.39145301224971\\
-1.22766287057359	-3.39151379982221\\
-1.39631320746139	-3.3808496940076\\
-1.55447908508147	-3.36071226228474\\
-1.70287462151121	-3.3321530445752\\
-1.84220946327544	-3.29603688523293\\
-1.97316492744097	-3.25305814882564\\
-2.09637808130158	-3.20375749061753\\
-2.21243149987578	-3.14853775640611\\
-2.32184688148563	-3.08767822001648\\
-2.42508109894911	-3.02134678958923\\
-2.52252359246908	-2.9496100841685\\
-2.61449426618703	-2.87244144934862\\
-2.70124124044346	-2.78972708358762\\
-2.78293794738014	-2.70127051464626\\
-2.85967915133165	-2.60679572088432\\
-2.9314755405021	-2.50594925269726\\
-2.9982465862832	-2.3983017905733\\
-3.05981141648412	-2.28334969254005\\
-3.11587751739229	-2.16051724954773\\
-3.16602719136474	-2.02916059682455\\
-3.20970188427344	-1.88857453477377\\
-3.24618480397113	-1.73800390071199\\
-3.27458273194635	-1.57666159427501\\
-3.29380864897059	-1.40375585733705\\
-3.30256781250384	-1.21852985551498\\
-3.29935127053315	-1.02031683366934\\
-3.28244242619718	-0.808613839943074\\
-3.24994397454638	-0.583175817221317\\
-3.19983384934379	-0.344129229116059\\
-3.13005892010776	-0.0920998288786608\\
-3.03867289103494	0.171657472809961\\
-2.92401889067576	0.445146795822683\\
-2.78494685465088	0.725565289186103\\
-2.62104180619464	1.00931428445458\\
-2.43282490921739	1.29211422453723\\
-2.22188069184517	1.56923392380142\\
-1.99086776908621	1.83581845968917\\
-1.74339015709075	2.08727071340673\\
-1.4837382877407	2.31962080554035\\
-1.21654221879537	2.52981570238502\\
-0.946400969228186	2.71588066109658\\
-0.677553065468151	2.87693743867862\\
-0.413635536840815	3.0130978833525\\
-0.15755103135502	3.12527387942904\\
0.088563304520537	3.21495096129171\\
0.323288483199085	3.28396596140827\\
0.545827235646421	3.33431531819545\\
0.755885505103046	3.36800638923301\\
0.953552195606826	3.38695306611053\\
1.13918986385114	3.39291030692128\\
};
\addplot [color=mycolor1, forget plot]
  table[row sep=crcr]{%
1.07681390015377	3.50387265912038\\
1.25290541920648	3.49833572423572\\
1.4189356767722	3.48252854212395\\
1.57549793994767	3.45751789326495\\
1.72320292935161	3.4241930275202\\
1.86265234412909	3.3832777999272\\
1.99442009920366	3.33534439184401\\
2.11903930354307	3.2808270371971\\
2.23699331814286	3.22003479545064\\
2.34870954353418	3.15316285038042\\
2.45455486234239	3.08030211046827\\
2.55483188912201	3.0014470836132\\
2.64977535661494	2.91650213026931\\
2.73954810062175	2.82528629261796\\
2.82423620396988	2.72753697483497\\
2.90384293447019	2.62291282871868\\
2.9782811746746	2.51099629514403\\
3.04736410730181	2.39129637879328\\
3.11079400754534	2.26325240417023\\
3.16814912553205	2.12623972620519\\
3.21886884942524	1.97957865614578\\
3.26223766078859	1.82254821170636\\
3.29736887515341	1.65440669150492\\
3.32318985127375	1.47442145935703\\
3.33843129116964	1.28191060678544\\
3.34162444246868	1.07629917335834\\
3.33111137687374	0.857192084217782\\
3.30507483070997	0.624464560729761\\
3.26159491273558	0.378368078115702\\
3.19873959952593	0.119645680832525\\
3.11469341897808	-0.15035532707082\\
3.00792316781258	-0.429591854598313\\
2.87737058282481	-0.715278695886092\\
2.7226505650479	-1.00391302290729\\
2.54422273820986	-1.29138984491412\\
2.34349842605664	-1.5732142596992\\
2.12284929698522	-1.8447947354008\\
1.88550002685379	-2.10177872423388\\
1.63531234685439	-2.34037630997864\\
1.37649370738826	-2.55761666979082\\
1.1132808549074	-2.75149747541878\\
0.849650718702709	-2.9210133765116\\
0.589098633806478	-3.06607653770328\\
0.334503372642803	-3.18736095127435\\
0.0880778652927617	-3.28610893254169\\
-0.148610152184288	-3.36393423857315\\
-0.374569942243917	-3.42264617084894\\
-0.589294217619091	-3.46410761740004\\
-0.792658164053592	-3.49013049208802\\
-0.984823260130143	-3.50240573644614\\
-1.16615279750645	-3.50246184449687\\
-1.337141908017	-3.49164495734264\\
-1.49836219357036	-3.4711140416334\\
-1.65041952382722	-3.44184577780922\\
-1.79392288646242	-3.40464505649268\\
-1.92946203233454	-3.36015815167484\\
-2.05759181094738	-3.30888659825559\\
-2.17882137624947	-3.25120053029595\\
-2.29360675984949	-3.18735076066673\\
-2.40234560498806	-3.11747924444821\\
-2.50537310652952	-3.04162781050253\\
-2.6029584036181	-2.95974520593235\\
-2.69530082563154	-2.87169260754258\\
-2.78252550646702	-2.77724783745605\\
-2.86467796709813	-2.67610859633044\\
-2.94171733334199	-2.5678951135311\\
-3.01350791825165	-2.45215272354561\\
-3.07980897269412	-2.3283550251702\\
-3.14026251404752	-2.19590847669295\\
-3.19437930828088	-2.05415953611136\\
-3.24152333948906	-1.90240577421399\\
-3.28089549603646	-1.73991276179689\\
-3.31151778247802	-1.56593892917875\\
-3.33222017760411	-1.37977094642282\\
-3.3416333263787	-1.18077234614903\\
-3.33819154649102	-0.968447894925771\\
-3.32015200737686	-0.742525306038447\\
-3.28563707691093	-0.50305389184007\\
-3.23270715373773	-0.250516304671474\\
-3.15946997428325	0.0140555549435519\\
-3.06422843948946	0.288975714522802\\
-2.94566174443023	0.571831968071082\\
-2.80302423696485	0.85947189989199\\
-2.63633487342317	1.14807102780864\\
-2.44652124165192	1.43329715793668\\
-2.23548085738443	1.71056661957679\\
-2.00603259452234	1.97536480215196\\
-1.76175240525437	2.22358302957913\\
-1.5067141483345	2.45181490629269\\
-1.24517891109999	2.65756275019291\\
-0.981286178864548	2.83932648752665\\
-0.718794577087978	2.9965751045877\\
-0.460902438757844	3.12962440062159\\
-0.210156918784256	3.23945763471365\\
0.0315577520644397	3.32752657542855\\
0.262972928864498	3.39556279593601\\
0.483352111847407	3.44541778923692\\
0.69239179193768	3.47893974872642\\
0.890121699398104	3.49788690805546\\
1.07681390015377	3.50387265912038\\
};
\addplot [color=mycolor1, forget plot]
  table[row sep=crcr]{%
1.02515979667224	3.60981291354574\\
1.20311562436717	3.6042131621784\\
1.37165196277825	3.58816339150138\\
1.53128009861078	3.56265918140955\\
1.68253149195655	3.52853058333861\\
1.8259333787484	3.48645225249679\\
1.96199093354743	3.43695506518422\\
2.09117436581869	3.38043785585983\\
2.21390955397894	3.31717842954101\\
2.33057106041103	3.24734337988316\\
2.44147658709197	3.17099650562936\\
2.54688211315672	3.08810580046505\\
2.64697710091582	2.99854912171739\\
2.74187926984612	2.90211874503844\\
2.83162852659825	2.79852510431886\\
2.91617971318818	2.68740011465374\\
2.99539390766968	2.56830059516605\\
3.06902809714852	2.44071246018251\\
3.13672316160073	2.30405654196441\\
3.19799028367838	2.15769715263519\\
3.25219616601802	2.00095478670175\\
3.29854783101757	1.8331246947889\\
3.33607833874827	1.65350338667567\\
3.36363552033777	1.46142537282874\\
3.37987679574304	1.25631249938791\\
3.38327427887542	1.03773787322616\\
3.37213551898953	0.805505339863061\\
3.34464607516443	0.559743452978278\\
3.29894015671415	0.301009592182757\\
3.23320408185208	0.0303953020543156\\
3.14581355627779	-0.25038151573407\\
3.03549928853415	-0.538917019771991\\
2.90152659503935	-0.8321166106038\\
2.74386511438148	-1.12626478789704\\
2.56331773962297	-1.41718043855402\\
2.36157729213811	-1.70045233319201\\
2.14118811785052	-1.97173013328136\\
1.90540730117711	-2.22702947852382\\
1.65798200631524	-2.46300249869556\\
1.40287823587041	-2.67713114916541\\
1.14400535857114	-2.8678184835532\\
0.88497745374461	-3.03437591453482\\
0.628939318653582	-3.17692474229225\\
0.378467674378691	-3.29624212085679\\
0.13554262520618	-3.39358379222262\\
-0.0984255763087338	-3.47051059170827\\
-0.322533207424746	-3.52873675093961\\
-0.536300907622331	-3.57000886379482\\
-0.739586635061441	-3.59601713677922\\
-0.932503172145304	-3.60833591349153\\
-1.11534548375721	-3.60838819588231\\
-1.2885300856312	-3.59742832529505\\
-1.45254649037625	-3.57653744089128\\
-1.60791959998377	-3.54662725568829\\
-1.75518136156198	-3.50844872441816\\
-1.89484986610153	-3.46260312808364\\
-2.0274141676177	-3.40955388545592\\
-2.15332330756317	-3.34963800690096\\
-2.27297826969436	-3.28307654969192\\
-2.38672582070896	-3.20998374826903\\
-2.49485339177604	-3.13037471143959\\
-2.5975843192724	-3.04417173168765\\
-2.69507289134802	-2.95120936519182\\
-2.78739874658233	-2.85123853588039\\
-2.87456025092164	-2.74393001033717\\
-2.95646655049722	-2.62887769751419\\
-3.03292807461189	-2.50560236111094\\
-3.10364536250223	-2.37355650455739\\
-3.16819623162121	-2.23213140760739\\
-3.22602152236636	-2.08066756335117\\
-3.27640997908704	-1.91847007885575\\
-3.31848329936319	-1.74483093760243\\
-3.35118304149935	-1.5590603234897\\
-3.37326194829048	-1.36052937479863\\
-3.38328330904302	-1.14872660980604\\
-3.37963314927423	-0.923329607521185\\
-3.36055108527246	-0.684292037527928\\
-3.32418620024484	-0.431943504613437\\
-3.26868367304024	-0.167095722866397\\
-3.19230536147705	0.108856576005162\\
-3.09358244533766	0.393857068754825\\
-2.97149044096019	0.685152865976423\\
-2.82562738894695	0.97932494000532\\
-2.65636726985971	1.27240021362405\\
-2.46495643298639	1.56004880299526\\
-2.25352466310871	1.83785155187066\\
-2.02499590669243	2.10160399929956\\
-1.7829041496629	2.34761020364682\\
-1.53114120525856	2.57291909951605\\
-1.27367773558674	2.77546850554358\\
-1.0143017090026	2.95412322218416\\
-0.756409657405247	3.10861616483928\\
-0.502870033834599	3.23941798981028\\
-0.255960938590626	3.34756760350115\\
-0.017371473825098	3.43449392828752\\
0.211750845111671	3.50185165385534\\
0.430725326573695	3.55138427531142\\
0.639252034058069	3.58481934966929\\
0.837326115959362	3.60379494983452\\
1.02515979667224	3.60981291354574\\
};
\addplot [color=mycolor1, forget plot]
  table[row sep=crcr]{%
0.9828277466891	3.70763537359447\\
1.16253075031462	3.70197719960511\\
1.33334009430001	3.68570766277502\\
1.49570695775239	3.65976271051159\\
1.65010344972799	3.62492141184049\\
1.79699993433113	3.58181474962233\\
1.9368480532624	3.53093576568061\\
2.07006805357902	3.47264986453593\\
2.19703921115515	3.40720453038006\\
2.31809233268834	3.33473804069562\\
2.43350349604434	3.25528699630026\\
2.5434883397133	3.16879265862485\\
2.64819633554898	3.07510621461667\\
2.74770457856682	2.97399319805948\\
2.8420107106544	2.86513740040777\\
2.93102467161595	2.74814471903087\\
3.01455905390315	2.62254752861117\\
3.09231794308403	2.487810332777\\
3.16388427545534	2.3433376651173\\
3.22870596337977	2.18848546303003\\
3.286081359338	2.02257742548568\\
3.33514508587561	1.84492815991116\\
3.37485588221517	1.65487516850558\\
3.40398892519859	1.45182182275065\\
3.42113604999295	1.2352932736457\\
3.42471832924838	1.00500652996698\\
3.41301635536507	0.760954447175441\\
3.38422393697969	0.5035008557295\\
3.33653022560203	0.233480401589032\\
3.26823289024617	-0.0477079186871799\\
3.17788032636966	-0.338029307421637\\
3.0644339748185	-0.634783043541202\\
2.92743353400122	-0.93463305884721\\
2.76714030521927	-1.233713376536\\
2.5846302238386	-1.52781119213057\\
2.38181130086192	-1.81261429556168\\
2.16135144700147	-2.08399295313932\\
1.92652001739265	-2.33827507327527\\
1.6809648359463	-2.57247228271275\\
1.4284597389176	-2.78442464350298\\
1.1726614366354	-2.97284981242987\\
0.916908237129788	-3.13730219312071\\
0.664080193666466	-3.2780628929517\\
0.416525582855244	-3.39598864294165\\
0.176046663487098	-3.49234732127893\\
-0.0560693645323572	-3.56866183959142\\
-0.278990371332361	-3.62657612494709\\
-0.492264951839194	-3.66774934602806\\
-0.695745042793501	-3.69377879807574\\
-0.889513035128932	-3.70614832570495\\
-1.0738175758965	-3.7061975177389\\
-1.24901973286528	-3.69510659111797\\
-1.41554952266051	-3.67389233770175\\
-1.57387183832527	-3.64341131302821\\
-1.72446036017558	-3.6043673252688\\
-1.86777791359634	-3.55732108663302\\
-2.00426180745468	-3.50270055590226\\
-2.13431284872163	-3.44081101890906\\
-2.25828692075098	-3.37184433929873\\
-2.37648819938399	-3.29588709096604\\
-2.48916324566855	-3.21292748404959\\
-2.59649535097355	-3.12286114391619\\
-2.69859862106026	-3.02549591924005\\
-2.79551137592754	-2.92055599967027\\
-2.88718852065041	-2.80768573144182\\
-2.97349262037157	-2.68645364415975\\
-3.05418350461285	-2.55635735543462\\
-3.12890635067281	-2.41683021112011\\
-3.19717837652471	-2.26725075241316\\
-3.25837453921522	-2.10695637406438\\
-3.31171301872151	-1.93526283367979\\
-3.35624180374542	-1.75149155094615\\
-3.39082841090712	-1.55500682278472\\
-3.41415566254215	-1.34526505202943\\
-3.4247274669697	-1.12187765939501\\
-3.42088954046595	-0.884688279919619\\
-3.40087069395281	-0.633862869992579\\
-3.36285022003324	-0.369988265853077\\
-3.30505544949928	-0.0941705558364476\\
-3.22589007815862	0.191880185460561\\
-3.12408805098337	0.485796593290265\\
-2.99888000213569	0.78455200019934\\
-2.85015098904599	1.08452724333672\\
-2.67856222643492	1.38165550846322\\
-2.48560896290965	1.67164033388676\\
-2.27359387819706	1.950224895958\\
-2.0455101495427	2.21347602806993\\
-1.80484699864818	2.45803975348327\\
-1.5553471104622	2.68132970671944\\
-1.30075416578548	2.88162457077831\\
-1.04458727672918	3.05807042243704\\
-0.789968913470217	3.21060202528641\\
-0.539518452389949	3.33980861790815\\
-0.295309740119573	3.44677296070807\\
-0.0588814470986692	3.53290880790546\\
0.168715467301757	3.59981463192038\\
0.386848938890716	3.6491533690256\\
0.595228759293788	3.68256119407468\\
0.793830724627843	3.70158370544942\\
0.9828277466891	3.70763537359447\\
};
\addplot [color=mycolor1]
  table[row sep=crcr]{%
0.948683298050514	3.79473319220206\\
1.12996891013208	3.78902241071535\\
1.3027838103604	3.77255915923418\\
1.46753552344581	3.74623053454947\\
1.62465224482371	3.71077288416293\\
1.77456152903207	3.66677970030574\\
1.91767403561802	3.61471073626391\\
2.05437110063153	3.55490128334521\\
2.18499505421831	3.4875709448019\\
2.30984136685525	3.41283153872394\\
2.42915185734506	3.33069397942336\\
2.54310832592638	3.24107414976038\\
2.65182608463458	3.14379790634541\\
2.75534694820212	3.03860547381077\\
2.8536313296431	2.92515559896292\\
2.9465491655584	2.80302996392973\\
3.0338694906698	2.67173851056894\\
3.11524860652315	2.53072651479627\\
3.19021696761039	2.3793844730251\\
3.25816516606007	2.21706211963439\\
3.31832976399147	2.0430881668871\\
3.36978023105031	1.85679760784185\\
3.41140891541445	1.65756857671014\\
3.44192680709927	1.44487070344488\\
3.45986879009691	1.21832646083672\\
3.46361298678702	0.977785965288651\\
3.45141941180798	0.723413822791541\\
3.4214930614047	0.455783738589644\\
3.37207522993717	0.175972765268488\\
3.30156372141992	-0.114357323644244\\
3.20865740947072	-0.412907981729085\\
3.0925136091427	-0.716739496073036\\
2.95289924526556	-1.02233055985344\\
2.79031115637498	-1.3257107709346\\
2.60603988731933	-1.6226624623078\\
2.40215712665582	-1.90897292279738\\
2.18141953804948	-2.18070442957845\\
1.94709823579504	-2.43444260737754\\
1.70275852491129	-2.66748659951055\\
1.45202370644265	-2.877956852001\\
1.19835703486066	-3.0648138241024\\
0.944888096380122	-3.22779787311511\\
0.694297500839398	-3.36731218002033\\
0.448761194715288	-3.48427490957016\\
0.209946190700892	-3.57996468520442\\
-0.0209555786883461	-3.65587741213202\\
-0.243169602253506	-3.71360522906444\\
-0.456269901983927	-3.7547419037936\\
-0.660108423391693	-3.78081428495334\\
-0.854748940791145	-3.79323662458319\\
-1.04040889805107	-3.79328335955738\\
-1.21741050351997	-3.78207578896275\\
-1.38614100991934	-3.76057854694004\\
-1.54702130890241	-3.72960250007616\\
-1.70048159286015	-3.6898114747465\\
-1.84694273429099	-3.64173092446581\\
-1.98680208954437	-3.58575723170233\\
-2.1204225680783	-3.52216679568137\\
-2.24812396922559	-3.45112440165093\\
-2.37017574651009	-3.37269062063008\\
-2.48679050050333	-3.2868281760587\\
-2.59811762050387	-3.19340735751618\\
-2.70423659456836	-3.09221068144532\\
-2.80514959232286	-2.98293711145668\\
-2.90077300437504	-2.86520627071399\\
-2.99092770791458	-2.73856321838213\\
-3.07532793559953	-2.60248453106578\\
-3.1535687736533	-2.45638663503084\\
-3.22511252958097	-2.29963757630341\\
-3.28927451889835	-2.13157368366421\\
-3.34520925490937	-1.95152284702656\\
-3.39189861348128	-1.7588363466938\\
-3.42814429814894	-1.5529312348605\\
-3.45256782598978	-1.33334504716437\\
-3.46362220116165	-1.0998039137315\\
-3.45962024522194	-0.852303713618117\\
-3.43878487541956	-0.591201557879526\\
-3.3993259814164	-0.317311504170883\\
-3.3395463824176	-0.0319942042762583\\
-3.25797518618734	0.262774139257094\\
-3.15352067565559	0.564371272065384\\
-3.02562738470585	0.869554764908886\\
-2.87441511977517	1.17455760801215\\
-2.70077402394131	1.47525676358082\\
-2.50639202484484	1.767403513355\\
-2.29370038309167	2.04688929885519\\
-2.06573813670639	2.31000991273335\\
-1.8259528645897	2.5536887433386\\
-1.57796799985389	2.77562773795799\\
-1.32535178123783	2.9743703475303\\
-1.07141885643386	3.14927865398809\\
-0.819084909373598	3.30044159903781\\
-0.570781691446472	3.42853925898355\\
-0.328428492520302	3.53468897189284\\
-0.093448740680559	3.62029468577783\\
0.13318251344032	3.68691393725669\\
0.350873927242831	3.73614982594868\\
0.55934713845833	3.76956970364108\\
0.758567824726602	3.78864856747597\\
0.948683298050513	3.79473319220206\\
};
\addlegendentry{Аппроксимации}

\addplot [color=mycolor2]
  table[row sep=crcr]{%
2.12132034355964	2.82842712474619\\
2.18582575000323	2.82640660874952\\
2.2459761483502	2.82067996094826\\
2.30316600770785	2.81153894842004\\
2.35833530419819	2.79908319427389\\
2.4121407651283	2.78328504568152\\
2.46505313939319	2.76402339892432\\
2.51741350921551	2.74110157112695\\
2.56946568278775	2.71425682483196\\
2.62137377470791	2.68316571665979\\
2.67322998558975	2.64744786912504\\
2.72505542741334	2.60667010622203\\
2.77679571750944	2.56035271147659\\
2.82831256449113	2.50797963523731\\
2.87937250576692	2.44901464186941\\
2.92963424895255	2.38292549191834\\
2.97863667364875	2.30921809960124\\
3.02579039162557	2.22748193169334\\
3.0703766721051	2.13744642550811\\
3.11155818595542	2.03904567139316\\
3.14840590189045	1.93248505458784\\
3.1799449834646	1.81829951655162\\
3.20521925970046	1.69738980361013\\
3.2233689191928	1.57102231442397\\
3.23371058284854	1.44078163681543\\
3.23580480244767	1.30847305653245\\
3.22949541157146	1.1759834831\\
3.21490900216393	1.04511946165368\\
3.19240995273442	0.917445713323391\\
3.16251366135309	0.794144173493542\\
3.12576385298257	0.675902054710668\\
3.08257536776515	0.562820516214273\\
3.03302813706062	0.454314775018543\\
2.97656426645917	0.348948817128694\\
2.91146921429886	0.244099783305823\\
2.83385680006128	0.135243682019644\\
2.73548220960228	0.0144130439331602\\
2.59871334341581	-0.133168683665202\\
2.38479638365891	-0.336180712113444\\
2.01057175466146	-0.648371525543042\\
1.34210041814469	-1.13791067535239\\
0.375352321166521	-1.75822148716338\\
-0.527448468220416	-2.26262801632552\\
-1.11365847778129	-2.5437869612038\\
-1.4567327767317	-2.68228003389134\\
-1.6691296256022	-2.75259678373195\\
-1.81428800439327	-2.79054335642282\\
-1.92320462744895	-2.81168897715777\\
-2.01132392805712	-2.82302365363059\\
-2.08680248126988	-2.82787464184343\\
-2.15422943553058	-2.82790871121207\\
-2.21634478307291	-2.8239827357917\\
-2.27487422572476	-2.8165264847195\\
-2.33095693947289	-2.80572465930185\\
-2.3853755403757	-2.79160756168836\\
-2.43868469976231	-2.77409782070174\\
-2.49128507004909	-2.75303495626884\\
-2.54346612600745	-2.7281884019911\\
-2.59543032955127	-2.69926454755406\\
-2.64730535029083	-2.66591103001887\\
-2.69914810041563	-2.62772046423944\\
-2.75094277189846	-2.58423541950896\\
-2.80259428647749	-2.53495641420315\\
-2.85391830194335	-2.47935483422118\\
-2.90462903903618	-2.41689283661657\\
-2.95432664759112	-2.34705230341292\\
-3.00248656709011	-2.26937453210355\\
-3.04845424185856	-2.18351130413452\\
-3.0914493816545	-2.08928598065547\\
-3.13058429431929	-1.98676020448289\\
-3.16490008683964	-1.87629788656397\\
-3.19342217916266	-1.75861429449935\\
-3.21523240499265	-1.63479579561046\\
-3.22954956680171	-1.50627700312196\\
-3.23580523135335	-1.37476797309193\\
-3.23369896357047	-1.2421341494958\\
-3.22321880671614	-1.11024302689198\\
-3.20461860896113	-0.980799507713277\\
-3.17835148049538	-0.85519274756288\\
-3.14496439092874	-0.734369602871579\\
-3.10495870881183	-0.618735254776386\\
-3.05861192744495	-0.508062722468019\\
-3.0057323937692	-0.401369642978784\\
-2.94527025004781	-0.296685462718974\\
-2.87460204254734	-0.190562899149911\\
-2.78805571026974	-0.0770301658026245\\
-2.67361421522119	0.0546866392302208\\
-2.50520482475694	0.225105040795738\\
-2.22541867988909	0.474087715816021\\
-1.7209533613293	0.868461990547897\\
-0.880186083660502	1.44500069250654\\
0.110323684998138	2.03883159526995\\
0.859259206598599	2.42737283484743\\
1.30741523004191	2.62511520505649\\
1.57426953542748	2.72304263504847\\
1.74766615441039	2.77435113512036\\
1.87207484559155	2.80265686669549\\
1.96924085114733	2.81832059141065\\
2.0502990590364	2.82612977760851\\
2.12132034355964	2.82842712474619\\
};
\addlegendentry{Сумма Минковского}

\end{axis}

\begin{axis}[%
width=0.798\linewidth,
height=0.597\linewidth,
at={(-0.104\linewidth,-0.066\linewidth)},
scale only axis,
xmin=0,
xmax=1,
ymin=0,
ymax=1,
axis line style={draw=none},
ticks=none,
axis x line*=bottom,
axis y line*=left,
legend style={legend cell align=left, align=left, draw=white!15!black}
]
\end{axis}
\end{tikzpicture}%
        \caption{Эллипсоидальные аппроксимации для 100 направлений.}
\end{figure}

%%%%%%%%%%%%%%%%%%%%%%%%%%%%%%%%%%%%%%%%%%%%%%%%%%%%%%%%%%%%%%%%%%%%%%%%%%%%%%%%
\clearpage
\section{Внутренняя оценка суммы эллипсоидов}

\begin{definition}
        \textit{Сингулярным разложением} матрицы $A \in \setR^{n \times m}$ называется представление матрицы в виде
$$
        A = V \mathit{\Sigma} U^*, 
$$
        где
$$
\begin{aligned}
&V \in \setR^{n \times n}\::\: V^* = V^{-1},\\
&U \in \setR^{m \times m}\::\: U^* = U^{-1},\\
&\mathit{\Sigma} = \mathrm{diag}\left(\sigma_1,\,\ldots,\,\sigma_{\min\{n,\,m\}}\right) \in \setR^{n \times m}\::\:\sigma_1 \geqslant \sigma_2 \geqslant \ldots \geqslant \sigma_{\min\{n,\,m\}}.
\end{aligned}
$$ 
\end{definition}

\begin{theorem}
        Сингулярное разложение
        $A = V \mathit{\Sigma} U^*$
        существует для любой комплексной матрицы $A$.
        Если матрица $A$ вещественная, то матрицы $V$, $\mathit{\Sigma}$ и $U$ также можно выбрать вещественными. 
\end{theorem}

\begin{theorem}
        Старшее сингулярное число $\sigma_1$ матрицы $A = V \mathit{\Sigma} U^*$ является её нормой.
\end{theorem}

\begin{definition}
        Назовём линейное преобразование $\mathcal{A}$ \textit{ортогональным}, если оно сохраняет скалярное произведение, то есть
$$
        \langle \mathcal{A}(x),\,\mathcal{A}(y)\rangle = \langle x,\,y\rangle.
$$
\end{definition}

\begin{theorem}\label{th:unitarnost}
        Необходимым и достаточным условием ортогональности линейного преобразования $\mathcal{A}$ в конечномерном пространстве является унитарность матрицы преобразования $A$, то есть
$$
        A^* = A^{-1}.
$$
\end{theorem}

\begin{assertion}
        Для произвольных векторов $a,\,b\in\setR^{n}$ таких, что $\|a\| = \|b\|$, существует матрица ортогонального преобразования, переводящего $a$ в $b$.
\end{assertion}
\begin{proof}
        
        Построим сингулярное разложение для векторов $a$ и $b$:
$$
        a = V_a \mathit{\Sigma}_a u_a,
        \qquad
        b = V_b \mathit{\Sigma}_b u_b,
$$        
причем $V_a,\,V_b \in \setR^{n \times n}$~--- унитарные матрицы, $u_a,\,u_b\in\{-1,\,1\}\in\setR^1$,
$$
        \mathit{\Sigma}_a = [\sigma_a,\,0,\,\ldots,\,0]\T\in\setR^{n \times 1},\quad
        \mathit{\Sigma}_b = [\sigma_b,\,0,\,\ldots,\,0]\T\in\setR^{n \times 1},\quad
        \sigma_a,\,\sigma_b > 0.
$$
Согласно Теореме~\ref{th:unitarnost} $\sigma_a = \sigma_b$.
Тогда преобразуем выражение для вектора $b$:
\begin{multline*}
        b
        =
        V_b \mathit{\Sigma}_b u_b
        =
        V_b (V_a\T V_a)\mathit{\Sigma}_b u_b
        =
        V_bV_a\T V_a \left( \mathit{\Sigma}_a \frac{\sigma_b}{\sigma_a} \right) \left( u_a\frac{u_b}{u_a} \right)
        =\\=
        V_bV_a\T \frac{\sigma_b\cdot u_b}{\sigma_a\cdot u_a}V_a\mathit{\Sigma}_au_a
        =
        \left(V_b V_a\T \frac{\sigma_b\cdot u_b}{\sigma_a\cdot u_a}\right)a.
\end{multline*}
Так как произведение унитарных матриц есть унитарная матрица, теорема доказана.

\end{proof}

\textbf{Следствие 1.} Далее под ортогональным преобразованием из вектора $a$ в вектор $b$ таких, что $\|a\| = \|b\|$, будем понимать
$$
        \mathrm{Orth}(a,\,b) = u_au_bV_bV_a\T.
$$

\begin{assertion}
        Для суммы Минковского эллипсоидов справедлива следующая оценка
$$
        \sum_{i=1}^n \Varepsilon(q_i,\,Q_i) = 
        \bigcup\limits_{\|l\|=1}\Varepsilon(q_-(l),\,Q_-(l)),
$$
        где
$$
        \begin{aligned}
q_-(l) &= \sum_{i=1}^n q_i,
\\
Q_-(l) &= Q_*\T(l) Q_*(l),
\quad
Q_*(l) = \sum_{i=1}^{n}S_i(l)Q_i^{\nicefrac12},
\\
S_i(l) &= \mathrm{Orth}(Q_i^{\nicefrac12}l,\,\lambda_iQ_1^{\nicefrac12}l),
\quad
\lambda_i = \frac{\langle l,\,Q_il\rangle^{\nicefrac12}}{\langle l,\,Q_1l\rangle^{\nicefrac12}}.
        \end{aligned}
$$
\end{assertion}

\begin{proof}

Будем доказывать для случая $q_i = 0$, $i=\overline{1,\,n}$.
Случай с произвольными центрами~--- аналогично.

Итак рассмотрим эллипсоид $\Varepsilon_- = \Varepsilon(0,\,Q_-)$, $Q_- = Q_*\T Q_*$,
$$
        Q_* = \sum_{i=1}^n S_i Q_i^{\nicefrac12},
$$
где $S_i$~--- некоторые унитарные матрицы. Распишем квадрат опорной функции этого эллипсоида:
\begin{multline*}
        \rho^2(l\,|\,\Varepsilon_-)
        =
        \langle l,\,Q_-l \rangle
        =
        \langle Q_*l,\,Q_*l \rangle
        =
        \sum_{i=1}^n \langle l,\,Q_il \rangle
        +
        \sum_{i \neq j} \left \langle
        S_i Q_i^{\nicefrac12}l,\, S_j Q_j^{\nicefrac12}l
        \right\rangle
        \leqslant\\\leqslant
        \{
        \mbox{Неравенство Коши--Буняковского}
        \}
        \leqslant\\\leqslant
        \sum_{i=1}^n \langle l,\,Q_il \rangle
        +
        \sum_{i \neq j}
        \langle l,\,Q_il \rangle^{\nicefrac12}
        \langle l,\,Q_jl \rangle^{\nicefrac12}
        =
        \left(
        \sum_{i=1}^n \langle l,\,Q_il \rangle^{\nicefrac12}
        \right)^2
        =
        \rho^2\left(
        l\left|
        \sum_{i=1}^n \Varepsilon(q_i,\,Q_i)
        \right.
        \right).
\end{multline*}
Таким образом, получили, что $\Varepsilon_-\subseteq\sum_{i=1}^n\Varepsilon(q_i,\,Q_i)$.

Заметим, что равенство в последней формуле при фиксированном направлении $l \neq 0$ достигается при
$$
        S_iQ_i^{\nicefrac12}l = \lambda_i S_1Q_1^{\nicefrac12}l, 
$$
где $\lambda_i$~--- произвольные неотрицательные константы. Если положить $S_1 = I$, а $\lambda_i$ выбирать, исходя из условий нормировки ($\|Q_i^{\nicefrac12}l\| = \|\lambda_iQ_1^{\nicefrac12}l\|$):
$$
        \lambda_i = \frac{\langle l,\,Q_il\rangle^{\nicefrac12}}{\langle l,\,Q_1l\rangle^{\nicefrac12}},
$$
то получим утверждение теоремы.

\end{proof}
\vfill
\begin{figure}[h]

        \centering
        % This file was created by matlab2tikz.
%
%The latest updates can be retrieved from
%  http://www.mathworks.com/matlabcentral/fileexchange/22022-matlab2tikz-matlab2tikz
%where you can also make suggestions and rate matlab2tikz.
%
\definecolor{mycolor1}{rgb}{0.00000,0.44700,0.74100}%
\definecolor{mycolor2}{rgb}{0.85000,0.32500,0.09800}%
%
\begin{tikzpicture}

\begin{axis}[%
width=0.618\linewidth,
height=0.487\linewidth,
at={(0\linewidth,0\linewidth)},
scale only axis,
xmin=-4,
xmax=4,
xlabel style={font=\color{white!15!black}},
xlabel={$x_1$},
ymin=-3,
ymax=3,
ylabel style={font=\color{white!15!black}},
ylabel={$x_2$},
axis background/.style={fill=white},
axis x line*=bottom,
axis y line*=left,
xmajorgrids,
ymajorgrids,
legend style={at={(0.03,0.97)}, anchor=north west, legend cell align=left, align=left, draw=white!15!black}
]
\addplot [color=mycolor1, forget plot]
  table[row sep=crcr]{%
2.36712178368749	2.74633057020532\\
2.41188305682678	2.74494651398331\\
2.45036041459025	2.74130089565813\\
2.48383536069915	2.73596692388682\\
2.51327948444918	2.72933446739078\\
2.53943919311047	2.72166753585199\\
2.56289492242467	2.71314181506891\\
2.58410302011837	2.70386940753685\\
2.60342571833248	2.69391520541791\\
2.62115280082523	2.68330766515176\\
2.63751738560017	2.67204572812381\\
2.65270745907852	2.66010298488441\\
2.66687427014079	2.64742976011011\\
2.68013832873481	2.63395351039579\\
2.69259349381253	2.61957772004863\\
2.70430943819101	2.6041793136995\\
2.71533261385189	2.58760445132293\\
2.72568568587575	2.56966240776041\\
2.73536523285703	2.55011704085516\\
2.74433729805113	2.52867508999008\\
2.75253007967915	2.50497017880853\\
2.75982261143016	2.47854086090236\\
2.76602760995085	2.44880025137829\\
2.77086559717422	2.41499358451471\\
2.77392567575526	2.37613820122431\\
2.77460549370625	2.33093765236215\\
2.77201821305984	2.27765729690984\\
2.7648464218491	2.21394231571241\\
2.7511098920125	2.13654986655322\\
2.72779317877431	2.04095569483249\\
2.69024836192134	1.92078707468285\\
2.6312539409082	1.7670493095065\\
2.53961413480675	1.56721869756115\\
2.39837999194539	1.30464987086518\\
2.18360687141246	0.959750006540987\\
1.8667826621648	0.516209188824812\\
1.42701784641989	-0.0238781630653391\\
0.87502831245873	-0.620875059217897\\
0.268936641059173	-1.19901922110702\\
-0.309576813214938	-1.68530797756881\\
-0.802100907580029	-2.04896030059078\\
-1.19206161730571	-2.300440537754\\
-1.49041992551985	-2.46708460024485\\
-1.7168759123608	-2.57532489307401\\
-1.88995458136084	-2.64492059461505\\
-2.02413470299114	-2.68919932969544\\
-2.12995725581116	-2.71680265831965\\
-2.21490837909904	-2.73327951527395\\
-2.284286999297	-2.74220891806166\\
-2.34187032729419	-2.74592425079378\\
-2.39038313346388	-2.74596637420131\\
-2.4318186528206	-2.74336542452957\\
-2.46765634360889	-2.73881708355027\\
-2.49901014208278	-2.7327940719517\\
-2.52673028332805	-2.72561759993438\\
-2.551474024202	-2.71750375208859\\
-2.57375536742722	-2.70859393338899\\
-2.5939804436843	-2.69897499500867\\
-2.61247296631128	-2.688692537203\\
-2.6294927100655	-2.67775958603992\\
-2.64524900273341	-2.66616202895009\\
-2.6599105764423	-2.65386167486455\\
-2.67361268902278	-2.64079746083638\\
-2.68646212009346	-2.62688508668123\\
-2.69854042224194	-2.61201517643097\\
-2.70990563030842	-2.59604990865956\\
-2.72059247497691	-2.57881790224895\\
-2.73061098716967	-2.5601069666803\\
-2.73994319179238	-2.53965410007404\\
-2.74853734053432	-2.51713180921804\\
-2.75629877602211	-2.49212938365471\\
-2.76307597883326	-2.46412710489964\\
-2.76863950239953	-2.43246039507853\\
-2.77265014411951	-2.39626942454063\\
-2.7746104875259	-2.35442742359956\\
-2.77379029091263	-2.3054374557388\\
-2.76911009981251	-2.24728211523106\\
-2.75895730729457	-2.17720283029247\\
-2.74089227231279	-2.0913749275097\\
-2.71117633487659	-1.98443346302753\\
-2.66401914162082	-1.84880387542276\\
-2.59041768625865	-1.67383876852635\\
-2.47652724106683	-1.44496938777841\\
-2.30193857203366	-1.14371563581315\\
-2.0396739695219	-0.750852651889909\\
-1.66265564107253	-0.256803878437201\\
-1.16248546087895	0.319450601512933\\
-0.573614265552006	0.917547528939285\\
0.0284025713818862	1.45675943888985\\
0.5685251002902	1.88250662286626\\
1.00960568842464	2.18713123722861\\
1.35150239782658	2.39258093926768\\
1.61139397123953	2.52710219556684\\
1.80908182046774	2.61398894502504\\
1.96116532773652	2.66959557253549\\
2.08006223005889	2.7046815507453\\
2.17466815864962	2.726172144186\\
2.25127885929965	2.73851896753466\\
2.3143624731471	2.74460741055068\\
2.36712178368749	2.74633057020532\\
};
\addplot [color=mycolor1, forget plot]
  table[row sep=crcr]{%
2.10959639276541	2.82833170033868\\
2.13005277719103	2.82769680200034\\
2.14802287471214	2.82599213309054\\
2.16399176571052	2.82344580125796\\
2.17833205481083	2.82021393645668\\
2.19133411253028	2.81640177024618\\
2.20322718958303	2.81207744870569\\
2.21419439157868	2.80728111824648\\
2.22438344129317	2.80203084312009\\
2.23391449132048	2.79632631929519\\
2.24288582434271	2.79015097768998\\
2.25137799857592	2.78347282702145\\
2.25945680590028	2.7762442174586\\
2.26717527446035	2.7684005761966\\
2.27457484262639	2.75985805103207\\
2.28168573976336	2.75050987848557\\
2.28852651598563	2.7402211492275\\
2.29510255122665	2.72882145096567\\
2.30140322103382	2.71609459233055\\
2.30739716790373	2.70176419574736\\
2.31302476509234	2.68547330303474\\
2.31818626415068	2.66675511304507\\
2.32272310676726	2.64499030425581\\
2.32638812037928	2.61934362489206\\
2.32879716902256	2.58866772920285\\
2.32934906157394	2.5513540908222\\
2.32708969033154	2.5050964728495\\
2.32047561983038	2.44650691445407\\
2.3069520645374	2.37047913665329\\
2.28218252151097	2.26911845209114\\
2.23862573128952	2.12995002321369\\
2.16294815306376	1.93306507749366\\
2.03172785596982	1.64737025756866\\
1.80647115907523	1.22912474961823\\
1.43674425455007	0.635722130761753\\
0.895415675386827	-0.122678493566976\\
0.24506740533721	-0.923290250180105\\
-0.373302970705661	-1.59439810765956\\
-0.861342602045722	-2.06153432057587\\
-1.21051470427877	-2.35582021146555\\
-1.45350302607409	-2.53553747281427\\
-1.62427912108232	-2.6457756283534\\
-1.74735161713476	-2.71454779213463\\
-1.83866892421111	-2.75820072395713\\
-1.90838558633582	-2.78623139232265\\
-1.96302614690974	-2.80425772800326\\
-2.00687082453237	-2.81568963402831\\
-2.04279713811814	-2.82265357129101\\
-2.07278812116456	-2.82650998992637\\
-2.09824304490626	-2.82814934742327\\
-2.12017116470066	-2.82816582669431\\
-2.13931522377064	-2.826961933092\\
-2.15623195525253	-2.82481302044249\\
-2.17134569327871	-2.82190798040682\\
-2.18498480181322	-2.81837538766676\\
-2.19740689106534	-2.81430054547294\\
-2.20881656655742	-2.80973669054728\\
-2.21937810527049	-2.8047123437088\\
-2.22922461613086	-2.79923603208632\\
-2.23846471192486	-2.79329914096155\\
-2.24718737595216	-2.78687735484354\\
-2.25546547731453	-2.7799309467999\\
-2.26335822936359	-2.77240402898462\\
-2.27091276797016	-2.76422275734835\\
-2.27816493006503	-2.75529236858525\\
-2.28513922246038	-2.74549279836288\\
-2.2918478707394	-2.73467246525985\\
-2.2982887089987	-2.72263957559321\\
-2.30444148590314	-2.70914996681804\\
-2.31026187654082	-2.69388999154343\\
-2.31567202720427	-2.67645213420482\\
-2.32054568700573	-2.65629974947181\\
-2.32468465040066	-2.63271516862803\\
-2.32778088753628	-2.60472181958332\\
-2.32935449079792	-2.57096482990594\\
-2.32864968427513	-2.52952378807117\\
-2.32445618534273	-2.47761220956851\\
-2.31479431693115	-2.41108425761014\\
-2.29634617602986	-2.32361020748856\\
-2.26340897464661	-2.20528850710705\\
-2.20596668980778	-2.04035887822781\\
-2.10628907021382	-1.80379023894369\\
-1.93390602004408	-1.45787754842354\\
-1.64266626013909	-0.955835256982624\\
-1.18636447694275	-0.272301910216555\\
-0.575627521932152	0.529268583185396\\
0.0765700150810036	1.28296729173816\\
0.635827895370091	1.85302389135356\\
1.05156023208183	2.22653834661497\\
1.34296588926262	2.45673552212428\\
1.54615322639242	2.59724775265813\\
1.69061445079945	2.68411603245696\\
1.79623883340552	2.73880286760147\\
1.87575398878987	2.77375114085251\\
1.93728218443528	2.79624367598011\\
1.98609242944908	2.81064247686136\\
2.02568323586637	2.81963136032678\\
2.05843611040654	2.82490602040426\\
2.08601207953629	2.8275641952934\\
2.10959639276541	2.82833170033868\\
};
\addplot [color=mycolor1, forget plot]
  table[row sep=crcr]{%
-0.751180563789503	1.29301100092275\\
-0.644509272490707	1.2895621961483\\
-0.525853075685341	1.27816235586954\\
-0.394733067383687	1.25710640312682\\
-0.251255417385498	1.22462204940121\\
-0.0963657956622014	1.17906474936042\\
0.0679313764001297	1.11919457949777\\
0.238479475719274	1.04449792159326\\
0.411096645609142	0.95547160953947\\
0.580968788823447	0.85375980073978\\
0.743253360925444	0.742056725339432\\
0.893734219404683	0.623769327860435\\
1.02933068138786	0.502533729359883\\
1.14832654684568	0.381735283717928\\
1.25030593415255	0.264160768651649\\
1.33588484834454	0.151837028466942\\
1.40636299856342	0.046035698606773\\
1.46339657703116	-0.0526160638365817\\
1.50874510492206	-0.143981414805869\\
1.54410401747092	-0.228260871982706\\
1.57101084154411	-0.305863800112672\\
1.59080459914876	-0.37731143443009\\
1.60461898267407	-0.443169442531953\\
1.61339434135726	-0.50400430761343\\
1.6178983941152	-0.560357233657007\\
1.61874953068363	-0.612730111534175\\
1.61643932306818	-0.661579338697342\\
1.6113526202479	-0.707314476704708\\
1.60378462375957	-0.750299690229399\\
1.59395489259106	-0.790856617640623\\
1.58201848840448	-0.829267814089403\\
1.56807456975614	-0.865780234303465\\
1.55217275313418	-0.900608430316382\\
1.53431752429447	-0.933937265273067\\
1.51447093083552	-0.965924014217765\\
1.49255372986368	-0.996699753941291\\
1.4684451102011	-1.02636994778256\\
1.44198106109596	-1.05501411446045\\
1.41295142229282	-1.08268443616728\\
1.38109562819774	-1.10940311195514\\
1.34609715924291	-1.13515819874194\\
1.30757674909159	-1.15989760548665\\
1.26508448786515	-1.18352082038956\\
1.2180911421837	-1.20586786686862\\
1.16597933286764	-1.22670492454442\\
1.10803574356846	-1.24570606215447\\
1.04344637620195	-1.26243069409395\\
0.971298136519321	-1.27629683490271\\
0.890591822483024	-1.2865512138292\\
0.800273888148872	-1.29223914571393\\
0.699296856787975	-1.29218010554772\\
0.586720020892429	-1.28495946080335\\
0.461861094092619	-1.26895243972215\\
0.3245024552969	-1.24240144182672\\
0.175138600776365	-1.2035682010117\\
0.0152226545865632	-1.15097166888395\\
-0.152664711632565	-1.08369478588514\\
-0.324821834412255	-1.00170023256538\\
-0.496685757869856	-0.90605470141019\\
-0.663344832918517	-0.798955642596777\\
-0.820192630077424	-0.683508161867404\\
-0.963532782997699	-0.563297645942172\\
-1.09095770410823	-0.441888842741164\\
-1.20142500062305	-0.322399917701083\\
-1.29507612370551	-0.207246320149147\\
-1.37291296977355	-0.0980684877283813\\
-1.43645044441771	0.0042020468738688\\
-1.48742259418976	0.0992032593483852\\
-1.52757278250983	0.186984897840184\\
-1.55852539637732	0.267866024583987\\
-1.58172126356109	0.342322104544315\\
-1.59839612803496	0.410903378933661\\
-1.60958478267695	0.474180138513564\\
-1.61613840476429	0.532708621189078\\
-1.61874713092658	0.587011554073861\\
-1.61796325588701	0.637568493346236\\
-1.61422266111825	0.684812373734955\\
-1.60786343226419	0.729129764048409\\
-1.59914138437268	0.770863155573305\\
-1.58824260169102	0.810314202630113\\
-1.57529326680601	0.847747236816572\\
-1.56036710004137	0.883392638422727\\
-1.54349071326817	0.917449811620675\\
-1.52464713656047	0.950089605203132\\
-1.50377772010479	0.981456069358695\\
-1.4807825573749	1.0116674553813\\
-1.45551952420259	1.04081635809509\\
-1.42780198561346	1.06896887513935\\
-1.39739519197381	1.09616261560587\\
-1.3640113739163	1.12240333396511\\
-1.32730356145226	1.14765989469051\\
-1.28685821295664	1.17185719094692\\
-1.24218687070083	1.19486655389881\\
-1.19271730234616	1.21649311338865\\
-1.13778500336415	1.23645953866669\\
-1.07662660918946	1.2543856619263\\
-1.00837780643194	1.26976377842146\\
-0.932079852456276	1.28193010776828\\
-0.846700871404774	1.2900342666388\\
-0.751180563789503	1.29301100092275\\
};
\addplot [color=mycolor1, forget plot]
  table[row sep=crcr]{%
1.7832981678353	1.36546589275169\\
2.05592568115162	1.3571744194569\\
2.26714164491445	1.33728673652804\\
2.43111853478521	1.31126156725633\\
2.55943271347477	1.28244136021363\\
2.66093354729971	1.25276022517939\\
2.74218962233567	1.22327936216145\\
2.80802606279813	1.19453958765605\\
2.86198660051068	1.16677896496198\\
2.90668763683666	1.14006289696381\\
2.94407647731309	1.11436054257093\\
2.97561575220362	1.08958916115493\\
3.00241420566575	1.06563949952862\\
3.02531948618789	1.04239000684265\\
3.04498425031883	1.0197144567338\\
3.06191351815595	0.997485660984366\\
3.07649877708056	0.975576844041151\\
3.08904261974966	0.95386158977852\\
3.09977651875239	0.932212879917087\\
3.10887352086999	0.910501505895431\\
3.11645707010401	0.888593987389955\\
3.12260675718842	0.866350032379446\\
3.12736148556441	0.843619502128469\\
3.13072029732992	0.820238784686037\\
3.13264088455171	0.796026421968093\\
3.13303559280109	0.770777769825975\\
3.13176447555128	0.744258389841893\\
3.12862464535523	0.716195767549336\\
3.12333474415734	0.686268814798392\\
3.1155127546208	0.654094433393716\\
3.10464449980951	0.619210182674226\\
3.09003888625681	0.58105180074941\\
3.07076402681151	0.538923992473511\\
3.04555554798029	0.49196258500875\\
3.01268429798848	0.439086064580252\\
2.96976506994043	0.37893515870206\\
2.91348118717393	0.309801736731583\\
2.83919435906196	0.229555616004797\\
2.74041386938704	0.135595692071535\\
2.60813878932407	0.024891213502135\\
2.43022180279671	-0.105743493774361\\
2.19124961386915	-0.258874318262122\\
1.87411209379186	-0.434804268222937\\
1.46518116380506	-0.628916844714506\\
0.964245059921326	-0.829016773640375\\
0.395207777190676	-1.01569772268623\\
-0.193748734909378	-1.16861428361279\\
-0.747622298683287	-1.2757260230298\\
-1.22738617820585	-1.33743352552652\\
-1.61919514519511	-1.36280865849501\\
-1.92824504599151	-1.36321391987884\\
-2.16827266488954	-1.34828035900592\\
-2.35424440438222	-1.32479177202571\\
-2.49912053823983	-1.29705421238368\\
-2.61307727724595	-1.26762654584206\\
-2.7037556868631	-1.23795184232903\\
-2.77678844889193	-1.20879671834807\\
-2.83630900148409	-1.18052984170452\\
-2.88535963560319	-1.15329067060285\\
-2.92619505892335	-1.12708932719676\\
-2.96050096090471	-1.10186489456276\\
-2.98954936666697	-1.07751903751993\\
-3.01430876229406	-1.0539350695173\\
-3.03552236243188	-1.03098844046375\\
-3.05376401994576	-1.00855215247121\\
-3.06947838995699	-0.98649915687812\\
-3.08300991163942	-0.964702930102648\\
-3.09462374740807	-0.943036918843075\\
-3.10452083500761	-0.921373240822967\\
-3.11284852374633	-0.899580840330464\\
-3.11970778215795	-0.877523178040282\\
-3.12515761091966	-0.855055451989272\\
-3.12921702232672	-0.832021282451107\\
-3.13186471906918	-0.808248735438496\\
-3.13303639024817	-0.783545498449621\\
-3.13261931307637	-0.757692949906779\\
-3.13044367314647	-0.730438772495175\\
-3.1262696548114	-0.701487641376271\\
-3.11976885109067	-0.670489360837328\\
-3.1104978196536	-0.63702361653161\\
-3.09786054958415	-0.600580246600419\\
-3.08105502897536	-0.560533616143727\\
-3.05899676892009	-0.516109340945298\\
-3.03020872281921	-0.466341368002129\\
-2.99266220849364	-0.410017606742597\\
-2.94354713401342	-0.345613712643242\\
-2.87894316526232	-0.271219097624817\\
-2.7933613052015	-0.18447081327201\\
-2.67914274745981	-0.0825377845677899\\
-2.5257783825343	0.0377458731415321\\
-2.31943501649626	0.179398198627669\\
-2.04347928473513	0.34412630878845\\
-1.68159158956786	0.530151902243493\\
-1.22537153655025	0.729304947360203\\
-0.685671303026337	0.925396871033157\\
-0.0996443290375696	1.09741544503237\\
0.478086993696042	1.22814867876394\\
0.998202340115934	1.3118123673045\\
1.43427748430264	1.35394217752062\\
1.78329816783529	1.36546589275169\\
};
\addplot [color=mycolor1, forget plot]
  table[row sep=crcr]{%
2.35571328277463	2.22100511820195\\
2.47826782286047	2.21724110275804\\
2.57945563798566	2.2076760066786\\
2.66391338264297	2.19423739652515\\
2.73516709681387	2.17820355908947\\
2.79590243487287	2.16041750763504\\
2.84817524353824	2.1414300272445\\
2.89356994912398	2.12159441970805\\
2.93331677967345	2.10112910266619\\
2.96837799604573	2.08015897733583\\
2.99951128269362	2.05874275651048\\
3.02731643733883	2.03689093924234\\
3.05226984512698	2.01457747210173\\
3.07474996249443	1.99174706294535\\
3.09505610683312	1.96831941038495\\
3.11342217070063	1.94419114633847\\
3.13002638320828	1.91923597242133\\
3.14499787010008	1.89330324619182\\
3.1584204744478	1.86621510225032\\
3.17033405755072	1.83776204934554\\
3.18073327556166	1.8076968479169\\
3.18956359405484	1.77572632588045\\
3.19671403015796	1.74150061739947\\
3.20200576274209	1.70459909187564\\
3.20517527509796	1.66451195769695\\
3.20585002004533	1.62061615374963\\
3.20351362162625	1.57214365762075\\
3.19745620422277	1.51813972897166\\
3.1867033712263	1.45740789312791\\
3.16991440413369	1.38843778139177\\
3.14523622615357	1.30931167074949\\
3.11009471123801	1.2175867111612\\
3.06090031723225	1.11015476668775\\
2.9926453648678	0.983095758531635\\
2.89838838626637	0.831573980390453\\
2.76868761057062	0.649898587637061\\
2.59122233945366	0.432001663954325\\
2.35121103645622	0.172770931598334\\
2.03377775386478	-0.129248435576604\\
1.62959742771365	-0.46779854288743\\
1.14346100946033	-0.825320879963705\\
0.600821886747986	-1.17394011308631\\
0.0442409056935615	-1.4838164622843\\
-0.48163880163311	-1.73456167138362\\
-0.945768934604242	-1.92088020146989\\
-1.33603704925895	-2.04954181981907\\
-1.65480624610948	-2.13266138371325\\
-1.91166875393097	-2.18249310676955\\
-2.11807157549804	-2.2090843405378\\
-2.28459218053074	-2.21985724993794\\
-2.42000314217803	-2.220001836486\\
-2.53121317335989	-2.21304488666874\\
-2.62353282613246	-2.2013486995232\\
-2.70100594205337	-2.18648395128973\\
-2.76670839564354	-2.16948958718696\\
-2.82298812770595	-2.15104809534126\\
-2.87164830808901	-2.13160199457462\\
-2.91408376440492	-2.11143085198241\\
-2.95138155578939	-2.09070216769097\\
-2.98439489737335	-2.06950500443297\\
-3.01379754605468	-2.04787217232614\\
-3.0401239091731	-2.02579474465893\\
-3.06379868525868	-2.00323135043708\\
-3.08515876112665	-1.98011382155867\\
-3.10446929599717	-1.95635020164888\\
-3.12193534452632	-1.93182574111754\\
-3.13770994317351	-1.90640223807004\\
-3.15189925874524	-1.87991589072489\\
-3.16456513585618	-1.85217367237726\\
-3.17572515028752	-1.82294810191056\\
-3.18535005003472	-1.79197014323174\\
-3.19335821697194	-1.75891980928172\\
-3.19960647651815	-1.7234138533874\\
-3.20387617749746	-1.68498968341194\\
-3.20585289974103	-1.64308431012539\\
-3.20509733712814	-1.59700671612584\\
-3.20100372563178	-1.54590148473199\\
-3.19274046786317	-1.48870085916713\\
-3.17916512589608	-1.4240616742884\\
-3.15870247919262	-1.35028304642604\\
-3.12916978909147	-1.26520095598012\\
-3.08752833431522	-1.16605847020069\\
-3.02953722802248	-1.04935888480526\\
-2.9492923192467	-0.910731054901148\\
-2.83866985696159	-0.744886013906675\\
-2.68680635565555	-0.545843632021727\\
-2.48001009197639	-0.307772295691794\\
-2.20297704173767	-0.0269519369913502\\
-1.84266886104903	0.294767101014322\\
-1.39569223068177	0.645751374552216\\
-0.876809173496224	1.0027034275221\\
-0.321315333824451	1.33532954261279\\
0.224920657710854	1.61726471950061\\
0.722602501490466	1.83555657428617\\
1.15023955592394	1.99171058627205\\
1.50385186179304	2.09599160249212\\
1.79026056897879	2.16104741003003\\
2.02047904511769	2.19817110296771\\
2.20572333459932	2.21607809894601\\
2.35571328277463	2.22100511820195\\
};
\addplot [color=mycolor1, forget plot]
  table[row sep=crcr]{%
2.41112293345451	2.67293497202217\\
2.46888392391395	2.67115131797152\\
2.51815038737103	2.6664855191518\\
2.56067745902692	2.65971096292733\\
2.59779343839778	2.65135197568606\\
2.6305157273557	2.64176311344075\\
2.65963253350677	2.63118102002161\\
2.68576094455292	2.61975863731399\\
2.70938859486226	2.60758789630244\\
2.73090382849039	2.59471475314211\\
2.75061769692551	2.58114902195038\\
2.76878007240405	2.56687056330675\\
2.78559143844458	2.55183281196918\\
2.8012114227226	2.53596424504605\\
2.81576478577869	2.51916812567583\\
2.82934531927977	2.50132065759491\\
2.84201790198104	2.48226751893041\\
2.85381878116153	2.46181858333906\\
2.8647539665374	2.43974046017921\\
2.87479541554687	2.41574626808028\\
2.8838744197952	2.3894817673208\\
2.89187122478307	2.36050657363737\\
2.89859935644222	2.32826859854975\\
2.90378227390621	2.29206901942597\\
2.90701863342752	2.2510138451484\\
2.90773033309651	2.20394632395823\\
2.90508412791903	2.14935179353612\\
2.89787220142401	2.08522283788636\\
2.88432854807521	2.00886770995555\\
2.86184503952644	1.916639659751\\
2.82653307728443	1.80356260538734\\
2.77255796402825	1.66284105249546\\
2.69117522918125	1.4853017815288\\
2.56949523498214	1.25900114520614\\
2.38939363438971	0.969699040009248\\
2.12804201406743	0.603783618187781\\
1.76333966100237	0.155977290279563\\
1.28774343536766	-0.358089056184036\\
0.725951136675671	-0.893477459637415\\
0.136840561367889	-1.38815401247061\\
-0.41314368327355	-1.79385084762324\\
-0.881710561305587	-2.09580917159059\\
-1.25843603279283	-2.30612464980497\\
-1.55288592047271	-2.44682454430528\\
-1.7813060775171	-2.53866033056596\\
-1.9593657775584	-2.59741700772015\\
-2.09975332575718	-2.63403834464104\\
-2.21203071942399	-2.65581837430199\\
-2.3032006562605	-2.66755558712192\\
-2.37835240081262	-2.67240733578158\\
-2.44119571607529	-2.67246443748957\\
-2.49445761919681	-2.66912333754563\\
-2.54016488193194	-2.66332432155596\\
-2.57984205540937	-2.6557040973601\\
-2.61464993480629	-2.64669419424165\\
-2.6454827915952	-2.63658501085421\\
-2.67303717734763	-2.62556789562519\\
-2.69786105735113	-2.61376299788296\\
-2.72038922461991	-2.60123775195561\\
-2.74096903869299	-2.58801907026067\\
-2.75987924735955	-2.57410120073673\\
-2.77734377856985	-2.55945048935761\\
-2.793541793656	-2.54400782159856\\
-2.80861487682785	-2.52768920006705\\
-2.82267193569293	-2.51038468758181\\
-2.83579215890869	-2.491955765212\\
-2.84802618740108	-2.47223099410424\\
-2.85939547818274	-2.45099970448246\\
-2.86988964901186	-2.42800324168044\\
-2.87946135839992	-2.40292305034405\\
-2.8880179592727	-2.37536453831611\\
-2.89540870763175	-2.34483518069009\\
-2.90140561910968	-2.31071462828254\\
-2.90567500103259	-2.27221356472263\\
-2.90773500983351	-2.22831655519435\\
-2.90689191136059	-2.17770193064862\\
-2.90214344552749	-2.11862858948262\\
-2.8920308887691	-2.04877526035613\\
-2.87441080959479	-1.96501245936017\\
-2.84610196161767	-1.86308285568593\\
-2.8023432419264	-1.73716857524885\\
-2.7359855348757	-1.5793538010083\\
-2.63637493654125	-1.37909972675918\\
-2.48809271214244	-1.12315205734959\\
-2.27038916682889	-0.796974956580139\\
-1.95964380346866	-0.38978904595946\\
-1.53877736296124	0.0949035646598871\\
-1.01455397500912	0.626922435424875\\
-0.430388009563244	1.14961064156151\\
0.146485075591148	1.60386358275269\\
0.65890842805611	1.95746634107526\\
1.08124245096797	2.21110290073615\\
1.41497076391917	2.38377830939657\\
1.67431219460888	2.49773918811789\\
1.87576040675745	2.57138888298126\\
2.03360307134768	2.61796802131199\\
2.15891606343668	2.64643668053227\\
2.25989814780917	2.66271445002484\\
2.34251989232733	2.67069163243849\\
2.41112293345451	2.67293497202217\\
};
\addplot [color=mycolor1, forget plot]
  table[row sep=crcr]{%
2.2306613868761	2.81803070297518\\
2.25810690410877	2.81718012527407\\
2.28201516003601	2.81491320421253\\
2.30308950693452	2.81155368741527\\
2.32186667064398	2.80732269213538\\
2.33876212010734	2.80236972560499\\
2.35410163785533	2.79679295696782\\
2.36814362598294	2.79065256203978\\
2.38109507726808	2.78397948565953\\
2.39312313224099	2.77678107656757\\
2.40436349690098	2.76904449916777\\
2.41492657297905	2.76073847420239\\
2.42490186835632	2.75181366374805\\
2.43436105653859	2.74220184468911\\
2.44335990651309	2.73181387613749\\
2.45193918373476	2.72053633609572\\
2.46012451013591	2.70822655989054\\
2.4679250472588	2.69470563459613\\
2.47533070951499	2.67974865995893\\
2.48230739246368	2.66307123355111\\
2.48878936392124	2.64431058778368\\
2.49466743013696	2.6229989883018\\
2.4997706103052	2.59852571315108\\
2.50383757236253	2.57008185871995\\
2.50647153235448	2.53657882915073\\
2.50706783155106	2.49652574468635\\
2.50469535887311	2.44784158954154\\
2.49789834998327	2.38756214474631\\
2.48435833264454	2.31137583767289\\
2.46030778970481	2.21288289339427\\
2.41950575059641	2.08242347523106\\
2.35147553534986	1.90531321212694\\
2.23868402405164	1.65958770416935\\
2.05300806937287	1.31467312543502\\
1.75510971736689	0.836517617528178\\
1.30849126874066	0.211189888458031\\
0.720798419557309	-0.511317529079945\\
0.0779140443342507	-1.20791854545575\\
-0.504310404303506	-1.76449354912077\\
-0.963651051034921	-2.15132519346601\\
-1.30133101338113	-2.40097689803586\\
-1.54462840311249	-2.55800318024844\\
-1.72127575581048	-2.65671069052081\\
-1.85211382956641	-2.71925994897991\\
-1.95136885095432	-2.75917169432432\\
-2.02850063951073	-2.78462216400938\\
-2.08981496010035	-2.80061234470425\\
-2.13957654655819	-2.81026070860493\\
-2.18072560715238	-2.81555396271566\\
-2.21533107455486	-2.81778429561671\\
-2.2448783990183	-2.81780785993293\\
-2.27045467227012	-2.81620060789198\\
-2.29286973444643	-2.81335422309845\\
-2.31273685207127	-2.80953638074975\\
-2.33052744438039	-2.80492929976484\\
-2.34660885364804	-2.79965478587334\\
-2.36127083449701	-2.7937906722762\\
-2.37474439994278	-2.78738164589066\\
-2.38721539229624	-2.78044630443655\\
-2.3988343416865	-2.77298159220663\\
-2.40972365384161	-2.76496532407901\\
-2.41998282343293	-2.75635722071773\\
-2.42969213301304	-2.74709867891067\\
-2.43891512794588	-2.73711134941806\\
-2.44770002625738	-2.72629446298113\\
-2.45608010789502	-2.71452071131373\\
-2.46407301201541	-2.70163033252864\\
-2.47167873369492	-2.68742284384416\\
-2.4788759266952	-2.67164557285018\\
-2.48561584719577	-2.65397770782051\\
-2.49181285049689	-2.63400793094603\\
-2.49732966899515	-2.61120267356518\\
-2.50195456246426	-2.58486040061168\\
-2.50536549382862	-2.554044685875\\
-2.50707310865982	-2.51748448173663\\
-2.5063283000101	-2.47342271799767\\
-2.50196930299109	-2.41938216717487\\
-2.49216346758393	-2.35179720417731\\
-2.47396280197052	-2.26542754675948\\
-2.4425289071623	-2.15242386901544\\
-2.38978363183492	-2.00087480010687\\
-2.30214558738244	-1.79274231006888\\
-2.15720197537847	-1.50172612091604\\
-1.92078092466796	-1.09405563207464\\
-1.55158685210626	-0.541139515258476\\
-1.02868761619051	0.144455732587051\\
-0.398536047886839	0.87154781456737\\
0.226482120535896	1.50768862618519\\
0.750336570209395	1.97786021388668\\
1.14618923806126	2.29038532705206\\
1.43290798360277	2.48861007795899\\
1.63980650934666	2.61301346167155\\
1.79139786468341	2.69150173946714\\
1.90500767470046	2.74143986091333\\
1.99224745257609	2.77333611988895\\
2.06083057765806	2.79357153066645\\
2.11593135374411	2.80608480485869\\
2.16108190935215	2.81335835477259\\
2.19874233023691	2.81699046255361\\
2.2306613868761	2.81803070297518\\
};
\addplot [color=mycolor1, forget plot]
  table[row sep=crcr]{%
1.34916822602105	2.61884380257413\\
1.36255348536219	2.61842535800733\\
1.37480683187091	2.61726037653587\\
1.38612752561835	2.61545290337742\\
1.39667669922158	2.61307332288931\\
1.40658638148471	2.61016585503015\\
1.41596613668646	2.60675348673496\\
1.42490800439879	2.60284105382679\\
1.43349020388433	2.598416912692\\
1.44177991815768	2.59345345180439\\
1.44983536863337	2.5879065527139\\
1.45770731417958	2.58171399239607\\
1.46544004576872	2.57479266390242\\
1.47307188948927	2.56703436125339\\
1.48063516614287	2.55829970555158\\
1.48815547206729	2.54840955221251\\
1.49565002401685	2.53713286811873\\
1.50312461938875	2.52416952842272\\
1.5105684469781	2.50912563214587\\
1.51794544380942	2.49147756217942\\
1.52517994674875	2.47051874931118\\
1.53213268434057	2.4452792850012\\
1.53856001621684	2.41440198359606\\
1.5440434126235	2.37594709462455\\
1.54786479332811	2.32707784393597\\
1.54878118856405	2.26354409964061\\
1.54460916042535	2.17882346641676\\
1.53144942360156	2.06269711309166\\
1.50225566229098	1.89898924372039\\
1.44436890588002	1.66251346478269\\
1.33620870262374	1.31721418768455\\
1.14673141186483	0.824026867167908\\
0.849782608285716	0.175946554623806\\
0.461685962801752	-0.548018956645931\\
0.0565516337693074	-1.20228090782282\\
-0.290753796572131	-1.6917502022023\\
-0.554835633179842	-2.01816874854801\\
-0.746358627160081	-2.22639575742822\\
-0.884817909103233	-2.3589451739538\\
-0.986659923035106	-2.44472049770824\\
-1.0634337309332	-2.50144568703854\\
-1.12282433770986	-2.53974002418693\\
-1.16991128279668	-2.5660218412918\\
-1.20809299884232	-2.58425316968421\\
-1.23968783716858	-2.59694161257033\\
-1.26631202944042	-2.60571445312303\\
-1.28911705407786	-2.61165259675128\\
-1.30894048914882	-2.61548905733766\\
-1.32640356331209	-2.617729744945\\
-1.34197546868493	-2.61872870075034\\
-1.35601659862283	-2.61873598932488\\
-1.36880817441296	-2.61792878298912\\
-1.38057291538259	-2.61643186607089\\
-1.39148970600711	-2.61433131778502\\
-1.40170416735516	-2.6116836843858\\
-1.4113363853036	-2.608522081478\\
-1.42048663037264	-2.6048601311055\\
-1.42923963244693	-2.60069429697314\\
-1.43766779293806	-2.5960049543598\\
-1.44583359306231	-2.59075637031308\\
-1.45379136809174	-2.58489564336501\\
-1.46158854894265	-2.57835053799512\\
-1.46926641317011	-2.57102602853142\\
-1.47686032733914	-2.56279922031462\\
-1.48439939058245	-2.55351211746392\\
-1.49190528944329	-2.54296141961962\\
-1.49939002248124	-2.53088409664441\\
-1.50685190825312	-2.51693681519435\\
-1.5142688787915	-2.50066621316221\\
-1.52158734818893	-2.48146525837629\\
-1.5287036798198	-2.4585079950171\\
-1.53543297058567	-2.43064999568616\\
-1.54145557404778	-2.39627321363943\\
-1.5462236004709	-2.35303882637033\\
-1.54879377227572	-2.29748516474355\\
-1.54752207454328	-2.22436239290059\\
-1.53949639900754	-2.12552421645835\\
-1.51947921042305	-1.98811596811773\\
-1.47800070676288	-1.79185510593083\\
-1.39834292507678	-1.50606847102606\\
-1.25373377570828	-1.09096289236972\\
-1.01217443181423	-0.516876345325173\\
-0.663483593147455	0.185309718307808\\
-0.255650441715128	0.892281146234285\\
0.127141962370146	1.46932437295849\\
0.432991150703737	1.87280614464764\\
0.658442536758492	2.13409571476104\\
0.821074581862312	2.30001032816487\\
0.939487523829889	2.40636173951891\\
1.02762555404293	2.47592484154283\\
1.09493604306106	2.52242215305209\\
1.14766234075684	2.55409174160465\\
1.18995036713984	2.57596124909194\\
1.22459961632259	2.59117267921784\\
1.25354049557583	2.60173993143481\\
1.27813357584593	2.60898559194935\\
1.29935845276156	2.61379764430181\\
1.31793477108438	2.61678384887558\\
1.33440123755647	2.61836680110905\\
1.34916822602105	2.61884380257413\\
};
\addplot [color=mycolor1, forget plot]
  table[row sep=crcr]{%
1.33832785075596	1.05856770426014\\
1.71708873794132	1.04706270790821\\
2.00524308812722	1.01997360948505\\
2.22158293985867	0.98568232028984\\
2.38421685443873	0.949191101343268\\
2.5076145272811	0.91313611260959\\
2.6024663989092	0.87874493050746\\
2.67642270413419	0.846477797541033\\
2.7349083019084	0.816402633286111\\
2.78177963434076	0.78840014646188\\
2.81980118326959	0.762271550607141\\
2.85097669866004	0.737793146276745\\
2.87677601172378	0.714742657350381\\
2.89829006124147	0.692910811688235\\
2.91633737206639	0.672105353139954\\
2.93153778559506	0.652151275517667\\
2.9443639890652	0.632889257135411\\
2.95517784931997	0.614173310232574\\
2.96425621180258	0.595868146084142\\
2.97180927822365	0.577846482438437\\
2.97799364973239	0.559986372597205\\
2.982921432524	0.542168554727318\\
2.98666632910257	0.524273773221959\\
2.9892673015886	0.506179992999476\\
2.99073014005198	0.487759402000392\\
2.99102706084517	0.46887507006363\\
2.99009426746616	0.449377098720827\\
2.9878272023742	0.429098051468598\\
2.98407297275139	0.407847392434451\\
2.97861910870765	0.38540457648988\\
2.97117735555587	0.36151031754284\\
2.96136053270739	0.335855404031068\\
2.94864948746336	0.308066221173719\\
2.93234564125112	0.27768587209012\\
2.91150227341512	0.244149475236976\\
2.88482407104405	0.206751914642895\\
2.8505189770917	0.164606224499872\\
2.80607829448728	0.11659141871253\\
2.74795009456766	0.0612912218850788\\
2.67105933973035	-0.00306711709908142\\
2.56812617464324	-0.0786451112657921\\
2.4287765554169	-0.167922293729583\\
2.23862361750258	-0.273355629826909\\
1.97903645009366	-0.396452773264885\\
1.62946597084443	-0.535843685786365\\
1.17549772978648	-0.684372465629206\\
0.623913899953506	-0.827066784603145\\
0.0156603832950979	-0.944197989256905\\
-0.582713904244943	-1.02082682525928\\
-1.1109120882187	-1.05489053430368\\
-1.54004749962896	-1.05543374616649\\
-1.87136800357853	-1.03485403068786\\
-2.12118690488697	-1.00334708083351\\
-2.30861435903366	-0.967504177395819\\
-2.45007070231361	-0.931008400315427\\
-2.55806939607659	-0.895691241500445\\
-2.64167391994001	-0.862335439946601\\
-2.70732795253916	-0.831170891254812\\
-2.75960186319709	-0.802153688786899\\
-2.80175664512797	-0.775115414285042\\
-2.83614241827917	-0.749840185559272\\
-2.86447275609878	-0.726102868844767\\
-2.88801214554155	-0.703686833934361\\
-2.90770435438771	-0.682391113449945\\
-2.92426093572038	-0.662032191603923\\
-2.9382227944058	-0.642443165421628\\
-2.95000341218912	-0.623471699475948\\
-2.9599194431015	-0.604977491641939\\
-2.96821248644699	-0.58682959233689\\
-2.97506458571853	-0.568903718994405\\
-2.98060916195791	-0.55107959889393\\
-2.98493852032551	-0.533238312675461\\
-2.98810867137945	-0.515259573693341\\
-2.99014191896274	-0.497018851100946\\
-2.99102743998762	-0.47838421890837\\
-2.99071988462755	-0.459212783487904\\
-2.98913583058736	-0.439346503305103\\
-2.98614770441886	-0.418607162070534\\
-2.981574503371	-0.396790184171176\\
-2.97516826859042	-0.373656881793238\\
-2.96659470971629	-0.348924587454162\\
-2.95540556317111	-0.322253943280443\\
-2.94099902838946	-0.293232380082817\\
-2.92256272972903	-0.261352524893812\\
-2.898990733363	-0.22598395477037\\
-2.86876167890882	-0.186336478897755\\
-2.82975838133331	-0.141413299954517\\
-2.77899972415068	-0.08995380351499\\
-2.71224382577774	-0.030370353344629\\
-2.62341239539384	0.0393040537726912\\
-2.50379953054959	0.121410226009802\\
-2.34112351143517	0.218486204153281\\
-2.11880645853817	0.332683514259896\\
-1.81669174283804	0.464386512203611\\
-1.41581099600258	0.609717854256119\\
-0.910199242096203	0.757659551907771\\
-0.32300024499543	0.8900063214995\\
0.289093551297695	0.98803779850949\\
0.858233627317148	1.04278813347991\\
1.33832785075596	1.05856770426014\\
};
\addplot [color=mycolor1]
  table[row sep=crcr]{%
2.22721472973942	1.94935886896179\\
2.3886648425121	1.94441487511101\\
2.5195634483927	1.93205457572182\\
2.62670779849127	1.91501766628131\\
2.71531853364	1.89508786650117\\
2.78936864949369	1.87341097537522\\
2.85187814440773	1.85071228241933\\
2.90515032165589	1.82744075521499\\
2.9509520177711	1.80386332654994\\
2.99064783647547	1.78012618777659\\
3.02529909977035	1.7562945351756\\
3.0557365623221	1.73237827445755\\
3.08261387510984	1.7083485310185\\
3.10644696879361	1.68414807630787\\
3.12764310033082	1.65969765855822\\
3.14652223914943	1.63489950162782\\
3.16333268711427	1.60963876367151\\
3.17826225469757	1.5837834336332\\
3.19144589188063	1.55718292750329\\
3.20297034763997	1.52966548843736\\
3.2128761683912	1.50103436869378\\
3.2211571124149	1.47106265802835\\
3.22775682617807	1.43948650786522\\
3.23256237104755	1.40599637010381\\
3.23539387180292	1.3702257107414\\
3.23598913837185	1.33173645766061\\
3.23398152983158	1.29000018434938\\
3.22886850103885	1.24437370419161\\
3.21996707976175	1.19406735000614\\
3.20635080827997	1.13810376656035\\
3.18676025898309	1.07526465092859\\
3.15947593702972	1.0040228169715\\
3.12213828317303	0.922457920675385\\
3.07149545967596	0.828157721349833\\
3.00305879148097	0.718116213117772\\
2.91065699973037	0.588662002756959\\
2.785925244978	0.435496623074383\\
2.61788823849866	0.254008232649376\\
2.39306798253518	0.0401523940333445\\
2.09699730361163	-0.20771900183087\\
1.7183819635995	-0.485932899271695\\
1.25639683089422	-0.782386182967809\\
0.728319451607428	-1.07597695395358\\
0.17048677629427	-1.34156752261761\\
-0.372344302750219	-1.55919115860543\\
-0.863264272549092	-1.72086595365401\\
-1.28299919742863	-1.8302347441912\\
-1.62890152595063	-1.89731652670458\\
-1.90833376158261	-1.93331843320936\\
-2.13240654410545	-1.94782687646357\\
-2.3122537087605	-1.94803363323748\\
-2.45746747168844	-1.93896359662991\\
-2.57574556593285	-1.92399109460735\\
-2.67305603610964	-1.90533078490532\\
-2.75395564612754	-1.88441460443591\\
-2.82190758462804	-1.86215611801615\\
-2.87954766726586	-1.83912801979308\\
-2.92889157233296	-1.81567886095776\\
-2.97149068204595	-1.79200910352774\\
-3.00854741762708	-1.76822049003994\\
-3.04100007190525	-1.74434802267285\\
-3.0695851473103	-1.7203805908067\\
-3.09488323143276	-1.69627413213334\\
-3.11735281839452	-1.67195981527391\\
-3.13735524534064	-1.64734883071173\\
-3.15517299859553	-1.62233479306363\\
-3.17102297575804	-1.59679437416259\\
-3.18506579905015	-1.570586527561\\
-3.19741190629232	-1.54355048208521\\
-3.20812485602917	-1.51550254281508\\
-3.21722203851139	-1.48623162017347\\
-3.22467275544594	-1.45549329519237\\
-3.23039339092638	-1.42300210774249\\
-3.23423911335776	-1.38842161161915\\
-3.23599118590349	-1.35135156242766\\
-3.23533847055249	-1.31131137681683\\
-3.23185101774822	-1.2677187105326\\
-3.22494264006069	-1.21986163848029\\
-3.2138179373885	-1.16686249014231\\
-3.19739719631547	-1.10763095340298\\
-3.17420973909784	-1.0408037811618\\
-3.142242561071	-0.964668760862181\\
-3.09872682936675	-0.877072578179989\\
-3.0398417586358	-0.77531812311728\\
-2.96031892266926	-0.656071422287709\\
-2.85295425309884	-0.515330753928937\\
-2.70811161177175	-0.348574764250833\\
-2.51348980394932	-0.151314840167112\\
-2.25479111797749	0.0795960527623154\\
-1.91840618384291	0.343563906305597\\
-1.49721915275415	0.633014609293368\\
-0.998726638487032	0.931115827109969\\
-0.450328181715587	1.21375185121395\\
0.105465468340197	1.45718721810745\\
0.625945116714116	1.64701594620063\\
1.08251666630864	1.78154116777267\\
1.46486936946524	1.86834353319716\\
1.77626331377619	1.91854897142989\\
2.02656829443923	1.94275375393752\\
2.22721472973941	1.94935886896179\\
};
\addlegendentry{Аппроксимации}

\addplot [color=mycolor2]
  table[row sep=crcr]{%
2.12132034355964	2.82842712474619\\
2.18582575000323	2.82640660874952\\
2.2459761483502	2.82067996094826\\
2.30316600770785	2.81153894842004\\
2.35833530419819	2.79908319427389\\
2.4121407651283	2.78328504568152\\
2.46505313939319	2.76402339892432\\
2.51741350921551	2.74110157112695\\
2.56946568278775	2.71425682483196\\
2.62137377470791	2.68316571665979\\
2.67322998558975	2.64744786912504\\
2.72505542741334	2.60667010622203\\
2.77679571750944	2.56035271147659\\
2.82831256449113	2.50797963523731\\
2.87937250576692	2.44901464186941\\
2.92963424895255	2.38292549191834\\
2.97863667364875	2.30921809960124\\
3.02579039162557	2.22748193169334\\
3.0703766721051	2.13744642550811\\
3.11155818595542	2.03904567139316\\
3.14840590189045	1.93248505458784\\
3.1799449834646	1.81829951655162\\
3.20521925970046	1.69738980361013\\
3.2233689191928	1.57102231442397\\
3.23371058284854	1.44078163681543\\
3.23580480244767	1.30847305653245\\
3.22949541157146	1.1759834831\\
3.21490900216393	1.04511946165368\\
3.19240995273442	0.917445713323391\\
3.16251366135309	0.794144173493542\\
3.12576385298257	0.675902054710668\\
3.08257536776515	0.562820516214273\\
3.03302813706062	0.454314775018543\\
2.97656426645917	0.348948817128694\\
2.91146921429886	0.244099783305823\\
2.83385680006128	0.135243682019644\\
2.73548220960228	0.0144130439331602\\
2.59871334341581	-0.133168683665202\\
2.38479638365891	-0.336180712113444\\
2.01057175466146	-0.648371525543042\\
1.34210041814469	-1.13791067535239\\
0.375352321166521	-1.75822148716338\\
-0.527448468220416	-2.26262801632552\\
-1.11365847778129	-2.5437869612038\\
-1.4567327767317	-2.68228003389134\\
-1.6691296256022	-2.75259678373195\\
-1.81428800439327	-2.79054335642282\\
-1.92320462744895	-2.81168897715777\\
-2.01132392805712	-2.82302365363059\\
-2.08680248126988	-2.82787464184343\\
-2.15422943553058	-2.82790871121207\\
-2.21634478307291	-2.8239827357917\\
-2.27487422572476	-2.8165264847195\\
-2.33095693947289	-2.80572465930185\\
-2.3853755403757	-2.79160756168836\\
-2.43868469976231	-2.77409782070174\\
-2.49128507004909	-2.75303495626884\\
-2.54346612600745	-2.7281884019911\\
-2.59543032955127	-2.69926454755406\\
-2.64730535029083	-2.66591103001887\\
-2.69914810041563	-2.62772046423944\\
-2.75094277189846	-2.58423541950896\\
-2.80259428647749	-2.53495641420315\\
-2.85391830194335	-2.47935483422118\\
-2.90462903903618	-2.41689283661657\\
-2.95432664759112	-2.34705230341292\\
-3.00248656709011	-2.26937453210355\\
-3.04845424185856	-2.18351130413452\\
-3.0914493816545	-2.08928598065547\\
-3.13058429431929	-1.98676020448289\\
-3.16490008683964	-1.87629788656397\\
-3.19342217916266	-1.75861429449935\\
-3.21523240499265	-1.63479579561046\\
-3.22954956680171	-1.50627700312196\\
-3.23580523135335	-1.37476797309193\\
-3.23369896357047	-1.2421341494958\\
-3.22321880671614	-1.11024302689198\\
-3.20461860896113	-0.980799507713277\\
-3.17835148049538	-0.85519274756288\\
-3.14496439092874	-0.734369602871579\\
-3.10495870881183	-0.618735254776386\\
-3.05861192744495	-0.508062722468019\\
-3.0057323937692	-0.401369642978784\\
-2.94527025004781	-0.296685462718974\\
-2.87460204254734	-0.190562899149911\\
-2.78805571026974	-0.0770301658026245\\
-2.67361421522119	0.0546866392302208\\
-2.50520482475694	0.225105040795738\\
-2.22541867988909	0.474087715816021\\
-1.7209533613293	0.868461990547897\\
-0.880186083660502	1.44500069250654\\
0.110323684998138	2.03883159526995\\
0.859259206598599	2.42737283484743\\
1.30741523004191	2.62511520505649\\
1.57426953542748	2.72304263504847\\
1.74766615441039	2.77435113512036\\
1.87207484559155	2.80265686669549\\
1.96924085114733	2.81832059141065\\
2.0502990590364	2.82612977760851\\
2.12132034355964	2.82842712474619\\
};
\addlegendentry{Сумма Минковского}

\end{axis}

\begin{axis}[%
width=0.798\linewidth,
height=0.597\linewidth,
at={(-0.104\linewidth,-0.066\linewidth)},
scale only axis,
xmin=0,
xmax=1,
ymin=0,
ymax=1,
axis line style={draw=none},
ticks=none,
axis x line*=bottom,
axis y line*=left,
legend style={legend cell align=left, align=left, draw=white!15!black}
]
\end{axis}
\end{tikzpicture}%
        \caption{Эллипсоидальные аппроксимации для 10 направлений.}
\end{figure}
\clearpage
\begin{figure}[t]

        \centering
        % This file was created by matlab2tikz.
%
%The latest updates can be retrieved from
%  http://www.mathworks.com/matlabcentral/fileexchange/22022-matlab2tikz-matlab2tikz
%where you can also make suggestions and rate matlab2tikz.
%
\definecolor{mycolor1}{rgb}{0.00000,0.44700,0.74100}%
\definecolor{mycolor2}{rgb}{0.85000,0.32500,0.09800}%
%
\begin{tikzpicture}

\begin{axis}[%
width=0.618\linewidth,
height=0.487\linewidth,
at={(0\linewidth,0\linewidth)},
scale only axis,
xmin=-4,
xmax=4,
xlabel style={font=\color{white!15!black}},
xlabel={$x_1$},
ymin=-3,
ymax=3,
ylabel style={font=\color{white!15!black}},
ylabel={$x_2$},
axis background/.style={fill=white},
axis x line*=bottom,
axis y line*=left,
xmajorgrids,
ymajorgrids,
legend style={at={(0.03,0.97)}, anchor=north west, legend cell align=left, align=left, draw=white!15!black}
]
\addplot [color=mycolor1, forget plot]
  table[row sep=crcr]{%
2.36712178368749	2.74633057020532\\
2.41188305682678	2.74494651398331\\
2.45036041459025	2.74130089565813\\
2.48383536069915	2.73596692388682\\
2.51327948444918	2.72933446739078\\
2.53943919311047	2.72166753585199\\
2.56289492242467	2.71314181506891\\
2.58410302011837	2.70386940753685\\
2.60342571833248	2.69391520541791\\
2.62115280082523	2.68330766515176\\
2.63751738560017	2.67204572812381\\
2.65270745907852	2.66010298488441\\
2.66687427014079	2.64742976011011\\
2.68013832873481	2.63395351039579\\
2.69259349381253	2.61957772004863\\
2.70430943819101	2.6041793136995\\
2.71533261385189	2.58760445132293\\
2.72568568587575	2.56966240776041\\
2.73536523285703	2.55011704085516\\
2.74433729805113	2.52867508999008\\
2.75253007967915	2.50497017880853\\
2.75982261143016	2.47854086090236\\
2.76602760995085	2.44880025137829\\
2.77086559717422	2.41499358451471\\
2.77392567575526	2.37613820122431\\
2.77460549370625	2.33093765236215\\
2.77201821305984	2.27765729690984\\
2.7648464218491	2.21394231571241\\
2.7511098920125	2.13654986655322\\
2.72779317877431	2.04095569483249\\
2.69024836192134	1.92078707468285\\
2.6312539409082	1.7670493095065\\
2.53961413480675	1.56721869756115\\
2.39837999194539	1.30464987086518\\
2.18360687141246	0.959750006540987\\
1.8667826621648	0.516209188824812\\
1.42701784641989	-0.0238781630653391\\
0.87502831245873	-0.620875059217897\\
0.268936641059173	-1.19901922110702\\
-0.309576813214938	-1.68530797756881\\
-0.802100907580029	-2.04896030059078\\
-1.19206161730571	-2.300440537754\\
-1.49041992551985	-2.46708460024485\\
-1.7168759123608	-2.57532489307401\\
-1.88995458136084	-2.64492059461505\\
-2.02413470299114	-2.68919932969544\\
-2.12995725581116	-2.71680265831965\\
-2.21490837909904	-2.73327951527395\\
-2.284286999297	-2.74220891806166\\
-2.34187032729419	-2.74592425079378\\
-2.39038313346388	-2.74596637420131\\
-2.4318186528206	-2.74336542452957\\
-2.46765634360889	-2.73881708355027\\
-2.49901014208278	-2.7327940719517\\
-2.52673028332805	-2.72561759993438\\
-2.551474024202	-2.71750375208859\\
-2.57375536742722	-2.70859393338899\\
-2.5939804436843	-2.69897499500867\\
-2.61247296631128	-2.688692537203\\
-2.6294927100655	-2.67775958603992\\
-2.64524900273341	-2.66616202895009\\
-2.6599105764423	-2.65386167486455\\
-2.67361268902278	-2.64079746083638\\
-2.68646212009346	-2.62688508668123\\
-2.69854042224194	-2.61201517643097\\
-2.70990563030842	-2.59604990865956\\
-2.72059247497691	-2.57881790224895\\
-2.73061098716967	-2.5601069666803\\
-2.73994319179238	-2.53965410007404\\
-2.74853734053432	-2.51713180921804\\
-2.75629877602211	-2.49212938365471\\
-2.76307597883326	-2.46412710489964\\
-2.76863950239953	-2.43246039507853\\
-2.77265014411951	-2.39626942454063\\
-2.7746104875259	-2.35442742359956\\
-2.77379029091263	-2.3054374557388\\
-2.76911009981251	-2.24728211523106\\
-2.75895730729457	-2.17720283029247\\
-2.74089227231279	-2.0913749275097\\
-2.71117633487659	-1.98443346302753\\
-2.66401914162082	-1.84880387542276\\
-2.59041768625865	-1.67383876852635\\
-2.47652724106683	-1.44496938777841\\
-2.30193857203366	-1.14371563581315\\
-2.0396739695219	-0.750852651889909\\
-1.66265564107253	-0.256803878437201\\
-1.16248546087895	0.319450601512933\\
-0.573614265552006	0.917547528939285\\
0.0284025713818862	1.45675943888985\\
0.5685251002902	1.88250662286626\\
1.00960568842464	2.18713123722861\\
1.35150239782658	2.39258093926768\\
1.61139397123953	2.52710219556684\\
1.80908182046774	2.61398894502504\\
1.96116532773652	2.66959557253549\\
2.08006223005889	2.7046815507453\\
2.17466815864962	2.726172144186\\
2.25127885929965	2.73851896753466\\
2.3143624731471	2.74460741055068\\
2.36712178368749	2.74633057020532\\
};
\addplot [color=mycolor1, forget plot]
  table[row sep=crcr]{%
2.35240505464064	2.76075021334929\\
2.39429606969185	2.75945449074689\\
2.43037411286026	2.75603583797069\\
2.46182071964198	2.75102474749105\\
2.48953226372909	2.7447822797979\\
2.51419796801997	2.7375529548978\\
2.53635434440832	2.72949928507746\\
2.55642367540865	2.72072453208959\\
2.57474154346732	2.71128775569393\\
2.59157672930874	2.70121369370841\\
2.6071457034148	2.69049907088173\\
2.62162321073745	2.67911633833517\\
2.63514996235992	2.66701545720637\\
2.6478381125071	2.65412407506413\\
2.65977495873579	2.64034624885663\\
2.67102511914481	2.62555970842345\\
2.68163128510145	2.60961150375056\\
2.69161349794549	2.59231171409547\\
2.70096672992684	2.57342469308632\\
2.70965633462077	2.55265704944946\\
2.71761062979593	2.52964117359486\\
2.72470942365474	2.50391254885291\\
2.73076659186681	2.4748782286687\\
2.73550368771706	2.44177255370809\\
2.73850973115383	2.40359416717204\\
2.73917927620848	2.35901526004097\\
2.73661574356727	2.30624913708954\\
2.72947837780359	2.24285482082059\\
2.7157367198957	2.16544669277417\\
2.69227293033385	2.06926342687234\\
2.65423709364454	1.94753924024194\\
2.59402035163186	1.79063563981677\\
2.49971323572562	1.58501269692596\\
2.35315024241768	1.31256503039365\\
2.12864777932873	0.952067155541005\\
1.79621437753681	0.486680047250032\\
1.33625613088531	-0.078254314338472\\
0.766437217218612	-0.69465416704619\\
0.154503796467904	-1.27852819912494\\
-0.414737908033144	-1.75715359300286\\
-0.888451837771183	-2.10699420135133\\
-1.25739472197187	-2.34495625430441\\
-1.53680289125242	-2.50103042562183\\
-1.74770650766029	-2.60184210371291\\
-1.90850464434176	-2.66650103010414\\
-2.03309498455462	-2.70761509614372\\
-2.13140980780297	-2.73325953784195\\
-2.21042927392142	-2.7485852910304\\
-2.27506472386636	-2.75690363778087\\
-2.32880559720436	-2.76037053426479\\
-2.37416495660386	-2.76040947101594\\
-2.41298006388656	-2.75797261878991\\
-2.44661469143309	-2.75370354023524\\
-2.4760961740651	-2.74803989675378\\
-2.50220924915406	-2.74127920764433\\
-2.52556111884979	-2.73362152710032\\
-2.54662714872804	-2.72519745309993\\
-2.56578336978446	-2.71608663470736\\
-2.583329856457	-2.70632998808298\\
-2.59950769563699	-2.69593763465725\\
-2.61451137202192	-2.68489382800076\\
-2.6284978033132	-2.67315965738214\\
-2.64159285649488	-2.66067399757508\\
-2.65389589416861	-2.64735294979028\\
-2.66548269162024	-2.63308784501451\\
-2.67640689871448	-2.61774172870901\\
-2.68670007084269	-2.60114409074919\\
-2.69637013715653	-2.58308342333688\\
-2.70539798684275	-2.56329695495395\\
-2.71373160169024	-2.54145658319533\\
-2.72127679579021	-2.51714955916172\\
-2.72788306153676	-2.48985177771614\\
-2.73332213400089	-2.45889047089402\\
-2.73725545152916	-2.42339148023845\\
-2.73918432926723	-2.38220377433743\\
-2.73837272033678	-2.33378998386214\\
-2.73372579881979	-2.27606573156975\\
-2.7235964056328	-2.20616156525457\\
-2.70547282555558	-2.12006886105957\\
-2.67547208091285	-2.01211722390011\\
-2.62752302154944	-1.87422773696123\\
-2.55209340449926	-1.69493840971088\\
-2.43439384879093	-1.45844070952189\\
-2.25251213766288	-1.14463119254633\\
-1.97767200659054	-0.732949898706597\\
-1.5822742236057	-0.214800426101576\\
-1.06187245772022	0.384846887446284\\
-0.460163049994717	0.99612986835765\\
0.139878547911511	1.53372324957911\\
0.664968942177641	1.94772499231274\\
1.08537079974014	2.23812186948185\\
1.40696804069012	2.43139655318979\\
1.64956497932395	2.55697472989305\\
1.83339591262838	2.63777387015065\\
1.97462394608275	2.68941180394685\\
2.08504289043958	2.72199559336845\\
2.17298471767437	2.74197177729419\\
2.24429964729586	2.75346450848608\\
2.30312113316005	2.75914104476399\\
2.35240505464064	2.76075021334929\\
};
\addplot [color=mycolor1, forget plot]
  table[row sep=crcr]{%
2.33637221543638	2.77359127925803\\
2.37555924354049	2.77237880820087\\
2.40937123794229	2.76917454596508\\
2.43889707827725	2.76446924085856\\
2.46496343861871	2.7585971151577\\
2.48820653315317	2.75178450780297\\
2.50912212950084	2.744181630855\\
2.52810088298891	2.73588349316341\\
2.54545360964711	2.7269437243832\\
2.56142955275294	2.71738362781807\\
2.57622968413772	2.70719792374284\\
2.5900164143854	2.69635809673865\\
2.60292063833293	2.68481390212879\\
2.61504673328514	2.67249333997666\\
2.62647590475164	2.65930122141101\\
2.63726810274461	2.64511629832438\\
2.64746258445559	2.62978677865868\\
2.65707705402952	2.61312388268925\\
2.66610514396634	2.59489288554136\\
2.67451178632173	2.57480080444892\\
2.68222571341034	2.55247947808503\\
2.68912786135759	2.52746217653571\\
2.69503371742093	2.49915095925159\\
2.69966647054546	2.46677058143266\\
2.70261587912763	2.42930254372515\\
2.70327451243641	2.38538942141003\\
2.70073749781637	2.33319418508968\\
2.69364247410982	2.27019083236438\\
2.67991042087211	2.19285019863564\\
2.65632152704606	2.09616833205675\\
2.61781990553633	1.97297012766626\\
2.556393576908	1.81293522681722\\
2.4593772940689	1.60143082913913\\
2.30730201772547	1.31876620611291\\
2.07262878699913	0.94196376050997\\
1.72391573965135	0.453789171162566\\
1.24344969741539	-0.136391318768668\\
0.656926099516598	-0.770996753093629\\
0.0417886619641087	-1.35809301456615\\
-0.515623703650036	-1.82689899376119\\
-0.969303541781385	-2.16201421670412\\
-1.31723069543787	-2.38645255455632\\
-1.57829933202544	-2.53229442217931\\
-1.77441694676618	-2.62604232656544\\
-1.92364912764283	-2.68605128040974\\
-2.03924856860398	-2.72419816479556\\
-2.13053605171207	-2.74800903746349\\
-2.20400284943405	-2.76225725612886\\
-2.26419305112633	-2.77000296444939\\
-2.31432632834073	-2.77323664854273\\
-2.35671847301735	-2.77327262353084\\
-2.39306188920309	-2.77099058863201\\
-2.42461308321403	-2.76698563780904\\
-2.45231913795599	-2.76166279258598\\
-2.47690408935424	-2.75529748631566\\
-2.49892872559441	-2.74807481098525\\
-2.51883255679587	-2.74011527302354\\
-2.53696365504808	-2.73149180432721\\
-2.55360011571802	-2.72224097327856\\
-2.56896563426422	-2.71237024001503\\
-2.583240872228	-2.7018624133194\\
-2.59657174112777	-2.69067802547869\\
-2.60907536260046	-2.67875604645876\\
-2.62084420261484	-2.66601314846319\\
-2.63194868411551	-2.65234156671407\\
-2.64243842589445	-2.63760545389755\\
-2.65234211207123	-2.62163547080534\\
-2.66166584405338	-2.60422117073813\\
-2.67038963981693	-2.58510049165977\\
-2.6784614893414	-2.5639453284014\\
-2.68578799793619	-2.54034165851501\\
-2.69222006701896	-2.51376194823722\\
-2.69753113435073	-2.48352642453764\\
-2.70138398314494	-2.44874803365179\\
-2.70327961671456	-2.40825314607774\\
-2.70247745867555	-2.3604657339348\\
-2.69786892269196	-2.3032359796149\\
-2.68777407981268	-2.23358396910862\\
-2.66961041986487	-2.14731447222314\\
-2.63934946132841	-2.03844174415473\\
-2.59063069771097	-1.89835713280234\\
-2.51336604572977	-1.71472897173176\\
-2.3917587955001	-1.47040741966589\\
-2.20228433366304	-1.14352791485559\\
-1.91428524384337	-0.712154976758204\\
-1.49989381707048	-0.169092576727588\\
-0.959545669549715	0.453636680936888\\
-0.346998703073263	1.076089367383\\
0.248235503995792	1.60952777200189\\
0.756340489353792	2.01023492636414\\
1.15552289245518	2.28602185753826\\
1.45720069343643	2.46734487511429\\
1.6832308024427	2.58435439754378\\
1.8539597705315	2.65939683811413\\
1.984991847698	2.70730692674371\\
2.087470984034	2.73754728674876\\
2.16917584595401	2.75610613263204\\
2.23553100450468	2.7667989929561\\
2.29035472161473	2.77208920980101\\
2.33637221543638	2.77359127925803\\
};
\addplot [color=mycolor1, forget plot]
  table[row sep=crcr]{%
2.31899511579583	2.78496710127215\\
2.35562981603182	2.78383324360938\\
2.38729775592243	2.78083185798896\\
2.41500170562571	2.77641662221972\\
2.43950357273108	2.77089669699762\\
2.46139029705334	2.76448142446218\\
2.48111974715084	2.75730951010792\\
2.49905313489982	2.74946824664576\\
2.51547819869842	2.74100620271409\\
2.53062596016773	2.731941508788\\
2.54468292505632	2.72226707646172\\
2.55779998553553	2.71195358311898\\
2.57009886950765	2.70095072299508\\
2.58167669807872	2.6891869960311\\
2.59260900645553	2.676568132452\\
2.60295142313444	2.66297410276083\\
2.61274006261332	2.64825451572018\\
2.6219905463938	2.63222203817396\\
2.63069540277576	2.61464325412882\\
2.63881937832462	2.59522608137787\\
2.64629187947257	2.57360243066363\\
2.65299528241172	2.54930414537656\\
2.65874708838325	2.52172927256302\\
2.66327266338513	2.49009418446174\\
2.66616324468682	2.45336466284387\\
2.66681042018401	2.41015524261165\\
2.66430232883478	2.35858004977787\\
2.65725653684722	2.29602884307349\\
2.64354680791206	2.21882754576371\\
2.61985118912713	2.12172286615602\\
2.58090258556044	1.99711165873042\\
2.51826721495296	1.83394833927724\\
2.41847530624473	1.61641992585627\\
2.26065458999511	1.32310807005003\\
2.01528220827559	0.929154871667727\\
1.64952589629397	0.417120007567265\\
1.1482714408185	-0.198666982227323\\
0.54640905154937	-0.850020916681251\\
-0.0691017291823024	-1.43764535137375\\
-0.612199948864583	-1.89453488182135\\
-1.04479878658608	-2.21414158478757\\
-1.3718153396137	-2.42511771476334\\
-1.61516534393746	-2.56107110241224\\
-1.79722379344277	-2.64810143533772\\
-1.93555021217121	-2.70372546121445\\
-2.04270539109275	-2.73908544849924\\
-2.12740163597375	-2.76117652473781\\
-2.19565920649712	-2.77441387849936\\
-2.25167424964972	-2.78162177519906\\
-2.29841322926979	-2.7846360705932\\
-2.33800775259184	-2.78466928587429\\
-2.37201536576894	-2.78253358634189\\
-2.40159285489621	-2.7787788787706\\
-2.42761273157058	-2.77377972370747\\
-2.45074261926824	-2.76779091832854\\
-2.47150014893828	-2.7609835610781\\
-2.49029146552292	-2.75346871855796\\
-2.50743860077189	-2.74531304937084\\
-2.52319916028231	-2.73654908333665\\
-2.53778061266345	-2.72718184373065\\
-2.55135071340469	-2.71719286899586\\
-2.56404509506734	-2.70654228408706\\
-2.57597271479835	-2.69516929825076\\
-2.58721960988191	-2.68299130905499\\
-2.5978512323804	-2.66990163492385\\
-2.60791348668932	-2.65576575356513\\
-2.61743245645768	-2.64041576869778\\
-2.62641265810546	-2.62364263858406\\
-2.63483347157999	-2.60518544727419\\
-2.64264313965648	-2.5847166411094\\
-2.6497493406152	-2.56182162521395\\
-2.65600473681119	-2.53597031735907\\
-2.66118493374687	-2.50647702921167\\
-2.6649546923366	-2.47244312764962\\
-2.66681556845443	-2.43267390052964\\
-2.66602360944957	-2.38555624036334\\
-2.66145791012851	-2.32887614154429\\
-2.65140730054231	-2.25954319945489\\
-2.63321932490039	-2.17317210160491\\
-2.60271796065926	-2.06345017956284\\
-2.5532428781369	-1.92121005516382\\
-2.47411910018783	-1.73318767435553\\
-2.34847036311905	-1.48077624467292\\
-2.1510355617258	-1.1401956316382\\
-1.84919435755573	-0.688107897947388\\
-1.41514372797696	-0.119250957563104\\
-0.855280865430319	0.526083098359011\\
-0.234160930659898	1.1574202658589\\
0.353373767113646	1.68411073867882\\
0.842696187852734	2.07009612486439\\
1.22026952463506	2.33099564995725\\
1.50246043079355	2.50062249797124\\
1.71263134581766	2.60942771688335\\
1.87096320639038	2.67902252218549\\
1.99240428591782	2.72342573228369\\
2.08743319127747	2.75146710747536\\
2.16328836969436	2.76869663302595\\
2.22498824023365	2.77863875701106\\
2.27605403234581	2.78356586654782\\
2.31899511579583	2.78496710127215\\
};
\addplot [color=mycolor1, forget plot]
  table[row sep=crcr]{%
2.30020669994137	2.79497003360658\\
2.33442779348364	2.79391054443853\\
2.36406366192883	2.79110146412444\\
2.39003684377099	2.78696181192526\\
2.41304890569281	2.78177729330581\\
2.43364086342606	2.77574132765827\\
2.45223524061943	2.7689818362945\\
2.46916576788949	2.76157888616564\\
2.48469862514458	2.75357632318545\\
2.4990477973216	2.74498934528772\\
2.51238625496893	2.73580923500262\\
2.52485410803677	2.72600600801785\\
2.53656450354794	2.71552942862473\\
2.54760777625928	2.70430862911225\\
2.55805417112397	2.69225040614971\\
2.56795530659792	2.67923612396515\\
2.57734441530391	2.66511700844601\\
2.58623526248155	2.6497074455038\\
2.5946194802314	2.6327756742544\\
2.60246183676749	2.61403095386041\\
2.60969263975395	2.59310582709883\\
2.61619597887432	2.56953141746383\\
2.62179172332008	2.542702640977\\
2.62620789476024	2.51182856207531\\
2.62903786604208	2.47586050220534\\
2.62967313221243	2.43338630950981\\
2.62719598509371	2.38247244008694\\
2.62020520353049	2.32042471414226\\
2.60652826894451	2.24342194673642\\
2.58274014246277	2.14595321992017\\
2.54335564670146	2.01996541621218\\
2.47949678751822	1.85363764394082\\
2.37683220232506	1.62987621008494\\
2.21297126498285	1.32537261555615\\
1.95626592344349	0.913253136607868\\
1.57259905293611	0.376138452149806\\
1.05034058210994	-0.265538845068828\\
0.43480429548368	-0.931861720771676\\
-0.178039901245535	-1.5171177814555\\
-0.704415039910225	-1.96005554934772\\
-1.11504702584882	-2.26348951010003\\
-1.42134106334064	-2.46111901777462\\
-1.64759369620293	-2.58752837680277\\
-1.81627967256246	-2.66816836568712\\
-1.94431008888774	-2.7196523161846\\
-2.04352060001263	-2.75239019587104\\
-2.12202234173039	-2.77286498390856\\
-2.185382705902	-2.78515204327266\\
-2.237468123753	-2.79185378693279\\
-2.28100707369837	-2.79466128154541\\
-2.31795885555641	-2.79469191916393\\
-2.34975517450592	-2.7926947809758\\
-2.37745987784314	-2.78917754473946\\
-2.40187600344441	-2.78448627762359\\
-2.42361860980879	-2.7788564533485\\
-2.44316509317433	-2.77244605686927\\
-2.46089046981923	-2.76535730926546\\
-2.47709245601405	-2.75765100206573\\
-2.49200950805543	-2.74935591027353\\
-2.50583391707323	-2.74047482637293\\
-2.51872135965986	-2.73098817820737\\
-2.53079784572123	-2.72085581955979\\
-2.54216469215797	-2.71001732897607\\
-2.55290192959034	-2.69839096767582\\
-2.56307038255415	-2.68587129689785\\
-2.5727125250979	-2.67232531332748\\
-2.58185208201898	-2.65758680599976\\
-2.59049219988251	-2.64144844511845\\
-2.59861182582681	-2.62365085150195\\
-2.606159669656	-2.60386752009051\\
-2.61304472916205	-2.5816839133836\\
-2.6191217364694	-2.55656819154151\\
-2.62416887494901	-2.52782972793532\\
-2.6278534437985	-2.49455948149816\\
-2.62967831816298	-2.4555429826136\\
-2.62889718603392	-2.40913136117688\\
-2.62437806539607	-2.35304728947931\\
-2.61437975905681	-2.28408921703587\\
-2.59618015937168	-2.19767714406857\\
-2.56545255417668	-2.08715767216677\\
-2.51522386318292	-1.94277105292442\\
-2.43419553653303	-1.7502485606065\\
-2.30432773508839	-1.48939355868306\\
-2.09848152744881	-1.1343369241126\\
-1.78199733962051	-0.660334255898569\\
-1.32757728226521	-0.0647399507045235\\
-0.748830647813808	0.60249490401494\\
-0.121708119297749	1.24011936632773\\
0.455176324749901	1.75741345798711\\
0.924069208136567	2.12736739288647\\
1.27977406430857	2.37319242181238\\
1.5429475354994	2.53140131594123\\
1.73794202389831	2.63235402061753\\
1.88453375381058	2.69678935047547\\
1.99693909722587	2.73788853009942\\
2.08496394786265	2.76386257147755\\
2.15532142036926	2.77984277321716\\
2.21264266329337	2.78907880461998\\
2.26016874412744	2.79366392340583\\
2.30020669994137	2.79497003360658\\
};
\addplot [color=mycolor1, forget plot]
  table[row sep=crcr]{%
2.27989423025359	2.80367012813669\\
2.31182909506837	2.80268110738935\\
2.33953609037404	2.80005458825546\\
2.36386281781723	2.79617711440432\\
2.38545448717227	2.79131239293695\\
2.40480920655615	2.78563889722791\\
2.42231645590045	2.77927442324971\\
2.43828425414907	2.7722922606117\\
2.45295859430733	2.76473184231069\\
2.46653749575951	2.7566056535963\\
2.4791812362186	2.74790351148229\\
2.49101981016599	2.73859490116034\\
2.50215831474622	2.72862977367262\\
2.51268072396356	2.71793801000599\\
2.52265233631157	2.70642760149479\\
2.53212104114996	2.69398145776753\\
2.54111742329624	2.68045260898249\\
2.54965359336202	2.66565739609871\\
2.55772047072465	2.64936601366224\\
2.56528302613558	2.63128944511472\\
2.57227266530954	2.61106135187156\\
2.57857542753652	2.58821275073957\\
2.58401385557852	2.56213618744709\\
2.58831904026199	2.53203433579653\\
2.5910870560991	2.4968451036069\\
2.59171006355139	2.45513071157973\\
2.58926545338738	2.40491069026903\\
2.58233419211587	2.34340654738009\\
2.56869787676544	2.26664664449753\\
2.54482641466212	2.16885201665013\\
2.50500751445182	2.04149340089486\\
2.4398917399604	1.87191587913018\\
2.33421854445463	1.64162796522165\\
2.16394499060052	1.32524422830162\\
1.8951439591019	0.893735738631335\\
1.49258625080902	0.330161276340648\\
0.949220739602002	-0.337555860020286\\
0.322047854491551	-1.01666579883714\\
-0.284862839313155	-1.59643693298567\\
-0.79218484411334	-2.02345199049009\\
-1.18011059909213	-2.31015726127132\\
-1.4659367423745	-2.49459919940231\\
-1.67570670773632	-2.61180569021119\\
-1.8316675525631	-2.68636393057746\\
-1.94996553587042	-2.73393415501005\\
-2.04168892856542	-2.76420086393755\\
-2.11435811405785	-2.78315379997754\\
-2.1731056775555	-2.79454577443937\\
-2.22148547198689	-2.80077024555496\\
-2.26200198522569	-2.80338244448942\\
-2.29645300988559	-2.80341066898991\\
-2.3261525625886	-2.80154493910368\\
-2.35207766536885	-2.79825337903581\\
-2.37496647227013	-2.79385534504486\\
-2.39538494038225	-2.78856817949742\\
-2.41377286330605	-2.78253755533497\\
-2.43047614459481	-2.77585739089817\\
-2.44576974145904	-2.7685829824573\\
-2.459874171878	-2.76073961003977\\
-2.47296749859146	-2.75232802354111\\
-2.48519406814302	-2.74332768503392\\
-2.49667086234683	-2.73369829912388\\
-2.50749203283999	-2.7233799285648\\
-2.51773198565752	-2.71229181905291\\
-2.52744722803038	-2.70032991313774\\
-2.53667705929853	-2.68736289432676\\
-2.54544306149733	-2.67322644674672\\
-2.55374720206252	-2.65771521859413\\
-2.56156817541707	-2.64057170660506\\
-2.56885534467735	-2.62147088627847\\
-2.57551924031023	-2.59999882451805\\
-2.58141693080415	-2.57562260863434\\
-2.58632953165119	-2.54764751297536\\
-2.58992736438644	-2.51515507764702\\
-2.59171528196931	-2.47691215235485\\
-2.59094547146427	-2.43123506817258\\
-2.58647587216081	-2.37578350774903\\
-2.57653607048487	-2.30724324881995\\
-2.55833386982738	-2.22083344899758\\
-2.52738724967351	-2.10954318377554\\
-2.47639423807661	-1.96298063156187\\
-2.39338883531493	-1.76578796581867\\
-2.2590688978225	-1.49602457183854\\
-2.04425621765017	-1.12553780210027\\
-1.71218911508775	-0.628210747673746\\
-1.23666007004454	-0.00489431482113671\\
-0.63993220037813	0.68322735642396\\
-0.00973267835069244	1.32417897840899\\
0.553493837725301	1.82937408032667\\
1.00045453225166	2.18209994984005\\
1.33414363399572	2.41274133225413\\
1.57879406457006	2.55982595724772\\
1.75926752282458	2.6532640326178\\
1.89473119297408	2.71280842369238\\
1.99861121015519	2.75079006261327\\
2.08003965083966	2.77481706052269\\
2.14522029762639	2.78962086867523\\
2.19841515595272	2.79819150981641\\
2.24260078315941	2.80245391783868\\
2.27989423025359	2.80367012813669\\
};
\addplot [color=mycolor1, forget plot]
  table[row sep=crcr]{%
2.25788870292592	2.81111255677442\\
2.287654916644	2.81019040218352\\
2.3135286657741	2.80773741291854\\
2.33628740238797	2.8041096403732\\
2.35652357987319	2.79955011947793\\
2.37469512569472	2.79422326875967\\
2.39116055804213	2.78823736127986\\
2.40620378693669	2.78165931984663\\
2.42005186230477	2.77452444794562\\
2.43288781059253	2.766842718315\\
2.44485998250258	2.75860262975216\\
2.45608886382766	2.74977325298756\\
2.46667198535894	2.74030482662775\\
2.47668734784351	2.73012807860742\\
2.48619561592421	2.71915230142912\\
2.49524120452002	2.70726207500777\\
2.50385226155616	2.69431238739626\\
2.51203942295326	2.68012172829357\\
2.5197930568574	2.6644624943301\\
2.52707849316102	2.64704770760462\\
2.53382840332611	2.62751254658506\\
2.53993097497982	2.60538841995103\\
2.54521167952737	2.58006611276808\\
2.54940501935253	2.55074262249637\\
2.55211023338693	2.51634321065097\\
2.55272075042301	2.47540513158836\\
2.55030976292687	2.42590114020182\\
2.5434409975938	2.36496711601297\\
2.5298498499946	2.28847602158638\\
2.50589782519448	2.19036749895199\\
2.46563359806467	2.06160428840486\\
2.399202721541	1.88862746060108\\
2.29033437288762	1.65140967199981\\
2.11317626587243	1.32227278785799\\
1.83135769246533	0.869896633398846\\
1.40881012514373	0.278313060321905\\
0.844418841760294	-0.415371904508185\\
0.208111714600051	-1.10458374955839\\
-0.389353359035489	-1.67551535364681\\
-0.875374687306539	-2.08470429598352\\
-1.23998841711074	-2.35422379466866\\
-1.50565468425842	-2.52567217950787\\
-1.69954564436864	-2.63401111711252\\
-1.84339117402125	-2.70277841441165\\
-1.95247849740747	-2.746644511391\\
-2.03713505805024	-2.77457870618071\\
-2.10430312402153	-2.79209626869318\\
-2.15869821427116	-2.80264369279087\\
-2.20357767997811	-2.80841735787632\\
-2.24123489720429	-2.81084482758961\\
-2.27331599547042	-2.81087078825918\\
-2.30102469660419	-2.80912984941941\\
-2.32525672835316	-2.80605301082385\\
-2.34668950945295	-2.80193453872268\\
-2.36584302602081	-2.7969747305353\\
-2.3831218493064	-2.79130767928843\\
-2.39884459497255	-2.78501949729019\\
-2.41326487171072	-2.77816032594382\\
-2.42658635807579	-2.77075218758484\\
-2.43897375075651	-2.76279396009856\\
-2.45056074742211	-2.75426426916161\\
-2.46145584307613	-2.74512277687487\\
-2.47174645662899	-2.73531012829537\\
-2.48150171724134	-2.72474665468707\\
-2.49077409655662	-2.71332979434373\\
-2.49959995033285	-2.70093005551038\\
-2.50799891171573	-2.68738518956101\\
-2.51597193812646	-2.67249204113197\\
-2.52349762850909	-2.65599526143237\\
-2.53052615923393	-2.63757166082288\\
-2.536969773856	-2.61680835723503\\
-2.54268810059327	-2.59317191853194\\
-2.54746548162895	-2.56596418469749\\
-2.55097565952651	-2.53425802893047\\
-2.55272599747514	-2.49680236516834\\
-2.55196784675265	-2.45187920609796\\
-2.54754975556874	-2.39708483508081\\
-2.53767237367441	-2.32898961584025\\
-2.51947199549834	-2.24260356335451\\
-2.48830469404539	-2.13053737931832\\
-2.43651919657954	-1.98171946016395\\
-2.35142882017189	-1.77960281834186\\
-2.21235184848992	-1.50032235638658\\
-1.98788615858815	-1.11322470622252\\
-1.63913213049626	-0.590916485747844\\
-1.14175486096006	0.0611110024336226\\
-0.528318729159554	0.768681689213856\\
0.101619538049815	1.40957805269479\\
0.648126500738803	1.89991988974157\\
1.07179145063932	2.23433020905319\\
1.38341441913172	2.44974622718435\\
1.61005221608102	2.58601032593121\\
1.77663186292935	2.67225729403111\\
1.90153812700941	2.72716117174515\\
1.99736313946378	2.76219713392125\\
2.07256887160442	2.78438736883833\\
2.13286638976315	2.79808153592387\\
2.18216585835773	2.8060240588645\\
2.22319384147655	2.80998144231508\\
2.25788870292592	2.81111255677442\\
};
\addplot [color=mycolor1, forget plot]
  table[row sep=crcr]{%
2.23394855765595	2.81731297801188\\
2.26165542881003	2.81645433828003\\
2.28578522134835	2.81416644391003\\
2.30704958441646	2.81077666426061\\
2.32599148360855	2.80650857474673\\
2.34303114263776	2.8015133554433\\
2.35849801089891	2.79589030904853\\
2.37265334861332	2.78970036790849\\
2.38570639666023	2.7829749647209\\
2.39782607668333	2.7757217405338\\
2.40914951273934	2.76792800626899\\
2.4197882375849	2.7595625174909\\
2.42983265885312	2.75057588292467\\
2.43935515927449	2.74089975437909\\
2.44841205590173	2.7304448060995\\
2.45704452161019	2.71909738093224\\
2.46527845854013	2.70671453784107\\
2.47312318888068	2.69311705736275\\
2.48056867109479	2.67807971880391\\
2.48758072771249	2.66131781199672\\
2.49409343444114	2.64246831961024\\
2.49999728651665	2.62106339378108\\
2.50512088308885	2.59649247127797\\
2.50920239819958	2.56794731733937\\
2.51184457318115	2.53434093478312\\
2.51244251282485	2.49418572108672\\
2.51006559670834	2.44540696967048\\
2.50326034293511	2.38505228302863\\
2.48971464771938	2.30883202088359\\
2.46567636093457	2.21038701377491\\
2.42493925477907	2.08013241442987\\
2.35710180056824	1.90352032711532\\
2.24478420350019	1.65882250405807\\
2.06013938703502	1.31581788713266\\
1.7641832420183	0.840776377405985\\
1.32043053375636	0.219467138432777\\
0.73538690017579	-0.499760516135056\\
0.0930291108947909	-1.1957590007653\\
-0.491215767880153	-1.75424119406673\\
-0.953776314805486	-2.14377244520633\\
-1.29459482026794	-2.39573992748687\\
-1.54045239114532	-2.55441710429746\\
-1.71905444770675	-2.65421650753045\\
-1.85135954830114	-2.71746709182542\\
-1.95172065990269	-2.75782369546833\\
-2.02969800390966	-2.78355326332276\\
-2.09166991701261	-2.79971502556886\\
-2.14195212072707	-2.80946440618582\\
-2.1835205310336	-2.81481166388316\\
-2.21846928352491	-2.81706417099303\\
-2.24830184335604	-2.81708800387865\\
-2.27411836743984	-2.81546568889455\\
-2.29673831343415	-2.81259331688718\\
-2.31678213074571	-2.8087415448277\\
-2.33472667692994	-2.80409461954302\\
-2.35094346387928	-2.79877572531461\\
-2.36572547954666	-2.79286362401391\\
-2.37930626916953	-2.78640361375265\\
-2.39187367454519	-2.77941467487347\\
-2.4035798143547	-2.77189396598476\\
-2.41454836077782	-2.76381938913056\\
-2.42487981794438	-2.7551506532804\\
-2.43465526858326	-2.74582906428706\\
-2.44393888363475	-2.73577611664615\\
-2.45277935678078	-2.72489082997507\\
-2.46121031049194	-2.71304563903389\\
-2.46924960371401	-2.70008048887152\\
-2.47689733375108	-2.68579458068648\\
-2.48413214009023	-2.66993492373347\\
-2.49090514657346	-2.65218042025219\\
-2.49713045655451	-2.63211955817063\\
-2.50267043458998	-2.60921876941312\\
-2.50731287652463	-2.58277689405653\\
-2.51073524368639	-2.55185857118274\\
-2.51244778648669	-2.51519506832468\\
-2.51170143795885	-2.4710338852086\\
-2.50733563119917	-2.41690644816279\\
-2.49752163633762	-2.34926324313888\\
-2.47932146413312	-2.26289383276216\\
-2.44791996027284	-2.15000411186928\\
-2.39529035629046	-1.99878420638165\\
-2.30795966087572	-1.79137743947912\\
-2.1637256527534	-1.50178070923687\\
-1.9287515730817	-1.09660022170732\\
-1.56201415832366	-0.547363379121762\\
-1.04210332896129	0.134299496204066\\
-0.41373736915889	0.85930176547767\\
0.21212340676912	1.49627049229307\\
0.73880059729606	1.96895776582428\\
1.1379412125118	2.28407109237519\\
1.42753200855995	2.48427879822855\\
1.63667724327522	2.6100323100656\\
1.78996263355016	2.68939756091493\\
1.90484457399082	2.73989491206942\\
1.99304948264513	2.77214413487499\\
2.06237662154612	2.79259916061378\\
2.11806121525891	2.80524509917386\\
2.16367804052527	2.81259382941613\\
2.2017171408998	2.81626251283113\\
2.23394855765595	2.81731297801188\\
};
\addplot [color=mycolor1, forget plot]
  table[row sep=crcr]{%
2.20773459083003	2.8222495289209\\
2.23348473718759	2.82145125522293\\
2.2559548318002	2.8193204935936\\
2.27579468679104	2.81615759288709\\
2.29350076287215	2.81216778393027\\
2.30945784242059	2.80748975993564\\
2.32396803739161	2.80221436187556\\
2.33727129498891	2.79639687695855\\
2.34956008794141	2.79006510231986\\
2.36099004991404	2.78322450845005\\
2.37168772374717	2.77586133018681\\
2.38175620219995	2.7679440877519\\
2.3912791795804	2.75942382039327\\
2.40032374934845	2.75023315421115\\
2.40894214567091	2.74028419309749\\
2.41717251349424	2.72946509451973\\
2.42503868370206	2.71763504945624\\
2.43254880918091	2.70461720498769\\
2.43969256182161	2.6901888180198\\
2.44643636739362	2.6740675635357\\
2.45271581340241	2.65589236827855\\
2.4584238176108	2.63519628236681\\
2.4633922399022	2.61136753834277\\
2.4673630846604	2.5835927402505\\
2.46994277399634	2.550772487407\\
2.4705282381474	2.51139364570898\\
2.46818499718316	2.46333216102615\\
2.46144163796091	2.40354278661851\\
2.44793587476921	2.32756282820322\\
2.42379395539544	2.22871109864616\\
2.38253334906901	2.09680478313707\\
2.31315170340179	1.91620172798163\\
2.19703815751015	1.66327247075068\\
2.00413044856463	1.30496254027981\\
1.69266803865093	0.805057180161984\\
1.22639854495036	0.152164411992385\\
0.621530473963147	-0.591629130270673\\
-0.0230689005423035	-1.29031100961773\\
-0.590042352737044	-1.83246425399385\\
-1.02707637532975	-2.20058405277435\\
-1.34373007916348	-2.43471755813798\\
-1.57016562545039	-2.58086921940254\\
-1.73405441270537	-2.67244926556035\\
-1.85536224159972	-2.73044230004995\\
-1.94744880260296	-2.76747089490889\\
-2.0191063409059	-2.79111442363814\\
-2.07616448903187	-2.80599406743089\\
-2.12255587999007	-2.81498850965447\\
-2.16098911599277	-2.81993202932399\\
-2.19337006048283	-2.82201867867359\\
-2.22106776878537	-2.82204050841864\\
-2.24508498412103	-2.82053101044253\\
-2.26616947929479	-2.8178533986692\\
-2.28488817049593	-2.81425608103213\\
-2.30167738166643	-2.80990816867924\\
-2.3168775422682	-2.80492255517029\\
-2.33075753221499	-2.79937107088254\\
-2.34353201168744	-2.79329445550569\\
-2.3553739069556	-2.78670884188176\\
-2.3664234841957	-2.77960980324398\\
-2.37679496514577	-2.77197461198927\\
-2.38658132134539	-2.76376309272819\\
-2.39585766622337	-2.7549172664856\\
-2.40468350734871	-2.74535983930341\\
-2.41310399834668	-2.73499146118265\\
-2.42115022146144	-2.72368654912106\\
-2.42883841979367	-2.71128730962606\\
-2.4361679632172	-2.69759538525143\\
-2.44311764748936	-2.6823602491983\\
-2.44963965175959	-2.6652630245746\\
-2.45565004906105	-2.64589371824984\\
-2.46101406296593	-2.62371878033168\\
-2.46552308833301	-2.59803416998312\\
-2.46885847617421	-2.56789628085639\\
-2.47053353886647	-2.53201838011155\\
-2.46979887629439	-2.48861229487933\\
-2.46548451503161	-2.4351416266843\\
-2.4557308880001	-2.367930037746\\
-2.43752103026658	-2.28153102169332\\
-2.40585540154001	-2.16771136528395\\
-2.35229751005661	-2.01384958783898\\
-2.26250565626076	-1.80063144759285\\
-2.11258557093622	-1.49966066179349\\
-1.86602728037014	-1.07454491114853\\
-1.4797877932246	-0.496095355052655\\
-0.936805752189655	0.215941648042236\\
-0.295978143629587	0.955565328864142\\
0.321438639117415	1.5841689940824\\
0.825136322831452	2.0363613926554\\
1.19865633278655	2.33130040158839\\
1.46632323755697	2.51636869632571\\
1.65850147464714	2.63192520700021\\
1.79906607079213	2.70470478186873\\
1.90442332523672	2.75101495342029\\
1.98541221796483	2.78062512601975\\
2.0491795001313	2.79943900899214\\
2.10050143774229	2.81109359900658\\
2.14263303395119	2.8178803855583\\
2.17784033336939	2.82127556070905\\
2.20773459083003	2.8222495289209\\
};
\addplot [color=mycolor1, forget plot]
  table[row sep=crcr]{%
2.17877089191157	2.8258491631413\\
2.20266195475253	2.82510825619919\\
2.22355303970033	2.82312699966302\\
2.24203576130143	2.82018025751118\\
2.25856281479918	2.8164559480498\\
2.27348562190955	2.81208097374269\\
2.28708055244381	2.80713818397963\\
2.29956747509971	2.80167753770813\\
2.31112305815506	2.79572340861465\\
2.32189040561136	2.78927923667156\\
2.33198608000072	2.78233027086249\\
2.34150521308389	2.77484485181766\\
2.35052516945291	2.76677448124427\\
2.35910806153742	2.75805277512825\\
2.36730228879221	2.74859327147649\\
2.37514316822329	2.73828593921934\\
2.38265262068332	2.72699209253963\\
2.38983775991052	2.7145372308512\\
2.39668807680493	2.70070106681206\\
2.4031706870237	2.68520362462547\\
2.40922276235853	2.66768571092972\\
2.41473970458509	2.6476811525825\\
2.41955668385167	2.62457674239169\\
2.42341956070791	2.59755346032928\\
2.42593840079165	2.56549858692832\\
2.42651175108008	2.52687163564035\\
2.42420061301231	2.47949555154823\\
2.41751383304105	2.42022482642527\\
2.40403424708072	2.34440848059438\\
2.37975463609059	2.24501166999273\\
2.33788680348285	2.11118778930979\\
2.26675737864986	1.92606679741267\\
2.14637058166322	1.66387081861997\\
1.94418590217523	1.28837621186511\\
1.61553544103642	0.760902998569861\\
1.12539644326509	0.0745020623084606\\
0.502231119522599	-0.692028792653104\\
-0.139898794240548	-1.3883092500649\\
-0.685264967956727	-1.90997587733958\\
-1.09481076588507	-2.25501661218449\\
-1.38703683868072	-2.47111344877806\\
-1.59446713793099	-2.60500512550983\\
-1.74420428353216	-2.68867837342763\\
-1.85503000691998	-2.74165972225064\\
-1.93926554994634	-2.77553050794186\\
-2.00493893990024	-2.79719875117125\\
-2.0573470000794	-2.81086507442919\\
-2.10005537972538	-2.81914490964075\\
-2.13551861595477	-2.82370597694059\\
-2.16546446419626	-2.82563535550257\\
-2.19113517288083	-2.82565529704955\\
-2.21344172129508	-2.82425307049057\\
-2.23306440504989	-2.821760889237\\
-2.2505197612451	-2.81840617041766\\
-2.26620596485001	-2.81434373682659\\
-2.28043418264582	-2.80967676338909\\
-2.29345058827861	-2.80447053516409\\
-2.30545204646153	-2.79876149299468\\
-2.31659742199454	-2.79256309496412\\
-2.32701580304022	-2.78586944178028\\
-2.33681249704063	-2.77865724760119\\
-2.3460733713891	-2.770886495416\\
-2.35486791396654	-2.76249994441075\\
-2.36325124550464	-2.75342152166876\\
-2.37126520234114	-2.74355350788806\\
-2.37893850604596	-2.7327722961707\\
-2.38628592868827	-2.72092234306972\\
-2.39330622982574	-2.70780771495221\\
-2.39997845694397	-2.69318032125385\\
-2.40625592354149	-2.67672345836231\\
-2.41205673908975	-2.65802856431732\\
-2.4172490422457	-2.63656193879927\\
-2.42162786717519	-2.61161633019505\\
-2.42487845667244	-2.58223923626837\\
-2.42651708281877	-2.54712463397145\\
-2.42579363098841	-2.50444610188262\\
-2.42152761816296	-2.45159422103287\\
-2.41182571193496	-2.38475522131539\\
-2.39358448774803	-2.29822459719431\\
-2.36160130059077	-2.18328435227048\\
-2.30698424867367	-2.02640706535026\\
-2.21441724776119	-1.80663567049949\\
-2.05810232219544	-1.49287274945234\\
-1.79859143345776	-1.0454639060061\\
-1.39108370550384	-0.435140869323708\\
-0.82480235487054	0.307613699130504\\
-0.174921739515048	1.05796528414176\\
0.429057793926385	1.67312163418713\\
0.906601218695649	2.10195276089735\\
1.25353586858761	2.37594377011342\\
1.4994538511774	2.5459881258258\\
1.67519385396721	2.65166344876194\\
1.80358749812364	2.71814129356404\\
1.89989066764397	2.7604709173079\\
1.97404145527801	2.78758017016175\\
2.03254641694308	2.8048407202027\\
2.0797395803064	2.81555711454915\\
2.11857098081715	2.82181180533255\\
2.15109432677297	2.82494776723096\\
2.17877089191157	2.8258491631413\\
};
\addplot [color=mycolor1, forget plot]
  table[row sep=crcr]{%
2.14638286171033	2.82796423666817\\
2.1685094971813	2.82727778496317\\
2.18790043752099	2.82543857464301\\
2.20509246287125	2.82269741653787\\
2.22049703837008	2.81922588001395\\
2.23443415506741	2.81513972630638\\
2.24715592449306	2.81051424969982\\
2.25886328880231	2.8053943710103\\
2.2697180143095	2.79980122861067\\
2.27985138828809	2.79373634775355\\
2.28937056027584	2.78718405488052\\
2.29836315520485	2.78011253502649\\
2.30690057317799	2.77247374551488\\
2.315040240036	2.76420225953563\\
2.32282695786662	2.75521299287993\\
2.33029340631105	2.74539764554835\\
2.33745974772071	2.73461954720598\\
2.34433217475774	2.72270640758823\\
2.3509000855707	2.70944020643105\\
2.3571313455368	2.6945430608253\\
2.36296474045027	2.6776572978299\\
2.36829814852421	2.6583169976477\\
2.37296998737321	2.63590671969835\\
2.37672981520411	2.60960056517011\\
2.37919099741342	2.57827042825493\\
2.37975296956227	2.54034492229864\\
2.37747065233213	2.4935876616492\\
2.37082972145653	2.43474115356513\\
2.35735041420007	2.35894366511076\\
2.33287430374968	2.25876240221542\\
2.29026634313273	2.12259835813708\\
2.21708805566767	1.93217933656367\\
2.09176081156176	1.65926844659496\\
1.87895213931415	1.26409107160541\\
1.53103973962933	0.705711375456089\\
1.01576355532262	-0.0160168121401194\\
0.376896810506748	-0.802150579667117\\
-0.256940380314764	-1.48973187510391\\
-0.776081907587097	-1.98647819331637\\
-1.15629517575985	-2.30686986686515\\
-1.423932858916	-2.50480321137639\\
-1.61280149684316	-2.6267181121602\\
-1.74893632450621	-2.70279042246697\\
-1.84977142266498	-2.75099469191492\\
-1.9265562643376	-2.78186855061526\\
-1.98656202171616	-2.80166594608952\\
-2.03456899024968	-2.81418391236825\\
-2.07379132074096	-2.82178736325071\\
-2.10644193588614	-2.82598625428656\\
-2.13407993566315	-2.82776659093309\\
-2.1578277979326	-2.82778475114576\\
-2.17850995701177	-2.82648439457411\\
-2.19674311789018	-2.8241684818417\\
-2.21299635522654	-2.82104461336558\\
-2.22763190433368	-2.81725411473201\\
-2.24093335951189	-2.81289097509428\\
-2.25312549401949	-2.80801429077649\\
-2.26438839561164	-2.80265643842267\\
-2.27486766896483	-2.79682835032\\
-2.28468185966512	-2.79052274220007\\
-2.29392786811972	-2.78371581231931\\
-2.3026848646701	-2.7763677098011\\
-2.31101703952361	-2.76842191179093\\
-2.31897539098827	-2.75980352179609\\
-2.32659865096067	-2.75041638306813\\
-2.33391335056484	-2.74013877136725\\
-2.3409329250638	-2.72881726957966\\
-2.34765562652752	-2.71625820452968\\
-2.35406082818988	-2.70221570273356\\
-2.36010302314961	-2.68637493163734\\
-2.36570236978041	-2.66832832882923\\
-2.37072988989644	-2.64754140238456\\
-2.37498415356	-2.62330269720879\\
-2.37815406021903	-2.59464921277142\\
-2.37975834147586	-2.56025294094827\\
-2.37904510441422	-2.51824449556296\\
-2.37482104878933	-2.46593286163663\\
-2.36515394587688	-2.39935067236393\\
-2.34684223351532	-2.31250396918446\\
-2.31445313585167	-2.19612735400624\\
-2.2585767975166	-2.03566252518764\\
-2.16278387532862	-1.80827193345053\\
-1.99910792181745	-1.47978249385533\\
-1.72488195578284	-1.00704113098572\\
-1.29408696393288	-0.361795492332068\\
-0.704865784088944	0.41126327946243\\
-0.0506195584320581	1.16697280720031\\
0.534227949968228	1.76287629141842\\
0.982439525342405	2.16547356894941\\
1.3019570069277	2.41784782802175\\
1.52636237507873	2.57302658281129\\
1.68619549819523	2.66913835746718\\
1.80294774058144	2.72958802234225\\
1.890643171054	2.76813310488603\\
1.95831241293265	2.79287176878262\\
2.01183572375014	2.80866181552588\\
2.05512133356105	2.81849028502511\\
2.09082835129564	2.82424123560066\\
2.12080904146535	2.82713164121354\\
2.14638286171033	2.82796423666817\\
};
\addplot [color=mycolor1, forget plot]
  table[row sep=crcr]{%
2.10959639276541	2.82833170033868\\
2.13005277719103	2.82769680200034\\
2.14802287471214	2.82599213309054\\
2.16399176571052	2.82344580125796\\
2.17833205481083	2.82021393645668\\
2.19133411253028	2.81640177024618\\
2.20322718958303	2.81207744870569\\
2.21419439157868	2.80728111824648\\
2.22438344129317	2.80203084312009\\
2.23391449132048	2.79632631929519\\
2.24288582434271	2.79015097768998\\
2.25137799857592	2.78347282702145\\
2.25945680590028	2.7762442174586\\
2.26717527446035	2.7684005761966\\
2.27457484262639	2.75985805103207\\
2.28168573976336	2.75050987848557\\
2.28852651598563	2.7402211492275\\
2.29510255122665	2.72882145096567\\
2.30140322103382	2.71609459233055\\
2.30739716790373	2.70176419574736\\
2.31302476509234	2.68547330303474\\
2.31818626415068	2.66675511304507\\
2.32272310676726	2.64499030425581\\
2.32638812037928	2.61934362489206\\
2.32879716902256	2.58866772920285\\
2.32934906157394	2.5513540908222\\
2.32708969033154	2.5050964728495\\
2.32047561983038	2.44650691445407\\
2.3069520645374	2.37047913665329\\
2.28218252151097	2.26911845209114\\
2.23862573128952	2.12995002321369\\
2.16294815306376	1.93306507749366\\
2.03172785596982	1.64737025756866\\
1.80647115907523	1.22912474961823\\
1.43674425455007	0.635722130761753\\
0.895415675386827	-0.122678493566976\\
0.24506740533721	-0.923290250180105\\
-0.373302970705661	-1.59439810765956\\
-0.861342602045722	-2.06153432057587\\
-1.21051470427877	-2.35582021146555\\
-1.45350302607409	-2.53553747281427\\
-1.62427912108232	-2.6457756283534\\
-1.74735161713476	-2.71454779213463\\
-1.83866892421111	-2.75820072395713\\
-1.90838558633582	-2.78623139232265\\
-1.96302614690974	-2.80425772800326\\
-2.00687082453237	-2.81568963402831\\
-2.04279713811814	-2.82265357129101\\
-2.07278812116456	-2.82650998992637\\
-2.09824304490626	-2.82814934742327\\
-2.12017116470066	-2.82816582669431\\
-2.13931522377064	-2.826961933092\\
-2.15623195525253	-2.82481302044249\\
-2.17134569327871	-2.82190798040682\\
-2.18498480181322	-2.81837538766676\\
-2.19740689106534	-2.81430054547294\\
-2.20881656655742	-2.80973669054728\\
-2.21937810527049	-2.8047123437088\\
-2.22922461613086	-2.79923603208632\\
-2.23846471192486	-2.79329914096155\\
-2.24718737595216	-2.78687735484354\\
-2.25546547731453	-2.7799309467999\\
-2.26335822936359	-2.77240402898462\\
-2.27091276797016	-2.76422275734835\\
-2.27816493006503	-2.75529236858525\\
-2.28513922246038	-2.74549279836288\\
-2.2918478707394	-2.73467246525985\\
-2.2982887089987	-2.72263957559321\\
-2.30444148590314	-2.70914996681804\\
-2.31026187654082	-2.69388999154343\\
-2.31567202720427	-2.67645213420482\\
-2.32054568700573	-2.65629974947181\\
-2.32468465040066	-2.63271516862803\\
-2.32778088753628	-2.60472181958332\\
-2.32935449079792	-2.57096482990594\\
-2.32864968427513	-2.52952378807117\\
-2.32445618534273	-2.47761220956851\\
-2.31479431693115	-2.41108425761014\\
-2.29634617602986	-2.32361020748856\\
-2.26340897464661	-2.20528850710705\\
-2.20596668980778	-2.04035887822781\\
-2.10628907021382	-1.80379023894369\\
-1.93390602004408	-1.45787754842354\\
-1.64266626013909	-0.955835256982624\\
-1.18636447694275	-0.272301910216555\\
-0.575627521932152	0.529268583185396\\
0.0765700150810036	1.28296729173816\\
0.635827895370091	1.85302389135356\\
1.05156023208183	2.22653834661497\\
1.34296588926262	2.45673552212428\\
1.54615322639242	2.59724775265813\\
1.69061445079945	2.68411603245696\\
1.79623883340552	2.73880286760147\\
1.87575398878987	2.77375114085251\\
1.93728218443528	2.79624367598011\\
1.98609242944908	2.81064247686136\\
2.02568323586637	2.81963136032678\\
2.05843611040654	2.82490602040426\\
2.08601207953629	2.8275641952934\\
2.10959639276541	2.82833170033868\\
};
\addplot [color=mycolor1, forget plot]
  table[row sep=crcr]{%
2.0669689142298	2.8265001166603\\
2.08585218487773	2.82591377173339\\
2.10248435713509	2.824335791098\\
2.11730176510645	2.82197287100359\\
2.13064039591691	2.81896657267359\\
2.14276272243324	2.81541218008658\\
2.15387647302746	2.8113710713481\\
2.16414796811114	2.80687885374025\\
2.17371172139776	2.80195064490338\\
2.18267741955627	2.79658435510239\\
2.19113501896617	2.79076249383181\\
2.19915845114387	2.78445280557154\\
2.2068082596289	2.77760788518744\\
2.21413336951179	2.77016380211266\\
2.22117209536673	2.76203765196474\\
2.227952408255	2.75312383596871\\
2.23449139325266	2.74328872273134\\
2.24079371916766	2.73236314815236\\
2.24684878895986	2.72013192028773\\
2.25262600707546	2.70631905782578\\
2.25806722804191	2.69056680598815\\
2.26307483266911	2.67240537597125\\
2.26749281998738	2.65120855342423\\
2.27107644143286	2.62612730046412\\
2.27344254173279	2.59598829347994\\
2.27398653889816	2.55913525875812\\
2.27174012871236	2.51317476307575\\
2.26512077120442	2.45455886598506\\
2.25147861067759	2.37788450758131\\
2.22625743325768	2.27469939362629\\
2.18142130782272	2.13147407201847\\
2.10255582239114	1.92633668518003\\
1.96404509251501	1.62482097010485\\
1.72381856839937	1.17882240111403\\
1.329183818134	0.545395040165709\\
0.761790068907563	-0.249787526505068\\
0.106628164730735	-1.05674278201958\\
-0.487505353795216	-1.70185391858681\\
-0.939357911409621	-2.13448330265905\\
-1.25594103658005	-2.40134149815461\\
-1.47431937685956	-2.56286478580134\\
-1.62749848260503	-2.66174359928501\\
-1.73804377195806	-2.72351381077397\\
-1.82030363294048	-2.76283520906365\\
-1.8833232358948	-2.78817183992138\\
-1.93289296053334	-2.80452423009049\\
-1.97280938807871	-2.81493111993595\\
-2.00562765622305	-2.82129200404624\\
-2.03311197406603	-2.82482564197129\\
-2.05651006259846	-2.82633216974524\\
-2.07672408698264	-2.82634706367953\\
-2.09441965051151	-2.82523401181922\\
-2.11009691749222	-2.82324233404957\\
-2.12413805178504	-2.82054327423604\\
-2.13683950901854	-2.81725336930394\\
-2.1484344317427	-2.81344970908028\\
-2.15910844240422	-2.80917997274218\\
-2.16901094268674	-2.80446900297834\\
-2.178263291945	-2.79932300498311\\
-2.18696477081792	-2.79373204110065\\
-2.19519693282776	-2.78767122462506\\
-2.20302674361282	-2.78110083463718\\
-2.21050876536302	-2.77396543908935\\
-2.2176865377223	-2.76619199986685\\
-2.22459321797863	-2.75768682147306\\
-2.23125145805402	-2.74833107563165\\
-2.23767239863978	-2.73797446554853\\
-2.24385353281524	-2.72642635543562\\
-2.24977500441801	-2.71344333660651\\
-2.25539361411527	-2.69871165550837\\
-2.26063332867985	-2.68182206458618\\
-2.26537028309853	-2.66223325496721\\
-2.26940886619963	-2.63921770416699\\
-2.27244298687703	-2.61177982613277\\
-2.27399205537368	-2.57852946857343\\
-2.27329264767083	-2.53748169639891\\
-2.26911034445943	-2.4857320441213\\
-2.25940393141311	-2.41891713013389\\
-2.24071038667139	-2.33030101858926\\
-2.20699631199343	-2.20921528422324\\
-2.1475106522554	-2.03845428181591\\
-2.04296071306462	-1.79036922549699\\
-1.85994625905196	-1.42317525718662\\
-1.5486658362723	-0.886598994775004\\
-1.06463898401359	-0.161388990802826\\
-0.435691160374269	0.6644529412044\\
0.205745366158514	1.40610510429633\\
0.732167458520347	1.9429031736525\\
1.11235206856804	2.28455348115237\\
1.37509630217415	2.492128076491\\
1.55741791816583	2.61821320681005\\
1.68704870813498	2.69616214335981\\
1.78204832273847	2.74534615469033\\
1.85379818103074	2.77687987495302\\
1.90951601160325	2.79724714593898\\
1.95387541865583	2.81033207185613\\
1.98998084771081	2.81852894203929\\
2.01994885409201	2.82335459591496\\
2.04525882008697	2.8257939325355\\
2.0669689142298	2.8265001166603\\
};
\addplot [color=mycolor1, forget plot]
  table[row sep=crcr]{%
2.01629649870323	2.8216947796992\\
2.03371192787167	2.82115372757937\\
2.04909758148215	2.8196937697169\\
2.06284386947368	2.81750145142994\\
2.07525232111241	2.81470461391784\\
2.08655918486678	2.81138915448681\\
2.09695197926938	2.80761003738521\\
2.10658128152195	2.80339853083658\\
2.11556923308719	2.79876688561593\\
2.12401573403612	2.79371120614847\\
2.13200297131505	2.78821297113196\\
2.13959870991857	2.78223946498994\\
2.14685862733202	2.77574324077276\\
2.15382786330377	2.76866062152778\\
2.16054187042183	2.76090914028469\\
2.16702657163593	2.75238370062266\\
2.17329774507153	2.74295109035814\\
2.17935944791232	2.73244227361185\\
2.1852011364565	2.72064158145531\\
2.1907929012884	2.7072714544067\\
2.19607785082874	2.69197065410794\\
2.20096002912548	2.67426267219099\\
2.20528513440281	2.65350909646456\\
2.20880931586443	2.62883936170034\\
2.21114769507717	2.59904253518429\\
2.2116874471933	2.56239654634731\\
2.20943715928125	2.51639175507511\\
2.20275831617271	2.45727185066445\\
2.188872968427	2.37925324006683\\
2.1629384810545	2.27317642871619\\
2.11628700004405	2.1241865211064\\
2.03314705487453	1.90797639336758\\
1.88522789170033	1.58603353217718\\
1.62647712510769	1.10567006058556\\
1.20337195083471	0.426430541886378\\
0.611915909614419	-0.402858236553646\\
-0.0377639343599761	-1.20354789310859\\
-0.597100482331314	-1.81117113185058\\
-1.00757427958943	-2.20428707001593\\
-1.29021696967227	-2.44256084179465\\
-1.48412953405143	-2.58599032846721\\
-1.62023730138053	-2.67384690614635\\
-1.71879227503168	-2.72891443587442\\
-1.79245059734471	-2.76412199699261\\
-1.84914108871065	-2.78691245609935\\
-1.89393410940838	-2.80168792222779\\
-1.93015885351913	-2.81113148167195\\
-1.96006170027521	-2.81692666023866\\
-1.98519848830143	-2.82015798991299\\
-2.00667318546355	-2.82154029086012\\
-2.02528665005883	-2.82155369216577\\
-2.0416316546784	-2.82052532925747\\
-2.05615503547555	-2.81868002036959\\
-2.06919925101121	-2.81617239719858\\
-2.08103074658255	-2.81310765133649\\
-2.09185967675809	-2.80955510860833\\
-2.10185384863354	-2.80555716250382\\
-2.11114872107	-2.80113511420794\\
-2.11985465684564	-2.79629287490345\\
-2.12806221881559	-2.79101911843369\\
-2.13584603650612	-2.78528823438026\\
-2.14326759134354	-2.77906026763665\\
-2.15037714299369	-2.77227990629132\\
-2.15721492382378	-2.76487447169454\\
-2.16381164722527	-2.75675075432964\\
-2.17018829476462	-2.74779040808994\\
-2.17635505229918	-2.73784344117483\\
-2.18230913715818	-2.72671909182738\\
-2.1880310680229	-2.7141730012679\\
-2.19347862762006	-2.69988901191023\\
-2.19857727062877	-2.68345298618764\\
-2.20320488089757	-2.66431451591879\\
-2.20716729498335	-2.64172983860735\\
-2.21015833063649	-2.61467490171091\\
-2.2116931024764	-2.58170984489889\\
-2.21099398493157	-2.54076243038014\\
-2.20679020984049	-2.48877291954632\\
-2.19695552159172	-2.42109700701258\\
-2.17783500661387	-2.33048097653156\\
-2.142969037832	-2.20528818446957\\
-2.08067540438981	-2.02650626200969\\
-1.96972237526817	-1.76327837650438\\
-1.77324535376412	-1.36912188000483\\
-1.43794190280115	-0.791111827159751\\
-0.924545711399306	-0.0216540259654149\\
-0.283953095218948	0.819957185386306\\
0.335043173914783	1.53607343206824\\
0.820645184263823	2.03143236592395\\
1.16236468113286	2.33857029213583\\
1.39605637439492	2.52320245266567\\
1.55792497526828	2.63514208823294\\
1.67327854150371	2.70450306114709\\
1.75815357520141	2.74844279268488\\
1.82254887470734	2.77674235019102\\
1.87278527967306	2.79510456930404\\
1.91295729365938	2.80695333270049\\
1.94579044293073	2.81440658009161\\
1.97314833573224	2.81881138097168\\
1.99633775367937	2.82104590983873\\
2.01629649870323	2.8216947796992\\
};
\addplot [color=mycolor1, forget plot]
  table[row sep=crcr]{%
1.95408713286161	2.81255877444144\\
1.97015665108589	2.81205922536446\\
1.98440383011239	2.81070703591457\\
1.99717590274892	2.80866986239339\\
2.00874206890337	2.80606267207798\\
2.01931402447436	2.80296252225127\\
2.02906040923129	2.79941828627224\\
2.03811713116402	2.79545704113902\\
2.04659483732625	2.79108817309935\\
2.05458436756647	2.78630585336903\\
2.06216074721035	2.78109027757561\\
2.06938608809292	2.77540788802964\\
2.07631163782244	2.76921066958993\\
2.08297912134429	2.76243450284066\\
2.08942144040516	2.75499645363844\\
2.09566272220139	2.74679075818894\\
2.10171762517856	2.73768310713825\\
2.10758970122228	2.72750261231144\\
2.11326845507579	2.71603051211328\\
2.1187244940024	2.70298416525528\\
2.12390175446257	2.68799407744965\\
2.12870510331387	2.67057039352651\\
2.13298040690076	2.65005309663722\\
2.13648199862587	2.62553640960846\\
2.13881848174384	2.5957513300483\\
2.13936021939874	2.55887846075589\\
2.13707705910718	2.51224175080742\\
2.13024522821122	2.4517937944577\\
2.11590219113997	2.37122954964349\\
2.08880702511242	2.26043600118698\\
2.03943638612533	2.10280117567733\\
1.95023384582017	1.87087470725969\\
1.78957421552096	1.52125205780096\\
1.50724163123535	0.997097348433753\\
1.05218124740212	0.266322314398444\\
0.442891559848105	-0.58852021357033\\
-0.185833290433792	-1.36392590129485\\
-0.698009436945026	-1.92057888359518\\
-1.06198920354467	-2.26924086719763\\
-1.30958905743568	-2.47798367576543\\
-1.4792957117902	-2.60350567562873\\
-1.59889578491176	-2.68070188805705\\
-1.68600931275967	-2.72937275373668\\
-1.75152908476803	-2.76068754710666\\
-1.80226698636391	-2.78108311856666\\
-1.84258863188146	-2.79438234601673\\
-1.87537167226418	-2.80292773170737\\
-1.90256666928812	-2.80819742761536\\
-1.92553095322346	-2.81114894635193\\
-1.94523210358837	-2.81241666419805\\
-1.96237516540708	-2.81242866498418\\
-1.97748420843269	-2.81147777919341\\
-1.99095587711757	-2.80976585505906\\
-2.00309533870785	-2.80743194832296\\
-2.01414091135235	-2.80457058999968\\
-2.02428125088093	-2.80124376919405\\
-2.03366754272059	-2.79748882437222\\
-2.0424222723539	-2.79332358755083\\
-2.05064560332622	-2.78874961214502\\
-2.05842004429573	-2.78375399369517\\
-2.06581385882923	-2.77831008210545\\
-2.07288351712786	-2.77237723631783\\
-2.07967537858105	-2.76589965726303\\
-2.08622670867571	-2.75880423126099\\
-2.09256605897104	-2.75099720604875\\
-2.09871296179467	-2.7423593867516\\
-2.10467679770348	-2.73273935563581\\
-2.1104545639566	-2.72194395253888\\
-2.11602707550614	-2.70972484716044\\
-2.12135281425098	-2.69575939801092\\
-2.12635811517624	-2.67962296813396\\
-2.1309214701633	-2.66074817727884\\
-2.13484812202904	-2.63836371385647\\
-2.13782819258103	-2.61140038594983\\
-2.13936610387029	-2.57834332604201\\
-2.13865849025632	-2.53699337538637\\
-2.13437691750635	-2.48407135459853\\
-2.12426956517293	-2.41454448693609\\
-2.10441034382447	-2.32045540155919\\
-2.06775511912624	-2.18887241749459\\
-2.00137768373468	-1.99841461647744\\
-1.8815491498074	-1.71418308494054\\
-1.6673253143189	-1.28445643542233\\
-1.30292284263472	-0.656185427944006\\
-0.760561802985589	0.157087992584297\\
-0.120400018076629	0.99873582485722\\
0.46087903608505	1.67162366241341\\
0.897137918336744	2.11679559453497\\
1.19773552170667	2.38700539604759\\
1.40216485547805	2.54851917659725\\
1.54406119305533	2.64664238340825\\
1.64571156583332	2.70775941662276\\
1.72097033471938	2.74671763256643\\
1.77842930535205	2.7719666864416\\
1.82352311167654	2.78844764991827\\
1.85978356502206	2.79914158907219\\
1.88957179553683	2.80590281834557\\
1.91450981240302	2.80991739269252\\
1.93574032729193	2.81196268811317\\
1.95408713286161	2.81255877444144\\
};
\addplot [color=mycolor1, forget plot]
  table[row sep=crcr]{%
1.87457785195785	2.79663504639062\\
1.88945545218441	2.79617219509606\\
1.90270374359799	2.79491450953976\\
1.91462952242024	2.79301206176028\\
1.92547182633201	2.79056780972743\\
1.93541953820537	2.78765050384051\\
1.94462383638467	2.78430319829366\\
1.95320713164883	2.78054883237584\\
1.96126955988095	2.77639378663779\\
1.96889373707553	2.7718299711776\\
1.97614824742985	2.76683577821289\\
1.98309017679418	2.76137607604001\\
1.98976689233669	2.7554013038897\\
1.99621718508488	2.74884562490728\\
2.0024718207702	2.74162398977357\\
2.00855347402061	2.73362783867479\\
2.01447593856784	2.7247190026417\\
2.02024239483717	2.71472112543319\\
2.02584235012124	2.70340756507238\\
2.03124660207401	2.69048416881407\\
2.03639913691442	2.67556440825667\\
2.04120411953491	2.65813286935502\\
2.04550480009076	2.63749057763998\\
2.04904874009366	2.61267129583654\\
2.05142923430248	2.58231024186075\\
2.05198409645007	2.54443273360228\\
2.04961575014777	2.49610446347231\\
2.04246165430275	2.4328367742712\\
2.02727238009869	2.3475506408997\\
1.99820950966423	2.22874757739577\\
1.9445086891535	2.05733205488968\\
1.84614113374679	1.80162784832694\\
1.66730383173018	1.41247595138888\\
1.35437802897999	0.831416506907343\\
0.866004494455484	0.0466867728527042\\
0.253507304421055	-0.813411613844761\\
-0.332010221993016	-1.53608422884368\\
-0.783291433215933	-2.02675635656377\\
-1.0960575116577	-2.32639806983683\\
-1.30784508177712	-2.50494509822122\\
-1.45374421032671	-2.612849065112\\
-1.55745382918991	-2.67978157783237\\
-1.63370370471678	-2.7223779641566\\
-1.6915732725475	-2.75003312998556\\
-1.73676242378907	-2.76819602222803\\
-1.77294782937584	-2.78012947651889\\
-1.80257071204522	-2.78785003090417\\
-1.82729792907952	-2.79264072957431\\
-1.84829752840052	-2.79533912096034\\
-1.8664076462336	-2.79650397780766\\
-1.8822427616688	-2.79651467256384\\
-1.89626224752837	-2.79563203275335\\
-1.90881565562071	-2.79403651912157\\
-1.92017329436437	-2.79185267951468\\
-1.93054729272853	-2.78916507242008\\
-1.94010637508435	-2.78602874578762\\
-1.94898639167708	-2.78247614163633\\
-1.9572979261832	-2.7785215770886\\
-1.9651318482346	-2.7741640131358\\
-1.9725633874208	-2.76938854442878\\
-1.97965511282255	-2.76416685834055\\
-1.98645907017771	-2.75845677846827\\
-1.99301823301222	-2.75220090018766\\
-1.99936734793394	-2.74532422449441\\
-2.00553318496923	-2.73773058396824\\
-2.0115341292134	-2.72929751211378\\
-2.01737895576195	-2.71986900884221\\
-2.02306449523449	-2.70924536160379\\
-2.02857168889946	-2.69716873091932\\
-2.03385919308024	-2.6833024952447\\
-2.03885311904793	-2.66720119032324\\
-2.04343049559734	-2.64826594721389\\
-2.04739225140634	-2.62567703875512\\
-2.05041821524351	-2.5982893830741\\
-2.05199037599498	-2.56446652853883\\
-2.05125843748935	-2.52180971963489\\
-2.04679724268251	-2.46670334973505\\
-2.03615563241697	-2.39353204771595\\
-2.01499378005436	-2.29330477661769\\
-1.97540492246692	-2.15123132676112\\
-1.90269410937579	-1.94265037912345\\
-1.76980293844462	-1.62748139356961\\
-1.53124519946957	-1.14892440398809\\
-1.13209359195833	-0.460484046162954\\
-0.566808533969557	0.387816394586287\\
0.0520366019978877	1.20215165598319\\
0.576535381829118	1.80967253888786\\
0.954864286568924	2.19582800001589\\
1.21211636398751	2.42708312606015\\
1.38729554854803	2.56547899463559\\
1.50978522747304	2.65017391625239\\
1.59834333757868	2.70341329806673\\
1.66452000788624	2.73766611350783\\
1.71548697704509	2.7600597349861\\
1.7558056309989	2.77479364434049\\
1.7884611523011	2.78442313328132\\
1.81546395096188	2.7905512025093\\
1.83820504074484	2.794211411217\\
1.8576710696949	2.7960861786337\\
1.87457785195785	2.79663504639062\\
};
\addplot [color=mycolor1, forget plot]
  table[row sep=crcr]{%
1.76784387513612	2.76928910145383\\
1.78174633759923	2.76885615522374\\
1.79419693541599	2.76767383044685\\
1.80546484911209	2.76587600985613\\
1.81576126131739	2.76355453796707\\
1.82525415931308	2.76077035293493\\
1.83407887909368	2.75756084426591\\
1.84234572307335	2.75394466141325\\
1.85014552584246	2.74992473191324\\
1.85755374913669	2.7454899539701\\
1.8646334942403	2.74061583435693\\
1.87143768860931	2.73526420470525\\
1.87801060916971	2.72938204000661\\
1.88438883113162	2.72289930259853\\
1.89060162587557	2.7157256259312\\
1.89667076319854	2.70774551650771\\
1.90260958893445	2.69881156524049\\
1.90842113042871	2.68873488479607\\
1.91409480045653	2.67727156911894\\
1.91960097508577	2.66410330775859\\
1.92488222337915	2.6488092124617\\
1.92983910128206	2.63082412809549\\
1.9343068731227	2.60937566180876\\
1.93801667340984	2.58338686500575\\
1.94052922309179	2.55132202917289\\
1.94111869405914	2.51093572559874\\
1.93856325046462	2.45885292681666\\
1.93075569678326	2.38984749944438\\
1.91395891517441	2.29557553517638\\
1.88135412086077	2.16233871312354\\
1.82023064403477	1.96727962024497\\
1.70698513939229	1.67293893640523\\
1.5009626367681	1.22458307939976\\
1.14920561952737	0.571075287988628\\
0.634671529376524	-0.256529150340702\\
0.0487615290360517	-1.08023744885568\\
-0.464160373241134	-1.71379205799029\\
-0.840794541274635	-2.12341792386813\\
-1.09857110524551	-2.37037717746715\\
-1.27425441166773	-2.51846834010297\\
-1.39693091863683	-2.60918252109282\\
-1.48545234652581	-2.66630303200044\\
-1.55147603147699	-2.70318039482592\\
-1.60224137064384	-2.72743650225245\\
-1.64234632461511	-2.74355321834049\\
-1.67479425152706	-2.75425225602386\\
-1.70160378481403	-2.76123825526265\\
-1.72416901355081	-2.76560911350238\\
-1.74347710347502	-2.76808940820817\\
-1.76024319539013	-2.76916723159805\\
-1.77499620104009	-2.7691767210339\\
-1.78813472615172	-2.76834914861335\\
-1.79996436786742	-2.76684528428748\\
-1.8107231218077	-2.76477629836321\\
-1.82059902686968	-2.76221746239001\\
-1.82974263443093	-2.75921720197294\\
-1.83827595617522	-2.75580306113597\\
-1.84629896779417	-2.75198554221772\\
-1.85389438056917	-2.7477604169714\\
-1.86113115584889	-2.7431098675017\\
-1.8680670790099	-2.73800265396649\\
-1.87475059904002	-2.73239338527917\\
-1.88122205763355	-2.72622086635664\\
-1.88751436365687	-2.71940539282717\\
-1.89365310351775	-2.71184474423268\\
-1.89965600355715	-2.70340846884428\\
-1.90553156197023	-2.69392982790673\\
-1.91127652217843	-2.68319442880026\\
-1.91687162919518	-2.67092405027451\\
-1.92227472893751	-2.6567533208856\\
-1.92740961653825	-2.64019553048539\\
-1.93214788588446	-2.62059153274098\\
-1.93627893869243	-2.59703169657767\\
-1.93945940251928	-2.56823380041851\\
-1.9411256975553	-2.53234698407273\\
-1.94033865257595	-2.48662825826946\\
-1.93549900187693	-2.42689383328213\\
-1.92381069939923	-2.34656509677893\\
-1.90024268269791	-2.23498378595543\\
-1.85550126153486	-2.07446712487143\\
-1.77220791794184	-1.83557709268416\\
-1.61884814998775	-1.47187517105885\\
-1.34621301713298	-0.924796279998125\\
-0.909481515083475	-0.170949326882521\\
-0.340914514642232	0.683260693916561\\
0.223422225044149	1.42661131604494\\
0.669486332528397	1.9435442286894\\
0.982190677905218	2.26275433782589\\
1.19459131176185	2.45367579279806\\
1.34083225948148	2.56919246942168\\
1.44460153384385	2.6409310904603\\
1.52074563315668	2.686699595453\\
1.57843171408775	2.71655270800445\\
1.6234100295506	2.73631178759889\\
1.65938328561586	2.74945551357152\\
1.68880511280075	2.75812989491714\\
1.71334758481246	2.76369848551413\\
1.73418031619097	2.76705069509871\\
1.75214124503063	2.76877985050119\\
1.76784387513612	2.76928910145383\\
};
\addplot [color=mycolor1, forget plot]
  table[row sep=crcr]{%
1.61613855159024	2.72141247921817\\
1.62941391966627	2.72099848689358\\
1.64139747806358	2.71986002293162\\
1.65232379011111	2.71811627703685\\
1.66237878254442	2.71584885191853\\
1.67171183356178	2.71311119457135\\
1.68044447792192	2.70993483944171\\
1.68867675842975	2.70633345329975\\
1.69649190694504	2.70230529579381\\
1.70395981265679	2.69783446704066\\
1.71113958433924	2.69289114772267\\
1.71808140792551	2.68743091323758\\
1.72482782235928	2.68139309808504\\
1.73141447211586	2.67469808160963\\
1.73787033314867	2.66724324384313\\
1.74421733919648	2.65889717938977\\
1.75046924315976	2.64949152834833\\
1.75662941219957	2.63880944005132\\
1.76268703975575	2.62656915235071\\
1.76861090034744	2.61240031796586\\
1.77433915991991	2.59580931610619\\
1.77976267171791	2.57612745194278\\
1.78469722220312	2.55243193700056\\
1.78883652472459	2.5234224961629\\
1.79167072701979	2.48722377925767\\
1.79234135213091	2.44106055482288\\
1.789375703746	2.38070977151951\\
1.78018693291654	2.29955531830014\\
1.76011199813813	2.1869387238229\\
1.72055198175909	2.02533260345223\\
1.64553280840373	1.7859655735198\\
1.50647561611139	1.42449133190946\\
1.25979031965253	0.887322769791866\\
0.867422530010792	0.157466147395609\\
0.358701421246773	-0.662129182426281\\
-0.149473375810737	-1.37749528953538\\
-0.557458530087428	-1.88173099747749\\
-0.848618942779954	-2.19841186792069\\
-1.04949403742542	-2.39082010930412\\
-1.18952358701084	-2.50882522014053\\
-1.28985205145128	-2.58299321675379\\
-1.36403888976222	-2.63085119744051\\
-1.42059441277052	-2.66243215945022\\
-1.46492211832912	-2.68360718214946\\
-1.50053416058839	-2.69791494306816\\
-1.52977527810634	-2.70755423580167\\
-1.55425307769521	-2.71393093611635\\
-1.57509782069847	-2.71796727543747\\
-1.5931228018649	-2.72028177271289\\
-1.60892578158929	-2.72129691153713\\
-1.62295469491897	-2.72130530601542\\
-1.63555120422068	-2.72051134533719\\
-1.64698019083996	-2.71905795828844\\
-1.65745012410178	-2.71704410986192\\
-1.66712738619165	-2.71453637411781\\
-1.67614651188316	-2.71157661673854\\
-1.68461761334463	-2.70818704280787\\
-1.69263182725761	-2.70437339026219\\
-1.70026534302248	-2.70012674922752\\
-1.70758238708614	-2.69542428859632\\
-1.71463741304862	-2.69022902975082\\
-1.72147665723247	-2.68448869518187\\
-1.72813914936835	-2.67813355668702\\
-1.73465720626276	-2.67107309642847\\
-1.74105637209399	-2.66319115557991\\
-1.74735469006647	-2.65433905492944\\
-1.75356107936389	-2.64432589293457\\
-1.7596724213162	-2.6329048006232\\
-1.76566868254957	-2.61975326128974\\
-1.7715049366081	-2.60444451716606\\
-1.77709833358423	-2.58640528663938\\
-1.78230661343683	-2.56485196448675\\
-1.78689208331813	-2.5386921773767\\
-1.79045991879808	-2.50636914342008\\
-1.79234981279428	-2.46560917445441\\
-1.79144039174069	-2.41300112137448\\
-1.78578601903851	-2.34327839389174\\
-1.77192472130788	-2.24807066239538\\
-1.74353879464788	-2.11373299263696\\
-1.68889795825026	-1.91774897557392\\
-1.58643016183899	-1.6238749767559\\
-1.39964365988511	-1.18075199110278\\
-1.08242507481589	-0.543631661955198\\
-0.621555837314145	0.25307854518944\\
-0.09641245414283	1.04328968947428\\
0.36874050269807	1.65658250176025\\
0.716347118554585	2.05952788811988\\
0.958339109629879	2.30652938594418\\
1.12558565639423	2.45682608727719\\
1.2436561905534	2.55006469283883\\
1.32958975333439	2.60945667416062\\
1.39412727092238	2.64823864447461\\
1.44403321655362	2.67405908965072\\
1.48364979593202	2.69145857862251\\
1.51583699422256	2.70321614386247\\
1.54253000575099	2.71108399841515\\
1.56507272792322	2.71619739538668\\
1.58442127374257	2.71930968084827\\
1.60127112895018	2.72093100837461\\
1.61613855159024	2.72141247921817\\
};
\addplot [color=mycolor1, forget plot]
  table[row sep=crcr]{%
1.38732790552402	2.63465819500296\\
1.40062943147086	2.6342425157038\\
1.41278139958255	2.63308730039004\\
1.42398694382011	2.63129832534375\\
1.43440982960286	2.62894733458447\\
1.44418383730999	2.62607976906692\\
1.45341964378235	2.62271986095831\\
1.4622099269933	2.61887384173129\\
1.47063318375156	2.61453172508766\\
1.47875659248889	2.60966793076692\\
1.48663814362397	2.60424087250832\\
1.49432817971773	2.59819151597249\\
1.5018704234068	2.5914408000625\\
1.50930251238285	2.58388568931988\\
1.51665599722341	2.57539346488786\\
1.52395567717654	2.56579363848176\\
1.53121803284551	2.55486654492852\\
1.53844833348756	2.54232716514553\\
1.54563569904242	2.52780193788258\\
1.55274489072743	2.51079503816241\\
1.55970271801618	2.49063848845685\\
1.56637535903263	2.46641691101483\\
1.57252996400346	2.43685161615735\\
1.57776839447886	2.40011803141076\\
1.5814103256968	2.35355156732736\\
1.58228214227961	2.29316361847308\\
1.57832723800457	2.21283228039354\\
1.5658755767825	2.10294548415539\\
1.53827929241975	1.94819438992214\\
1.48349357885858	1.72440073080168\\
1.38055319409598	1.39581707532157\\
1.19781585501386	0.920306801172927\\
0.904628098562938	0.280688803757821\\
0.509489430092227	-0.456114835477672\\
0.0855409875837257	-1.14058062935781\\
-0.283257072849337	-1.66029752573636\\
-0.564207735286064	-2.00757771839611\\
-0.766873947036603	-2.22793675491897\\
-0.912235725107974	-2.36710609782894\\
-1.01829550715567	-2.45644119828739\\
-1.09766249484591	-2.51508683724071\\
-1.15866437634191	-2.55442282650588\\
-1.20675941635745	-2.58126903813666\\
-1.24557031288401	-2.59980192357757\\
-1.27755060032258	-2.61264593418094\\
-1.30439982093182	-2.62149347003746\\
-1.32732197226194	-2.62746251368772\\
-1.34718845519265	-2.63130761041243\\
-1.36464270165228	-2.63354740438033\\
-1.38016871998496	-2.63454360949568\\
-1.3941369208368	-2.63455102054594\\
-1.40683536048265	-2.63374982835509\\
-1.41849144100645	-2.63226685697424\\
-1.42928724575547	-2.63018969575471\\
-1.43937055152318	-2.62757615751248\\
-1.44886285155347	-2.62446057350835\\
-1.45786527494397	-2.62085787180965\\
-1.46646299787239	-2.61676602896612\\
-1.47472855005451	-2.61216724990523\\
-1.48272428903746	-2.60702806608776\\
-1.4905042219424	-2.60129841465141\\
-1.49811528344374	-2.59490964867501\\
-1.50559811854367	-2.58777131199924\\
-1.51298735896656	-2.57976637202622\\
-1.52031131173482	-2.57074441656157\\
-1.52759088287588	-2.56051205144964\\
-1.53483741525638	-2.54881933036228\\
-1.54204888849166	-2.53534041788314\\
-1.54920354220679	-2.51964568188709\\
-1.55624931637771	-2.50116077122663\\
-1.56308631896982	-2.47910550075426\\
-1.56953737967292	-2.45240071311581\\
-1.57529774141912	-2.41952321802537\\
-1.57984730402533	-2.37827470462384\\
-1.58229399048494	-2.32540535526911\\
-1.58108764032128	-2.25598889593011\\
-1.5734870898689	-2.16237360962026\\
-1.55455945900639	-2.03243925699405\\
-1.51534183875637	-1.84688071019744\\
-1.43979552430412	-1.57587327333968\\
-1.30141382543751	-1.17872959520849\\
-1.06600085542706	-0.619443701596676\\
-0.716526411785143	0.0840243813567158\\
-0.295063910234367	0.814367939190121\\
0.109186222039112	1.42365553711546\\
0.434684938257419	1.85305282279251\\
0.67402696988451	2.1304579326942\\
0.845463387434966	2.30536939321677\\
0.969271852192695	2.41657625809892\\
1.06071279964974	2.48875193121301\\
1.13006486318988	2.53666300667769\\
1.1840651285775	2.56909994731102\\
1.22715045655147	2.59138317239276\\
1.26229400270659	2.60681253448976\\
1.29153199480616	2.61748891908155\\
1.31629096703128	2.62478392175053\\
1.33759255451442	2.62961371435005\\
1.35618374488957	2.63260257864501\\
1.3726213487659	2.63418297028529\\
1.38732790552402	2.63465819500296\\
};
\addplot [color=mycolor1, forget plot]
  table[row sep=crcr]{%
1.02739733741893	2.47348557476368\\
1.04219991052851	2.47302138058375\\
1.0559950337039	2.47170853917827\\
1.06895728691558	2.46963781625163\\
1.08123207453146	2.46686791514125\\
1.09294215182295	2.46343119833349\\
1.10419258201004	2.45933728611963\\
1.11507453640509	2.45457500278217\\
1.12566822414637	2.44911293646631\\
1.13604514731848	2.44289872233356\\
1.14626980813599	2.43585702324449\\
1.15640093689847	2.42788604669408\\
1.16649225345643	2.41885228009767\\
1.17659271108451	2.40858292391278\\
1.18674608692328	2.39685521926411\\
1.19698965773281	2.38338145237328\\
1.20735150090413	2.36778779134156\\
1.21784563190738	2.34958414090792\\
1.2284636298166	2.32812067278729\\
1.2391604290419	2.30252424743977\\
1.24983023012145	2.2716040012872\\
1.26026537740859	2.23370898637197\\
1.2700854025811	2.18651048837497\\
1.27861313515567	2.12666571765968\\
1.28465631246388	2.04929712833414\\
1.286121874765	1.94719876205682\\
1.2793458872528	1.80969238589744\\
1.25799632015502	1.62123209077316\\
1.21156045628604	1.36058221253479\\
1.12428293704218	1.00343260412228\\
0.977774300260783	0.534507895522493\\
0.762495229105826	-0.0276962808078741\\
0.494580699093191	-0.614415920687728\\
0.21512444262883	-1.13704908249558\\
-0.0363748816684043	-1.54366353098258\\
-0.242170781785461	-1.83370229251349\\
-0.403006881609096	-2.03239387455115\\
-0.5271200897453	-2.16723258118143\\
-0.623437344275255	-2.25937028699492\\
-0.699259962306375	-2.32318702120188\\
-0.760009693129276	-2.36804447342719\\
-0.809580221423752	-2.39998858735189\\
-0.850750354948673	-2.42295559567968\\
-0.885515681979411	-2.43954711168527\\
-0.915326660655649	-2.45151305628743\\
-0.941252924563969	-2.4600514134627\\
-0.96409544920585	-2.46599588098061\\
-0.984463303547178	-2.46993500067531\\
-1.0028266286991	-2.47228901446572\\
-1.0195536418362	-2.47336025691699\\
-1.03493682563864	-2.473366694953\\
-1.0492117187949	-2.47246453206161\\
-1.06257058284439	-2.47076357138599\\
-1.07517247546841	-2.46833767408633\\
-1.08715077022958	-2.46523180445155\\
-1.09861883702821	-2.46146661564033\\
-1.10967437803078	-2.4570411775543\\
-1.1204027631107	-2.4519342071522\\
-1.13087960249329	-2.44610398495604\\
-1.1411727158489	-2.43948699841739\\
-1.15134359471152	-2.43199521991977\\
-1.16144839927243	-2.42351178387252\\
-1.17153847205639	-2.41388465128115\\
-1.18166027864401	-2.4029176125877\\
-1.19185458362132	-2.3903576388808\\
-1.20215451307336	-2.37587708363649\\
-1.21258190036276	-2.35904845940414\\
-1.22314088434211	-2.33930829887762\\
-1.23380699293633	-2.31590468120806\\
-1.24450865180094	-2.28781990495046\\
-1.25509574716695	-2.25365477084764\\
-1.2652856878568	-2.2114528242258\\
-1.27456978008644	-2.1584300336945\\
-1.28204888502655	-2.09055610856803\\
-1.28614302408898	-2.00190929011678\\
-1.28408101466374	-1.88371283442261\\
-1.27103285053692	-1.72302784374413\\
-1.238776102119	-1.50145517093768\\
-1.17419465210873	-1.19547538391306\\
-1.05944501671691	-0.782915478248143\\
-0.878472672005636	-0.26187964957388\\
-0.632896379370723	0.323789932064092\\
-0.353427167312457	0.888326759231871\\
-0.0842104180387018	1.35587313105521\\
0.145234985297924	1.70192712671374\\
0.327761757720484	1.94263941038087\\
0.469069566250933	2.10629884367719\\
0.578250597443543	2.21759988385927\\
0.663530707415728	2.29413829132118\\
0.731244886631844	2.34754665688213\\
0.785996928136564	2.38534607007067\\
0.831075464460832	2.41240712373885\\
0.868832417770454	2.43192315659928\\
0.900966209411553	2.44602308723871\\
0.928720063416563	2.45615169171875\\
0.953017804165661	2.46330641382896\\
0.974556567004055	2.46818659097927\\
0.99387047974921	2.47128894435639\\
1.01137486888268	2.47296968927403\\
1.02739733741893	2.47348557476368\\
};
\addplot [color=mycolor1, forget plot]
  table[row sep=crcr]{%
0.47620713588277	2.18437004943608\\
0.496625391790089	2.18372579956675\\
0.516344844398201	2.18184549881844\\
0.535514439991744	2.17877969559682\\
0.554268779286526	2.17454430135902\\
0.5727315313037	2.16912250662038\\
0.591018288661362	2.16246485952589\\
0.60923897433689	2.15448754932154\\
0.627499870189142	2.14506878020796\\
0.645905296015645	2.13404295309354\\
0.664558918125901	2.12119217378726\\
0.683564598818918	2.10623435253214\\
0.703026598024824	2.0888068213686\\
0.723048782054685	2.06844393344197\\
0.743732243240506	2.0445464723026\\
0.76517032472752	2.01633983223327\\
0.787439373166844	1.98281678410573\\
0.810582443206192	1.94265922636535\\
0.834581400690629	1.89413184551802\\
0.859310076920423	1.83493981906352\\
0.884456983623625	1.76204465156248\\
0.909400721240225	1.6714420062521\\
0.933016540050802	1.55793417073609\\
0.953395887840482	1.41499903073798\\
0.967492159003808	1.23499780370961\\
0.970805297122381	1.01018388457675\\
0.957426852106025	0.735142385460073\\
0.921010892175009	0.410912963375699\\
0.857092982242873	0.0493880011383427\\
0.766035925310678	-0.325839751088052\\
0.654323155910683	-0.6855301599157\\
0.53246034017998	-1.00515751171179\\
0.410885578518104	-1.27203135999729\\
0.296971860217316	-1.48520273938611\\
0.194415318420365	-1.65092506949056\\
0.104138541697818	-1.77801892895423\\
0.0254860864408128	-1.87506077908946\\
-0.0428805888734549	-1.94924200381424\\
-0.102465461955605	-2.00617381507678\\
-0.154689883582295	-2.05008085854172\\
-0.200798345906226	-2.08409273935246\\
-0.241840318683732	-2.11051574146561\\
-0.278684796607555	-2.13105091112399\\
-0.312045929833457	-2.14695799135387\\
-0.342509935822128	-2.15917489270236\\
-0.370559393274054	-2.16840355335845\\
-0.396593767743847	-2.17517133861407\\
-0.420946197254415	-2.17987493192\\
-0.443897002523521	-2.18281175264095\\
-0.465684490818418	-2.18420246095624\\
-0.486513590890022	-2.18420703600205\\
-0.50656277870641	-2.18293614879374\\
-0.525989668611247	-2.18045901049187\\
-0.544935567027176	-2.17680849067284\\
-0.56352922005612	-2.17198401849787\\
-0.581889931748121	-2.16595256483222\\
-0.600130183878707	-2.15864782773647\\
-0.618357847292737	-2.14996758545978\\
-0.636678034922541	-2.13976902101053\\
-0.655194601897781	-2.1278616415685\\
-0.674011240920889	-2.11399719299014\\
-0.693232039511397	-2.09785567790774\\
-0.71296124121597	-2.07902619108133\\
-0.733301755241592	-2.05698074329029\\
-0.754351638642492	-2.0310385007615\\
-0.776197250911278	-2.00031686454427\\
-0.79890092114673	-1.96366452461323\\
-0.822479567427569	-1.91957013046458\\
-0.846868468608206	-1.86603893716692\\
-0.871860950673296	-1.80042997719165\\
-0.89700991179925	-1.71925125011644\\
-0.921471626576344	-1.61792774868017\\
-0.943770128972071	-1.49060270372679\\
-0.961474196303445	-1.33013337698189\\
-0.970838148433354	-1.12862641091917\\
-0.966609375019177	-0.879085111628614\\
-0.942457436254818	-0.578726314182426\\
-0.892610136125284	-0.233577186077621\\
-0.814687491133952	0.138325729779689\\
-0.712153021676426	0.509387866614885\\
-0.593959795072005	0.851454979769253\\
-0.471086238914411	1.14547925137416\\
-0.352660605060765	1.38504616933722\\
-0.244166059846886	1.57346115459866\\
-0.147758427943964	1.7187390553571\\
-0.0634346279898705	1.82980920482635\\
0.009888947335459	1.91462255823226\\
0.0736783576399879	1.97957090125034\\
0.129415354679625	2.02953758502394\\
0.178438527373515	2.06816309229685\\
0.221894380239587	2.09813451538813\\
0.260739303706182	2.12143181788064\\
0.295761446410512	2.1395179456237\\
0.327607743123823	2.15347913236043\\
0.356809782611324	2.16412628696682\\
0.38380626313929	2.1720676188883\\
0.408961606107032	2.17776054426846\\
0.432581034590971	2.18154881346394\\
0.454922659440202	2.18368910282942\\
0.47620713588277	2.18437004943608\\
};
\addplot [color=mycolor1, forget plot]
  table[row sep=crcr]{%
-0.218432368692388	1.75006885244458\\
-0.178478874550534	1.74879469991791\\
-0.137402821732203	1.74486457968917\\
-0.0950083444367945	1.73807094169262\\
-0.0510875079760354	1.72813839612093\\
-0.00542085812119394	1.71471388417246\\
0.0422207735712061	1.69735461930125\\
0.0920724954094568	1.67551384024463\\
0.144368606585523	1.64852472000514\\
0.199331491102878	1.61558333583957\\
0.257154326987754	1.57573256717545\\
0.317975337776623	1.52785033449011\\
0.38184103876284	1.47064791060494\\
0.448656175742909	1.40268720968177\\
0.518119426896689	1.32242971859341\\
0.589647322399636	1.22833301036837\\
0.662295171885714	1.11901109920361\\
0.734693343049372	0.993468019785329\\
0.805028151840547	0.851394813056192\\
0.871103239544788	0.69348622035243\\
0.93051018123024	0.521692030405634\\
0.98090791377875	0.33929151936184\\
1.02036266320661	0.150699125173134\\
1.04765596401809	-0.0390090651407171\\
1.06246077478911	-0.224743756910975\\
1.06532985788274	-0.402041008726533\\
1.05751428953611	-0.567514609618132\\
1.04068841502765	-0.719035767469351\\
1.01667106016879	-0.85566276912545\\
0.987206502897894	-0.97741184673942\\
0.953828795059489	-1.08497118276009\\
0.91780206834694	-1.17943256047259\\
0.880115114284985	-1.26207804852348\\
0.841507293520577	-1.33423012225647\\
0.802507999458383	-1.397157525876\\
0.763478381307864	-1.45202337733812\\
0.724649355479105	-1.49986213387365\\
0.686153487547168	-1.54157460811198\\
0.648050358305878	-1.57793325628698\\
0.610346013277956	-1.60959257302992\\
0.573007458842296	-1.63710138046797\\
0.535973205639494	-1.66091513805353\\
0.499160751098688	-1.68140726081827\\
0.462471736243642	-1.69887895800313\\
0.425795355670333	-1.71356740441454\\
0.389010464018177	-1.7256522148187\\
0.351986713401945	-1.7352602619678\\
0.314584973745266	-1.74246889659897\\
0.276657229172496	-1.7473076147454\\
0.238046105928604	-1.74975818716593\\
0.198584169077412	-1.74975322574803\\
0.158093126445626	-1.74717311794589\\
0.116383100910635	-1.74184121782058\\
0.0732521805527515	-1.73351714785357\\
0.0284865375923655	-1.72188805042904\\
-0.0181384683047321	-1.70655765052496\\
-0.0668556056456335	-1.68703308344085\\
-0.117900656747315	-1.66270965522525\\
-0.171503689684931	-1.63285412029728\\
-0.227875117489127	-1.596587801435\\
-0.287184486792264	-1.55287210664826\\
-0.349529547676552	-1.50050090893681\\
-0.414893083737573	-1.43810700479428\\
-0.483085696016494	-1.36419340466288\\
-0.553674989290907	-1.27720393252153\\
-0.625906368285954	-1.1756498278953\\
-0.69862865302886	-1.05830639631927\\
-0.770248394068362	-0.92448131101201\\
-0.838746521318586	-0.774329508164403\\
-0.901792002438366	-0.609150208681351\\
-0.956969634936358	-0.431563950437059\\
-1.00209899767848	-0.245460409394229\\
-1.03557164603545	-0.0556587079783228\\
-1.0566040611219	0.132673297390674\\
-1.06532245902061	0.314687447316354\\
-1.06265948272779	0.48641850018189\\
-1.05011439250815	0.645103748192245\\
-1.0294659048525	0.789229178833693\\
-1.00251785216747	0.918366425926476\\
-0.970921189464369	1.03290364240365\\
-0.936078507461518	1.13376138894612\\
-0.899114494127462	1.22214923861014\\
-0.860888801271196	1.29938442668872\\
-0.822030537946831	1.36677135618774\\
-0.78297990227181	1.42553035213702\\
-0.74402853215203	1.47676173626029\\
-0.705354597016382	1.52143297586337\\
-0.667051389562434	1.56037963029418\\
-0.629149625407273	1.59431370034004\\
-0.591634288114514	1.62383527509477\\
-0.554457028715936	1.64944500234484\\
-0.51754507661473	1.67155598997075\\
-0.48080747746677	1.69050442247485\\
-0.444139313285142	1.70655857697429\\
-0.407424412692465	1.71992614383045\\
-0.370536936812692	1.7307598657518\\
-0.333342130857583	1.73916154979285\\
-0.295696461168918	1.74518450695214\\
-0.257447309526412	1.74883445104043\\
-0.218432368692388	1.75006885244458\\
};
\addplot [color=mycolor1, forget plot]
  table[row sep=crcr]{%
-0.751180563789503	1.29301100092275\\
-0.644509272490707	1.2895621961483\\
-0.525853075685341	1.27816235586954\\
-0.394733067383687	1.25710640312682\\
-0.251255417385498	1.22462204940121\\
-0.0963657956622014	1.17906474936042\\
0.0679313764001297	1.11919457949777\\
0.238479475719274	1.04449792159326\\
0.411096645609142	0.95547160953947\\
0.580968788823447	0.85375980073978\\
0.743253360925444	0.742056725339432\\
0.893734219404683	0.623769327860435\\
1.02933068138786	0.502533729359883\\
1.14832654684568	0.381735283717928\\
1.25030593415255	0.264160768651649\\
1.33588484834454	0.151837028466942\\
1.40636299856342	0.046035698606773\\
1.46339657703116	-0.0526160638365817\\
1.50874510492206	-0.143981414805869\\
1.54410401747092	-0.228260871982706\\
1.57101084154411	-0.305863800112672\\
1.59080459914876	-0.37731143443009\\
1.60461898267407	-0.443169442531953\\
1.61339434135726	-0.50400430761343\\
1.6178983941152	-0.560357233657007\\
1.61874953068363	-0.612730111534175\\
1.61643932306818	-0.661579338697342\\
1.6113526202479	-0.707314476704708\\
1.60378462375957	-0.750299690229399\\
1.59395489259106	-0.790856617640623\\
1.58201848840448	-0.829267814089403\\
1.56807456975614	-0.865780234303465\\
1.55217275313418	-0.900608430316382\\
1.53431752429447	-0.933937265273067\\
1.51447093083552	-0.965924014217765\\
1.49255372986368	-0.996699753941291\\
1.4684451102011	-1.02636994778256\\
1.44198106109596	-1.05501411446045\\
1.41295142229282	-1.08268443616728\\
1.38109562819774	-1.10940311195514\\
1.34609715924291	-1.13515819874194\\
1.30757674909159	-1.15989760548665\\
1.26508448786515	-1.18352082038956\\
1.2180911421837	-1.20586786686862\\
1.16597933286764	-1.22670492454442\\
1.10803574356846	-1.24570606215447\\
1.04344637620195	-1.26243069409395\\
0.971298136519321	-1.27629683490271\\
0.890591822483024	-1.2865512138292\\
0.800273888148872	-1.29223914571393\\
0.699296856787975	-1.29218010554772\\
0.586720020892429	-1.28495946080335\\
0.461861094092619	-1.26895243972215\\
0.3245024552969	-1.24240144182672\\
0.175138600776365	-1.2035682010117\\
0.0152226545865632	-1.15097166888395\\
-0.152664711632565	-1.08369478588514\\
-0.324821834412255	-1.00170023256538\\
-0.496685757869856	-0.90605470141019\\
-0.663344832918517	-0.798955642596777\\
-0.820192630077424	-0.683508161867404\\
-0.963532782997699	-0.563297645942172\\
-1.09095770410823	-0.441888842741164\\
-1.20142500062305	-0.322399917701083\\
-1.29507612370551	-0.207246320149147\\
-1.37291296977355	-0.0980684877283813\\
-1.43645044441771	0.0042020468738688\\
-1.48742259418976	0.0992032593483852\\
-1.52757278250983	0.186984897840184\\
-1.55852539637732	0.267866024583987\\
-1.58172126356109	0.342322104544315\\
-1.59839612803496	0.410903378933661\\
-1.60958478267695	0.474180138513564\\
-1.61613840476429	0.532708621189078\\
-1.61874713092658	0.587011554073861\\
-1.61796325588701	0.637568493346236\\
-1.61422266111825	0.684812373734955\\
-1.60786343226419	0.729129764048409\\
-1.59914138437268	0.770863155573305\\
-1.58824260169102	0.810314202630113\\
-1.57529326680601	0.847747236816572\\
-1.56036710004137	0.883392638422727\\
-1.54349071326817	0.917449811620675\\
-1.52464713656047	0.950089605203132\\
-1.50377772010479	0.981456069358695\\
-1.4807825573749	1.0116674553813\\
-1.45551952420259	1.04081635809509\\
-1.42780198561346	1.06896887513935\\
-1.39739519197381	1.09616261560587\\
-1.3640113739163	1.12240333396511\\
-1.32730356145226	1.14765989469051\\
-1.28685821295664	1.17185719094692\\
-1.24218687070083	1.19486655389881\\
-1.19271730234616	1.21649311338865\\
-1.13778500336415	1.23645953866669\\
-1.07662660918946	1.2543856619263\\
-1.00837780643194	1.26976377842146\\
-0.932079852456276	1.28193010776828\\
-0.846700871404774	1.2900342666388\\
-0.751180563789503	1.29301100092275\\
};
\addplot [color=mycolor1, forget plot]
  table[row sep=crcr]{%
-0.792711621111321	0.98157953768228\\
-0.524039126534901	0.972855450713703\\
-0.223155287672921	0.943970299352618\\
0.0970845287357334	0.89265178624017\\
0.418361554453311	0.820112091604991\\
0.721568299281959	0.731202980446408\\
0.992069737039553	0.632937470787459\\
1.22249537797474	0.532313651163314\\
1.41221608098798	0.434728026924781\\
1.56495976330159	0.343484831498922\\
1.68635092136143	0.260092913835551\\
1.78224001120135	0.18483850219387\\
1.8578635205525	0.11730988759003\\
1.9175611884238	0.0567670347987813\\
1.96478181381882	0.00236466154353296\\
2.00220485764216	-0.0467286451778812\\
2.03188865709344	-0.0912751891546783\\
2.05540712823268	-0.131948588281725\\
2.07396278328999	-0.169332489372267\\
2.08847528569881	-0.203927707183566\\
2.09964889554858	-0.236162518549415\\
2.10802305513758	-0.266403470902291\\
2.11401000820589	-0.294965496142634\\
2.11792262303556	-0.322120867955179\\
2.11999485572392	-0.348106911291836\\
2.12039666474396	-0.3731325425674\\
2.11924469208138	-0.39738378512693\\
2.11660964448466	-0.421028418889408\\
2.11252101571752	-0.444219913015945\\
2.10696956246702	-0.467100769619532\\
2.09990776123253	-0.489805381061177\\
2.09124831286927	-0.512462475132387\\
2.08086060958534	-0.53519719055088\\
2.06856492130052	-0.558132786625655\\
2.05412387949755	-0.581391940158127\\
2.03723062115011	-0.605097510735529\\
2.01749268582802	-0.629372548500999\\
1.99441041849142	-0.654339154204018\\
1.96734820656221	-0.680115545407622\\
1.93549637849465	-0.706810282115394\\
1.89782106738893	-0.734511980507324\\
1.85299897045652	-0.763271883400435\\
1.79933414963148	-0.793075222898754\\
1.73465580805775	-0.823795289805376\\
1.6562014095423	-0.855121592693686\\
1.56050256136341	-0.886451175769476\\
1.44331848846043	-0.916732514784459\\
1.29971241791409	-0.944260595272382\\
1.12444293603457	-0.966452087976906\\
0.912917778625771	-0.979697607096246\\
0.662923339355622	-0.979494985082252\\
0.37698478839068	-0.961143526917589\\
0.0643945057451018	-0.92113151248099\\
-0.258836596504117	-0.858806199962149\\
-0.573293343729396	-0.77729038059577\\
-0.86154610949324	-0.682778190636289\\
-1.11246542293541	-0.58253462290651\\
-1.32228123826288	-0.482886497173722\\
-1.4928788430727	-0.388183675967567\\
-1.62920072357579	-0.300767744684041\\
-1.73713626895234	-0.221463029054589\\
-1.82229189975114	-0.15015026080482\\
-1.88946780204592	-0.0862173983858334\\
-1.9425473720943	-0.0288514537115337\\
-1.98457653540813	0.0227960100964114\\
-2.01790588850602	0.0695260087674356\\
-2.04433584795107	0.112057504232829\\
-2.06524208259898	0.151018597474137\\
-2.08167602871728	0.186950344449971\\
-2.09444236973088	0.220315931258629\\
-2.10415752737457	0.251511445216503\\
-2.11129331982863	0.280876428860417\\
-2.11620933433931	0.308703443840538\\
-2.11917680800605	0.335246401829789\\
-2.12039612423771	0.360727673755501\\
-2.1200094724428	0.385344097860135\\
-2.11810978260752	0.409272042657603\\
-2.11474671244081	0.432671680630897\\
-2.10993020723372	0.455690611939787\\
-2.10363194826658	0.478466953807131\\
-2.09578483471324	0.50113198435242\\
-2.08628048961741	0.523812399829979\\
-2.07496462746163	0.546632209530777\\
-2.06162995442018	0.569714248797749\\
-2.04600607729657	0.593181230636013\\
-2.02774565683554	0.617156169080839\\
-2.0064057380819	0.641761875216961\\
-1.98142280905275	0.667119021938585\\
-1.95207967357141	0.693341953353095\\
-1.91746169957402	0.720530913984127\\
-1.87639952256073	0.748758596599108\\
-1.82739512479565	0.778047728646369\\
-1.76852903775977	0.808334698837547\\
-1.69734967106284	0.839411914748699\\
-1.61075435885193	0.870838996005181\\
-1.50489091967973	0.901811446283295\\
-1.37514641412866	0.930979107423511\\
-1.21635403674298	0.95622448292148\\
-1.02343225058082	0.974458275051307\\
-0.792711621111322	0.98157953768228\\
};
\addplot [color=mycolor1, forget plot]
  table[row sep=crcr]{%
-0.392413722930891	0.84554181943936\\
0.0574991168375939	0.831154097362173\\
0.505147659949656	0.788519822044919\\
0.911684194589035	0.723753194528262\\
1.25373099685688	0.64685994138717\\
1.52646987318868	0.567132827487189\\
1.73723815412086	0.490727610699746\\
1.89787424528086	0.420675286662048\\
2.02003209129655	0.3578942088356\\
2.11336301864759	0.302168930108651\\
2.18525659578056	0.252792732245178\\
2.24116986978698	0.208915948011435\\
2.28507546096545	0.169710004235972\\
2.31985801824868	0.134432497075667\\
2.34762016050889	0.102444204275014\\
2.36990526938917	0.0732049691631955\\
2.38785567468071	0.0462616886765516\\
2.40232392619075	0.0212344929143888\\
2.41395099003992	-0.00219632298715742\\
2.42322138148395	-0.0243017186686428\\
2.43050222936105	-0.0453135288483911\\
2.43607108364639	-0.0654324372620509\\
2.44013575490199	-0.0848345583017478\\
2.44284842789974	-0.103676892510789\\
2.44431557320258	-0.122101899568361\\
2.44460468239501	-0.140241400069661\\
2.44374849975782	-0.158219985998165\\
2.44174716327257	-0.17615809341235\\
2.43856846387418	-0.194174870617511\\
2.43414625602581	-0.212390960687017\\
2.42837688181337	-0.230931307666362\\
2.42111328258965	-0.249928089576639\\
2.41215624176518	-0.269523875938011\\
2.40124189793098	-0.289875098650625\\
2.38802424562686	-0.311155904824948\\
2.37205073996559	-0.333562413755809\\
2.35272825286137	-0.357317299644011\\
2.32927537032245	-0.382674413614582\\
2.30065521410733	-0.409922742037863\\
2.26548044975565	-0.439388182657007\\
2.22187884264946	-0.471430051093477\\
2.16730403669588	-0.506426266136148\\
2.0982739817986	-0.544735706230507\\
2.01002436055964	-0.586616641754953\\
1.89609176532488	-0.632064808809404\\
1.74792684764033	-0.680515169409561\\
1.55484842567927	-0.730343280848899\\
1.30505799123314	-0.778160890317086\\
0.988919447812327	-0.818148046207319\\
0.605416543083506	-0.842192477576616\\
0.169844949337834	-0.841990436587597\\
-0.284265742050115	-0.81313340510902\\
-0.715481032707434	-0.758309777154651\\
-1.09145467912348	-0.686181762958299\\
-1.39848995418638	-0.60688389900676\\
-1.63889402833189	-0.528262755877788\\
-1.82305794230454	-0.454813385393407\\
-1.96310514827734	-0.388371966541922\\
-2.06978862969178	-0.329187492207117\\
-2.15160841140705	-0.27673952309024\\
-2.21493276229014	-0.230220283082745\\
-2.26442186776805	-0.188777889956994\\
-2.30346035018527	-0.151622578146054\\
-2.33450919511601	-0.118063158297597\\
-2.35936804367039	-0.0875111519170718\\
-2.37936328361864	-0.0594716327281494\\
-2.3954807891641	-0.033529829150519\\
-2.40845918011913	-0.00933749359531244\\
-2.41885544449055	0.0133993926007298\\
-2.42709131700411	0.0349311323748096\\
-2.43348622202874	0.0554730815850008\\
-2.43828076060055	0.0752128930362168\\
-2.44165345782466	0.0943165027594429\\
-2.44373262028527	0.112933115758461\\
-2.44460455599637	0.131199419032673\\
-2.44431899168683	0.149243217004679\\
-2.44289222122969	0.167186655319756\\
-2.44030829102082	0.185149175644964\\
-2.43651834129811	0.203250326852542\\
-2.43143805178636	0.22161254615892\\
-2.42494296339643	0.240364016156838\\
-2.41686124184706	0.259641698268294\\
-2.40696318622118	0.279594636712781\\
-2.39494642873272	0.300387613833229\\
-2.38041526965797	0.322205206779803\\
-2.36285186960453	0.345256226766583\\
-2.34157597558486	0.369778376738065\\
-2.31568834666596	0.396042667919569\\
-2.28399090003332	0.424356551513887\\
-2.24487367983333	0.45506358830937\\
-2.19615515333315	0.488535316920607\\
-2.13485898358148	0.525146943201869\\
-2.05691063622175	0.565221183974557\\
-1.95675047473234	0.608912266997585\\
-1.82691017093183	0.655983984208223\\
-1.65773700274838	0.705418249335526\\
-1.43775505952901	0.75480505417595\\
-1.15564363113835	0.799599025224289\\
-0.805088588744831	0.832719416165173\\
-0.392413722930893	0.84554181943936\\
};
\addplot [color=mycolor1, forget plot]
  table[row sep=crcr]{%
0.127583927346025	0.822354782701324\\
0.648213232803182	0.80605800795337\\
1.10318107339043	0.76304053956409\\
1.46974443062957	0.704857395593081\\
1.7506277306205	0.641835239258246\\
1.96083296670893	0.580446894605021\\
2.1172920353496	0.523753890913169\\
2.23435929047659	0.472709568819359\\
2.32289332314376	0.427209264636958\\
2.39071777840021	0.386710389660878\\
2.44337011656711	0.350545072678569\\
2.48476144153763	0.318059921922197\\
2.51767062198877	0.288669271144002\\
2.54409160763447	0.261868466494899\\
2.56547156713278	0.237230258626693\\
2.58287306167405	0.214395113582127\\
2.5970847403339	0.193060239851003\\
2.60869741614965	0.172969283164263\\
2.61815680955908	0.153903340296842\\
2.62580044199271	0.135673377279496\\
2.63188363663354	0.118113912642419\\
2.63659792076735	0.101077754940444\\
2.64008402751842	0.0844315774446671\\
2.6424409643375	0.0680521305771022\\
2.64373212040189	0.0518229159299322\\
2.64398903950231	0.0356311665529758\\
2.64321323224248	0.0193649933701248\\
2.64137620229844	0.00291056587197173\\
2.63841768691518	-0.0138508038767062\\
2.63424193709751	-0.0310458117512815\\
2.62871166336885	-0.0488126995318836\\
2.62163901919879	-0.0673053570165812\\
2.61277264673791	-0.0866981974870052\\
2.60177931095254	-0.107192055562649\\
2.58821791195275	-0.129021398355355\\
2.57150255840735	-0.152463170319034\\
2.55084970446004	-0.177847574298192\\
2.52520179584149	-0.205570946037465\\
2.49311601422567	-0.236110429802094\\
2.45260104892505	-0.270039025142066\\
2.40087706065533	-0.308036934055735\\
2.3340250863367	-0.350889310674258\\
2.24648755503755	-0.39944821162938\\
2.13040049228801	-0.454512077012021\\
1.97483523001082	-0.516532619024773\\
1.76533290111822	-0.585000852211627\\
1.48484120470863	-0.657355061917321\\
1.11829318779626	-0.727527135003885\\
0.663073259146776	-0.785196445728787\\
0.142231167785234	-0.818072895775999\\
-0.393472173993291	-0.818155783814035\\
-0.886049713082783	-0.787199461778638\\
-1.29779320358844	-0.73511883652189\\
-1.62009373036179	-0.673452659624767\\
-1.86347479476686	-0.610680751481175\\
-2.04480161083383	-0.551419036273427\\
-2.17999142771539	-0.497516541074019\\
-2.28164162385726	-0.449294850545918\\
-2.35900309711399	-0.406375673156854\\
-2.41866449572068	-0.368128112625037\\
-2.46527788530321	-0.33388094484314\\
-2.5021365149178	-0.303011228645238\\
-2.53159107832171	-0.274973555532582\\
-2.55533779171109	-0.249302966880185\\
-2.57461494025345	-0.225607431139841\\
-2.59033677587351	-0.203557141000713\\
-2.6031851909469	-0.182873735592355\\
-2.61367299057704	-0.163320619972174\\
-2.62218795812845	-0.144694689635349\\
-2.62902380288774	-0.126819405365888\\
-2.63440202887595	-0.109539031254651\\
-2.63848741459588	-0.092713817117327\\
-2.641398900009	-0.0762159155355208\\
-2.64321707729127	-0.059925845691677\\
-2.6439890701039	-0.043729338835614\\
-2.64373129273313	-0.0275144184669946\\
-2.64243035866194	-0.0111685801382348\\
-2.64004222456362	0.00542405958235494\\
-2.63648948404545	0.0223857783896487\\
-2.63165654158267	0.0398485720872036\\
-2.62538217455736	0.0579579507065615\\
-2.61744869653946	0.0768773987940208\\
-2.60756652062552	0.0967937282660691\\
-2.59535231747372	0.117923594147524\\
-2.58029806141494	0.140521481901149\\
-2.56172689521631	0.164889487239771\\
-2.53872967067157	0.191389142111881\\
-2.51007287587656	0.220455274617765\\
-2.47406396369724	0.252611164735925\\
-2.42835338315418	0.288482511170148\\
-2.3696439929403	0.328803780544085\\
-2.29327057097972	0.374401992519135\\
-2.19261540300272	0.426125476488849\\
-2.05837353081808	0.484651867778157\\
-1.87786003310808	0.550056379298748\\
-1.635040280013	0.620972440860603\\
-1.31294328821675	0.693264653039546\\
-0.901053209393393	0.758704786362719\\
-0.408199139990258	0.805422528040251\\
0.127583927346023	0.822354782701324\\
};
\addplot [color=mycolor1, forget plot]
  table[row sep=crcr]{%
0.5647960623138	0.853812859999501\\
1.07128673553608	0.838198408636122\\
1.48039777394432	0.799662583060077\\
1.79266817538587	0.750166491000075\\
2.02479185719599	0.698110293912658\\
2.19632768966487	0.648020529098301\\
2.32386018402829	0.601806757171454\\
2.4198235022426	0.559959493407524\\
2.49307769479211	0.522306821137164\\
2.54982610613474	0.488416853654953\\
2.59440759520677	0.457790987159596\\
2.62987906959654	0.429948280836494\\
2.65841847283103	0.404457029515137\\
2.68159732786201	0.380942142385574\\
2.70056417724772	0.359082240673385\\
2.71616855270952	0.338602997986055\\
2.72904546107385	0.319269651181613\\
2.73967355164051	0.300879897931048\\
2.74841557811197	0.283257601678915\\
2.75554679891187	0.266247368715491\\
2.76127503362634	0.249709913689685\\
2.76575483893485	0.233518077021313\\
2.76909744165955	0.21755334638614\\
2.77137751332221	0.201702738965491\\
2.77263748964209	0.185855909231112\\
2.77288986443052	0.169902352837051\\
2.77211767622501	0.153728577765967\\
2.77027322578096	0.1372151074127\\
2.76727488704598	0.120233165192337\\
2.76300167813254	0.102640864674644\\
2.7572850126839	0.0842786907811072\\
2.74989671666541	0.0649640032548598\\
2.7405319138553	0.044484220158523\\
2.72878466789091	0.0225882441383666\\
2.71411318339692	-0.00102442090286279\\
2.69578969943864	-0.0267175372222971\\
2.67282762221409	-0.0549357456712831\\
2.64387443893777	-0.086226620103167\\
2.60705283594023	-0.121267363051761\\
2.55972344891701	-0.160894741313181\\
2.49813067786754	-0.206132626129549\\
2.41688109230449	-0.258201271274281\\
2.30820646225063	-0.31846933337672\\
2.16102794156049	-0.388261527787755\\
1.9600942649918	-0.468349327874165\\
1.68618532970365	-0.557854874809137\\
1.31978937039226	-0.652388661475404\\
0.851701177658143	-0.742091957403252\\
0.2996878226251	-0.812222786327461\\
-0.283403565055562	-0.849311024806142\\
-0.829131909311639	-0.849671124327323\\
-1.28856516732028	-0.820991188416604\\
-1.64777878514603	-0.775658460547183\\
-1.91749270169338	-0.724097527473434\\
-2.11699928820803	-0.67265504472608\\
-2.26472299505105	-0.624376210638953\\
-2.37516131876585	-0.58033858751573\\
-2.45884957200914	-0.540632718202448\\
-2.52320806872325	-0.504922443915455\\
-2.57342209009238	-0.472726917545838\\
-2.61312907901512	-0.44355003443715\\
-2.64490534747896	-0.416933488381197\\
-2.67059795700049	-0.392473816110554\\
-2.69154829428359	-0.36982347867641\\
-2.70874274362947	-0.34868555022563\\
-2.72291490439807	-0.328806414191347\\
-2.73461560117952	-0.309968379463334\\
-2.74426133726505	-0.291982958094855\\
-2.75216816003143	-0.274685010667489\\
-2.75857551538068	-0.257927732765276\\
-2.7636631154935	-0.241578364791952\\
-2.76756282819312	-0.225514479799037\\
-2.7703669221514	-0.209620702756656\\
-2.77213354502348	-0.193785722015356\\
-2.77288999071339	-0.177899461140576\\
-2.77263407401168	-0.161850282759114\\
-2.77133373868317	-0.145522093310267\\
-2.7689248503333	-0.128791207021445\\
-2.76530694288232	-0.111522807349164\\
-2.76033647049569	-0.0935668124380141\\
-2.75381683187417	-0.0747529051583457\\
-2.7454840339232	-0.054884424790499\\
-2.73498627666103	-0.033730733354312\\
-2.7218548619301	-0.01101756388009\\
-2.70546248424923	0.0135852635734658\\
-2.68496288535053	0.040479485056165\\
-2.65920263327697	0.0701581547916423\\
-2.62659082253035	0.103230057451741\\
-2.58490501885585	0.140448374709387\\
-2.53100118982908	0.18274056801885\\
-2.4603826165084	0.23122979984345\\
-2.36657498260569	0.287222701791902\\
-2.24027879008065	0.35210449981766\\
-2.06840652840338	0.42701652250624\\
-1.8335570524236	0.512090553182067\\
-1.51554768948008	0.604966091936343\\
-1.09817415321068	0.698692925584068\\
-0.583632372478345	0.780585967110578\\
-0.00768759183282086	0.835429000391897\\
0.564796062313798	0.853812859999501\\
};
\addplot [color=mycolor1, forget plot]
  table[row sep=crcr]{%
0.890712145070811	0.909450418757661\\
1.35615885261011	0.895223268347426\\
1.7178770133649	0.861204771906555\\
1.98883562135947	0.81827185341504\\
2.18934297244442	0.773305016262608\\
2.33813014010674	0.729851990461745\\
2.44972390595194	0.68940712468328\\
2.5346150158692	0.652382072672868\\
2.60017637536743	0.618678388031736\\
2.65156020319445	0.58798790340672\\
2.69238512568549	0.559939164936878\\
2.72521891863578	0.534163927537106\\
2.75190661716258	0.510324101099737\\
2.77379135916652	0.488119891841291\\
2.79186327696568	0.467289454487744\\
2.80686061899949	0.447605063345819\\
2.81933911290857	0.428868163364543\\
2.82972005140612	0.410904358831404\\
2.83832395807279	0.393558763793993\\
2.84539433942666	0.376691836434651\\
2.85111450317749	0.360175680602173\\
2.85561942298616	0.343890737238368\\
2.85900396567129	0.327722764353783\\
2.86132834493843	0.311559995282265\\
2.86262134708813	0.29529036032451\\
2.86288163617489	0.278798650749535\\
2.86207725184185	0.261963492909228\\
2.8601432339406	0.244653981187701\\
2.85697711819763	0.226725788861691\\
2.85243181853318	0.208016532305813\\
2.84630510839095	0.18834010198256\\
2.83832448478395	0.167479587806611\\
2.82812556800403	0.145178310381336\\
2.81522123766127	0.121128317454752\\
2.79895724360707	0.094955515907978\\
2.7784477619024	0.0662003995574959\\
2.75248083374686	0.0342931637967893\\
2.71937815937494	-0.00147795233693322\\
2.67678546413278	-0.0420055273043307\\
2.62135794787402	-0.0884069185173281\\
2.54829130368562	-0.142064245531875\\
2.45064075021913	-0.204634033811873\\
2.31839972834169	-0.277960726236148\\
2.13747019310304	-0.363747848256886\\
1.88917402634202	-0.46271092344986\\
1.55217347467057	-0.572854125554091\\
1.11031757645939	-0.686931045624751\\
0.5686004428524	-0.790900393956026\\
-0.0317095263207011	-0.867395661633784\\
-0.621049791188676	-0.905112993988739\\
-1.13650530825543	-0.905617152832956\\
-1.5495086819587	-0.87991946706539\\
-1.86345932543428	-0.840329516553253\\
-2.0966362049343	-0.795758438133868\\
-2.26919022981483	-0.751261298010872\\
-2.3978358398636	-0.70921082060183\\
-2.49498482746347	-0.67046583142385\\
-2.56944700736389	-0.635131647095147\\
-2.62738491506972	-0.602979271852111\\
-2.67311188480114	-0.573656849805782\\
-2.70967167159994	-0.546789350398643\\
-2.73923734713559	-0.522021743950802\\
-2.7633805729318	-0.49903476354143\\
-2.78325281465181	-0.477547928176154\\
-2.79970790674062	-0.457317036615965\\
-2.81338569230257	-0.438129586166547\\
-2.82476970026708	-0.419799710568105\\
-2.83422733473163	-0.402163320645339\\
-2.84203813121379	-0.385073692666452\\
-2.84841374148752	-0.368397543383253\\
-2.85351207522891	-0.352011538208387\\
-2.85744721363654	-0.335799140336732\\
-2.86029616421148	-0.319647693949695\\
-2.86210314856951	-0.303445628825438\\
-2.86288184224531	-0.28707966883864\\
-2.8626157732811	-0.270431918524477\\
-2.86125690299825	-0.253376687088789\\
-2.85872223112257	-0.235776885282436\\
-2.8548880620305	-0.217479794390573\\
-2.84958130814093	-0.198311954384588\\
-2.84256684818636	-0.178072845120389\\
-2.83352944016367	-0.156526934314425\\
-2.82204791572865	-0.133393532481749\\
-2.80755820480412	-0.108333723894724\\
-2.78929991967601	-0.0809334392355528\\
-2.76623839708287	-0.050681533377657\\
-2.73694969564132	-0.016941637589268\\
-2.69944929699592	0.0210831417040737\\
-2.65093531208999	0.0643921817503108\\
-2.58740373957987	0.114231187293853\\
-2.5030803834195	0.172122051690198\\
-2.38961812563455	0.239836476890025\\
-2.23508483033654	0.319213739722195\\
-2.0230669203918	0.411616201279232\\
-1.73304317172197	0.516683621167623\\
-1.34476477329517	0.630124557518481\\
-0.850287967294496	0.741278414960865\\
-0.271751240440297	0.833554342235435\\
0.332303788006903	0.891315090999931\\
0.890712145070809	0.909450418757661\\
};
\addplot [color=mycolor1, forget plot]
  table[row sep=crcr]{%
1.13228895415094	0.976232367083907\\
1.55453529055446	0.963380785631541\\
1.87735344896523	0.933035113576346\\
2.1184845775288	0.894825371350644\\
2.29787683748837	0.854585882214751\\
2.43227481251057	0.815326896230432\\
2.53422473485194	0.778370124569441\\
2.61269575478856	0.744139423495457\\
2.67399866107858	0.712620330446479\\
2.72257214445541	0.683604727923391\\
2.76156142008329	0.656814173994275\\
2.79322006604216	0.631958960343927\\
2.81918314674446	0.608764303135775\\
2.84065202094255	0.586980174694858\\
2.85851989743146	0.56638325395648\\
2.87345784029931	0.546775281493775\\
2.88597427105367	0.527979946664411\\
2.8964565493718	0.509839330570172\\
2.90520028440166	0.492210366795147\\
2.91243011729296	0.474961497954436\\
2.91831446402342	0.457969563902499\\
2.92297587905606	0.441116886757648\\
2.92649814196841	0.424288482122503\\
2.92893078207311	0.407369305689946\\
2.93029147526219	0.390241429259922\\
2.93056652851896	0.372781023828083\\
2.92970947811905	0.354855005560475\\
2.92763764022047	0.336317169435942\\
2.92422624024647	0.317003591131751\\
2.91929947813593	0.296727015455653\\
2.91261751677722	0.275269863030624\\
2.90385784701549	0.252375368188296\\
2.89258868678762	0.227736201213942\\
2.87823086126403	0.200979719574153\\
2.86000274508799	0.171648736327014\\
2.83683995383567	0.139176417712259\\
2.8072769916719	0.102853728926342\\
2.76927123580572	0.0617880301729651\\
2.71993968118953	0.0148527389768783\\
2.65516586661613	-0.0393677412042734\\
2.56902248807935	-0.102622103283209\\
2.45296226998788	-0.176981508802543\\
2.29481334781964	-0.264668724270688\\
2.07792820730373	-0.367504549350102\\
1.78167534396632	-0.485598501371426\\
1.38600553881976	-0.614970273259653\\
0.883559945743928	-0.744805485886824\\
0.297063929592292	-0.857551335033007\\
-0.314995550482536	-0.935748231473546\\
-0.881964052990284	-0.972192941185301\\
-1.35656539196237	-0.972743377863418\\
-1.72742597329563	-0.949699043438517\\
-2.00684087248312	-0.914467679869741\\
-2.21475183366217	-0.874719671788787\\
-2.36982151425778	-0.834722796055881\\
-2.48667231671519	-0.796519916364948\\
-2.57594861399704	-0.760908291075883\\
-2.64517968856829	-0.728051258689338\\
-2.69965514074735	-0.697816255793652\\
-2.74310677989146	-0.669949578355601\\
-2.77819289831537	-0.644162341846309\\
-2.80683000624538	-0.620170316931991\\
-2.83041747520835	-0.597710497828228\\
-2.8499897378227	-0.57654625934367\\
-2.86632008078914	-0.55646714197932\\
-2.87999208997297	-0.537286293160247\\
-2.89144933133931	-0.518837048204695\\
-2.90103022890587	-0.500969346659074\\
-2.90899273449339	-0.483546279852304\\
-2.91553183885285	-0.466440864345504\\
-2.9207919572066	-0.449533035674002\\
-2.9248755439451	-0.432706806771718\\
-2.92784882832088	-0.415847509214427\\
-2.9297452356072	-0.398839018717003\\
-2.9305668130403	-0.381560851171685\\
-2.93028377937382	-0.36388499688355\\
-2.92883213178227	-0.345672334683861\\
-2.9261090476084	-0.326768430524965\\
-2.92196558218108	-0.306998472609269\\
-2.91619585082817	-0.286161021550744\\
-2.90852144106672	-0.264020152464737\\
-2.89856915113163	-0.24029542772493\\
-2.88583917141691	-0.214648955648693\\
-2.86965932428903	-0.186668556753896\\
-2.8491186548788	-0.155845786349874\\
-2.8229700613926	-0.121547306806751\\
-2.78948610531244	-0.0829780416000335\\
-2.74624382975354	-0.0391351503135862\\
-2.6898027911544	0.0112456755571244\\
-2.61522702752864	0.0697429611071754\\
-2.5153954382363	0.13827444486632\\
-2.38007896894432	0.21902535174251\\
-2.19493334032315	0.314123658023469\\
-1.9410880248057	0.424762276878524\\
-1.59722619747505	0.549365716412347\\
-1.14760097929895	0.6808116647911\\
-0.597875475194193	0.804535971041456\\
0.0103610704461334	0.901750895964801\\
0.607930437625238	0.959080304254248\\
1.13228895415094	0.976232367083907\\
};
\addplot [color=mycolor1, forget plot]
  table[row sep=crcr]{%
1.31724795584908	1.04879793416756\\
1.70068896800859	1.03714926463591\\
1.99243412563981	1.00972335390571\\
2.21128528948877	0.975035371074232\\
2.37559068430626	0.938170427444379\\
2.50007029924285	0.901800397049627\\
2.59560899674705	0.867161038389809\\
2.66999240803846	0.834708218963402\\
2.72873547749773	0.804501169978652\\
2.77575356376531	0.776411417061493\\
2.81384956025627	0.750231993046598\\
2.84505236265008	0.725732440318642\\
2.87084839353232	0.702685120391255\\
2.89233964894694	0.680876613176355\\
2.91035210944293	0.660111518528291\\
2.92551071394318	0.640212499121538\\
2.93829169714277	0.621018557932336\\
2.9490594557421	0.602382565596242\\
2.95809270468511	0.58416853462776\\
2.96560310047765	0.56624886235795\\
2.97174845934883	0.548501617650293\\
2.97664199416488	0.530807866768432\\
2.98035851181535	0.513048988155423\\
2.98293817100018	0.495103896110834\\
2.98438814435474	0.476846068882062\\
2.9846823200263	0.458140250826432\\
2.98375898607868	0.438838666034884\\
2.98151623928633	0.41877653748519\\
2.9778046186795	0.397766646238988\\
2.9724161465914	0.375592583047378\\
2.96506851384193	0.352000231958747\\
2.95538249317645	0.326686872401422\\
2.94284968615133	0.299287082441922\\
2.92678621646407	0.269354364850047\\
2.90626568799413	0.236337108267457\\
2.88002119272049	0.199547193952261\\
2.84630076925584	0.158119443323986\\
2.80265275945175	0.110960658291323\\
2.74560663983224	0.0566894522876568\\
2.67020291053394	-0.00642470003833229\\
2.56932201190236	-0.0804962544596073\\
2.43279967300743	-0.167962268872052\\
2.24648657562714	-0.271266028968002\\
1.9919207905751	-0.391979782927913\\
1.64840702094397	-0.528949767320616\\
1.20066961434231	-0.675428087235624\\
0.65373091120456	-0.816901827471298\\
0.0466851295588091	-0.933778186550022\\
-0.554257251248188	-1.01071739348441\\
-1.08724660964121	-1.04507959019524\\
-1.52142481475842	-1.04562549335195\\
-1.85690865491003	-1.02478708833223\\
-2.10975536194308	-0.992899466037865\\
-2.29924077089661	-0.956664390850311\\
-2.44204554602821	-0.919821926019723\\
-2.55090826368052	-0.884223183205733\\
-2.63505602361024	-0.850651395512304\\
-2.70104332736356	-0.819329232452809\\
-2.75351347363108	-0.790203558051256\\
-2.79577501936664	-0.76309717269976\\
-2.83020913831105	-0.737786709899157\\
-2.85854980616021	-0.714040992926969\\
-2.88207503608867	-0.691638662656213\\
-2.90173766121912	-0.670375131436804\\
-2.9182553876661	-0.650064167795247\\
-2.93217336191435	-0.630536882903907\\
-2.94390805059626	-0.61163954606285\\
-2.95377827091342	-0.593230944566961\\
-2.9620272587906	-0.575179625835013\\
-2.96883837402682	-0.557361158923949\\
-2.97434618406172	-0.539655444784479\\
-2.97864408732369	-0.521944044947472\\
-2.98178923332578	-0.50410746231682\\
-2.98380520340743	-0.48602228163461\\
-2.98468268775975	-0.467558052713413\\
-2.98437819755078	-0.448573771047592\\
-2.98281065760069	-0.428913773217576\\
-2.9798555075506	-0.408402813770303\\
-2.97533566543103	-0.386840020298865\\
-2.96900833349175	-0.363991327101009\\
-2.96054608874203	-0.339579856150916\\
-2.9495099034091	-0.313273536888723\\
-2.93531053375228	-0.284669024188519\\
-2.91715286645675	-0.253270685689284\\
-2.89395496323989	-0.218463112035864\\
-2.86422917280108	-0.179475359715508\\
-2.82590609659725	-0.135335264769788\\
-2.77607276040422	-0.0848134410459427\\
-2.7105844092644	-0.0263608877946367\\
-2.62349944761116	0.041944146798071\\
-2.5062973391022	0.122395838482514\\
-2.34692584809889	0.217499779025785\\
-2.12902993939484	0.329424887102003\\
-1.83250714465007	0.458686424420379\\
-1.4379374353601	0.601721274169193\\
-0.938024113384351	0.74798013060162\\
-0.353923457943785	0.879607203169263\\
0.258945755246423	0.977741613950653\\
0.832029281899767	1.0328566526933\\
1.31724795584908	1.04879793416756\\
};
\addplot [color=mycolor1, forget plot]
  table[row sep=crcr]{%
1.46475086798591	1.12486267541903\\
1.81455064496186	1.11424173504341\\
2.08101811692771	1.0891848432148\\
2.28239477396513	1.05725633865747\\
2.43516101366466	1.02297096369503\\
2.55222786165662	0.988759043180935\\
2.6431073278131	0.955802910425065\\
2.7146373091078	0.924590286752199\\
2.77170413818777	0.895241476523484\\
2.81781236316433	0.867692314960797\\
2.85549699948198	0.841793134484645\\
2.88661151674617	0.817360852200531\\
2.91252656590965	0.794205432401846\\
2.93426699299933	0.772142515743081\\
2.95260677318198	0.750998663464851\\
2.96813529734919	0.73061271717463\\
2.98130405122357	0.710835161507174\\
2.9924597435533	0.691526494724127\\
3.00186794581836	0.672555128082024\\
3.00972997636107	0.653795067109678\\
3.01619486979958	0.63512347785137\\
3.02136766535384	0.616418154327055\\
3.02531482399341	0.597554849653668\\
3.02806727630516	0.578404395087712\\
3.02962136384206	0.558829498131974\\
3.02993773187217	0.538681075652315\\
3.02893803222613	0.517793934940117\\
3.02649907337657	0.495981559240114\\
3.0224437786017	0.473029678094332\\
3.0165279391859	0.4486881991649\\
3.00842121624705	0.422660937455598\\
2.9976800587588	0.394592389157719\\
2.98370902355612	0.364050550863591\\
2.96570518575104	0.330504479387123\\
2.94257758377414	0.293294950382676\\
2.91282947836327	0.251596311720892\\
2.8743850248728	0.204367746024086\\
2.82433329036807	0.150293442953563\\
2.75855200597052	0.0877155387590932\\
2.67116583336082	0.0145754690441363\\
2.55380702427253	-0.0715921184010218\\
2.39473271379309	-0.173506451243602\\
2.17814159007815	-0.293605490771841\\
1.88475266322618	-0.432752885013195\\
1.49594165548874	-0.587840504337622\\
1.00421008637066	-0.748816520881164\\
0.428204497449267	-0.897960863886098\\
-0.181069342161435	-1.01542826939592\\
-0.758071027725421	-1.08942430520555\\
-1.25383701882624	-1.12144949613573\\
-1.65114711302072	-1.12196791913176\\
-1.95706424890432	-1.10296348206271\\
-2.18873552451611	-1.0737368824625\\
-2.36396063075085	-1.04021870875464\\
-2.49749047958932	-1.00576048059793\\
-2.60046006633146	-0.972081963203389\\
-2.68094737638745	-0.93996520250237\\
-2.74473272823448	-0.909684049981042\\
-2.79595080402652	-0.881250064378457\\
-2.83757850259748	-0.854547535635447\\
-2.87178039260318	-0.829405541040784\\
-2.90014809676444	-0.805635276710233\\
-2.92386524475683	-0.783048536119421\\
-2.94382142623585	-0.761466065933136\\
-2.96069143501722	-0.74072055377238\\
-2.97499083610984	-0.720656820926039\\
-2.98711525579079	-0.701130600286\\
-2.99736835333593	-0.682006626587287\\
-3.00598180523958	-0.66315640643376\\
-3.01312954511842	-0.644455835646819\\
-3.01893776823853	-0.625782717974056\\
-3.02349170434944	-0.607014171680475\\
-3.02683980356726	-0.588023866192023\\
-3.0289957113926	-0.568678996357998\\
-3.02993819078351	-0.548836868510815\\
-3.02960895059597	-0.528340934081635\\
-3.02790813295673	-0.507016057545366\\
-3.02468696775183	-0.484662740045331\\
-3.01973678363577	-0.461049931197972\\
-3.01277312067384	-0.435905940663681\\
-3.00341304428323	-0.408906797741841\\
-2.99114279818484	-0.379661190623549\\
-2.97527147807427	-0.347690840003647\\
-2.95486418674673	-0.312404833834234\\
-2.92864474542825	-0.273066128691932\\
-2.89485294071803	-0.228748292490477\\
-2.85103389105989	-0.178281118439137\\
-2.79372728790955	-0.120186213158118\\
-2.71801409109878	-0.0526109543084696\\
-2.6168773368519	0.0267122368710863\\
-2.48037329926872	0.120411627235978\\
-2.29477385383028	0.231169688316104\\
-2.04231301726284	0.360863536376334\\
-1.7031762285242	0.50873932397908\\
-1.26255393483827	0.668547632236554\\
-0.724234652787264	0.82617364302703\\
-0.12359505347142	0.961691386230613\\
0.477364392125094	1.05806587603998\\
1.01792006988566	1.11014470800911\\
1.46475086798591	1.12486267541903\\
};
\addplot [color=mycolor1, forget plot]
  table[row sep=crcr]{%
1.58697213121126	1.20336136317783\\
1.9076340390308	1.19362316957183\\
2.1529623596323	1.17054474461314\\
2.33998746848524	1.14088153826959\\
2.48338418761784	1.10869017780001\\
2.59449987839305	1.07621039709076\\
2.68170112752524	1.04458260799585\\
2.75104372054698	1.01432015572903\\
2.80689578120153	0.985592670342716\\
2.8524221421894	0.958388406892165\\
2.88993526501501	0.932604821861844\\
2.92114215149815	0.908098078821548\\
2.94731633975157	0.884709440025805\\
2.96941768474312	0.86227875839831\\
2.98817617409695	0.840650810103552\\
3.00415099310331	0.819677676168766\\
3.01777246146683	0.799218958659433\\
3.02937199965124	0.77914081640432\\
3.03920361601012	0.759314352276649\\
3.04745928077584	0.739613624834528\\
3.05427978771138	0.719913405723052\\
3.05976217492766	0.700086712141951\\
3.06396439936458	0.680002083025932\\
3.0669076774521	0.659520521792054\\
3.06857667662451	0.638491987024067\\
3.06891753693403	0.616751267716092\\
3.06783349085648	0.594113025247864\\
3.06517760298044	0.570365713580987\\
3.06074183339464	0.545263994757396\\
3.05424118955854	0.518519139618314\\
3.04529109862083	0.489786733138532\\
3.03337519605724	0.458650778994819\\
3.01779932079387	0.424603012444904\\
2.99762538192359	0.387015895099739\\
2.97157555727582	0.345107440606509\\
2.93789252757304	0.297895896533356\\
2.894134677273	0.244142888401816\\
2.83687646773538	0.182286152050274\\
2.76127581055905	0.110370308801734\\
2.66047170131774	0.0260025409344433\\
2.52481587585663	-0.0735981350095048\\
2.34109787068459	-0.191304729055157\\
2.09235121362609	-0.329248668687806\\
1.75971183833163	-0.48704866481498\\
1.32881027591122	-0.658999800722057\\
0.802036457999092	-0.831568463994606\\
0.210734221357192	-0.984824584999507\\
-0.38777641280512	-1.10035461917923\\
-0.934360738786419	-1.17053852025654\\
-1.39336140266842	-1.20022669008197\\
-1.75779374171375	-1.20070866807381\\
-2.03862293124132	-1.18325589981369\\
-2.25276893646319	-1.15622999995266\\
-2.41635313761951	-1.12492897948904\\
-2.54239200337834	-1.09239593812663\\
-2.64066508308718	-1.06024728385831\\
-2.71829858239608	-1.02926440743518\\
-2.78043476129419	-0.999762345382136\\
-2.83078849227627	-0.971805143099149\\
-2.87206179875188	-0.945327442609912\\
-2.90623884510072	-0.920201620544917\\
-2.93479211592366	-0.896274071926745\\
-2.95882589717626	-0.873384213184259\\
-2.97917639446744	-0.851373868808164\\
-2.99648202677968	-0.830091335297682\\
-3.01123315083976	-0.80939251750509\\
-3.02380748740073	-0.789140465686734\\
-3.03449549213506	-0.769204040001337\\
-3.04351854501094	-0.749456087159715\\
-3.05104190548527	-0.72977131651809\\
-3.05718374616062	-0.710023945644536\\
-3.062021132934	-0.690085111628488\\
-3.06559349611069	-0.669819992783676\\
-3.06790388643737	-0.649084542835442\\
-3.06891809699603	-0.627721697409026\\
-3.06856152709846	-0.605556863823372\\
-3.06671343951736	-0.582392443487743\\
-3.06319798543667	-0.558001054676279\\
-3.05777100043022	-0.532117013905994\\
-3.0501010498453	-0.504425486699422\\
-3.03974243419978	-0.474548522123226\\
-3.02609671962065	-0.442026930521274\\
-3.0083576305549	-0.406296649987609\\
-2.98543152971664	-0.366657904102167\\
-2.95582179344142	-0.322235192168796\\
-2.91745967145321	-0.271926286852041\\
-2.86745638659513	-0.214339737284079\\
-2.80174211919316	-0.147724783229074\\
-2.71455205963207	-0.0699093564136836\\
-2.59773493723347	0.0217100643328273\\
-2.43994355498867	0.130022399139329\\
-2.22603357434645	0.257683214119972\\
-1.93763912498983	0.405860982827894\\
-1.55696130812789	0.571904053643628\\
-1.07620703943236	0.74636557032752\\
-0.511274730013405	0.911924373514858\\
0.0916394559526066	1.04810571255123\\
0.670514117540115	1.14105418007271\\
1.17582961916804	1.18979941964901\\
1.58697213121126	1.20336136317783\\
};
\addplot [color=mycolor1, forget plot]
  table[row sep=crcr]{%
1.69154852785871	1.28370401175513\\
1.98672928216181	1.27473415003994\\
2.21390108424491	1.25335406547444\\
2.3886721671133	1.22562492412593\\
2.5240760054015	1.19521991116207\\
2.63011715172874	1.1642170823112\\
2.71419350409143	1.13371768638542\\
2.78169954664035	1.10425279247967\\
2.83656251862296	1.07603088498327\\
2.88165563630413	1.04908292777186\\
2.91909840184877	1.02334554935167\\
2.9504695155905	0.998707999571044\\
2.97695656811281	0.975038180312408\\
2.99946129654955	0.952196653784043\\
3.01867392699092	0.930043754760351\\
3.03512601603836	0.908442745974919\\
3.04922825187172	0.887260696633115\\
3.06129762746649	0.866368037516756\\
3.07157699757694	0.845637323770712\\
3.08024907326989	0.824941487340937\\
3.08744624698973	0.804151710236768\\
3.0932571761229	0.783134953971837\\
3.0977307141138	0.761751114076917\\
3.10087751639272	0.739849715679729\\
3.10266942737847	0.717266016282757\\
3.10303654479367	0.693816327060933\\
3.10186163011529	0.66929229710414\\
3.0989712573794	0.643453818580479\\
3.09412272589533	0.616020096217404\\
3.0869852483128	0.586658271738282\\
3.07711317949214	0.554968791982241\\
3.06390794727981	0.520466449283701\\
3.04656369490717	0.482555705357621\\
3.0239891727972	0.440498567965223\\
2.99469475660702	0.393373038710646\\
2.95662820936091	0.340020310101148\\
2.90693577599555	0.278980258156785\\
2.84161740355516	0.208419226490437\\
2.75504125426462	0.126065791553188\\
2.63929931271235	0.0291976991417976\\
2.4834665440008	-0.0852183864878517\\
2.27307022427829	-0.220025901007928\\
1.99064539634526	-0.376669674047728\\
1.61917471026681	-0.552941606294356\\
1.15055721309653	-0.740034199829811\\
0.598048773833731	-0.921163600349624\\
0.00314733408814727	-1.07549448712044\\
-0.575841737161633	-1.18736911049668\\
-1.08929438614472	-1.25336201237515\\
-1.51347869615247	-1.28082049737753\\
-1.84866477490039	-1.2812637280361\\
-2.10779694582459	-1.26515094305336\\
-2.3069495686061	-1.24000737687748\\
-2.46060174261895	-1.21059800012478\\
-2.58025054920589	-1.17970719790656\\
-2.67452367686858	-1.1488614252525\\
-2.74974342309568	-1.11883741880872\\
-2.8105111479618	-1.08998157033276\\
-2.86018303277302	-1.06240009352769\\
-2.90122398017439	-1.03606910801023\\
-2.93546093651639	-1.01089725903163\\
-2.964261486689	-0.986760774587864\\
-2.98865933226719	-0.963522655409174\\
-3.00944268677859	-0.941042759048854\\
-3.02721690008306	-0.919182661092823\\
-3.04244911808689	-0.897807516718047\\
-3.05550031833885	-0.876786189958432\\
-3.06664836696297	-0.855990364817512\\
-3.07610458458484	-0.835293028743695\\
-3.08402551471939	-0.814566525998151\\
-3.09052103557451	-0.79368025920963\\
-3.09565956145848	-0.772498038722634\\
-3.09947078479444	-0.750875021230973\\
-3.10194617235881	-0.728654128915337\\
-3.10303721737287	-0.705661788879955\\
-3.10265123399875	-0.681702772720765\\
-3.10064423317521	-0.656553840372643\\
-3.09681010303249	-0.629955793032172\\
-3.09086488530946	-0.601603407626702\\
-3.08242432191236	-0.571132549326719\\
-3.07097193975157	-0.538103528363932\\
-3.05581359250599	-0.501979477312295\\
-3.03601235654665	-0.462098187829636\\
-3.01029466309495	-0.417635526260147\\
-2.97691413610257	-0.367558441195495\\
-2.93345344412972	-0.310566192624902\\
-2.87653678974171	-0.245021002147175\\
-2.80141888517497	-0.168876660325787\\
-2.70141956169065	-0.0796317756926746\\
-2.56721359818832	0.0256249862756282\\
-2.38613071547999	0.149929294633429\\
-2.14200616450009	0.295636668224313\\
-1.81689468557862	0.46271509430863\\
-1.3968259799178	0.646009298609677\\
-0.882832291688758	0.832645321473129\\
-0.302368412278713	1.00289687492031\\
0.29193820929031	1.13726589176048\\
0.842910913972665	1.22582272036213\\
1.31296805875141	1.27120728874838\\
1.69154852785871	1.28370401175513\\
};
\addplot [color=mycolor1, forget plot]
  table[row sep=crcr]{%
1.7832981678353	1.36546589275169\\
2.05592568115162	1.3571744194569\\
2.26714164491445	1.33728673652804\\
2.43111853478521	1.31126156725633\\
2.55943271347477	1.28244136021363\\
2.66093354729971	1.25276022517939\\
2.74218962233567	1.22327936216145\\
2.80802606279813	1.19453958765605\\
2.86198660051068	1.16677896496198\\
2.90668763683666	1.14006289696381\\
2.94407647731309	1.11436054257094\\
2.97561575220362	1.08958916115493\\
3.00241420566575	1.06563949952862\\
3.02531948618789	1.04239000684265\\
3.04498425031883	1.0197144567338\\
3.06191351815595	0.997485660984367\\
3.07649877708056	0.975576844041152\\
3.08904261974966	0.95386158977852\\
3.09977651875239	0.932212879917087\\
3.10887352086999	0.910501505895431\\
3.11645707010401	0.888593987389956\\
3.12260675718842	0.866350032379446\\
3.12736148556441	0.84361950212847\\
3.13072029732992	0.820238784686038\\
3.13264088455171	0.796026421968094\\
3.13303559280109	0.770777769825975\\
3.13176447555128	0.744258389841893\\
3.12862464535523	0.716195767549337\\
3.12333474415734	0.686268814798393\\
3.1155127546208	0.654094433393716\\
3.10464449980951	0.619210182674226\\
3.09003888625681	0.581051800749411\\
3.07076402681151	0.538923992473511\\
3.04555554798029	0.491962585008751\\
3.01268429798848	0.439086064580252\\
2.96976506994043	0.37893515870206\\
2.91348118717393	0.309801736731583\\
2.83919435906196	0.229555616004797\\
2.74041386938704	0.135595692071535\\
2.60813878932407	0.0248912135021348\\
2.43022180279671	-0.105743493774361\\
2.19124961386915	-0.258874318262122\\
1.87411209379186	-0.434804268222938\\
1.46518116380506	-0.628916844714507\\
0.964245059921325	-0.829016773640376\\
0.395207777190675	-1.01569772268623\\
-0.193748734909379	-1.1686142836128\\
-0.747622298683288	-1.2757260230298\\
-1.22738617820585	-1.33743352552652\\
-1.61919514519511	-1.36280865849501\\
-1.92824504599151	-1.36321391987884\\
-2.16827266488954	-1.34828035900592\\
-2.35424440438222	-1.32479177202571\\
-2.49912053823983	-1.29705421238368\\
-2.61307727724595	-1.26762654584206\\
-2.7037556868631	-1.23795184232903\\
-2.77678844889193	-1.20879671834807\\
-2.83630900148409	-1.18052984170452\\
-2.88535963560319	-1.15329067060285\\
-2.92619505892335	-1.12708932719676\\
-2.96050096090471	-1.10186489456276\\
-2.98954936666697	-1.07751903751993\\
-3.01430876229406	-1.0539350695173\\
-3.03552236243188	-1.03098844046375\\
-3.05376401994576	-1.00855215247121\\
-3.06947838995699	-0.98649915687812\\
-3.08300991163942	-0.964702930102649\\
-3.09462374740807	-0.943036918843076\\
-3.10452083500761	-0.921373240822968\\
-3.11284852374633	-0.899580840330465\\
-3.11970778215795	-0.877523178040283\\
-3.12515761091966	-0.855055451989273\\
-3.12921702232672	-0.832021282451107\\
-3.13186471906918	-0.808248735438497\\
-3.13303639024817	-0.783545498449622\\
-3.13261931307637	-0.757692949906779\\
-3.13044367314647	-0.730438772495175\\
-3.1262696548114	-0.701487641376272\\
-3.11976885109067	-0.670489360837329\\
-3.11049781965361	-0.63702361653161\\
-3.09786054958415	-0.600580246600419\\
-3.08105502897536	-0.560533616143728\\
-3.05899676892009	-0.516109340945298\\
-3.03020872281921	-0.466341368002129\\
-2.99266220849364	-0.410017606742598\\
-2.94354713401342	-0.345613712643242\\
-2.87894316526232	-0.271219097624817\\
-2.7933613052015	-0.18447081327201\\
-2.67914274745981	-0.0825377845677898\\
-2.5257783825343	0.0377458731415323\\
-2.31943501649625	0.179398198627669\\
-2.04347928473513	0.344126308788451\\
-1.68159158956786	0.530151902243493\\
-1.22537153655025	0.729304947360204\\
-0.685671303026336	0.925396871033158\\
-0.099644329037568	1.09741544503237\\
0.478086993696043	1.22814867876394\\
0.998202340115935	1.3118123673045\\
1.43427748430264	1.35394217752062\\
1.78329816783529	1.36546589275169\\
};
\addplot [color=mycolor1, forget plot]
  table[row sep=crcr]{%
1.86527941841213	1.44825240370401\\
2.11771325155239	1.44056775786807\\
2.3146438691887	1.42201660055715\\
2.46888019254694	1.39752963749764\\
2.59071684335125	1.37015790542439\\
2.68800519087505	1.34170343400817\\
2.76659677043846	1.31318515783504\\
2.83082010235443	1.2851462271798\\
2.88388025783308	1.25784608835794\\
2.92816364116387	1.23137735726302\\
2.96546102639441	1.20573592840368\\
2.99712789566332	1.1808626349622\\
3.02419904593052	1.15666774659049\\
3.04747055975572	1.13304512390001\\
3.06755865595758	1.10988011611193\\
3.08494214443492	1.08705364341692\\
3.09999317660819	1.06444391896843\\
3.11299954624969	1.0419266711386\\
3.12418079062586	1.01937436463494\\
3.13369963737515	0.996654693924047\\
3.1416698415423	0.973628477145177\\
3.14816109038466	0.950146978784336\\
3.15320137109218	0.926048613223259\\
3.1567769610436	0.901154914561983\\
3.15882998113876	0.875265590354205\\
3.15925322167056	0.848152399268906\\
3.15788167619843	0.819551496607721\\
3.15447986339269	0.789153767778748\\
3.14872352711161	0.756592507909527\\
3.14017360532388	0.721427595068109\\
3.12823933821891	0.683125037112243\\
3.1121258845059	0.641030451681497\\
3.09075960873231	0.594334702650082\\
3.06268100836197	0.542029687526621\\
3.02589079683469	0.482852475934354\\
2.97762897175065	0.415217431606972\\
2.91406095861527	0.33714041809399\\
2.82984383017217	0.246170597184527\\
2.71756319068948	0.139371466952623\\
2.56710335408355	0.0134469599782818\\
2.36521648184592	-0.134794739196671\\
2.09600634147434	-0.307322133291899\\
1.74374612949788	-0.502780130865966\\
1.29973652733544	-0.713622372095883\\
0.772685574497584	-0.924268036294543\\
0.195840532604639	-1.1136404366881\\
-0.379672184980449	-1.26317811794792\\
-0.905127232671636	-1.36486458989047\\
-1.35185024182953	-1.42235333482624\\
-1.71378205463731	-1.44579970039503\\
-1.99931648626235	-1.44616912203778\\
-2.2222613254083	-1.43229003289289\\
-2.39639418110116	-1.41028844269828\\
-2.53330505257749	-1.38406879379923\\
-2.64202280717088	-1.35598825095052\\
-2.72933700426184	-1.32740987741486\\
-2.80028183599667	-1.29908456453855\\
-2.85858020554726	-1.27139510629733\\
-2.90699530226382	-1.24450638015938\\
-2.947592193925	-1.21845598805502\\
-2.98192717794727	-1.19320836075959\\
-3.01118336923205	-1.16868675508286\\
-3.03626759247698	-1.14479193034465\\
-3.05787979721232	-1.1214127859477\\
-3.07656301310694	-1.09843211952335\\
-3.09273946787627	-1.0757293913073\\
-3.10673677727398	-1.05318161540029\\
-3.11880691461837	-1.03066303560194\\
-3.12913982678066	-1.00804395896556\\
-3.13787297099832	-0.985188940451968\\
-3.14509762024053	-0.961954392812934\\
-3.15086246552017	-0.938185609944088\\
-3.15517478867469	-0.913713121998979\\
-3.15799925541525	-0.888348234484449\\
-3.1592541569643	-0.86187753192624\\
-3.15880468027373	-0.83405604094017\\
-3.1564524775992	-0.804598638814661\\
-3.1519203913931	-0.773169152187795\\
-3.14483060713148	-0.739366405369039\\
-3.13467366330692	-0.70270623935128\\
-3.12076451078705	-0.662598226770153\\
-3.1021799921693	-0.618315471439615\\
-3.07766944933406	-0.568955574380272\\
-3.04552637025706	-0.513390775284264\\
-3.00340387231053	-0.450205952538559\\
-2.94805082150653	-0.377625791225276\\
-2.87494099838897	-0.293439685493208\\
-2.77777324861214	-0.194950415666673\\
-2.64785882128024	-0.0790107997470756\\
-2.47353686337519	0.0577132524403142\\
-2.24007102783224	0.217997577923155\\
-1.93107839760373	0.402477340836931\\
-1.53320039607419	0.607063582008598\\
-1.04509831762917	0.820231299118718\\
-0.487511116244679	1.02294809819989\\
0.0954524566832367	1.19419249706947\\
0.651027569613964	1.32000441649981\\
1.13916833508337	1.39857409095482\\
1.54310927986971	1.43761596787514\\
1.86527941841213	1.44825240370401\\
};
\addplot [color=mycolor1, forget plot]
  table[row sep=crcr]{%
1.93943026851774	1.53163957203499\\
2.17360049581859	1.52450375594013\\
2.35755267889386	1.507167450353\\
2.50283805042945	1.48409467494661\\
2.61862701362222	1.45807589297092\\
2.711904720556	1.43078980113568\\
2.78789794972007	1.40321063191535\\
2.85049812106648	1.37587728963447\\
2.90260827294845	1.34906340471532\\
2.94640665985586	1.32288242012228\\
2.98354018983974	1.29735179089412\\
3.01526428573785	1.27243191584892\\
3.04254355050506	1.24804954958427\\
3.06612428546851	1.22411167389182\\
3.08658691232949	1.20051347002365\\
3.10438401945045	1.17714260332082\\
3.1198680486297	1.15388115911178\\
3.1333114246421	1.13060603329294\\
3.14492107221718	1.10718824793783\\
3.15484865623774	1.08349144998392\\
3.16319744069907	1.05936970978761\\
3.17002633210458	1.03466463528547\\
3.17535141046174	1.00920173680874\\
3.17914502232256	0.982785903580248\\
3.18133228639511	0.955195775051067\\
3.1817846142805	0.926176699198573\\
3.18030954356728	0.895431856345077\\
3.17663577468156	0.862610980826365\\
3.17039173647708	0.827295923087385\\
3.16107519214855	0.788982051656446\\
3.14801021191297	0.747054194608803\\
3.13028610628902	0.700755480999907\\
3.10667040258241	0.649147136556571\\
3.0754844031259	0.591057219627259\\
3.03442515971809	0.525016963546477\\
2.98031231722511	0.449186006034351\\
2.9087346617035	0.361274943673765\\
2.81357717269425	0.258490731375111\\
2.68644585098409	0.137567426068143\\
2.51612277515245	-0.00498383250365952\\
2.28846610535659	-0.172159159422162\\
1.98770377648844	-0.364935344947061\\
1.60066040195149	-0.579748947424175\\
1.1249587311468	-0.805731300108648\\
0.578529135762317	-1.02424182226127\\
0.0018151964207621	-1.21369394943156\\
-0.554595484228764	-1.35836326169271\\
-1.04995644620431	-1.4542802652993\\
-1.46493689603643	-1.50770529730677\\
-1.79936987895032	-1.52937259081181\\
-2.06360046551447	-1.52970881922359\\
-2.27107724781439	-1.51678493296196\\
-2.43440000580924	-1.4961417936104\\
-2.56393648952145	-1.47132809713904\\
-2.66771714564749	-1.44451759463789\\
-2.75179187549888	-1.41699534929173\\
-2.82067103513665	-1.38949139261048\\
-2.87771388935971	-1.36239547667343\\
-2.92543291381231	-1.33589101005423\\
-2.96572023314904	-1.3100372813353\\
-3.00001227802128	-1.28481949674121\\
-3.02940845809071	-1.26017900999828\\
-3.05475657477219	-1.23603138924932\\
-3.07671444666484	-1.21227698949568\\
-3.09579454834497	-1.18880686910994\\
-3.11239646055606	-1.16550577220273\\
-3.12683048862806	-1.14225321584285\\
-3.13933478547637	-1.11892329975785\\
-3.15008759391245	-1.09538359128941\\
-3.15921570692029	-1.07149326640162\\
-3.1667998656831	-1.04710056941051\\
-3.1728775236196	-1.02203956524038\\
-3.17744316287597	-0.996126082116127\\
-3.18044612691394	-0.969152667859541\\
-3.18178570066164	-0.940882299647012\\
-3.18130289768291	-0.911040485996277\\
-3.17876806450486	-0.879305271180267\\
-3.17386293502143	-0.845294485638946\\
-3.16615509087293	-0.808549370543012\\
-3.15506180294167	-0.768513432790178\\
-3.13979879576737	-0.724505062774383\\
-3.11930738664225	-0.675682109879071\\
-3.09215045321041	-0.620996384380717\\
-3.05636355479973	-0.559136266347788\\
-3.00924235488592	-0.488457035834185\\
-2.94704246831324	-0.406902965112206\\
-2.86456748527221	-0.311936416048813\\
-2.75463822885288	-0.200514624736791\\
-2.6075037101399	-0.0692071419161662\\
-2.41043964816916	0.0853601783129438\\
-2.14818441113122	0.265428168733131\\
-1.80548913225713	0.470070070247356\\
-1.37333584735421	0.692353225606642\\
-0.858513701023624	0.917297761136096\\
-0.29079596436584	1.12382307127971\\
0.281914664267367	1.29216766341901\\
0.811647335494608	1.41220223974685\\
1.26785920299175	1.48566882945716\\
1.64171802014129	1.52181328657332\\
1.93943026851773	1.53163957203499\\
};
\addplot [color=mycolor1, forget plot]
  table[row sep=crcr]{%
2.00696136401616	1.61515115761543\\
2.2244763751215	1.60851615796231\\
2.39650471451627	1.59229668680998\\
2.53345495378374	1.57054151686175\\
2.64351148367024	1.54580578256746\\
2.73290434414652	1.51965195360396\\
2.80631279325486	1.49300744753006\\
2.86724111811561	1.46640128633536\\
2.91832130806057	1.44011502928095\\
2.96154209634674	1.41427730154465\\
2.99841739547142	1.38892246172996\\
3.03010867183627	1.36402680232611\\
3.057513552894	1.3395307305017\\
3.08133005773032	1.31535217735229\\
3.10210330264744	1.29139447609956\\
3.12025956983251	1.26755070454325\\
3.1361311872701	1.24370571539386\\
3.14997463568489	1.219736596381\\
3.16198356202128	1.19551199641695\\
3.17229784946792	1.17089055449665\\
3.1810095042562	1.14571853095339\\
3.18816581973541	1.11982663858188\\
3.19377003078697	1.09302598566912\\
3.19777944566942	1.06510296096169\\
3.20010080960119	1.03581280137891\\
3.20058238477912	1.00487147697869\\
3.19900188865654	0.971945394114084\\
3.1950489674993	0.936638245959665\\
3.18830022702405	0.898474118681664\\
3.17818389819098	0.856875683725423\\
3.16392984492455	0.811135976447896\\
3.14449863349101	0.760381917347674\\
3.11848055397372	0.703527498527821\\
3.08395162400545	0.639214758512009\\
3.03826882809506	0.565742073837507\\
2.97778233785116	0.480983635635591\\
2.89744247597732	0.382314922117848\\
2.79029578776044	0.266583707551525\\
2.64692727178493	0.130216750300535\\
2.45507544953773	-0.0303578786756252\\
2.20001268640501	-0.217674749787857\\
1.86685064606414	-0.431254689935425\\
1.44620976602339	-0.664782761045597\\
0.943127699983683	-0.903873517789795\\
0.38427994838395	-1.12747132535379\\
-0.185353231066425	-1.31470988115095\\
-0.718599426479986	-1.45343541687645\\
-1.18333170544918	-1.54346315654748\\
-1.5681980712356	-1.59302554230365\\
-1.87733032808084	-1.61305360077976\\
-2.12214127410076	-1.61335939866085\\
-2.31547278170568	-1.60130961030503\\
-2.46880280663759	-1.58192293699056\\
-2.59141439882153	-1.55843016806524\\
-2.69046679613017	-1.53283655796367\\
-2.77136399372475	-1.50635073996917\\
-2.83815504437088	-1.47967750060326\\
-2.89387509055317	-1.45320736984484\\
-2.94081000509277	-1.42713625155903\\
-2.98069325725668	-1.40153995435671\\
-3.01484960365108	-1.37642030837201\\
-3.04429922644163	-1.35173353392588\\
-3.06983315014823	-1.32740752815835\\
-3.09206798947786	-1.3033521953867\\
-3.11148582640992	-1.27946536477639\\
-3.12846332614156	-1.25563585908362\\
-3.14329298033504	-1.2317446693827\\
-3.15619849444608	-1.20766480801879\\
-3.16734571270759	-1.18326016620529\\
-3.17685002186825	-1.15838353860248\\
-3.18478083537304	-1.13287386037785\\
-3.19116349026798	-1.10655261035256\\
-3.19597865600009	-1.07921925153787\\
-3.19915912844725	-1.05064549600183\\
-3.20058363534984	-1.0205680844799\\
-3.20006697823932	-0.988679652526426\\
-3.19734543923824	-0.954617103743766\\
-3.19205583125112	-0.917946715583595\\
-3.18370578514848	-0.878144954382671\\
-3.17163173039152	-0.8345736703227\\
-3.15493937296599	-0.786447997426901\\
-3.13241909538237	-0.732794970696447\\
-3.10242537324393	-0.672400795585811\\
-3.06270493050221	-0.603745367988815\\
-3.01015344216256	-0.524925208081585\\
-2.9404774856664	-0.433572977477883\\
-2.84774445136671	-0.326798395754649\\
-2.7238374474047	-0.201211234613841\\
-2.55793839316163	-0.0531560686324172\\
-2.33641919046957	0.120602410235937\\
-2.04400323109662	0.321404499629473\\
-1.66760176663596	0.546225264813016\\
-1.20384676724017	0.784847999392526\\
-0.66815026467459	1.01902764897822\\
-0.0976764318487073	1.22667643555794\\
0.459022871868458	1.3904123014413\\
0.960698937316442	1.50414893360095\\
1.38575196981399	1.57262394183115\\
1.73160305051423	1.60606634792779\\
2.00696136401616	1.61515115761543\\
};
\addplot [color=mycolor1, forget plot]
  table[row sep=crcr]{%
2.06860660762213	1.6982564075881\\
2.27083390287344	1.69208158484801\\
2.43181623095735	1.67689744800409\\
2.56093236612886	1.65638138520399\\
2.66550083508625	1.63287461855828\\
2.75109090401043	1.60782966645149\\
2.82189989396105	1.58212559838171\\
2.88108778101375	1.55627688899532\\
2.93104242025631	1.53056767567032\\
2.9735794665914	1.50513681595202\\
3.01008960653438	1.48003139638351\\
3.04164595097124	1.45524024098408\\
3.06908218678636	1.43071476405002\\
3.09304952831468	1.4063817792773\\
3.11405833439906	1.3821511459237\\
3.13250858656795	1.35792004598616\\
3.14871219749462	1.3335750035248\\
3.16290923474801	1.30899232412895\\
3.17527950851923	1.28403735238804\\
3.18595050823006	1.25856275768579\\
3.19500232471316	1.23240592553061\\
3.20246991898054	1.20538542817751\\
3.20834286164587	1.17729645745394\\
3.21256243981641	1.14790501164691\\
3.21501578237673	1.11694052610887\\
3.21552635813266	1.08408651333649\\
3.21383981384117	1.04896862160267\\
3.20960358553135	1.01113932017236\\
3.20233795866104	0.970058163217434\\
3.19139515971534	0.925066269822954\\
3.17590148070908	0.875353300549936\\
3.15467517274135	0.819914882186446\\
3.12610969095244	0.757498330656529\\
3.08800775922697	0.686535139968228\\
3.03734714023266	0.605061175825374\\
2.96995616008087	0.510632303323317\\
2.8800827372595	0.400259326796594\\
2.75987252483657	0.270420895487558\\
2.59886936571566	0.117280023737274\\
2.38388581270329	-0.0626631847495668\\
2.10003245751582	-0.271146178427355\\
1.73420453571025	-0.505713869055155\\
1.28204877937458	-0.756817313998955\\
0.756667781044695	-1.00661495204683\\
0.1922733449297	-1.23255116919391\\
-0.364437061939789	-1.41564239051156\\
-0.871800672737549	-1.54769918540689\\
-1.30616008159199	-1.63187423596462\\
-1.66268725551021	-1.67779647802568\\
-1.94852783530275	-1.69631403914339\\
-2.17554581779588	-1.69659209740831\\
-2.35584395382825	-1.6853483502887\\
-2.49985475124989	-1.66713418663842\\
-2.61590047339757	-1.64489452404878\\
-2.71037792764455	-1.62047891864067\\
-2.78812467574338	-1.59502118880521\\
-2.85278177135636	-1.56919735303032\\
-2.90709431663539	-1.54339350921435\\
-2.95314268875547	-1.51781282664167\\
-2.99251442364579	-1.49254305174153\\
-3.02643004080702	-1.46759887639381\\
-3.0558346556689	-1.44294839981031\\
-3.08146467381201	-1.41852950986356\\
-3.10389645791012	-1.39425983034538\\
-3.12358193679098	-1.37004250838304\\
-3.14087468969792	-1.34576925536863\\
-3.15604899636746	-1.32132151185452\\
-3.16931359396485	-1.2965702596509\\
-3.18082134004983	-1.2713747764242\\
-3.19067558088456	-1.24558047154152\\
-3.19893371714453	-1.21901582613657\\
-3.20560820661074	-1.19148836502251\\
-3.21066501481064	-1.16277949856979\\
-3.21401929166534	-1.13263797735477\\
-3.21552778542046	-1.10077159067195\\
-3.21497716899955	-1.06683660114395\\
-3.21206700150163	-1.03042423025091\\
-3.20638541338284	-0.991043282215236\\
-3.1973746947755	-0.948097708047462\\
-3.18428265137816	-0.900857571416836\\
-3.16609369567306	-0.848421520785672\\
-3.14143095431247	-0.789668619176481\\
-3.10841702795366	-0.723197550674446\\
-3.06447656286716	-0.647252563797837\\
-3.00605961778449	-0.559639709938102\\
-2.9282649168792	-0.457647549503845\\
-2.82435759990059	-0.338010483311221\\
-2.68523395877069	-0.197001944847137\\
-2.49904103887176	-0.0308321456585343\\
-2.25149282157391	0.163357481220727\\
-1.92794704359109	0.385569683479203\\
-1.5185982643987	0.630131842732555\\
-1.02683517234253	0.883261618608591\\
-0.476530090381141	1.12394512252728\\
0.0899721270408058	1.33026131171234\\
0.626255381738577	1.48807579992018\\
1.09878526349904	1.59525133632222\\
1.49389074129826	1.65892050951114\\
1.81374487298738	1.68985191952004\\
2.06860660762213	1.6982564075881\\
};
\addplot [color=mycolor1, forget plot]
  table[row sep=crcr]{%
2.12479054205496	1.78038134069434\\
2.31291656161758	1.77463153108908\\
2.46360610366426	1.76041283002961\\
2.58531390643501	1.7410691972296\\
2.68459681276724	1.71874663610339\\
2.76644398208781	1.69479360268786\\
2.83462713558066	1.67003996425573\\
2.8919991948091	1.64498193509097\\
2.94072770888523	1.61990174728722\\
2.98247012200416	1.59494421232009\\
3.01850295537561	1.57016544374821\\
3.04981632875234	1.54556375037565\\
3.0771830307112	1.52109910765688\\
3.10120907214989	1.49670526622932\\
3.12237077455906	1.47229705683084\\
3.1410420104021	1.44777449975262\\
3.15751415986059	1.42302472183765\\
3.17201058532855	1.39792229432523\\
3.18469687103318	1.37232834788918\\
3.19568766556411	1.34608864468236\\
3.20505065072055	1.31903065742965\\
3.21280790299115	1.29095960015476\\
3.21893468318756	1.2616532579701\\
3.22335545718919	1.23085536183297\\
3.22593668726858	1.1982671369072\\
3.22647560444775	1.16353650877242\\
3.22468373243725	1.12624426815375\\
3.22016332006521	1.08588625994957\\
3.2123739627508	1.04185036683857\\
3.20058542991375	0.993386703785246\\
3.18381089780502	0.939569061922303\\
3.16071221931497	0.87924535425997\\
3.12946538640294	0.810974926595604\\
3.08757007304783	0.732951832674429\\
3.03158313835534	0.642917163875137\\
2.95675589737613	0.538073721361076\\
2.85656905904271	0.415039483078827\\
2.72221216084583	0.26992395087485\\
2.54219661269735	0.0986971519466855\\
2.30259833799656	-0.101860489881717\\
1.988914542344	-0.332282706047911\\
1.590833038462	-0.587588465309065\\
1.11013273152794	-0.854634144926016\\
0.568083887970402	-1.11247084944338\\
0.00463629186577767	-1.33814162830174\\
-0.53445543917634	-1.51552597345371\\
-1.01432355133255	-1.6404783228451\\
-1.41910545379632	-1.71894465365314\\
-1.74911013215975	-1.76145651305498\\
-2.01349136861769	-1.7785815895257\\
-2.22414161555226	-1.778834444578\\
-2.39236488665411	-1.76833812291586\\
-2.52763190541574	-1.75122476343541\\
-2.637414179359	-1.73018111639456\\
-2.72743941803059	-1.70691245766884\\
-2.80204676583525	-1.68247968465132\\
-2.86451532605169	-1.65752740762065\\
-2.91733005804378	-1.63243301732164\\
-2.96238471945608	-1.6074024926176\\
-3.00113255564112	-1.58253150520725\\
-3.03469685032081	-1.55784423610873\\
-3.06395171936462	-1.53331792796646\\
-3.08958118440682	-1.50889827723035\\
-3.11212246190421	-1.48450889048134\\
-3.13199774897241	-1.46005683471198\\
-3.14953755486476	-1.43543555342924\\
-3.16499773010165	-1.41052593585689\\
-3.17857169594814	-1.38519601128781\\
-3.19039890215735	-1.35929952905157\\
-3.20057018425654	-1.33267353477869\\
-3.20913041012869	-1.30513493834513\\
-3.21607856529649	-1.27647596930937\\
-3.22136519811773	-1.24645831777657\\
-3.22488690195018	-1.21480565062894\\
-3.22647721994893	-1.18119406369038\\
-3.22589298028387	-1.14523986795464\\
-3.22279455206183	-1.10648390022672\\
-3.21671778036694	-1.06437128407666\\
-3.20703430696839	-1.01822524069456\\
-3.19289546606382	-0.967213175566549\\
-3.17315277669444	-0.910302912228278\\
-3.14624504277518	-0.846206799830856\\
-3.11003815033464	-0.773311962456872\\
-3.06159927613152	-0.689597282062443\\
-2.99688444241334	-0.592544204307182\\
-2.91032342357126	-0.479064016328777\\
-2.79431509370997	-0.345497907382085\\
-2.63873561239191	-0.187811386672545\\
-2.43077687368962	-0.0022104490946153\\
-2.15584030237021	0.213483429925804\\
-1.80070129893057	0.457436150053204\\
-1.35998856707431	0.720810101212633\\
-0.844619331415884	0.986195276047295\\
-0.286098770515454	1.23058945444942\\
0.270545939309583	1.43341860165406\\
0.783237520117522	1.58436045908561\\
1.22637739195548	1.68490620595368\\
1.5930049503019	1.74399940268459\\
1.88877271927099	1.77260242882456\\
2.12479054205496	1.78038134069434\\
};
\addplot [color=mycolor1, forget plot]
  table[row sep=crcr]{%
2.1757455399383	1.86092947588109\\
2.35081968343383	1.85557361347775\\
2.4918814069607	1.84225863208274\\
2.60655824336578	1.8240283222734\\
2.70073573799555	1.80285014400824\\
2.77889236357058	1.77997422491894\\
2.84442307195399	1.75618106444592\\
2.89990649664374	1.73194580475101\\
2.94731124659916	1.70754511313093\\
2.98815020451916	1.68312612891192\\
3.02359428704186	1.6587507692563\\
3.0545558807579	1.63442411379075\\
3.08175001423489	1.61011247806171\\
3.10573928683399	1.58575475408463\\
3.1269669296582	1.56126929111975\\
3.14578113135792	1.5365577545954\\
3.1624528498072	1.51150686354687\\
3.17718866730384	1.48598855560677\\
3.19013976072305	1.45985889229939\\
3.20140769291341	1.43295585051682\\
3.21104744439648	1.40509601867434\\
3.21906786098303	1.37607010761512\\
3.22542946417315	1.3456370815782\\
3.23003932927788	1.31351660071788\\
3.23274245037814	1.27937933174952\\
3.23330864339078	1.2428345148618\\
3.23141353732528	1.20341395994266\\
3.22661149742104	1.16055137087074\\
3.21829731089289	1.11355555601054\\
3.20565200704221	1.06157568726971\\
3.18756610020821	1.00355637933419\\
3.16253064865351	0.938180154314155\\
3.12848273803986	0.86379528743524\\
3.08258772403524	0.778329148142527\\
3.02093765895846	0.679193262511462\\
2.93814948064986	0.563201169264176\\
2.82687253540112	0.426552585933946\\
2.6772961224536	0.265001045394384\\
2.4769450478802	0.0744276433387469\\
2.21143140255826	-0.147838875464467\\
1.86731172809882	-0.400655940343124\\
1.43813288752117	-0.675973742953314\\
0.932668968336354	-0.956875982837009\\
0.379868962039353	-1.21994253881364\\
-0.176755743005223	-1.44298613217454\\
-0.69465073899282	-1.61347070951245\\
-1.1462983201646	-1.73111613176759\\
-1.52265813963826	-1.80408987649808\\
-1.82793983029911	-1.84342025116485\\
-2.07253666415207	-1.85926110979957\\
-2.26808650642126	-1.85949111268943\\
-2.42508096947947	-1.84969048308533\\
-2.55211304014538	-1.83361451173693\\
-2.65590054705101	-1.81371615279375\\
-2.74158252091892	-1.79156687627748\\
-2.81305835960616	-1.7681569245645\\
-2.87328539301801	-1.7440977085159\\
-2.92451497364393	-1.71975452013379\\
-2.96847135624044	-1.69533243440478\\
-3.0064843306294	-1.67093159810833\\
-3.03958667374776	-1.64658270937511\\
-3.06858558161	-1.62226969316863\\
-3.09411507998217	-1.59794405612958\\
-3.11667455875612	-1.57353377393765\\
-3.13665713747485	-1.54894852044796\\
-3.15437050281795	-1.52408237866027\\
-3.17005208114538	-1.49881474013717\\
-3.18387984234486	-1.47300981269993\\
-3.19597961166421	-1.44651495919136\\
-3.20642944476207	-1.41915794600328\\
-3.21526135973226	-1.39074306432466\\
-3.22246048722934	-1.36104598217436\\
-3.2279614680494	-1.32980707750475\\
-3.23164166782781	-1.29672287980269\\
-3.23331045712067	-1.26143509754435\\
-3.23269337812526	-1.22351651876016\\
-3.22940942610948	-1.18245282867667\\
-3.22293882882447	-1.1376190811391\\
-3.21257749123535	-1.08824918889927\\
-3.19737252597229	-1.03339639195026\\
-3.17603082171242	-0.971882330094916\\
-3.14678925483359	-0.902232374478002\\
-3.10723101369695	-0.8225959581938\\
-3.054028521735	-0.730654356005426\\
-2.98259295602703	-0.62352803360588\\
-2.88662269817132	-0.497717904034039\\
-2.75759083133511	-0.349160982008038\\
-2.58434167608116	-0.173565222105404\\
-2.35324979693792	0.0326908767820533\\
-2.04985818817793	0.270734018763883\\
-1.66325731093179	0.536350555225738\\
-1.19359855779719	0.817107480987676\\
-0.659653834827181	1.09216856819398\\
-0.0991467107972801	1.3375460775094\\
0.442715382529131	1.53508043637685\\
0.929717232907285	1.67851643798467\\
1.34384825322769	1.77250800435862\\
1.68360721994237	1.82727950637311\\
1.95708707666518	1.85372676041588\\
2.1757455399383	1.86092947588109\\
};
\addplot [color=mycolor1, forget plot]
  table[row sep=crcr]{%
2.22159676622086	1.93930856982817\\
2.38456348915545	1.93431869239652\\
2.51659941474423	1.92185152071188\\
2.62459243618175	1.90468011902611\\
2.71383537803312	1.88460851340267\\
2.7883560991336	1.86279418799137\\
2.85121539348207	1.83996875348906\\
2.90474663628562	1.81658429638293\\
2.95073880373225	1.79290902967154\\
2.99057285119828	1.76908942219381\\
3.02532224916302	1.7451904488079\\
3.05582684850611	1.72122159542384\\
3.08274717489659	1.69715354787979\\
3.10660441249865	1.67292872537141\\
3.127809887305	1.64846767764792\\
3.14668677481328	1.62367262888872\\
3.16348596110748	1.59842897211503\\
3.17839740498815	1.57260520022359\\
3.19155791838624	1.54605154151156\\
3.20305595341838	1.51859740878168\\
3.21293371855797	1.49004764492252\\
3.22118671215024	1.4601774351994\\
3.227760530906	1.42872564251366\\
3.23254455564754	1.39538619351755\\
3.23536180344011	1.35979698777029\\
3.23595382175374	1.32152560535341\\
3.23395892799329	1.28005083611285\\
3.22888128420597	1.23473873318586\\
3.22004712651428	1.18481150089176\\
3.2065427863991	1.12930708628236\\
3.18712675833704	1.06702695026784\\
3.1601048199004	0.996469414928267\\
3.12315314447153	0.915746872100953\\
3.07307027181208	0.822488485651201\\
3.0054376929282	0.713739031638763\\
2.91417909926155	0.585885607914834\\
2.7910507029328	0.434688447808658\\
2.62521279462519	0.255575061560006\\
2.4032948862381	0.044481007647173\\
2.11081059009323	-0.200384751809641\\
1.73616153777326	-0.475677649798832\\
1.27780703965718	-0.769790461836722\\
0.752028056882593	-1.06208929013006\\
0.194399578661223	-1.32756797160026\\
-0.350316856720065	-1.54593554218786\\
-0.844475622829684	-1.70866935710458\\
-1.26787394401055	-1.81898888345277\\
-1.61719807411477	-1.88673288114333\\
-1.89950611666071	-1.92310517359197\\
-2.12585585919881	-1.93776143897764\\
-2.30744776846662	-1.937970749563\\
-2.453976108978	-1.92881915090876\\
-2.57323699998103	-1.91372272803877\\
-2.67128003518321	-1.89492236072458\\
-2.75272515508203	-1.87386549694145\\
-2.82108293983526	-1.85147437242084\\
-2.87902424937034	-1.82832618816331\\
-2.92859056026966	-1.80477156292672\\
-2.97135227022156	-1.78101165755016\\
-3.00852588244833	-1.75714819470257\\
-3.04106019066816	-1.73321582747918\\
-3.06969959513531	-1.70920299863323\\
-3.09503068404653	-1.68506523966372\\
-3.11751656635825	-1.66073343720329\\
-3.13752218229105	-1.63611867783153\\
-3.1553328871236	-1.61111468952351\\
-3.17116792414091	-1.58559850903179\\
-3.1851899036903	-1.55942974263533\\
-3.19751103107226	-1.53244860321826\\
-3.20819653289102	-1.5044727669215\\
-3.21726548504536	-1.47529297523436\\
-3.22468901655486	-1.44466719677413\\
-3.2303856241741	-1.41231304342215\\
-3.23421305309102	-1.37789799510356\\
-3.235955841838	-1.34102681304299\\
-3.23530714524298	-1.30122529861235\\
-3.2318427684834	-1.25791926980496\\
-3.22498437026108	-1.21040727021301\\
-3.2139473887235	-1.15782510271635\\
-3.1976672355856	-1.09909984365504\\
-3.17469450283523	-1.03289070644772\\
-3.14304623510161	-0.957514408594192\\
-3.09999606227808	-0.870854547522102\\
-3.04178279133616	-0.770260111763278\\
-2.96322004042956	-0.652452227928976\\
-2.85721209015153	-0.513489278488834\\
-2.71425395122918	-0.348902459295057\\
-2.52217422497574	-0.15421940103064\\
-2.26673597004074	0.0737793821462952\\
-1.93418981696836	0.334730225925443\\
-1.51691609518709	0.621483026274706\\
-1.02151355087482	0.917723764851995\\
-0.474421557665009	1.19967161875598\\
0.0822682578220813	1.4434858734613\\
0.60542847324188	1.63428244591121\\
1.06555720769938	1.76985019514753\\
1.45151187150373	1.85746785386418\\
1.76607348285571	1.908183370787\\
2.0189513713455	1.93263715122637\\
2.22159676622086	1.93930856982817\\
};
\addplot [color=mycolor1, forget plot]
  table[row sep=crcr]{%
2.26242433708805	2.01495975701713\\
2.41414617789258	2.01031029942162\\
2.53771272321917	1.99863920907641\\
2.6393508791053	1.9824750944135\\
2.72382923025413	1.96347238228063\\
2.79477724470694	1.94270157923801\\
2.8549589261095	1.92084645526483\\
2.90648822956053	1.89833479687986\\
2.95099201579987	1.8754241778549\\
2.98973104374063	1.85225799090357\\
3.02368912523749	1.82890199066744\\
3.05363870720377	1.80536806266584\\
3.0801891695655	1.78162956416864\\
3.10382246092602	1.75763103475301\\
3.12491940707309	1.73329407085491\\
3.14377907181281	1.70852050670912\\
3.16063284991186	1.68319361541998\\
3.1756544582572	1.65717775520509\\
3.18896660690415	1.63031668309961\\
3.20064483218905	1.60243060602038\\
3.21071872453098	1.57331191267627\\
3.2191705545088	1.54271941140635\\
3.22593106467386	1.51037077388933\\
3.23087192170737	1.47593273907774\\
3.23379397786403	1.43900845148347\\
3.23441002415314	1.39912107769105\\
3.23232006341119	1.35569254873925\\
3.22697619457315	1.30801590090295\\
3.21763284831761	1.25521923424936\\
3.20327617608307	1.19621881822066\\
3.18252367432787	1.12965848973131\\
3.15348149758124	1.05383259621917\\
3.11354260866627	0.966591258648163\\
3.05910531395786	0.865231748365437\\
2.98519334228269	0.746392684964905\\
2.88497752433065	0.605997106085439\\
2.74926341418225	0.43935036706806\\
2.56617333589524	0.241604392998829\\
2.32158625195492	0.00893748363705662\\
2.0013801507348	-0.259164636303382\\
1.59667626378294	-0.556598110954975\\
1.11180299427944	-0.867816262874491\\
0.570628563809619	-1.16878568287357\\
0.0138354865941234	-1.43397669740479\\
-0.51477784008836	-1.64597382440369\\
-0.983585324057899	-1.80041115102751\\
-1.37923667624514	-1.90352622692107\\
-1.70304879787909	-1.96633146823376\\
-1.96406395146819	-1.9999609332631\\
-2.17358366620682	-2.01352459856801\\
-2.3422597299364	-2.01371519107483\\
-2.47902161586027	-2.00516977239792\\
-2.59094452144229	-1.99099881414488\\
-2.68348498162809	-1.97325066716072\\
-2.76080429913143	-1.95325798961672\\
-2.82606866053358	-1.93187800249144\\
-2.88169362637296	-1.90965337745034\\
-2.92953211223144	-1.88691822322425\\
-2.97101501167211	-1.86386742935304\\
-3.00725510852754	-1.84060192483462\\
-3.03912354064474	-1.81715816586222\\
-3.06730606704634	-1.79352725982213\\
-3.09234454601841	-1.76966721413769\\
-3.11466755692612	-1.7455105517505\\
-3.13461298630067	-1.72096872693834\\
-3.15244458048663	-1.6959342476818\\
-3.16836386841219	-1.67028106039175\\
-3.18251841524694	-1.64386351247528\\
-3.19500703061484	-1.61651403425936\\
-3.20588228444575	-1.58803954494798\\
-3.21515044773085	-1.55821646686304\\
-3.22276874634791	-1.52678411218415\\
-3.22863956562872	-1.49343607275455\\
-3.23260093919311	-1.45780908245487\\
-3.23441225649563	-1.41946861840475\\
-3.23373357302114	-1.37789024595265\\
-3.23009612544867	-1.33243537787069\\
-3.22286052918501	-1.28231970232306\\
-3.21115751492079	-1.22657205308672\\
-3.19380375532933	-1.1639810301653\\
-3.16918216277058	-1.09302646845147\\
-3.13507199916185	-1.01179349857178\\
-3.08840988556377	-0.917869845171345\\
-3.02496091257697	-0.808235214543548\\
-2.93888704748241	-0.679171346043859\\
-2.82223666826812	-0.526263839427488\\
-2.66448461949251	-0.344648301373574\\
-2.45249402739633	-0.129781031181872\\
-2.17169849900763	0.120868484719446\\
-1.80972935627272	0.404948643356206\\
-1.3632561340928	0.711840294406582\\
-0.84598563937002	1.02126035664147\\
-0.291314470632887	1.30722853904193\\
0.256386012127547	1.54720538202523\\
0.757911370961256	1.73018016716924\\
1.19073558268672	1.85774057613982\\
1.54966147374486	1.93923803701874\\
1.84070705774828	1.98616599240323\\
2.07456070878453	2.00877850877765\\
2.26242433708805	2.01495975701713\\
};
\addplot [color=mycolor1, forget plot]
  table[row sep=crcr]{%
2.29830749724921	2.08738533347698\\
2.4395811740186	2.08305263854005\\
2.55520061617609	2.07212901265497\\
2.65080211567494	2.05692218109676\\
2.73068984585204	2.03894971670617\\
2.79814044812796	2.01920078004988\\
2.85565401536398	1.99831282384068\\
2.90514813770187	1.97668873039448\\
2.94810347329143	1.9545738782152\\
2.98567147129506	1.9321067243545\\
3.01875369636157	1.90935197166206\\
3.04806022379772	1.88632225308333\\
3.07415270212239	1.86299217817905\\
3.09747616469615	1.83930722377463\\
3.11838252328554	1.81518906403112\\
3.13714782939007	1.79053835531842\\
3.15398476936351	1.7652356059088\\
3.16905140194706	1.73914049674307\\
3.1824568002718	1.71208982970108\\
3.1942639845052	1.68389413204003\\
3.20449029378011	1.65433281748077\\
3.21310511863907	1.62314767854759\\
3.22002467010205	1.59003434623226\\
3.22510316709981	1.55463118714243\\
3.22811944042731	1.51650489916663\\
3.22875742400764	1.47513179685374\\
3.22657825605299	1.42987342926179\\
3.22098063445581	1.37994473301228\\
3.2111445119643	1.32437239859616\\
3.19595098563718	1.2619405822732\\
3.17386813346458	1.19112073535739\\
3.14278851341989	1.10998268755293\\
3.09979955470441	1.01608649672324\\
3.04086530125618	0.906361827493813\\
2.96040347532639	0.776999705219256\\
2.85077245154363	0.623421613075426\\
2.70177576333412	0.440470155704622\\
2.50051308750921	0.223096331657488\\
2.2323179458462	-0.0320410681959813\\
1.88399246213748	-0.323721965763938\\
1.45030338557161	-0.642524096561426\\
0.942220642271537	-0.968737361939567\\
0.390812497285217	-1.2755120554387\\
-0.159963629894488	-1.53794078881014\\
-0.669198713934636	-1.74224053078579\\
-1.11183040302902	-1.88809740401201\\
-1.48062795069045	-1.98423320154642\\
-1.78052030750788	-2.04240516957082\\
-2.02184432018373	-2.0734973663325\\
-2.21584486452425	-2.08605362771345\\
-2.37256441865051	-2.08622730090106\\
-2.50021033879442	-2.0782481673088\\
-2.60520716935758	-2.0649511812546\\
-2.69248456073879	-2.04820986400352\\
-2.76579786844252	-2.02925083248949\\
-2.82800777561613	-2.00886956147172\\
-2.88130214750432	-1.98757446254011\\
-2.92736452028104	-1.96568194613657\\
-2.96749945605449	-1.94337886547424\\
-3.00272500583715	-1.9207634826045\\
-3.03384076001914	-1.8978723035551\\
-3.06147798141842	-1.87469756194609\\
-3.08613661192687	-1.85119844159636\\
-3.10821261761388	-1.82730802935857\\
-3.128018148681	-1.80293727330643\\
-3.14579626615936	-1.77797674994651\\
-3.16173145589966	-1.75229672683094\\
-3.17595675407387	-1.72574578494571\\
-3.18855800163099	-1.69814809973293\\
-3.19957549213643	-1.6692993440335\\
-3.2090030483687	-1.63896105116648\\
-3.21678433039166	-1.60685314597231\\
-3.22280591241952	-1.57264420130932\\
-3.2268863325603	-1.53593879205867\\
-3.22875987206479	-1.4962610814842\\
-3.22805319432644	-1.45303346778393\\
-3.22425207670579	-1.40554872534058\\
-3.21665417134161	-1.35293358966853\\
-3.20430186210206	-1.29410118586178\\
-3.18588664028215	-1.22768920689045\\
-3.15961284078826	-1.15198064948813\\
-3.12300419233677	-1.06480505265109\\
-3.07263255395004	-0.963422490360464\\
-3.00374830420568	-0.844404217321268\\
-2.90980681553143	-0.703551162898204\\
-2.78194086289829	-0.535948802319476\\
-2.60857622479041	-0.336361891275331\\
-2.37570095198536	-0.100320780841988\\
-2.06878304888541	0.173669299639037\\
-1.67760012151999	0.480726745207704\\
-1.20406911546428	0.80630008205186\\
-0.669318227607618	1.12628547167269\\
-0.112520488089901	1.41346196908147\\
0.421789369974737	1.64766180912662\\
0.899662855971952	1.82206522437116\\
1.3053487755271	1.94165717647392\\
1.63860176170692	2.01733702594874\\
1.90778748090552	2.06074259319959\\
2.12409187880228	2.08165636290458\\
2.29830749724921	2.08738533347698\\
};
\addplot [color=mycolor1, forget plot]
  table[row sep=crcr]{%
2.32935350426214	2.15617173173143\\
2.46092075586192	2.15213372489249\\
2.56908832940956	2.14191139927175\\
2.65896471538462	2.12761282504564\\
2.73444205483902	2.1106304941265\\
2.79848402521174	2.09187777804395\\
2.85335580032668	2.07194771005019\\
2.90079919596766	2.05121820007574\\
2.94216311250622	2.02992142129865\\
2.97849976370664	2.00818952500258\\
3.01063548002523	1.98608474349786\\
3.03922285172473	1.96361914362191\\
3.06477921334039	1.94076744429417\\
3.08771509049744	1.91747510467296\\
3.10835519914207	1.89366310208727\\
3.12695383187255	1.86923030050607\\
3.14370591329698	1.84405396201361\\
3.15875459586971	1.81798871118315\\
3.17219595232114	1.79086408294747\\
3.18408106369956	1.76248063963401\\
3.19441557281967	1.7326045112283\\
3.20315654396047	1.70096007769185\\
3.21020621229256	1.66722035772634\\
3.21540188668415	1.63099447862448\\
3.21850084250327	1.59181135878443\\
3.21915844446517	1.54909841790525\\
3.21689688452632	1.50215371941553\\
3.21106068009092	1.45010943213044\\
3.20075328108649	1.39188388794337\\
3.18474656782996	1.32611890190494\\
3.16135148736598	1.25109869757437\\
3.12823359553641	1.16464748759466\\
3.08215268206203	1.06400625588533\\
3.01860416618456	0.945699465708896\\
2.93135083717691	0.805427264160013\\
2.81187975223788	0.638072770612336\\
2.64894909567976	0.438017195115797\\
2.42867976698624	0.200113082811465\\
2.13613065485561	-0.0782098645709827\\
1.75967758187403	-0.393487132995316\\
1.29866158436661	-0.732453427227594\\
0.771202537901096	-1.07121256327827\\
0.214728980558727	-1.38091874404249\\
-0.325494961838313	-1.6384149650196\\
-0.812965102505627	-1.83404646195004\\
-1.22924624190243	-1.97125460517688\\
-1.57235707714977	-2.06070931296474\\
-1.84993959636118	-2.11455784134355\\
-2.07308909793711	-2.14330769833361\\
-2.25278582519434	-2.1549355994704\\
-2.39843774449618	-2.15509398815662\\
-2.51757797964365	-2.14764365112562\\
-2.61604479854629	-2.13517105857273\\
-2.6982992584146	-2.11939098848893\\
-2.76773681442649	-2.1014323062276\\
-2.82694677743785	-2.08203220493066\\
-2.87791380142478	-2.06166558681209\\
-2.92216930744609	-2.04063052932208\\
-2.96090359475513	-2.01910460671347\\
-2.99504838736969	-1.9971819926349\\
-3.02533758060229	-1.97489786432459\\
-3.05235202489591	-1.95224434892787\\
-3.07655261064923	-1.92918075751268\\
-3.09830472024621	-1.90563987805827\\
-3.11789623002166	-1.88153146092839\\
-3.13555059924298	-1.85674360684358\\
-3.15143610824056	-1.83114247836945\\
-3.16567195008156	-1.80457054930332\\
-3.1783315980804	-1.77684344724499\\
-3.1894436318143	-1.74774530867145\\
-3.19898997864918	-1.71702243439424\\
-3.20690128843532	-1.68437489038377\\
-3.21304887521986	-1.64944552896162\\
-3.21723229289842	-1.61180569135058\\
-3.21916110914594	-1.57093657538571\\
-3.21842872934844	-1.52620489105142\\
-3.21447509338645	-1.47683096313165\\
-3.20653357442546	-1.4218468714199\\
-3.19355525520003	-1.36004158880333\\
-3.17410072549746	-1.28988955477375\\
-3.14618550490828	-1.20945917541377\\
-3.10706046072684	-1.11629952448398\\
-3.05290489374579	-1.00730967183723\\
-2.978412885197	-0.878611223092022\\
-2.87627789570386	-0.72548180564268\\
-2.73666064701904	-0.542483349892518\\
-2.54692508885849	-0.324051637474437\\
-2.29232017876867	-0.0659769457238991\\
-1.9587960087572	0.231793697508506\\
-1.53911030580676	0.561283420912842\\
-1.04127224491097	0.903661418785349\\
-0.493744333777135	1.23140511216232\\
0.0600689388895753	1.51714943946215\\
0.577425408082559	1.74399826308549\\
1.03044074659414	1.90937750121675\\
1.40961161396351	2.02117636793413\\
1.71867229167716	2.09137080250741\\
1.96760462743302	2.13151173759376\\
2.16773707616045	2.15085998477536\\
2.32935350426214	2.15617173173143\\
};
\addplot [color=mycolor1, forget plot]
  table[row sep=crcr]{%
2.35571328277463	2.22100511820195\\
2.47826782286047	2.21724110275804\\
2.57945563798566	2.2076760066786\\
2.66391338264297	2.19423739652515\\
2.73516709681387	2.17820355908947\\
2.79590243487287	2.16041750763504\\
2.84817524353824	2.1414300272445\\
2.89356994912398	2.12159441970805\\
2.93331677967345	2.10112910266619\\
2.96837799604573	2.08015897733583\\
2.99951128269362	2.05874275651048\\
3.02731643733883	2.03689093924234\\
3.05226984512698	2.01457747210173\\
3.07474996249443	1.99174706294535\\
3.09505610683312	1.96831941038495\\
3.11342217070063	1.94419114633847\\
3.13002638320828	1.91923597242133\\
3.14499787010008	1.89330324619182\\
3.1584204744478	1.86621510225032\\
3.17033405755072	1.83776204934554\\
3.18073327556166	1.8076968479169\\
3.18956359405484	1.77572632588045\\
3.19671403015796	1.74150061739947\\
3.20200576274209	1.70459909187564\\
3.20517527509796	1.66451195769695\\
3.20585002004533	1.62061615374963\\
3.20351362162625	1.57214365762075\\
3.19745620422277	1.51813972897166\\
3.1867033712263	1.45740789312791\\
3.16991440413369	1.38843778139177\\
3.14523622615357	1.30931167074949\\
3.11009471123801	1.2175867111612\\
3.06090031723225	1.11015476668775\\
2.9926453648678	0.983095758531635\\
2.89838838626637	0.831573980390453\\
2.76868761057062	0.649898587637061\\
2.59122233945366	0.432001663954325\\
2.35121103645622	0.172770931598334\\
2.03377775386478	-0.129248435576604\\
1.62959742771365	-0.46779854288743\\
1.14346100946033	-0.825320879963705\\
0.600821886747986	-1.17394011308631\\
0.0442409056935615	-1.4838164622843\\
-0.48163880163311	-1.73456167138362\\
-0.945768934604242	-1.92088020146989\\
-1.33603704925895	-2.04954181981907\\
-1.65480624610948	-2.13266138371325\\
-1.91166875393097	-2.18249310676955\\
-2.11807157549804	-2.2090843405378\\
-2.28459218053074	-2.21985724993794\\
-2.42000314217803	-2.220001836486\\
-2.53121317335989	-2.21304488666874\\
-2.62353282613246	-2.2013486995232\\
-2.70100594205337	-2.18648395128973\\
-2.76670839564354	-2.16948958718696\\
-2.82298812770595	-2.15104809534126\\
-2.87164830808901	-2.13160199457462\\
-2.91408376440492	-2.11143085198241\\
-2.95138155578939	-2.09070216769097\\
-2.98439489737335	-2.06950500443297\\
-3.01379754605468	-2.04787217232614\\
-3.0401239091731	-2.02579474465893\\
-3.06379868525868	-2.00323135043708\\
-3.08515876112665	-1.98011382155867\\
-3.10446929599717	-1.95635020164888\\
-3.12193534452632	-1.93182574111754\\
-3.13770994317351	-1.90640223807004\\
-3.15189925874524	-1.87991589072489\\
-3.16456513585618	-1.85217367237726\\
-3.17572515028752	-1.82294810191056\\
-3.18535005003472	-1.79197014323174\\
-3.19335821697194	-1.75891980928172\\
-3.19960647651815	-1.7234138533874\\
-3.20387617749746	-1.68498968341194\\
-3.20585289974103	-1.64308431012539\\
-3.20509733712814	-1.59700671612584\\
-3.20100372563178	-1.54590148473199\\
-3.19274046786317	-1.48870085916713\\
-3.17916512589608	-1.4240616742884\\
-3.15870247919262	-1.35028304642604\\
-3.12916978909147	-1.26520095598012\\
-3.08752833431522	-1.16605847020069\\
-3.02953722802248	-1.04935888480526\\
-2.9492923192467	-0.910731054901148\\
-2.83866985696159	-0.744886013906675\\
-2.68680635565555	-0.545843632021727\\
-2.48001009197639	-0.307772295691794\\
-2.20297704173767	-0.0269519369913502\\
-1.84266886104903	0.294767101014322\\
-1.39569223068177	0.645751374552216\\
-0.876809173496224	1.0027034275221\\
-0.321315333824451	1.33532954261279\\
0.224920657710854	1.61726471950061\\
0.722602501490466	1.83555657428617\\
1.15023955592394	1.99171058627205\\
1.50385186179304	2.09599160249212\\
1.79026056897879	2.16104741003003\\
2.02047904511769	2.19817110296771\\
2.20572333459932	2.21607809894601\\
2.35571328277463	2.22100511820195\\
};
\addplot [color=mycolor1, forget plot]
  table[row sep=crcr]{%
2.37758606528632	2.28167835318404\\
2.49177779236238	2.27816889033819\\
2.58643654129522	2.26921889562557\\
2.66577691988729	2.25659271656946\\
2.73299918861372	2.24146446362388\\
2.79054164262947	2.22461204330309\\
2.84027340016404	2.20654634251511\\
2.88363789642987	2.18759675296512\\
2.92175850758457	2.16796776122089\\
2.95551605130682	2.14777643322451\\
2.98560570694247	2.12707723384355\\
3.01257893511641	2.10587836965161\\
3.03687443210908	2.08415237025142\\
3.05884100197166	2.06184266324354\\
3.07875438884017	2.03886726768229\\
3.09682950155217	2.01512031012179\\
3.11322901475573	1.99047177773649\\
3.12806899273829	1.96476571310981\\
3.14142191386205	1.93781689048177\\
3.1533172425677	1.90940586879712\\
3.16373947432695	1.87927217334283\\
3.17262333881433	1.8471051976523\\
3.17984555601306	1.81253222273927\\
3.18521215787343	1.77510270119484\\
3.18843985746846	1.73426762499693\\
3.18912918620102	1.68935235999834\\
3.18672600829574	1.63952075883262\\
3.18046639082213	1.58372764217373\\
3.16929742868451	1.52065589925095\\
3.15176323195933	1.4486336729205\\
3.12584069311508	1.36552687954528\\
3.08870415640836	1.26860400463144\\
3.03639360410928	1.1543768731755\\
2.96336381683274	1.01843987857661\\
2.8619195262561	0.855374897665057\\
2.72163445749676	0.658883198778094\\
2.52908689796336	0.422472560497084\\
2.26870515950629	0.141233696761046\\
1.92608821230028	-0.184774505377792\\
1.49499092845071	-0.545931274374462\\
0.986418384452357	-0.920048933956963\\
0.432982829304749	-1.27571827247748\\
-0.119136128434278	-1.58321574433414\\
-0.627655027613959	-1.82575970727758\\
-1.06757582968432	-2.00240478044851\\
-1.43255385347586	-2.12275069775533\\
-1.72842728644552	-2.19990845867801\\
-1.9661116709102	-2.24602123664282\\
-2.15710471968667	-2.27062600113633\\
-2.31149491765359	-2.28061196117877\\
-2.43743478465075	-2.28074408647642\\
-2.5412582089816	-2.27424702197866\\
-2.62780100444458	-2.26328076567257\\
-2.70073509484606	-2.24928521530659\\
-2.76285212574483	-2.23321675902357\\
-2.81628506334122	-2.21570678829857\\
-2.86267486836922	-2.19716683987995\\
-2.9032937035981	-2.17785815738066\\
-2.93913541648268	-2.15793774301906\\
-2.97098193864479	-2.13748886740034\\
-2.99945211478397	-2.11654123389789\\
-3.02503771560578	-2.09508417081845\\
-3.04813004952554	-2.07307503562399\\
-3.06903960224466	-2.05044423866413\\
-3.08801041732097	-2.02709778038385\\
-3.10523040918521	-2.00291784861494\\
-3.12083841282735	-1.97776177798815\\
-3.13492847566302	-1.95145948973794\\
-3.14755165113287	-1.92380937821806\\
-3.15871533073761	-1.89457246868706\\
-3.16837992399641	-1.86346452125125\\
-3.17645243491076	-1.83014558084612\\
-3.1827761531404	-1.79420625383446\\
-3.18711523016356	-1.755149705987\\
-3.18913227693457	-1.71236799799602\\
-3.18835620094771	-1.66511087430589\\
-3.18413615430636	-1.61244447464527\\
-3.17557549147682	-1.55319664750823\\
-3.16143678281318	-1.48588469655782\\
-3.14000494688445	-1.40862079042619\\
-3.10889042539018	-1.31899075571488\\
-3.06474890541543	-1.21390562553849\\
-3.00289199147913	-1.0894368677975\\
-2.91677516560724	-0.940675663175271\\
-2.79740245163396	-0.761722570024749\\
-2.63283920452629	-0.546042725412211\\
-2.40836566454116	-0.287620868514534\\
-2.10836455303723	0.0164973432651121\\
-1.72141396712113	0.362048522724962\\
-1.24883324577633	0.733216621124839\\
-0.712552708207109	1.10224434366042\\
-0.153813549914169	1.43692681283272\\
0.38088733213369	1.71300064169735\\
0.856965914982096	1.92187713206091\\
1.25926017399376	2.0688092073437\\
1.58849820036728	2.16591618255055\\
1.85380508806444	2.22618318092815\\
2.06677018214786	2.26052458407289\\
2.23831992589926	2.27710592776164\\
2.37758606528632	2.28167835318404\\
};
\addplot [color=mycolor1, forget plot]
  table[row sep=crcr]{%
2.39521560661132	2.33808947020594\\
2.50165301737442	2.33481631441034\\
2.59021226763556	2.32644121742552\\
2.66473014948301	2.31458089640988\\
2.72811656402173	2.30031450520744\\
2.78258937535318	2.28435986778607\\
2.8298509249221	2.26719041706212\\
2.87121826114054	2.24911257660912\\
2.90771855167311	2.23031702064168\\
2.94015893115056	2.21091268772948\\
2.96917775336511	2.19094933533025\\
2.9952823265977	2.17043239030596\\
3.01887677215071	2.14933252757017\\
3.04028259129508	2.12759154668238\\
3.05975376304627	2.10512554827704\\
3.07748764274719	2.08182603017976\\
3.0936325248088	2.05755925643596\\
3.1082924230745	2.03216405481763\\
3.12152937147492	2.0054480379043\\
3.13336332535661	1.97718209618812\\
3.14376952253335	1.94709285958934\\
3.15267291408491	1.91485264802381\\
3.15993896395509	1.88006621244632\\
3.16535969787814	1.84225328166619\\
3.16863329139493	1.8008255486039\\
3.16933462925266	1.75505621888867\\
3.16687300610201	1.7040395707407\\
3.16043127369357	1.646637119965\\
3.14887800857505	1.5814059914449\\
3.13064037732106	1.50650418960118\\
3.1035201403668	1.41956730718509\\
3.06442914594481	1.31755356243158\\
3.00901637102516	1.19656307143056\\
2.93116469893273	1.05166211781725\\
2.82237535363197	0.876802042322309\\
2.67118404936331	0.665044383789976\\
2.46305877837921	0.409511599850905\\
2.18178892819626	0.105702096678543\\
1.81392666502375	-0.244362060607632\\
1.35711711704518	-0.62712930014674\\
0.829177392341424	-1.01559675794281\\
0.269344016277399	-1.37549260102969\\
-0.274259788466607	-1.67834639642003\\
-0.763154680849728	-1.91159759274296\\
-1.17858291271912	-2.07844526629714\\
-1.51926750260519	-2.19079797487089\\
-1.7937312306146	-2.26237864675983\\
-2.013711434124	-2.3050575482378\\
-2.19053980187294	-2.32783626535857\\
-2.33376756044192	-2.33709825588292\\
-2.4509528573853	-2.33721912976606\\
-2.54790269211076	-2.3311502847474\\
-2.62902585113351	-2.32086907669468\\
-2.69766227901308	-2.30769673096066\\
-2.7563504132456	-2.2925139611881\\
-2.80703147496583	-2.2759046309315\\
-2.85120130171226	-2.25825088157889\\
-2.89002186142181	-2.2397960981004\\
-2.92440287539562	-2.22068665899547\\
-2.95506162620203	-2.20099964421522\\
-2.98256691656624	-2.18076116422536\\
-3.00737148513045	-2.1599583329753\\
-3.02983594952401	-2.13854684018862\\
-3.05024644954002	-2.11645537990111\\
-3.06882751475763	-2.09358772798619\\
-3.08575120843401	-2.06982294411827\\
-3.10114324655063	-2.0450139461929\\
-3.11508651472267	-2.01898452948588\\
-3.12762217267833	-1.99152475185492\\
-3.13874831762105	-1.96238445921168\\
-3.14841594642258	-1.9312645639131\\
-3.15652168143442	-1.89780549389601\\
-3.16289636663906	-1.86157198093439\\
-3.16728814587444	-1.82203302644447\\
-3.16933792492491	-1.77853544149122\\
-3.16854407998	-1.73026876914547\\
-3.16421174030815	-1.67621863279578\\
-3.15537971184138	-1.6151046176509\\
-3.14071483019245	-1.54529779542672\\
-3.11835896528835	-1.46471234062786\\
-3.0857080945164	-1.37066645943963\\
-3.03909709476125	-1.25971278195924\\
-2.97336311152426	-1.12745369940347\\
-2.88127896804591	-0.968395991395105\\
-2.7529218658074	-0.775985408972052\\
-2.57524500381626	-0.543126418552202\\
-2.33255204446239	-0.263728412551782\\
-2.00920815844371	0.0640732522834683\\
-1.59607758629773	0.433054732063467\\
-1.10000687731179	0.822759794085273\\
-0.550220947233858	1.20119336628135\\
0.00730350945160745	1.53525815237335\\
0.527202673577963	1.80377330441763\\
0.980457052624716	2.00268705038353\\
1.35787365899489	2.14055887331388\\
1.66406241824938	2.23087830678547\\
1.90978990059699	2.28670063179813\\
2.10687494963863	2.31848086352427\\
2.26583641719627	2.33384372724978\\
2.39521560661132	2.33808947020594\\
};
\addplot [color=mycolor1, forget plot]
  table[row sep=crcr]{%
2.40888077172283	2.3902330200096\\
2.50813233444798	2.38717908606175\\
2.59100005011747	2.37934066202057\\
2.66098213322652	2.3682008737186\\
2.72072924962287	2.35475236287878\\
2.77226248661235	2.33965762874679\\
2.81713501474985	2.32335511691346\\
2.85655047962942	2.30612938999659\\
2.8914493925071	2.28815765007762\\
2.92257224517012	2.26954064736369\\
2.95050578023133	2.25032319062776\\
2.97571704516746	2.23050763640493\\
2.99857852155886	2.21006254104203\\
3.01938665427382	2.18892788205039\\
3.03837541110589	2.16701774112359\\
3.05572600135215	2.14422099246346\\
3.07157351078336	2.12040029297962\\
3.0860109244689	2.09538948320704\\
3.09909077261541	2.06898934996484\\
3.110824418657	2.04096155145672\\
3.1211787858935	2.01102034331287\\
3.13007005924853	1.97882155055495\\
3.13735356516316	1.94394798350517\\
3.1428085744848	1.90589016811499\\
3.14611611703187	1.86402081910777\\
3.14682693374144	1.81756088702465\\
3.14431526271719	1.76553421586626\\
3.1377120291608	1.70670683383351\\
3.12580787897313	1.63950571972651\\
3.10691202114415	1.56191082000341\\
3.07864686425438	1.47131399451154\\
3.03765166922578	1.36434169898715\\
2.97916448488124	1.236649989726\\
2.89646197940779	1.08273284380471\\
2.78019192232968	0.895861341069844\\
2.61780086671964	0.668427872973931\\
2.39365140476653	0.393224344394331\\
2.09108675215881	0.066400864885246\\
1.69815533443762	-0.30756101600652\\
1.21720249138408	-0.710637308388417\\
0.673242697314047	-1.11100197657112\\
0.111271203340768	-1.47238612026827\\
-0.420354927699846	-1.76865799316422\\
-0.888053846144788	-1.99185475738594\\
-1.2791714796078	-2.14896926102479\\
-1.59673873350113	-2.253711661735\\
-1.85127279553356	-2.3200990957414\\
-2.05494113784792	-2.35961363973481\\
-2.21875836531109	-2.3807149952148\\
-2.35171732991723	-2.38931115041\\
-2.4608135960206	-2.38942186162935\\
-2.55137267379465	-2.38375139117903\\
-2.62741912650439	-2.37411210124081\\
-2.69199612519804	-2.36171752052997\\
-2.7474161315611	-2.34737908193473\\
-2.79544906409573	-2.33163659063348\\
-2.83746076631175	-2.31484448727144\\
-2.87451414789845	-2.29722896426605\\
-2.90744300093537	-2.27892588091402\\
-2.93690601309933	-2.260005954406\\
-2.96342644474103	-2.24049142405698\\
-2.98742137697517	-2.22036690515763\\
-3.00922329903649	-2.19958618681875\\
-3.02909598395469	-2.17807609740442\\
-3.04724601225948	-2.15573813949756\\
-3.06383087348644	-2.13244830482443\\
-3.0789642517126	-2.10805526648643\\
-3.09271884414577	-2.08237697620772\\
-3.10512683912712	-2.05519554276558\\
-3.11617796394417	-2.02625011386991\\
-3.12581477556484	-1.99522730843816\\
-3.13392457607716	-1.96174852900089\\
-3.1403269462786	-1.92535320050831\\
-3.14475534472522	-1.88547660169896\\
-3.14683042644459	-1.84142044089567\\
-3.14602156332116	-1.79231363703443\\
-3.14159130388004	-1.73705986251555\\
-3.13251492454502	-1.67426729172221\\
-3.11736346176195	-1.60215481634377\\
-3.09413337798508	-1.51842823891945\\
-3.05999944396679	-1.42012104981175\\
-3.01096128934895	-1.30340086206869\\
-2.94135499957542	-1.16336257543353\\
-2.8432282197083	-0.993879982512098\\
-2.70567713390113	-0.787699302400684\\
-2.51450836283949	-0.537165999597471\\
-2.2531264418093	-0.236249327948188\\
-1.90623194822527	0.115453118936052\\
-1.46769549096299	0.507185376642161\\
-0.950611710694782	0.913494425757623\\
-0.391314880114672	1.29859008381683\\
0.16092568827981	1.62959380211794\\
0.663450670397169	1.88921030518518\\
1.09326257240482	2.07787932923935\\
1.44658202591583	2.2069691762133\\
1.73111770281516	2.2909095372093\\
1.95873315187099	2.34261929249117\\
2.14121952233708	2.37204479129414\\
2.28861434227596	2.38628816417568\\
2.40888077172283	2.3902330200096\\
};
\addplot [color=mycolor1, forget plot]
  table[row sep=crcr]{%
2.41888318380138	2.43818635730555\\
2.51147824027556	2.43533570150822\\
2.58904001753741	2.42799778514815\\
2.654762915214	2.41753476294698\\
2.71106571560561	2.40486046826886\\
2.75979350888519	2.39058653892867\\
2.80236582063492	2.37511886450643\\
2.83988441701825	2.35872135292565\\
2.87321169884619	2.34155823262177\\
2.9030278718053	2.32372217986472\\
2.92987282283087	2.3052529845246\\
2.95417692552194	2.28614979942586\\
2.97628375721661	2.26637893920662\\
2.99646682350968	2.24587849060945\\
3.01494175254589	2.22456052875055\\
3.0318749635532	2.20231141395228\\
3.04738947390946	2.17899041324983\\
3.06156824349506	2.15442671091284\\
3.07445523077159	2.12841471565484\\
3.08605412360892	2.10070741684961\\
3.09632448182186	2.07100736819664\\
3.10517475647273	2.03895466388302\\
3.11245129298451	2.00411099441553\\
3.11792192439974	1.96593849533151\\
3.12125203444865	1.92377159147383\\
3.12196989283308	1.87677934295999\\
3.11941645171259	1.82391486427375\\
3.11267237480603	1.76384718481904\\
3.100451490343	1.69486951063386\\
3.08094472458089	1.61477657455123\\
3.05159173440928	1.52070370295161\\
3.008749919873	1.40892425665714\\
2.94722700922035	1.27461720696809\\
2.85965903539448	1.11165831545422\\
2.73578992327268	0.912586932136786\\
2.56192833078652	0.66909961063489\\
2.32135179263591	0.373729959642227\\
1.99719390871752	0.0235653536649638\\
1.57960139546763	-0.373916177171587\\
1.07639734702074	-0.79572889688355\\
0.519932676754131	-1.20541169193261\\
-0.0401808144590513	-1.56571184119036\\
-0.556982479305206	-1.85380500958858\\
-1.00251770771059	-2.06647459713139\\
-1.36985885057439	-2.21406294236225\\
-1.66558777908723	-2.31161292294382\\
-1.90163229697547	-2.37318111115254\\
-2.0902909759994	-2.40978353887551\\
-2.24216045712678	-2.4293446369676\\
-2.36567306671119	-2.43732844588558\\
-2.46729667711223	-2.4374299717228\\
-2.55191766208865	-2.43212985980174\\
-2.62321463845191	-2.42309129638358\\
-2.68396502687506	-2.41143003860427\\
-2.7362792241262	-2.39789414199754\\
-2.78177383166558	-2.3829826772989\\
-2.82169808008363	-2.36702412299338\\
-2.85702574275742	-2.35022828069243\\
-2.88852204894739	-2.33272076557404\\
-2.9167925847934	-2.3145659349788\\
-2.94231919346127	-2.29578204363779\\
-2.9654864259931	-2.27635107291352\\
-2.98660104464018	-2.25622481115633\\
-3.00590633144478	-2.23532819032211\\
-3.02359241755068	-2.213560499177\\
-3.03980345588516	-2.19079482401433\\
-3.05464216166413	-2.16687586668619\\
-3.06817200400236	-2.14161612451635\\
-3.08041711743815	-2.11479026317304\\
-3.09135978700224	-2.0861273514492\\
-3.10093511580077	-2.05530043606938\\
-3.10902217497754	-2.02191269248365\\
-3.11543051501338	-1.98547906615454\\
-3.11988031622032	-1.94540188226001\\
-3.1219735731995	-1.90093830482429\\
-3.12115239057052	-1.85115671672689\\
-3.11663849111376	-1.79487802364494\\
-3.10734508823608	-1.73059656028233\\
-3.09174796201318	-1.65637386358516\\
-3.06769657194691	-1.56969770035753\\
-3.03213858413043	-1.4673001803593\\
-2.98072466317665	-1.3449370790949\\
-2.90726347981168	-1.19715618329567\\
-2.80303457019989	-1.01714771007273\\
-2.65609915783797	-0.796913211476596\\
-2.45108995970766	-0.528249316603974\\
-2.17061792601485	-0.205349676514746\\
-1.80012956956224	0.170304936508782\\
-1.33725525923846	0.583846334044822\\
-0.801923235957124	1.0045985704694\\
-0.237080966282234	1.39362898194891\\
0.306283202348292	1.71941108888218\\
0.789517259102474	1.96912730676105\\
1.19575951981373	2.14748633998509\\
1.52597909755489	2.26815294435095\\
1.79027540467186	2.3461286629649\\
2.00117230186604	2.39404152151433\\
2.17024730915772	2.42130366179175\\
2.3070153558866	2.43451861795335\\
2.41888318380138	2.43818635730555\\
};
\addplot [color=mycolor1, forget plot]
  table[row sep=crcr]{%
2.42553419358814	2.48209316550803\\
2.51196378081259	2.47943096870429\\
2.58458213605233	2.47255952928049\\
2.64631060549358	2.46273135324056\\
2.69936012467214	2.45078846621064\\
2.74541804078164	2.43729579639707\\
2.78578391756268	2.42262902919096\\
2.82146785269512	2.4070326697568\\
2.85326176109995	2.39065857672067\\
2.88179127793504	2.37359159422191\\
2.9075537462819	2.35586655222134\\
2.93094614675948	2.33747938774235\\
2.95228567701671	2.31839416027534\\
2.97182487470824	2.29854709497804\\
2.98976259824536	2.27784836062498\\
3.00625176062514	2.25618199419038\\
3.02140439839821	2.23340416756501\\
3.03529440988456	2.20933981837323\\
3.04795808237633	2.18377751003194\\
3.0593923196453	2.15646222458848\\
3.06955025068456	2.1270856048844\\
3.07833361533312	2.09527292729243\\
3.08558093838671	2.06056577433141\\
3.09104995792431	2.02239895096344\\
3.09439197162826	1.98006960102621\\
3.09511456410917	1.93269567074746\\
3.09252736404776	1.87915976927787\\
3.08566273996497	1.81803305009449\\
3.07315925787811	1.7474720507549\\
3.05308983755072	1.66507989221115\\
3.02270871379862	1.56772318318652\\
2.97808286959396	1.45130101909892\\
2.91357071143426	1.31048152477025\\
2.82113288193912	1.13847444146357\\
2.68955830356001	0.927033947156868\\
2.50397138595715	0.667137086569504\\
2.24660171292782	0.351150802508159\\
1.9006557586107	-0.022570977245387\\
1.45903192426012	-0.442983784762838\\
0.935743711773808	-0.881729012214833\\
0.370352399760586	-1.29810141004348\\
-0.184260873698683	-1.65496980145545\\
-0.683992062919049	-1.93362144158048\\
-1.10690190382878	-2.13553351675148\\
-1.45125304056741	-2.27390535791362\\
-1.72646580436798	-2.36469590442473\\
-1.94539735940974	-2.42180269160389\\
-2.12025416477838	-2.45572707555127\\
-2.26115164240814	-2.47387374299914\\
-2.37597227077536	-2.48129425404923\\
-2.47069213137682	-2.48138746964015\\
-2.54979752218387	-2.47643153711324\\
-2.61665524824535	-2.46795461928887\\
-2.67380430738313	-2.45698365254657\\
-2.7231740280917	-2.44420873454906\\
-2.76624351461178	-2.43009134652392\\
-2.804157207019	-2.41493573690072\\
-2.83780856311567	-2.39893619990212\\
-2.86790084645478	-2.38220850145124\\
-2.89499150134421	-2.36481077198873\\
-2.91952471037678	-2.34675729424175\\
-2.94185536793424	-2.32802739607041\\
-2.96226673544778	-2.30857086859801\\
-2.98098335819641	-2.28831080921709\\
-2.99818033221118	-2.26714443625605\\
-3.01398964954104	-2.24494217170217\\
-3.02850407393302	-2.22154509715157\\
-3.04177877142498	-2.19676072585496\\
-3.0538307123217	-2.17035687706612\\
-3.06463564530312	-2.14205326725037\\
-3.0741221912185	-2.11151022469609\\
-3.08216227591155	-2.07831366445504\\
-3.08855666595789	-2.04195509708647\\
-3.09301371161753	-2.00180494507621\\
-3.09511842166653	-1.95707675054247\\
-3.09428751965502	-1.90677891236173\\
-3.08970390002161	-1.8496493307119\\
-3.08022054787565	-1.78406675893942\\
-3.06421905138112	-1.70793096236617\\
-3.03940094597951	-1.61850272426761\\
-3.00248164581652	-1.51219654758801\\
-2.9487497377427	-1.38432932031328\\
-2.87146019074656	-1.22886076983121\\
-2.76108079899122	-1.03824471047934\\
-2.60458386264906	-0.803692396134978\\
-2.38540845463613	-0.516471258475225\\
-2.08550773718212	-0.171195817604975\\
-1.69154178596567	0.228301500494895\\
-1.20566764318991	0.662469355804623\\
-0.655061178092835	1.09533772275853\\
-0.0884979704197673	1.48566918100562\\
0.442917165571432	1.80437867094387\\
0.905533049896758	2.04349773065613\\
1.28846169219662	2.21165095136388\\
1.59671409466504	2.32430362517328\\
1.84216252941488	2.39672320743641\\
2.0376485110943	2.44113560505875\\
2.19440565187908	2.46641069964569\\
2.3214083155111	2.47868070833737\\
2.42553419358814	2.48209316550803\\
};
\addplot [color=mycolor1, forget plot]
  table[row sep=crcr]{%
2.42914281428483	2.52214630241406\\
2.5098606425793	2.51965884751003\\
2.57787455974418	2.51322203854237\\
2.63586005304591	2.50398887156166\\
2.68584133577255	2.4927359032191\\
2.7293640482531	2.47998518591892\\
2.76761984560266	2.46608444165716\\
2.80153619318871	2.45126002906961\\
2.83184130663175	2.4356521034578\\
2.8591113714304	2.41933800609414\\
2.88380507716818	2.40234775957729\\
2.90628899315419	2.38467416302029\\
2.92685624829824	2.36627908944838\\
2.94574022954792	2.3470970047594\\
2.96312448231146	2.32703633653899\\
2.97914961171824	2.30597904757141\\
2.99391769395348	2.2837785644528\\
3.00749447417415	2.2602560429447\\
3.01990942140661	2.23519479358935\\
3.03115350419836	2.20833252217004\\
3.04117431513899	2.17935083817664\\
3.04986787277575	2.14786122530522\\
3.05706601807852	2.11338631890824\\
3.06251772928118	2.07533485311084\\
3.06586179724353	2.03296796742609\\
3.06658697040766	1.9853536256505\\
3.06397364630225	1.93130461639889\\
3.05700809149553	1.86929391704424\\
3.04425552116582	1.79733918115981\\
3.02367162803375	1.71284623573629\\
2.99232317303891	1.61240136761138\\
2.94597874879659	1.49150834188721\\
2.87852863649959	1.34428980571425\\
2.78122268186052	1.16323950414677\\
2.64184244700154	0.939270709619785\\
2.44428403194152	0.662620627599521\\
2.16978442306006	0.325602725991836\\
1.80195354986152	-0.071787254720533\\
1.33713714180375	-0.514344623845644\\
0.796155506874123	-0.968029684742976\\
0.225385408830048	-1.38848241834113\\
-0.320495137767467	-1.7398323501747\\
-0.801466793838679	-2.00808984704342\\
-1.20169669685893	-2.19920949262367\\
-1.52401280356067	-2.3287433088123\\
-1.7800298904326	-2.41320755995004\\
-1.9831462530216	-2.46619055938935\\
-2.14531363799268	-2.49765257942019\\
-2.27613067877928	-2.51449980177372\\
-2.38294898519408	-2.52140184213171\\
-2.47128834382054	-2.52148752949664\\
-2.54527068598905	-2.51685142827531\\
-2.60798137799079	-2.50889931954025\\
-2.66174506026186	-2.49857737059843\\
-2.70832843285511	-2.4865226709097\\
-2.74908701514419	-2.47316205670539\\
-2.78507089292095	-2.45877723410144\\
-2.81710103606706	-2.44354791224032\\
-2.845824633339	-2.427580480807\\
-2.87175544628144	-2.41092706846434\\
-2.89530339855213	-2.39359809108922\\
-2.91679634871061	-2.37557028981545\\
-2.936496102415	-2.35679153970488\\
-2.95461009070748	-2.33718323428111\\
-2.97129969082941	-2.31664072628661\\
-2.9866858343625	-2.29503207100744\\
-3.00085229047382	-2.27219513542462\\
-3.01384679610838	-2.24793297581879\\
-3.02568000197646	-2.22200722554269\\
-3.03632198587308	-2.19412905234739\\
-3.04569582223605	-2.1639470178792\\
-3.05366734848216	-2.13103087246186\\
-3.06002977695509	-2.09484990883986\\
-3.0644810800858	-2.0547439292298\\
-3.06659099391686	-2.00988408513625\\
-3.06575284006289	-1.95921974918372\\
-3.06111285637371	-1.90140609570132\\
-3.05146592213324	-1.83470518767195\\
-3.03510092726093	-1.75685131364158\\
-3.00957113235041	-1.66487000420575\\
-2.97135518288351	-1.5548423264199\\
-2.91536692282082	-1.42161891556544\\
-2.83428115319192	-1.25852896894489\\
-2.7177092002887	-1.05723449872739\\
-2.55148008890396	-0.808110195610234\\
-2.31782762169535	-0.501924537373715\\
-1.99821561543265	-0.133944263918715\\
-1.58104134546577	0.289132020717469\\
-1.07374775719885	0.742526958041193\\
-0.510972336618491	1.18507853541027\\
0.0537157539135096	1.57423152002543\\
0.570635840131045	1.88433173513163\\
1.01181445364394	2.11241994787035\\
1.37197100770443	2.27059809947887\\
1.65945988956961	2.37567301381497\\
1.88740169647059	2.44293063520625\\
2.06869210106127	2.48411823489821\\
2.21413317989098	2.5075678533355\\
2.3321570981413	2.51896913700286\\
2.42914281428483	2.52214630241406\\
};
\addplot [color=mycolor1, forget plot]
  table[row sep=crcr]{%
2.43000560883911	2.55857157411362\\
2.50542930976853	2.5562462145367\\
2.5691541465995	2.55021438887263\\
2.62363377252617	2.54153864695314\\
2.67072424682261	2.53093579785419\\
2.71184359297447	2.51888853629303\\
2.74808617215659	2.50571873005342\\
2.78030480632662	2.49163582113813\\
2.80917004144808	2.47676895742148\\
2.83521318831917	2.46118836197196\\
2.85885777907958	2.44491946682238\\
2.88044266719758	2.42795207482972\\
2.90023901419008	2.41024599991185\\
2.91846271780422	2.39173410286394\\
2.93528334893509	2.37232328020922\\
2.9508303105995	2.35189370892216\\
2.9651966635197	2.33029645555811\\
2.97844084339309	2.30734939306867\\
2.99058629571449	2.28283120819943\\
3.00161884823388	2.25647310514837\\
3.01148139949471	2.22794759411449\\
3.02006518730663	2.19685346859266\\
3.02719646128919	2.16269568598247\\
3.032616740877	2.12485832201185\\
3.03595387466936	2.08256800212379\\
3.03667964313811	2.03484413297623\\
3.03404737943109	1.98043076123442\\
3.02699959897154	1.91770289468106\\
3.01403034841165	1.84453769696113\\
2.99297927404867	1.75813866559983\\
2.9607241064281	1.65480066598533\\
2.91272749734146	1.52961106448614\\
2.84239275664357	1.37611132381162\\
2.74022248164007	1.18602667489593\\
2.59293678693402	0.949371056469283\\
2.38316159844013	0.655625245963671\\
2.09121669429894	0.297186618404667\\
1.70149648802604	-0.123883606133013\\
1.21452152414961	-0.58761333599355\\
0.6584099902644	-1.05409912183383\\
0.0857008954183244	-1.47609947001611\\
-0.448648923464263	-1.82012192354314\\
-0.909667247310918	-2.07730900092981\\
-1.28747756439644	-2.25775275935131\\
-1.58881445245579	-2.3788685303394\\
-1.82692255391993	-2.45742904132522\\
-2.01543400573807	-2.50660329190737\\
-2.16593074767542	-2.53580052536536\\
-2.287479025774	-2.5514529918328\\
-2.38692357646327	-2.55787740509025\\
-2.46936206415589	-2.55795626182771\\
-2.53858447918773	-2.55361744838998\\
-2.59742172857421	-2.54615564023323\\
-2.64800528589124	-2.53644346222473\\
-2.69195542105061	-2.52506949611656\\
-2.73051630339294	-2.51242866609342\\
-2.76465287164565	-2.49878175250296\\
-2.79512054459071	-2.48429480801279\\
-2.82251568425663	-2.46906536428564\\
-2.84731236472344	-2.45313983471818\\
-2.8698893167037	-2.43652493831251\\
-2.89054973919312	-2.41919495783558\\
-2.909535846854	-2.40109598879854\\
-2.92703944376479	-2.38214789987525\\
-2.94320940052873	-2.36224442497792\\
-2.95815660549424	-2.34125158732077\\
-2.97195672065852	-2.31900447936387\\
-2.98465086686821	-2.29530226248473\\
-2.99624416382948	-2.26990108413472\\
-3.00670183073618	-2.24250441619777\\
-3.01594228058818	-2.21275007129988\\
-3.02382627194575	-2.18019282211358\\
-3.03014065252787	-2.14428108934219\\
-3.03457444334637	-2.10432551822\\
-3.03668382089796	-2.05945635209584\\
-3.03584072775242	-2.00856523662128\\
-3.03115702909942	-1.95022534945272\\
-3.02137183160458	-1.88258151604598\\
-3.00468315655043	-1.80319948472831\\
-2.97849609747813	-1.70886187224915\\
-2.93904850659789	-1.59530079115891\\
-2.88086707514858	-1.45687279619227\\
-2.79601948321354	-1.28623227530518\\
-2.6732142885432	-1.07419103369826\\
-2.49708205345216	-0.810240129508162\\
-2.24864845706541	-0.484691325075486\\
-1.90909185335305	-0.0937332482755964\\
-1.46912479590529	0.352510784013221\\
-0.942205516224551	0.823542318810933\\
-0.370427105395691	1.27329407034341\\
0.18907995510647	1.65898618259533\\
0.689464296947835	1.9592427279986\\
1.10880856622092	2.17608537037877\\
1.44693610665946	2.32460882647534\\
1.71488697109807	2.42255091117002\\
1.92659462130799	2.48502084216652\\
2.09481030058443	2.52323798408973\\
2.22984903760658	2.54500951727224\\
2.33961026878607	2.55561145439992\\
2.43000560883911	2.55857157411362\\
};
\addplot [color=mycolor1, forget plot]
  table[row sep=crcr]{%
2.42839893897157	2.59161347785827\\
2.49891160776086	2.58943859617106\\
2.5586393815682	2.58378428441539\\
2.60983529655238	2.57563073807164\\
2.65420358518009	2.56564016929899\\
2.69304703887209	2.55425912691462\\
2.72737207287741	2.54178559052053\\
2.75796392474883	2.52841326757055\\
2.78544082091949	2.51426099890381\\
2.8102932758938	2.4993923045474\\
2.83291279628798	2.48382828371773\\
2.85361294631788	2.46755592724537\\
2.87264482072889	2.45053315679859\\
2.89020833727843	2.43269141648883\\
2.90646031256835	2.41393631034023\\
2.92151995843505	2.39414654095953\\
2.93547218584112	2.37317121902336\\
2.94836889534608	2.35082545040057\\
2.96022823981116	2.32688394425288\\
2.97103163931986	2.30107219914192\\
2.98071808016543	2.27305459043335\\
2.98917489968384	2.24241837019282\\
2.99622378987899	2.20865215841507\\
3.00160005897145	2.17111689337597\\
3.00492213694687	2.12900633796492\\
3.00564668835006	2.08129299979105\\
3.00300217589035	2.02665358622318\\
2.99588981025141	1.96336577112433\\
2.98273484323523	1.88916514773764\\
2.9612623557401	1.80104840378502\\
2.92815985732262	1.69500827984193\\
2.8785767525192	1.56569451323293\\
2.80541021102086	1.40603032094314\\
2.69837761964472	1.20691688874364\\
2.54308121316076	0.957407214341849\\
2.32083702813122	0.646213341099079\\
2.01114531096811	0.265981362292742\\
1.59961914501329	-0.178687219208589\\
1.09170162030625	-0.662444078680426\\
0.523148560624259	-1.13948493470032\\
-0.0482280155187187	-1.56062127744243\\
-0.568683111535514	-1.89578522694911\\
-1.00897944928141	-2.1414632250918\\
-1.36486403376478	-2.31146022447935\\
-1.64632583661454	-2.42459818521446\\
-1.86775609182295	-2.49765963341006\\
-2.04278178152133	-2.54331658749513\\
-2.18253654625458	-2.57042917760987\\
-2.29555273001758	-2.58498190658312\\
-2.38819486713318	-2.59096580666851\\
-2.46517079139088	-2.59103845424658\\
-2.52996784564671	-2.58697614108782\\
-2.58518641381692	-2.5799724839866\\
-2.6327834639797	-2.57083303956825\\
-2.67424707129207	-2.56010195883131\\
-2.71072081715863	-2.54814477250849\\
-2.74309261413974	-2.53520286296155\\
-2.77205847163025	-2.52142953749779\\
-2.79816859225998	-2.50691400797144\\
-2.82186093299831	-2.49169729619974\\
-2.84348578504061	-2.47578263684834\\
-2.86332383214565	-2.45914202411449\\
-2.88159938833896	-2.4417199475055\\
-2.89848998425722	-2.42343496121357\\
-2.91413309060001	-2.40417945271752\\
-2.92863048354321	-2.3838177685781\\
-2.94205053165504	-2.36218268433478\\
-2.95442848652006	-2.33907004504167\\
-2.96576466316621	-2.31423123091869\\
-2.97602017368924	-2.28736289594039\\
-2.98510959416056	-2.25809315870114\\
-2.99288955404653	-2.2259630587995\\
-2.99914166929046	-2.19040157886496\\
-3.00354738752565	-2.15069180315162\\
-3.00565100845932	-2.10592474384398\\
-3.00480512126117	-2.05493589499166\\
-3.00008956030611	-1.99621754249474\\
-2.99019013318851	-1.92779720802456\\
-2.97321606849211	-1.8470695906231\\
-2.9464247454543	-1.75056724231749\\
-2.90580954052106	-1.63365798584864\\
-2.84549761729985	-1.49017579735687\\
-2.75692173228313	-1.31205379608571\\
-2.62783923073872	-1.08919163341809\\
-2.44162569282153	-0.810148681924428\\
-2.17810548285373	-0.464835980349436\\
-1.81841420228852	-0.0506761365086647\\
-1.35621023012268	0.418182881435659\\
-0.811643913041554	0.905094544320476\\
-0.234026790647956	1.35956208676017\\
0.31732084940483	1.73973430997547\\
0.799593610230153	2.02919135077307\\
1.19704506377141	2.23474970256805\\
1.51401904710699	2.37399798823262\\
1.76364336609869	2.46524767908681\\
1.96030951561495	2.52328093884892\\
2.11647783080989	2.55876082706519\\
2.24194455351926	2.57898823216638\\
2.3440931018766	2.58885379549762\\
2.42839893897157	2.59161347785827\\
};
\addplot [color=mycolor1, forget plot]
  table[row sep=crcr]{%
2.42457355287415	2.621523428073\\
2.49052555348901	2.61948838573613\\
2.54652557549044	2.61418624100554\\
2.5946447751745	2.60652205238196\\
2.63644992345651	2.59710806248537\\
2.67313947238414	2.58635759423051\\
2.70564012646988	2.57454655749056\\
2.73467577519072	2.56185404543694\\
2.76081706891956	2.54838927883922\\
2.78451735616315	2.53420950163965\\
2.80613891728896	2.51933176099744\\
2.82597220488806	2.50374044594455\\
2.84424995897383	2.48739177711343\\
2.86115748122019	2.47021599066643\\
2.8768399395112	2.45211765197708\\
2.89140727205672	2.43297431092441\\
2.90493702657521	2.41263353221843\\
2.91747527242207	2.39090817293977\\
2.92903553501016	2.36756961221875\\
2.93959549574904	2.34233844187454\\
2.94909094576697	2.31487187547121\\
2.95740613582958	2.28474679228606\\
2.96435916638781	2.2514368545102\\
2.96968031544688	2.21428145292397\\
2.97298005798564	2.17244325254973\\
2.97370175105131	2.12484969488447\\
2.97105116935237	2.0701118059777\\
2.96389070753553	2.00641091054438\\
2.95057931564856	1.93134038439232\\
2.92872919451826	1.84168607285294\\
2.8948366770195	1.73312814808155\\
2.8437306653131	1.59985712136739\\
2.76778238029053	1.43413919446595\\
2.65588398862367	1.22599238599821\\
2.49246037204388	0.963443405009668\\
2.25748017035919	0.634428293595332\\
1.92974695414895	0.232038141126992\\
1.49658300369732	-0.236056243872965\\
0.969109087334938	-0.738532999794152\\
0.390884592398836	-1.22381266260643\\
-0.17609913651268	-1.64182639793885\\
-0.680710189455144	-1.96686693340394\\
-1.0998697153845	-2.20079527770734\\
-1.43448717000876	-2.36065433874548\\
-1.69718709536818	-2.46625907080926\\
-1.90310149597363	-2.53420367432117\\
-2.06566949584141	-2.57661116539494\\
-2.19552571443911	-2.60180275093068\\
-2.30067674173315	-2.61534176757871\\
-2.3870346266155	-2.62091880392253\\
-2.45894748078343	-2.62098579549941\\
-2.51962636315109	-2.61718088248714\\
-2.57146235558115	-2.61060556819087\\
-2.61625436119793	-2.60200413307862\\
-2.65537077903324	-2.59187997978379\\
-2.68986407397265	-2.5805715534327\\
-2.72055229588059	-2.56830226729342\\
-2.74807748000692	-2.5552135604492\\
-2.77294781574204	-2.54138686457017\\
-2.79556832752743	-2.52685815230927\\
-2.81626333304886	-2.51162741180158\\
-2.83529292886278	-2.49566454406377\\
-2.85286505359682	-2.47891262867918\\
-2.86914418934422	-2.46128913358444\\
-2.88425741063562	-2.44268538492361\\
-2.89829822701907	-2.42296441587734\\
-2.91132845317397	-2.4019571466252\\
-2.92337815065671	-2.37945668613773\\
-2.93444349154553	-2.35521036779604\\
-2.94448216812327	-2.32890891078919\\
-2.95340567856407	-2.30017180805798\\
-2.96106740583477	-2.26852763901446\\
-2.96724479935986	-2.23338743466187\\
-2.97161304719748	-2.19400840336833\\
-2.97370620162521	-2.1494441447997\\
-2.97285949413804	-2.09847578960964\\
-2.96812308252839	-2.03951613824585\\
-2.95813202714109	-1.97047573735495\\
-2.9409090029799	-1.88857618308768\\
-2.91356436109427	-1.79009320308102\\
-2.87184350497758	-1.67001501497011\\
-2.8094614865568	-1.52162358128125\\
-2.71718707040197	-1.33608164949733\\
-2.58177514845009	-1.10231058322602\\
-2.38528794053165	-0.807888873787678\\
-2.10636620236644	-0.442398846979578\\
-1.72638849802968	-0.00485656194973809\\
-1.24263964739572	0.48592730484768\\
-0.682563345208584	0.986820029965961\\
-0.102218521560992	1.44355819563319\\
0.438335201205228	1.8163867024402\\
0.901333779174422	2.09433643676666\\
1.27709655064445	2.28870855265296\\
1.57386992374557	2.41909611378573\\
1.80634015035226	2.50408009849308\\
1.98907227577506	2.55800280219291\\
2.1341303634353	2.59095821724414\\
2.25077741915636	2.60976288276926\\
2.34590198860912	2.6189491003489\\
2.42457355287415	2.621523428073\\
};
\addplot [color=mycolor1, forget plot]
  table[row sep=crcr]{%
2.41875121389676	2.64855056580978\\
2.4804621851032	2.64664564603231\\
2.53298198384958	2.64167236070519\\
2.57821643935736	2.63446706530605\\
2.61760751089612	2.62559618406988\\
2.65225890000336	2.61544246022642\\
2.68302485996611	2.60426140570847\\
2.71057344301215	2.59221854971585\\
2.73543192786561	2.57941415423297\\
2.75801972560871	2.56589961230934\\
2.77867238846366	2.55168820942244\\
2.79765920720138	2.53676195645932\\
2.81519610462298	2.52107557701103\\
2.83145499419961	2.50455831766021\\
2.84657039211788	2.48711396420712\\
2.86064379112477	2.46861923583169\\
2.87374608562724	2.44892055691462\\
2.88591814898051	2.42782904582382\\
2.89716947953687	2.40511338836097\\
2.90747462505364	2.38049005708748\\
2.91676683306505	2.35361006794347\\
2.92492801306316	2.32404109433784\\
2.93177356786976	2.29124323227207\\
2.93702985183273	2.25453595001245\\
2.940300775593	2.21305264996538\\
2.94101813331877	2.16567766229958\\
2.93836715219469	2.11095817986936\\
2.93117389733477	2.04698043291975\\
2.91773356609498	1.971195270361\\
2.89554727910843	1.88017399131407\\
2.86091934145589	1.76927385505018\\
2.80835074464214	1.63220393324588\\
2.72966595866595	1.4605325021307\\
2.61288926072011	1.24333100600682\\
2.44120489286289	0.967530144238057\\
2.19319903262895	0.620288794588344\\
1.84713034824417	0.195375852040556\\
1.39258091998237	-0.295881881399157\\
0.847097432884703	-0.815618147791448\\
0.262016003075326	-1.30678087031817\\
-0.297752906266213	-1.71958679315429\\
-0.784953348055309	-2.0334850455994\\
-1.18284747663061	-2.25558368389165\\
-1.49696516448511	-2.40566648821301\\
-1.74199732638048	-2.50417547685707\\
-1.93348132882357	-2.56736047948983\\
-2.08453129304809	-2.60676337977591\\
-2.20525285892648	-2.63018218870162\\
-2.303141400249	-2.64278522805763\\
-2.38368413694762	-2.64798585573839\\
-2.45089726047376	-2.64804768345522\\
-2.50773920845619	-2.64448267291651\\
-2.55641057010319	-2.6383081754167\\
-2.59856667747303	-2.63021237239539\\
-2.63546727246695	-2.62066123673404\\
-2.66808204557137	-2.60996823327106\\
-2.69716550575917	-2.59834013178333\\
-2.7233105241102	-2.58590733562874\\
-2.74698695784711	-2.57274402416935\\
-2.76856973595611	-2.55888146843808\\
-2.78835940702573	-2.54431666008423\\
-2.80659720941267	-2.52901761584986\\
-2.82347607789838	-2.51292621290439\\
-2.83914854982753	-2.49595906868554\\
-2.85373220922532	-2.47800673592301\\
-2.8673130622666	-2.45893129569143\\
-2.87994703717352	-2.43856226799976\\
-2.89165961838025	-2.41669059623414\\
-2.902443432702	-2.39306027588406\\
-2.91225337540026	-2.36735696387606\\
-2.92099855885655	-2.33919258990395\\
-2.92852993172743	-2.30808455024994\\
-2.9346217685964	-2.27342743290474\\
-2.93894423737863	-2.23445430645854\\
-2.94102270266134	-2.19018327078028\\
-2.94017697916224	-2.13934303530154\\
-2.93542987776555	-2.08026855077139\\
-2.92536828840323	-2.01075402551339\\
-2.90793065937658	-1.92784624848952\\
-2.88008112556126	-1.82755761831534\\
-2.83731364442982	-1.70448125563007\\
-2.77291809674371	-1.55131641033625\\
-2.67696845008103	-1.35840314840475\\
-2.53516236288159	-1.11361346343406\\
-2.32818792810975	-0.803494524702402\\
-2.0335326165553	-0.417390918102354\\
-1.63315184890276	0.0436749996658289\\
-1.12868460204515	0.55555786694248\\
-0.555370186023146	1.06841079251652\\
0.0246852414975189	1.52504568575042\\
0.552154009179338	1.88894204508582\\
0.995072749419049	2.15489125427073\\
1.34954746142368	2.33827777055983\\
1.62710857431764	2.46023536325952\\
1.84354179456123	2.53936050375954\\
2.01336098445757	2.58947345133322\\
2.1481605412785	2.62009781566314\\
2.25666811573677	2.63758949473537\\
2.34530095960704	2.64614792059426\\
2.41875121389676	2.64855056580978\\
};
\addplot [color=mycolor1, forget plot]
  table[row sep=crcr]{%
2.41112293345451	2.67293497202217\\
2.46888392391395	2.67115131797152\\
2.51815038737103	2.6664855191518\\
2.56067745902692	2.65971096292733\\
2.59779343839778	2.65135197568606\\
2.6305157273557	2.64176311344075\\
2.65963253350677	2.63118102002161\\
2.68576094455292	2.61975863731399\\
2.70938859486226	2.60758789630244\\
2.73090382849039	2.59471475314211\\
2.75061769692551	2.58114902195038\\
2.76878007240405	2.56687056330675\\
2.78559143844458	2.55183281196918\\
2.8012114227226	2.53596424504605\\
2.81576478577869	2.51916812567583\\
2.82934531927976	2.50132065759491\\
2.84201790198104	2.48226751893041\\
2.85381878116153	2.46181858333906\\
2.8647539665374	2.43974046017921\\
2.87479541554687	2.41574626808028\\
2.8838744197952	2.3894817673208\\
2.89187122478307	2.36050657363737\\
2.89859935644222	2.32826859854975\\
2.90378227390621	2.29206901942597\\
2.90701863342752	2.2510138451484\\
2.90773033309651	2.20394632395823\\
2.90508412791903	2.14935179353612\\
2.89787220142401	2.08522283788636\\
2.88432854807521	2.00886770995555\\
2.86184503952644	1.916639659751\\
2.82653307728443	1.80356260538734\\
2.77255796402825	1.66284105249546\\
2.69117522918125	1.4853017815288\\
2.56949523498214	1.25900114520614\\
2.38939363438971	0.969699040009248\\
2.12804201406743	0.60378361818778\\
1.76333966100236	0.155977290279563\\
1.28774343536765	-0.358089056184037\\
0.72595113667567	-0.893477459637415\\
0.136840561367889	-1.38815401247061\\
-0.413143683273551	-1.79385084762324\\
-0.881710561305588	-2.09580917159059\\
-1.25843603279283	-2.30612464980497\\
-1.55288592047271	-2.44682454430528\\
-1.7813060775171	-2.53866033056596\\
-1.9593657775584	-2.59741700772015\\
-2.09975332575718	-2.63403834464104\\
-2.21203071942399	-2.65581837430199\\
-2.3032006562605	-2.66755558712192\\
-2.37835240081262	-2.67240733578158\\
-2.44119571607529	-2.67246443748957\\
-2.49445761919681	-2.66912333754563\\
-2.54016488193194	-2.66332432155596\\
-2.57984205540937	-2.6557040973601\\
-2.61464993480629	-2.64669419424165\\
-2.6454827915952	-2.63658501085421\\
-2.67303717734763	-2.62556789562519\\
-2.69786105735113	-2.61376299788296\\
-2.72038922461991	-2.60123775195561\\
-2.74096903869299	-2.58801907026067\\
-2.75987924735955	-2.57410120073672\\
-2.77734377856985	-2.55945048935761\\
-2.793541793656	-2.54400782159856\\
-2.80861487682785	-2.52768920006705\\
-2.82267193569293	-2.51038468758181\\
-2.83579215890868	-2.491955765212\\
-2.84802618740108	-2.47223099410424\\
-2.85939547818274	-2.45099970448246\\
-2.86988964901186	-2.42800324168044\\
-2.87946135839992	-2.40292305034405\\
-2.8880179592727	-2.37536453831611\\
-2.89540870763175	-2.34483518069009\\
-2.90140561910968	-2.31071462828254\\
-2.90567500103259	-2.27221356472263\\
-2.90773500983351	-2.22831655519435\\
-2.90689191136059	-2.17770193064862\\
-2.90214344552749	-2.11862858948262\\
-2.8920308887691	-2.04877526035613\\
-2.87441080959479	-1.96501245936017\\
-2.84610196161767	-1.86308285568593\\
-2.8023432419264	-1.73716857524884\\
-2.7359855348757	-1.5793538010083\\
-2.63637493654125	-1.37909972675918\\
-2.48809271214243	-1.12315205734959\\
-2.27038916682889	-0.796974956580139\\
-1.95964380346866	-0.389789045959459\\
-1.53877736296124	0.0949035646598881\\
-1.01455397500912	0.626922435424875\\
-0.430388009563244	1.14961064156151\\
0.146485075591149	1.60386358275269\\
0.65890842805611	1.95746634107526\\
1.08124245096797	2.21110290073615\\
1.41497076391917	2.38377830939657\\
1.67431219460888	2.49773918811789\\
1.87576040675745	2.57138888298126\\
2.03360307134768	2.61796802131199\\
2.15891606343668	2.64643668053227\\
2.25989814780917	2.66271445002484\\
2.34251989232733	2.67069163243849\\
2.41112293345451	2.67293497202217\\
};
\addplot [color=mycolor1, forget plot]
  table[row sep=crcr]{%
2.40184832977892	2.69490294654388\\
2.45592399283202	2.69323249711691\\
2.50214465959359	2.68885462745153\\
2.54212772027489	2.68248487007762\\
2.57709765951613	2.67460878682083\\
2.60799303796255	2.66555490609426\\
2.63554167666137	2.65554239876409\\
2.6603140114805	2.64471252194086\\
2.68276133876676	2.6331494654426\\
2.7032434880069	2.62089414511776\\
2.72204899549531	2.60795318726871\\
2.73940987242113	2.59430452761714\\
2.75551239638478	2.57990051837748\\
2.77050489726047	2.56466908339177\\
2.78450318336705	2.54851321266359\\
2.79759401240609	2.53130889797802\\
2.80983681818068	2.51290144869255\\
2.82126373107969	2.49309996625585\\
2.83187775268152	2.47166957438063\\
2.84164873539826	2.44832077323747\\
2.85050654188269	2.42269497733214\\
2.85833036546549	2.39434486122415\\
2.86493260415113	2.36270750643164\\
2.87003477162677	2.32706741444136\\
2.87323149515886	2.28650507241896\\
2.87393635669052	2.23982470919166\\
2.87129963211028	2.18545186014563\\
2.86408200330457	2.12128703196035\\
2.85045876424552	2.04449595538954\\
2.82771432238764	1.95121039336567\\
2.79176614553139	1.8361102048244\\
2.73643546713606	1.69187094277892\\
2.65238485586856	1.5085310432453\\
2.52576058108595	1.27305716438514\\
2.33705616610302	0.969957779644263\\
2.06200028345344	0.584866453040137\\
1.67835833154431	0.113785735228833\\
1.18214610812648	-0.42263604568819\\
0.605896103323672	-0.971925405430728\\
0.0155723827383446	-1.46775406562183\\
-0.522312519362957	-1.86462709186122\\
-0.971324398289734	-2.15404224631969\\
-1.32715043845893	-2.35271822870257\\
-1.60279530089054	-2.48444397602867\\
-1.81560852479424	-2.57000911025981\\
-1.98117106386415	-2.62464285417094\\
-2.1116732147888	-2.65868520750944\\
-2.21612975662973	-2.67894743218132\\
-2.30107153790832	-2.68988207556836\\
-2.37121551086357	-2.69440981171719\\
-2.42998826643122	-2.69446257712194\\
-2.4799043774965	-2.69133079768861\\
-2.52283159053019	-2.68588400343966\\
-2.56017497653528	-2.67871156117301\\
-2.59300495343547	-2.67021324110272\\
-2.62214686783574	-2.66065811175114\\
-2.64824425085508	-2.65022322168541\\
-2.6718039350366	-2.63901919428186\\
-2.69322855158082	-2.6271071999645\\
-2.7128401395691	-2.61451012441759\\
-2.73089740259439	-2.60121972067263\\
-2.74760834240366	-2.58720087567082\\
-2.76313944898201	-2.57239369124244\\
-2.77762224240739	-2.55671378533672\\
-2.79115768371606	-2.54005100487191\\
-2.80381875835273	-2.52226656886728\\
-2.81565135572206	-2.50318850185678\\
-2.82667339617656	-2.48260504946375\\
-2.83687196698051	-2.46025556649031\\
-2.84619799109553	-2.43581810407097\\
-2.85455762548908	-2.40889255730016\\
-2.86179910668912	-2.37897771193176\\
-2.86769303203279	-2.3454397646135\\
-2.87190292593754	-2.30746876063673\\
-2.8739411298695	-2.26401771203514\\
-2.87310214305253	-2.2137166667641\\
-2.86836083396254	-2.15475036650014\\
-2.85821536473773	-2.08468305677749\\
-2.84044272991812	-2.00020763943901\\
-2.81171705992515	-1.8967905929009\\
-2.76701826400222	-1.7681864806928\\
-2.69874331942785	-1.60582994541959\\
-2.59547450038752	-1.39824243423878\\
-2.44061218668422	-1.13095957820658\\
-2.21190189403025	-0.78830978415099\\
-1.88467872020738	-0.359531309080791\\
-1.44327960281451	0.148851239629981\\
-0.900402958899709	0.699900932966638\\
-0.307870210739911	1.23020995198984\\
0.263072675579177	1.67991415655052\\
0.758799419972088	2.02207454581275\\
1.16029175073245	2.26323582897871\\
1.47391126142261	2.42552507639461\\
1.71600758027358	2.53191511984887\\
1.90345289959078	2.60044748272965\\
2.050174427334	2.64374495208587\\
2.16669927035574	2.67021656755708\\
2.26070957776861	2.6853697803666\\
2.33775391788163	2.6928077177244\\
2.40184832977892	2.69490294654388\\
};
\addplot [color=mycolor1, forget plot]
  table[row sep=crcr]{%
2.39105565046959	2.71466394891098\\
2.44168643913923	2.7130993739671\\
2.48505087718409	2.70899156370742\\
2.52264008371797	2.70300275790928\\
2.5555834337588	2.69558274045283\\
2.58474724027934	2.68703595981662\\
2.61080394512603	2.67756538313932\\
2.63428115432691	2.66730141297161\\
2.65559676206356	2.6563210478661\\
2.67508435447998	2.64466053875074\\
2.69301172311061	2.63232359642322\\
2.70959440943117	2.61928645031054\\
2.72500558850878	2.60550057022014\\
2.73938317702514	2.59089353517344\\
2.75283475043425	2.57536830065925\\
2.76544062920564	2.55880093494172\\
2.7772553117192	2.54103673622695\\
2.78830726502361	2.5218844812342\\
2.79859690981943	2.50110836859014\\
2.80809242516848	2.47841698035865\\
2.8167227152342	2.4534482560457\\
2.82436647131996	2.42574900413702\\
2.83083564371333	2.39474678989435\\
2.83585067333367	2.35971101846401\\
2.8390032983272	2.31969850350355\\
2.83970027217576	2.27347651354509\\
2.83707728986047	2.21941285894894\\
2.82986581942565	2.15531758873091\\
2.81618486650372	2.07821402943217\\
2.79321303813971	1.98400893839282\\
2.75667255046015	1.86702687505075\\
2.70003133071598	1.71938838577005\\
2.61333263100642	1.53029269807504\\
2.48170337858554	1.28553493342318\\
2.28417482600422	0.968284901730114\\
1.99500961002305	0.563450340431757\\
1.592112701967	0.0687015669963673\\
1.07581729000886	-0.489513382033007\\
0.487110705977186	-1.05080876082787\\
-0.101641479202603	-1.54545169258212\\
-0.625362480657865	-1.93196938522036\\
-1.05415748912718	-2.20840572355328\\
-1.38948135606153	-2.39565815264789\\
-1.64718968517319	-2.51882184377782\\
-1.84534326727985	-2.59849595956324\\
-1.99925943132172	-2.64928709852718\\
-2.12058058446563	-2.68093414564767\\
-2.21777859831219	-2.69978770858757\\
-2.29693437442552	-2.70997680820576\\
-2.36241671214485	-2.71420298673928\\
-2.4173901737337	-2.71425176296527\\
-2.46417386728752	-2.71131600828767\\
-2.50448964864668	-2.70620012054386\\
-2.53963310878231	-2.69944981929434\\
-2.57059186364337	-2.69143552820743\\
-2.59812807799416	-2.68240655774621\\
-2.62283663401238	-2.67252668224414\\
-2.64518657839968	-2.66189767317864\\
-2.6655509604152	-2.65057488898442\\
-2.68422850198075	-2.6385775051426\\
-2.70145943030626	-2.62589502007647\\
-2.71743705844548	-2.61249106726922\\
-2.73231619149175	-2.59830516623178\\
-2.74621908146117	-2.58325277124674\\
-2.7592393960627	-2.5672237744592\\
-2.77144446655925	-2.55007945339229\\
-2.7828759086264	-2.5316476957986\\
-2.79354854264693	-2.51171616361689\\
-2.80344735085607	-2.49002284799106\\
-2.81252196710058	-2.46624318803999\\
-2.82067785705647	-2.43997253467787\\
-2.82776284595137	-2.4107021746446\\
-2.83354688066196	-2.37778629462227\\
-2.83769169898016	-2.34039601769445\\
-2.83970513134109	-2.29745476968859\\
-2.83887159838726	-2.2475464208599\\
-2.83414520261156	-2.18878349072564\\
-2.82398340099586	-2.11861679335381\\
-2.80608582235899	-2.03356028041366\\
-2.77698255603545	-1.92879753547014\\
-2.73139010013438	-1.79763801733551\\
-2.6612351704066	-1.63082968596641\\
-2.55429669422483	-1.41588772712857\\
-2.39272325480453	-1.13704585751673\\
-2.15268490071564	-0.77744335145647\\
-1.8085585557974	-0.326512109778409\\
-1.34662019034505	0.205578402133016\\
-0.786342626004456	0.774402472667871\\
-0.188013121231037	1.31003964582746\\
0.374409853793793	1.75315073077823\\
0.852071731470989	2.08291484474334\\
1.23266559621533	2.3115591569358\\
1.52687413548001	2.46381909057387\\
1.75266676341099	2.56305005688437\\
1.92702017200067	2.62679740425486\\
2.06339979492001	2.66704294905067\\
2.17176767297531	2.69166092518201\\
2.2593053602677	2.70577013195945\\
2.33116355135832	2.71270670876987\\
2.39105565046959	2.71466394891098\\
};
\addplot [color=mycolor1, forget plot]
  table[row sep=crcr]{%
2.37884203763194	2.73240879223995\\
2.4262463534695	2.73094342879247\\
2.46692757621165	2.72708936391518\\
2.50226073696543	2.72145961913605\\
2.53328781311229	2.71447087784343\\
2.56080871004439	2.70640526457967\\
2.5854449290535	2.69745069568468\\
2.60768463890829	2.68772748032498\\
2.62791494206404	2.67730593553107\\
2.6464452044794	2.6662179994832\\
2.66352405277528	2.65446472577958\\
2.6793518021746	2.64202084628609\\
2.69408951242514	2.62883713957879\\
2.70786547887196	2.61484103799892\\
2.72077968780334	2.59993568796996\\
2.73290655569953	2.58399750554662\\
2.74429609976839	2.5668721135433\\
2.75497352690937	2.5483683842569\\
2.76493705589793	2.52825011894172\\
2.77415357525633	2.50622464329364\\
2.78255144923767	2.48192724851407\\
2.79000936000601	2.4548999034889\\
2.79633942553089	2.42456191960397\\
2.80126181244299	2.39016913416001\\
2.80436642492361	2.35075648925176\\
2.80505458049549	2.30505631604419\\
2.80244918709069	2.25138075418868\\
2.79525467387766	2.18745099670203\\
2.78153604010664	2.11014802606856\\
2.758367560554	2.01514983613783\\
2.7212744512416	1.89641365784961\\
2.66336095320443	1.74547682738926\\
2.57402172341646	1.55064360100373\\
2.43730295388958	1.29644711496952\\
2.23068579536726	0.964623854548406\\
1.92695107239325	0.53940119782852\\
1.50447501405737	0.0205785582624191\\
0.968746023487389	-0.558742255112305\\
0.369736970208452	-1.13000185512183\\
-0.214705012903247	-1.62115746921311\\
-0.722436941866099	-1.9959639182891\\
-1.13057305149817	-2.25912794505457\\
-1.44588359012524	-2.43522464577601\\
-1.68651158363303	-2.55023300337369\\
-1.87089179250685	-2.62437145274093\\
-2.01394001635376	-2.67157654255079\\
-2.1267181216797	-2.70099465656543\\
-2.21716486116874	-2.71853801731161\\
-2.29093332892937	-2.72803299245666\\
-2.35206672882154	-2.7319778944991\\
-2.40348677360069	-2.73202299109313\\
-2.44733230308416	-2.72927115191948\\
-2.48519096075237	-2.72446665978726\\
-2.51825772237007	-2.71811489139163\\
-2.54744411116073	-2.71055909230139\\
-2.57345419832581	-2.70203023728123\\
-2.59683809457397	-2.69267976219929\\
-2.61803003408079	-2.68260120792063\\
-2.63737578015784	-2.67184454230672\\
-2.65515252366643	-2.66042552932652\\
-2.67158341559993	-2.64833164167971\\
-2.68684818699216	-2.63552545590418\\
-2.70109084093572	-2.62194610110999\\
-2.71442507394136	-2.60750907732493\\
-2.7269378446073	-2.59210456801815\\
-2.73869132010812	-2.57559421036609\\
-2.74972326774459	-2.55780613071517\\
-2.76004579562609	-2.53852787825735\\
-2.76964215815257	-2.51749667161289\\
-2.7784610964171	-2.49438607796518\\
-2.78640783518803	-2.46878782609997\\
-2.79333033579016	-2.44018684377489\\
-2.79899859316409	-2.40792670023395\\
-2.80307347538253	-2.37116126318891\\
-2.80505951574642	-2.32878629733392\\
-2.80423264437882	-2.27934157092956\\
-2.79952819480964	-2.22086929786549\\
-2.78936520895674	-2.15070788412002\\
-2.7713680059692	-2.06519087440544\\
-2.74192293759335	-1.95921180708588\\
-2.69547797082562	-1.82561616358676\\
-2.62347135048132	-1.65442475362061\\
-2.51283474043797	-1.43207320182174\\
-2.34438634992853	-1.1413920454077\\
-2.09264625946526	-0.76427828495814\\
-1.73114810967095	-0.2905765432957\\
-1.24871320700027	0.265184724864449\\
-0.672449713736175	0.850361898452582\\
-0.0709690600813887	1.38896482709428\\
0.480509122770547	1.82356632238975\\
0.938992152198613	2.14015562258572\\
1.29878931037204	2.35633720929923\\
1.57431739364831	2.4989422377873\\
1.78470493929849	2.59140743991512\\
1.94680749926314	2.65067669043398\\
2.0735538240495	2.68807927237218\\
2.1743349153968	2.71097316510059\\
2.2558500138841	2.72411098961202\\
2.3228751080575	2.73058038763443\\
2.37884203763194	2.73240879223995\\
};
\addplot [color=mycolor1, forget plot]
  table[row sep=crcr]{%
2.36527365438272	2.74830870908072\\
2.40964991746339	2.7469364995168\\
2.44780579915534	2.74332128955086\\
2.48100929698089	2.73803052651152\\
2.51022183713887	2.73145019589874\\
2.53618210125969	2.72384168192079\\
2.55946459271266	2.71537889659003\\
2.58052106015224	2.70617275202405\\
2.59971013934627	2.69628735428491\\
2.61731878089818	2.68575065701308\\
2.63357785800937	2.6745612993535\\
2.64867357201913	2.66269271269721\\
2.6627557512408	2.65009516481937\\
2.67594377880519	2.63669612760843\\
2.68833062790302	2.62239914930745\\
2.69998528743098	2.60708124673624\\
2.71095369814491	2.59058868007056\\
2.7212581648169	2.57273080906794\\
2.73089503983224	2.55327153086106\\
2.73983025987883	2.5319175355474\\
2.74799202070095	2.50830224501375\\
2.75525943556605	2.48196376072738\\
2.76144534500368	2.45231434242239\\
2.7662703691099	2.4185977235074\\
2.76932355026417	2.37982870970839\\
2.77000206564684	2.33470665069354\\
2.76741771764363	2.28149000033681\\
2.76024993931903	2.2178126088649\\
2.74651183528815	2.14041300741655\\
2.72317454125283	2.04473625024061\\
2.68556393474381	1.92435911002144\\
2.62640871451058	1.77020478075194\\
2.53442204404956	1.56962081682207\\
2.39250057485056	1.30577768731714\\
2.17647867610017	0.958875721384523\\
1.8576501387814	0.512529816055839\\
1.41526549058398	-0.030780494230754\\
0.860889959227299	-0.630372554180624\\
0.253891702488302	-1.20940152139326\\
-0.323558108606942	-1.69481349550972\\
-0.813700808611469	-2.05671810266489\\
-1.20091960518074	-2.30643522281392\\
-1.4967680434039	-2.47167953487244\\
-1.72114720592667	-2.57892790820529\\
-1.89257879775753	-2.64786151485634\\
-2.02546999690482	-2.69171490991658\\
-2.13028266636831	-2.71905474556955\\
-2.21443591196754	-2.73537676422754\\
-2.28317683298629	-2.74422401082905\\
-2.3402439663865	-2.74790596622875\\
-2.38833355093365	-2.7479476596335\\
-2.42941776898398	-2.74536870731669\\
-2.46496045199247	-2.74085775992939\\
-2.49606383485234	-2.73488281086318\\
-2.52356930303391	-2.72776187791145\\
-2.54812735187924	-2.71970888677516\\
-2.57024676673503	-2.7108637874166\\
-2.59032961545124	-2.70131246095496\\
-2.60869642198736	-2.69109987555867\\
-2.6256044407356	-2.68023866329648\\
-2.6412609982819	-2.66871448687934\\
-2.65583323412694	-2.65648905152365\\
-2.66945514000038	-2.64350127674358\\
-2.68223249488475	-2.6296669046439\\
-2.69424607072992	-2.61487663978795\\
-2.70555330792793	-2.59899275958483\\
-2.71618850375176	-2.58184397872029\\
-2.72616139775262	-2.56321817320308\\
-2.73545385022647	-2.54285234259458\\
-2.74401406053032	-2.52041887789477\\
-2.75174741329837	-2.49550675676421\\
-2.75850249706639	-2.46759563065094\\
-2.76404998802014	-2.43601978130414\\
-2.76805072479322	-2.39991742195724\\
-2.77000706768743	-2.35815851420924\\
-2.76918793999981	-2.30924073259587\\
-2.76451178202781	-2.25113782679321\\
-2.75436136354981	-2.18107670228624\\
-2.73628754151374	-2.09520878021411\\
-2.70653284298508	-1.98812972299983\\
-2.65927065768527	-1.85220040887384\\
-2.58543021080492	-1.67666991121794\\
-2.47104662203931	-1.44681282868073\\
-2.29552004164257	-1.14394425989429\\
-2.03164242041711	-0.748667517327314\\
-1.65225577728419	-0.251513538736312\\
-1.14943023701283	0.327810235700239\\
-0.558776498007111	0.927735895787799\\
0.0431409482925311	1.46687835816249\\
0.581416088700065	1.89118337284762\\
1.01983159465553	2.19397491574896\\
1.35905688462949	2.39782266432595\\
1.61664699951616	2.53115395667853\\
1.8124797275961	2.61722576532144\\
1.96310543470836	2.67229944803121\\
2.08086225396267	2.70704894646139\\
2.17457171154062	2.72833581772816\\
2.2504702101494	2.74056777650213\\
2.31298100828986	2.74660085820289\\
2.36527365438272	2.74830870908072\\
};
\addplot [color=mycolor1, forget plot]
  table[row sep=crcr]{%
2.3503853119908	2.76251494643398\\
2.39191393217561	2.76123037943232\\
2.42768859660251	2.75784042750851\\
2.45887833723077	2.75287022845848\\
2.48637012136388	2.74667722880223\\
2.510846019112	2.73950350164073\\
2.53283704608621	2.73150990440398\\
2.55276121997183	2.72279858834094\\
2.57095078490094	2.71342788164366\\
2.58767188916159	2.70342205775786\\
2.60313891570007	2.69277756784621\\
2.61752494851065	2.68146672721784\\
2.63096937682453	2.66943946143598\\
2.64358330724213	2.656623455247\\
2.65545321580629	2.64292285417333\\
2.66664308967188	2.62821550969925\\
2.67719515377777	2.61234860840996\\
2.68712912855599	2.59513236017421\\
2.69643979677686	2.57633121563916\\
2.70509244243733	2.55565180726134\\
2.71301542150625	2.53272641584916\\
2.72008867039546	2.50709018833066\\
2.7261262507295	2.47814946676669\\
2.73084989653595	2.44513726724934\\
2.73384867932385	2.40704990747715\\
2.73451683301892	2.36255561318433\\
2.73195661543024	2.30986101940349\\
2.72482435363388	2.2465139818327\\
2.711083155337	2.16911018188178\\
2.68760184537471	2.07285693063183\\
2.64950384721276	1.95093593613899\\
2.58912858477636	1.79362188109626\\
2.49447036384301	1.58723660965141\\
2.34719855894177	1.31347506200585\\
2.12139403076715	0.950889813008775\\
1.78687361851418	0.482581582351276\\
1.32425335231122	-0.0856291170117338\\
0.752183398073065	-0.704480587244766\\
0.139677587891934	-1.28892186282411\\
-0.42816231067355	-1.76638553844743\\
-0.899324293288053	-2.11435121474952\\
-1.26551887600693	-2.35054509822281\\
-1.54249592627137	-2.50526302268155\\
-1.7514250245256	-2.60513147346486\\
-1.91067266168712	-2.66916706266552\\
-2.03405548013089	-2.70988262961109\\
-2.13142588241301	-2.73528065477131\\
-2.20969916046123	-2.75046160356716\\
-2.27373753915316	-2.75870303328772\\
-2.3269942216892	-2.76213862713827\\
-2.37195570736334	-2.7621771646753\\
-2.41043972602469	-2.75976104871229\\
-2.44379557720665	-2.75552731042341\\
-2.47303976527881	-2.74990921479259\\
-2.49894883600904	-2.74320130790478\\
-2.52212372944933	-2.73560163068733\\
-2.54303497524366	-2.72723942343221\\
-2.5620548349764	-2.71819343096336\\
-2.57948042108719	-2.70850398289026\\
-2.59555047760972	-2.69818083952949\\
-2.61045762762849	-2.68720805526125\\
-2.62435730680135	-2.67554663761915\\
-2.63737420432198	-2.66313546515162\\
-2.6496067533685	-2.64989070435837\\
-2.66113000678477	-2.63570379354718\\
-2.67199706851311	-2.62043790966508\\
-2.68223910229461	-2.60392267912295\\
-2.69186378362787	-2.5859467119911\\
-2.70085187355923	-2.56624730312205\\
-2.70915133990608	-2.54449631638781\\
-2.71666808282094	-2.52028079436336\\
-2.72325175723032	-2.49307613112037\\
-2.72867429229312	-2.46220857806186\\
-2.73259726361307	-2.42680221280929\\
-2.73452189320905	-2.38570295979474\\
-2.73371147248218	-2.33736830171394\\
-2.72906928957888	-2.27970522948764\\
-2.71894380765585	-2.20982984310806\\
-2.70081399747552	-2.12370930394393\\
-2.67077795434588	-2.01563260904588\\
-2.62272724530581	-1.87745314027105\\
-2.54705860167537	-1.69759855679207\\
-2.42885477702257	-1.46009112163033\\
-2.24599939558024	-1.14460545563963\\
-1.96947512171323	-0.730403961985268\\
-1.57163178709569	-0.209047206830658\\
-1.04860508517012	0.3936365298367\\
-0.445360833644853	1.00649873650033\\
0.154214806613381	1.54369452870495\\
0.67719354757618	1.95604462684509\\
1.09485028267557	2.24455198983996\\
1.41382211716279	2.43625168009219\\
1.65421359806668	2.5606892612365\\
1.8362909434323	2.6407179570803\\
1.97615061287271	2.69185560424471\\
2.08550273385151	2.7241245251592\\
2.17260633020613	2.74391022237171\\
2.24325488576203	2.7552954844211\\
2.30153959553947	2.76092014780299\\
2.3503853119908	2.76251494643398\\
};
\addplot [color=mycolor1, forget plot]
  table[row sep=crcr]{%
2.33417922251919	2.77515857935272\\
2.37302446345421	2.77395663381085\\
2.40654963067401	2.77077951041994\\
2.43583199686454	2.76611296825101\\
2.46168950562335	2.76028785819516\\
2.48475173000904	2.7535282337604\\
2.50550933109755	2.74598276018145\\
2.5243489983959	2.73774540873181\\
2.54157844074649	2.72886912755511\\
2.55744444898051	2.71937479102455\\
2.57214604857724	2.70925687233551\\
2.58584410069049	2.6984867418507\\
2.59866826704092	2.68701413897711\\
2.61072194848004	2.67476712102847\\
2.62208558667726	2.66165061021449\\
2.63281854814552	2.64754350692396\\
2.6429596636201	2.6322941888767\\
2.65252635132696	2.61571404867556\\
2.6615120867452	2.59756851130044\\
2.66988176493976	2.57756468474824\\
2.6775641922802	2.55533438297514\\
2.68444047616295	2.53041064659606\\
2.69032634517252	2.50219495706871\\
2.69494524328088	2.46991090852706\\
2.69788708180155	2.4325378702118\\
2.69854424714761	2.38871466854988\\
2.69601088276543	2.33659781368121\\
2.68892192270199	2.27365026140483\\
2.67519210905055	2.19632400672814\\
2.65158830987743	2.09958293128851\\
2.61302737146264	1.9761971289107\\
2.55144333649516	1.81575407886478\\
2.45406877280891	1.60347200034296\\
2.30125726814113	1.31944292483288\\
2.06521822926748	0.94045104662107\\
1.71432392256784	0.44922297385281\\
1.23115674064735	-0.144283789810344\\
0.64254576172766	-0.781166412816022\\
0.0271944808310615	-1.36848884553553\\
-0.528487228620339	-1.83585571054603\\
-0.979468638449927	-2.16898652663144\\
-1.32465584313549	-2.39166122610503\\
-1.58337414476832	-2.5361915531421\\
-1.77761447908935	-2.62904253610715\\
-1.92538542429413	-2.68846398527979\\
-2.039851616804	-2.72623686500676\\
-2.13025406291663	-2.74981681312022\\
-2.20302147092732	-2.76392931196305\\
-2.26265139697322	-2.77160284798201\\
-2.31232951967749	-2.77480711141059\\
-2.35434685090081	-2.77484271481674\\
-2.39037762959059	-2.77258026463459\\
-2.42166492224548	-2.76860877200978\\
-2.44914575953804	-2.76332915972775\\
-2.47353657346962	-2.75701408575001\\
-2.495392336303	-2.74984676076529\\
-2.51514806328149	-2.74194642214541\\
-2.5331483197224	-2.73338515779818\\
-2.54966844276935	-2.72419899145912\\
-2.56492994429187	-2.71439505253689\\
-2.57911174959065	-2.7039559742482\\
-2.59235838752929	-2.69284222744608\\
-2.60478588133654	-2.68099280546934\\
-2.61648583152902	-2.66832446685929\\
-2.62752799079809	-2.65472957859868\\
-2.63796147539942	-2.64007245461603\\
-2.64781461497922	-2.62418392938969\\
-2.65709329070221	-2.60685372101366\\
-2.66577742457177	-2.58781989333687\\
-2.67381502638283	-2.56675438293266\\
-2.68111282640781	-2.54324305418456\\
-2.68752193708849	-2.51675799216027\\
-2.69281605417463	-2.48661859130609\\
-2.69665818485541	-2.45193621304128\\
-2.69854935756574	-2.41153439090227\\
-2.6977484920616	-2.36383216765948\\
-2.69314531457329	-2.30667127533311\\
-2.68305573365365	-2.23705737709181\\
-2.66488806280295	-2.15077062614394\\
-2.63459468046404	-2.04178326316977\\
-2.58577654841186	-1.90141537153563\\
-2.50827077513128	-1.7172172108897\\
-2.38614393387481	-1.47185548805787\\
-2.19565192404784	-1.1432245218687\\
-1.90588545328172	-0.709206749592182\\
-1.48896435522745	-0.162825728991575\\
-0.946038384383019	0.462888162769359\\
-0.332236635993454	1.0866376214684\\
0.262157127043303	1.6193428477606\\
0.767906914478278	2.01820505971951\\
1.16428518567672	2.2920606042203\\
1.46339148439475	2.47184040961561\\
1.68730986127867	2.58775757832882\\
1.85638013965914	2.66207117916182\\
1.98612588689922	2.70951094253041\\
2.08760480900204	2.73945608556077\\
2.16852420162778	2.75783643569324\\
2.23425447519047	2.76842852203312\\
2.28857408052101	2.77367002737662\\
2.33417922251919	2.77515857935272\\
};
\addplot [color=mycolor1, forget plot]
  table[row sep=crcr]{%
2.31662243599076	2.78635024525841\\
2.35293417322862	2.78522633722784\\
2.38433045880942	2.78225065824871\\
2.41180325991813	2.7778722261458\\
2.43610635920612	2.77239704930667\\
2.45782051375344	2.76603233073719\\
2.47739883494592	2.75891532736314\\
2.49519883713148	2.75113236092503\\
2.51150535913619	2.74273136444641\\
2.52654713200155	2.73373007205922\\
2.54050884222562	2.72412117266028\\
2.55353993303382	2.71387524921832\\
2.56576097919128	2.70294199797262\\
2.57726818947959	2.69124999428929\\
2.58813638716678	2.67870509974901\\
2.59842065983668	2.66518745748042\\
2.60815673126147	2.65054687583518\\
2.61735996812363	2.63459623152095\\
2.62602277037307	2.61710230605039\\
2.63410987622555	2.59777316859469\\
2.64155079762947	2.57624078212906\\
2.64822811992764	2.55203685768825\\
2.65395963466464	2.52455898553249\\
2.6584710289845	2.4930225250869\\
2.66135378373019	2.45639130137023\\
2.66199942680694	2.41327629119163\\
2.65949527222369	2.36178533355917\\
2.6524563644985	2.29929721566371\\
2.63875037745638	2.22211878641674\\
2.61504196243896	2.12496360453309\\
2.57603596220367	2.00017105922659\\
2.51324192059867	1.83659727800389\\
2.41308093908502	1.61826797727075\\
2.25448927532998	1.32352756237873\\
2.00767471046191	0.927262586440358\\
1.6396299310979	0.412023553211461\\
1.13564164668971	-0.207132141700543\\
0.531891017170401	-0.860550597649819\\
-0.0834486858662258	-1.44803462631231\\
-0.624497145372981	-1.90321556429516\\
-1.05427299723325	-2.22074456387053\\
-1.37856985036945	-2.4299693354227\\
-1.61965084392854	-2.56465624946155\\
-1.79992400599472	-2.65083349015314\\
-1.93687163112228	-2.70590311341063\\
-2.04296139720988	-2.74091147029444\\
-2.12682657384936	-2.76278570212073\\
-2.19442722084214	-2.77589557972898\\
-2.24991539535961	-2.78303561167106\\
-2.2962256307893	-2.78602219784378\\
-2.33546637180161	-2.78605506640721\\
-2.36917823068472	-2.78393789758166\\
-2.39850548094883	-2.78021491926245\\
-2.42431127936886	-2.77525686174071\\
-2.44725617194405	-2.7693159257264\\
-2.46785237352234	-2.76256144815528\\
-2.48650184003541	-2.75510330734979\\
-2.50352333224779	-2.74700737306404\\
-2.51917188068512	-2.7383056689549\\
-2.53365291367801	-2.72900291552327\\
-2.54713256339255	-2.7190804973541\\
-2.55974516930548	-2.7084984965352\\
-2.57159866170076	-2.69719616348986\\
-2.58277827002371	-2.68509100112774\\
-2.59334882299732	-2.67207648161534\\
-2.60335576135157	-2.65801827063435\\
-2.6128248474056	-2.64274867898989\\
-2.62176040693981	-2.62605887200621\\
-2.6301417522632	-2.60768811336371\\
-2.6379171756221	-2.58730895947424\\
-2.64499451442092	-2.56450678882722\\
-2.65122668480172	-2.53875124671864\\
-2.65638960695603	-2.50935594678835\\
-2.66014834317394	-2.47542083303262\\
-2.66200458033747	-2.43574854134014\\
-2.66121400291597	-2.38872122347622\\
-2.65665419266781	-2.33211656121882\\
-2.64660999519241	-2.26282968154675\\
-2.62841985146878	-2.17645012552654\\
-2.59788825718166	-2.06662154701748\\
-2.54831481543425	-1.92410119775107\\
-2.46894529719174	-1.73549809709094\\
-2.34275646897583	-1.48200569342412\\
-2.14425031032493	-1.1395812012755\\
-1.8405441962249	-0.68470252093548\\
-1.40387341614752	-0.112406976598289\\
-0.841502519951627	0.535834141894993\\
-0.219445282773875	1.16814747973172\\
0.366866749718658	1.69376187769669\\
0.853610391505406	2.07772463285486\\
1.22833872432155	2.33666348088705\\
1.50801765790865	2.50478235141778\\
1.71616753137343	2.61254193101756\\
1.87292919968403	2.68144670794859\\
1.99315921357654	2.72540708647408\\
2.08724855622801	2.75317114021806\\
2.16236616371458	2.770233056107\\
2.22347880009503	2.78008048176229\\
2.27407015977558	2.78496175200151\\
2.31662243599076	2.78635024525841\\
};
\addplot [color=mycolor1, forget plot]
  table[row sep=crcr]{%
2.29764238025503	2.79617948043294\\
2.33155776382023	2.7951294127764\\
2.36093593396223	2.79234472148207\\
2.38668934938194	2.78824006338051\\
2.40951199610384	2.78309819003935\\
2.4299391191885	2.77711051459377\\
2.44838879937177	2.77040359931626\\
2.46519131130015	2.76305660106635\\
2.48061012244609	2.75511277219028\\
2.49485707323489	2.74658694414486\\
2.50810342978124	2.73747019904336\\
2.5204879439354	2.72773247612557\\
2.53212268191843	2.71732355777748\\
2.54309712414534	2.70617266786296\\
2.55348085049531	2.69418675228871\\
2.56332497685799	2.68124736911181\\
2.57266237715152	2.66720596998822\\
2.58150658947687	2.65187718369467\\
2.58984914288974	2.63502948882822\\
2.5976548223296	2.61637234942624\\
2.60485406837627	2.59553842849492\\
2.61133121281677	2.57205880296976\\
2.61690645784271	2.54532803886171\\
2.62130820407685	2.51455431738734\\
2.62413014725095	2.47868715322246\\
2.62476382931063	2.43631099338295\\
2.62229085265392	2.38548613158552\\
2.61530762106444	2.32350741071506\\
2.60163561168301	2.24653420776192\\
2.57783619728613	2.14902122911547\\
2.53839509382159	2.02285473924832\\
2.47437437686926	1.85610842486001\\
2.37132537767582	1.63151301647124\\
2.20664964662485	1.32549983636793\\
1.94841036224136	0.910921515403966\\
1.5623335377777	0.370431734902438\\
1.03732013187301	-0.274642951463033\\
0.420138859990144	-0.942770060131745\\
-0.192120709557591	-1.52749140993308\\
-0.716137112867406	-1.96845936081391\\
-1.12384148689449	-2.26973704776009\\
-1.42744562508792	-2.46563371112109\\
-1.65151005015505	-2.59082142391287\\
-1.81849752043933	-2.67064968374397\\
-1.94522530502567	-2.72160981845203\\
-2.0434324665042	-2.75401654034908\\
-2.12115231049674	-2.77428731292085\\
-2.18389439956994	-2.78645439868368\\
-2.23548337572101	-2.79309220063738\\
-2.27861768269272	-2.7958735475304\\
-2.31523492728519	-2.79590386087272\\
-2.3467509926555	-2.7939242853969\\
-2.37421804826164	-2.79043718477528\\
-2.39843040416711	-2.78578503878942\\
-2.41999651456956	-2.78020088696098\\
-2.43938871911307	-2.77384106170873\\
-2.45697812031413	-2.76680666955355\\
-2.47305937586617	-2.75915776364104\\
-2.48786853217569	-2.75092264808629\\
-2.50159596948731	-2.74210383819441\\
-2.5143958431194	-2.73268162772698\\
-2.52639295089314	-2.72261584436207\\
-2.53768764755075	-2.71184612372187\\
-2.54835920793859	-2.70029084918318\\
-2.55846787556392	-2.68784475507024\\
-2.5680556957367	-2.67437504951772\\
-2.57714610153282	-2.65971575798996\\
-2.58574207511867	-2.64365979494904\\
-2.5938225208964	-2.62594800823379\\
-2.60133622402374	-2.60625406320411\\
-2.60819237113583	-2.58416347219998\\
-2.61424598537452	-2.55914421883699\\
-2.61927561430565	-2.53050509658149\\
-2.62294892622753	-2.49733578203864\\
-2.62476901977294	-2.45841930989898\\
-2.62398934180504	-2.41210221761324\\
-2.61947654268191	-2.35609894231874\\
-2.6094855718454	-2.28719332139928\\
-2.59128520730172	-2.2007795031393\\
-2.56052874427497	-2.09015840588625\\
-2.51020112791778	-1.94549010324041\\
-2.42891927198263	-1.75236866575036\\
-2.29848437511039	-1.49037886305939\\
-2.09150070389166	-1.13336474617869\\
-1.77303719770791	-0.656399625920843\\
-1.3159015701882	-0.0572398538969062\\
-0.734747581320188	0.612789276987113\\
-0.107048664920237	1.25102495865522\\
0.468223340686164	1.76689291123208\\
0.934333044396128	2.13466154427952\\
1.28716764037607	2.37850745298307\\
1.54789253602088	2.53524599458945\\
1.74095319874678	2.6351980127523\\
1.88605718476763	2.69897948316981\\
1.99732059302863	2.73966108812865\\
2.08446122808108	2.76537413976779\\
2.15412473300267	2.7811966420266\\
2.21089301465229	2.79034350550865\\
2.25797171794367	2.79488540293541\\
2.29764238025503	2.79617948043294\\
};
\addplot [color=mycolor1, forget plot]
  table[row sep=crcr]{%
2.27711966681336	2.80471326690848\\
2.30876470732822	2.80373318111351\\
2.33622690217155	2.80112983301123\\
2.36034442597795	2.79728567370915\\
2.38175539889466	2.7924616367204\\
2.40095253020421	2.786834310215\\
2.41832114071146	2.78052021289298\\
2.43416601174845	2.77359177997818\\
2.44873059324809	2.76608788938589\\
2.46221089244057	2.7580206871475\\
2.47476558676405	2.74937981003638\\
2.48652339508863	2.74013468284281\\
2.49758839947065	2.73023528881658\\
2.50804377223615	2.71961161443807\\
2.5179541893136	2.70817181549812\\
2.52736707217291	2.69579901338413\\
2.53631267572461	2.68234648611736\\
2.54480290812179	2.66763084538734\\
2.5528286079959	2.65142256070542\\
2.56035478463002	2.63343286564032\\
2.56731300020612	2.61329559929467\\
2.57358956422711	2.59054180412344\\
2.57900738873048	2.5645637648759\\
2.5832979924475	2.53456337815224\\
2.58605783945938	2.49947686287451\\
2.58667922539985	2.45786315029811\\
2.58423895852489	2.40773566618065\\
2.57731573035855	2.34630482938462\\
2.56368513748234	2.26957901976898\\
2.53980314901611	2.17174333944803\\
2.49992699240001	2.04420410512122\\
2.43464331397276	1.87419253977717\\
2.32856450729637	1.64302484427267\\
2.1574207074857	1.32502897461142\\
1.88697506485576	0.890884177558295\\
1.48187061299944	0.32374191689811\\
0.935748434048338	-0.347378077649403\\
0.307228859364719	-1.02797244346139\\
-0.298652652289612	-1.60678448308606\\
-0.803317499659814	-2.03157712530496\\
-1.18822916270127	-2.3160609748746\\
-1.47140286407627	-2.49879359146632\\
-1.67906419043248	-2.61482261881218\\
-1.83340828547876	-2.68860811215353\\
-1.95047408647239	-2.73568281866713\\
-2.04125105712786	-2.76563715013135\\
-2.11318329204814	-2.78439779818418\\
-2.17134788716336	-2.79567665281743\\
-2.21925906595674	-2.80184076968151\\
-2.25939302529013	-2.80442825172745\\
-2.29352720274335	-2.80445617227769\\
-2.3229607998811	-2.80260711193358\\
-2.34865991490779	-2.79934421143854\\
-2.37135453980611	-2.7949834601364\\
-2.39160445426495	-2.78973991383537\\
-2.40984471980418	-2.78375769234307\\
-2.42641757489392	-2.77712966684287\\
-2.44159511028663	-2.76991044141953\\
-2.45559558365701	-2.76212485746951\\
-2.46859526417699	-2.75377341099572\\
-2.48073706973861	-2.74483544778696\\
-2.49213684364279	-2.73527066118357\\
-2.50288783414984	-2.72501918483355\\
-2.51306373878632	-2.71400040098098\\
-2.52272052208447	-2.70211044165896\\
-2.53189708615449	-2.68921822164761\\
-2.54061474783449	-2.67515968626879\\
-2.54887533342882	-2.65972975940298\\
-2.55665751642393	-2.64267120488666\\
-2.56391075765847	-2.62365921967494\\
-2.57054580187642	-2.60227998503542\\
-2.57642004026832	-2.57800049216393\\
-2.58131499451859	-2.55012553322605\\
-2.58490141245528	-2.51773547914119\\
-2.58668444774294	-2.4795948022608\\
-2.58591614512812	-2.43401533480081\\
-2.58145318549996	-2.3786485182647\\
-2.57152137197258	-2.31016523795941\\
-2.55331927040942	-2.22375785222391\\
-2.52234413115481	-2.11236728680922\\
-2.4712496105744	-1.96551581309859\\
-2.38797880348207	-1.76769630314829\\
-2.25305643044805	-1.49672758793427\\
-2.03702474111524	-1.1241431238314\\
-1.70284397425902	-0.62365208058744\\
-1.22450190265727	0.00335988167970699\\
-0.625509477875658	0.694114905155601\\
0.00485503797733931	1.3352611742484\\
0.566072085352282	1.83867314563659\\
1.01006365019401	2.18906550931705\\
1.34087070187219	2.41771869486442\\
1.58313849312425	2.56337217786309\\
1.76176167440832	2.65585265256686\\
1.89581443079859	2.7147768970127\\
1.99861613727433	2.75236420962431\\
2.07921101972012	2.77614516660634\\
2.14373753399703	2.79080032912494\\
2.19641075219014	2.79928685625849\\
2.24017376828498	2.80350843912919\\
2.27711966681336	2.80471326690848\\
};
\addplot [color=mycolor1, forget plot]
  table[row sep=crcr]{%
2.25487688810311	2.81199323983976\\
2.28436799553838	2.81107956910574\\
2.3100089430402	2.80864861750409\\
2.33256834724573	2.80505258966712\\
2.35263201677768	2.8005319114714\\
2.37065281728478	2.79524922692905\\
2.38698536527165	2.7893116067201\\
2.40191053147013	2.78278517030559\\
2.41565297676594	2.77570470126259\\
2.42839383599721	2.76807985783397\\
2.4402799545843	2.7598989765926\\
2.45143061802544	2.75113108183596\\
2.4619424020688	2.74172645626618\\
2.47189255390494	2.73161594466065\\
2.48134115440656	2.72070901609231\\
2.49033218214529	2.7088904763179\\
2.49889348116304	2.69601557851705\\
2.50703550697458	2.68190310486602\\
2.51474856656858	2.66632575464071\\
2.52199804710781	2.64899683529037\\
2.52871679629322	2.62955174742179\\
2.53479329508715	2.6075219805043\\
2.54005341352389	2.5822981249259\\
2.54423212086449	2.55307647608831\\
2.54692909744006	2.51878068145456\\
2.54753797263876	2.47794475558929\\
2.54513142675877	2.4285353127635\\
2.53827094999473	2.36767688352728\\
2.52468583058409	2.29122165295407\\
2.5007230709378	2.19307131555196\\
2.46039904329072	2.06411951158295\\
2.3937904251023	1.89068329845897\\
2.28448754298865	1.65252313899048\\
2.10638869534037	1.32164367346809\\
1.82279138568968	0.86641600175977\\
1.39754469963532	0.271049611753693\\
0.830426111489532	-0.426006181532845\\
0.193139376052034	-1.11630813728006\\
-0.402818609233224	-1.68582385182827\\
-0.885895466495825	-2.09254689146715\\
-1.24742522357496	-2.35979211481426\\
-1.51048266855685	-2.52955874150764\\
-1.7023428922059	-2.63676350644936\\
-1.84464870306425	-2.7047947972659\\
-1.95256916176499	-2.74819161705105\\
-2.03633170215196	-2.77583072299165\\
-2.10280408108491	-2.79316676413249\\
-2.15664861365666	-2.80360735984377\\
-2.20108504615934	-2.80932396858313\\
-2.23838006958827	-2.81172803992354\\
-2.27016067988777	-2.81175371512871\\
-2.2976166988921	-2.81002861637584\\
-2.32163361419796	-2.80697906088408\\
-2.34288119505579	-2.8028961489738\\
-2.36187363699789	-2.79797802636934\\
-2.37901107343624	-2.79235732394082\\
-2.39460868067106	-2.78611916859778\\
-2.40891737554244	-2.77931305185168\\
-2.42213871244104	-2.77196058679941\\
-2.43443570141001	-2.76406041841396\\
-2.44594069599043	-2.75559107179853\\
-2.45676111987736	-2.74651221050273\\
-2.46698354232244	-2.73676456192228\\
-2.47667642714893	-2.72626860544654\\
-2.48589173823608	-2.7149219818025\\
-2.49466546269359	-2.70259544602571\\
-2.50301699232055	-2.68912703001946\\
-2.51094716403585	-2.67431387862586\\
-2.51843457479595	-2.65790094139657\\
-2.52542951768997	-2.63956528977856\\
-2.53184447167411	-2.61889420571681\\
-2.53753941350928	-2.59535422173673\\
-2.54229912529256	-2.56824676694474\\
-2.54579782094551	-2.53664362217958\\
-2.54754322323746	-2.49929139094935\\
-2.54678661119326	-2.45446760501918\\
-2.5423753367342	-2.39976018367811\\
-2.53250625450951	-2.3317241259899\\
-2.51430614558058	-2.24534136350177\\
-2.48310931004538	-2.13317135168002\\
-2.43121704898488	-1.98404962205009\\
-2.3458439814805	-1.78126542388781\\
-2.20611831389706	-1.50068724615164\\
-1.98033238609531	-1.11131775015247\\
-1.62930687482662	-0.585608586595307\\
-1.12902246122067	0.0702408160438668\\
-0.513522037910772	0.780218103330241\\
0.116111253157241	1.42083254405841\\
0.660204998816192	1.90902779295882\\
1.08073270905983	2.24097041048473\\
1.38947351393834	2.45439728713025\\
1.61379618081722	2.58927045308193\\
1.77860543104869	2.67460103391161\\
1.90217247786813	2.72891623217254\\
1.99697791304627	2.7635793316525\\
2.07139663550914	2.78553726263968\\
2.13107658113404	2.79909109356255\\
2.17988317892706	2.80695414441097\\
2.22051134717918	2.81087290798355\\
2.25487688810311	2.81199323983976\\
};
\addplot [color=mycolor1, forget plot]
  table[row sep=crcr]{%
2.2306613868761	2.81803070297518\\
2.25810690410877	2.81718012527407\\
2.28201516003601	2.81491320421253\\
2.30308950693452	2.81155368741527\\
2.32186667064398	2.80732269213538\\
2.33876212010734	2.80236972560499\\
2.35410163785533	2.79679295696782\\
2.36814362598294	2.79065256203978\\
2.38109507726808	2.78397948565953\\
2.39312313224099	2.77678107656757\\
2.40436349690098	2.76904449916777\\
2.41492657297905	2.76073847420239\\
2.42490186835632	2.75181366374805\\
2.43436105653859	2.74220184468911\\
2.44335990651309	2.73181387613749\\
2.45193918373476	2.72053633609572\\
2.46012451013591	2.70822655989054\\
2.4679250472588	2.69470563459613\\
2.47533070951499	2.67974865995893\\
2.48230739246368	2.66307123355111\\
2.48878936392124	2.64431058778368\\
2.49466743013696	2.6229989883018\\
2.4997706103052	2.59852571315108\\
2.50383757236253	2.57008185871996\\
2.50647153235448	2.53657882915073\\
2.50706783155106	2.49652574468635\\
2.50469535887311	2.44784158954154\\
2.49789834998327	2.38756214474631\\
2.48435833264454	2.31137583767289\\
2.46030778970481	2.21288289339427\\
2.41950575059641	2.08242347523106\\
2.35147553534986	1.90531321212694\\
2.23868402405164	1.65958770416935\\
2.05300806937287	1.31467312543502\\
1.75510971736689	0.836517617528177\\
1.30849126874066	0.21118988845803\\
0.720798419557306	-0.511317529079947\\
0.0779140443342484	-1.20791854545575\\
-0.504310404303508	-1.76449354912078\\
-0.963651051034922	-2.15132519346601\\
-1.30133101338113	-2.40097689803586\\
-1.54462840311249	-2.55800318024844\\
-1.72127575581048	-2.65671069052081\\
-1.85211382956641	-2.71925994897991\\
-1.95136885095432	-2.75917169432432\\
-2.02850063951073	-2.78462216400938\\
-2.08981496010035	-2.80061234470425\\
-2.13957654655818	-2.81026070860493\\
-2.18072560715238	-2.81555396271566\\
-2.21533107455486	-2.81778429561671\\
-2.2448783990183	-2.81780785993293\\
-2.27045467227012	-2.81620060789198\\
-2.29286973444643	-2.81335422309845\\
-2.31273685207127	-2.80953638074976\\
-2.33052744438039	-2.80492929976484\\
-2.34660885364804	-2.79965478587334\\
-2.36127083449701	-2.7937906722762\\
-2.37474439994278	-2.78738164589066\\
-2.38721539229624	-2.78044630443655\\
-2.3988343416865	-2.77298159220663\\
-2.40972365384161	-2.76496532407901\\
-2.41998282343293	-2.75635722071773\\
-2.42969213301304	-2.74709867891067\\
-2.43891512794588	-2.73711134941806\\
-2.44770002625738	-2.72629446298113\\
-2.45608010789501	-2.71452071131373\\
-2.46407301201541	-2.70163033252864\\
-2.47167873369492	-2.68742284384416\\
-2.4788759266952	-2.67164557285018\\
-2.48561584719577	-2.65397770782051\\
-2.49181285049689	-2.63400793094603\\
-2.49732966899515	-2.61120267356519\\
-2.50195456246426	-2.58486040061168\\
-2.50536549382862	-2.554044685875\\
-2.50707310865982	-2.51748448173663\\
-2.5063283000101	-2.47342271799767\\
-2.50196930299109	-2.41938216717487\\
-2.49216346758393	-2.35179720417731\\
-2.47396280197052	-2.26542754675948\\
-2.4425289071623	-2.15242386901544\\
-2.38978363183492	-2.00087480010687\\
-2.30214558738244	-1.79274231006888\\
-2.15720197537847	-1.50172612091604\\
-1.92078092466796	-1.09405563207464\\
-1.55158685210626	-0.541139515258474\\
-1.02868761619051	0.144455732587054\\
-0.398536047886836	0.871547814567372\\
0.226482120535897	1.50768862618519\\
0.750336570209397	1.97786021388669\\
1.14618923806126	2.29038532705206\\
1.43290798360277	2.48861007795899\\
1.63980650934666	2.61301346167155\\
1.79139786468341	2.69150173946715\\
1.90500767470046	2.74143986091333\\
1.99224745257609	2.77333611988895\\
2.06083057765806	2.79357153066645\\
2.11593135374411	2.80608480485869\\
2.16108190935215	2.81335835477259\\
2.1987423302369	2.81699046255361\\
2.2306613868761	2.81803070297518\\
};
\addplot [color=mycolor1, forget plot]
  table[row sep=crcr]{%
2.20411869675044	2.82279803928925\\
2.22962048977649	2.82200742847636\\
2.25187975884548	2.81989662779306\\
2.27153851214511	2.81676257166763\\
2.28908734150646	2.8128081720023\\
2.30490655602639	2.80817054307244\\
2.31929481642068	2.80293945579783\\
2.33248937844525	2.79716948382655\\
2.34468059668352	2.790887965464\\
2.35602242581111	2.78410009843371\\
2.36664007142398	2.77679198415477\\
2.37663555958338	2.76893211680475\\
2.38609173629602	2.76047159494436\\
2.39507502710029	2.75134317398984\\
2.40363715134788	2.74145914604152\\
2.41181587340279	2.73070790680094\\
2.41963476568831	2.71894892691552\\
2.42710183821288	2.70600566388891\\
2.43420673359399	2.69165569968379\\
2.44091596335171	2.67561702216299\\
2.44716531868489	2.65752881254623\\
2.45284803969333	2.63692423640233\\
2.45779641799448	2.61319136133464\\
2.46175296364436	2.58551609671097\\
2.464324582366	2.55279737375112\\
2.46490843664902	2.51351861804726\\
2.46256950944333	2.46554910456749\\
2.45583393945943	2.40583098262153\\
2.44233259023548	2.32987794822825\\
2.41817504775361	2.23096462491157\\
2.37684071941484	2.09882560974564\\
2.30724399164613	1.91766663513266\\
2.19060259752446	1.66359274901303\\
1.99654643199345	1.30315549704448\\
1.68294141792844	0.799814089190649\\
1.21362999681624	0.142652742032227\\
0.606266086420765	-0.604237935836287\\
-0.0382997540981736	-1.30291832432706\\
-0.602702520520728	-1.84263705063436\\
-1.03625418886858	-2.2078342624618\\
-1.34972907059829	-2.43962109553829\\
-1.573657269817	-2.58415548714345\\
-1.73566550234319	-2.67468489667198\\
-1.8555751156271	-2.73200944254198\\
-1.94661241757109	-2.76861601619204\\
-2.01746950767089	-2.79199533022614\\
-2.07390549224461	-2.80671263664546\\
-2.11980415561342	-2.81561147538022\\
-2.15784017121593	-2.82050384345695\\
-2.18989557116555	-2.82256946670836\\
-2.21732242434057	-2.82259104341717\\
-2.24111117383757	-2.82109587128214\\
-2.26200054487199	-2.81844301039579\\
-2.28055069787866	-2.81487805671626\\
-2.29719284121756	-2.81056820777978\\
-2.31226348081426	-2.80562505579916\\
-2.32602845262016	-2.80011955472088\\
-2.33870003161801	-2.79409186837045\\
-2.35044925968698	-2.78755777079688\\
-2.3614149052239	-2.78051263692365\\
-2.37170999556761	-2.77293366269971\\
-2.38142655029902	-2.76478069164113\\
-2.39063892875089	-2.75599584069135\\
-2.3994060499628	-2.74650197585847\\
-2.40777262176581	-2.73619996140739\\
-2.41576940802391	-2.72496447442836\\
-2.42341245167745	-2.71263801805474\\
-2.43070103652721	-2.69902255510953\\
-2.43761398623936	-2.68386788206692\\
-2.44410362433956	-2.66685541318944\\
-2.45008628715942	-2.64757535338349\\
-2.45542757757212	-2.62549415098482\\
-2.45991936623893	-2.5999073759985\\
-2.46324351677259	-2.56987031334723\\
-2.46491374110498	-2.53409380753901\\
-2.4641805794229	-2.49078487179822\\
-2.45987276887681	-2.4373979208442\\
-2.45012653546256	-2.37023935955886\\
-2.431914115951	-2.28383038390616\\
-2.40021064168641	-2.16987733211045\\
-2.34652301112763	-2.01564625766097\\
-2.25638881346475	-1.8016197458856\\
-2.10567806579636	-1.4990673696351\\
-1.85751247124311	-1.07118544502344\\
-1.46860003919241	-0.488731048675673\\
-0.922577686600991	0.22731151805632\\
-0.280350139184698	0.968585112721327\\
0.335608902264612	1.59573520344585\\
0.83607009051639	2.0450385502171\\
1.20616863030855	2.33728236795813\\
1.47098269446232	2.5203802444838\\
1.66098315442301	2.63462778072726\\
1.7999268326179	2.70656813576981\\
1.90407499279771	2.7523467538357\\
1.98414951579792	2.78162251588829\\
2.0472127716962	2.80022858519715\\
2.09798230247861	2.81175764958274\\
2.13967244255459	2.81847325874164\\
2.1745208493223	2.821833771635\\
2.20411869675044	2.82279803928925\\
};
\addplot [color=mycolor1, forget plot]
  table[row sep=crcr]{%
2.17475110014827	2.82621406094749\\
2.19840629905796	2.82548043297719\\
2.21909692391879	2.82351815726881\\
2.23740723939928	2.82059887586622\\
2.25378441387765	2.81690831762872\\
2.26857565585033	2.81257189340399\\
2.2820540835625	2.80767144137197\\
2.2944370290331	2.802256246321\\
2.30589916343327	2.79635024908732\\
2.31658200720803	2.78995663347652\\
2.3266008614282	2.7830605250051\\
2.33604985169515	2.77563024340975\\
2.34500554288033	2.7676173513012\\
2.35352941858832	2.75895559282036\\
2.36166939494934	2.74955869074923\\
2.36946043370963	2.7393168467454\\
2.37692421752611	2.7280916469946\\
2.38406773332394	2.71570889107058\\
2.39088045524881	2.70194860287711\\
2.39732959413596	2.68653110035243\\
2.40335253200047	2.66909741690414\\
2.40884499632096	2.64918145274385\\
2.41364258794858	2.62616976861731\\
2.41749166427431	2.59924253789691\\
2.42000274952655	2.56728517874866\\
2.42057456170659	2.5287534147332\\
2.41826742025695	2.48146286987346\\
2.41158738197971	2.42225418535761\\
2.39810962268442	2.34645031439725\\
2.37380833079006	2.24696727030339\\
2.3318526433345	2.11286577582151\\
2.26047519304961	1.92710354457398\\
2.13948575923478	1.66360065413531\\
1.93599803799582	1.28569213576058\\
1.6049581204052	0.754385732317796\\
1.1116028431317	0.0634672083856049\\
0.486210425470154	-0.705837383852517\\
-0.155193166771875	-1.40136778907375\\
-0.697401134220052	-1.9200360919257\\
-1.10321661268172	-2.26194328464883\\
-1.39223713720369	-2.47567262411068\\
-1.59721657004419	-2.60798306054165\\
-1.74514557972819	-2.69064598612555\\
-1.85463854625831	-2.74299009034185\\
-1.93787834147517	-2.77646035973368\\
-2.00279362154754	-2.79787835832038\\
-2.05461279398557	-2.81139102620275\\
-2.096854717804	-2.8195803565799\\
-2.13194167177978	-2.82409297020052\\
-2.16157885351556	-2.82600241442489\\
-2.18699247474654	-2.82602211732219\\
-2.20908191952115	-2.82463350550535\\
-2.22851896072046	-2.82216487363621\\
-2.24581377331859	-2.8188409846643\\
-2.26135971457817	-2.81481485408586\\
-2.27546425534414	-2.81018842718826\\
-2.28837070089014	-2.80502616081054\\
-2.30027366745368	-2.79936395201235\\
-2.31133024281672	-2.79321492096289\\
-2.32166810233478	-2.78657298305792\\
-2.33139142673926	-2.77941478337059\\
-2.34058518567037	-2.77170032704136\\
-2.34931815648383	-2.76337246931098\\
-2.35764490647126	-2.75435529488997\\
-2.36560685442574	-2.74455129426235\\
-2.37323242620906	-2.73383711406644\\
-2.38053621180108	-2.72205749859815\\
-2.38751689890873	-2.70901682261767\\
-2.39415357385661	-2.69446730265018\\
-2.40039970249558	-2.67809250334192\\
-2.40617366261997	-2.6594840268271\\
-2.41134397360421	-2.63810811816795\\
-2.41570614112739	-2.61325705037652\\
-2.41894590520874	-2.5839770657284\\
-2.42057989803758	-2.54895945903643\\
-2.41985785096006	-2.5063725201115\\
-2.41559776917939	-2.45359674618959\\
-2.40590158809854	-2.38679936516417\\
-2.38765383558727	-2.30024009281017\\
-2.3556235539057	-2.18513341673795\\
-2.30085516358809	-2.02782512370027\\
-2.2078958747088	-1.80712729616441\\
-2.05068074657191	-1.49156354189462\\
-1.78935724797637	-1.0410346164601\\
-1.37892684098203	-0.426330064342468\\
-0.809611013312882	0.320429150179243\\
-0.158861852805466	1.07182267954521\\
0.442957258278228	1.68480962051844\\
0.916848606964636	2.11037621371517\\
1.26024549942691	2.38157864841056\\
1.50333812085555	2.54967090031518\\
1.67696961654767	2.65407864288348\\
1.80381260079955	2.71975355658681\\
1.89896626186733	2.76157779184113\\
1.97225037057874	2.78837006608586\\
2.03008883770538	2.80543387882379\\
2.0767591327988	2.81603146028465\\
2.11517249949107	2.82221875161342\\
2.14735572388684	2.82532186585778\\
2.17475110014827	2.82621406094749\\
};
\addplot [color=mycolor1, forget plot]
  table[row sep=crcr]{%
2.14185167975353	2.82811886201345\\
2.16375472099596	2.82743931153048\\
2.1829554808675	2.82561810950134\\
2.19998381115138	2.82290302535754\\
2.21524596830833	2.81946356060583\\
2.22905797280445	2.81541406673926\\
2.24166887153974	2.81082888159783\\
2.25327721545887	2.80575228780801\\
2.26404288803612	2.80020501357924\\
2.27409568351791	2.79418834106226\\
2.28354156224498	2.78768647908714\\
2.29246720103243	2.78066759198863\\
2.30094324708977	2.77308369344406\\
2.30902653528962	2.76486947593528\\
2.31676141495498	2.75594002685406\\
2.32418023496131	2.74618726101363\\
2.33130293876362	2.73547475649129\\
2.33813560686435	2.72363049229966\\
2.3446676308791	2.71043671862902\\
2.35086697694649	2.69561579141908\\
2.35667264118912	2.67881018875666\\
2.36198282021328	2.65955395646265\\
2.36663634360813	2.63723126333636\\
2.37038322814755	2.61101516183607\\
2.37283722330557	2.57977529934503\\
2.37339778982183	2.54193586381645\\
2.37111887335268	2.49525205412532\\
2.36448275111581	2.43645056417114\\
2.3510006908997	2.36063993576219\\
2.32649277166447	2.26033118819243\\
2.2837747325805	2.12381882607868\\
2.21029948587487	1.93263172647691\\
2.084263967828	1.65818455267862\\
1.86994484410035	1.26020846482804\\
1.5193384815145	0.697504952025946\\
1.0006983097713	-0.0289546968303891\\
0.360049021024791	-0.817323633554986\\
-0.272206581709307	-1.50322890734363\\
-0.787566691977589	-1.99637801408196\\
-1.16381823972836	-2.31343899538887\\
-1.42823676075027	-2.50899376903536\\
-1.61471437745805	-2.62936567702651\\
-1.74911207095017	-2.70446720841903\\
-1.84867712973555	-2.75206417414387\\
-1.92451708705135	-2.78255795851524\\
-1.98380493175901	-2.80211837265008\\
-2.03125468735656	-2.81449094713054\\
-2.07003569180899	-2.82200876941633\\
-2.10233018110642	-2.82616180320071\\
-2.12967588442384	-2.82792326372033\\
-2.1531801245671	-2.82794119873254\\
-2.17365639415808	-2.82665375450862\\
-2.1917133623488	-2.82436019316507\\
-2.20781410744761	-2.82126560905936\\
-2.22231632455808	-2.81750962028811\\
-2.23550012474565	-2.81318505342203\\
-2.24758757759959	-2.808350221096\\
-2.25875665118115	-2.80303698494289\\
-2.26915127475398	-2.7972559568432\\
-2.27888866191015	-2.79099967643615\\
-2.2880646510665	-2.7842442757679\\
-2.29675756692326	-2.77694992385298\\
-2.30503093126348	-2.76906018715897\\
-2.31293522298438	-2.76050031582731\\
-2.32050878383483	-2.75117434743358\\
-2.32777787103906	-2.74096079067564\\
-2.33475575469844	-2.72970648920066\\
-2.34144062746594	-2.7172180427969\\
-2.34781190937511	-2.70324983788496\\
-2.3538242488797	-2.68748724591913\\
-2.35939806941366	-2.66952277861396\\
-2.3644047611806	-2.64882175930822\\
-2.36864333865978	-2.62467206252469\\
-2.37180314504003	-2.59610912865902\\
-2.37340316803587	-2.56180177567943\\
-2.37269115478618	-2.51987449893684\\
-2.36847187271704	-2.46762474132698\\
-2.35880747173788	-2.40106346667514\\
-2.34048239923269	-2.31415587911096\\
-2.30803121233605	-2.19755920777985\\
-2.25197055349214	-2.03656905299043\\
-2.15571304768967	-1.80808100704446\\
-1.99098591135525	-1.47748835353313\\
-1.71468104540268	-1.00116744224884\\
-1.28068252930592	-0.351115377356678\\
-0.68853915531236	0.425809568332435\\
-0.0341533716274803	1.18172463156856\\
0.547734367825091	1.77464307462038\\
0.991880077971176	2.17360147631448\\
1.30776112775099	2.42310769102853\\
1.52937635180833	2.57635806517277\\
1.6871706168914	2.67124391386155\\
1.80244009974611	2.73092572845202\\
1.88904281906363	2.76899036570571\\
1.95589065974379	2.79342860115457\\
2.00878314321231	2.8090324361412\\
2.05157406424349	2.81874849587472\\
2.08688552050022	2.824435670925\\
2.11654422500564	2.8272949813574\\
2.14185167975353	2.82811886201345\\
};
\addplot [color=mycolor1, forget plot]
  table[row sep=crcr]{%
2.10439665617207	2.8282298624729\\
2.12464188065137	2.82760148176032\\
2.14243234288805	2.82591382277774\\
2.15824659004586	2.82339212332122\\
2.17245231744745	2.82019056110771\\
2.18533615434218	2.8164130354986\\
2.19712446440675	2.81212678727802\\
2.2079981010471	2.80737135710589\\
2.21810301410286	2.80216441784895\\
2.22755795084046	2.79650543077886\\
2.23646007529932	2.79037770929812\\
2.24488905472067	2.78374923438941\\
2.25290997458345	2.77657239889961\\
2.26057530994044	2.76878272888608\\
2.26792607712629	2.76029651585636\\
2.27499219932393	2.75100717441069\\
2.2817920267666	2.74077999580701\\
2.28833084080954	2.7294447746609\\
2.29459801820751	2.7167855079095\\
2.30056230294151	2.70252594691874\\
2.306164269904	2.6863091344892\\
2.31130446643525	2.66766802527548\\
2.31582470136927	2.64598260556978\\
2.31947817809422	2.62041612789413\\
2.32188099444431	2.58981831435422\\
2.32243170733342	2.55257512298983\\
2.32017470859998	2.50637009777338\\
2.31356213260709	2.44779635654283\\
2.30002812104162	2.37171232033471\\
2.2752102358232	2.27015675826242\\
2.23150840732329	2.13052870772527\\
2.15545993513703	1.93268391936682\\
2.02337848732076	1.6451206627233\\
1.79632734061758	1.22354909389806\\
1.42353359793657	0.625220579339345\\
0.878771555207121	-0.138018972691501\\
0.227351529764195	-0.940000491971334\\
-0.388387094450216	-1.60829303205258\\
-0.871991154686601	-2.07120251933972\\
-1.21698954120204	-2.36197680636123\\
-1.45675809677698	-2.53931410504228\\
-1.62520636758421	-2.64804972105209\\
-1.74661180451078	-2.71589014876617\\
-1.83671977780629	-2.75896476121326\\
-1.90554053800263	-2.78663504150652\\
-1.95950195902525	-2.80443718838813\\
-2.00282013971824	-2.8157317148965\\
-2.03832968303853	-2.82261478727292\\
-2.06798441766638	-2.8264279081047\\
-2.09316335521594	-2.82804944289212\\
-2.1148614193132	-2.8280657096879\\
-2.13381102198757	-2.8268740115869\\
-2.15056131628629	-2.82474621291356\\
-2.16553098346993	-2.82186884012359\\
-2.17904410894577	-2.81836885520921\\
-2.19135502121799	-2.81433046201439\\
-2.20266577905565	-2.80980615459494\\
-2.21313866319811	-2.80482396386185\\
-2.22290520522737	-2.79939210980444\\
-2.23207276452311	-2.79350180566567\\
-2.24072932584376	-2.78712866617418\\
-2.24894696417187	-2.7802329739112\\
-2.25678426640306	-2.77275891335017\\
-2.26428788314733	-2.76463276304384\\
-2.27149328874582	-2.75575992192013\\
-2.27842473786471	-2.7460205166516\\
-2.28509430723748	-2.73526317200658\\
-2.2914997822963	-2.7232962958047\\
-2.29762096294645	-2.70987589060062\\
-2.3034136760499	-2.69468838494862\\
-2.30880031805775	-2.67732616054952\\
-2.3136549741375	-2.6572521367095\\
-2.31777982205	-2.633747608672\\
-2.32086716431094	-2.60583389424976\\
-2.32243714590556	-2.57215208683478\\
-2.32173325091376	-2.53077426119804\\
-2.31754253154096	-2.47890003515275\\
-2.30787822319011	-2.41235776020951\\
-2.28940538814651	-2.32476930466821\\
-2.25638205356715	-2.20614149639993\\
-2.19870458744134	-2.04054099553217\\
-2.09845541803663	-1.80262170813905\\
-1.92480803984779	-1.45417862427278\\
-1.63113994870503	-0.947953115606942\\
-1.1713375961867	-0.259161487476996\\
-0.557980698113119	0.545889993495298\\
0.0933622157606264	1.29865044866206\\
0.648758293724474	1.86480006513732\\
1.06001811194664	2.23430726860776\\
1.34770707165212	2.46157149408546\\
1.5481469960902	2.60018432009282\\
1.69063965301853	2.68586874206893\\
1.79484746233692	2.73982193018783\\
1.87332468029321	2.77431381426718\\
1.93407508714821	2.79652186191819\\
1.98228898802755	2.81074462611771\\
2.02141253032432	2.81962733025033\\
2.05379189979266	2.82484177055891\\
2.08106385814126	2.82747058619653\\
2.10439665617207	2.8282298624729\\
};
\addplot [color=mycolor1, forget plot]
  table[row sep=crcr]{%
2.06086420360006	2.82606288482026\\
2.07954946709803	2.82548265069804\\
2.09601333227688	2.82392060654805\\
2.11068597508046	2.82158074459897\\
2.12389875922843	2.81860278558362\\
2.13591063685797	2.81508075559912\\
2.14692662145578	2.81107517523556\\
2.15711091717045	2.8066210743889\\
2.16659637331059	2.80173319287336\\
2.17549135875597	2.79640920788889\\
2.18388478249776	2.79063150275514\\
2.19184974351591	2.78436777594098\\
2.19944612711559	2.77757063694663\\
2.20672234502208	2.77017621530973\\
2.21371632230606	2.76210169905723\\
2.22045574990295	2.75324160072957\\
2.22695753276466	2.74346240291337\\
2.23322625410922	2.73259503550656\\
2.23925132297113	2.72042434612167\\
2.24500223933518	2.70667428353889\\
2.25042103772944	2.69098682330781\\
2.25541034879725	2.67289155617102\\
2.25981445309361	2.65176103973526\\
2.26338882491472	2.62674395617998\\
2.26575027047244	2.59666286660416\\
2.26629346484606	2.55985413589397\\
2.26404769597393	2.51391112540796\\
2.2574242625088	2.45526195276396\\
2.24375881457129	2.37845948752776\\
2.21846221176634	2.274969050451\\
2.17342512124737	2.13110596293197\\
2.09407399513783	1.92471095921482\\
1.95447412087448	1.62083113027563\\
1.71205614108905	1.17076849580878\\
1.31391113289961	0.531696757122038\\
0.743197812322143	-0.268179794547498\\
0.0880735063622423	-1.07514858084357\\
-0.502153092938349	-1.71605812303739\\
-0.948894897869495	-2.14381008954486\\
-1.26111566886754	-2.40699538801196\\
-1.47628697745874	-2.56614725303258\\
-1.62720510986181	-2.66356637669721\\
-1.73615321181851	-2.72444379483013\\
-1.81726335384155	-2.76321533960986\\
-1.87943510189821	-2.78821089182298\\
-1.92836393046319	-2.8043517092476\\
-1.96778443891464	-2.81462919399292\\
-2.00021066507747	-2.82091400872068\\
-2.02737901295415	-2.82440695845339\\
-2.05051798869225	-2.8258967520207\\
-2.07051619978823	-2.82591144571268\\
-2.08802949379759	-2.82480982424699\\
-2.1035508863799	-2.82283791974444\\
-2.11745721549868	-2.82016474710658\\
-2.13004090895075	-2.81690532193666\\
-2.14153202214416	-2.81313569461374\\
-2.15211378353517	-2.80890283901094\\
-2.16193371986965	-2.80423112841719\\
-2.17111171051065	-2.79912646820911\\
-2.17974586158463	-2.79357874503894\\
-2.18791679254925	-2.78756298892149\\
-2.19569072795659	-2.78103946535897\\
-2.20312164729712	-2.77395278139286\\
-2.21025264096103	-2.76622997677119\\
-2.21711653288359	-2.75777745964334\\
-2.22373574576197	-2.7484765167114\\
-2.23012128791669	-2.73817695858281\\
-2.23627061292907	-2.72668822158121\\
-2.24216391578484	-2.71376689048698\\
-2.24775813592976	-2.6990990562707\\
-2.25297745790512	-2.68227505042866\\
-2.2576982894797	-2.66275268114385\\
-2.26172528803559	-2.63980274396403\\
-2.26475249081184	-2.61242658332343\\
-2.26629899592143	-2.57922854142314\\
-2.26559997429051	-2.53821383269632\\
-2.26141709345376	-2.48646024070003\\
-2.25169962860753	-2.41957196711612\\
-2.23296265066169	-2.33075292350808\\
-2.19912348829124	-2.20922159971954\\
-2.13932243627577	-2.03756007410029\\
-2.03403793857676	-1.78773830321613\\
-1.8494498252271	-1.41739349238135\\
-1.53527988504902	-0.875835879740883\\
-1.04748582114543	-0.144956876075536\\
-0.416552666778595	0.683547362854895\\
0.222686672033858	1.42271253383222\\
0.744243343940997	1.95457610884938\\
1.11956379822977	2.29186389120879\\
1.37853086268745	2.49645631370144\\
1.55815545418005	2.62067636550857\\
1.68588914043314	2.69748424140915\\
1.77953750501375	2.74596836911959\\
1.85030312395161	2.77706928414767\\
1.90528603804559	2.7971677366425\\
1.94908325868717	2.81008670307786\\
1.9847488763622	2.81818362897938\\
2.0143657302615	2.82295266501718\\
2.03939014961669	2.82536442418953\\
2.06086420360006	2.82606288482026\\
};
\addplot [color=mycolor1, forget plot]
  table[row sep=crcr]{%
2.00891888614438	2.82078560705859\\
2.02615105351259	2.82025020819442\\
2.04138134146949	2.81880495921137\\
2.05499436364695	2.81663386540661\\
2.06728730449655	2.81386303773614\\
2.07849310905539	2.81057718774165\\
2.0887967526755	2.8068304663362\\
2.09834683370745	2.80265358719083\\
2.1072639405311	2.79805842886958\\
2.11564674666219	2.79304085349802\\
2.12357646721724	2.78758219053757\\
2.13112009777789	2.78164964140228\\
2.13833271062673	2.77519572165534\\
2.14525897670313	2.76815674485985\\
2.15193399614463	2.7604502456637\\
2.15838344164903	2.75197112141917\\
2.1646229334839	2.74258612157438\\
2.1706564564729	2.73212610543142\\
2.17647347428268	2.72037518130714\\
2.18204415717123	2.70705536878776\\
2.18731175146273	2.69180468204638\\
2.19218046700744	2.67414532896778\\
2.19649613010137	2.65343672764823\\
2.2000148408192	2.62880466190764\\
2.20235120160236	2.59903203066476\\
2.2028907851417	2.56238623040231\\
2.20063819958206	2.51633934780944\\
2.19394581432378	2.45710273878048\\
2.18001545400075	2.37883434803851\\
2.15396010897569	2.27226688933048\\
2.10701532869381	2.12234499819615\\
2.02320496199439	1.90439750994316\\
1.87384438654784	1.57932410863569\\
1.61233813108486	1.0938464330092\\
1.18524109310871	0.408176807620506\\
0.590973948330326	-0.425113895273302\\
-0.0569771957156509	-1.22374196019026\\
-0.610890663283743	-1.82551182531642\\
-1.01557587894421	-2.21309612205472\\
-1.29369663941072	-2.44756019768786\\
-1.48442922186893	-2.58863746352841\\
-1.61834726432864	-2.67508007316861\\
-1.71537580525566	-2.72929425484456\\
-1.78794419320833	-2.76398050957151\\
-1.84383522588946	-2.78644933724247\\
-1.8880263018093	-2.80102607714932\\
-1.92378673181909	-2.81034847094563\\
-1.95332352668201	-2.81607261785346\\
-1.97816604215308	-2.81926604942691\\
-1.99940001727583	-2.8206328007354\\
-2.01781348573809	-2.82064601381138\\
-2.03399001271016	-2.81962821404895\\
-2.04836970007472	-2.81780113109942\\
-2.06128999748746	-2.81531730242975\\
-2.07301356805585	-2.81228048793155\\
-2.08374767205679	-2.8087590311147\\
-2.09365787595991	-2.80479465302263\\
-2.10287788718483	-2.80040819927139\\
-2.11151668915681	-2.79560327921136\\
-2.11966375314433	-2.79036837479183\\
-2.1273928436062	-2.78467776236583\\
-2.1347647587155	-2.77849142890533\\
-2.14182922407171	-2.77175404110533\\
-2.1486260635108	-2.76439291855656\\
-2.15518569053768	-2.75631485205166\\
-2.16152888367223	-2.7474014767175\\
-2.16766571440516	-2.73750273431448\\
-2.17359336831123	-2.72642770719064\\
-2.17929240878103	-2.71393172718295\\
-2.18472072980177	-2.69969807263371\\
-2.18980394329585	-2.68331162355013\\
-2.19442009162995	-2.66422030083433\\
-2.19837507527803	-2.64167752756128\\
-2.20136247904574	-2.61465451014093\\
-2.20289646410985	-2.58170334133977\\
-2.20219684149331	-2.54073794130088\\
-2.19798678378918	-2.48867432855378\\
-2.18812642630346	-2.42082483760648\\
-2.16893087296831	-2.32985653195647\\
-2.1338749289035	-2.20398573926515\\
-2.07113493439352	-2.02392816627601\\
-1.95918946251008	-1.75835244731557\\
-1.76067650217052	-1.36011692833705\\
-1.4218821918633	-0.776081944412222\\
-0.904606984461733	-0.000769264299073928\\
-0.26323362215036	0.841939097541932\\
0.351784892089782	1.55350881341773\\
0.831431807622443	2.04281570297915\\
1.1679223194889	2.34525963145284\\
1.39779849369872	2.52687794682429\\
1.55702877496454	2.63699263458927\\
1.67055879911057	2.70525662425323\\
1.7541482407559	2.7485304122958\\
1.8176132090598	2.77642083912154\\
1.86715809602388	2.79453009809739\\
1.90680295862112	2.80622323505773\\
1.93922490440843	2.81358303291265\\
1.96625534626517	2.81793503296059\\
1.98917915602654	2.82014390649056\\
2.00891888614438	2.82078560705859\\
};
\addplot [color=mycolor1, forget plot]
  table[row sep=crcr]{%
1.94484137383948	2.81093474107766\\
1.96074547841375	2.81044028934514\\
1.97485329975312	2.80910128846411\\
1.9875066311284	2.80708302175642\\
1.99897060680077	2.80449883772538\\
2.00945384083686	2.80142467844377\\
2.01912260826149	2.79790864308162\\
2.02811098393352	2.79397726781706\\
2.03652818247299	2.78963955849741\\
2.04446391838361	2.78488941551565\\
2.0519923310893	2.77970683640786\\
2.05917483669373	2.7740581098061\\
2.06606214117594	2.76789508755222\\
2.07269555549188	2.76115351547\\
2.07910767558369	2.7537502987825\\
2.08532241656818	2.74557945784403\\
2.0913543072874	2.73650637307818\\
2.09720684250933	2.72635969607218\\
2.10286953093216	2.71491997250002\\
2.10831302764683	2.7019035098721\\
2.11348133002652	2.68693920708129\\
2.11827931989926	2.66953473105818\\
2.12255271702205	2.64902619986201\\
2.12605531829587	2.6245017198388\\
2.12839434568635	2.59468243768769\\
2.12893701888778	2.55773275922355\\
2.12664639890571	2.51094936877903\\
2.11978435331607	2.45023777811195\\
2.10535805591883	2.36920952937128\\
2.07806245337108	2.25760066386937\\
2.0282387279616	2.09852454951016\\
1.93805368929678	1.8640500671643\\
1.77538712578065	1.51006526183966\\
1.48949970851246	0.979304317245783\\
1.03007431292167	0.241476455505507\\
0.419288702040341	-0.615550762137571\\
-0.205214267148519	-1.3858245453107\\
-0.710233576241256	-1.934731191373\\
-1.06777672003211	-2.27723504612438\\
-1.31073289439563	-2.48206333033099\\
-1.47730305556258	-2.6052645536573\\
-1.59479054014759	-2.68109634887966\\
-1.68045100632792	-2.72895477760171\\
-1.74494270113762	-2.75977779842846\\
-1.79493183413157	-2.77987210230925\\
-1.83469310631565	-2.79298630915439\\
-1.86704631039807	-2.80141951316405\\
-1.8939042740545	-2.80662379841013\\
-1.91659908828351	-2.80954060534081\\
-1.93608103539775	-2.81079415696333\\
-1.95304303474956	-2.81080598165623\\
-1.96800047775809	-2.80986459578593\\
-1.98134368007364	-2.80816896161207\\
-1.9933731150669	-2.80585617766109\\
-2.00432356697461	-2.80301943279653\\
-2.01438099583165	-2.79971978771933\\
-2.02369450740841	-2.79599393431396\\
-2.03238496811591	-2.79185925149242\\
-2.04055127247485	-2.78731697238645\\
-2.04827493068313	-2.78235396204598\\
-2.05562342072536	-2.77694339766353\\
-2.06265259796019	-2.77104449764707\\
-2.06940834678536	-2.76460133187308\\
-2.07592757486766	-2.75754064221577\\
-2.08223857638946	-2.74976849232457\\
-2.08836071411339	-2.74116543002282\\
-2.0943032764969	-2.73157966067934\\
-2.10006323587494	-2.7208174601858\\
-2.10562143584944	-2.70862964564853\\
-2.11093641784446	-2.69469227718988\\
-2.11593456480623	-2.67857872496947\\
-2.12049432251631	-2.65971851899145\\
-2.12442063129081	-2.63733549520008\\
-2.12740273215173	-2.61035271796791\\
-2.12894294327842	-2.57724272324503\\
-2.12823326147808	-2.53578540274212\\
-2.12393536894113	-2.4826658651701\\
-2.11377659503182	-2.41278885724156\\
-2.09378679643449	-2.31808502786372\\
-2.05682850107485	-2.18541886954297\\
-1.98978033523441	-1.99304251906782\\
-1.86853263904077	-1.70545123782526\\
-1.65156467308278	-1.27022149623518\\
-1.28293800086662	-0.634646104230499\\
-0.737052344015008	0.18398021550805\\
-0.098257033791466	1.02392006128289\\
0.476762529669879	1.68961487921372\\
0.905933475822523	2.12757086932723\\
1.20098228351461	2.39279553822506\\
1.40158390766722	2.55128457826433\\
1.54091048947349	2.64762983803274\\
1.64081458187349	2.70769620331022\\
1.71485558795971	2.74602353869501\\
1.77144028197587	2.77088807459421\\
1.81588842973417	2.7871328303387\\
1.85165952558558	2.79768228501973\\
1.88106812247197	2.80435722786817\\
1.90570546513969	2.80832331021653\\
1.92669343243673	2.81034517052612\\
1.94484137383948	2.81093474107766\\
};
\addplot [color=mycolor1, forget plot]
  table[row sep=crcr]{%
1.86245943044403	2.79384728481862\\
1.87719543502277	2.79338878598613\\
1.89032623322453	2.79214220986903\\
1.90215354364192	2.79025543168445\\
1.91291262912871	2.78782990592033\\
1.92278952809222	2.78493333572121\\
1.93193325090473	2.78160803032954\\
1.94046454086296	2.77787638382461\\
1.94848224298122	2.77374436072882\\
1.9560679706476	2.7692035335771\\
1.96328952997607	2.76423199665512\\
1.97020340681034	2.75879432743904\\
1.97685651220648	2.75284065093106\\
1.9832872994753	2.74630476024209\\
1.98952629549268	2.73910114180536\\
1.9955960190075	2.73112062779437\\
2.00151017625338	2.72222422964287\\
2.00727191226183	2.71223446324973\\
2.01287072863135	2.7009231085034\\
2.01827741105682	2.68799377045739\\
2.02343586479672	2.67305668527072\\
2.0282499908962	2.65559169200297\\
2.03256238150448	2.63489272300859\\
2.03611914740736	2.60998272367851\\
2.03851057489655	2.57948003739637\\
2.03906841393883	2.54138299418465\\
2.03668297320485	2.49271294721524\\
2.02946742013352	2.42890633522266\\
2.01412311941155	2.34275423411632\\
1.98471102608089	2.22252873580691\\
1.93026212035532	2.04873132203026\\
1.83035376990038	1.78902783774467\\
1.64857280295573	1.39347046401649\\
1.33103518000575	0.803821593145323\\
0.838526356380404	0.0123651422389971\\
0.227394302186296	-0.845931946906854\\
-0.350442449398308	-1.55919759768021\\
-0.792727001687744	-2.04011003837801\\
-1.09849427935741	-2.33304905062391\\
-1.30556878808597	-2.50762108621555\\
-1.44839577807619	-2.61325125302515\\
-1.5500787806155	-2.67887462553143\\
-1.62495459176856	-2.72070262573729\\
-1.68186320611161	-2.74789805962356\\
-1.72635986223458	-2.76578228194085\\
-1.76203238599039	-2.77754636466723\\
-1.79126603796787	-2.78516531107247\\
-1.81569144812738	-2.7898974168338\\
-1.83645257630568	-2.79256507392194\\
-1.85437113561667	-2.79371753788091\\
-1.87005014771785	-2.7937280692581\\
-1.88394082426028	-2.79285349083442\\
-1.89638679433492	-2.79127159116071\\
-1.9076539962759	-2.78910510451997\\
-1.9179512873667	-2.78643733752859\\
-1.92744491188462	-2.78332245749679\\
-1.93626882061528	-2.77979227183243\\
-1.9445321310142	-2.77586062439388\\
-1.95232457527549	-2.77152610482186\\
-1.9597204993907	-2.76677349436795\\
-1.96678178833739	-2.76157418987653\\
-1.97355996345272	-2.75588571626226\\
-1.98009760410597	-2.74965033109007\\
-1.98642917078006	-2.74279262357416\\
-1.99258123797982	-2.73521589730112\\
-1.99857207095989	-2.72679698194947\\
-2.00441038565776	-2.7173789181426\\
-2.01009299557332	-2.70676066179558\\
-2.01560083891203	-2.69468249590454\\
-2.02089253581957	-2.68080511086198\\
-2.02589404403578	-2.66467913195244\\
-2.03048196669752	-2.64569990157713\\
-2.03445624511432	-2.62303895654064\\
-2.03749460805865	-2.59553774446397\\
-2.03907476422507	-2.56153854272313\\
-2.0383378448221	-2.51860812301941\\
-2.03384155505112	-2.46307344849564\\
-2.02310021273392	-2.38922080245886\\
-2.00170367524217	-2.28788674924078\\
-1.96160130236399	-2.14397601173785\\
-1.88781061634304	-1.93230359683834\\
-1.75275925925105	-1.61201577623333\\
-1.51037323548226	-1.12576982920136\\
-1.10632840669905	-0.428838530197344\\
-0.539072435537281	0.422527341385257\\
0.0748267257916535	1.23045553389087\\
0.590332220910206	1.82760354405671\\
0.960489345185461	2.20542733579409\\
1.21194722364818	2.43147322229943\\
1.38332012285093	2.56686017357665\\
1.50332175285635	2.64983329006234\\
1.59021747160967	2.70207230964653\\
1.65524948912868	2.73573203272077\\
1.70540363675527	2.75776811117796\\
1.74512825602826	2.77228466203469\\
1.77733825522655	2.78178258186608\\
1.80399917194136	2.7878329240668\\
1.82647257532337	2.79144994386784\\
1.84572528341125	2.79330408577433\\
1.86245943044403	2.79384728481862\\
};
\addplot [color=mycolor1, forget plot]
  table[row sep=crcr]{%
1.75108552716988	2.76447222447479\\
1.76488221422702	2.76404250582548\\
1.77724897716815	2.7628680858765\\
1.78845029864056	2.76108084132467\\
1.79869390550384	2.75877123151757\\
1.80814521289382	2.7559992047306\\
1.81693762491815	2.75280140880389\\
1.8251799843703	2.7491959003046\\
1.83296202077276	2.74518509199559\\
1.84035836173541	2.74075739148653\\
1.84743148519643	2.73588779376549\\
1.85423386213361	2.73053755456242\\
1.86080944715795	2.72465296307848\\
1.86719460207298	2.7181631319376\\
1.8734184728415	2.71097661217749\\
1.87950277219999	2.70297650298354\\
1.88546083530285	2.69401353487635\\
1.89129569572855	2.68389632399633\\
1.89699674425462	2.67237756415484\\
1.90253423198378	2.65913424202541\\
1.90785037105443	2.64373885549779\\
1.912844900282	2.62561677736276\\
1.91735139494529	2.60398177517081\\
1.92109767190124	2.57773622969796\\
1.92363808680396	2.54531280800104\\
1.92423467860873	2.50441642898222\\
1.92164237946125	2.45159196649833\\
1.91370900678464	2.38148059663562\\
1.89660921227259	2.28551375431683\\
1.863350321132	2.14961015011902\\
1.80088724098503	1.95028198099332\\
1.68504321311595	1.64918888648009\\
1.47455060025472	1.19108672751144\\
1.1172436115952	0.527194676546508\\
0.600838439980181	-0.303565110777849\\
0.0217110314597016	-1.11787439832514\\
-0.479295792600991	-1.73676848784417\\
-0.845266816030316	-2.13480615233511\\
-1.09566377390525	-2.37469239957717\\
-1.26664301717949	-2.51881458141324\\
-1.38634200994895	-2.60732445286478\\
-1.47293770426302	-2.66320074056089\\
-1.53767767061608	-2.69936009220179\\
-1.58756039597431	-2.72319384635182\\
-1.62704090309671	-2.73905920237903\\
-1.65903569364839	-2.74960854146887\\
-1.68550910797049	-2.7565067516797\\
-1.70782031556884	-2.76082825634916\\
-1.72693340842951	-2.76328338763886\\
-1.74354789771438	-2.76435137514603\\
-1.75818186757464	-2.76436071503665\\
-1.77122627072982	-2.76353901027093\\
-1.78298119838695	-2.76204459183504\\
-1.79368062347662	-2.75998696923435\\
-1.80350960739666	-2.75744024887193\\
-1.81261647449423	-2.75445200543617\\
-1.82112155809316	-2.75104912596252\\
-1.82912356382167	-2.74724156652161\\
-1.83670424214872	-2.74302460222861\\
-1.84393183205323	-2.73837991938817\\
-1.85086358368856	-2.73327573973745\\
-1.85754756019065	-2.72766604744722\\
-1.86402383820955	-2.72148888718672\\
-1.87032515965923	-2.71466359834352\\
-1.87647702221151	-2.70708672901466\\
-1.8824971214059	-2.69862621252646\\
-1.88839395763576	-2.68911315896781\\
-1.89416427351294	-2.67833026766958\\
-1.89978875254881	-2.66599532654949\\
-1.90522502065326	-2.651737399139\\
-1.9103963233033	-2.63506187913752\\
-1.91517306923613	-2.61529820075033\\
-1.91934228271213	-2.59151986780491\\
-1.92255598803098	-2.56241917110889\\
-1.92424181720092	-2.52610575635664\\
-1.92344382818246	-2.47977378495112\\
-1.91853047983872	-2.4191366990536\\
-1.90664281269112	-2.33744360035013\\
-1.88262584056747	-2.2237426721095\\
-1.8369431681246	-2.05985534486197\\
-1.75176729647091	-1.81557078204867\\
-1.5949202394313	-1.44359327983578\\
-1.31697355185953	-0.885815607926106\\
-0.875685625693936	-0.123991058980787\\
-0.309394011606504	0.726956426715789\\
0.24468052449709	1.4568868622189\\
0.678903972736431	1.96012466793852\\
0.982576432505579	2.27011531894379\\
1.18906142341872	2.45571542385598\\
1.33156579852306	2.56827739628308\\
1.43294892172296	2.63836437217429\\
1.50752689674116	2.68319025734382\\
1.5641525322609	2.7124937892268\\
1.60839096210157	2.73192732354029\\
1.64383384795358	2.74487692142254\\
1.67286639677993	2.75343629233736\\
1.6971172922832	2.75893855142175\\
1.71772785572109	2.76225488140204\\
1.73551709087349	2.76396740628574\\
1.75108552716988	2.76447222447479\\
};
\addplot [color=mycolor1, forget plot]
  table[row sep=crcr]{%
1.59153714869271	2.71283556446399\\
1.60476885427873	2.71242284118941\\
1.61672829479412	2.71128658950149\\
1.62764574399131	2.70954418885353\\
1.63770404439344	2.70727595560098\\
1.64705034381964	2.70453435482357\\
1.65580456300521	2.7013500981366\\
1.66406558417514	2.69773608650507\\
1.67191581871882	2.69368979286982\\
1.67942459554526	2.68919444322856\\
1.68665066610256	2.68421919232426\\
1.69364402000437	2.67871836758366\\
1.70044712881062	2.67262974952996\\
1.7070956719489	2.66587175041363\\
1.71361873731923	2.65833922744017\\
1.72003841862168	2.64989750103547\\
1.7263686373693	2.64037391124481\\
1.73261287780803	2.6295458885489\\
1.73876030068066	2.61712396012184\\
1.74477933195565	2.60272722453063\\
1.75060718655548	2.58584737245532\\
1.75613266133152	2.56579488854982\\
1.76116748404913	2.54161687298595\\
1.7653976787231	2.51196854115029\\
1.76829905973407	2.4749071956872\\
1.76898648679696	2.42755320362948\\
1.76593736086363	2.36551782061183\\
1.75647060848263	2.28191688309793\\
1.73574466156984	2.16565555246177\\
1.69482869501793	1.99851591129642\\
1.61718261347504	1.75076654277894\\
1.47352994683164	1.37732437680515\\
1.22051805952349	0.826298262825352\\
0.824104530899889	0.088737425767855\\
0.320869760337405	-0.722231139262628\\
-0.172427576851556	-1.41677043474581\\
-0.564628150823845	-1.90152232511448\\
-0.844164857945204	-2.20555524352634\\
-1.03761732801986	-2.39084560899199\\
-1.17307874967628	-2.50499544609819\\
-1.27057497062747	-2.57706625586182\\
-1.34296497124043	-2.62376301758625\\
-1.39835026011063	-2.6546892307529\\
-1.44189706837267	-2.67549040360413\\
-1.47697734489243	-2.68958397191685\\
-1.50585079022601	-2.69910168381581\\
-1.53007197621501	-2.70541126204121\\
-1.5507371833947	-2.70941263399981\\
-1.56863738283019	-2.71171095290876\\
-1.58435533435496	-2.71272050602903\\
-1.59832868964385	-2.71272876601862\\
-1.61089193830001	-2.71193681655228\\
-1.6223048854811	-2.7104853955052\\
-1.6327723668004	-2.70847195331515\\
-1.64245814281193	-2.705961951801\\
-1.65149485039195	-2.70299636918225\\
-1.65999123121105	-2.6995966271188\\
-1.66803744320138	-2.69576769607146\\
-1.67570899375596	-2.69149984374145\\
-1.6830696565729	-2.68676929722642\\
-1.69017361294467	-2.68153795039362\\
-1.69706697091202	-2.67575213645695\\
-1.70378874713416	-2.6693403819491\\
-1.71037133513253	-2.66220994472263\\
-1.71684041912074	-2.65424179609826\\
-1.72321421250686	-2.64528351033067\\
-1.72950178671067	-2.63513923500997\\
-1.7357000808777	-2.62355547256329\\
-1.74178889769397	-2.61020070304512\\
-1.74772270731284	-2.59463574492772\\
-1.75341723808441	-2.57626987072965\\
-1.75872731967868	-2.55429450216108\\
-1.76340965469847	-2.52758076017022\\
-1.76705891020817	-2.4945172764498\\
-1.7689952370932	-2.45274676514304\\
-1.76806082381912	-2.39872691925673\\
-1.76224153191576	-2.32698079683655\\
-1.74794563915439	-2.22879585130807\\
-1.71861081452158	-2.08997419296628\\
-1.66206523014974	-1.88716184619098\\
-1.55606064235132	-1.58313617346189\\
-1.36362833150289	-1.12657390203955\\
-1.04038019630243	-0.477214984779318\\
-0.579488947169715	0.319746071660618\\
-0.0653357731346624	1.09359382563594\\
0.383438388475592	1.68535391512193\\
0.717113301025199	2.07215344123786\\
0.949730301037335	2.3095776614735\\
1.11114866251162	2.45462986861789\\
1.2256305822835	2.54503016285746\\
1.30931556895403	2.60286539536645\\
1.37240768645625	2.64077716802134\\
1.42136037563239	2.66610339414253\\
1.46033399259921	2.6832198294233\\
1.49207970831041	2.69481567438568\\
1.51846579304674	2.70259274207218\\
1.54079383062352	2.70765720831968\\
1.55999247373734	2.71074520454552\\
1.57673895742309	2.71235644715024\\
1.59153714869271	2.71283556446399\\
};
\addplot [color=mycolor1, forget plot]
  table[row sep=crcr]{%
1.34916822602105	2.61884380257413\\
1.36255348536219	2.61842535800733\\
1.37480683187091	2.61726037653587\\
1.38612752561835	2.61545290337742\\
1.39667669922158	2.61307332288931\\
1.40658638148471	2.61016585503015\\
1.41596613668646	2.60675348673496\\
1.42490800439879	2.60284105382679\\
1.43349020388433	2.598416912692\\
1.44177991815768	2.59345345180439\\
1.44983536863337	2.5879065527139\\
1.45770731417958	2.58171399239607\\
1.46544004576872	2.57479266390242\\
1.47307188948927	2.56703436125339\\
1.48063516614287	2.55829970555158\\
1.48815547206729	2.54840955221251\\
1.49565002401685	2.53713286811873\\
1.50312461938875	2.52416952842272\\
1.5105684469781	2.50912563214587\\
1.51794544380942	2.49147756217942\\
1.52517994674875	2.47051874931118\\
1.53213268434057	2.4452792850012\\
1.53856001621684	2.41440198359606\\
1.5440434126235	2.37594709462455\\
1.54786479332811	2.32707784393597\\
1.54878118856405	2.26354409964061\\
1.54460916042535	2.17882346641676\\
1.53144942360156	2.06269711309166\\
1.50225566229098	1.89898924372039\\
1.44436890588002	1.66251346478269\\
1.33620870262374	1.31721418768455\\
1.14673141186483	0.824026867167908\\
0.849782608285716	0.175946554623806\\
0.461685962801752	-0.548018956645931\\
0.0565516337693074	-1.20228090782282\\
-0.290753796572131	-1.6917502022023\\
-0.554835633179842	-2.01816874854801\\
-0.746358627160081	-2.22639575742822\\
-0.884817909103233	-2.3589451739538\\
-0.986659923035106	-2.44472049770824\\
-1.0634337309332	-2.50144568703854\\
-1.12282433770986	-2.53974002418693\\
-1.16991128279668	-2.5660218412918\\
-1.20809299884232	-2.58425316968421\\
-1.23968783716858	-2.59694161257033\\
-1.26631202944042	-2.60571445312303\\
-1.28911705407786	-2.61165259675128\\
-1.30894048914882	-2.61548905733766\\
-1.32640356331209	-2.617729744945\\
-1.34197546868493	-2.61872870075034\\
-1.35601659862283	-2.61873598932488\\
-1.36880817441296	-2.61792878298912\\
-1.38057291538259	-2.61643186607089\\
-1.39148970600711	-2.61433131778502\\
-1.40170416735516	-2.6116836843858\\
-1.4113363853036	-2.608522081478\\
-1.42048663037264	-2.6048601311055\\
-1.42923963244693	-2.60069429697314\\
-1.43766779293806	-2.5960049543598\\
-1.44583359306231	-2.59075637031308\\
-1.45379136809174	-2.58489564336501\\
-1.46158854894265	-2.57835053799512\\
-1.46926641317011	-2.57102602853142\\
-1.47686032733914	-2.56279922031462\\
-1.48439939058245	-2.55351211746392\\
-1.49190528944329	-2.54296141961962\\
-1.49939002248124	-2.53088409664441\\
-1.50685190825312	-2.51693681519435\\
-1.5142688787915	-2.50066621316221\\
-1.52158734818893	-2.48146525837629\\
-1.5287036798198	-2.4585079950171\\
-1.53543297058567	-2.43064999568616\\
-1.54145557404778	-2.39627321363943\\
-1.5462236004709	-2.35303882637033\\
-1.54879377227572	-2.29748516474355\\
-1.54752207454328	-2.22436239290059\\
-1.53949639900754	-2.12552421645835\\
-1.51947921042305	-1.98811596811773\\
-1.47800070676288	-1.79185510593083\\
-1.39834292507678	-1.50606847102606\\
-1.25373377570828	-1.09096289236972\\
-1.01217443181423	-0.516876345325173\\
-0.663483593147455	0.185309718307808\\
-0.255650441715128	0.892281146234285\\
0.127141962370146	1.46932437295849\\
0.432991150703737	1.87280614464764\\
0.658442536758492	2.13409571476104\\
0.821074581862312	2.30001032816487\\
0.939487523829889	2.40636173951891\\
1.02762555404293	2.47592484154283\\
1.09493604306106	2.52242215305209\\
1.14766234075684	2.55409174160465\\
1.18995036713984	2.57596124909194\\
1.22459961632259	2.59117267921784\\
1.25354049557583	2.60173993143481\\
1.27813357584593	2.60898559194935\\
1.29935845276156	2.61379764430181\\
1.31793477108438	2.61678384887558\\
1.33440123755647	2.61836680110905\\
1.34916822602105	2.61884380257413\\
};
\addplot [color=mycolor1, forget plot]
  table[row sep=crcr]{%
0.967127731322892	2.44416335727229\\
0.982332194010056	2.44368626762445\\
0.996552044814818	2.44233274530933\\
1.00995827727741	2.44019085862995\\
1.02269424379618	2.43731666682272\\
1.03488178584442	2.43373960992044\\
1.04662592338407	2.42946583601168\\
1.05801847480111	2.42447989317166\\
1.06914086623855	2.41874501885892\\
1.08006630690967	2.41220210613007\\
1.09086144272083	2.4047672886263\\
1.10158754476246	2.39632794378619\\
1.1123012330301	2.38673674442344\\
1.12305466889795	2.3758031658969\\
1.13389505829192	2.36328154206065\\
1.1448631692255	2.34885430132771\\
1.15599034564836	2.33210831537199\\
1.16729313058477	2.31250121396592\\
1.17876398162417	2.28931282755925\\
1.19035546509962	2.26157423670833\\
1.20195337503873	2.22796263054996\\
1.21333074848429	2.18664338354513\\
1.22406849048966	2.13503020764076\\
1.23341712189669	2.06941885226787\\
1.24005475817105	1.98443112174681\\
1.24166587486424	1.87219700793212\\
1.23423019400315	1.72125667828972\\
1.21092403527738	1.51545106534366\\
1.1608076636804	1.23401181971476\\
1.06853441869398	0.856195783028522\\
0.918509277596076	0.37568184812346\\
0.706921090474597	-0.177265218634865\\
0.454153574876131	-0.731103204332548\\
0.198053204246737	-1.21018280203609\\
-0.0297170443403976	-1.57844849886536\\
-0.216534441156716	-1.84171786003203\\
-0.364040069802484	-2.02391708406028\\
-0.479373086051614	-2.14919888770883\\
-0.570079113544991	-2.23595700009679\\
-0.642374348028329	-2.29679725082808\\
-0.700942767541163	-2.34003907604897\\
-0.74920207237023	-2.37113492729172\\
-0.789628565488254	-2.39368484478053\\
-0.824025247785614	-2.41009884374503\\
-0.853718838415478	-2.42201651725366\\
-0.879698514691617	-2.43057160046117\\
-0.902712302398931	-2.43655996902277\\
-0.923334050260447	-2.44054766045806\\
-0.942010285885441	-2.44294135282445\\
-0.959093334892685	-2.44403503307794\\
-0.974865011083859	-2.44404132101184\\
-0.989553777211925	-2.44311272620087\\
-1.00334733594278	-2.44135616819522\\
-1.01640198546763	-2.43884288475897\\
-1.02884965678783	-2.43561509341293\\
-1.04080326848977	-2.43169028198276\\
-1.05236084290732	-2.42706367824267\\
-1.06360869408717	-2.42170922170841\\
-1.07462390235819	-2.41557919023216\\
-1.08547621833959	-2.40860249140423\\
-1.0962294803223	-2.40068149139744\\
-1.10694257411915	-2.39168710090024\\
-1.11766990440936	-2.38145164553381\\
-1.12846126953048	-2.36975878532062\\
-1.13936092020188	-2.35632936855015\\
-1.15040540856971	-2.34080153894434\\
-1.161619549011	-2.32270254857145\\
-1.1730093313512	-2.30140838005199\\
-1.1845497978101	-2.27608515356535\\
-1.19616443885553	-2.24560290782785\\
-1.20769006987176	-2.20840694698507\\
-1.21881648820972	-2.16232344046854\\
-1.22898182152812	-2.10426306489148\\
-1.23718963783854	-2.02976886695529\\
-1.24168907404894	-1.93233760331378\\
-1.23942408473097	-1.80245496695398\\
-1.22513423035626	-1.62642196393417\\
-1.19008816520814	-1.38559159763217\\
-1.12101043541084	-1.05813668864252\\
-1.0014442958404	-0.627979956134983\\
-0.81975968717137	-0.104520799938123\\
-0.583468119003452	0.459358633320534\\
-0.324135709273898	0.983425913460733\\
-0.0793924327840767	1.4085297153344\\
0.12836183073586	1.72185853875973\\
0.294802905494589	1.94133196694191\\
0.425234391644833	2.09237330396133\\
0.527378472679993	2.19648606699859\\
0.608201715833219	2.26901494489367\\
0.673135098941547	2.32022387243635\\
0.726187730882803	2.35684603539622\\
0.770268586484437	2.38330546508546\\
0.807488419671494	2.40254198929109\\
0.839391495499743	2.41653933886846\\
0.86712141444169	2.42665821415481\\
0.891536825910838	2.43384682854805\\
0.913291732109646	2.43877538523974\\
0.932891444261152	2.44192316766833\\
0.950731920128378	2.44363578722103\\
0.967127731322892	2.44416335727229\\
};
\addplot [color=mycolor1, forget plot]
  table[row sep=crcr]{%
0.389833257663135	2.13494205399992\\
0.411637012190412	2.13425332784037\\
0.432830583088378	2.13223175394764\\
0.453560693998461	2.12891569696232\\
0.473962470616696	2.1243075922011\\
0.494162472193208	2.11837498054376\\
0.514281289056747	2.11104967794315\\
0.534435780021275	2.10222504132626\\
0.554740988262769	2.09175113467283\\
0.575311732094167	2.07942742190971\\
0.596263810567844	2.06499239801034\\
0.61771468137522	2.04810929394353\\
0.639783341588963	2.02834662784339\\
0.662588940613527	2.00515189276218\\
0.686247331235421	1.9778160401717\\
0.71086424216981	1.94542562462237\\
0.736522914514755	1.90679856576309\\
0.763262708752509	1.86039867194929\\
0.79104312719537	1.80422397231211\\
0.819684692661886	1.73566605619677\\
0.84877425633593	1.65134548513089\\
0.8775187845775	1.54694903865653\\
0.904532894477926	1.41714080153142\\
0.92756298448584	1.25570782784642\\
0.943208032311654	1.0562361419601\\
0.946821913125114	0.813724878526878\\
0.932953901423398	0.527382672453897\\
0.896710338603812	0.203985376296863\\
0.835916667242952	-0.140357545376625\\
0.752908499578301	-0.482727521836322\\
0.654348119758859	-0.800227870027934\\
0.548859926898402	-1.07695991065846\\
0.44418277966519	-1.30673881764637\\
0.345565275630988	-1.49126309862293\\
0.255665541234909	-1.63650765373541\\
0.175238506635367	-1.74971298877467\\
0.103924297554882	-1.83768399565498\\
0.0408371288553757	-1.90612416305119\\
-0.0150784938297504	-1.95954090229694\\
-0.0648626253155773	-2.00138952748404\\
-0.109458271066525	-2.03428043107457\\
-0.149685353120126	-2.06017497680803\\
-0.186241564546573	-2.08054652619508\\
-0.219714564140881	-2.09650461534767\\
-0.250597692592461	-2.10888771948177\\
-0.279305584592668	-2.11833147055275\\
-0.306188200451515	-2.12531846439242\\
-0.331542844513849	-2.13021452029508\\
-0.355624209521529	-2.13329503609864\\
-0.378652673165659	-2.13476408783637\\
-0.400821124569248	-2.13476816612125\\
-0.422300589621874	-2.13340588180217\\
-0.443244891729413	-2.13073456289361\\
-0.463794545707637	-2.12677436071603\\
-0.484080044336877	-2.12151025149346\\
-0.50422466156071	-2.11489213373237\\
-0.524346862829057	-2.10683306033622\\
-0.544562379318148	-2.09720548953563\\
-0.564985964787353	-2.0858352733634\\
-0.585732805647301	-2.07249290875139\\
-0.606919487137589	-2.0568813339019\\
-0.628664316710347	-2.03861923676051\\
-0.651086646328595	-2.01721842383059\\
-0.674304580568565	-1.9920532442971\\
-0.698430046329205	-1.96231935134776\\
-0.723559536426158	-1.92697821841399\\
-0.74975777688409	-1.88468292350284\\
-0.777029900022185	-1.8336801421848\\
-0.805275191803477	-1.7716840361193\\
-0.834211995258691	-1.69572212075661\\
-0.863259374614334	-1.60196627480951\\
-0.891359111601299	-1.48559354762292\\
-0.91672889412078	-1.34078686680571\\
-0.936571286668781	-1.16109958864773\\
-0.946851495651384	-0.940548679227383\\
-0.942415107275109	-0.675822351313804\\
-0.917855840029638	-0.369524113914628\\
-0.869352468419604	-0.0330937167554029\\
-0.79685460095464	0.313315627814919\\
-0.705042783184192	0.64580264069854\\
-0.601938366123092	0.944299305729285\\
-0.496026410028535	1.19775590143363\\
-0.393896012203038	1.4043430387089\\
-0.299443774043	1.56834629821078\\
-0.214272432014153	1.69666805894454\\
-0.138490527366839	1.79646760641875\\
-0.0714171469981265	1.87403677915985\\
-0.0120495608528579	1.93447222249387\\
0.0406752945867412	1.9817307191636\\
0.0877548063802227	2.01881897258306\\
0.130071524556737	2.04800035130604\\
0.168383188990257	2.07097447639075\\
0.203330610054358	2.08901939862155\\
0.235452276844371	2.10309921970234\\
0.265200305703159	2.11394374610395\\
0.292955363290979	2.12210681128536\\
0.319039707043503	2.12800878190824\\
0.34372819552645	2.13196747775331\\
0.3672574253493	2.13422062045018\\
0.389833257663135	2.13494205399992\\
};
\addplot [color=mycolor1, forget plot]
  table[row sep=crcr]{%
-0.306227207130532	1.68738862435859\\
-0.261379969916529	1.68595590112493\\
-0.214795732375831	1.68149627030184\\
-0.166246224137784	1.67371381817293\\
-0.115490049238434	1.66223302401073\\
-0.0622759237369259	1.64658738880165\\
-0.00634848347055719	1.62620670504928\\
0.0525421478533969	1.60040374645278\\
0.114625475832831	1.56836188210587\\
0.180088519800896	1.52912624559407\\
0.249044311164358	1.48160272123015\\
0.321488996788948	1.42457117256621\\
0.397246489756056	1.35672182432064\\
0.475901979302771	1.27672582252266\\
0.556730063408933	1.18335123520745\\
0.638630085725047	1.07563162763586\\
0.720089496514892	0.953082849635422\\
0.799202472199641	0.815943148058473\\
0.873769773125081	0.665385314239871\\
0.941490615215035	0.503628545179529\\
1.00022702608551	0.333879795582411\\
1.04828569470517	0.160073036160582\\
1.08464168652753	-0.0135568843967586\\
1.1090398574796	-0.182963393066682\\
1.12195182993611	-0.344713618929643\\
1.12441644185153	-0.496254799550155\\
1.11782296278774	-0.635992012894199\\
1.10369788255835	-0.763209101950142\\
1.08353585909217	-0.877896160245083\\
1.05868966724269	-0.980546613175455\\
1.03031467339685	-1.07196831526202\\
0.999354059786426	-1.1531310335292\\
0.966549613685775	-1.22505576466716\\
0.93246562408045	-1.28874165082839\\
0.897517326266407	-1.34512240587057\\
0.861998816568914	-1.39504383065693\\
0.826107890702841	-1.43925528372961\\
0.789966863130557	-1.47840971076188\\
0.753639324493781	-1.51306844756036\\
0.717143229840711	-1.54370829105481\\
0.680460870916968	-1.57072926138341\\
0.643546299255664	-1.59446211120164\\
0.606330711723059	-1.61517504828705\\
0.568726230975791	-1.6330793898466\\
0.530628433127489	-1.64833401267997\\
0.491917904745029	-1.66104853925081\\
0.45246105559447	-1.67128523096314\\
0.412110373885495	-1.67905956335708\\
0.370704287784431	-1.68433944479124\\
0.328066791675956	-1.68704301850036\\
0.28400701016003	-1.6870349644935\\
0.238318910849118	-1.68412120003892\\
0.1907814445447	-1.6780418757101\\
0.141159496489401	-1.66846259387974\\
0.0892061853211523	-1.65496386297681\\
0.0346672579765807	-1.63702898237799\\
-0.0227113930501748	-1.61403088740444\\
-0.0831717223145854	-1.58521905359259\\
-0.14692529189719	-1.54970847037251\\
-0.214126735869181	-1.50647406242591\\
-0.284836778083663	-1.45435584111197\\
-0.358973134768415	-1.39208244206276\\
-0.436249208267781	-1.31832313485091\\
-0.516103814772592	-1.23177981597386\\
-0.597630840345963	-1.13132891789912\\
-0.679525486295592	-1.01621571017297\\
-0.760071672073524	-0.886287460348263\\
-0.837198593872501	-0.742227607523675\\
-0.908626835330616	-0.585727435943091\\
-0.972101222376384	-0.419519954114775\\
-1.02567269773647	-0.247220301936271\\
-1.06796069544957	-0.0729730573829008\\
-1.09832162886203	0.0990210510858061\\
-1.11687756207078	0.264977523870333\\
-1.12440879632921	0.421882170576312\\
-1.12215751534027	0.567662524495283\\
-1.11160612860006	0.701180382220813\\
-1.09428272680449	0.822097483980406\\
-1.07162111358625	0.930680848521104\\
-1.04487938471457	1.02760274479467\\
-1.01510672826523	1.11376841378628\\
-0.983143214392898	1.19018455560286\\
-0.949638454263744	1.25786832554039\\
-0.915078603157057	1.31779009367543\\
-0.879814992991123	1.37084140862849\\
-0.844090707442856	1.41782027514089\\
-0.808063458887253	1.45942747136213\\
-0.771824346577996	1.49626934968383\\
-0.73541271568474	1.52886402113299\\
-0.698827615228074	1.55764892337876\\
-0.662036427415693	1.58298854439512\\
-0.624981212737102	1.60518158636642\\
-0.587583244066306	1.62446717822998\\
-0.549746121380202	1.64102993885522\\
-0.511357782671937	1.65500379988569\\
-0.472291663383129	1.66647454828795\\
-0.43240720889068	1.67548106428663\\
-0.391549913177206	1.68201522434357\\
-0.34955104250065	1.68602042054413\\
-0.306227207130532	1.68738862435859\\
};
\addplot [color=mycolor1, forget plot]
  table[row sep=crcr]{%
-0.788141800536172	1.24177643190908\\
-0.666143911461141	1.23782621644052\\
-0.529385546548301	1.22468264945793\\
-0.377598032925018	1.200305816704\\
-0.211503328462869	1.16270309148098\\
-0.033151035064384	1.11025323342796\\
0.153892741506608	1.04211045907784\\
0.344656280172416	0.958584774469102\\
0.533260689535979	0.861346057866915\\
0.713737270154117	0.753323480156497\\
0.880924531852972	0.63828695584622\\
1.03114656753452	0.52024313111183\\
1.16248175759686	0.402853793230899\\
1.27462492668523	0.28904266342515\\
1.36849308754741	0.180844573010999\\
1.44575813645	0.0794519643076454\\
1.50843747841042	-0.0146278478275331\\
1.55859879314411	-0.101382776585321\\
1.59817999649187	-0.181122138434487\\
1.62889940432551	-0.254339608923593\\
1.65222602201337	-0.321615605657848\\
1.66938482224438	-0.383553199496267\\
1.68137941856557	-0.440739234980044\\
1.68902122202907	-0.493722931482854\\
1.69295899129496	-0.543005874624971\\
1.69370577571634	-0.589039019145862\\
1.69166205939839	-0.632223739114816\\
1.68713488522892	-0.672915006659931\\
1.68035320404118	-0.711425502482463\\
1.67147987539793	-0.748029937218246\\
1.66062077750401	-0.782969164058048\\
1.64783144199166	-0.816453845999399\\
1.63312155709335	-0.848667545586881\\
1.61645760174872	-0.879769157777472\\
1.59776379377272	-0.909894624692575\\
1.57692146172094	-0.939157864666103\\
1.55376688433758	-0.967650822563707\\
1.52808758489613	-0.995442505897755\\
1.49961702314798	-1.02257681156117\\
1.46802760103714	-1.0490688693812\\
1.43292190168589	-1.07489952909151\\
1.39382213589897	-1.10000749601614\\
1.3501579134582	-1.12427848162136\\
1.30125274841025	-1.14753059274397\\
1.24631024376023	-1.16949507441256\\
1.18440182338321	-1.18979152402478\\
1.11445938149836	-1.20789696049985\\
1.03527852900531	-1.22310892923822\\
0.945541403358974	-1.23450458424801\\
0.843872161198516	-1.24090101572704\\
0.728942407956638	-1.24082761900118\\
0.599645483779305	-1.23252919451498\\
0.45535282880379	-1.21402729099758\\
0.296245415148313	-1.18327224528407\\
0.12367254535335	-1.13841001001318\\
-0.0595661179947407	-1.07815481856255\\
-0.249164163302196	-1.00219849022497\\
-0.439610195384136	-0.911522452056087\\
-0.624866148744066	-0.808460173525445\\
-0.799261981459283	-0.696429661152239\\
-0.958317522331549	-0.579398221318837\\
-1.099228135799	-0.461263700985717\\
-1.22091815479127	-0.345352723626748\\
-1.32375128971609	-0.23414923729855\\
-1.40907894887731	-0.129252037800879\\
-1.47879034409057	-0.0314877875681934\\
-1.53495731057438	0.0589073177118155\\
-1.57959879465035	0.142101542524296\\
-1.61454971079868	0.218510972211179\\
-1.64140474522503	0.288682572195118\\
-1.6615088143786	0.353214325527291\\
-1.67597278214956	0.412704793477366\\
-1.68570039259146	0.467723849275245\\
-1.69141815567655	0.518797624047021\\
-1.69370383155179	0.5664024541869\\
-1.69301154729384	0.61096420467578\\
-1.68969292068307	0.652860571574267\\
-1.68401425006494	0.692424841635785\\
-1.67617013167615	0.729950176179679\\
-1.66629395933414	0.765693866861409\\
-1.65446574894632	0.799881247186684\\
-1.64071766956537	0.83270908325435\\
-1.62503758428124	0.864348343181529\\
-1.60737082322744	0.894946278472889\\
-1.58762033419848	0.924627755549562\\
-1.56564528659377	0.953495759321603\\
-1.54125814301605	0.981630956651841\\
-1.51422016176231	1.00909015662244\\
-1.4842352568206	1.03590343561629\\
-1.45094212840502	1.06206960633996\\
-1.41390460262001	1.08754959936643\\
-1.37260021186315	1.11225719459006\\
-1.32640725558765	1.13604639638229\\
-1.27459098054248	1.15869461401389\\
-1.21629022788945	1.17988074243297\\
-1.15050707983079	1.19915734777188\\
-1.07610391598764	1.21591665067463\\
-0.991815076452057	1.22935121391861\\
-0.89628409556123	1.23841270101852\\
-0.788141800536172	1.24177643190908\\
};
\addplot [color=mycolor1, forget plot]
  table[row sep=crcr]{%
-0.759481808475352	0.954811303080049\\
-0.464414977518427	0.945239358918495\\
-0.137178419317819	0.913848151332419\\
0.205107567194767	0.859035732518902\\
0.540479022822526	0.783362229568471\\
0.848583269917223	0.693065650564576\\
1.11622371763869	0.595880372562254\\
1.33889721598359	0.498671849537412\\
1.51877538882173	0.406168785154704\\
1.66154340595806	0.320896882719042\\
1.7738798578266	0.243732277033466\\
1.86204322726675	0.174544792552995\\
1.93131125706806	0.1126930881872\\
1.98589505992412	0.0573370457405482\\
2.02905758504342	0.0076098029816707\\
2.06329054280205	-0.0372991895483263\\
2.09048543210269	-0.0781114817153033\\
2.11207672999923	-0.115453032641855\\
2.12915410498925	-0.149859865136441\\
2.14254744934964	-0.181788580002967\\
2.15289020636786	-0.211628032439913\\
2.16066612363249	-0.23971046643039\\
2.16624361766087	-0.266321434193131\\
2.16990095182832	-0.291708335604611\\
2.17184459428808	-0.316087645044589\\
2.17222246794996	-0.339650985413606\\
2.17113330922803	-0.362570232747666\\
2.16863298116275	-0.385001826950307\\
2.16473830582924	-0.407090443167509\\
2.15942876212331	-0.428972152863411\\
2.15264621487007	-0.450777177012557\\
2.14429268050241	-0.472632306258149\\
2.13422597635368	-0.494663032559375\\
2.12225292850496	-0.516995399962178\\
2.10811960970095	-0.539757532349487\\
2.09149782429426	-0.56308072320629\\
2.07196672858698	-0.587099860284975\\
2.04898804725626	-0.611952780355284\\
2.02187279783734	-0.63777786381841\\
1.98973676223325	-0.664708718811725\\
1.95144120150381	-0.692864065538602\\
1.90551470116897	-0.722329760700233\\
1.85005211103009	-0.753128096722297\\
1.78258862099129	-0.785166872057058\\
1.69995396028697	-0.818157275872694\\
1.59812923262143	-0.851486233524169\\
1.47216693255121	-0.884028798398856\\
1.31630617031597	-0.913898206754499\\
1.12452333785629	-0.938173284195572\\
0.891854337270322	-0.952738768752893\\
0.616726685582375	-0.952519250789799\\
0.303922965480029	-0.932459702043087\\
-0.0334113403580325	-0.889312146670426\\
-0.375047850181613	-0.823482720324207\\
-0.698972336787196	-0.739562366176011\\
-0.987934845355978	-0.644862943327322\\
-1.23316467192344	-0.546927933872179\\
-1.43387775149854	-0.451627708476155\\
-1.59438233199738	-0.362543496434898\\
-1.72110483307549	-0.281292466100066\\
-1.82062781970597	-0.208173978793897\\
-1.8987518250651	-0.142752382824162\\
-1.96021453734701	-0.0842579579464069\\
-2.00873229014713	-0.0318223597645517\\
-2.0471598069178	0.0153996296826504\\
-2.07766884025279	0.0581760899462323\\
-2.10190646029094	0.0971806018546626\\
-2.12112253576385	0.132993026489251\\
-2.13626770401609	0.166108231550759\\
-2.14806686295431	0.196947494882579\\
-2.15707362420401	0.225870042184848\\
-2.16371041277271	0.253183617661105\\
-2.16829789455687	0.279153715091212\\
-2.17107649345363	0.304011444539451\\
-2.17222201559771	0.327960160283018\\
-2.17185682732312	0.351181027234208\\
-2.17005760469696	0.373837707814064\\
-2.16686035050174	0.396080335234385\\
-2.16226312799609	0.418048915242808\\
-2.15622676391288	0.439876272143686\\
-2.14867360495218	0.461690627913561\\
-2.13948425462151	0.48361787463078\\
-2.12849205420061	0.505783567443355\\
-2.11547488614215	0.528314622955141\\
-2.10014365189416	0.551340648194742\\
-2.08212648761002	0.574994735392162\\
-2.06094740621443	0.599413416964235\\
-2.0359975682226	0.624735250132764\\
-2.00649677097877	0.651097138197798\\
-1.97144202503041	0.678626911798155\\
-1.92953936980179	0.70742976115612\\
-1.87911471112878	0.737564649699697\\
-1.81800030310009	0.769004637087864\\
-1.7433975081286	0.801571960947746\\
-1.65172775176408	0.834835094188598\\
-1.53850984188323	0.867952605280023\\
-1.39835502910763	0.899453036089182\\
-1.22526238417232	0.926964181094606\\
-1.01351140865888	0.94697152112484\\
-0.759481808475353	0.954811303080049\\
};
\addplot [color=mycolor1, forget plot]
  table[row sep=crcr]{%
-0.324618520603023	0.837628893195232\\
0.141649275301933	0.822760735661914\\
0.596881567596064	0.779453357448525\\
1.0021395912943	0.714932864138764\\
1.3371668551542	0.639646692660512\\
1.60075351461313	0.562611696507694\\
1.80261577585573	0.489443428944657\\
1.95563109944708	0.422718211680479\\
2.07167039486615	0.363082993210658\\
2.16024047217468	0.310200454343011\\
2.2284812634194	0.263332621855819\\
2.28160509550794	0.22164422755014\\
2.32338034966147	0.184339950814669\\
2.35653235040297	0.150715553516891\\
2.38304317205515	0.120168430000446\\
2.404366145207	0.0921909486791049\\
2.42157654482902	0.0663577986355005\\
2.43547688557959	0.0423123706696491\\
2.44667060615226	0.0197542087083929\\
2.45561387912914	-0.0015718189538681\\
2.46265225208374	-0.0218845843751558\\
2.46804668999558	-0.0413741934350738\\
2.47199212297803	-0.060208170850693\\
2.47463060535432	-0.078536598710198\\
2.47606051073955	-0.0964964436946383\\
2.47634271644072	-0.114215275566578\\
2.47550439601706	-0.131814549621168\\
2.47354079088737	-0.149412601717515\\
2.47041513469672	-0.167127486921179\\
2.46605673063414	-0.185079781328189\\
2.46035700892522	-0.203395460421398\\
2.45316319636694	-0.222208964955927\\
2.44426898542543	-0.241666564632794\\
2.4334012623827	-0.26193012664474\\
2.42020149376975	-0.283181382844243\\
2.4041997078324	-0.305626750788811\\
2.38477804171665	-0.329502671013201\\
2.36111941247666	-0.35508121863716\\
2.33213482280654	-0.382675321490496\\
2.29635992462281	-0.412642056840548\\
2.25180763874941	-0.445380800680903\\
2.1957592938514	-0.48131971990023\\
2.1244740268668	-0.520877920134854\\
2.03280193886641	-0.564379475232355\\
1.91371936967003	-0.611877464082452\\
1.75790851186681	-0.662822678866233\\
1.55376102383069	-0.715501866957153\\
1.28867822650361	-0.766243460037781\\
0.953077794663284	-0.808695686544094\\
0.547943309889831	-0.83411341498746\\
0.0929521451383333	-0.833937257179846\\
-0.373411518321853	-0.804349215721416\\
-0.807532662055344	-0.749203359525841\\
-1.17887417675215	-0.678000007762318\\
-1.47743180119758	-0.600914131355315\\
-1.70860296234771	-0.525324568057738\\
-1.88443811822814	-0.455202648837094\\
-2.01761943598438	-0.392020878970487\\
-2.11889173447411	-0.335838989529977\\
-2.19653701794795	-0.286066804736578\\
-2.25666864061581	-0.241892517007049\\
-2.3037204981097	-0.202490471586393\\
-2.34089583370865	-0.167107781766933\\
-2.3705167105481	-0.135091149775208\\
-2.39427849140452	-0.105886833462269\\
-2.41342985701326	-0.0790300757239985\\
-2.4288987866803	-0.0541315781974386\\
-2.44138060147967	-0.030864274957714\\
-2.4513997071776	-0.00895161256239827\\
-2.45935313333664	0.0118423818282623\\
-2.4655414085426	0.0317213486275997\\
-2.4701905387983	0.0508634012982991\\
-2.47346764623651	0.0694267567626212\\
-2.47549200150645	0.0875544409215963\\
-2.4763426177843	0.105378288172085\\
-2.4760631789495	0.123022421945924\\
-2.47466478847366	0.140606376124166\\
-2.47212680681249	0.158247996346675\\
-2.46839586278682	0.176066245797393\\
-2.46338295398581	0.194184031342863\\
-2.45695837006585	0.212731161827081\\
-2.4489439566827	0.231847549143729\\
-2.4391019566509	0.251686761427661\\
-2.42711927798657	0.272420030759712\\
-2.41258548757802	0.29424079441131\\
-2.39496202999984	0.317369787919756\\
-2.37353900245125	0.342060569360835\\
-2.34737411283272	0.368605057766904\\
-2.31520600437833	0.397338061428061\\
-2.27533076853194	0.428638560214717\\
-2.2254262700896	0.462923138093415\\
-2.16230492395649	0.500622438918312\\
-2.08157571644412	0.542123177117839\\
-1.97721206215354	0.587643828606011\\
-1.84108271073518	0.636990557641272\\
-1.66267094055474	0.689118820395049\\
-1.4295781838639	0.74144414246256\\
-1.12998729811571	0.789012731023362\\
-0.758476839361802	0.824122371675962\\
-0.324618520603024	0.837628893195232\\
};
\addplot [color=mycolor1, forget plot]
  table[row sep=crcr]{%
0.191200421204062	0.824265275129285\\
0.71325592275401	0.807963511485502\\
1.16345759404792	0.765425392551587\\
1.52252052148191	0.708448771096694\\
1.79587642946785	0.647122748221874\\
1.9997464806358	0.587587061317099\\
2.15129352137004	0.532674364167705\\
2.26469168011602	0.483229473891889\\
2.35052395916227	0.439117036612124\\
2.41636375999719	0.399802546765864\\
2.46755317204297	0.364641438372434\\
2.50785972415183	0.333007084390018\\
2.5399593089937	0.304338968099849\\
2.56577253858966	0.278154207307022\\
2.58669425502056	0.254043669529567\\
2.60374943665117	0.231662595783897\\
2.61769948976084	0.210720119939289\\
2.62911526822383	0.190969460988905\\
2.63842769866772	0.172199374681048\\
2.64596319880481	0.154226933192088\\
2.6519686414066	0.1368914962194\\
2.65662901872332	0.120049672038544\\
2.66007991001322	0.103571061667474\\
2.66241615420358	0.0873345957210633\\
2.66369765410891	0.071225294808439\\
2.66395290566747	0.0551313030145495\\
2.66318060062165	0.0389410570502846\\
2.66134945605483	0.0225404598091692\\
2.65839625066598	0.00580992571133416\\
2.65422187103344	-0.0113788442080452\\
2.64868496619965	-0.0291665176937357\\
2.64159254520199	-0.0477104051036007\\
2.63268648800917	-0.0671895773652435\\
2.62162441457903	-0.0878111005514818\\
2.60795257653423	-0.109817710668383\\
2.59106725814224	-0.133497293444217\\
2.57015937826012	-0.159194529467751\\
2.54413424393695	-0.187324931307483\\
2.51149425802899	-0.218391047844017\\
2.47016627992243	-0.252999427725799\\
2.41724696483688	-0.291874127123704\\
2.34862987349136	-0.335856254401098\\
2.25847372891593	-0.385865586756277\\
2.13849379012915	-0.442773260758363\\
1.97717055072386	-0.507086259797404\\
1.75931890171518	-0.578280346889523\\
1.46727702516943	-0.653613616569171\\
1.08617723612172	-0.726578100557776\\
0.615460439420353	-0.786230272795211\\
0.0822835338711193	-0.819919980671972\\
-0.458755077233071	-0.82004513526283\\
-0.94927335353285	-0.789253224250595\\
-1.35445595364017	-0.738024537394433\\
-1.66900199975997	-0.677853055182354\\
-1.90537723635566	-0.616892504600364\\
-2.08108780044098	-0.55946756251732\\
-2.21202005649414	-0.507262547905664\\
-2.31051905310019	-0.46053519056928\\
-2.38556530379483	-0.418899791716337\\
-2.44352386353713	-0.38174322250028\\
-2.48887853447455	-0.34842024462932\\
-2.52480076725475	-0.318334243633819\\
-2.55355433630708	-0.290963369918397\\
-2.57677354714022	-0.265862593178887\\
-2.59565238416989	-0.242656271453801\\
-2.61107314174097	-0.221027883359438\\
-2.62369442650236	-0.200709758129085\\
-2.63401188594597	-0.181473866908048\\
-2.64240050775657	-0.163123947349628\\
-2.6491443324095	-0.145488902178262\\
-2.65445744888948	-0.128417291597656\\
-2.65849884811449	-0.111772710978315\\
-2.66138285164333	-0.0954298538359701\\
-2.66318625772025	-0.0792710803233651\\
-2.66395295068996	-0.0631833320099094\\
-2.66369643639807	-0.0470552498199148\\
-2.66240055017646	-0.0307743616965002\\
-2.66001840294916	-0.0142242090219379\\
-2.65646945892056	0.00271872448881402\\
-2.65163445089512	0.0201884314122112\\
-2.64534760931634	0.0383333196120541\\
-2.63738537336875	0.0573208771826185\\
-2.62745031680829	0.0773433114590995\\
-2.61514838243709	0.0986244601731538\\
-2.5999565620817	0.121428321698415\\
-2.58117670599581	0.146069576303246\\
-2.55786892677725	0.172926416392208\\
-2.52875468402557	0.202455744887065\\
-2.49207457729266	0.235210057075738\\
-2.44537863236689	0.271853482986669\\
-2.38521758771343	0.313170239913061\\
-2.30669532153084	0.360049484777172\\
-2.20284719577246	0.413411291637233\\
-2.06386455985811	0.474001591350687\\
-1.87639089359498	0.541924817937139\\
-1.62366965142184	0.615730639568154\\
-1.2884049236082	0.690980647379509\\
-0.861085217365527	0.758884641393307\\
-0.353732957529172	0.807004156430304\\
0.19120042120406	0.824265275129285\\
};
\addplot [color=mycolor1, forget plot]
  table[row sep=crcr]{%
0.613287781213732	0.86010797340398\\
1.11511674731886	0.844659582097954\\
1.51768457820977	0.806751425025845\\
1.82376588317216	0.758240526604769\\
2.0509352373577	0.707296184889737\\
2.21880388620532	0.658276776173363\\
2.34371427611363	0.613012334620282\\
2.43782537863024	0.571971938560831\\
2.50977234567591	0.534990445442697\\
2.56559421684612	0.501653186781069\\
2.6095150415272	0.471480657201323\\
2.64451273544842	0.444009402733239\\
2.67271096584353	0.418822514218511\\
2.69564373909931	0.395556952080649\\
2.71443341082278	0.373900962812272\\
2.72991101586408	0.353587826889258\\
2.74269832317876	0.334388753816769\\
2.75326437201219	0.316106104164191\\
2.7619648292203	0.298567354270926\\
2.7690696320498	0.281619874727851\\
2.77478251664214	0.265126448280041\\
2.7792548183247	0.248961399474088\\
2.78259512976479	0.233007195475165\\
2.78487586620252	0.217151380124197\\
2.78613741627532	0.201283709565552\\
2.786390289109	0.185293361868146\\
2.78561546044436	0.169066091981578\\
2.78376294122543	0.152481195182978\\
2.78074841529247	0.135408125132969\\
2.77644759298665	0.117702584711849\\
2.77068767443536	0.099201866301543\\
2.76323496938589	0.0797191598005111\\
2.75377722007983	0.059036467714995\\
2.74189842845168	0.0368956642359729\\
2.72704285592108	0.0129871103311403\\
2.70846311833792	-0.0130648973866359\\
2.68514459016501	-0.0417206655341929\\
2.65569413444983	-0.0735483955114565\\
2.61817476350412	-0.109252478769324\\
2.56985843389196	-0.149705358213237\\
2.50685675263342	-0.195977005304862\\
2.4235774834724	-0.249345084109551\\
2.31195942416062	-0.311243951870691\\
2.16051209060582	-0.383058764199705\\
1.95347159308953	-0.465579411248454\\
1.67116729490645	-0.557829172076192\\
1.29418386758641	-0.655100904556978\\
0.814779939119185	-0.746989151120519\\
0.253872314130891	-0.818278428838929\\
-0.332556849944329	-0.855612892402552\\
-0.87571841253119	-0.8559992687482\\
-1.32918499612098	-0.827708316243736\\
-1.68184436630816	-0.783209949142495\\
-1.94594384520362	-0.732724453171794\\
-2.14116067344381	-0.682388056643545\\
-2.28577545587929	-0.635124543600925\\
-2.39400860433696	-0.591965379762945\\
-2.47614130760329	-0.552996748802563\\
-2.53940003140643	-0.517896024319721\\
-2.58883226427853	-0.486201189920134\\
-2.62798013867384	-0.457434674716157\\
-2.65935451449369	-0.431154363145532\\
-2.68475737519204	-0.406970189330942\\
-2.70549880674259	-0.384545403180371\\
-2.72254321948034	-0.363591643989305\\
-2.73660859151485	-0.343862040311453\\
-2.74823448933603	-0.325144182371768\\
-2.75782918081735	-0.307253691506714\\
-2.76570258746931	-0.290028595005323\\
-2.77208950790246	-0.273324488302417\\
-2.77716604157746	-0.257010376088385\\
-2.78106115814751	-0.240965055130788\\
-2.78386470419543	-0.225073898559042\\
-2.78563269502162	-0.209225906815504\\
-2.78639042625926	-0.1933108961482\\
-2.78613370669727	-0.177216697306981\\
-2.7848283235279	-0.160826232686187\\
-2.78240767628803	-0.144014327774542\\
-2.77876833081706	-0.126644090570417\\
-2.77376302238152	-0.108562658267982\\
-2.76719034317613	-0.089596061029038\\
-2.75877993484113	-0.069542884450331\\
-2.74817139791829	-0.0481663218772056\\
-2.73488421319197	-0.0251840934520167\\
-2.71827456526713	-0.000255576276756123\\
-2.69747278524308	0.0270346355114722\\
-2.67129175488589	0.0571975662114773\\
-2.63809141127379	0.0908656856475478\\
-2.59557666716171	0.128823353730755\\
-2.54049503654738	0.172038704817913\\
-2.46818719690168	0.221686691139371\\
-2.37193666515798	0.279136317997162\\
-2.24209351508817	0.345838650164078\\
-2.06510027125647	0.422981151964812\\
-1.82303820318996	0.510667469332616\\
-1.49546056309268	0.606340464715187\\
-1.06683713755812	0.702604396053219\\
-0.54172868935217	0.78620186581936\\
0.0405998642924217	0.84168534999174\\
0.61328778121373	0.86010797340398\\
};
\addplot [color=mycolor1, forget plot]
  table[row sep=crcr]{%
0.926393358475887	0.917706271260934\\
1.38607263076893	0.903665969310752\\
1.7422016593669	0.870176702341646\\
2.00870321996272	0.827950270283077\\
2.20597730285032	0.783707671619406\\
2.3525210968257	0.740908734638542\\
2.46258707079647	0.701016603024936\\
2.54644387593162	0.664441849350182\\
2.61130548498574	0.63109723987738\\
2.66221582625019	0.600689035826832\\
2.70272100981106	0.572859548493884\\
2.73534034840694	0.547252309736575\\
2.76188629984635	0.523538805080332\\
2.78367988503753	0.50142682268343\\
2.80169603717409	0.480660429960683\\
2.81666241613587	0.461016464022597\\
2.82912726059294	0.442299855659283\\
2.83950647434844	0.424338836503721\\
2.84811662193418	0.406980457837514\\
2.85519822357702	0.390086551090085\\
2.86093225475325	0.373530120957746\\
2.86545178198856	0.357192100266858\\
2.86885001841987	0.34095836999222\\
2.8711856407957	0.324716937328195\\
2.87248589696127	0.308355158613597\\
2.87274779812196	0.291756886363926\\
2.87193749709468	0.274799407022081\\
2.86998777400157	0.257350015428979\\
2.86679335848258	0.239262040489446\\
2.86220358397008	0.220370090483013\\
2.85601155816496	0.20048422133587\\
2.84793859215063	0.179382641104984\\
2.83761197920958	0.15680244216955\\
2.82453322963001	0.132427693010208\\
2.8080323544497	0.105874023092661\\
2.78720144224183	0.0766686140974801\\
2.76079711622966	0.0442243342015672\\
2.72709580713587	0.00780680891789201\\
2.68367726847776	-0.0335059966993132\\
2.62709978763967	-0.0808694015873919\\
2.55241651917453	-0.135713101802578\\
2.45247556986243	-0.199749541256519\\
2.31698107075966	-0.274879353275059\\
2.13146950088213	-0.362838550985554\\
1.87691333707886	-0.46429770138626\\
1.53194166745117	-0.577051012339622\\
1.08127310409524	-0.693415470461945\\
0.532134511957492	-0.798831354735789\\
-0.0714694655728092	-0.8757747316268\\
-0.659035959560127	-0.913403657671121\\
-1.16938961652807	-0.913918239494097\\
-1.57651842842924	-0.888592637089835\\
-1.88539534489321	-0.849644055527742\\
-2.11474579460562	-0.805803912152524\\
-2.2845970351354	-0.762002705065793\\
-2.41138888813924	-0.720557114614023\\
-2.50728045874624	-0.682312710598663\\
-2.58089194300237	-0.647381474947317\\
-2.63825421303898	-0.615547964013723\\
-2.68359190121149	-0.586474698142311\\
-2.71988950170122	-0.559799491160688\\
-2.74928039355392	-0.535177979701714\\
-2.77330943370495	-0.512299438072088\\
-2.79310978287705	-0.490890090764409\\
-2.80952261211717	-0.470710936834829\\
-2.82317887741413	-0.451553467220147\\
-2.83455576976416	-0.433234848244482\\
-2.84401608759779	-0.415593254807294\\
-2.85183594086373	-0.39848360497849\\
-2.85822435538924	-0.381773742998982\\
-2.86333714558525	-0.365341024272321\\
-2.86728663107626	-0.349069215733003\\
-2.87014823964302	-0.332845608679193\\
-2.87196466935385	-0.316558233809178\\
-2.87274801434064	-0.300093061929035\\
-2.87248004864847	-0.283331064086473\\
-2.87111067944473	-0.266144988608059\\
-2.86855439809411	-0.248395686846582\\
-2.86468434858169	-0.229927781173933\\
-2.85932336565911	-0.210564413826294\\
-2.85223096645201	-0.190100738411087\\
-2.84308474492088	-0.16829571084183\\
-2.83145381965049	-0.144861596391372\\
-2.81676076664323	-0.119450430002067\\
-2.79822658503956	-0.0916364535841698\\
-2.77479031253233	-0.0608933422327805\\
-2.74499035284064	-0.0265649348789936\\
-2.70678760765723	0.0121714882088764\\
-2.65730028311564	0.0563488156370922\\
-2.59240679033264	0.107255467266795\\
-2.50616063962901	0.166465520453224\\
-2.38996807466756	0.235808451596563\\
-2.23156557364936	0.317172385697494\\
-2.01416104841412	0.411922540994109\\
-1.71698590740431	0.519583279163353\\
-1.32012399554165	0.635540125842404\\
-0.817175390612008	0.748615574455404\\
-0.232995168271264	0.841817858572328\\
0.371751247386078	0.899672684197672\\
0.926393358475885	0.917706271260934\\
};
\addplot [color=mycolor1, forget plot]
  table[row sep=crcr]{%
1.15917423619427	0.985452996961435\\
1.57605581690044	0.97276915929865\\
1.89441343642087	0.942843264494399\\
2.13227683472436	0.905150342403449\\
2.30942407846641	0.865413163486818\\
2.44232774805049	0.826589535101777\\
2.54329864806614	0.789986729526301\\
2.62113458251352	0.756032335676635\\
2.6820299188861	0.724722222994033\\
2.73034635185163	0.69585971565669\\
2.76917854378417	0.669176729546374\\
2.8007467931326	0.644392180882967\\
2.82666408408834	0.621238171226077\\
2.84811697000639	0.599470039069359\\
2.86598860922422	0.578868577273424\\
2.88094314711286	0.559238633514245\\
2.8934841634484	0.540406200432067\\
2.90399555917694	0.522215017548692\\
2.91277040371961	0.504523151960812\\
2.92003140075226	0.487199742292122\\
2.92594540855611	0.470121947586638\\
2.93063364090557	0.453172070861024\\
2.93417862736339	0.43623478986836\\
2.93662863152781	0.419194406100511\\
2.93799994881924	0.401932006621231\\
2.93827728833417	0.384322415691087\\
2.93741225391446	0.366230789939307\\
2.93531975068679	0.34750867822912\\
2.93187192760633	0.327989321208702\\
2.92688899133663	0.307481900752102\\
2.92012584763972	0.285764359574226\\
2.91125297782018	0.262574288144739\\
2.89982913907324	0.237597210458558\\
2.88526223146898	0.210451384549692\\
2.86675275384526	0.180667969274259\\
2.84321129383027	0.147665126580317\\
2.81313689528631	0.110714438865427\\
2.77443615632001	0.0688982370962552\\
2.72415278113033	0.0210578615885913\\
2.6580642866756	-0.0342625301334779\\
2.57009133700856	-0.0988597098543036\\
2.45147526364566	-0.174856072119426\\
2.28977247488069	-0.26451353004594\\
2.068051001765	-0.369643323146168\\
1.76555664303862	-0.49022874188461\\
1.3626712298896	-0.62196851668595\\
0.853498448339034	-0.753558732612605\\
0.263067043959383	-0.867084769427221\\
-0.34858168173437	-0.945253034235647\\
-0.911523896811392	-0.981455505235464\\
-1.38070599042454	-0.982007452859509\\
-1.7465684190745	-0.959275748618546\\
-2.02212325732471	-0.924530624568435\\
-2.22731078007852	-0.885302064978562\\
-2.3805439581278	-0.845777644851068\\
-2.49618350570722	-0.807969731568821\\
-2.58467037086583	-0.77267217971083\\
-2.65339185681341	-0.740056351422403\\
-2.70754271270146	-0.71000099655222\\
-2.75079235057624	-0.682263459422644\\
-2.78575807593243	-0.656564373723719\\
-2.8143293235244	-0.632627245573377\\
-2.83788741038465	-0.610195162061931\\
-2.8574545906673	-0.589036205167183\\
-2.87379582783211	-0.568943497701945\\
-2.88748893302291	-0.549732869739\\
-2.89897339195555	-0.531239620165527\\
-2.90858467669404	-0.513315071683582\\
-2.91657853317752	-0.495823221250298\\
-2.92314822799873	-0.478637586915617\\
-2.92843674480804	-0.461638250687693\\
-2.93254525685895	-0.44470904568158\\
-2.9355387481116	-0.427734808183106\\
-2.93744933310586	-0.410598597259897\\
-2.93827758337044	-0.393178768144887\\
-2.93799196850087	-0.375345765687661\\
-2.93652633419291	-0.35695847674322\\
-2.93377514082864	-0.337859941550254\\
-2.92958594539539	-0.317872169395212\\
-2.92374828895184	-0.296789727417868\\
-2.91597769794034	-0.274371665981091\\
-2.90589283948323	-0.250331200815598\\
-2.89298286274102	-0.224322382197802\\
-2.87656041322932	-0.195922740098191\\
-2.85569341655181	-0.164610613804974\\
-2.82910502268571	-0.129735616258361\\
-2.79502541026458	-0.0904806364713247\\
-2.75097066118413	-0.0458144539665412\\
-2.69341216109772	0.00556331091612401\\
-2.61728671989778	0.0652755618461577\\
-2.51529289624149	0.135290740074671\\
-2.37695842808772	0.217842197730919\\
-2.18764946166002	0.315079127957206\\
-1.92825794555101	0.42813693392327\\
-1.57755809935925	0.555223929896933\\
-1.12070581839738	0.688795150854699\\
-0.565356701070268	0.813805801254907\\
0.04470746832057	0.911337695620967\\
0.639826601490393	0.968452871849045\\
1.15917423619427	0.985452996961435\\
};
\addplot [color=mycolor1, forget plot]
  table[row sep=crcr]{%
1.33832785075596	1.05856770426014\\
1.71708873794132	1.04706270790821\\
2.00524308812722	1.01997360948505\\
2.22158293985867	0.98568232028984\\
2.38421685443873	0.949191101343268\\
2.5076145272811	0.91313611260959\\
2.6024663989092	0.87874493050746\\
2.67642270413419	0.846477797541033\\
2.7349083019084	0.816402633286111\\
2.78177963434076	0.78840014646188\\
2.81980118326959	0.762271550607141\\
2.85097669866004	0.737793146276745\\
2.87677601172378	0.714742657350381\\
2.89829006124147	0.692910811688235\\
2.91633737206639	0.672105353139954\\
2.93153778559506	0.652151275517667\\
2.9443639890652	0.632889257135411\\
2.95517784931997	0.614173310232574\\
2.96425621180258	0.595868146084142\\
2.97180927822365	0.577846482438437\\
2.97799364973239	0.559986372597205\\
2.982921432524	0.542168554727318\\
2.98666632910257	0.524273773221959\\
2.9892673015886	0.506179992999476\\
2.99073014005198	0.487759402000392\\
2.99102706084517	0.46887507006363\\
2.99009426746616	0.449377098720827\\
2.9878272023742	0.429098051468598\\
2.98407297275139	0.407847392434451\\
2.97861910870765	0.38540457648988\\
2.97117735555587	0.36151031754284\\
2.96136053270739	0.335855404031068\\
2.94864948746336	0.308066221173719\\
2.93234564125112	0.27768587209012\\
2.91150227341512	0.244149475236976\\
2.88482407104405	0.206751914642895\\
2.8505189770917	0.164606224499872\\
2.80607829448728	0.11659141871253\\
2.74795009456766	0.0612912218850788\\
2.67105933973035	-0.00306711709908142\\
2.56812617464324	-0.0786451112657921\\
2.4287765554169	-0.167922293729583\\
2.23862361750258	-0.273355629826909\\
1.97903645009366	-0.396452773264885\\
1.62946597084443	-0.535843685786365\\
1.17549772978648	-0.684372465629206\\
0.623913899953506	-0.827066784603145\\
0.0156603832950979	-0.944197989256905\\
-0.582713904244943	-1.02082682525928\\
-1.1109120882187	-1.05489053430368\\
-1.54004749962896	-1.05543374616649\\
-1.87136800357853	-1.03485403068786\\
-2.12118690488697	-1.00334708083351\\
-2.30861435903366	-0.967504177395819\\
-2.45007070231361	-0.931008400315427\\
-2.55806939607659	-0.895691241500445\\
-2.64167391994001	-0.862335439946601\\
-2.70732795253916	-0.831170891254812\\
-2.75960186319709	-0.802153688786899\\
-2.80175664512797	-0.775115414285042\\
-2.83614241827917	-0.749840185559272\\
-2.86447275609878	-0.726102868844767\\
-2.88801214554155	-0.703686833934361\\
-2.90770435438771	-0.682391113449945\\
-2.92426093572038	-0.662032191603923\\
-2.9382227944058	-0.642443165421628\\
-2.95000341218912	-0.623471699475948\\
-2.9599194431015	-0.604977491641939\\
-2.96821248644699	-0.58682959233689\\
-2.97506458571853	-0.568903718994405\\
-2.98060916195791	-0.55107959889393\\
-2.98493852032551	-0.533238312675461\\
-2.98810867137945	-0.515259573693341\\
-2.99014191896274	-0.497018851100946\\
-2.99102743998762	-0.47838421890837\\
-2.99071988462755	-0.459212783487904\\
-2.98913583058736	-0.439346503305103\\
-2.98614770441886	-0.418607162070534\\
-2.981574503371	-0.396790184171176\\
-2.97516826859042	-0.373656881793238\\
-2.96659470971629	-0.348924587454162\\
-2.95540556317111	-0.322253943280443\\
-2.94099902838946	-0.293232380082817\\
-2.92256272972903	-0.261352524893812\\
-2.898990733363	-0.22598395477037\\
-2.86876167890882	-0.186336478897755\\
-2.82975838133331	-0.141413299954517\\
-2.77899972415068	-0.08995380351499\\
-2.71224382577774	-0.030370353344629\\
-2.62341239539384	0.0393040537726912\\
-2.50379953054959	0.121410226009802\\
-2.34112351143517	0.218486204153281\\
-2.11880645853817	0.332683514259896\\
-1.81669174283804	0.464386512203611\\
-1.41581099600258	0.609717854256119\\
-0.910199242096203	0.757659551907771\\
-0.32300024499543	0.8900063214995\\
0.289093551297695	0.98803779850949\\
0.858233627317148	1.04278813347991\\
1.33832785075596	1.05856770426014\\
};
\addplot [color=mycolor1, forget plot]
  table[row sep=crcr]{%
1.48195524741105	1.13499600551441\\
1.82771084245232	1.12449792923133\\
2.09121070631347	1.09971889927061\\
2.29056202585284	1.06811011905157\\
2.44200222497211	1.03412112200209\\
2.5582243973705	1.00015506933652\\
2.64857921337322	0.967388428797628\\
2.71979414609436	0.936312684715988\\
2.77668243987642	0.907055225989295\\
2.82270084726161	0.879559355166109\\
2.86035316545087	0.85368207890044\\
2.89147238204279	0.829245848782393\\
2.91741560862578	0.806065029860011\\
2.93919865663493	0.783958663217387\\
2.95758940438435	0.762755871451644\\
2.97317306982624	0.74229736750022\\
2.98639822665132	0.722434940873958\\
2.9976094924628	0.703029924367371\\
3.00707087054564	0.683951164254744\\
3.01498242644814	0.665072750318575\\
3.02149210661462	0.646271611685118\\
3.02670391018261	0.627424996845516\\
3.03068320802233	0.60840780146909\\
3.03345969887125	0.589089668398643\\
3.03502825503851	0.569331749997934\\
3.03534770547678	0.548982986595672\\
3.03433740348466	0.52787571031182\\
3.03187120178483	0.505820325345909\\
3.02776817638863	0.482598737354212\\
3.0217790587727	0.457956097902255\\
3.01356679025977	0.431590285446009\\
3.00267880784298	0.403138350895174\\
2.98850746087563	0.372158903955268\\
2.9702331191046	0.33810910603951\\
2.94674172850081	0.300314597460034\\
2.91650432599898	0.257930436006693\\
2.87739975294452	0.209891288074655\\
2.82645308816859	0.154850528636367\\
2.75945194470782	0.0911125846386886\\
2.67039604942962	0.0165753827145177\\
2.55075154984237	-0.0712701882940495\\
2.38857573601612	-0.175171812629412\\
2.16788371645304	-0.297546184921392\\
1.86937402705675	-0.439126103017382\\
1.47485565200519	-0.596498607285113\\
0.978024766685259	-0.759158540164599\\
0.399280619236233	-0.909031700309866\\
-0.209222691531337	-1.02637010365967\\
-0.782516564403628	-1.09990426092498\\
-1.27340608465234	-1.13162079951875\\
-1.66619789036641	-1.13213477534486\\
-1.96860925157471	-1.11334737001622\\
-2.19781258107627	-1.08443074321809\\
-2.37139232321267	-1.05122597349927\\
-2.50386060594156	-1.01704058497466\\
-2.60616221235153	-0.983579677611965\\
-2.68624082346912	-0.951625326146515\\
-2.74978669526022	-0.921457337484466\\
-2.80087536645179	-0.893094774491843\\
-2.84244511941849	-0.866429077725958\\
-2.87663525618585	-0.841295442698824\\
-2.90502076456061	-0.817510021061471\\
-2.92877422841182	-0.79488848704882\\
-2.94877782711592	-0.77325454830531\\
-2.9657013206889	-0.752443091626493\\
-2.98005679825384	-0.732300512698212\\
-2.99223742913519	-0.712683603755642\\
-3.00254507357401	-0.693457726910913\\
-3.0112100193343	-0.674494643567592\\
-3.01840504582332	-0.655670170619315\\
-3.0242552972786	-0.636861720027347\\
-3.02884494997068	-0.617945709956243\\
-3.0322213045204	-0.598794790290315\\
-3.03439666846907	-0.579274789687264\\
-3.03534817700269	-0.559241256781215\\
-3.03501550055304	-0.538535428402326\\
-3.03329617915488	-0.51697940709317\\
-3.03003807513199	-0.494370262762469\\
-3.02502811051957	-0.47047268187135\\
-3.01797600141845	-0.445009663280249\\
-3.00849104089377	-0.417650592306105\\
-2.99604899719174	-0.387995802807642\\
-2.97994470381437	-0.355556454808895\\
-2.95922364687653	-0.31972822362765\\
-2.93258239896873	-0.279756978818976\\
-2.89822256475082	-0.234694525213657\\
-2.85363542562227	-0.183343111872696\\
-2.79528466319329	-0.124190093700613\\
-2.71814479248552	-0.0553419146260905\\
-2.61505367687108	0.0255137852239566\\
-2.47588198710784	0.121044266512722\\
-2.28669913561442	0.233941461232089\\
-2.02961011797109	0.366015313191922\\
-1.68496239277464	0.516299897716017\\
-1.23873660878186	0.67815187200221\\
-0.696280598734906	0.83700673279516\\
-0.0946098487340973	0.972777609559879\\
0.503930688914962	1.06878127437136\\
1.03995889446949	1.12043375648529\\
1.48195524741105	1.13499600551441\\
};
\addplot [color=mycolor1, forget plot]
  table[row sep=crcr]{%
1.60151165444191	1.21376353859017\\
1.91865392825986	1.20413163687514\\
2.16146055251207	1.18128912214335\\
2.34678220315943	1.15189476983143\\
2.48906901537889	1.11995145575896\\
2.5994810729338	1.08767646335323\\
2.68624928650806	1.05620504317403\\
2.75533692442773	1.02605331630931\\
2.81105064312716	0.997396557024017\\
2.85651480026093	0.970229115480782\\
2.89401520611343	0.944453983022338\\
2.9252412308416	0.919931966326893\\
2.95145466158424	0.896508048638974\\
2.97360744874277	0.874024968197798\\
2.99242420165329	0.852329669320216\\
3.00846039089377	0.831275798408647\\
3.0221437148727	0.810724016880521\\
3.03380368338312	0.790541112544326\\
3.04369284201066	0.770598441972192\\
3.05200195991931	0.750769978441166\\
3.05887075305835	0.730930088589762\\
3.06439519462896	0.710951068268945\\
3.06863209306283	0.690700406598752\\
3.07160133872594	0.670037700497916\\
3.07328599384936	0.648811099381227\\
3.07363019438329	0.626853113692275\\
3.07253461944072	0.603975564926566\\
3.06984903396344	0.579963382149292\\
3.06536108641892	0.554566853084517\\
3.05878009508824	0.527491807468921\\
3.04971390971152	0.498387035823426\\
3.0376359782482	0.466828017325126\\
3.0218383119463	0.432295740259596\\
3.00136387252382	0.394149060626916\\
2.97490864014783	0.351588725528002\\
2.94067879239094	0.303611091883521\\
2.89618159570869	0.248950233714589\\
2.83791994276145	0.186009859425842\\
2.76095254672642	0.112794272297332\\
2.65828483846744	0.0268670495822667\\
2.52010059775551	-0.0745900511703682\\
2.33301050835554	-0.194457881804245\\
2.07994111740238	-0.334801432192292\\
1.74218505871029	-0.495034299008016\\
1.30608372407202	-0.669071004772725\\
0.775416924238896	-0.842931253667855\\
0.183026458477091	-0.996488532947787\\
-0.413353316298409	-1.11162359925376\\
-0.955704250961396	-1.18127373485712\\
-1.41002817879594	-1.21066319585894\\
-1.77044404461756	-1.21114015108714\\
-2.04826567042507	-1.19387320821478\\
-2.26032756649037	-1.16710896417554\\
-2.422531658596	-1.13607078891378\\
-2.54768386419904	-1.10376561082803\\
-2.64540258288297	-1.07179752334559\\
-2.72270144076245	-1.04094758760794\\
-2.78464714192121	-1.01153548207928\\
-2.83490465866034	-0.983631313694125\\
-2.8761431824996	-0.957175611324294\\
-2.91032521757151	-0.932045857202917\\
-2.93890886945483	-0.908092622076673\\
-2.96298879862279	-0.88515861147164\\
-2.98339470092799	-0.86308816155869\\
-3.00076053537135	-0.841731422246322\\
-3.01557354829834	-0.82094559995982\\
-3.02820923357816	-0.800594580985712\\
-3.03895638740859	-0.780547661154982\\
-3.04803507758448	-0.76067776795101\\
-3.05560943928836	-0.740859364190165\\
-3.06179658640004	-0.720966104807109\\
-3.06667248976738	-0.70086824380515\\
-3.07027535439509	-0.68042973592618\\
-3.07260677896861	-0.659504934019606\\
-3.07363076847567	-0.637934739640746\\
-3.0732704647743	-0.61554201418963\\
-3.07140223244709	-0.592125994445668\\
-3.06784645563073	-0.567455372607121\\
-3.06235402300107	-0.541259588542124\\
-3.05458694179213	-0.51321773089914\\
-3.04409073683727	-0.48294424294041\\
-3.03025511957711	-0.449970369159299\\
-3.01225764702602	-0.413719960902326\\
-2.98898242622094	-0.373477916714031\\
-2.95890193537095	-0.328349285909294\\
-2.91990424359213	-0.277207246542429\\
-2.86904005845689	-0.218629624450325\\
-2.80215512061681	-0.150828347646298\\
-2.71336889273852	-0.0715887173907743\\
-2.59437845303026	0.0217350657353069\\
-2.43365766506504	0.132058542911812\\
-2.21590322689587	0.262015160923235\\
-1.92274195279636	0.412645902646693\\
-1.53677203889217	0.581005147098066\\
-1.05126636470144	0.75720443482184\\
-0.483696118163764	0.923554599581091\\
0.118643869731508	1.05962467768482\\
0.694129470594868	1.1520422511544\\
1.19480147923485	1.20034609016361\\
1.60151165444191	1.21376353859017\\
};
\addplot [color=mycolor1, forget plot]
  table[row sep=crcr]{%
1.7041846009319	1.2943141820301\\
1.99626617055785	1.28543761178804\\
2.22124175614415	1.26426293819397\\
2.39453017416156	1.23676782393869\\
2.52896408637316	1.20657960123409\\
2.63438716042362	1.175756675815\\
2.71808169542672	1.14539515887809\\
2.78536305329583	1.11602783814595\\
2.84010543395788	1.08786756517456\\
2.88514673507583	1.06095024777421\\
2.9225829191502	1.03521712036419\\
2.95397691776185	1.01056136374741\\
2.98050567802373	0.986854067154508\\
3.00306368390844	0.963958280238492\\
3.02233615866452	0.941736206473757\\
3.03885115178787	0.920052443127552\\
3.05301683469795	0.898774933595854\\
3.06514832973588	0.87777458064277\\
3.07548702698937	0.85692405068559\\
3.08421440519897	0.836096051577946\\
3.09146172447689	0.81516121575212\\
3.0973165011175	0.793985624376277\\
3.1018263403014	0.772427941024521\\
3.10500044296881	0.750336069553689\\
3.10680888273521	0.727543199906796\\
3.10717953780926	0.703863049386844\\
3.10599233300521	0.679084038331082\\
3.10307016596914	0.652962050479293\\
3.09816551812357	0.625211310957289\\
3.09094122611171	0.595492758529565\\
3.08094312764365	0.563399082666323\\
3.0675611678925	0.528435331338291\\
3.04997386749585	0.489993674758581\\
3.0270685348402	0.447320569827702\\
2.99732588623989	0.399474334660572\\
2.95865242299797	0.34527135316296\\
2.90813688499269	0.283220630985283\\
2.84169952147278	0.211451189241431\\
2.75360015287075	0.127649168935877\\
2.63579015752125	0.0290504103860355\\
2.47718044409962	-0.0874048975346086\\
2.26315440049889	-0.224539491410399\\
1.97624101171958	-0.383676370857818\\
1.59977690800976	-0.562325166838529\\
1.12661068121587	-0.751246369018012\\
0.571426793541186	-0.933269951884109\\
-0.0231945330767768	-1.08754580197928\\
-0.599163055661934	-1.19885002630157\\
-1.10821263274332	-1.26428398020831\\
-1.5280243443977	-1.29146157859367\\
-1.85963020768602	-1.29189976322708\\
-2.11613328811536	-1.27594927750475\\
-2.31347293789909	-1.25103332489436\\
-2.46592186792284	-1.22185313681699\\
-2.58479381981758	-1.19116200190927\\
-2.67857891676389	-1.16047520182953\\
-2.75350344627299	-1.13056847341001\\
-2.81410388947056	-1.10179161552201\\
-2.86369313535435	-1.07425566637257\\
-2.90470726510581	-1.04794158899572\\
-2.93895397442576	-1.02276231742486\\
-2.96778790699597	-0.99859763792049\\
-2.99223398977876	-0.975313379666422\\
-3.01307442280304	-0.952771567434943\\
-3.03091037717764	-0.930835368339748\\
-3.0462060383433	-0.909371033620749\\
-3.05932022560958	-0.888248094174011\\
-3.07052916260407	-0.867338521470358\\
-3.08004284009438	-0.84651524436022\\
-3.08801663398478	-0.82565021998284\\
-3.09455929844392	-0.804612137619662\\
-3.09973806520067	-0.783263755080808\\
-3.10358128812587	-0.761458808292563\\
-3.10607883621961	-0.73903838347614\\
-3.10718022598436	-0.715826588655281\\
-3.10679026744153	-0.691625299739336\\
-3.10476174708494	-0.666207678808018\\
-3.10088434981452	-0.639310060399069\\
-3.09486858146107	-0.610621666132709\\
-3.08632282322089	-0.579771428161253\\
-3.07472072399117	-0.546310967265258\\
-3.0593547589155	-0.509692477113805\\
-3.03926972115841	-0.469239927412545\\
-3.01316684589102	-0.42411168591123\\
-2.97926479270935	-0.373252582288746\\
-2.9350975127546	-0.315334137109362\\
-2.87722141885989	-0.248684455917916\\
-2.80079795942275	-0.171217097031371\\
-2.69902250417858	-0.0803873652608693\\
-2.56241494987765	0.026752980820651\\
-2.37814204960349	0.153247837121803\\
-2.1299429759065	0.30138943179547\\
-1.80001296258586	0.470949407623367\\
-1.37501405570806	0.656404734221779\\
-0.857224232733782	0.844434285581143\\
-0.275494477662446	1.01507514468077\\
0.317042707769568	1.14906003328664\\
0.864103828984793	1.23699824488219\\
1.32963119236916	1.28194972632408\\
1.7041846009319	1.2943141820301\\
};
\addplot [color=mycolor1, forget plot]
  table[row sep=crcr]{%
1.79451333687795	1.37623258172961\\
2.06437894430077	1.36802412314627\\
2.27364278731825	1.34831904966155\\
2.43629400799905	1.32250318949409\\
2.56373188637826	1.29387890390197\\
2.66466706633369	1.26436245928459\\
2.74556856172955	1.2350096720366\\
2.81119281358986	1.20636206852063\\
2.86503696965654	1.17866094921984\\
2.90968598827017	1.15197566340107\\
2.94706591443675	1.12627917729905\\
2.97862492315894	1.10149207155819\\
3.00546185166876	1.07750782499653\\
3.02841748975463	1.05420703582886\\
3.04813968400119	1.03146509239063\\
3.06513002218468	1.00915594522162\\
3.07977748081739	0.987153535081539\\
3.09238274743192	0.965331782097081\\
3.10317577097379	0.943563653214854\\
3.11232829045301	0.921719589106468\\
3.1199625284934	0.89966542345103\\
3.12615683127849	0.877259828969968\\
3.13094873227639	0.854351252443963\\
3.13433567233171	0.830774240267158\\
3.13627339068008	0.806344996398444\\
3.13667178150951	0.780855947410237\\
3.13538775939515	0.754069006761072\\
3.13221435902192	0.725707123922968\\
3.12686486295436	0.695443563909494\\
3.11895013854585	0.662888178353199\\
3.10794647251382	0.627569690518339\\
3.09314987331339	0.588912719714877\\
3.07361085634227	0.546207932699933\\
3.04804084787039	0.498573404835621\\
3.01467720302991	0.444905216476886\\
2.97108820312788	0.38381604428972\\
2.91389271499949	0.313563319504904\\
2.83836419754006	0.231976297684306\\
2.73789479389656	0.136410169507705\\
2.60333854470715	0.0237965139468894\\
2.42239579536337	-0.109060546546738\\
2.17956106428595	-0.264668607987185\\
1.85784309023286	-0.443144421608707\\
1.4441719282026	-0.639516259375913\\
0.939457202380576	-0.841139668930312\\
0.368918685510627	-1.0283302451733\\
-0.218665528781946	-1.18090618141992\\
-0.768973098121028	-1.28733879088908\\
-1.24436510686666	-1.34848886147076\\
-1.63213016362057	-1.3736032385062\\
-1.93796849459237	-1.3740036794835\\
-2.1756594985174	-1.35921433796541\\
-2.36001623058481	-1.33592856434747\\
-2.50381134739832	-1.30839697445668\\
-2.61706185013918	-1.27915087612558\\
-2.70729043857883	-1.24962273688704\\
-2.78004683255238	-1.2205774311595\\
-2.83940783860141	-1.19238591443792\\
-2.88837747174335	-1.16519138917053\\
-2.92918459732128	-1.13900792106641\\
-2.96349738916597	-1.11377818139752\\
-2.99257591894407	-1.08940686552014\\
-3.01738044126433	-1.06577972293003\\
-3.03864843761554	-1.04277407823606\\
-3.05694970805371	-1.02026430306792\\
-3.07272598295598	-0.998124270223542\\
-3.08631952632432	-0.976227977914113\\
-3.09799380993182	-0.954449030816098\\
-3.1079483738202	-0.932659362888273\\
-3.11632931733683	-0.910727400847761\\
-3.12323638902248	-0.888515747455641\\
-3.12872729505156	-0.865878380687456\\
-3.13281957597797	-0.84265729997381\\
-3.13549017384011	-0.818678491650597\\
-3.1366725962201	-0.793747023326093\\
-3.13625135215725	-0.767641003014309\\
-3.13405305519659	-0.740104045476918\\
-3.12983322099334	-0.71083576621519\\
-3.1232572748037	-0.679479662680571\\
-3.11387354646997	-0.645607531759416\\
-3.10107494690942	-0.608699304418124\\
-3.08404441443837	-0.568116856511122\\
-3.06167684265302	-0.523070018181276\\
-3.03246673353144	-0.472572782098417\\
-2.9943459418341	-0.415387948339402\\
-2.94444958092933	-0.349959983556741\\
-2.87878171397394	-0.274340653051243\\
-2.79175109683607	-0.18612422426896\\
-2.67556694180055	-0.0824371959007606\\
-2.51956678073377	0.0399140019352076\\
-2.30978231456736	0.183929908341486\\
-2.02956844260352	0.351203194158337\\
-1.66291465278432	0.539685603279386\\
-1.20226587375373	0.740783582827908\\
-0.659782514063547	0.937902771892843\\
-0.0736961714596904	1.10995578588396\\
0.501412124460094	1.24010882029349\\
1.01738293052278	1.32311336938\\
1.44914772331576	1.36482948316528\\
1.79451333687795	1.37623258172961\\
};
\addplot [color=mycolor1, forget plot]
  table[row sep=crcr]{%
1.875380198175	1.45912264622704\\
2.12532607372983	1.45151275523711\\
2.32049313714941	1.43312663690383\\
2.47351961229483	1.40883077969227\\
2.59454470957599	1.38164056702845\\
2.69129962768452	1.35334146430337\\
2.76954939823841	1.32494670095129\\
2.8335622883477	1.29699922662878\\
2.88650196898471	1.26976072903191\\
2.93072649099663	1.2433268912786\\
2.96800716044344	1.21769670697208\\
2.99968601153698	1.19281378547366\\
3.02678847325747	1.16859071790335\\
3.05010403242527	1.14492320634226\\
3.07024420039637	1.12169798193793\\
3.08768436550023	1.09879692266456\\
3.10279412762091	1.07609881054382\\
3.11585930640243	1.05347958179882\\
3.12709783060887	1.0308115652349\\
3.13667102518602	1.00796198065811\\
3.14469132014911	0.984790824374708\\
3.15122704385755	0.961148168735424\\
3.15630468366305	0.936870825940929\\
3.15990876251869	0.91177825868291\\
3.16197926064529	0.885667551171414\\
3.1624062783217	0.858307174811905\\
3.16102135810059	0.829429184553865\\
3.15758452313218	0.798719355335712\\
3.15176558918009	0.765804602668832\\
3.14311759426987	0.73023681658467\\
3.13103914923143	0.691471966427712\\
3.11472098212371	0.648843010440361\\
3.09306970442142	0.601524809885759\\
3.06459858593356	0.548489034635153\\
3.02727063241863	0.488447303179922\\
2.97827358720255	0.419782361339985\\
2.91370094648205	0.340471870835639\\
2.82811273852315	0.248021431188159\\
2.71396950645002	0.139450861872087\\
2.56100997894323	0.011434072810402\\
2.35585577163837	-0.139207939611686\\
2.08258910796728	-0.314338081602671\\
1.72574780809793	-0.512344302816052\\
1.27738430552161	-0.725264891854567\\
0.747410964471658	-0.937093700635873\\
0.170151684734942	-1.12661859938555\\
-0.403149960273714	-1.27559530863182\\
-0.924739677714599	-1.37654191730408\\
-1.36723218025528	-1.43348961008313\\
-1.72543970187286	-1.45669521773722\\
-2.00807292192314	-1.45706014702998\\
-2.22891235056166	-1.44331104409805\\
-2.40157978026183	-1.42149356437748\\
-2.53749727933407	-1.39546326605403\\
-2.64555530926629	-1.36755240066736\\
-2.7324408719078	-1.33911374280202\\
-2.80311577466288	-1.31089573584426\\
-2.86125293005364	-1.28328246873827\\
-2.90958126713142	-1.25644161240548\\
-2.95014237282275	-1.2304139170949\\
-2.98447637900907	-1.20516677800092\\
-3.01375518981585	-1.18062600865979\\
-3.03887778105092	-1.15669444918053\\
-3.0605385408469	-1.13326260668455\\
-3.07927649620449	-1.11021444400938\\
-3.09551093137147	-1.08743018066274\\
-3.10956722998844	-1.06478721623195\\
-3.12169559672649	-1.04215982907142\\
-3.13208449045607	-1.0194180211406\\
-3.14087001922426	-0.9964257010892\\
-3.14814212743523	-0.97303827856067\\
-3.15394809027255	-0.949099656396012\\
-3.15829357749059	-0.924438536709023\\
-3.16114132532165	-0.898863889641476\\
-3.16240723260219	-0.872159360523098\\
-3.16195344608523	-0.844076303511913\\
-3.15957768583335	-0.814325018612185\\
-3.15499763915335	-0.782563624334002\\
-3.14782865669049	-0.748383809417997\\
-3.13755212396971	-0.711292464246528\\
-3.12347062022884	-0.670687892829954\\
-3.10464412095523	-0.625828968409226\\
-3.07979879451023	-0.575795296373511\\
-3.04719609928311	-0.519436404776792\\
-3.00444475117117	-0.455308738835184\\
-2.9482322176708	-0.381602060475576\\
-2.87394842006821	-0.296064553876908\\
-2.77518113622301	-0.195954326761797\\
-2.64310416802847	-0.0780848431611926\\
-2.46591133519486	0.0608914956915911\\
-2.22877348479082	0.223698787331001\\
-1.91540232194173	0.410797198251262\\
-1.5129282868656	0.617755172649118\\
-1.02102501419702	0.832596173905048\\
-0.461678914241442	1.03596879881983\\
0.120322339013655	1.20694603259918\\
0.672695012185963	1.33204369199023\\
1.15664250869975	1.40994398642384\\
1.5565348565067	1.44859625100981\\
1.875380198175	1.45912264622704\\
};
\addplot [color=mycolor1, forget plot]
  table[row sep=crcr]{%
1.94860967361305	1.54255432833086\\
2.1805183787397	1.53548649136023\\
2.36285721209787	1.5183012557299\\
2.50702155221514	1.49540564898627\\
2.62204553358807	1.46955805053239\\
2.71480973562695	1.44242159207405\\
2.79046529403399	1.41496449889465\\
2.85285038384193	1.387724683507\\
2.90483097030853	1.36097714777595\\
2.94855954345884	1.33483762364216\\
2.98566504672178	1.30932602862144\\
3.01739029515218	1.28440503990512\\
3.04469096502283	1.26000335267682\\
3.06830696683483	1.23602950040386\\
3.08881408202697	1.21237982371928\\
3.10666146617262	1.18894277034559\\
3.12219895498825	1.16560084981681\\
3.1356969214283	1.14223103857391\\
3.14736059175103	1.1187041016261\\
3.15734013078588	1.09488308635246\\
3.16573737353531	1.07062110330677\\
3.17260975468686	1.04575840770177\\
3.17797172733908	1.02011871395454\\
3.18179373411327	0.993504600659352\\
3.18399856897966	0.965691784078038\\
3.18445471760671	0.936421945322298\\
3.18296595418294	0.905393680381121\\
3.17925605963709	0.872250992770669\\
3.17294694888811	0.836568554960588\\
3.16352766531549	0.79783271720215\\
3.15031049243682	0.755416938424182\\
3.13236866825522	0.70854997319778\\
3.10844763558118	0.656274849324365\\
3.07683817380715	0.59739663051999\\
3.03519503220057	0.530417717636257\\
2.98027938081541	0.453462247419244\\
2.90760015624451	0.364198731229946\\
2.81093650936482	0.259788036163911\\
2.6817630125737	0.136922409161699\\
2.50872093805967	-0.00790498392072498\\
2.2775720519001	-0.177646504168353\\
1.97261341510108	-0.373116462736028\\
1.58109419230744	-0.590422057659901\\
1.10155200657516	-0.818241223317062\\
0.553103473504285	-1.03757477755222\\
-0.023057140602249	-1.22686053176458\\
-0.576640773712478	-1.37080621485596\\
-1.06801045513199	-1.46595664204834\\
-1.4789594019965	-1.51886494276565\\
-1.80996604361589	-1.54031035988743\\
-2.07155855448015	-1.54064245584021\\
-2.27711730768827	-1.52783705568473\\
-2.43909171547293	-1.50736341137314\\
-2.56770027554192	-1.48272667724163\\
-2.67085281140652	-1.45607779784481\\
-2.75450984985187	-1.42869176395499\\
-2.82311814696074	-1.4012955399815\\
-2.87999250812671	-1.37427930959536\\
-2.92761450788321	-1.34782843907536\\
-2.96785477081937	-1.32200465604213\\
-3.00213469242636	-1.29679556598416\\
-3.03154308672126	-1.27214464320716\\
-3.05692021372257	-1.24796920448344\\
-3.0789184572603	-1.22417095912209\\
-3.098046311607	-1.20064193399821\\
-3.11470037786735	-1.17726747430447\\
-3.1291886612983	-1.15392734812871\\
-3.14174746141081	-1.13049556681816\\
-3.15255343902769	-1.10683927077373\\
-3.16173193739188	-1.0828168593773\\
-3.16936226115321	-1.05827542579354\\
-3.17548032873782	-1.03304746813961\\
-3.18007887317197	-1.00694677181647\\
-3.18310514345613	-0.979763281858598\\
-3.18445582470331	-0.951256699217872\\
-3.1839686197656	-0.921148431652043\\
-3.18140957999802	-0.889111398528125\\
-3.17645478670739	-0.854757018710966\\
-3.16866429456784	-0.817618491103143\\
-3.15744524822052	-0.777129201102349\\
-3.1419996226369	-0.732594759038347\\
-3.12124991115256	-0.683156840821503\\
-3.09373304327001	-0.627746789871623\\
-3.05744864793884	-0.56502719964194\\
-3.00964261337217	-0.493321254018828\\
-2.94650206075707	-0.410534317308221\\
-2.86273816213399	-0.314084086158319\\
-2.75105243695848	-0.200882268889368\\
-2.60155406129097	-0.0674650913852828\\
-2.40138990312832	0.0895347979608175\\
-2.13526444165352	0.272262905902826\\
-1.78815012598701	0.479549466436689\\
-1.35169578093293	0.704055024761356\\
-0.833802555613541	0.930355778532272\\
-0.26531739322843	1.13717633843827\\
0.305598348497514	1.30500708020502\\
0.831763869877206	1.42424215832089\\
1.28384927376364	1.49704828505071\\
1.65393178900214	1.5328286466653\\
1.94860967361305	1.54255432833086\\
};
\addplot [color=mycolor1, forget plot]
  table[row sep=crcr]{%
2.01533812164163	1.62604405276322\\
2.23078249240264	1.61947135062162\\
2.40132090076523	1.60339148684674\\
2.53722108374555	1.58180236172912\\
2.64654784552852	1.55723001253831\\
2.73543985727512	1.53122219873246\\
2.80850990487305	1.50470009413296\\
2.86921499079895	1.47819106428946\\
2.92015375315593	1.45197729022359\\
2.96329141677968	1.42618900054377\\
3.00012525770858	1.40086244377317\\
3.03180488741777	1.37597573400241\\
3.05921940824373	1.35147086268574\\
3.08306063861848	1.32726703647262\\
3.10386911975224	1.30326853228884\\
3.12206769401675	1.27936903780104\\
3.1379860358974	1.25545368599787\\
3.15187850511821	1.2313995175315\\
3.16393696944925	1.20707480229813\\
3.17429972452134	1.18233745375814\\
3.18305725410281	1.15703263297448\\
3.19025527804017	1.13098953708787\\
3.1958952892501	1.1040172808161\\
3.19993255524039	1.07589969643826\\
3.20227132538617	1.04638878691185\\
3.20275671237797	1.01519645831191\\
3.20116236797684	0.981984021383658\\
3.19717260000278	0.946348776642342\\
3.19035690968367	0.907806772208036\\
3.18013396639077	0.865770541065649\\
3.1657206394711	0.819520290456693\\
3.14605968485133	0.768166672827977\\
3.11971681260298	0.710603047605297\\
3.08473396396719	0.645445388655745\\
3.03842085756421	0.570959515297282\\
2.9770625239435	0.484979939102609\\
2.89552117857411	0.384836153263169\\
2.78672909121182	0.267328088123135\\
2.64113583717848	0.128845047388666\\
2.44635087067706	-0.0341853588117058\\
2.18759873351322	-0.224214143276535\\
1.85017027932879	-0.440534454589518\\
1.42527425881919	-0.676434248589246\\
0.918975998779657	-0.917067119843662\\
0.359018686232294	-1.14112455367898\\
-0.209233416725197	-1.32792310389627\\
-0.739226389230772	-1.46581173965135\\
-1.19996215139496	-1.55507001437202\\
-1.58102344345021	-1.60414390620066\\
-1.88700244748479	-1.62396754081746\\
-2.12940113609882	-1.62426956441044\\
-2.32097043321478	-1.61232871930589\\
-2.4730471142931	-1.59309969429799\\
-2.59478196509427	-1.56977421460537\\
-2.69322879116622	-1.54433650080771\\
-2.77371333840301	-1.51798531534492\\
-2.84022849832474	-1.49142186722937\\
-2.89576968323859	-1.46503638282487\\
-2.94259477266895	-1.43902599538774\\
-2.98241747368191	-1.41346832118405\\
-3.01654851341223	-1.38836707616566\\
-3.04599803371542	-1.36368019711536\\
-3.07154980001532	-1.33933701616698\\
-3.09381511158766	-1.31524854837838\\
-3.11327209412083	-1.29131339930806\\
-3.13029440294031	-1.26742083623812\\
-3.14517216961788	-1.24345196690316\\
-3.15812717037828	-1.2192795917351\\
-3.16932358304173	-1.19476705222343\\
-3.17887525422785	-1.16976623492159\\
-3.18685006373035	-1.14411477395435\\
-3.19327170606398	-1.11763240256517\\
-3.1981189768976	-1.090116321054\\
-3.20132242555929	-1.06133536274487\\
-3.2027579853432	-1.03102264123454\\
-3.20223688804579	-0.998866241078235\\
-3.19949076565381	-0.964497359602382\\
-3.19415028193235	-0.927475108504999\\
-3.1857148364916	-0.887266930464672\\
-3.1735097247936	-0.843223275531408\\
-3.15662545454323	-0.794544834323392\\
-3.1338315005953	-0.740240317604256\\
-3.10345340779718	-0.679072720821543\\
-3.06319775596855	-0.6094927493767\\
-3.00990465381885	-0.529560827287165\\
-2.93920465871892	-0.436866525052448\\
-2.84506407514115	-0.328471731263217\\
-2.71923958708128	-0.200941316919633\\
-2.55077516534008	-0.0505963821734142\\
-2.32594077898868	0.125764016628224\\
-2.02950284685265	0.329331705284922\\
-1.64873641905498	0.556766844342745\\
-1.18109928201364	0.797398721387658\\
-0.643132582441513	1.03258585633054\\
-0.0728119622014619	1.24019417476993\\
0.481442381812748	1.40322281470281\\
0.979351688292782	1.51611243646592\\
1.40041618766564	1.5839478109451\\
1.74275843194902	1.61705145338868\\
2.01533812164163	1.62604405276322\\
};
\addplot [color=mycolor1, forget plot]
  table[row sep=crcr]{%
2.07625084362876	1.70905454066351\\
2.27657171102993	1.7029371442451\\
2.43616868402318	1.68788290929235\\
2.56429398464666	1.66752361618015\\
2.66816102380818	1.64417396855024\\
2.7532589566817	1.61927255932821\\
2.82372647505532	1.59369206321064\\
2.88268142697289	1.56794476004416\\
2.93248158859062	1.54231477478672\\
2.97492112869881	1.51694197169365\\
3.01137531548528	1.49187481670289\\
3.04290611533935	1.46710353913919\\
3.07033908435744	1.44258080620787\\
3.09431943631997	1.41823444553482\\
3.1153530375915	1.39397505235657\\
3.13383644291012	1.36970025017026\\
3.15007888440732	1.34529670157174\\
3.1643182593009	1.3206405388398\\
3.17673253706303	1.29559660685734\\
3.18744755077955	1.27001672492065\\
3.19654179403765	1.24373704136143\\
3.20404857173142	1.21657445123097\\
3.20995561726957	1.18832195568965\\
3.2142020609549	1.15874274941089\\
3.21667238637311	1.12756271826917\\
3.21718671125588	1.09446090317859\\
3.21548633545506	1.05905732599157\\
3.21121295525575	1.02089736817394\\
3.2038791708604	0.979431632163099\\
3.19282679991494	0.933989895601004\\
3.17716789949916	0.88374740859986\\
3.15570109554437	0.827681459205336\\
3.12679262240682	0.764516053245468\\
3.08820733464323	0.692653240751548\\
3.03687042097796	0.610092261916117\\
2.96853800615594	0.51434486165235\\
2.8773614535736	0.402372075235921\\
2.75536443681216	0.27060407493264\\
2.59195470341043	0.115173991350096\\
2.37383613389199	-0.0673944967456265\\
2.08613598372201	-0.278706153942123\\
1.71605520807854	-0.516007374804534\\
1.25998293614149	-0.769296722337668\\
0.732095567027905	-1.02030053062889\\
0.167467140251177	-1.24634563118402\\
-0.387183572051126	-1.42877179658333\\
-0.891025359473241	-1.55991959660899\\
-1.32146597833854	-1.64333876514649\\
-1.6744246737234	-1.68880234054156\\
-1.95736105292541	-1.70713148781447\\
-2.18216337866193	-1.70740611297996\\
-2.36083222890854	-1.69626317837733\\
-2.50366964198596	-1.67819669876535\\
-2.618880768512	-1.6561163679259\\
-2.71276999592862	-1.63185226658598\\
-2.79010612252365	-1.60652857080875\\
-2.8544804346866	-1.58081732677704\\
-2.90860242718634	-1.55510371807489\\
-2.9545270643359	-1.52959151788961\\
-2.99382369479805	-1.50436972387311\\
-3.02769974333328	-1.47945445076631\\
-3.05709081854048	-1.45481514211133\\
-3.08272635217036	-1.43039082628619\\
-3.10517752533419	-1.40610000507192\\
-3.12489235374168	-1.38184641564834\\
-3.14222139692114	-1.35752206036053\\
-3.15743653465407	-1.33300836373815\\
-3.17074451852888	-1.30817597341254\\
-3.18229647416781	-1.28288349586974\\
-3.19219413596961	-1.25697530235906\\
-3.20049329266957	-1.23027842460988\\
-3.20720467138463	-1.20259846415877\\
-3.21229225948476	-1.17371434856197\\
-3.21566882949251	-1.14337167090636\\
-3.21718816250861	-1.11127423516403\\
-3.21663312446418	-1.07707328820045\\
-3.213698289137	-1.0403537381232\\
-3.2079651557215	-1.00061642671769\\
-3.19886708211181	-0.957255233004433\\
-3.18563971506946	-0.909527440378471\\
-3.16725076793578	-0.856515441995987\\
-3.1423002681693	-0.797077617113549\\
-3.1088787136591	-0.729786419488706\\
-3.06436609974594	-0.652853169202736\\
-3.00515074354965	-0.564043495916574\\
-2.92624747803168	-0.460598558864161\\
-2.82081182873313	-0.339202275322706\\
-2.67960829495215	-0.196085812311738\\
-2.49065346318377	-0.027450510797915\\
-2.23959860846959	0.169492024219199\\
-1.91193949263403	0.394534069462105\\
-1.49838501044073	0.641617587095279\\
-1.00327221339828	0.896484499037834\\
-0.451531360927953	1.13781127815106\\
0.113997515571919	1.34378714168818\\
0.647354154723447	1.50075046197496\\
1.11604264788142	1.60706016135398\\
1.50733999281103	1.67011773368875\\
1.82394103035216	1.70073476993218\\
2.07625084362876	1.70905454066351\\
};
\addplot [color=mycolor1, forget plot]
  table[row sep=crcr]{%
2.1317414423095	1.79100753091255\\
2.31810513092557	1.78531088071093\\
2.46750057327107	1.77121361329816\\
2.58826913120071	1.75201867108555\\
2.6868750793552	1.7298478205699\\
2.76823712944396	1.70603634702307\\
2.83607514814713	1.68140766269239\\
2.89320444625466	1.65645536988618\\
2.94176541488176	1.63146116440283\\
2.9833958590435	1.60657035406556\\
3.01935802552462	1.58183998349703\\
3.05063158093467	1.55726939291785\\
3.07798158660956	1.5328195071364\\
3.102008275447	1.50842484691612\\
3.12318358668422	1.48400078265371\\
3.14187800771922	1.45944761539587\\
3.15838023912751	1.43465247446497\\
3.17291145028976	1.40948963655129\\
3.1856353486624	1.3838196170224\\
3.19666488245242	1.35748720900995\\
3.20606608599172	1.33031851643538\\
3.21385932206391	1.30211692135325\\
3.22001794513899	1.27265782792174\\
3.22446417591456	1.24168192229003\\
3.22706171143083	1.20888656827981\\
3.22760426130214	1.17391481141768\\
3.22579875322447	1.1363412764689\\
3.22124132591228	1.09565400392063\\
3.21338333449859	1.0512309698093\\
3.20148330571805	1.00230967401717\\
3.18453892980651	0.947947802132128\\
3.16119056676129	0.886972688037045\\
3.12958422734702	0.817917450365342\\
3.08717771115147	0.738943010388461\\
3.03046968605458	0.647749429602825\\
2.9546318816494	0.541490732974257\\
2.85303999781711	0.416731621646288\\
2.71675510959344	0.269534018621001\\
2.53415623186526	0.0958496398074653\\
2.29124129015289	-0.107486009875524\\
1.97360510470139	-0.340815789433208\\
1.57137861878927	-0.598787897134276\\
1.08721218216131	-0.867771357129065\\
0.543419735910371	-1.1264524054491\\
-0.0194546428704111	-1.35190795116236\\
-0.555956970215128	-1.52845094451582\\
-1.03216247322138	-1.65245572918629\\
-1.4331592886672	-1.73019094646337\\
-1.75983247900571	-1.77227422021084\\
-2.02153685237199	-1.78922555794723\\
-2.23014518586385	-1.7894752978946\\
-2.39685537138576	-1.77907268401652\\
-2.53101872335505	-1.76209832106834\\
-2.6400030291327	-1.74120709171542\\
-2.72945476977381	-1.71808621165136\\
-2.80365248373414	-1.6937872099948\\
-2.86583106514623	-1.66895044366125\\
-2.91844350981957	-1.6439518970657\\
-2.96336049115918	-1.61899762638342\\
-3.00201852468984	-1.59418407199124\\
-3.03552869182068	-1.56953642646061\\
-3.06475613262073	-1.54503293912038\\
-3.0903781973561	-1.52062017474582\\
-3.11292708052982	-1.4962223993832\\
-3.13282113780473	-1.4717470929754\\
-3.150387877346	-1.44708784287421\\
-3.16588073720748	-1.42212539481393\\
-3.17949112278348	-1.39672732661129\\
-3.19135671140971	-1.3707466002794\\
-3.20156667958285	-1.34401909927058\\
-3.21016422967143	-1.31636014226515\\
-3.21714655389592	-1.28755986472617\\
-3.22246214491573	-1.25737726036396\\
-3.22600511642653	-1.22553256485005\\
-3.2276059021702	-1.1916975322311\\
-3.22701731790171	-1.15548298868904\\
-3.22389444409388	-1.11642283618932\\
-3.21776604144455	-1.07395340886645\\
-3.20799413922056	-1.02738675304841\\
-3.19371689091799	-0.975876023855393\\
-3.17376758645825	-0.918370838279727\\
-3.14655965614383	-0.853560298705166\\
-3.10992354808926	-0.7798020024311\\
-3.06087701812257	-0.69503783253758\\
-2.9953078345687	-0.596704182786383\\
-2.9075537768713	-0.481660592789777\\
-2.78989601647332	-0.346195893384541\\
-2.63207603324366	-0.186238557768118\\
-2.42116974669818	0.00199399013596253\\
-2.14256889254755	0.220565435545492\\
-1.78329827876146	0.467362244745593\\
-1.33864880940448	0.733098980637337\\
-0.820555955097994	0.999900587110749\\
-0.261428776333567	1.24457532845698\\
0.293542875111646	1.44680762682395\\
0.802975958022219	1.59679851197821\\
1.24229289017552	1.69648104919221\\
1.60531664101736	1.75499481314277\\
1.89807274077814	1.78330660078127\\
2.1317414423095	1.79100753091255\\
};
\addplot [color=mycolor1, forget plot]
  table[row sep=crcr]{%
2.18202401463123	1.87130548893601\\
2.35546460910354	1.8659989740022\\
2.49531402264379	1.85279783305457\\
2.60909837557366	1.83470888706013\\
2.70262154246927	1.81367741452399\\
2.78029989221005	1.79094111660265\\
2.84548241910839	1.76727406506573\\
2.90071413752772	1.74314848567836\\
2.94793913106009	1.71884008947843\\
2.98865234844797	1.69449608538423\\
3.02401151948419	1.67017893583495\\
3.05491925982564	1.64589442281208\\
3.08208328967226	1.62160953907504\\
3.10606067976851	1.59726372493208\\
3.12729042043629	1.57277568874334\\
3.14611738903238	1.54804722797296\\
3.16280989598737	1.52296493819135\\
3.17757233784711	1.49740035074913\\
3.19055400743643	1.47120880606978\\
3.20185475129721	1.4442272037834\\
3.21152788042525	1.41627064377201\\
3.21958049832703	1.38712786329406\\
3.2259711816471	1.35655526949354\\
3.23060470516282	1.32426925100999\\
3.23332321384119	1.28993631496692\\
3.23389287124117	1.25316042374075\\
3.23198450364748	1.21346668630825\\
3.22714603978494	1.17028027905468\\
3.21876351368187	1.12289912394304\\
3.20600591176761	1.07045845048474\\
3.18774702488831	1.01188497623521\\
3.16245452589533	0.945838247822802\\
3.12803267039657	0.870637166155992\\
3.08160075712392	0.784171977175059\\
3.01918676590626	0.683808451374954\\
2.93532042725459	0.566306538405956\\
2.82253783362597	0.427809633335044\\
2.67089509680695	0.26402657173068\\
2.46779379775515	0.0708365130902191\\
2.19881035353723	-0.1543370372527\\
1.85069554742905	-0.410094596109592\\
1.41757943731687	-0.687947573759316\\
0.909196433566452	-0.970485391769853\\
0.355432238463405	-1.23402580139253\\
-0.199907051267903	-1.45656752040886\\
-0.714824033161	-1.62608083233949\\
-1.16276950598761	-1.74276677810291\\
-1.53551456365519	-1.81504155347651\\
-1.83769667471329	-1.85397291447147\\
-2.07982491225251	-1.86965351824121\\
-2.27348859687057	-1.86988069903074\\
-2.42907384371724	-1.86016742457127\\
-2.55506502775769	-1.84422262288643\\
-2.6580880518937	-1.82447035710515\\
-2.74321038548067	-1.80246535066994\\
-2.81427788442654	-1.77918880243121\\
-2.87420854073542	-1.75524769527952\\
-2.92522499346317	-1.73100553369938\\
-2.96903053613899	-1.70666703771278\\
-3.00693959320492	-1.68233271304135\\
-3.03997360262699	-1.65803391122278\\
-3.06893132085974	-1.63375526384628\\
-3.09444042702253	-1.60944889993664\\
-3.11699547832575	-1.58504325462593\\
-3.13698585647355	-1.56044825124497\\
-3.15471629743739	-1.53555798040615\\
-3.17042183318646	-1.51025157238872\\
-3.18427841674602	-1.48439267580391\\
-3.19641008883668	-1.4578277602352\\
-3.20689322696851	-1.4303833171076\\
-3.21575815862114	-1.40186191715796\\
-3.22298818821358	-1.37203697719307\\
-3.22851585503685	-1.34064597961906\\
-3.23221597727604	-1.30738176323769\\
-3.2338947115149	-1.27188135077356\\
-3.23327342309961	-1.23371158453574\\
-3.2299655588147	-1.19235059325209\\
-3.22344385254215	-1.14716379976709\\
-3.21299395546891	-1.09737280136007\\
-3.1976488032231	-1.04201504444782\\
-3.17609552255039	-0.979891886962476\\
-3.14654328864766	-0.909502698571209\\
-3.10653638634683	-0.828963820567699\\
-3.05269282312502	-0.735915134149284\\
-2.98034873213667	-0.627427148135196\\
-2.88310229766976	-0.499944753247218\\
-2.75230159650874	-0.349351808342394\\
-2.57665665987365	-0.171327690835457\\
-2.34244890996455	0.0377108016772661\\
-2.03527926982411	0.278721927644738\\
-1.64461155087895	0.547139911134911\\
-1.17139168307723	0.830037161564239\\
-0.635415342870461	1.10615942595841\\
-0.0750890768055588	1.35147228458665\\
0.464530598780093	1.54820045262041\\
0.948070919357095	1.69062377979606\\
1.35846538811358	1.78377048989356\\
1.69483670196028	1.83799684704665\\
1.96553148603393	1.86417465145828\\
2.18202401463123	1.87130548893601\\
};
\addplot [color=mycolor1]
  table[row sep=crcr]{%
2.22721472973942	1.94935886896179\\
2.3886648425121	1.94441487511101\\
2.5195634483927	1.93205457572182\\
2.62670779849127	1.91501766628131\\
2.71531853364	1.89508786650117\\
2.78936864949369	1.87341097537522\\
2.85187814440773	1.85071228241933\\
2.90515032165589	1.82744075521499\\
2.9509520177711	1.80386332654994\\
2.99064783647547	1.78012618777659\\
3.02529909977035	1.7562945351756\\
3.0557365623221	1.73237827445755\\
3.08261387510984	1.7083485310185\\
3.10644696879361	1.68414807630787\\
3.12764310033082	1.65969765855822\\
3.14652223914943	1.63489950162782\\
3.16333268711427	1.60963876367151\\
3.17826225469757	1.5837834336332\\
3.19144589188063	1.55718292750329\\
3.20297034763997	1.52966548843736\\
3.2128761683912	1.50103436869378\\
3.2211571124149	1.47106265802835\\
3.22775682617807	1.43948650786522\\
3.23256237104755	1.40599637010381\\
3.23539387180292	1.3702257107414\\
3.23598913837185	1.33173645766061\\
3.23398152983158	1.29000018434938\\
3.22886850103885	1.24437370419161\\
3.21996707976175	1.19406735000614\\
3.20635080827997	1.13810376656035\\
3.18676025898309	1.07526465092859\\
3.15947593702972	1.0040228169715\\
3.12213828317303	0.922457920675385\\
3.07149545967596	0.828157721349833\\
3.00305879148097	0.718116213117772\\
2.91065699973037	0.588662002756959\\
2.785925244978	0.435496623074383\\
2.61788823849866	0.254008232649376\\
2.39306798253518	0.0401523940333445\\
2.09699730361163	-0.20771900183087\\
1.7183819635995	-0.485932899271695\\
1.25639683089422	-0.782386182967809\\
0.728319451607428	-1.07597695395358\\
0.17048677629427	-1.34156752261761\\
-0.372344302750219	-1.55919115860543\\
-0.863264272549092	-1.72086595365401\\
-1.28299919742863	-1.8302347441912\\
-1.62890152595063	-1.89731652670458\\
-1.90833376158261	-1.93331843320936\\
-2.13240654410545	-1.94782687646357\\
-2.3122537087605	-1.94803363323748\\
-2.45746747168844	-1.93896359662991\\
-2.57574556593285	-1.92399109460735\\
-2.67305603610964	-1.90533078490532\\
-2.75395564612754	-1.88441460443591\\
-2.82190758462804	-1.86215611801615\\
-2.87954766726586	-1.83912801979308\\
-2.92889157233296	-1.81567886095776\\
-2.97149068204595	-1.79200910352774\\
-3.00854741762708	-1.76822049003994\\
-3.04100007190525	-1.74434802267285\\
-3.0695851473103	-1.7203805908067\\
-3.09488323143276	-1.69627413213334\\
-3.11735281839452	-1.67195981527391\\
-3.13735524534064	-1.64734883071173\\
-3.15517299859553	-1.62233479306363\\
-3.17102297575804	-1.59679437416259\\
-3.18506579905015	-1.570586527561\\
-3.19741190629232	-1.54355048208521\\
-3.20812485602917	-1.51550254281508\\
-3.21722203851139	-1.48623162017347\\
-3.22467275544594	-1.45549329519237\\
-3.23039339092638	-1.42300210774249\\
-3.23423911335776	-1.38842161161915\\
-3.23599118590349	-1.35135156242766\\
-3.23533847055249	-1.31131137681683\\
-3.23185101774822	-1.2677187105326\\
-3.22494264006069	-1.21986163848029\\
-3.2138179373885	-1.16686249014231\\
-3.19739719631547	-1.10763095340298\\
-3.17420973909784	-1.0408037811618\\
-3.142242561071	-0.964668760862181\\
-3.09872682936675	-0.877072578179989\\
-3.0398417586358	-0.77531812311728\\
-2.96031892266926	-0.656071422287709\\
-2.85295425309884	-0.515330753928937\\
-2.70811161177175	-0.348574764250833\\
-2.51348980394932	-0.151314840167112\\
-2.25479111797749	0.0795960527623154\\
-1.91840618384291	0.343563906305597\\
-1.49721915275415	0.633014609293368\\
-0.998726638487032	0.931115827109969\\
-0.450328181715587	1.21375185121395\\
0.105465468340197	1.45718721810745\\
0.625945116714116	1.64701594620063\\
1.08251666630864	1.78154116777267\\
1.46486936946524	1.86834353319716\\
1.77626331377619	1.91854897142989\\
2.02656829443923	1.94275375393752\\
2.22721472973941	1.94935886896179\\
};
\addlegendentry{Аппроксимации}

\addplot [color=mycolor2]
  table[row sep=crcr]{%
2.12132034355964	2.82842712474619\\
2.18582575000323	2.82640660874952\\
2.2459761483502	2.82067996094826\\
2.30316600770785	2.81153894842004\\
2.35833530419819	2.79908319427389\\
2.4121407651283	2.78328504568152\\
2.46505313939319	2.76402339892432\\
2.51741350921551	2.74110157112695\\
2.56946568278775	2.71425682483196\\
2.62137377470791	2.68316571665979\\
2.67322998558975	2.64744786912504\\
2.72505542741334	2.60667010622203\\
2.77679571750944	2.56035271147659\\
2.82831256449113	2.50797963523731\\
2.87937250576692	2.44901464186941\\
2.92963424895255	2.38292549191834\\
2.97863667364875	2.30921809960124\\
3.02579039162557	2.22748193169334\\
3.0703766721051	2.13744642550811\\
3.11155818595542	2.03904567139316\\
3.14840590189045	1.93248505458784\\
3.1799449834646	1.81829951655162\\
3.20521925970046	1.69738980361013\\
3.2233689191928	1.57102231442397\\
3.23371058284854	1.44078163681543\\
3.23580480244767	1.30847305653245\\
3.22949541157146	1.1759834831\\
3.21490900216393	1.04511946165368\\
3.19240995273442	0.917445713323391\\
3.16251366135309	0.794144173493542\\
3.12576385298257	0.675902054710668\\
3.08257536776515	0.562820516214273\\
3.03302813706062	0.454314775018543\\
2.97656426645917	0.348948817128694\\
2.91146921429886	0.244099783305823\\
2.83385680006128	0.135243682019644\\
2.73548220960228	0.0144130439331602\\
2.59871334341581	-0.133168683665202\\
2.38479638365891	-0.336180712113444\\
2.01057175466146	-0.648371525543042\\
1.34210041814469	-1.13791067535239\\
0.375352321166521	-1.75822148716338\\
-0.527448468220416	-2.26262801632552\\
-1.11365847778129	-2.5437869612038\\
-1.4567327767317	-2.68228003389134\\
-1.6691296256022	-2.75259678373195\\
-1.81428800439327	-2.79054335642282\\
-1.92320462744895	-2.81168897715777\\
-2.01132392805712	-2.82302365363059\\
-2.08680248126988	-2.82787464184343\\
-2.15422943553058	-2.82790871121207\\
-2.21634478307291	-2.8239827357917\\
-2.27487422572476	-2.8165264847195\\
-2.33095693947289	-2.80572465930185\\
-2.3853755403757	-2.79160756168836\\
-2.43868469976231	-2.77409782070174\\
-2.49128507004909	-2.75303495626884\\
-2.54346612600745	-2.7281884019911\\
-2.59543032955127	-2.69926454755406\\
-2.64730535029083	-2.66591103001887\\
-2.69914810041563	-2.62772046423944\\
-2.75094277189846	-2.58423541950896\\
-2.80259428647749	-2.53495641420315\\
-2.85391830194335	-2.47935483422118\\
-2.90462903903618	-2.41689283661657\\
-2.95432664759112	-2.34705230341292\\
-3.00248656709011	-2.26937453210355\\
-3.04845424185856	-2.18351130413452\\
-3.0914493816545	-2.08928598065547\\
-3.13058429431929	-1.98676020448289\\
-3.16490008683964	-1.87629788656397\\
-3.19342217916266	-1.75861429449935\\
-3.21523240499265	-1.63479579561046\\
-3.22954956680171	-1.50627700312196\\
-3.23580523135335	-1.37476797309193\\
-3.23369896357047	-1.2421341494958\\
-3.22321880671614	-1.11024302689198\\
-3.20461860896113	-0.980799507713277\\
-3.17835148049538	-0.85519274756288\\
-3.14496439092874	-0.734369602871579\\
-3.10495870881183	-0.618735254776386\\
-3.05861192744495	-0.508062722468019\\
-3.0057323937692	-0.401369642978784\\
-2.94527025004781	-0.296685462718974\\
-2.87460204254734	-0.190562899149911\\
-2.78805571026974	-0.0770301658026245\\
-2.67361421522119	0.0546866392302208\\
-2.50520482475694	0.225105040795738\\
-2.22541867988909	0.474087715816021\\
-1.7209533613293	0.868461990547897\\
-0.880186083660502	1.44500069250654\\
0.110323684998138	2.03883159526995\\
0.859259206598599	2.42737283484743\\
1.30741523004191	2.62511520505649\\
1.57426953542748	2.72304263504847\\
1.74766615441039	2.77435113512036\\
1.87207484559155	2.80265686669549\\
1.96924085114733	2.81832059141065\\
2.0502990590364	2.82612977760851\\
2.12132034355964	2.82842712474619\\
};
\addlegendentry{Сумма Минковского}

\end{axis}

\begin{axis}[%
width=0.798\linewidth,
height=0.597\linewidth,
at={(-0.104\linewidth,-0.066\linewidth)},
scale only axis,
xmin=0,
xmax=1,
ymin=0,
ymax=1,
axis line style={draw=none},
ticks=none,
axis x line*=bottom,
axis y line*=left,
legend style={legend cell align=left, align=left, draw=white!15!black}
]
\end{axis}
\end{tikzpicture}%
        \caption{Эллипсоидальные аппроксимации для 100 направлений.}
\end{figure}
\begin{figure}[b]

        \centering
        % This file was created by matlab2tikz.
%
%The latest updates can be retrieved from
%  http://www.mathworks.com/matlabcentral/fileexchange/22022-matlab2tikz-matlab2tikz
%where you can also make suggestions and rate matlab2tikz.
%
\definecolor{mycolor1}{rgb}{0.00000,0.44700,0.74100}%
\definecolor{mycolor2}{rgb}{0.85000,0.32500,0.09800}%
%
\begin{tikzpicture}

\begin{axis}[%
width=0.618\linewidth,
height=0.471\linewidth,
at={(0\linewidth,0\linewidth)},
scale only axis,
xmin=-8,
xmax=8,
xlabel style={font=\color{white!15!black}},
xlabel={$x_1$},
ymin=-6,
ymax=6,
ylabel style={font=\color{white!15!black}},
ylabel={$x_2$},
axis background/.style={fill=white},
axis x line*=bottom,
axis y line*=left,
xmajorgrids,
ymajorgrids,
legend style={at={(0.03,0.97)}, anchor=north west, legend cell align=left, align=left, draw=white!15!black}
]
\addplot [color=mycolor1, forget plot]
  table[row sep=crcr]{%
1.1711787112841	3.34230777773576\\
1.34444389618673	3.33686722689733\\
1.50648318534277	3.32144701348819\\
1.65806242150429	3.29723890999844\\
1.79995419924852	3.26523166869947\\
1.93290799118999	3.22622794766654\\
2.05763032437368	3.18086276334402\\
2.17477223041259	3.12962150018857\\
2.28492171716205	3.07285634836238\\
2.3885995032216	3.01080060567632\\
2.48625667387058	2.94358064140543\\
2.57827324651315	2.87122553999274\\
2.66495688065443	2.79367457178694\\
2.74654114431786	2.71078271324095\\
2.82318287015103	2.62232448808267\\
2.8949582146523	2.52799644473262\\
2.96185708693385	2.42741864049226\\
3.02377565347494	2.3201355847161\\
3.0805066685131	2.20561721602438\\
3.13172744617751	2.08326066796132\\
3.17698540702193	1.95239382914097\\
3.21568133490354	1.81228204209558\\
3.24705081945703	1.66213971753183\\
3.27014489780722	1.50114915910447\\
3.28381172002077	1.32848945666784\\
3.28668221628109	1.14337881243048\\
3.2771642749897	0.945133918403077\\
3.25345178932588	0.733249674250213\\
3.21355684151227	0.507501133889645\\
3.15537469570401	0.268066487797808\\
3.07679115926015	0.015664562009573\\
2.97583883725379	-0.248307401701393\\
2.85090136682411	-0.521659161442773\\
2.70095216778915	-0.801347823435408\\
2.52579793500058	-1.08350783899399\\
2.32628156602984	-1.3635980584876\\
2.10439191407115	-1.63667268179468\\
1.86323617293213	-1.89775167737354\\
1.60685736899554	-2.14223401489281\\
1.33991789027605	-2.3662777508081\\
1.06730585933085	-2.56707514274059\\
0.793739425298697	-2.74297851870391\\
0.523437853608312	-2.89347215049445\\
0.259902596754298	-3.0190207516634\\
0.00581903236219554	-3.1208447467499\\
-0.2369372555121	-3.20067372983419\\
-0.467220354507	-3.2605178574641\\
-0.684493265340969	-3.30248032509504\\
-0.888694234930048	-3.32861897022119\\
-1.0801092196493	-3.34085455680139\\
-1.25926106809304	-3.34091776153004\\
-1.42681950330378	-3.33032515794113\\
-1.58353177449541	-3.31037507736533\\
-1.73017158467123	-3.2821558685242\\
-1.86750300244094	-3.24656096722397\\
-1.99625599202063	-3.20430689704777\\
-2.11711054093165	-3.15595168681459\\
-2.23068687011842	-3.10191219445658\\
-2.33753972875556	-3.04247952074268\\
-2.43815523449964	-2.97783215102365\\
-2.5329490934295	-2.90804674702345\\
-2.6222653208492	-2.83310668006404\\
-2.70637479428206	-2.75290849531896\\
-2.78547311726505	-2.66726655590999\\
-2.85967737154948	-2.5759161597878\\
-2.92902140016315	-2.47851546990414\\
-2.99344930844609	-2.37464666482809\\
-3.05280690958282	-2.26381681735669\\
-3.10683089311493	-2.14545915779927\\
-3.15513558252945	-2.01893559215682\\
-3.19719730262603	-1.88354163878489\\
-3.23233664163192	-1.73851533219976\\
-3.25969932318791	-1.58305212075492\\
-3.27823706849968	-1.41632833369867\\
-3.28669080350863	-1.23753634540448\\
-3.28357990805066	-1.04593497683649\\
-3.26720291066679	-0.840918687724783\\
-3.23565696131938	-0.622108318461704\\
-3.18688516768166	-0.389463978954261\\
-3.11876167995196	-0.143416533926881\\
-3.02922302993614	0.11499241610004\\
-2.91644915653916	0.383980836121862\\
-2.77908754849337	0.660934801326638\\
-2.61649915361415	0.942389492277687\\
-2.42898799999709	1.22411526429499\\
-2.21796402213223	1.50132757866017\\
-1.9859883878281	1.76901289011117\\
-1.7366683581316	2.02232922526914\\
-1.47440268220478	2.25701269019761\\
-1.20401775520347	2.46971278707818\\
-0.930363383198716	2.65819606075749\\
-0.657943143023997	2.82139307813531\\
-0.390637102663735	2.95930311865273\\
-0.131543811712393	3.07279931782965\\
0.117062449314586	3.1633871238911\\
0.353684227850852	3.23296276733506\\
0.577496523037587	3.28360333410188\\
0.788216624026908	3.3174035782658\\
0.985971690059736	3.33636162404406\\
1.1711787112841	3.34230777773576\\
};
\addplot [color=mycolor2, forget plot]
  table[row sep=crcr]{%
2.36712178368749	2.74633057020532\\
2.41188305682678	2.74494651398331\\
2.45036041459025	2.74130089565813\\
2.48383536069915	2.73596692388682\\
2.51327948444918	2.72933446739078\\
2.53943919311047	2.72166753585199\\
2.56289492242467	2.71314181506891\\
2.58410302011837	2.70386940753685\\
2.60342571833248	2.69391520541791\\
2.62115280082523	2.68330766515176\\
2.63751738560017	2.67204572812381\\
2.65270745907852	2.66010298488441\\
2.66687427014079	2.64742976011011\\
2.68013832873481	2.63395351039579\\
2.69259349381253	2.61957772004863\\
2.70430943819101	2.6041793136995\\
2.71533261385189	2.58760445132293\\
2.72568568587575	2.56966240776041\\
2.73536523285703	2.55011704085516\\
2.74433729805113	2.52867508999008\\
2.75253007967915	2.50497017880853\\
2.75982261143016	2.47854086090236\\
2.76602760995085	2.44880025137829\\
2.77086559717422	2.41499358451471\\
2.77392567575526	2.37613820122431\\
2.77460549370625	2.33093765236215\\
2.77201821305984	2.27765729690984\\
2.7648464218491	2.21394231571241\\
2.7511098920125	2.13654986655322\\
2.72779317877431	2.04095569483249\\
2.69024836192134	1.92078707468285\\
2.6312539409082	1.7670493095065\\
2.53961413480675	1.56721869756115\\
2.39837999194539	1.30464987086518\\
2.18360687141246	0.959750006540987\\
1.8667826621648	0.516209188824812\\
1.42701784641989	-0.0238781630653391\\
0.87502831245873	-0.620875059217897\\
0.268936641059173	-1.19901922110702\\
-0.309576813214938	-1.68530797756881\\
-0.802100907580029	-2.04896030059078\\
-1.19206161730571	-2.300440537754\\
-1.49041992551985	-2.46708460024485\\
-1.7168759123608	-2.57532489307401\\
-1.88995458136084	-2.64492059461505\\
-2.02413470299114	-2.68919932969544\\
-2.12995725581116	-2.71680265831965\\
-2.21490837909904	-2.73327951527395\\
-2.284286999297	-2.74220891806166\\
-2.34187032729419	-2.74592425079378\\
-2.39038313346388	-2.74596637420131\\
-2.4318186528206	-2.74336542452957\\
-2.46765634360889	-2.73881708355027\\
-2.49901014208278	-2.7327940719517\\
-2.52673028332805	-2.72561759993438\\
-2.551474024202	-2.71750375208859\\
-2.57375536742722	-2.70859393338899\\
-2.5939804436843	-2.69897499500867\\
-2.61247296631128	-2.688692537203\\
-2.6294927100655	-2.67775958603992\\
-2.64524900273341	-2.66616202895009\\
-2.6599105764423	-2.65386167486455\\
-2.67361268902278	-2.64079746083638\\
-2.68646212009346	-2.62688508668123\\
-2.69854042224194	-2.61201517643097\\
-2.70990563030842	-2.59604990865956\\
-2.72059247497691	-2.57881790224895\\
-2.73061098716967	-2.5601069666803\\
-2.73994319179238	-2.53965410007404\\
-2.74853734053432	-2.51713180921804\\
-2.75629877602211	-2.49212938365471\\
-2.76307597883326	-2.46412710489964\\
-2.76863950239953	-2.43246039507853\\
-2.77265014411951	-2.39626942454063\\
-2.7746104875259	-2.35442742359956\\
-2.77379029091263	-2.3054374557388\\
-2.76911009981251	-2.24728211523106\\
-2.75895730729457	-2.17720283029247\\
-2.74089227231279	-2.0913749275097\\
-2.71117633487659	-1.98443346302753\\
-2.66401914162082	-1.84880387542276\\
-2.59041768625865	-1.67383876852635\\
-2.47652724106683	-1.44496938777841\\
-2.30193857203366	-1.14371563581315\\
-2.0396739695219	-0.750852651889909\\
-1.66265564107253	-0.256803878437201\\
-1.16248546087895	0.319450601512933\\
-0.573614265552006	0.917547528939285\\
0.0284025713818862	1.45675943888985\\
0.5685251002902	1.88250662286626\\
1.00960568842464	2.18713123722861\\
1.35150239782658	2.39258093926768\\
1.61139397123953	2.52710219556684\\
1.80908182046774	2.61398894502504\\
1.96116532773652	2.66959557253549\\
2.08006223005889	2.7046815507453\\
2.17466815864962	2.726172144186\\
2.25127885929965	2.73851896753466\\
2.3143624731471	2.74460741055068\\
2.36712178368749	2.74633057020532\\
};
\addplot [color=mycolor1, forget plot]
  table[row sep=crcr]{%
1.27349527731192	3.20471819063596\\
1.44428602357601	3.19936336562619\\
1.60261405164943	3.18430366081914\\
1.7494617470495	3.16085789974953\\
1.88579739700425	3.13011008910517\\
2.0125428514028	3.09293326009606\\
2.13055396048019	3.05001429713906\\
2.24060988431472	3.00187741815919\\
2.34340825178317	2.94890508541948\\
2.43956391894906	2.89135583426738\\
2.5296096946948	2.82937892878458\\
2.61399786648816	2.76302598507574\\
2.69310169254936	2.69225981668208\\
2.76721625454656	2.61696080274643\\
2.83655821210403	2.53693109253768\\
2.90126408806333	2.45189696354759\\
2.96138675922161	2.36150966207012\\
3.01688984642981	2.26534508952985\\
3.06763970522629	2.16290276913649\\
3.11339473035919	2.05360465173142\\
3.15379172653358	1.93679451605467\\
3.18832919463209	1.81173900963237\\
3.21634758276645	1.67763178656044\\
3.23700691962578	1.53360274915386\\
3.24926287231705	1.3787350990062\\
3.25184326398312	1.21209372070441\\
3.24322856924441	1.03276925597542\\
3.22164196864756	0.839942844419659\\
3.18505716005814	0.632976472406207\\
3.13123499683714	0.411532478777436\\
3.05780235675828	0.175722049785833\\
2.96238692637022	-0.0737245209502354\\
2.84281753478955	-0.335283926070913\\
2.69738875896765	-0.606494094576824\\
2.52516943358411	-0.883881067768981\\
2.32630972784737	-1.16300759537571\\
2.10227863372139	-1.43868074143903\\
1.85595607299018	-1.70532290329812\\
1.59152340317857	-1.95746223706772\\
1.31414419128342	-2.19025259681242\\
1.02948815699354	-2.39991296505422\\
0.743198214823109	-2.58399557142257\\
0.460411488715003	-2.74144341261505\\
0.185417358609968	-2.87245752802465\\
-0.0785138856106611	-2.97823672044315\\
-0.329145589188702	-3.06066541941308\\
-0.565175433800149	-3.12201330013307\\
-0.786073059350422	-3.16468601912692\\
-0.991897989417819	-3.1910422753473\\
-1.18312736420307	-3.20327509398365\\
-1.36050921912967	-3.20334606094207\\
-1.52494663407333	-3.19295843529209\\
-1.67741164040707	-3.17355605030546\\
-1.81888459259207	-3.14633753472363\\
-1.95031370082384	-3.11227829447245\\
-2.07258962512226	-3.07215522866845\\
-2.18653078029436	-3.02657109539301\\
-2.29287589627036	-2.97597680757907\\
-2.39228121567003	-2.92069083506565\\
-2.48532040738634	-2.86091544019825\\
-2.57248581455441	-2.79674979051335\\
-2.65419005132678	-2.72820015727833\\
-2.73076724021758	-2.65518748343299\\
-2.80247336639283	-2.57755263051448\\
-2.86948534028096	-2.49505961995269\\
-2.93189842458056	-2.40739718984169\\
-2.98972171238905	-2.31417900956967\\
-3.04287135449774	-2.21494294566854\\
-3.09116124130123	-2.10914986787104\\
-3.13429086713906	-1.99618264208025\\
-3.1718301681864	-1.87534619761917\\
-3.20320126676684	-1.74586990360306\\
-3.2276573295482	-1.60691396750067\\
-3.24425923086295	-1.45758219425911\\
-3.25185150619326	-1.29694420959691\\
-3.24904030415707	-1.12407109656362\\
-3.2341778115443	-0.938089160971005\\
-3.2053599817185	-0.738256903482426\\
-3.16044720340501	-0.524069668561859\\
-3.09712028967762	-0.295394018093037\\
-3.01298571486023	-0.0526286044658788\\
-2.90574246407405	0.203120663078268\\
-2.77341566658946	0.469875931666726\\
-2.61464718527509	0.744671830950842\\
-2.42901069043125	1.02353733575874\\
-2.21729346307338	1.3016207686597\\
-1.98167017289334	1.57348138033746\\
-1.72569877677608	1.83352877918035\\
-1.45410337080575	2.0765414570174\\
-1.17236633997143	2.2981599756885\\
-0.886209684202	2.49524939806107\\
-0.601076144049998	2.66606314379628\\
-0.321711220923973	2.81019956313654\\
-0.0519055926422069	2.92839608031109\\
0.205593759258074	3.02223382531029\\
0.449034858659099	3.09382481911079\\
0.677523760761812	3.14553385255023\\
0.890847711899254	3.1797617627868\\
1.08929520758861	3.1987956202668\\
1.27349527731192	3.20471819063596\\
};
\addplot [color=mycolor2, forget plot]
  table[row sep=crcr]{%
2.33603117668886	2.77383773533797\\
2.37516464725374	2.77262691357743\\
2.40893170213665	2.76942690337039\\
2.43841940062857	2.76472767070704\\
2.46445304757259	2.75886290926017\\
2.48766781562485	2.75205859965667\\
2.50855867157367	2.74446471145197\\
2.52751564883675	2.7361760908235\\
2.54484907620749	2.7272462605596\\
2.56080781241566	2.7176964566711\\
2.57559252411427	2.7075213606192\\
2.58936537895104	2.69669243896537\\
2.60225707842181	2.68515944438462\\
2.61437184672533	2.67285038571794\\
2.62579076958642	2.65967009126052\\
2.63657370548617	2.64549733588026\\
2.64675984467711	2.63018035376686\\
2.65636684635875	2.61353039177146\\
2.66538831824105	2.59531274799315\\
2.67378918633813	2.57523445329204\\
2.68149819424238	2.55292734178153\\
2.68839630448344	2.52792464682231\\
2.69429904156482	2.49962833648105\\
2.69892963366187	2.46726298324861\\
2.70187786222841	2.42980975324494\\
2.70253626684757	2.38591063440619\\
2.69999981971571	2.33372758652888\\
2.69290573546456	2.27073288158737\\
2.67917401784381	2.19339441362953\\
2.65558277526102	2.09670321392352\\
2.61707186745425	1.97347563859052\\
2.55562090465979	1.81337698205752\\
2.45854873336742	1.60175127155448\\
2.30635872094887	1.31887400601422\\
2.07147274626518	0.941730524350173\\
1.72241984531358	0.453080187872498\\
1.24153219533735	-0.137619467154939\\
0.654680997818204	-0.772582252750343\\
0.0395069632936128	-1.35971688279848\\
-0.517637861558398	-1.82830042522886\\
-0.97089736627434	-2.16310655256113\\
-1.31839640890298	-2.38726933266721\\
-1.57909721866367	-2.53290594747214\\
-1.7749207828929	-2.62651336246446\\
-1.92392397638943	-2.6864302500371\\
-2.03934578994682	-2.72451851239282\\
-2.13049454066881	-2.74829318691323\\
-2.20385170818753	-2.76252013130226\\
-2.26395409160786	-2.77025452721126\\
-2.31401603417494	-2.7734836002745\\
-2.35634944249354	-2.7735195169949\\
-2.39264387200647	-2.77124055076024\\
-2.42415372127008	-2.76724084171689\\
-2.45182449666335	-2.76192476872171\\
-2.47637904157075	-2.75556732998468\\
-2.49837723207992	-2.74835332254689\\
-2.51825787341866	-2.74040305387863\\
-2.53636848776742	-2.731789323547\\
-2.5529867379886	-2.72254861451314\\
-2.56833597715801	-2.7126883350026\\
-2.58259659458752	-2.70219126617895\\
-2.59591428500911	-2.69101793056547\\
-2.60840599783681	-2.67910730161753\\
-2.62016406333094	-2.66637606485201\\
-2.63125879931482	-2.65271647587191\\
-2.64173974572939	-2.63799271230598\\
-2.65163553102222	-2.62203546177158\\
-2.66095222192739	-2.60463430292545\\
-2.66966982116945	-2.58552719290596\\
-2.67773632155005	-2.56438603244357\\
-2.68505834758345	-2.54079678062751\\
-2.69148683325652	-2.51423184316855\\
-2.69679525610873	-2.48401131577275\\
-2.70064643363102	-2.44924789564855\\
-2.70254137209289	-2.40876750845558\\
-2.70173941511501	-2.36099335456907\\
-2.69713171116671	-2.30377429501523\\
-2.68703767939672	-2.23412816295588\\
-2.6688733753315	-2.1478558869707\\
-2.63860733006011	-2.03896517257946\\
-2.58987303019969	-1.89883626876177\\
-2.51257074237411	-1.71511913654336\\
-2.39088248288237	-1.47063540976362\\
-2.20124949426826	-1.14348303786958\\
-1.91297510885859	-0.711698128877647\\
-1.49818929428948	-0.168118349712293\\
-0.95743799505463	0.455077656114234\\
-0.344692469317055	1.07773548420951\\
0.250413715583845	1.61106228502936\\
0.758152826604524	2.01148285385328\\
1.1568976469304	2.28696840373081\\
1.45817328343662	2.46805007856471\\
1.68387274034641	2.58488856364122\\
1.85434182773653	2.65981681155943\\
1.98517230036076	2.70765318707294\\
2.0874946534908	2.73784727198925\\
2.1690763903508	2.75637813802873\\
2.23533360006066	2.76705520229934\\
2.29007830562403	2.77233778434643\\
2.33603117668886	2.77383773533797\\
};
\addplot [color=mycolor1, forget plot]
  table[row sep=crcr]{%
1.39922803212736	3.07685806631181\\
1.5674959045967	3.07159193962536\\
1.72184120893004	3.05691962276783\\
1.86354618020044	3.03430258576604\\
1.99383941594411	3.00492434527477\\
2.11386334227955	2.96972519188087\\
2.22465770222629	2.92943638880441\\
2.32715347530744	2.88461124035927\\
2.42217316267472	2.83565189476223\\
2.51043459705609	2.78283160405981\\
2.59255635115887	2.726312621507\\
2.66906346531654	2.66616012709227\\
2.74039265271167	2.60235263950445\\
2.80689642206507	2.53478936498328\\
2.86884572677734	2.46329489204857\\
2.92643083902505	2.38762159185311\\
2.9797601812272	2.30745004343857\\
3.02885684319924	2.22238778377882\\
3.0736524851218	2.13196669573504\\
3.1139782873357	2.03563940659629\\
3.14955257373531	1.9327751939472\\
3.1799647289397	1.82265610911853\\
3.20465508608693	1.70447436470255\\
3.22289063840819	1.57733253330384\\
3.23373681095027	1.44024881653547\\
3.23602624800484	1.29217060644661\\
3.22832680506483	1.13200078307626\\
3.20891290060994	0.958642592487318\\
3.17574729649874	0.771070271643144\\
3.12648431983601	0.568433254718831\\
3.05851021117547	0.350200754759043\\
2.96904057175168	0.116349140091352\\
2.85529634653472	-0.132415162797099\\
2.71477438513296	-0.394420939308942\\
2.54561149927938	-0.666831451081862\\
2.34700920116711	-0.945545149042266\\
2.11964414530881	-1.22527369541402\\
1.86595263790311	-1.4998524733049\\
1.59017201622944	-1.76278083859751\\
1.29806813358121	-2.00790742861739\\
0.996371540126414	-2.23010725848011\\
0.692045715840486	-2.42578400848801\\
0.391565820616375	-2.59308707986694\\
0.100367539191781	-2.73183034725394\\
-0.177450479556643	-2.84318720481823\\
-0.439174975715654	-2.92927763544535\\
-0.683341813504384	-2.99275420405671\\
-0.909499160727534	-3.03645626209875\\
-1.11794272622867	-3.06316020920615\\
-1.30947089185102	-3.07542341942466\\
-1.48518319381155	-3.075503866566\\
-1.64632800142023	-3.06533330557286\\
-1.79419521627932	-3.04652411049664\\
-1.93004558742833	-3.02039455884332\\
-2.05506754143312	-2.98800216298559\\
-2.17035346301742	-2.95017858038136\\
-2.27688899630141	-2.90756246369496\\
-2.37555057125685	-2.86062847315891\\
-2.46710774227685	-2.80971180667064\\
-2.55222799210507	-2.7550282383983\\
-2.63148242817394	-2.69668997476434\\
-2.70535133355913	-2.63471776521598\\
-2.77422888817933	-2.56904972849322\\
-2.83842659644711	-2.49954732641521\\
-2.89817508338303	-2.42599886934325\\
-2.95362398038452	-2.34812089091296\\
-3.00483963484408	-2.26555769827939\\
-3.05180036004648	-2.17787939952491\\
-3.09438890639883	-2.08457874433518\\
-3.13238179607603	-1.98506720320871\\
-3.16543513925834	-1.87867087564694\\
-3.19306656963785	-1.76462708792787\\
-3.21463304463943	-1.64208295384835\\
-3.22930452229022	-1.5100977722518\\
-3.23603405874339	-1.36765196884681\\
-3.23352582137385	-1.21366638503477\\
-3.22020408341029	-1.04703704589849\\
-3.19418868326632	-0.866691949135152\\
-3.15328586629968	-0.671677510030317\\
-3.0950078113619	-0.461282278775401\\
-3.01663883454107	-0.23520307506533\\
-2.91536954396166	0.00624810083846552\\
-2.788518847893	0.261910999084898\\
-2.63385304009768	0.529552178029452\\
-2.44998681340866	0.805705536654112\\
-2.23681291575937	1.0856526413263\\
-1.99586497695243	1.36361286468693\\
-1.7304935342104	1.63317401023785\\
-1.44575426300667	1.88792082565777\\
-1.14798021434768	2.12213786617229\\
-0.844113546482297	2.33141799237157\\
-0.540956133719094	2.51303014802294\\
-0.244517307879121	2.66598261140652\\
0.0404151513087646	2.79081670105124\\
0.310442717751683	2.88923292693133\\
0.563501955589499	2.96366615687849\\
0.798667665060156	3.0168997946765\\
1.01589444674839	3.05176677289582\\
1.21575754923351	3.07094824312334\\
1.39922803212736	3.07685806631181\\
};
\addplot [color=mycolor2, forget plot]
  table[row sep=crcr]{%
2.2994083961655	2.79535024862062\\
2.33353375244637	2.79429370988155\\
2.36308892670794	2.79149226656542\\
2.38899329883575	2.78736357111642\\
2.41194606003088	2.78219240353602\\
2.43248641775227	2.77617155467401\\
2.45103550347488	2.76942852007211\\
2.46792596493415	2.76204308154774\\
2.48342313216485	2.75405889901689\\
2.49774031819916	2.74549105552355\\
2.5110499595793	2.73633077068861\\
2.52349174076519	2.72654803607802\\
2.53517847016565	2.71609262142336\\
2.54620021486251	2.70489368741015\\
2.55662701141105	2.69285807709674\\
2.56651032075193	2.67986721498817\\
2.57588326302111	2.6657723971624\\
2.58475953213234	2.65038808497362\\
2.59313072766097	2.63348259180601\\
2.60096162271741	2.61476524012883\\
2.60818256613162	2.59386860939319\\
2.61467772279225	2.57032380743149\\
2.62026706536589	2.54352563953533\\
2.62467873315634	2.51268289255035\\
2.62750619885164	2.47674632157668\\
2.6281409707708	2.43430271030891\\
2.62566512363811	2.38341658835518\\
2.61867669255461	2.32139034728934\\
2.60500128248795	2.24439673584546\\
2.58120961318637	2.14691411622279\\
2.54180743093121	2.02087048189424\\
2.47789806938556	1.85441211532424\\
2.37511364375075	1.63039068861648\\
2.21099885010276	1.32541676630578\\
1.95381570727021	0.912531393401144\\
1.56939776571841	0.374364860947135\\
1.046278826301	-0.268373525565837\\
0.430224874707573	-0.935264083236797\\
-0.182442693596096	-1.52035883315628\\
-0.708085217542624	-1.96268482913067\\
-1.11780394439882	-2.26544616804894\\
-1.42325717122235	-2.46253403104913\\
-1.64882507222525	-2.58856111738334\\
-1.81697924574442	-2.66894695501801\\
-1.94460168654153	-2.72026685041513\\
-2.04349791307372	-2.7529009999111\\
-2.12175472716191	-2.7733118805893\\
-2.18492142143171	-2.78556135910919\\
-2.23685134067867	-2.79224307449155\\
-2.28026355484299	-2.79504237948199\\
-2.31711056707747	-2.79507291549566\\
-2.34881912484968	-2.79308127701272\\
-2.37644941612019	-2.78957347677697\\
-2.40080174380478	-2.78489445799613\\
-2.42248909688625	-2.7792789317652\\
-2.44198728711015	-2.77288436531976\\
-2.45967010440917	-2.7658126305574\\
-2.47583430670791	-2.75812428765506\\
-2.49071759523835	-2.74984796403495\\
-2.50451166243361	-2.74098636524612\\
-2.51737170827005	-2.73151987698026\\
-2.52942336291049	-2.72140834464276\\
-2.54076764180827	-2.71059136440109\\
-2.55148433878095	-2.69898723541734\\
-2.56163409628376	-2.68649057275093\\
-2.57125925396019	-2.67296843883943\\
-2.58038344507941	-2.65825469619767\\
-2.58900976450084	-2.64214209085254\\
-2.59711714568452	-2.62437131392721\\
-2.60465432163687	-2.60461591281531\\
-2.61153034875796	-2.58246136459062\\
-2.61760004960281	-2.55737577301318\\
-2.6226417206743	-2.52866832876134\\
-2.62632277559825	-2.49542958836365\\
-2.62814615814032	-2.45644430058206\\
-2.6273654794016	-2.41006215886969\\
-2.62284832830332	-2.3540032622465\\
-2.61285230238781	-2.28506149819036\\
-2.59465244750208	-2.19864880205721\\
-2.56391581427681	-2.08809753480014\\
-2.51365623921814	-1.94362297330129\\
-2.43254884392664	-1.75091368452257\\
-2.30250431096509	-1.48970502660703\\
-2.09630378072396	-1.13403862962795\\
-1.77920298094463	-0.659113081286025\\
-1.32393593034075	-0.0624067233417369\\
-0.744435555399311	0.605702954626298\\
-0.117127565699744	1.24352382354984\\
0.459258666404913	1.76037734431491\\
0.927284812249035	2.12965076349379\\
1.28209324409541	2.37485769803888\\
1.54450086032372	2.53260670741068\\
1.73889001480468	2.63324617543375\\
1.88501585025053	2.69747674676842\\
1.99706351776802	2.73844513430852\\
2.0848113928928	2.76433742219006\\
2.15495148237977	2.78026822507319\\
2.21209951830831	2.7894763258861\\
2.25948547345075	2.79404790700794\\
2.2994083961655	2.79535024862062\\
};
\addplot [color=mycolor1, forget plot]
  table[row sep=crcr]{%
1.55639256905055	2.96512930676979\\
1.72194558400121	2.95995959644899\\
1.87187651283173	2.94571684865297\\
2.0078786804364	2.92401863646909\\
2.13152133501261	2.89614746612155\\
2.24422281966148	2.86310233390609\\
2.34724285915032	2.82564636890681\\
2.44168604901987	2.78434809112897\\
2.52851125160194	2.73961565129869\\
2.60854350893253	2.69172436094\\
2.68248638804546	2.64083822956509\\
2.7509335267552	2.58702633781514\\
2.81437867968246	2.53027483711349\\
2.87322387711878	2.4704952622475\\
2.92778547640811	2.40752971955752\\
2.97829795635211	2.34115339425146\\
3.02491531197481	2.27107471816354\\
3.06770987022764	2.19693346115496\\
3.10666827982388	2.1182969614416\\
3.14168433945825	2.03465470072291\\
3.17254822689084	1.94541147237283\\
3.19893158908426	1.8498795065186\\
3.22036787230806	1.74727013771703\\
3.23622724996468	1.63668597826971\\
3.24568561424095	1.51711516213764\\
3.24768745381299	1.38743014141285\\
3.24090323560218	1.24639485411533\\
3.22368343638461	1.09268593009691\\
3.19401403685822	0.924935969093117\\
3.14948258066916	0.741809589413677\\
3.08727020568363	0.542125192126739\\
3.00419323787546	0.325035616898807\\
2.89682647873737	0.0902762099333626\\
2.76174504315679	-0.161524827654143\\
2.59591463370637	-0.428505535065347\\
2.39723087888561	-0.707269802183286\\
2.16514959094201	-0.992739868771152\\
1.90127011825929	-1.27829123417025\\
1.60966980423026	-1.55625824365859\\
1.29679620703209	-1.81878348201875\\
0.970842704091363	-2.05883160002194\\
0.640728210317661	-2.2710859078631\\
0.31496544686145	-2.45247227019177\\
0.000731420951075753	-2.60220433110341\\
-0.296661411711697	-2.72142462868948\\
-0.573868667402983	-2.81262680605443\\
-0.829299485755192	-2.87905047295545\\
-1.06273726530523	-2.92417697995347\\
-1.27492670448002	-2.95137672496134\\
-1.4672079009809	-2.96370222005834\\
-1.64123122171026	-2.96379418017215\\
-1.79875555766596	-2.95386278896524\\
-1.94151727446253	-2.93571227958493\\
-2.07115260235749	-2.91078617505879\\
-2.18915727467812	-2.88021893913876\\
-2.29687043821154	-2.84488604726812\\
-2.39547334929136	-2.8054486049037\\
-2.48599634346008	-2.7623911007221\\
-2.56932981975949	-2.71605221826704\\
-2.64623656817436	-2.66664926811465\\
-2.71736382975639	-2.61429703952844\\
-2.78325415636491	-2.55902189368348\\
-2.84435454835788	-2.50077184161734\\
-2.90102358102023	-2.43942323201905\\
-2.95353634450248	-2.37478455042308\\
-3.00208705769334	-2.30659771959634\\
-3.04678919949292	-2.23453720000039\\
-3.08767294758247	-2.15820712535994\\
-3.12467963559481	-2.07713667842455\\
-3.157652843028	-1.9907739264093\\
-3.18632562825139	-1.89847841126238\\
-3.21030331962046	-1.79951295359335\\
-3.22904122207912	-1.6930354213926\\
-3.24181662900644	-1.57809169429292\\
-3.2476947425595	-1.45361180105303\\
-3.24548865203891	-1.31841232158201\\
-3.23371463787672	-1.17120972867802\\
-3.21054610855636	-1.01065146104482\\
-3.17377289507163	-0.835374085829839\\
-3.12077789444459	-0.644100498574099\\
-3.04855037558751	-0.435789619224154\\
-2.95376394575681	-0.209850277976997\\
-2.83295454904904	0.0335776083465671\\
-2.68283396756642	0.293291226760688\\
-2.50075751710368	0.566692571995787\\
-2.28532087855846	0.849548411045377\\
-2.03698913670611	1.13596864265365\\
-1.75858277962264	1.41872109049825\\
-1.45541039978119	1.68992187208685\\
-1.13490047418897	1.94199817831076\\
-0.805751468807456	2.1686805329637\\
-0.476814596794987	2.36573772500202\\
-0.156028488172868	2.53126267406597\\
0.150323962960791	2.66549690559156\\
0.437924755038436	2.77033638872705\\
0.70434694889485	2.84871913201416\\
0.948738849466358	2.9040595499326\\
1.17141446008507	2.93981790811621\\
1.3734572074647	2.95922355824186\\
1.55639256905055	2.96512930676979\\
};
\addplot [color=mycolor2, forget plot]
  table[row sep=crcr]{%
2.25648148144183	2.81152788377295\\
2.28611854761594	2.81060971185361\\
2.31188301568343	2.80816706740784\\
2.33454818820421	2.80455419539165\\
2.3547033978906	2.80001290555633\\
2.37280419498227	2.79470678315618\\
2.38920726532027	2.78874353664228\\
2.40419509161646	2.78218971157443\\
2.41799360188431	2.77508036725945\\
2.43078493427593	2.76742532867936\\
2.44271673263075	2.75921301833719\\
2.45390891910292	2.75041248499556\\
2.46445857600876	2.74097398775105\\
2.47444335022407	2.73082830913038\\
2.48392363222904	2.71988482420959\\
2.49294363197708	2.70802821854096\\
2.50153135458867	2.69511360418144\\
2.50969735110584	2.68095960758999\\
2.51743196071794	2.66533876684148\\
2.52470053987407	2.64796423727343\\
2.53143584233227	2.62847130085012\\
2.53752619284462	2.60639140337675\\
2.5427972491997	2.58111523783408\\
2.54698373201901	2.5518394717744\\
2.54968508637137	2.51748860956842\\
2.55029483420252	2.47659838720192\\
2.54788592718438	2.42713868356417\\
2.54102104395717	2.36624006572208\\
2.52743272100318	2.28976577528399\\
2.50347567868718	2.19163770921451\\
2.46318348874838	2.06278631195425\\
2.39666951885793	1.88959472276959\\
2.28759809150629	1.6519365163752\\
2.11000044012681	1.3219854402897\\
1.82735074161516	0.868276902077198\\
1.4035409516527	0.274924351724607\\
0.837870821427439	-0.420341133225879\\
0.201098027006392	-1.1100709252071\\
-0.395668872192818	-1.68034674015666\\
-0.880314982894813	-2.08838402879004\\
-1.24348455333275	-2.35683855332146\\
-1.507927557234	-2.52749837896194\\
-1.70086564924503	-2.63530514001686\\
-1.8439884403667	-2.7037269596881\\
-1.95252793963369	-2.74737273901532\\
-2.03676473790973	-2.77516838314625\\
-2.10360615644679	-2.79260071553722\\
-2.15774274589135	-2.80309798095731\\
-2.2024142036615	-2.80884485754463\\
-2.23990137594796	-2.81126134200048\\
-2.27184140715831	-2.81128716861249\\
-2.29943149129235	-2.80955366557274\\
-2.32356254689679	-2.8064896336933\\
-2.3449084001642	-2.80238785252393\\
-2.36398631770982	-2.79744760908678\\
-2.38119878662054	-2.79180230981551\\
-2.39686280734885	-2.78553760429208\\
-2.41123072445566	-2.7787033287183\\
-2.42450521898287	-2.77132131320246\\
-2.43685019558303	-2.76339032616339\\
-2.44839872074248	-2.75488894591073\\
-2.45925878639586	-2.74577683503627\\
-2.46951741246095	-2.73599467701678\\
-2.4792434156444	-2.72546287236615\\
-2.48848902910439	-2.71407895406076\\
-2.49729043540429	-2.70171354578068\\
-2.50566715422489	-2.6882045300937\\
-2.5136200862035	-2.67334889198013\\
-2.52112782906662	-2.65689142202918\\
-2.52814061359786	-2.63850905241165\\
-2.53457079335508	-2.61778897725139\\
-2.54027815951578	-2.59419774705328\\
-2.54504725993472	-2.56703700835965\\
-2.54855205743463	-2.53537912159193\\
-2.55030008292073	-2.49797191743159\\
-2.54954265294916	-2.45309530925004\\
-2.54512775459313	-2.39834166414063\\
-2.53525426020177	-2.33027415548639\\
-2.51705401219814	-2.24388966249411\\
-2.48587290072106	-2.13177492923707\\
-2.43403748503547	-1.98281496038808\\
-2.34881497238625	-1.7803862861264\\
-2.20943487700202	-1.50049942326001\\
-1.98435235053856	-1.11234032997585\\
-1.63453666967953	-0.588442579984248\\
-1.13579855446416	0.0653739113129525\\
-0.521391255039876	0.774076688632134\\
0.108412266451223	1.41484922999526\\
0.65379499857227	1.90419112353122\\
1.07599241105833	2.23744713376225\\
1.38626469515738	2.45193098293142\\
1.61181650075467	2.58754262213015\\
1.77756525843863	2.67335950991766\\
1.90184282954443	2.72798703390261\\
1.99718914272427	2.76284793443485\\
2.07202535032604	2.78492909576245\\
2.13203289943638	2.79855737073537\\
2.18110095632948	2.80646258005474\\
2.2219412271938	2.8104018315038\\
2.25648148144183	2.81152788377295\\
};
\addplot [color=mycolor1, forget plot]
  table[row sep=crcr]{%
1.75789046365152	2.87851795059347\\
1.92027634890559	2.8734607469727\\
2.06510472919776	2.85971412359751\\
2.19462378613701	2.83905983815503\\
2.31083799198744	2.81287128241266\\
2.41549985134058	2.78219036762057\\
2.51012168931292	2.74779378118756\\
2.59599693355803	2.71024730376353\\
2.67422461521597	2.66994888995881\\
2.74573358597395	2.62716200532674\\
2.81130464942246	2.58204086930475\\
2.87158979656993	2.53464911970734\\
2.92712827874576	2.48497317127276\\
2.9783595203119	2.43293127841629\\
3.02563297884943	2.37837907014263\\
3.06921506861588	2.32111211682189\\
3.10929321275395	2.2608659162411\\
3.14597700189229	2.19731354772933\\
3.17929632141098	2.13006113583178\\
3.20919617033611	2.05864118915142\\
3.23552773207612	1.98250384064836\\
3.25803507156029	1.9010060256971\\
3.27633662949837	1.81339871826247\\
3.2899004779828	1.71881254651789\\
3.29801213040874	1.61624249714943\\
3.29973364297103	1.50453310189055\\
3.29385296085882	1.38236664313758\\
3.27882323389149	1.2482587430566\\
3.25269363956878	1.10056848829626\\
3.2130368754626	0.937534243902808\\
3.15688500245051	0.757351543854944\\
3.08069596130842	0.558315211712442\\
2.98038849216385	0.339051842429407\\
2.85150160724088	0.0988657329701925\\
2.68954871829273	-0.161797793059755\\
2.49062888672968	-0.440814094262145\\
2.25230211025796	-0.733886406655619\\
1.97461137213715	-1.03430825388099\\
1.66096254637496	-1.33322930069096\\
1.31845337095567	-1.62057461227626\\
0.95732198834882	-1.88650319769001\\
0.589521285820144	-2.12298216439976\\
0.226850550200004	-2.32492867802353\\
-0.120713759157173	-2.49056304573379\\
-0.446003476157265	-2.62099290234318\\
-0.74485004764125	-2.71934212688711\\
-1.01568637964931	-2.78979839031806\\
-1.25887356310461	-2.83683344653703\\
-1.47600882825676	-2.86468792887994\\
-1.66935421231228	-2.87709929910705\\
-1.8414272424568	-2.87720508913988\\
-1.99473930098236	-2.86755172407349\\
-2.13164670382921	-2.85015597603359\\
-2.25427862994713	-2.82658529012176\\
-2.36451303746788	-2.79803830517911\\
-2.46398008560633	-2.76541678541352\\
-2.5540796809791	-2.72938590743431\\
-2.63600495018158	-2.69042280163042\\
-2.71076690996416	-2.64885455478097\\
-2.77921779159952	-2.60488729808164\\
-2.84207178352668	-2.55862798542887\\
-2.89992269720416	-2.51010026245484\\
-2.9532584501465	-2.45925556676253\\
-3.00247243669148	-2.40598034463359\\
-3.04787190743458	-2.35010004350221\\
-3.08968345364505	-2.29138034936013\\
-3.12805562209542	-2.2295259831704\\
-3.16305858336976	-2.1641772475904\\
-3.19468064919358	-2.0949044234456\\
-3.22282128331157	-2.02120005676339\\
-3.24728007602649	-1.94246916029047\\
-3.26774095669969	-1.8580173964057\\
-3.28375071035319	-1.76703744404503\\
-3.29469066885278	-1.66859403579765\\
-3.29974031987945	-1.56160867076005\\
-3.29783163093767	-1.44484589838781\\
-3.28759333321043	-1.3169045219713\\
-3.26728562584676	-1.17621934205837\\
-3.23472836344095	-1.02108243033109\\
-3.18723070850975	-0.849697569083211\\
-3.12153866748348	-0.660287153423712\\
-3.03382996236572	-0.451276164627668\\
-2.91980308567388	-0.221579177725888\\
-2.77492503123962	0.0289933634714278\\
-2.59490818701817	0.299224837097703\\
-2.37645882199743	0.585956016402782\\
-2.11825076772733	0.883689041791611\\
-1.82192293749608	1.18457256302336\\
-1.49273614167424	1.47898872208505\\
-1.1394873165461	1.75677715837975\\
-0.773492177610049	2.00881906414316\\
-0.406862835189353	2.22845893954485\\
-0.0506604067013423	2.41227436255297\\
0.286471226804994	2.5600189838237\\
0.598883778632005	2.67393043307189\\
0.883783632138136	2.75777669318722\\
1.14065779203266	2.8159688844512\\
1.3705662997484	2.85291106863863\\
1.57550018828592	2.87261358525458\\
1.75789046365152	2.87851795059347\\
};
\addplot [color=mycolor2, forget plot]
  table[row sep=crcr]{%
2.20548089943823	2.82259492056614\\
2.2310756781062	2.82180144067627\\
2.25341388640125	2.81968316600984\\
2.27314045325173	2.81653830900669\\
2.29074816700108	2.81257064955839\\
2.30661901094925	2.80791789296031\\
2.32105293751804	2.80267021067225\\
2.33428820981845	2.79688244343441\\
2.34651597522875	2.7905821012801\\
2.35789081685965	2.78377448399526\\
2.36853844083513	2.77644574287847\\
2.37856127247992	2.76856438176995\\
2.38804247531326	2.76008147691584\\
2.39704872485898	2.75092973518643\\
2.40563193312944	2.74102137806329\\
2.41383000688234	2.7302447118763\\
2.42166661519277	2.71845910236796\\
2.42914982141766	2.70548789061109\\
2.43626927894883	2.69110853666488\\
2.44299146696314	2.67503891108904\\
2.4492521000945	2.65691809977387\\
2.45494429738749	2.63627922524096\\
2.45990018838628	2.6125104174842\\
2.46386209338476	2.58479784748374\\
2.46643673675452	2.55204107421729\\
2.46702119429798	2.51272481779732\\
2.4646806518639	2.46472086352629\\
2.45794217040651	2.40497610469501\\
2.44443919338808	2.32901312638822\\
2.42028758225714	2.23012319567484\\
2.37898108438301	2.0980720291474\\
2.30946546712203	1.91712270122127\\
2.19302308476548	1.663480368929\\
1.99939986372173	1.30384507534387\\
1.68660213849871	0.801798313227344\\
1.21843474900127	0.146242380030686\\
0.612004139575645	-0.599490011561192\\
-0.0325838159621133	-1.29818124981711\\
-0.597959798471228	-1.83882174568265\\
-1.03282192882622	-2.20511881340256\\
-1.34748998865958	-2.43778649253415\\
-1.57235808532948	-2.58292713607252\\
-1.73507076486851	-2.67385011156894\\
-1.85550395067858	-2.73142497897582\\
-1.94693408294812	-2.76818955272608\\
-2.01809082677471	-2.79166777757401\\
-2.07475971740468	-2.80644585476573\\
-2.1208428346447	-2.8153804828952\\
-2.15902755535032	-2.8202920002674\\
-2.19120483273771	-2.82236549524384\\
-2.21873309374681	-2.82238716666129\\
-2.24260738515186	-2.82088663063703\\
-2.26356981838853	-2.81822450201677\\
-2.28218308232773	-2.81464742928703\\
-2.29888029991503	-2.81032332627052\\
-2.31399944553803	-2.80536427229711\\
-2.32780749443154	-2.79984154937507\\
-2.34051761431891	-2.7937955368736\\
-2.35230155238913	-2.78724214311511\\
-2.36329863765224	-2.78017681720398\\
-2.37362234450946	-2.77257678363149\\
-2.38336504885646	-2.76440187873777\\
-2.39260139227276	-2.75559418343563\\
-2.40139051405918	-2.74607650370016\\
-2.40977728886199	-2.73574962341048\\
-2.41779259964182	-2.72448812209074\\
-2.42545256414998	-2.71213439161339\\
-2.43275649824911	-2.69849027468424\\
-2.43968321494595	-2.6833054465566\\
-2.44618498344281	-2.66626121236649\\
-2.45217804115527	-2.64694770291678\\
-2.45752784854098	-2.62483136757586\\
-2.46202609767439	-2.59920792305946\\
-2.46535445985736	-2.56913307163661\\
-2.46702649734364	-2.5333185694399\\
-2.46629277352527	-2.48997324111257\\
-2.46198251161568	-2.43655495878129\\
-2.45223352200701	-2.3693766236903\\
-2.43402210812818	-2.28297159406618\\
-2.40233294227727	-2.169068980355\\
-2.34869425491878	-2.01497722303928\\
-2.25868913883791	-1.80125551068714\\
-2.10827647682834	-1.49929940663201\\
-1.86071669856813	-1.07246009907866\\
-1.47281070198137	-0.491513662073968\\
-0.927929490211162	0.223025382031232\\
-0.286220268576624	0.963687847270177\\
0.330295777802333	1.59139355984857\\
0.83197763016014	2.04178659571054\\
1.20336178905379	2.3350431603245\\
1.4692458554266	2.51888008351382\\
1.66006234090684	2.63361810391247\\
1.79961307762266	2.7058727610355\\
1.90421394291801	2.75185040212565\\
1.98463077787894	2.78125135738429\\
2.04795758667964	2.79993522196825\\
2.09893391063224	2.81151127678533\\
2.14078931321682	2.81825353172206\\
2.17577207927344	2.82162702115639\\
2.20548089943823	2.82259492056614\\
};
\addplot [color=mycolor1, forget plot]
  table[row sep=crcr]{%
2.02528855722988	2.8314627757905\\
2.18364256490326	2.82654712415527\\
2.32232089875638	2.81339711001726\\
2.44429966663803	2.79395569352328\\
2.55211787844742	2.7696676812432\\
2.64791266060073	2.74159314789719\\
2.73346857162937	2.71049808923055\\
2.81026898835822	2.67692434239347\\
2.87954376279983	2.64124227745091\\
2.94231081353296	2.60368983175353\\
2.99941111331336	2.56440098104945\\
3.05153738352734	2.52342612389607\\
3.09925713864025	2.48074627096032\\
3.14303078842218	2.43628243603582\\
3.18322543891858	2.38990122822675\\
3.22012490614952	2.34141733232042\\
3.25393630364273	2.29059331914839\\
3.28479339967968	2.23713703273042\\
3.31275676441611	2.18069664229565\\
3.33781053586055	2.12085331497088\\
3.3598554178422	2.05711135386713\\
3.37869727070575	1.98888555754715\\
3.3940303537022	1.91548550066498\\
3.40541391624517	1.83609643696888\\
3.41224041172014	1.74975663464195\\
3.41369314422302	1.65533126155474\\
3.40869073030275	1.55148360479601\\
3.39581554937713	1.43664570536127\\
3.37322376767736	1.30899285086508\\
3.33853634462256	1.16643045198713\\
3.28871512464255	1.0066085044921\\
3.21993811678531	0.826988980881285\\
3.12750686843875	0.625005188811786\\
3.0058500389319	0.398366670510614\\
2.84873066938348	0.145569184539406\\
2.64980675565619	-0.133354564135314\\
2.40369384305287	-0.435895060253338\\
2.10755788715585	-0.756164807355284\\
1.76295726042551	-1.08448738339765\\
1.37722220051641	-1.40802207807992\\
0.96344170103583	-1.71267650518036\\
0.538553849418496	-1.98584885182753\\
0.120068015783935	-2.21889121823761\\
-0.277199285299691	-2.40824585880885\\
-0.643099822950995	-2.55500109057541\\
-0.972465706932159	-2.66343648969624\\
-1.26423038869403	-2.73937577017955\\
-1.52009738124336	-2.78889636149337\\
-1.74329963282843	-2.81755650507628\\
-1.93768265558867	-2.83005667671636\\
-2.10712845574068	-2.83017869236823\\
-2.25524470362919	-2.82086681791378\\
-2.38522909864705	-2.80436229258904\\
-2.4998363994733	-2.78234341125037\\
-2.60139901110545	-2.75604987509515\\
-2.69187128999656	-2.72638475821845\\
-2.77288092865411	-2.69399440086462\\
-2.84577893581548	-2.65932925322217\\
-2.91168441125529	-2.62268931527768\\
-2.97152283345203	-2.58425754092257\\
-3.02605783251237	-2.54412399449862\\
-3.07591697091745	-2.50230293447381\\
-3.12161223055129	-2.45874445601906\\
-3.16355589123708	-2.41334187899278\\
-3.20207238300434	-2.36593571503202\\
-3.2374065516678	-2.31631477087814\\
-3.26972861743574	-2.26421472691255\\
-3.29913593606771	-2.20931435452912\\
-3.32565148958407	-2.15122939143132\\
-3.34921883136506	-2.089503972563\\
-3.36969297794654	-2.02359941386525\\
-3.38682646430956	-1.95288007154884\\
-3.40024944867019	-1.87659596779902\\
-3.40944235934206	-1.7938619205939\\
-3.41369912774013	-1.70363310855882\\
-3.41207859133372	-1.60467746319032\\
-3.40334129865304	-1.49554621970708\\
-3.3858689805257	-1.37454571988291\\
-3.35756493606156	-1.23971668570371\\
-3.31573660462241	-1.08883244372372\\
-3.25696856650511	-0.919435900458483\\
-3.17700812873283	-0.728947083919059\\
-3.07071029907343	-0.514887822395679\\
-2.93212679730042	-0.275282186335697\\
-2.75486957540281	-0.00928591918375054\\
-2.53290747287493	0.28195292691212\\
-2.26190501443769	0.594330814017856\\
-1.94100156034264	0.92006421124295\\
-1.57453530079942	1.24773500771413\\
-1.17283460989327	1.56356523148982\\
-0.751263100438952	1.85385259421138\\
-0.327490030079096	2.10772970392302\\
0.0819728852720324	2.31905616496465\\
0.464503781835654	2.48673557094366\\
0.81249731627353	2.61366335439495\\
1.12299079725902	2.70508271574215\\
1.39647235888047	2.76707313014238\\
1.63554733779064	2.80551841565643\\
1.8438470703118	2.82556921699742\\
2.02528855722988	2.8314627757905\\
};
\addplot [color=mycolor2, forget plot]
  table[row sep=crcr]{%
2.14285046027138	2.82808756846245\\
2.16480232914021	2.82740651091117\\
2.18404462096453	2.82558137619315\\
2.2011087002191	2.82286059781474\\
2.21640196057306	2.81941412880184\\
2.23024128938241	2.81535662847848\\
2.24287640264083	2.81076264354564\\
2.25450637336062	2.80567659600452\\
2.26529149609173	2.80011930365334\\
2.27536189104698	2.79409210175755\\
2.28482377819513	2.78757922486641\\
2.2937640413391	2.78054884188152\\
2.30225349204603	2.7729529542352\\
2.3103490941841	2.76472622844156\\
2.31809529588709	2.75578371453312\\
2.32552451819258	2.74601728057389\\
2.33265675227913	2.73529045062384\\
2.33949810305412	2.72343114522586\\
2.34603796346328	2.7102215560015\\
2.3522442775425	2.69538398746693\\
2.35805599539808	2.67856088584384\\
2.36337124406967	2.65928630618278\\
2.36802876326672	2.63694450529085\\
2.37177846998737	2.61070876898528\\
2.37423403079971	2.57944924242777\\
2.37479490319824	2.54159109198231\\
2.37251525076091	2.49489137507254\\
2.36587810379294	2.43608027600758\\
2.35239671276938	2.36027289623144\\
2.32789590352291	2.25999265404638\\
2.28520228049655	2.12355755666052\\
2.2117927329946	1.93254039078384\\
2.08591373136419	1.65843271354778\\
1.87192808224271	1.26107452856134\\
1.52191544631949	0.699323977495348\\
1.00401328873093	-0.0260975653771542\\
0.363748071527552	-0.813984659430715\\
-0.2688644928636	-1.50026846347405\\
-0.78505984725054	-1.99421220096836\\
-1.1621814512543	-2.3120046844514\\
-1.42730501323027	-2.50808040891141\\
-1.61430555121665	-2.62878977651985\\
-1.74908287219487	-2.70410346009035\\
-1.84892536871792	-2.75183309069718\\
-1.92497168994207	-2.78240988395908\\
-1.98441631057396	-2.80202204953594\\
-2.03198774293826	-2.81442637276709\\
-2.07086511691551	-2.82196289321932\\
-2.1032373715361	-2.82612594060473\\
-2.1306469031497	-2.82789152306617\\
-2.15420434386284	-2.82790950724923\\
-2.17472557537473	-2.82661924323466\\
-2.19282102135335	-2.8243208006842\\
-2.20895506929689	-2.82121982114766\\
-2.22348640556655	-2.81745629556059\\
-2.23669590171479	-2.81312330436156\\
-2.24880621761687	-2.80827933132791\\
-2.2599957841219	-2.80295635057308\\
-2.27040889628117	-2.79716504396977\\
-2.28016305849344	-2.79089798964273\\
-2.2893543410282	-2.78413133413755\\
-2.29806125319669	-2.77682524217822\\
-2.30634746269321	-2.76892326079756\\
-2.31426356178228	-2.7603506082126\\
-2.32184797734626	-2.75101127969837\\
-2.32912702634906	-2.74078373327658\\
-2.33611401487627	-2.72951475592708\\
-2.34280714846345	-2.717010888234\\
-2.34918583682532	-2.70302646046144\\
-2.35520469439475	-2.68724680041715\\
-2.36078408665524	-2.66926440501115\\
-2.36579532337417	-2.64854464007714\\
-2.37003732217863	-2.62437552995137\\
-2.37319932996421	-2.59579286104035\\
-2.37480028000697	-2.56146615364026\\
-2.3740880014168	-2.51952125498264\\
-2.36986769099099	-2.46725815618681\\
-2.3602027423829	-2.40069260335542\\
-2.34188069376843	-2.31379881220432\\
-2.30944331514121	-2.19725108711924\\
-2.25342347579248	-2.03637727552634\\
-2.15726868574138	-1.8081318719905\\
-1.99277374169129	-1.47800379565238\\
-1.7169275365176	-1.00247264364514\\
-1.2836338369358	-0.353478106408908\\
-0.692128251530688	0.422602884473332\\
-0.0377635087113686	1.17848387603558\\
0.544781901732044	1.77206572627127\\
0.989822588600999	2.17182516324887\\
1.30650097833272	2.42196025609287\\
1.52872677843643	2.57563263041337\\
1.68696675115438	2.67078647683365\\
1.80256019550217	2.73063605644361\\
1.88940158578403	2.76880563330631\\
1.95642880997747	2.79330947859752\\
2.00945904775913	2.80895397684342\\
2.05235799123028	2.81869458288088\\
2.08775582377096	2.82439568419062\\
2.11748483909939	2.82726178475665\\
2.14285046027137	2.82808756846245\\
};
\addplot [color=mycolor1, forget plot]
  table[row sep=crcr]{%
2.39662868959425	2.84951082055756\\
2.54951631809144	2.84478316146543\\
2.68057577202693	2.83236964475803\\
2.79368931059282	2.81435210655368\\
2.89200956708609	2.79221227117425\\
2.97808052474476	2.76699436724974\\
3.05395082474626	2.73942510677382\\
3.12127101874159	2.71000031675759\\
3.18137334210608	2.67904668602785\\
3.23533549826592	2.6467653770816\\
3.28403088381477	2.61326255303785\\
3.32816773512192	2.5785704660063\\
3.36831937711644	2.54266168096823\\
3.40494735390697	2.50545821962719\\
3.43841882493674	2.46683683518139\\
3.46901925630134	2.42663121128447\\
3.49696112960295	2.3846315684237\\
3.52238912121525	2.34058192200471\\
3.54538195800655	2.29417504129906\\
3.56595091331318	2.24504498662911\\
3.58403464861197	2.19275693830407\\
3.599489807854	2.13679386361574\\
3.61207640388388	2.0765393910213\\
3.62143656457363	2.01125607413664\\
3.6270645884653	1.9400580467324\\
3.62826545094727	1.86187693524784\\
3.62409786740233	1.77541990303787\\
3.61329677027294	1.67911905078622\\
3.59416873522277	1.57107248739018\\
3.56445296290204	1.44897996970295\\
3.52114107269472	1.31008145360664\\
3.46025386082096	1.15111750821774\\
3.3765877477028	0.96834975663678\\
3.26347767903156	0.757711152712086\\
3.11269107120892	0.515200371534801\\
2.91468049121388	0.237677533331435\\
2.65956107621816	-0.075792005356255\\
2.33922687728764	-0.42207479588027\\
1.95070870148828	-0.79208503593864\\
1.49992499673157	-1.17005005898236\\
1.00369620923486	-1.53532883829985\\
0.487714276647776	-1.86704741987091\\
-0.0196037104394338	-2.14958815963688\\
-0.494266489177459	-2.37589438155797\\
-0.921094490007363	-2.54715856090294\\
-1.29407185464677	-2.67002059913229\\
-1.61405066260037	-2.7533617882459\\
-1.88582793930159	-2.80600842284946\\
-2.11580931472076	-2.8355752154477\\
-2.31055960343822	-2.8481265843826\\
-2.4760753310711	-2.84826673357076\\
-2.61751342695858	-2.83939071981517\\
-2.73916306057416	-2.82395684149332\\
-2.84452673110119	-2.8037235494515\\
-2.93643778478825	-2.77993637218657\\
-3.01717928551152	-2.7534680827133\\
-3.08858977970994	-2.72492085634415\\
-3.15215174070506	-2.69469950321393\\
-3.20906307246607	-2.66306341554223\\
-3.26029379479034	-2.63016310153615\\
-3.30663043348983	-2.59606561165381\\
-3.3487104699092	-2.56077192816399\\
-3.38704883368716	-2.52422846610237\\
-3.42205801644831	-2.48633416034531\\
-3.45406300707221	-2.44694412473863\\
-3.48331191905296	-2.40587051060485\\
-3.5099828937072	-2.36288092139323\\
-3.53418760693175	-2.31769452613295\\
-3.55597146508519	-2.26997583310268\\
-3.57531032803307	-2.21932591896988\\
-3.59210332225446	-2.1652707442698\\
-3.60616097787423	-2.10724601437021\\
-3.61718750864978	-2.04457786249847\\
-3.62475551469728	-1.97645844436705\\
-3.62827067991085	-1.90191536813603\\
-3.6269231170911	-1.8197738030447\\
-3.61962086504921	-1.72861025449313\\
-3.60489972458324	-1.62669764617954\\
-3.58080240451189	-1.51194306149107\\
-3.5447195921549	-1.38182328893359\\
-3.49318786598444	-1.23333098397222\\
-3.42164818690977	-1.06295867761356\\
-3.32419146548788	-0.866772876631447\\
-3.19336668810003	-0.64066903948441\\
-3.0202170318666	-0.380945389129189\\
-2.79484166778538	-0.0853597202615685\\
-2.50789777204493	0.24524215579012\\
-2.15336789331032	0.60494911172521\\
-1.7323041212614	0.981292060555261\\
-1.25603671291713	1.35564225777066\\
-0.74637262927147	1.70653835040823\\
-0.23129746125784	2.01512023532681\\
0.262194573726511	2.26986342003333\\
0.714231227266017	2.46807926651014\\
1.11438689035123	2.61410444947348\\
1.46044698874666	2.71605977891121\\
1.75558068285544	2.78301068879355\\
2.00562516112094	2.82326121103597\\
2.21719544458287	2.84365845322455\\
2.39662868959425	2.84951082055756\\
};
\addplot [color=mycolor2, forget plot]
  table[row sep=crcr]{%
2.06125160838935	2.82609185066612\\
2.07994925124581	2.82551123450406\\
2.09642363792935	2.82394819411031\\
2.11110532962897	2.82160689086134\\
2.12432597927662	2.81862716062496\\
2.13634475923024	2.81510310819691\\
2.14736685278038	2.81109530783169\\
2.15755659642858	2.80663882557853\\
2.16704694356782	2.80174842493572\\
2.17594634547779	2.79642179774859\\
2.18434377663302	2.79064133531615\\
2.19231238805512	2.78437473909726\\
2.19991210618597	2.77757461782079\\
2.20719137482391	2.77017709745021\\
2.2141881433638	2.76209936047244\\
2.22093012022366	2.7532359127787\\
2.22743422158551	2.74345323023988\\
2.23370503599168	2.73258223742678\\
2.23973197208311	2.72040778022596\\
2.24548452388585	2.70665381282011\\
2.25090471474374	2.69096232906942\\
2.25589515985366	2.67286296058742\\
2.26030012253552	2.65172834480686\\
2.26387506387652	2.62670731102676\\
2.26623679389436	2.59662268469411\\
2.26678003680328	2.55981130214178\\
2.26453423629103	2.51386736871076\\
2.25791108650606	2.45522053002431\\
2.24424716451873	2.37842646554096\\
2.21895543141708	2.27495574519942\\
2.17393123321108	2.13113357503807\\
2.09461115491916	1.92481897622353\\
1.95508074839113	1.62108993957651\\
1.71280218315468	1.17128586036876\\
1.31487939823906	0.532571479500302\\
0.744373465911252	-0.267011737689928\\
0.0892415081571163	-1.0739861548735\\
-0.501235778002658	-1.71516536094982\\
-0.94830109085819	-2.14322609187765\\
-1.26079643502598	-2.40664257753268\\
-1.47616902870883	-2.56594328497232\\
-1.62722885702031	-2.66345388897609\\
-1.73627697297688	-2.72438721319827\\
-1.81745907450743	-2.76319317280894\\
-1.87968387289177	-2.78821006640888\\
-1.92865279428784	-2.80436411932649\\
-1.96810432160467	-2.8146496981553\\
-2.00055506641086	-2.82093927034229\\
-2.02774317329863	-2.82443476449945\\
-2.05089835097381	-2.82592560443978\\
-2.0709100552968	-2.8259403106581\\
-2.08843474420915	-2.82483797460678\\
-2.103965880597	-2.8228648340703\\
-2.11788063559145	-2.82019004338289\\
-2.13047168893757	-2.81692871332125\\
-2.1419692891369	-2.81315695930412\\
-2.15255681437367	-2.80892179935339\\
-2.16238190864915	-2.80424763616303\\
-2.17156454387382	-2.79914039392197\\
-2.18020289961054	-2.79358997034857\\
-2.18837765371054	-2.78757140076561\\
-2.19615507704615	-2.78104495165163\\
-2.2035891855187	-2.77395522776637\\
-2.21072309758051	-2.76622926419841\\
-2.21758965797856	-2.75777346288815\\
-2.22421130370965	-2.74846910370544\\
-2.23059905134031	-2.73816599101705\\
-2.23675035689961	-2.72667355727781\\
-2.2426454121733	-2.7137483885527\\
-2.24824114796816	-2.69907658675329\\
-2.25346173532385	-2.68224851142628\\
-2.25818356520954	-2.66272202846742\\
-2.26221127873464	-2.63976804230512\\
-2.26523890612181	-2.61238809494481\\
-2.26678556693555	-2.57918688165498\\
-2.26608652368133	-2.53817024701707\\
-2.26190369584654	-2.48641710820626\\
-2.25218696779675	-2.41953373467694\\
-2.23345280986463	-2.3307278718439\\
-2.19962169333636	-2.20922521273513\\
-2.13984083508523	-2.03762134861856\\
-2.0346032412399	-1.78791047607263\\
-1.85011538205858	-1.41776640448773\\
-1.53612890561482	-0.87652508965204\\
-1.04857221250265	-0.146003337612458\\
-0.417760327509739	0.682338124572095\\
0.221622946878704	1.42166635464838\\
0.74348915317282	1.95384390101691\\
1.11911647718661	2.2914069222994\\
1.37832089111808	2.49618674388445\\
1.55811461255222	2.62052375633131\\
1.68596713518099	2.69740311002973\\
1.77970009060015	2.74593105205812\\
1.85052730166635	2.77705905146031\\
1.90555619448917	2.79717432215525\\
1.94938858132061	2.81010366980761\\
1.98508170672757	2.8182068466805\\
2.0147205208524	2.8229794234728\\
2.03976279585362	2.82539290711205\\
2.06125160838935	2.82609185066612\\
};
\addplot [color=mycolor1, forget plot]
  table[row sep=crcr]{%
2.94343378702665	2.98150695436743\\
3.08885591882588	2.97703004882834\\
3.21054796459534	2.96551824604259\\
3.31341894940247	2.94914292861528\\
3.40124887065957	2.92937341442907\\
3.47695420897439	2.90719878294888\\
3.54279594379573	2.88327856753722\\
3.60053768793696	2.85804436184619\\
3.65156420497661	2.83176816902748\\
3.69696972781973	2.8046083443135\\
3.73762369200408	2.77664040394839\\
3.77421969304537	2.74787752274462\\
3.80731196955673	2.7182838987095\\
3.83734254235399	2.68778306693339\\
3.86466125736386	2.65626250988181\\
3.88954032096338	2.62357541000337\\
3.91218441887612	2.58954003765049\\
3.93273712261889	2.55393700506867\\
3.95128396880596	2.51650440568945\\
3.96785231077364	2.47693066782098\\
3.98240775591448	2.43484475941707\\
3.99484668103311	2.38980316445684\\
4.0049839205663	2.34127278980445\\
4.0125341945373	2.28860863059909\\
4.01708510995117	2.23102459677709\\
4.01805852587469	2.16755535943817\\
4.01465557245049	2.09700640285185\\
4.00577846405327	2.01788869870269\\
3.98991922134651	1.92833370145539\\
3.96500134333717	1.8259841226427\\
3.92815550330806	1.7078572334151\\
3.87540572228343	1.57018268928028\\
3.80124242044465	1.40823126569638\\
3.69807474295245	1.21618474376124\\
3.55561366713805	0.987168345099138\\
3.36039355718265	0.713697986754594\\
3.0959705528776	0.388984026151238\\
2.74485482730161	0.00965943621581311\\
2.29357605845332	-0.419860896077798\\
1.74123809312363	-0.88273264558884\\
1.10809223296134	-1.3486339523367\\
0.436257105221601	-1.78050448022638\\
-0.222411249521406	-2.14740128269557\\
-0.824456983247826	-2.43456471604976\\
-1.34594972456221	-2.64394799960773\\
-1.78189217938341	-2.78766825269008\\
-2.13923520644798	-2.8808316329203\\
-2.42990512098094	-2.93720386055972\\
-2.66642312116667	-2.96765779361241\\
-2.85990342575475	-2.98016035853\\
-3.01947047104845	-2.980318962509\\
-3.15233141227306	-2.97199814102729\\
-3.26407678003267	-2.95783324946641\\
-3.35901422996584	-2.93961142538769\\
-3.44046250987799	-2.91853917939846\\
-3.51098770154975	-2.89542548863729\\
-3.57258504950268	-2.87080559978002\\
-3.62681601558868	-2.84502437405132\\
-3.67491062596942	-2.81829233073741\\
-3.71784366349641	-2.79072328945494\\
-3.75639138919313	-2.76235953884773\\
-3.7911738055063	-2.73318844920671\\
-3.82268613730342	-2.70315310412262\\
-3.85132218756576	-2.67215862982249\\
-3.87739146216411	-2.64007529563112\\
-3.90113138683763	-2.60673904059147\\
-3.9227155029489	-2.57194977960228\\
-3.94225818044096	-2.53546760974441\\
-3.95981608835485	-2.49700683985336\\
-3.97538638144289	-2.45622757786813\\
-3.98890126247118	-2.41272440886056\\
-4.00021822633899	-2.36601146055877\\
-4.00910483726414	-2.31550286003529\\
-4.01521627070026	-2.26048721032088\\
-4.01806297880681	-2.20009423360216\\
-4.01696458777847	-2.1332511189094\\
-4.01098433732702	-2.05862538227507\\
-3.99883581308979	-1.97455027346157\\
-3.97875018437911	-1.87892820292819\\
-3.94828757872079	-1.76910799411692\\
-3.90407115412907	-1.64173460153032\\
-3.84141925200013	-1.49257885102639\\
-3.75385682688332	-1.31637716254411\\
-3.63252009480728	-1.10676098963577\\
-3.46556723356335	-0.856454205776585\\
-3.23794182258222	-0.55808102507739\\
-2.93227133418999	-0.206114291382088\\
-2.53220302067924	0.199545148698177\\
-2.02933798624782	0.648742309963167\\
-1.43256471133232	1.11760073735716\\
-0.773780891425569	1.57105760630653\\
-0.102173643872332	1.97343260536243\\
0.532508754172318	2.30116044932552\\
1.09595877732084	2.54836533236184\\
1.57436796768815	2.72307496482937\\
1.96969470722219	2.83964895043039\\
2.29210292472253	2.9128645884254\\
2.55419399350715	2.95510989188397\\
2.76792508339582	2.97575476480715\\
2.94343378702665	2.98150695436743\\
};
\addplot [color=mycolor2, forget plot]
  table[row sep=crcr]{%
1.94396230060006	2.81077700198304\\
1.95985108879152	2.81028302218281\\
1.97394601517608	2.80894524160945\\
1.98658836727855	2.80692872305329\\
1.99804290145849	2.80434666455556\\
2.00851794541265	2.80127490447696\\
2.01817955450266	2.79776146990237\\
2.02716163337153	2.7938328465329\\
2.0355732636903	2.7894980045526\\
2.04350405557244	2.78475081876226\\
2.05102806631827	2.77957126772946\\
2.0582066475911	2.77392562507895\\
2.0650904552572	2.76776572935734\\
2.07172076202316	2.761027312666\\
2.07813013562979	2.75362726375721\\
2.08434247168474	2.74545958092923\\
2.09037228714733	2.73638961317069\\
2.09622307156667	2.726245965862\\
2.10188433396748	2.71480911571078\\
2.10732673361643	2.70179526631605\\
2.11249427288296	2.68683315867038\\
2.11729183363615	2.66943021738284\\
2.12156511946277	2.64892218404249\\
2.12506787274033	2.62439657120363\\
2.12740717858999	2.59457357186253\\
2.12794994908125	2.55761602734519\\
2.12565858578613	2.51081799353142\\
2.11879356735678	2.4500804507019\\
2.10435914717095	2.36900692454446\\
2.077044104215	2.25731897225445\\
2.02717662267011	2.09810365785094\\
1.93689704487503	1.86338398381203\\
1.77403815419916	1.50898111967002\\
1.48781236794104	0.977591282821741\\
1.02797794570424	0.239102328779913\\
0.417065751156591	-0.618110712345637\\
-0.207022888965703	-1.38788049639511\\
-0.71136150653178	-1.93605025310214\\
-1.0682992731387	-2.27797487868556\\
-1.31082101086512	-2.48243692785583\\
-1.47709853453944	-2.60542161615692\\
-1.59438921386909	-2.68112630281884\\
-1.67991444787889	-2.72890911966843\\
-1.74431053192608	-2.75968640544852\\
-1.794230072076	-2.77975270835501\\
-1.83393929055117	-2.79284972826202\\
-1.8662525883997	-2.80127251714169\\
-1.89307927889216	-2.80647073286878\\
-1.91574910331315	-2.80938432065387\\
-1.93521073419012	-2.81063655925458\\
-1.95215596271729	-2.81064836758024\\
-1.96709937311284	-2.80970786100419\\
-1.98043069252134	-2.80801373357572\\
-1.99244995795617	-2.80570290192155\\
-2.00339162522548	-2.8028684301293\\
-2.01344140390623	-2.79957129251754\\
-2.02274820662335	-2.79584812070656\\
-2.03143274948098	-2.79171625123768\\
-2.0395938092478	-2.7871768871126\\
-2.04731280353105	-2.78221687147284\\
-2.05465713752028	-2.77680936482239\\
-2.06168260965477	-2.77091357168549\\
-2.06843506041373	-2.76447354870151\\
-2.0749513644328	-2.75741602296175\\
-2.08125979218648	-2.74964703925448\\
-2.08737969086333	-2.74104711921343\\
-2.09332034049265	-2.73146443019397\\
-2.09907871112645	-2.72070519171487\\
-2.10463564888983	-2.70852013631634\\
-2.10994970029888	-2.69458519617113\\
-2.11494725179667	-2.6784735460981\\
-2.11950674301536	-2.65961441457825\\
-2.12343308270449	-2.63723116573116\\
-2.12641542374375	-2.61024611324654\\
-2.12795587734824	-2.57713057502082\\
-2.12724598828813	-2.53566242125648\\
-2.12294648080199	-2.48252332081042\\
-2.112782671042	-2.41261201185111\\
-2.09278017382524	-2.31784838961673\\
-2.05579255081375	-2.18507740519626\\
-1.98867972953625	-1.99251610396963\\
-1.86729573432262	-1.70460212758402\\
-1.65006553490393	-1.26884645769957\\
-1.2810391545868	-0.632579777231207\\
-0.734829643605394	0.186538733238462\\
-0.0961807704141871	1.02629442439069\\
0.478237086983504	1.69129758873056\\
0.906738389298453	2.12857176341927\\
1.20126694458971	2.39332900349002\\
1.4015110305165	2.55153550099923\\
1.54059805830277	2.64771501683872\\
1.64033955717226	2.70768355489814\\
1.7142672731182	2.74595219806201\\
1.77077068523864	2.77078098484228\\
1.81515884049049	2.78700379272891\\
1.85088449176798	2.79753982960281\\
1.88025785289774	2.80420676365557\\
1.90486730397671	2.8081683476314\\
1.92583278470842	2.81018803520323\\
1.94396230060006	2.81077700198304\\
};
\addplot [color=mycolor1, forget plot]
  table[row sep=crcr]{%
3.8069742121015	3.32601021063417\\
3.94310767731051	3.32183896413045\\
4.05420275841036	3.31134312526451\\
4.14616621219568	3.29671363519818\\
4.22331079729383	3.27935618385384\\
4.28882087296	3.26017297175802\\
4.3450761920556	3.23973943178931\\
4.39387674716691	3.2184157825262\\
4.43659966989989	3.19641802187611\\
4.47430959606783	3.17386344125139\\
4.50783701640624	3.15079991969402\\
4.53783440608565	3.12722471731482\\
4.56481674587108	3.10309632370882\\
4.58919091662597	3.07834157938611\\
4.61127701392177	3.05285944670632\\
4.63132365137258	3.02652226330007\\
4.64951864016164	2.99917494528458\\
4.66599594085826	2.9706323448972\\
4.68083940812854	2.94067475863317\\
4.69408353516085	2.90904139284285\\
4.70571110826766	2.87542139564259\\
4.71564736063074	2.83944182921608\\
4.7237498186394	2.80065165304653\\
4.7297924996768	2.75850037441778\\
4.73344235137475	2.71230943916012\\
4.73422467258974	2.66123359920901\\
4.73147249347479	2.60420828388831\\
4.72425214485595	2.53987725159715\\
4.71125292455703	2.46649229210029\\
4.69062195021153	2.38177326153695\\
4.65971462450232	2.28271219372795\\
4.61471493250551	2.16530026432807\\
4.55005689082875	2.0241538357331\\
4.45755195938648	1.85202485053174\\
4.32511660075767	1.63922707388558\\
4.13506999383621	1.3731567878721\\
3.86233660637841	1.03846759065608\\
3.47398655764083	0.619251405989274\\
2.93385906115449	0.105608985940811\\
2.21819414526292	-0.493647269701428\\
1.34285086028235	-1.13739433911321\\
0.381733763959551	-1.75511273586638\\
-0.555945978397205	-2.27758418022185\\
-1.3801338200065	-2.67098549388462\\
-2.05273030094319	-2.94130697961128\\
-2.57903978312717	-3.11502187892367\\
-2.98402766689781	-3.22074360659178\\
-3.29550436360794	-3.28123933156228\\
-3.53714176607316	-3.31240885437513\\
-3.72707871394694	-3.32471889480433\\
-3.87861866314147	-3.32489355374631\\
-4.00137183341736	-3.31722202297675\\
-4.10227530238411	-3.30444276587309\\
-4.18636976282832	-3.28831018307363\\
-4.2573555868939	-3.26995072743412\\
-4.31798139885556	-3.25008592041586\\
-4.37031381264431	-3.22917262122432\\
-4.41592509522117	-3.2074919276244\\
-4.4560246114975	-3.18520597582556\\
-4.4915517058208	-3.16239444919179\\
-4.52324194637074	-3.13907806679447\\
-4.55167477689722	-3.11523355816124\\
-4.57730801706905	-3.09080293236409\\
-4.60050290583896	-3.06569879140361\\
-4.62154219991242	-3.03980676424577\\
-4.64064302591274	-3.01298569489415\\
-4.65796560952048	-2.9850659110852\\
-4.67361857917333	-2.95584566959024\\
-4.68766120322395	-2.92508567906909\\
-4.7001026196294	-2.89250141118904\\
-4.71089781383341	-2.85775269790244\\
-4.71993974885331	-2.82042984810432\\
-4.72704659614644	-2.7800351641179\\
-4.73194237861246	-2.73595824833843\\
-4.73422840048079	-2.68744279296103\\
-4.73334141867878	-2.63354154044063\\
-4.72849231313563	-2.57305464653371\\
-4.71857557025886	-2.50444457872921\\
-4.70203446371177	-2.42571771042901\\
-4.67665827273233	-2.33425874303738\\
-4.63927465991455	-2.22659917351108\\
-4.58528080037595	-2.09809665954561\\
-4.50793119077666	-1.94250354931282\\
-4.39727709648894	-1.75142557911221\\
-4.23867072522465	-1.51375682493256\\
-4.01092918938735	-1.21541927095236\\
-3.68491262374884	-0.840304133414801\\
-3.22495230318177	-0.374306646763107\\
-2.59825348631594	0.185029069108261\\
-1.79721195477659	0.813917342883822\\
-0.866410835864464	1.45435571377579\\
0.0967248201703723	2.03142822392588\\
0.98588338027817	2.4907984854268\\
1.73578374400764	2.82009224229463\\
2.33281856586493	3.03836073769983\\
2.79490690793268	3.17478938223685\\
3.14984561154891	3.25550255453373\\
3.42379321322665	3.29972949659059\\
3.6376413178944	3.32043092197198\\
3.8069742121015	3.32601021063417\\
};
\addplot [color=mycolor2, forget plot]
  table[row sep=crcr]{%
1.74677956590797	2.76321383348015\\
1.76055079162803	2.76278489084878\\
1.77289752304441	2.76161235880782\\
1.78408307816614	2.75982761735725\\
1.79431432821077	2.75752078244666\\
1.80375605241051	2.75475155614031\\
1.81254117813544	2.75155640064587\\
1.82077818881318	2.74795322267997\\
1.82855654270017	2.74394430324449\\
1.83595066345758	2.73951792268229\\
1.84302287744233	2.73464894166046\\
1.84982554550421	2.72929846348412\\
1.85640254543905	2.72341259492248\\
1.86279018923641	2.71692022200251\\
1.86901759477247	2.70972960689701\\
1.87510646340486	2.70172347333093\\
1.881070129905	2.69275205592092\\
1.88691163070849	2.68262330610839\\
1.89262035073271	2.67108901358273\\
1.89816650666999	2.6578249161294\\
1.90349221348034	2.64240175754852\\
1.90849698956012	2.62424240164907\\
1.91301395760529	2.60255695453783\\
1.91677004982509	2.57624233466723\\
1.91931793079299	2.54372286171832\\
1.91991642535571	2.50269036399665\\
1.91731432610301	2.44966862491135\\
1.90934758813476	2.37926393771651\\
1.8921676553373	2.28284881141417\\
1.858736269888	2.14624189106937\\
1.79592121051184	1.94579193806572\\
1.67939844049359	1.64293488187096\\
1.46775176340378	1.18231590857994\\
1.10905206980362	0.515816473456423\\
0.592268408768907	-0.315590085670382\\
0.0149776246806529	-1.12735122067658\\
-0.482966401320024	-1.74247563274645\\
-0.846247138018683	-2.13758925789272\\
-1.09479727169634	-2.37570531565814\\
-1.26460364302433	-2.51883787815067\\
-1.38356177823104	-2.60679926878341\\
-1.46967907361087	-2.66236645582977\\
-1.53410063802864	-2.69834771482707\\
-1.58376484002151	-2.7220768972298\\
-1.62309106660028	-2.73788014820263\\
-1.65497417750504	-2.74839259095045\\
-1.68136499101193	-2.75526922566731\\
-1.70361398362453	-2.75957864139342\\
-1.72267950622259	-2.76202763288094\\
-1.73925718341232	-2.76309323105423\\
-1.75386240879692	-2.76310253391312\\
-1.76688423497511	-2.76228223573638\\
-1.77862138744392	-2.76079006369596\\
-1.78930684247465	-2.75873511590543\\
-1.79912492476823	-2.75619120953072\\
-1.80822340856862	-2.75320570703901\\
-1.8167222134742	-2.749805330294\\
-1.82471973285041	-2.74599989645953\\
-1.83229748174183	-2.74178455266795\\
-1.83952352294432	-2.73714085580965\\
-1.84645497692253	-2.73203688565233\\
-1.85313981422446	-2.72642646053219\\
-1.85961804889195	-2.72024742257549\\
-1.86592238451269	-2.71341885602795\\
-1.87207829967102	-2.70583698036699\\
-1.8781034848286	-2.6973692982315\\
-1.88400644278413	-2.68784634661996\\
-1.88978391652406	-2.67705005108697\\
-1.89541657260374	-2.6646971390544\\
-1.90086197669035	-2.65041519717875\\
-1.90604322535744	-2.63370752631203\\
-1.91083040880429	-2.61390053769103\\
-1.91500991591535	-2.5900632758779\\
-1.91823254593586	-2.5608813002931\\
-1.91992359964181	-2.52445383804068\\
-1.91912269779814	-2.47795849089767\\
-1.91418977930217	-2.41708166155436\\
-1.9022493147976	-2.33502720544078\\
-1.8781137711729	-2.22076643402019\\
-1.83218329233983	-2.05599163654765\\
-1.74651403576271	-1.81029310733755\\
-1.58876039137854	-1.43616386293894\\
-1.30945691793092	-0.875652336430001\\
-0.867067023985944	-0.111895638767759\\
-0.301474533825516	0.73804118811955\\
0.249914151277534	1.46445740076682\\
0.681128085091329	1.96421367923717\\
0.982532897783201	2.27188940194\\
1.18754066861232	2.45616064097104\\
1.32911446155926	2.56798673754701\\
1.42990435001473	2.63766308133445\\
1.50409356324664	2.68225497396183\\
1.56045640237134	2.71142230728431\\
1.60451183081133	2.73077531966039\\
1.63982383437755	2.7436770095946\\
1.66876057492792	2.75220807282934\\
1.69293993199161	2.75769405610699\\
1.71349618514047	2.76100161394126\\
1.73124362836514	2.7627100897579\\
1.74677956590797	2.76321383348015\\
};
\addplot [color=mycolor1, forget plot]
  table[row sep=crcr]{%
5.23359968598068	4.06228569073568\\
5.36158749680671	4.05838065589803\\
5.46374098524715	4.04874039012966\\
5.54679183289556	4.03553600666112\\
5.61543665918267	4.02009609472798\\
5.67301780996828	4.00323838663666\\
5.7219588007418	3.98546437097936\\
5.76404748331935	3.96707561940243\\
5.80062350346106	3.94824452309343\\
5.83270461781342	3.92905791881896\\
5.86107325043917	3.90954423800289\\
5.88633671013261	3.8896904063681\\
5.90896962530503	3.86945219768965\\
5.92934413252931	3.84876026701216\\
5.94775144446764	3.82752320063858\\
5.96441718888765	3.80562836909689\\
5.9795120919587	3.78294101085854\\
5.99315901563786	3.75930172366362\\
6.00543694852082	3.73452234366319\\
6.0163822227875	3.70838001337468\\
6.0259869344175	3.6806090475804\\
6.0341942325797	3.65088997141281\\
6.0408897648076	3.61883478967854\\
6.04588804778265	3.58396709869379\\
6.04891177486158	3.54569499249963\\
6.04956090262756	3.50327371982051\\
6.0472664971646	3.45575351572038\\
6.04122127904133	3.40190563377832\\
6.03027372644046	3.34011579792157\\
6.0127639465138	3.26822818248925\\
5.98626454625468	3.18331316289279\\
5.94716346988753	3.08131622318111\\
5.8899795581516	2.95652060026942\\
5.80622154889659	2.8007204513202\\
5.68247087909964	2.60196159196587\\
5.49719491920497	2.34271072987884\\
5.21573614177751	1.99755522791973\\
4.7837443466677	1.53164430053319\\
4.12360982185974	0.904557796144839\\
3.15088009759361	0.0909888040006891\\
1.84002957784789	-0.872152820396208\\
0.319106348884118	-1.84937470406752\\
-1.15275137205982	-2.66989860688977\\
-2.36798134970148	-3.25059532097248\\
-3.27589286565043	-3.61600344575553\\
-3.92595046499675	-3.83088160761399\\
-4.38901224547138	-3.95194490186727\\
-4.72349231149159	-4.01701000298456\\
-4.97038901434058	-4.04891587652087\\
-5.15700419762603	-4.0610447774668\\
-5.3013489603052	-4.06123206442097\\
-5.41541488601347	-4.0541167851449\\
-5.507322473188	-4.04248567917992\\
-5.58268038608153	-4.02803518021852\\
-5.6454410834196	-4.0118073560535\\
-5.69844424824095	-3.99444337905209\\
-5.74376716383899	-3.97633364667063\\
-5.7829545913717	-3.95770831182488\\
-5.81717231615215	-3.93869275066441\\
-5.84731150137254	-3.91934195400816\\
-5.87406075604608	-3.89966196286193\\
-5.89795661305006	-3.87962314242784\\
-5.91941928808485	-3.85916816308982\\
-5.93877819398976	-3.83821641512527\\
-5.95629015390419	-3.81666588762123\\
-5.97215225555894	-3.79439310081417\\
-5.98651061370172	-3.77125138458311\\
-5.9994658310114	-3.74706757739965\\
-6.01107558656411	-3.72163703610851\\
-6.02135447578981	-3.69471666499943\\
-6.03027092845338	-3.6660154633198\\
-6.03774069318905	-3.63518182071692\\
-6.04361593943317	-3.60178641648343\\
-6.04766840603931	-3.56529903705841\\
-6.04956408873287	-3.5250568181969\\
-6.0488254886897	-3.48022018583099\\
-6.04477507184966	-3.42971085631841\\
-6.03644966793568	-3.37212324025763\\
-6.02246892500576	-3.30559577658956\\
-6.00082957465836	-3.22762096104647\\
-5.96857745114017	-3.13476030577434\\
-5.9212743543892	-3.02221052364306\\
-5.85211579503254	-2.88313695112032\\
-5.75045222363542	-2.70765027157872\\
-5.5993092530868	-2.48127354568202\\
-5.37134478901386	-2.18282749426356\\
-5.02288978461698	-1.78221201710447\\
-4.48777477941243	-1.24061089878179\\
-3.68049222212999	-0.520924958516225\\
-2.53409425373193	0.378121776297124\\
-1.0909212452937	1.37045141300698\\
0.439385935937216	2.287447501653\\
1.79910038993385	2.99051600874801\\
2.85847914257785	3.4563076483258\\
3.62862072756977	3.73827308049239\\
4.17677865185423	3.90035497170982\\
4.56938522686428	3.98976950678617\\
4.85591727395001	4.03610457934639\\
5.06993291680334	4.05686649742183\\
5.23359968598067	4.06228569073568\\
};
\addplot [color=mycolor2, forget plot]
  table[row sep=crcr]{%
1.33301670838737	2.61204888329464\\
1.34644320755614	2.61162908613279\\
1.35874493797112	2.61045944960314\\
1.37011960182753	2.60864331060449\\
1.38072728200983	2.60625048816175\\
1.39069931887731	2.60332468400919\\
1.40014485071414	2.59988834624712\\
1.40915568633019	2.59594569807516\\
1.41780996332873	2.59148436178709\\
1.42617490018531	2.58647582139012\\
1.43430884830736	2.58087482761002\\
1.44226277441712	2.57461773110508\\
1.45008124159532	2.56761961353\\
1.45780289896554	2.55976995270321\\
1.46546042493265	2.55092638520112\\
1.47307978398209	2.54090588631409\\
1.48067853267487	2.52947232623802\\
1.48826271427597	2.5163188064435\\
1.49582155700891	2.50104230408664\\
1.50331863627902	2.4831067372466\\
1.51067718701888	2.46178822934725\\
1.51775549857982	2.43609242260905\\
1.52430509226474	2.40462695806892\\
1.52989828989807	2.36540053415675\\
1.53380007949194	2.31549947077581\\
1.53473644177133	2.25055725852743\\
1.53046737484438	2.16387448001793\\
1.51699215125942	2.0449675132812\\
1.48709098642182	1.87729236425733\\
1.42784761001922	1.63526643156743\\
1.31745996182202	1.28282952524043\\
1.12526661095823	0.78250845831663\\
0.827178623124766	0.131829563845013\\
0.442626569180643	-0.585643728223974\\
0.045598959997802	-1.22687943553216\\
-0.292887299848114	-1.7039329392721\\
-0.550174643240503	-2.02194789246879\\
-0.737231321118399	-2.22531219891865\\
-0.872923550490124	-2.35520786936517\\
-0.973073002559847	-2.43955468949996\\
-1.04880622763155	-2.49550923857833\\
-1.10755169111278	-2.53338648794925\\
-1.15423717537672	-2.55944352147708\\
-1.19217088269001	-2.57755596109554\\
-1.22361657877387	-2.59018418731236\\
-1.25015675706226	-2.59892911557715\\
-1.27292158124762	-2.60485662231768\\
-1.29273488333345	-2.60869099274247\\
-1.31020886281663	-2.61093297797532\\
-1.32580668819217	-2.61193351417705\\
-1.33988468529741	-2.61194075318417\\
-1.35272130697319	-2.61113064541139\\
-1.36453738377374	-2.60962714494715\\
-1.37551051859768	-2.6075157086508\\
-1.38578547830666	-2.60485235082933\\
-1.39548180163462	-2.60166966586222\\
-1.4046994376114	-2.59798070619169\\
-1.41352296466125	-2.59378126805965\\
-1.42202476441441	-2.58905091388597\\
-1.43026740319638	-2.58375290074115\\
-1.43830538703167	-2.57783305831397\\
-1.44618638843291	-2.5712175450445\\
-1.45395198429384	-2.56380928878409\\
-1.46163788389146	-2.55548276837135\\
-1.46927355295561	-2.54607658901317\\
-1.47688103818478	-2.53538300946849\\
-1.48447264150035	-2.52313313308341\\
-1.49204684223526	-2.5089757795822\\
-1.49958144278572	-2.49244694411555\\
-1.50702218051136	-2.47292493719459\\
-1.51426374550244	-2.44956327808078\\
-1.5211177698076	-2.42118828226447\\
-1.52725792808844	-2.38613941931682\\
-1.53212386393177	-2.34201503276749\\
-1.53474935346189	-2.28525898061997\\
-1.5334484297272	-2.21047785433787\\
-1.52523299326183	-2.10930790940293\\
-1.50473158650961	-1.96857741655307\\
-1.4622569408778	-1.7675997104122\\
-1.38081867461681	-1.47540980909169\\
-1.23360993864157	-1.0527993701081\\
-0.989734196601563	-0.473118481173226\\
-0.641946753959142	0.227373334762711\\
-0.240284393996546	0.923745133831825\\
0.133534335238773	1.48729623857675\\
0.431414965234564	1.88026635099027\\
0.651290951575217	2.13508720401846\\
0.810394867871117	2.29739657590908\\
0.926643646894778	2.40180047104697\\
1.01345598548921	2.47031496538093\\
1.07994787151582	2.51624539095708\\
1.13216521285489	2.54760839904196\\
1.17413710720869	2.56931384510598\\
1.2085930466781	2.58444002420209\\
1.23742064298458	2.59496564299668\\
1.26195372251315	2.60219342990986\\
1.28315480571647	2.60699994060218\\
1.30173242034655	2.60998623944193\\
1.31821791114351	2.61157092933178\\
1.33301670838737	2.61204888329464\\
};
\addplot [color=mycolor1, forget plot]
  table[row sep=crcr]{%
7.0392772787345	5.11903163413247\\
7.16626142891188	5.11516704726827\\
7.26629530947662	5.10573296770008\\
7.34678121583159	5.0929404252796\\
7.41275009540204	5.07810515705988\\
7.46770818903169	5.06201734166849\\
7.51415523785521	5.04515049114053\\
7.55390997795022	5.0277825360959\\
7.58831969447633	5.01006760274933\\
7.6183982542733	4.99207933311409\\
7.64491893503354	4.97383734027816\\
7.66847801198178	4.95532339772637\\
7.68953899601874	4.93649119519741\\
7.70846377852075	4.91727191806879\\
7.72553470301668	4.89757698320867\\
7.7409701764394	4.87729870260255\\
7.75493552033362	4.85630928720692\\
7.76755014829301	4.83445835564298\\
7.77889171864666	4.81156891940326\\
7.78899757154672	4.78743163993666\\
7.79786345915379	4.76179696211644\\
7.80543926619556	4.73436449125311\\
7.81162104112765	4.70476865681686\\
7.81623814294102	4.67255923743224\\
7.81903354520929	4.63717461823094\\
7.81963414633169	4.59790456633448\\
7.81750600161046	4.5538375966568\\
7.8118861644611	4.50378523524493\\
7.80167728653956	4.44617092905127\\
7.78528137648615	4.37886367333006\\
7.76033149490167	4.29892322355866\\
7.72324748141428	4.2022006198842\\
7.6684797357689	4.08269666016919\\
7.58718501700106	3.93150794554762\\
7.46484515066429	3.73506430634936\\
7.27690174669529	3.47216907133276\\
6.98076794077052	3.10918441338855\\
6.50205235605686	2.59321050777568\\
5.71605754331805	1.84720902799201\\
4.44720912225841	0.787085083320875\\
2.56880033484852	-0.591757799732749\\
0.257878165560372	-2.07608138743818\\
-1.95873600611616	-3.31246837059941\\
-3.67246583887669	-4.13229019081978\\
-4.84890959120493	-4.60637352971057\\
-5.62924820005475	-4.86462617628409\\
-6.152693041282	-5.00162775078038\\
-6.51414732826121	-5.07201566016901\\
-6.77218075665923	-5.10540017020252\\
-6.96238590458728	-5.11778413520988\\
-7.10672674008617	-5.11798405531009\\
-7.21911589964707	-5.1109811032576\\
-7.3086237166577	-5.09965865272981\\
-7.38133267142339	-5.08571943340421\\
-7.4414301898953	-5.07018253002399\\
-7.4918688504861	-5.05366035915892\\
-7.53477577997466	-5.03651722848336\\
-7.57171304439667	-5.01896232518367\\
-7.60384725922387	-5.00110537118762\\
-7.63206250849633	-4.98299044337303\\
-7.65703701014054	-4.96461668067336\\
-7.67929606109214	-4.9459508960254\\
-7.69924911114715	-4.92693503066006\\
-7.71721597128642	-4.90749018638949\\
-7.73344539439188	-4.88751825480663\\
-7.74812813760957	-4.86690171678298\\
-7.76140587128517	-4.84550189100654\\
-7.77337678567625	-4.82315569550259\\
-7.78409836629924	-4.79967080559342\\
-7.79358749509538	-4.77481891194556\\
-7.80181773494674	-4.74832657256971\\
-7.80871331842344	-4.7198628778369\\
-7.81413892636756	-4.68902276011232\\
-7.81788371978822	-4.65530420766401\\
-7.8196371394281	-4.61807677139272\\
-7.81895247498845	-4.57653739295516\\
-7.81519171634902	-4.52964741199717\\
-7.80744098485871	-4.47604106957693\\
-7.79437851885377	-4.41388992474236\\
-7.77406412241355	-4.34069755793578\\
-7.74359506366882	-4.25298149230635\\
-7.69852849172415	-4.14576846289678\\
-7.63188425699374	-4.01177432194864\\
-7.53237456803583	-3.84004333817358\\
-7.38118488394934	-3.61366203084588\\
-7.14605406356131	-3.3059544239265\\
-6.77063751144112	-2.87457534687001\\
-6.15761509575811	-2.25458614919317\\
-5.15347938851971	-1.36027702764371\\
-3.58354295228443	-0.130374145500047\\
-1.43877767222122	1.34333463692235\\
0.89679834516737	2.74303223250638\\
2.88793902590139	3.77351412635487\\
4.3202076583399	4.40403635626887\\
5.27891129380902	4.75547535999178\\
5.91612850328448	4.94410764971809\\
6.34932856187132	5.04287378380005\\
6.65346014382276	5.09210914897521\\
6.87414979638642	5.11354760941571\\
7.0392772787345	5.11903163413247\\
};
\addplot [color=mycolor2, forget plot]
  table[row sep=crcr]{%
0.340159402357135	2.10600086215548\\
0.3628465262389	2.10528376142592\\
0.384983538536663	2.10317175270087\\
0.406716283993941	2.09969488583627\\
0.428180678011108	2.09484635528048\\
0.449505533367152	2.08858296968783\\
0.470815018388807	2.08082372707867\\
0.492230799808478	2.07144640903074\\
0.513873889130809	2.06028194820058\\
0.535866166731864	2.04710614126222\\
0.558331494769348	2.03162805630885\\
0.581396234946618	2.01347419948199\\
0.605188838922992	1.99216713731932\\
0.629837943593957	1.96709679723808\\
0.655468026444841	1.93748207838122\\
0.682191073509881	1.90231972724602\\
0.710091758583656	1.86031679879876\\
0.739202155707758	1.80980281214685\\
0.769459818573213	1.74861884556833\\
0.80064008913943	1.67398536129912\\
0.832250213425196	1.58236272978444\\
0.863371344598456	1.46934596208096\\
0.892440839855638	1.32968951028285\\
0.916994864544829	1.1576478771209\\
0.933461665095238	0.947922142732608\\
0.93722024091161	0.697507134450695\\
0.923254154813351	0.408373538265392\\
0.887613191091914	0.0899584839014675\\
0.829292619497089	-0.240624003826465\\
0.751350217007357	-0.562241095216022\\
0.660192239960307	-0.855958820248454\\
0.563363666195249	-1.10998962709869\\
0.46734720222843	-1.32075071805171\\
0.376485188537052	-1.49074974756309\\
0.293014529295588	-1.62559329921078\\
0.21763453804227	-1.73168296113876\\
0.150124373920323	-1.8149523813449\\
0.0898064836170812	-1.88038152662915\\
0.0358346873675908	-1.93193627925224\\
-0.0126491668834016	-1.97268813284077\\
-0.0564408523233183	-2.00498324783389\\
-0.0962456055018899	-2.03060375647545\\
-0.132673687998756	-2.05090219860937\\
-0.166246925494325	-2.06690671537743\\
-0.197409467037838	-2.07940074765622\\
-0.226539405413789	-2.08898241554729\\
-0.253959729761681	-2.09610838452086\\
-0.279948024299138	-2.10112612642857\\
-0.304744788313741	-2.10429756290364\\
-0.328560461293696	-2.10581629915087\\
-0.351581317507659	-2.10582004535157\\
-0.373974412350106	-2.10439936011743\\
-0.395891752286059	-2.10160350283493\\
-0.417473838109259	-2.09744391778838\\
-0.438852705211635	-2.09189566598317\\
-0.460154557460578	-2.08489694885098\\
-0.481502063043633	-2.07634671334331\\
-0.503016349197679	-2.06610017415172\\
-0.524818694049379	-2.05396192018484\\
-0.547031861225919	-2.03967607218156\\
-0.569780945882671	-2.02291270740463\\
-0.593193482477847	-2.00324944399919\\
-0.617398378026468	-1.98014665881395\\
-0.64252293665151	-1.95291428015224\\
-0.668686764263143	-1.92066745506512\\
-0.695990582565688	-1.88226770787954\\
-0.724496790253406	-1.83624571219609\\
-0.754196798300473	-1.78070209222836\\
-0.78495757674282	-1.71318514683432\\
-0.816436598886198	-1.63055203632806\\
-0.84795154121966	-1.5288386027832\\
-0.878292379011432	-1.40320237913186\\
-0.905478215418113	-1.24807521708215\\
-0.926507279688136	-1.0577656117023\\
-0.937246943050733	-0.827828665751056\\
-0.932746846728689	-0.557384782911339\\
-0.90829524435065	-0.251913355457592\\
-0.861193639531173	0.0751070859219576\\
-0.792430440350056	0.40386223059073\\
-0.7069485204366	0.713524546104237\\
-0.612037533875673	0.988336489411677\\
-0.514928609605447	1.22072882229166\\
-0.421088281198913	1.41053544394599\\
-0.333753957128362	1.56216496577172\\
-0.254311790829908	1.68184199557282\\
-0.182931519533185	1.77583455664028\\
-0.119117675453817	1.84962629757004\\
-0.062081846382775	1.90768220695569\\
-0.0109584532815586	1.95350093424583\\
0.035085202877312	1.98976990269858\\
0.0768012723721879	2.0185345950784\\
0.114847160032888	2.04134741988353\\
0.149787719570286	2.05938727992309\\
0.182104487988261	2.07355139534386\\
0.212207153711672	2.08452419672195\\
0.240444960310469	2.09282839920086\\
0.267117068662005	2.09886263746276\\
0.292481564777994	2.1029290972206\\
0.316763114824385	2.10525371981033\\
0.340159402357135	2.10600086215548\\
};
\addplot [color=mycolor1, forget plot]
  table[row sep=crcr]{%
6.61613611895751	4.8641942745862\\
6.74267149711494	4.86034166598725\\
6.84257530198247	4.85091880632069\\
6.92309873222533	4.83811961587769\\
6.98919202422371	4.82325590380537\\
7.0443173727696	4.80711879880973\\
7.09095015006891	4.79018426152558\\
7.13089554947986	4.77273282959086\\
7.16549338683827	4.75492090558833\\
7.19575351193484	4.73682393903221\\
7.22244715811728	4.71846288110962\\
7.24616967660651	4.69982042039387\\
7.26738427455025	4.68085079423991\\
7.28645286113455	4.66148541809789\\
7.30365793713208	4.64163566081575\\
7.31921809172996	4.62119353646822\\
7.33329877818253	4.60003072596646\\
7.34601943704234	4.5779960943986\\
7.357457604939	4.55491167731867\\
7.36765031073442	4.53056693316483\\
7.37659276261626	4.5047108687695\\
7.38423401969428	4.47704140898084\\
7.39046896592001	4.44719106021156\\
7.39512539095489	4.41470745448598\\
7.39794422323946	4.37902666722518\\
7.39854977688489	4.33943613597095\\
7.39640496071006	4.29502233073107\\
7.39074321390929	4.24459563420822\\
7.38046349486532	4.18658047707388\\
7.36396512538055	4.11885139407387\\
7.33888219380859	4.03848309344072\\
7.30164578431777	3.94136087081416\\
7.24674328402024	3.82155971649234\\
7.16543172741847	3.67033470293414\\
7.04344941432565	3.47445678193261\\
6.85688671052396	3.21347769087934\\
6.56480894289311	2.85543554581792\\
6.09704081110383	2.35120251270526\\
5.33922949371652	1.631832926147\\
4.13741497712578	0.627506738731737\\
2.39201362139747	-0.653956054596282\\
0.271129584082802	-2.0163128519437\\
-1.76706418105212	-3.15304630900311\\
-3.36477841515265	-3.9171910106998\\
-4.48057736347614	-4.36672322110522\\
-5.23178650862857	-4.61527876569066\\
-5.74140679225126	-4.74863483186194\\
-6.09621456353023	-4.81771508798844\\
-6.35101845621712	-4.85067488062163\\
-6.53967088879576	-4.86295399720598\\
-6.68330764764067	-4.86315077538039\\
-6.79543275686343	-4.85616295793073\\
-6.88490770779658	-4.8448438231133\\
-6.95770489220665	-4.83088712753518\\
-7.01795224457751	-4.81531109806863\\
-7.06856961994611	-4.79873010511294\\
-7.11166597664871	-4.78151108154078\\
-7.14879330431607	-4.76386568823666\\
-7.18111271547939	-4.74590569465811\\
-7.20950540074214	-4.7276767463185\\
-7.23464817377506	-4.7091791018496\\
-7.25706576499147	-4.69038029875894\\
-7.27716750980255	-4.67122266148723\\
-7.29527332439532	-4.65162737865237\\
-7.31163214276176	-4.63149616603733\\
-7.32643488682237	-4.6107110889196\\
-7.33982331218368	-4.58913282362824\\
-7.35189556689479	-4.56659742372425\\
-7.36270892512731	-4.54291147591349\\
-7.37227984687155	-4.51784535141028\\
-7.38058121681288	-4.49112404964878\\
-7.38753628018219	-4.46241485842959\\
-7.39300835961935	-4.43131067100693\\
-7.39678481775374	-4.39730723605238\\
-7.39855278713633	-4.35977175928978\\
-7.39786269021448	-4.31789894229912\\
-7.39407311222847	-4.27064842485589\\
-7.38626644212047	-4.21665415887213\\
-7.37311752223128	-4.15409054931505\\
-7.35268482515135	-4.08047058639661\\
-7.32207055776348	-3.99233467925752\\
-7.27685311900696	-3.88476016134848\\
-7.21011400666836	-3.75057132588405\\
-7.1107263430797	-3.57904450869508\\
-6.96028340185397	-3.35377011677173\\
-6.72755817023565	-3.04918973401285\\
-6.3588483963266	-2.62547601246506\\
-5.76350289791787	-2.02328116456653\\
-4.80347182629616	-1.16809173024485\\
-3.33099395850477	-0.0142908269286488\\
-1.3533002898372	1.34482324438388\\
0.788163082342835	2.6281575200801\\
2.62994410822462	3.58116089787993\\
3.97684522927382	4.1739541276016\\
4.89332317159997	4.50983315037412\\
5.51049081152359	4.69249098100198\\
5.93412383523154	4.78905695763134\\
6.23362438927599	4.83753309979793\\
6.45206700619369	4.8587482316194\\
6.61613611895751	4.8641942745862\\
};
\addplot [color=mycolor2, forget plot]
  table[row sep=crcr]{%
-0.808897193889222	1.2068070685132\\
-0.67471894432654	1.20245833535103\\
-0.523610279594059	1.1879328887286\\
-0.355602286690072	1.16095101337708\\
-0.172113986428172	1.11941457990547\\
0.0236781959832208	1.06184570982191\\
0.226763932915542	0.987874892639806\\
0.430695552019423	0.898606246212317\\
0.628471241032463	0.796666634737447\\
0.813666088851403	0.685849809625889\\
0.981411473905152	0.570458525142341\\
1.12889670178906	0.454591664773143\\
1.25532005519737	0.341614771537379\\
1.3614533580257	0.233920459000961\\
1.44906751721652	0.132944344606522\\
1.52041068524428	0.0393323749214226\\
1.57782741047141	-0.0468418764510821\\
1.62352634881597	-0.125874872234994\\
1.65946499918193	-0.198273711617701\\
1.68731177841586	-0.264643643790927\\
1.70845233617472	-0.325615284853456\\
1.7240172970881	-0.381801300816121\\
1.7349175934188	-0.433773178884676\\
1.74187988803718	-0.482050834467961\\
1.74547854069128	-0.527099946580128\\
1.74616281641526	-0.569333639502137\\
1.74427919808968	-0.609116372085116\\
1.74008919611336	-0.64676873530157\\
1.73378323461243	-0.682572399495952\\
1.72549120574174	-0.716774788177067\\
1.71529021613791	-0.749593254844176\\
1.70320995215191	-0.78121865142323\\
1.68923598730834	-0.811818232943133\\
1.67331125692419	-0.841537862784264\\
1.65533583439673	-0.870503477728018\\
1.63516506164871	-0.898821748127393\\
1.61260601211971	-0.926579827945359\\
1.58741219891264	-0.953844031504873\\
1.55927638618584	-0.980657195786712\\
1.52782132637932	-1.00703438474726\\
1.49258824550724	-1.03295646066862\\
1.45302296379723	-1.05836088380746\\
1.40845972292123	-1.08312890825763\\
1.35810318442154	-1.10706813633384\\
1.30100981290468	-1.12988922487335\\
1.23607118507752	-1.1511755157412\\
1.16200398222345	-1.17034471327827\\
1.0773548933259	-1.18660286237797\\
0.980533637945768	-1.19889346213648\\
0.869893534347268	-1.20584951255666\\
0.743884766397251	-1.2057645440841\\
0.601306128617595	-1.19661014135249\\
0.441667761094257	-1.17613898136997\\
0.265638700923344	-1.14211484190948\\
0.075483669219649	-1.09268866130737\\
-0.124686947267792	-1.02687859490506\\
-0.329061590529294	-0.945022081749565\\
-0.530781194663546	-0.849003795889475\\
-0.722990736443141	-0.742102114593396\\
-0.899945764438998	-0.62845748146599\\
-1.05777997546494	-0.512352747117631\\
-1.1947231245095	-0.397568796071803\\
-1.31082900078583	-0.286996637511487\\
-1.40744077312449	-0.182536253222041\\
-1.48662530858524	-0.0852023743823495\\
-1.55071680461233	0.0046725042194259\\
-1.60201264240033	0.0872227383389218\\
-1.64260458317222	0.162867011670197\\
-1.67430690470771	0.232172845558747\\
-1.69864409443724	0.295765188241698\\
-1.71687011040416	0.354269643004955\\
-1.73000118943305	0.408280244530071\\
-1.73885186708886	0.458343413176045\\
-1.74406894338996	0.504951945121495\\
-1.74616114088827	0.548544855252462\\
-1.74552383846293	0.589510369296379\\
-1.74245906604218	0.628190390074527\\
-1.73719127666227	0.664885439876693\\
-1.72987949587503	0.699859508876439\\
-1.72062641237825	0.733344499759449\\
-1.70948488701277	0.765544109744199\\
-1.6964622550066	0.796637072143682\\
-1.68152269475052	0.82677971561631\\
-1.66458784175479	0.856107805505731\\
-1.64553574026817	0.884737616685108\\
-1.62419814702131	0.912766155007379\\
-1.60035613145138	0.940270395453863\\
-1.57373385601163	0.967305337574049\\
-1.54399037377056	0.993900589216507\\
-1.51070925984532	1.02005507320129\\
-1.47338592058589	1.04572930415321\\
-1.43141254031662	1.0708345031942\\
-1.38406089980276	1.09521761509475\\
-1.33046384917194	1.11864109614215\\
-1.26959722289676	1.14075622649914\\
-1.20026571257888	1.16106882787121\\
-1.12109901138918	1.1788969420808\\
-1.0305687562377	1.19332178033102\\
-0.927042489768694	1.20313689765039\\
-0.808897193889222	1.2068070685132\\
};
\addplot [color=mycolor1, forget plot]
  table[row sep=crcr]{%
4.56917235808472	3.70268895532509\\
4.69993758331261	3.69869279347102\\
4.80517945949634	3.68875692811782\\
4.891315800367	3.67505917995538\\
4.96290040132152	3.65895608509231\\
5.02321828383969	3.64129574673471\\
5.07467792544773	3.62260597381011\\
5.11907237396694	3.60320901432834\\
5.15775539338562	3.58329249739931\\
5.19176199696943	3.56295381135076\\
5.22189209387517	3.54222806388184\\
5.24876929621458	3.52110568085189\\
5.27288272085366	3.49954330350572\\
5.29461693707984	3.4774702130657\\
5.31427347814993	3.45479163779903\\
5.33208619457745	3.43138974763909\\
5.34823195812602	3.40712277935365\\
5.36283768770315	3.38182247942628\\
5.3759842700924	3.3552898503993\\
5.3877076260163	3.32728900317498\\
5.39799687590262	3.29753872307469\\
5.40678924438351	3.26570112173645\\
5.41396095682356	3.23136643455092\\
5.41931285689038	3.19403258541623\\
5.42254870979102	3.15307750500636\\
5.42324298996123	3.10772124497457\\
5.42079311891639	3.05697350605851\\
5.41434817106706	2.9995600221317\\
5.40270123876636	2.93381789265783\\
5.38412463252741	2.85754479201215\\
5.35611365638706	2.767779099135\\
5.31498215350318	2.66047633057464\\
5.25521576712938	2.53003141868109\\
5.16843041292733	2.36857973242306\\
5.04170453134492	2.1650099448242\\
4.85499737117693	1.90370247719085\\
4.57753953354758	1.56336054093121\\
4.16419338875672	1.11739931029442\\
3.55651701769009	0.539884205672537\\
2.70077846049034	-0.176194625640385\\
1.59598179398859	-0.988275705995668\\
0.34581999886164	-1.79163729311536\\
-0.868505547238941	-2.46843827368793\\
-1.89989521026677	-2.96104252617115\\
-2.70104992150966	-3.28328830450124\\
-3.2968512620211	-3.4801094636453\\
-3.73503601365084	-3.59459902752057\\
-4.0596436972177	-3.65770455314172\\
-4.30398704887353	-3.68925815552074\\
-4.49148882414584	-3.7014315428108\\
-4.63824137087147	-3.70161391312862\\
-4.75529535780737	-3.69430712149698\\
-4.85031570845045	-3.68227869809872\\
-4.92869739665469	-3.66724603675176\\
-4.99430017400813	-3.65028169006803\\
-5.04993128854323	-3.63205557469415\\
-5.09766502239726	-3.61298162986896\\
-5.13905677865899	-3.59330788855458\\
-5.17528853037799	-3.57317251755698\\
-5.2072690591085	-3.55263902527036\\
-5.23570398384655	-3.53171846300699\\
-5.26114528157326	-3.51038332169017\\
-5.28402664366843	-3.4885759784651\\
-5.30468886032385	-3.46621343317026\\
-5.32339802304281	-3.44318938443409\\
-5.34035840243682	-3.41937425196426\\
-5.35572121908679	-3.39461345056524\\
-5.3695900665462	-3.36872399804908\\
-5.38202339202143	-3.34148935081348\\
-5.39303413670034	-3.31265217544548\\
-5.40258633778172	-3.28190455342899\\
-5.41058815087062	-3.24887484707502\\
-5.41688030653274	-3.21311008692461\\
-5.42121838611503	-3.17405221510248\\
-5.42324636218192	-3.13100574642952\\
-5.4224573917147	-3.08309325166024\\
-5.4181355313962	-3.02919330887705\\
-5.40926827985627	-2.9678528711087\\
-5.39441364137628	-2.89716183780545\\
-5.37149503593939	-2.81457122256595\\
-5.33747996512858	-2.71662666136708\\
-5.28786926267767	-2.59857519079825\\
-5.2158766204963	-2.45378606231255\\
-5.11110805423888	-2.27291421151429\\
-4.9574722721946	-2.0427622478448\\
-4.730065934337	-1.7449762831817\\
-4.39127956248136	-1.35535458235569\\
-3.88852005730529	-0.846294424062207\\
-3.16171272147315	-0.198036167364282\\
-2.17576608285899	0.575550783655259\\
-0.978405967993526	1.39910336774656\\
0.277135513527748	2.15141677350008\\
1.41235144448438	2.73817910064784\\
2.32873253767492	3.14086783121862\\
3.0217321914352	3.3944321931087\\
3.53266232012548	3.54541338136133\\
3.90919611930924	3.63111514804455\\
4.19018187717937	3.67652388323898\\
4.40369542251996	3.6972200520764\\
4.56917235808472	3.70268895532509\\
};
\addplot [color=mycolor2, forget plot]
  table[row sep=crcr]{%
-0.264450999198444	0.83214045157284\\
0.214248321186214	0.816914604110463\\
0.67404969566841	0.773214467044598\\
1.07668900558516	0.709144913196186\\
1.40495128677245	0.635400964505414\\
1.66060100929812	0.560697628911048\\
1.85510232803173	0.490203198683265\\
2.00198977903837	0.426152500911911\\
2.11319212545656	0.36900377396721\\
2.19803948641445	0.31834386558543\\
2.26344449923751	0.27342323238087\\
2.31441326848521	0.23342546846939\\
2.35454816309746	0.197585432330489\\
2.38644761730663	0.165230879414624\\
2.41199890705233	0.135788848554691\\
2.43258512757489	0.108777538152803\\
2.44922954376228	0.0837934455534409\\
2.46269600712489	0.0604980669807242\\
2.47355903123215	0.0386058415277215\\
2.48225297757613	0.017873829332876\\
2.48910679927453	-0.00190689675315377\\
2.49436870924686	-0.0209183594280127\\
2.49822372585025	-0.0393215490951768\\
2.50080609268576	-0.0572613155803396\\
2.50220791857965	-0.0748705063894331\\
2.50248493397078	-0.092273546530673\\
2.5016599400423	-0.109589626731456\\
2.49972428823439	-0.126935644927175\\
2.49663753563351	-0.144429030651753\\
2.49232524956892	-0.162190573051497\\
2.48667475943331	-0.180347369905173\\
2.47952845167494	-0.199036016081142\\
2.47067394697809	-0.218406153292496\\
2.45983014943958	-0.238624505032164\\
2.44662766294206	-0.259879513653218\\
2.43058135236097	-0.28238666560122\\
2.41105177329823	-0.306394505602213\\
2.38719064187274	-0.33219114112874\\
2.35786324977618	-0.360110606123845\\
2.32153750736643	-0.390537555348389\\
2.27612499249618	-0.423906948327799\\
2.2187544573104	-0.460691807730163\\
2.14545511915088	-0.501365289826523\\
2.05073370658385	-0.546310803193739\\
1.92706761617213	-0.59563317347751\\
1.76445875015479	-0.648796620887027\\
1.55049610602075	-0.704004144224202\\
1.27194805388467	-0.75732112120375\\
0.919473693297603	-0.801912560176133\\
0.49613899942984	-0.828489649186642\\
0.025692080922134	-0.828340341005733\\
-0.449320029613414	-0.798245514008142\\
-0.88412407220191	-0.743051905881648\\
-1.25034969255114	-0.672857552252268\\
-1.54125404579476	-0.597764286432564\\
-1.76463721550565	-0.524729734527018\\
-1.93369422062092	-0.457314658533467\\
-2.06140671142647	-0.396728677678376\\
-2.15842705775399	-0.34290581788556\\
-2.23282103499755	-0.295217495165058\\
-2.29048061161413	-0.252858746887685\\
-2.33565292845559	-0.215030107629663\\
-2.37139548208082	-0.181010551314087\\
-2.39992046853641	-0.150177905945483\\
-2.42284161642236	-0.122006247725035\\
-2.44134718298104	-0.0960546201814722\\
-2.45632033484567	-0.0719536257249904\\
-2.46842298187979	-0.0493926339595112\\
-2.47815443889097	-0.0281085739030853\\
-2.48589273303814	-0.00787648831769808\\
-2.49192386517619	0.0114982926954471\\
-2.49646261608196	0.0301866086436648\\
-2.49966732599551	0.0483408202054516\\
-2.5016502881176	0.0660992978795976\\
-2.50248485714598	0.0835902585406819\\
-2.50220999635652	0.100935130134395\\
-2.50083271248245	0.11825159952906\\
-2.49832861592892	0.135656479949484\\
-2.49464066471791	0.153268522416851\\
-2.48967597965003	0.171211289612278\\
-2.48330043261038	0.189616209620683\\
-2.47533048441306	0.208625929581672\\
-2.46552145099452	0.228398092645316\\
-2.45355096278771	0.249109660476085\\
-2.4389957876638	0.270961887204528\\
-2.42129931899874	0.294185997820034\\
-2.39972575103622	0.319049491914846\\
-2.37329508448751	0.345862697986741\\
-2.34069038956722	0.374984577974526\\
-2.30012498648155	0.40682550239901\\
-2.24915245698217	0.441842163594816\\
-2.18439784997777	0.480514825696315\\
-2.101188611267	0.523287783563198\\
-1.99308212590181	0.570437524813583\\
-1.85135785953935	0.621808199153835\\
-1.66474020557732	0.67632943271341\\
-1.42005374316342	0.731253673206172\\
-1.10516915155054	0.781251166446899\\
-0.715719010491333	0.818066616494719\\
-0.264450999198446	0.83214045157284\\
};
\addplot [color=mycolor1, forget plot]
  table[row sep=crcr]{%
3.17677417039348	3.06220363740278\\
3.31935043399715	3.05782096511845\\
3.43769887712985	3.046630132178\\
3.53706168261554	3.03081663195866\\
3.62140584972235	3.01183421791775\\
3.6937481576491	2.9906465340053\\
3.75639911613872	2.96788699145282\\
3.81114261030455	2.94396424135472\\
3.85936723600453	2.91913183226332\\
3.90216223782819	2.89353430231897\\
3.94038776148057	2.86723767038152\\
3.97472648412622	2.84024947963139\\
4.0057216751557	2.81253172303048\\
4.03380526928449	2.78400879558825\\
4.05931847405358	2.75457184164777\\
4.08252666933673	2.72408034563199\\
4.10362979670892	2.69236145438735\\
4.12276901302913	2.6592072545807\\
4.14003004327764	2.62437001620671\\
4.15544337087205	2.58755522164173\\
4.16898111242984	2.54841200325468\\
4.1805501005934	2.50652038741957\\
4.189980298719	2.46137446315747\\
4.19700713642881	2.41236022919505\\
4.20124560249939	2.35872638667083\\
4.20215284043759	2.29954569137659\\
4.19897438473111	2.23366361187286\\
4.19066679605515	2.15962992618775\\
4.17578595253724	2.07560756621235\\
4.15232521745815	1.97925172399264\\
4.11748082877196	1.86755173387132\\
4.0673135628575	1.73663058035474\\
3.99626893137781	1.58150700167523\\
3.89652289800053	1.39585416370095\\
3.75716409757238	1.17185931374062\\
3.56337212632166	0.900438705505084\\
3.29612665441547	0.572327965997373\\
2.93370576079621	0.180882711667256\\
2.45705472102703	-0.272674545189229\\
1.86050778278718	-0.772482316597883\\
1.16451685236404	-1.2845483777012\\
0.419258436400468	-1.76359479187261\\
-0.310451065573799	-2.17009581352747\\
-0.970512410976555	-2.48499188487467\\
-1.53304219695017	-2.71091402645631\\
-1.99465023348299	-2.86314635273768\\
-2.36617534434433	-2.96004404359145\\
-2.6633874611004	-3.01771046191725\\
-2.90173787310418	-3.04841749658235\\
-3.09431332561097	-3.06087319520195\\
-3.25147897947426	-3.06103735904053\\
-3.38119118486251	-3.05291928331043\\
-3.48947990122615	-3.03919651729106\\
-3.58090349168907	-3.02165201979975\\
-3.65891865332526	-3.00147015489306\\
-3.72616295980993	-2.97943339228997\\
-3.78466451464927	-2.95605216745467\\
-3.83599561684905	-2.9316505741551\\
-3.88138499309718	-2.90642301715518\\
-3.92179985990072	-2.88047171211381\\
-3.95800612186526	-2.85383144118423\\
-3.99061268873467	-2.82648570719649\\
-4.02010416898246	-2.7983769604186\\
-4.04686494837156	-2.76941261446765\\
-4.07119676264246	-2.73946793465286\\
-4.09333122120634	-2.70838645170393\\
-4.11343825448174	-2.67597824779936\\
-4.13163108182871	-2.64201622778269\\
-4.14796798375716	-2.60623028989809\\
-4.16245087258137	-2.56829911936056\\
-4.17502035336242	-2.52783912031028\\
-4.18554661162926	-2.48438975275164\\
-4.19381500561436	-2.43739422323724\\
-4.19950460942144	-2.38617405770586\\
-4.20215704988444	-2.32989552089704\\
-4.20113165723341	-2.26752509195969\\
-4.1955409938594	-2.19777021687624\\
-4.18415793433676	-2.11900032717899\\
-4.16528125285283	-2.02914175339791\\
-4.13654073378355	-1.92553911355845\\
-4.0946150919557	-1.8047762910793\\
-4.03482779602556	-1.66245544209737\\
-3.9505828429046	-1.4929500758959\\
-3.83262165276153	-1.28919420220208\\
-3.66816704733271	-1.04267411268889\\
-3.44026358955362	-0.743995026519903\\
-3.12816604364904	-0.384708680621449\\
-2.71048582054154	0.0387050776214842\\
-2.17322418619597	0.518512363739549\\
-1.52242222011607	1.02971849467443\\
-0.794040529230507	1.53102853372387\\
-0.0485561944193281	1.97767233534558\\
0.651596897764944	2.33925712839912\\
1.26464629340802	2.60828678671132\\
1.77598975688712	2.79508091712544\\
2.19072196670672	2.91742086361568\\
2.52306177366165	2.99292260089135\\
2.78903776233008	3.03581499533663\\
3.00303791269362	3.05649995546078\\
3.17677417039348	3.06220363740278\\
};
\addplot [color=mycolor2, forget plot]
  table[row sep=crcr]{%
0.661273782960425	0.866962186619865\\
1.15796205284672	0.851693038651411\\
1.55385438750588	0.814423660462159\\
1.85381732281985	0.766885959235898\\
2.07617283338148	0.717021594888785\\
2.24051638617859	0.669030895999442\\
2.36292503570404	0.624672111410093\\
2.45527929738384	0.584396976084643\\
2.52599281815815	0.548048739106013\\
2.5809449574233	0.515230268539075\\
2.62424905343567	0.485480899947359\\
2.65880723615788	0.458354204400774\\
2.68669128762984	0.433447573305808\\
2.70939944994863	0.410409558737287\\
2.72802917122473	0.388937630991543\\
2.74339392644928	0.368772335789921\\
2.75610293693549	0.349690567260384\\
2.76661614353836	0.331499105872901\\
2.77528250747898	0.314028834342263\\
2.78236692940358	0.297129709844776\\
2.78806927369543	0.280666427697226\\
2.79253780974802	0.264514657597637\\
2.79587860673463	0.248557718915537\\
2.79816189726337	0.232683562264001\\
2.79942606438194	0.216781929025155\\
2.79967964434784	0.200741562886482\\
2.79890153260681	0.184447344700957\\
2.79703940171752	0.167777212056612\\
2.79400616164061	0.150598705922155\\
2.78967408895369	0.132764956360156\\
2.78386599120974	0.114109874654288\\
2.77634241366005	0.0944422566234181\\
2.76678337573911	0.0735384172667591\\
2.75476234844049	0.0511328667796173\\
2.73970900106703	0.0269064030239178\\
2.72085542073849	0.000470848901667607\\
2.69715767306884	-0.028650448966868\\
2.66718017648142	-0.0610471935020732\\
2.62892365232438	-0.0974520813058362\\
2.57956760595732	-0.138774628197668\\
2.51508546665734	-0.186132564282447\\
2.42967877912178	-0.240862688463524\\
2.31498413577605	-0.304466200882717\\
2.15909101178252	-0.378387607580392\\
1.94572497836975	-0.463428493655018\\
1.65480195610755	-0.558496052466978\\
1.26711459734576	-0.658537049264833\\
0.77657514989749	-0.752577661258321\\
0.207369162339691	-0.824951408390095\\
-0.381574349653424	-0.862479910369312\\
-0.921511686533697	-0.862890828216087\\
-1.36871821476444	-0.835005611462908\\
-1.71481278672137	-0.791341871441737\\
-1.97341590826614	-0.741908687426995\\
-2.16448928144107	-0.692640408697396\\
-2.30612758268735	-0.646348835619765\\
-2.41226322272021	-0.604025166399864\\
-2.49292438342471	-0.565753910039911\\
-2.55514806487159	-0.531226821686693\\
-2.60384842392396	-0.500000681614672\\
-2.64247594030227	-0.471616062174229\\
-2.67347880035625	-0.445646546514242\\
-2.69861595862575	-0.421714983761723\\
-2.71916769755915	-0.399494981073737\\
-2.73607754777726	-0.378706371822785\\
-2.75004866970255	-0.359108715867074\\
-2.76160996555653	-0.340494620357544\\
-2.77116190922859	-0.322683592178605\\
-2.77900862574165	-0.305516634102342\\
-2.78538051207993	-0.288851575556715\\
-2.7904502368164	-0.272559038920184\\
-2.7943440032954	-0.25651891211297\\
-2.79714932739429	-0.240617193347299\\
-2.79892014917985	-0.224743077476029\\
-2.79967979249433	-0.20878615729131\\
-2.79942205736934	-0.192633613239204\\
-2.79811054174722	-0.176167258913038\\
-2.79567611360316	-0.15926029546696\\
-2.79201226690305	-0.141773603712912\\
-2.78696786713369	-0.123551365565637\\
-2.7803364887445	-0.104415753390757\\
-2.7718411169053	-0.0841603527255041\\
-2.7611123526863	-0.0625418867677141\\
-2.74765730453616	-0.0392696879289773\\
-2.73081488156519	-0.0139922179360294\\
-2.70969093033265	0.0137202006216811\\
-2.68306312346527	0.0443973335648985\\
-2.64924005970246	0.0786963249046509\\
-2.60585085623834	0.117433966448259\\
-2.54953005584578	0.161620602371638\\
-2.47544936887848	0.212484709782595\\
-2.37664170939938	0.271459228033626\\
-2.24309728766334	0.34006138581561\\
-2.06078357220008	0.419521432230619\\
-1.81127955202615	0.509903493366639\\
-1.47394379994429	0.608430286702134\\
-1.03407711740967	0.707231481400013\\
-0.498790464781053	0.792473568594955\\
0.0891670364609763	0.848526699392398\\
0.661273782960423	0.866962186619865\\
};
\addplot [color=mycolor1, forget plot]
  table[row sep=crcr]{%
2.35723974553073	2.84416812417343\\
2.51069738272667	2.83942105928315\\
2.64251651332025	2.82693423750504\\
2.75649096143632	2.80877852846073\\
2.85571591958509	2.78643416026401\\
2.94269893019672	2.76094838995348\\
3.0194661373763	2.73305269663057\\
3.08765471098761	2.70324792118058\\
3.14858937420538	2.6718652646589\\
3.20334402719018	2.63910956301037\\
3.25279054010171	2.60508969437066\\
3.29763695043819	2.56983965516583\\
3.33845707939824	2.53333281809185\\
3.37571323580098	2.49549112362842\\
3.40977331694871	2.45619039921715\\
3.44092328605599	2.4152625915511\\
3.46937571450358	2.37249539254628\\
3.49527481797884	2.32762950385245\\
3.51869817541443	2.28035359214749\\
3.53965508182247	2.23029681840946\\
3.55808123108122	2.17701866432022\\
3.57382912947303	2.11999561772275\\
3.58665327772499	2.05860411151641\\
3.5961886953016	1.99209893948848\\
3.60192075740274	1.91958621630789\\
3.60314353362642	1.83998985438197\\
3.59890283311901	1.75201060116323\\
3.5879190029878	1.65407712794404\\
3.56848336662896	1.54428989508587\\
3.53832153606045	1.42036130172444\\
3.49441792939479	1.27956130976565\\
3.43280137970586	1.11868852519635\\
3.34830708785551	0.934105777173234\\
3.23436466491018	0.72190970105947\\
3.08292848391465	0.478344752388961\\
2.88477324192349	0.200607294019178\\
2.63049975041719	-0.111837186200972\\
2.31261787611998	-0.455485174088717\\
1.92874622554254	-0.82108653917307\\
1.48506617658905	-1.1931092075257\\
0.998035770061693	-1.55162561181268\\
0.492319908450856	-1.87674651350774\\
-0.00500095663271922	-2.15371625030846\\
-0.47105533379721	-2.37591157248377\\
-0.891240774070467	-2.54450272708476\\
-1.25958652926944	-2.66583184732752\\
-1.57666258567759	-2.74841093472104\\
-1.84686814876165	-2.80074834040904\\
-2.07622761179615	-2.83023154463703\\
-2.27099612042322	-2.84278135592224\\
-2.43694060189584	-2.84291981785256\\
-2.57905701509396	-2.83399968754863\\
-2.70152578587721	-2.81846070138456\\
-2.80777793317631	-2.79805587627732\\
-2.90060091198471	-2.7740319654112\\
-2.98224904339135	-2.74726589081205\\
-3.05454342556257	-2.71836484841287\\
-3.11895641284093	-2.68773847340791\\
-3.1766804691153	-2.65565027605371\\
-3.22868309348335	-2.62225396515838\\
-3.27575004481665	-2.58761881845915\\
-3.31851901725403	-2.55174708859276\\
-3.35750561464144	-2.51458554782193\\
-3.39312311075937	-2.47603262256475\\
-3.42569713509985	-2.43594209198227\\
-3.45547611349754	-2.39412397304877\\
-3.48263801896802	-2.35034294797793\\
-3.50729374033659	-2.30431447852061\\
-3.52948713952374	-2.25569857292167\\
-3.54919162456181	-2.20409100832365\\
-3.56630279339919	-2.14901165176407\\
-3.58062637817612	-2.08988935876831\\
-3.59186031064004	-2.02604275824673\\
-3.599569200735	-1.95665606520587\\
-3.60314883250387	-1.88074892881337\\
-3.60177739968447	-1.79713929342847\\
-3.59434912274173	-1.70439847331892\\
-3.57938469372077	-1.60079841368833\\
-3.55491199194827	-1.48425299354551\\
-3.51831051117761	-1.35225924191538\\
-3.46611579371748	-1.20185223707173\\
-3.39378973718194	-1.02960194588312\\
-3.295486126987	-0.831704729697735\\
-3.16388956206597	-0.604258631528481\\
-2.99029254310449	-0.343853312370394\\
-2.76519701600118	-0.0486215255346563\\
-2.4798214882969	0.280188756232953\\
-2.12877973569313	0.636373170420278\\
-1.71359192911296	1.00747964391791\\
-1.24556624327728	1.37536326684738\\
-0.745794375128828	1.71945418960585\\
-0.241015860211289	2.0218665981489\\
0.243063883633684	2.27174582229565\\
0.687439762821932	2.46659503015395\\
1.08197874485966	2.6105629683591\\
1.42431832352954	2.71141548526532\\
1.7172655368299	2.77786498936206\\
1.96625697011267	2.81794182510089\\
2.17755897671675	2.83831004944716\\
2.35723974553073	2.84416812417343\\
};
\addplot [color=mycolor2, forget plot]
  table[row sep=crcr]{%
1.18991530942429	0.996510309110826\\
1.60054088847078	0.984021430456015\\
1.91377244156376	0.954577655347213\\
2.14791203027807	0.917473554270012\\
2.32251535465373	0.878305494358859\\
2.45373375878422	0.839972813807476\\
2.55360600181979	0.803767192939426\\
2.63073362704449	0.77012093230025\\
2.69117823317898	0.739041899974995\\
2.73921392535886	0.710346566448502\\
2.77787795324581	0.683778702006049\\
2.80935286990539	0.659067073284671\\
2.83522664770944	0.635951634127522\\
2.85666909055672	0.614193833395493\\
2.874552002102	0.593579139214968\\
2.8895317218807	0.573915920283808\\
2.90210637368735	0.555032767841512\\
2.91265596359902	0.536775278792383\\
2.92147069986961	0.51900277283484\\
2.92877109843297	0.501585135454873\\
2.93472224979762	0.484399835099438\\
2.93944383389818	0.467329089405828\\
2.94301693519272	0.450257116591205\\
2.94548833765155	0.43306738503193\\
2.94687270661257	0.415639756143704\\
2.94715284928651	0.397847396552401\\
2.94627805622044	0.379553310728468\\
2.94416033513793	0.360606310769947\\
2.94066812865059	0.340836191532789\\
2.93561682499492	0.320047811504501\\
2.92875498041601	0.298013685916316\\
2.91974460543797	0.274464570164272\\
2.90813302103366	0.24907733921039\\
2.89331250200592	0.221459244560428\\
2.87446193899167	0.191127356986687\\
2.85046167457741	0.157481714562739\\
2.81976792364993	0.119770512339495\\
2.78022600914904	0.0770459308095315\\
2.72879132806117	0.0281107725380253\\
2.66111397018444	-0.0285389374000364\\
2.57093264307373	-0.0947569876766271\\
2.44923757809347	-0.17272541723286\\
2.28326719813619	-0.264748920737332\\
2.05576903546383	-0.372618964547878\\
1.74588642840482	-0.496154129648031\\
1.33459016541791	-0.630655150650412\\
0.817792878440208	-0.764236171010543\\
0.223210775629237	-0.878588321361184\\
-0.387488105528008	-0.956662577001607\\
-0.94544396218029	-0.992562807753562\\
-1.40823102692931	-0.993115571787897\\
-1.7683138275155	-0.970744942497015\\
-2.03945352735151	-0.936555785635127\\
-2.24154691063925	-0.89791728601457\\
-2.39270406004862	-0.858926887362741\\
-2.50698083273759	-0.821563309124965\\
-2.59458434079788	-0.786617165495715\\
-2.66273967323147	-0.754269284531161\\
-2.71653355994551	-0.724411464036381\\
-2.75956449421103	-0.696813713045393\\
-2.79440326481326	-0.671207546223913\\
-2.8229086170185	-0.64732529779844\\
-2.84644142929244	-0.62491699822928\\
-2.86601018748154	-0.603756084633368\\
-2.88237045822303	-0.58363974573412\\
-2.8960935416649	-0.564386845692692\\
-2.90761432763437	-0.545834891706278\\
-2.91726496392581	-0.527836746921811\\
-2.92529870926235	-0.510257397710516\\
-2.93190687920515	-0.49297088342211\\
-2.93723082678732	-0.475857394184777\\
-2.94137025218947	-0.4588004893601\\
-2.94438869164446	-0.441684360051673\\
-2.94631671937614	-0.424391039555058\\
-2.94715315692405	-0.406797447755458\\
-2.94686438542758	-0.388772134015282\\
-2.94538166956615	-0.370171553943397\\
-2.94259620012493	-0.350835674540475\\
-2.93835131578995	-0.330582644817653\\
-2.93243103516107	-0.309202189085831\\
-2.92454356182793	-0.286447270097373\\
-2.91429773507599	-0.262023419987743\\
-2.90116935647461	-0.235574939359954\\
-2.88445272456195	-0.206666914582644\\
-2.86319023870879	-0.174761714560406\\
-2.83606910795067	-0.13918836809934\\
-2.80126834127463	-0.099103196869256\\
-2.75623050704679	-0.0534408333077383\\
-2.69732087016683	-0.000857646209580513\\
-2.61932360436197	0.0603221566113784\\
-2.51472211999705	0.132126713227889\\
-2.37275701371166	0.21684430587183\\
-2.17846019986387	0.316643567600139\\
-1.9124699695339	0.43258002030732\\
-1.55372938073449	0.562588128013722\\
-1.08855103789696	0.69860911923824\\
-0.526985364418182	0.825043987363956\\
0.0847239315437372	0.922867884974586\\
0.676587249516098	0.979694143100387\\
1.18991530942428	0.996510309110826\\
};
\addplot [color=mycolor1, forget plot]
  table[row sep=crcr]{%
1.84790736958456	2.85539624508498\\
2.00892487226021	2.85038734938052\\
2.15161460587838	2.83684839195735\\
2.27847365068873	2.81662215054955\\
2.39169364626266	2.79111153615171\\
2.49316493440788	2.76136860581943\\
2.58449971813475	2.72816920137418\\
2.66706285803228	2.69207280870998\\
2.7420039377821	2.65346917087495\\
2.81028731684173	2.61261379962829\\
2.87271868648525	2.56965450771895\\
2.92996763362065	2.52465080306163\\
2.98258621771325	2.47758763347676\\
3.03102378366391	2.42838463079329\\
3.07563829063698	2.37690170719065\\
3.1167044047097	2.32294161098691\\
3.15441852079681	2.2662498495781\\
3.18890076607437	2.20651222680153\\
3.22019390087017	2.14335011429239\\
3.24825887456952	2.07631347802289\\
3.27296661059502	2.00487161356855\\
3.29408538192824	1.9284015154322\\
3.311262895117	1.84617383684868\\
3.32400193215177	1.7573365242676\\
3.33162812979233	1.66089649997351\\
3.33324826531379	1.55570032408616\\
3.32769739679072	1.44041576290418\\
3.31347363266625	1.31351787885098\\
3.28866065417024	1.1732859918368\\
3.25084120056975	1.01782206668806\\
3.19701083076644	0.84510710050348\\
3.12351212828304	0.653119751975032\\
3.02602680639661	0.440049138668592\\
2.89968682572986	0.204636619901509\\
2.73938992651037	-0.0533306138224429\\
2.54041231199207	-0.332393904670119\\
2.29936787692003	-0.628772510767444\\
2.01543038031951	-0.935917839041958\\
1.6915119616486	-1.24459598152934\\
1.33488281903928	-1.54376444301386\\
0.956733747481965	-1.8222114167435\\
0.570572069272705	-2.07049238790772\\
0.189937153897144	-2.28244672825619\\
-0.173716559496603	-2.45575876703295\\
-0.512282604500782	-2.5915247044428\\
-0.821236103988375	-2.6932132753532\\
-1.09911022893855	-2.76551271300529\\
-1.34663993521902	-2.81339851505886\\
-1.56590874874089	-2.84153589706066\\
-1.7596689634603	-2.85398154078695\\
-1.93087453770099	-2.85409308307999\\
-2.08239763586714	-2.84455751625081\\
-2.21687874717862	-2.82747429843868\\
-2.33666353451119	-2.80445436217213\\
-2.44379094050302	-2.77671491761417\\
-2.54000866850522	-2.74516153003493\\
-2.62680121605391	-2.71045525623359\\
-2.70542187198806	-2.67306562981485\\
-2.77692404965429	-2.63331144754917\\
-2.84218970045887	-2.59139154037586\\
-2.9019538958854	-2.54740752817672\\
-2.95682537871724	-2.50138022522718\\
-3.00730322378136	-2.45326101069311\\
-3.05378987429397	-2.40293915875533\\
-3.09660082588792	-2.35024585224454\\
-3.13597116970703	-2.2949553819004\\
-3.17205910656379	-2.23678385430369\\
-3.20494641896013	-2.17538558811419\\
-3.23463574062408	-2.11034726537688\\
-3.26104429275771	-2.04117982108518\\
-3.28399355861275	-1.96730800458932\\
-3.30319413986797	-1.88805754448315\\
-3.31822478069885	-1.80263992193353\\
-3.32850427156098	-1.71013495560625\\
-3.33325469357533	-1.6094718081165\\
-3.33145432608481	-1.4994097747024\\
-3.32177869960428	-1.37852151887596\\
-3.30252908668592	-1.24518358106745\\
-3.27154981225758	-1.09758240218248\\
-3.22614016820033	-0.933749189986639\\
-3.16297496042255	-0.751643881709563\\
-3.07806162632474	-0.549316485789908\\
-2.96678257255502	-0.325180258718806\\
-2.82409665235852	-0.0784282751791394\\
-2.64499270725013	0.190400163799164\\
-2.42527520518298	0.478760764606481\\
-2.16267778572755	0.781519542224893\\
-1.85811745768085	1.09072924063741\\
-1.51666503575119	1.39608862323249\\
-1.14768296517624	1.68623139338685\\
-0.763781266444752	1.95059658468864\\
-0.378778991262895	2.18124436867327\\
-0.00539508952684124	2.37393399176962\\
0.346498809368338	2.52815978543668\\
0.670614869027265	2.64635173309969\\
0.964053154865023	2.73272374538172\\
1.22656050661286	2.79220371918969\\
1.45964412621222	2.82966611705173\\
1.66579343430517	2.84949388873804\\
1.84790736958455	2.85539624508498\\
};
\addplot [color=mycolor2, forget plot]
  table[row sep=crcr]{%
1.50442737389697	1.14866712477609\\
1.84486918014159	1.13833029946581\\
2.10448822423007	1.11391457163613\\
2.3011971502059	1.08272290206621\\
2.45090974726946	1.04912002215542\\
2.56603302766523	1.01547382385565\\
2.65570683287098	0.982953137479371\\
2.72651451942016	0.952054315709736\\
2.78317393134374	0.922913951356243\\
2.82907939203983	0.895485069268227\\
2.86669388594656	0.869633380781392\\
2.89782359894184	0.845188565340106\\
2.9238079099883	0.821970740313312\\
2.94565078800753	0.799803392836786\\
2.96411211305387	0.778518992555047\\
2.97977162374811	0.757960692973232\\
2.99307406546933	0.737981977689385\\
3.00436130164132	0.718445251619882\\
3.01389526417741	0.699219902810375\\
3.02187435680076	0.680180095175765\\
3.02844507448732	0.661202401810001\\
3.03371002059801	0.642163299896373\\
3.03773309491289	0.6229364922162\\
3.04054232644606	0.603389979670131\\
3.04213058976255	0.583382773532295\\
3.04245423894509	0.562761098074081\\
3.0414294909429	0.541353887770826\\
3.03892616141411	0.518967322661358\\
3.03475806787604	0.495378063865181\\
3.02866902212632	0.470324740603633\\
3.02031277145188	0.443497090399019\\
3.00922441801632	0.414521954395087\\
2.99477959777149	0.382945070774552\\
2.97613580484955	0.348207292672608\\
2.95214736276928	0.309613518652939\\
2.92124119161162	0.266292391618661\\
2.88123413210468	0.217145053754499\\
2.8290638188195	0.160782847536577\\
2.760394989693	0.0954589923200107\\
2.66905771559782	0.0190128403057332\\
2.54629428349928	-0.0711224631742372\\
2.37989684982899	-0.177729197414635\\
2.15364768360016	-0.303187106141744\\
1.84825198609768	-0.448038536550649\\
1.44615947755651	-0.60844406975839\\
0.942723826257913	-0.773286400684488\\
0.360664582674148	-0.924044527718796\\
-0.246475983542887	-1.04114576216124\\
-0.814639713355748	-1.11403929809116\\
-1.29900475074848	-1.1453421728842\\
-1.6858359904969	-1.14584998396187\\
-1.98365383877806	-1.12734683822528\\
-2.2096341597025	-1.09883498124102\\
-2.38106886874338	-1.06603877517639\\
-2.51215503156031	-1.03220860298567\\
-2.61358846378269	-0.999030512193673\\
-2.69313749782962	-0.967286591441651\\
-2.75637485266642	-0.937264373264917\\
-2.80729884540031	-0.908992680864494\\
-2.8487972968282	-0.882372273200659\\
-2.88297646159004	-0.857246331048154\\
-2.91138955008335	-0.83343748109554\\
-2.93519467329699	-0.810766471372551\\
-2.95526429330422	-0.789060880396075\\
-2.9722615635818	-0.768158466556946\\
-2.98669400668361	-0.747907675110618\\
-2.99895155926237	-0.728166667978184\\
-3.00933370906833	-0.708801604667799\\
-3.01806890596801	-0.68968454834918\\
-3.02532839426926	-0.670691171741342\\
-3.03123591202487	-0.651698322615945\\
-3.0358742176301	-0.632581439134721\\
-3.03928905678174	-0.61321175856261\\
-3.04149092036105	-0.593453225817017\\
-3.04245472768803	-0.573158972216394\\
-3.04211736953351	-0.552167193253061\\
-3.04037283349348	-0.530296201450184\\
-3.0370643805986	-0.507338360176015\\
-3.03197290808334	-0.483052509307378\\
-3.0248001654215	-0.45715436481122\\
-3.01514480943405	-0.429304200989976\\
-3.00246826818674	-0.399090895578792\\
-2.98604584657951	-0.366011128581075\\
-2.96489616756454	-0.329442190017435\\
-2.9376784926671	-0.288606540810655\\
-2.90254216532864	-0.242526202439522\\
-2.85690484271088	-0.189965799366191\\
-2.79712643370924	-0.12936605430645\\
-2.71803656827352	-0.0587780226198686\\
-2.6122765961754	0.0241704844411688\\
-2.4694686152808	0.122197004054777\\
-2.27541753962575	0.238000463103092\\
-2.01207771082531	0.373289058075866\\
-1.66006143102537	0.526794996331922\\
-1.20647397080343	0.69133291546151\\
-0.658778300565647	0.851746047432427\\
-0.0560913168613796	0.987773404311966\\
0.53895165990967	1.08323818102995\\
1.06884545129749	1.13431195086951\\
1.50442737389697	1.14866712477609\\
};
\addplot [color=mycolor1, forget plot]
  table[row sep=crcr]{%
1.51239328471355	2.99187321897874\\
1.67867446084939	2.98667759938608\\
1.82979360645117	2.9723192337742\\
1.96732600132283	2.95037450977582\\
2.09274479981551	2.92210090208251\\
2.20739186151525	2.88848350304977\\
2.31246699395216	2.85027875729118\\
2.40902834647901	2.80805283095023\\
2.49799897887384	2.76221379370481\\
2.58017633617323	2.71303772572031\\
2.65624255967014	2.66068929612474\\
2.72677436488139	2.60523750766455\\
2.79225172868971	2.54666729889171\\
2.85306493714009	2.48488762098636\\
2.909519716944	2.41973650692446\\
2.96184025249874	2.35098355132742\\
3.01016990623873	2.27833013390277\\
3.05456943315731	2.20140765669543\\
3.09501242365782	2.11977403322802\\
3.13137763273854	2.03290867596972\\
3.16343776850579	1.940206292353\\
3.19084423431321	1.84096994158089\\
3.21310727192629	1.73440405875188\\
3.22957098227844	1.61960856793017\\
3.23938288106664	1.49557584701932\\
3.24145810240536	1.36119325479827\\
3.23443928798276	1.21525526659365\\
3.21665487213498	1.05649103857775\\
3.1860812605918	0.883615367048722\\
3.14031868860227	0.69541319788971\\
3.07659653554911	0.490869227322065\\
2.99183113068201	0.269353035253762\\
2.88276573014075	0.0308638592568937\\
2.74622404177374	-0.223675933520155\\
2.57949812677245	-0.492115900945798\\
2.38086022397749	-0.77083354274097\\
2.15013257224817	-1.05465556042521\\
1.88918173053705	-1.33705294216358\\
1.60215899641311	-1.61066869021488\\
1.29533267398154	-1.86812846675396\\
0.976472892525725	-2.10295720056563\\
0.653915809529938	-2.31035347869866\\
0.335560089471391	-2.48761382957053\\
0.0280584572646769	-2.63413417386167\\
-0.263636859153305	-2.75106554317231\\
-0.536360822565975	-2.84078738522603\\
-0.78854748650978	-2.9063621604053\\
-1.01989715152375	-2.95108008829427\\
-1.2310105092283	-2.97813745863803\\
-1.42306013077515	-2.99044419746322\\
-1.59753024291552	-2.9905329869835\\
-1.75602894663267	-2.98053721568007\\
-1.90016322894113	-2.96220964925042\\
-2.03146246118451	-2.93696139900913\\
-2.15133644489144	-2.90590801895207\\
-2.26105652077574	-2.8699151112142\\
-2.36175113824257	-2.82963956649311\\
-2.45440983880741	-2.78556486918132\\
-2.53989160079024	-2.73803019190504\\
-2.61893493296653	-2.68725365585423\\
-2.69216809009164	-2.63335040332824\\
-2.76011842697453	-2.57634618838206\\
-2.82322030867127	-2.51618714518441\\
-2.88182122806917	-2.45274630271093\\
-2.93618590272331	-2.38582731274327\\
-2.9864981671721	-2.31516576443959\\
-3.03286046949205	-2.24042838372721\\
-3.07529073763118	-2.16121036730898\\
-3.11371631342569	-2.07703108769674\\
-3.14796457012237	-1.98732843946967\\
-3.17774974499562	-1.89145219638467\\
-3.20265545172952	-1.7886569420333\\
-3.22211232123201	-1.67809546434025\\
-3.23537031247882	-1.55881402314529\\
-3.24146553402507	-1.42975168383362\\
-3.23918207764659	-1.2897470423625\\
-3.22701062328347	-1.13755722026301\\
-3.20310775495492	-0.971895988183196\\
-3.16526342101049	-0.791500110601493\\
-3.11088911190784	-0.595234941923725\\
-3.03704607498221	-0.382250722140419\\
-2.94054020210392	-0.152197712258686\\
-2.81811512770277	0.09450194458729\\
-2.66677171975208	0.356348286764419\\
-2.48422218721787	0.630477714379279\\
-2.26944337938398	0.912487333112226\\
-2.02322943415994	1.19648116206932\\
-1.74858204526264	1.475429817124\\
-1.45076091491646	1.74185422835084\\
-1.13688708903981	1.98871806337898\\
-0.815141623939223	2.21030456709899\\
-0.49376054385753	2.40283490282641\\
-0.180100480124219	2.56467997680183\\
0.120003514970815	2.69617191608695\\
0.40250210416808	2.79914631394991\\
0.665065380523587	2.87638840670827\\
0.906806359627473	2.93112342909832\\
1.12792007247506	2.96662628190493\\
1.32933028778223	2.98596700459762\\
1.51239328471354	2.99187321897874\\
};
\addplot [color=mycolor2, forget plot]
  table[row sep=crcr]{%
1.72274085757999	1.31026474427909\\
2.01026614301254	1.30152528012605\\
2.23201584329209	1.28065232276943\\
2.40312518212022	1.25350117177955\\
2.5361316963277	1.22363200205133\\
2.6406436810082	1.1930742780253\\
2.72377440734015	1.16291636519221\\
2.79072342258328	1.13369337333612\\
2.84528715314021	1.10562441056408\\
2.89025159926759	1.07875253892653\\
2.9276780567195	1.05302569164154\\
2.95910618091147	1.02834278534591\\
2.98569717105014	1.00457957025386\\
3.00833475323026	0.98160273795666\\
3.02769671139579	0.959277228056095\\
3.04430586543995	0.937469588196718\\
3.05856662245834	0.916049031872386\\
3.07079129744337	0.89488713481783\\
3.081219074456	0.87385669836927\\
3.09002956953075	0.85283006267768\\
3.09735232592094	0.831677002423555\\
3.10327312584447	0.810262240935\\
3.10783767485561	0.788442550430583\\
3.11105295868559	0.766063350952403\\
3.1128863527598	0.742954667906426\\
3.11326235223077	0.718926249874456\\
3.11205655667378	0.693761577131293\\
3.10908625587767	0.667210399324376\\
3.10409657886324	0.638979319139999\\
3.09674062726076	0.608719777182073\\
3.08655122798497	0.576012580850285\\
3.07290077666084	0.540347848747005\\
3.05494390693558	0.501098916829594\\
3.03153513242212	0.457488415542408\\
3.00110980286464	0.408544518113625\\
2.96151132350235	0.353045653896105\\
2.90974057504121	0.289453696806686\\
2.84159626443886	0.215840907962605\\
2.75117357532672	0.129829395360097\\
2.63021169159603	0.0285929215176733\\
2.46737755444351	-0.0909646386826638\\
2.24784969233456	-0.231626738984467\\
1.95417622824219	-0.394518792626553\\
1.57027438833949	-0.576708641133046\\
1.0904661840107	-0.768301080556169\\
0.531564905798336	-0.951569183923089\\
-0.0623418971493884	-1.10568563341469\\
-0.633617989988535	-1.2161021722935\\
-1.13605816033267	-1.28069638907403\\
-1.54939419097584	-1.30745770207334\\
-1.87572926586082	-1.30788835936333\\
-2.12836987817241	-1.29217629577013\\
-2.32304612299803	-1.26759473529956\\
-2.47372557267457	-1.23875160323609\\
-2.591453345156	-1.20835455614213\\
-2.68451842074422	-1.17790230524224\\
-2.75900664524931	-1.14816890818611\\
-2.81935945503074	-1.11950898553585\\
-2.8688261931858	-1.09204052968102\\
-2.90980054357077	-1.06575153233741\\
-2.94406181904487	-1.04056117627549\\
-2.97294553842738	-1.01635444607972\\
-2.99746363021922	-0.993001310143848\\
-3.01838936394926	-0.970366966950246\\
-3.03631768842475	-0.948316912478931\\
-3.05170836803424	-0.926718996332547\\
-3.06491698868893	-0.905443712048879\\
-3.07621730464331	-0.884363429157137\\
-3.08581730005135	-0.863350957130833\\
-3.09387058304999	-0.842277640219219\\
-3.10048420139264	-0.821011062591899\\
-3.10572358823947	-0.799412363166568\\
-3.10961505950969	-0.777333099373584\\
-3.11214605015043	-0.754611546325656\\
-3.11326306418744	-0.731068263346829\\
-3.11286709410248	-0.706500695997612\\
-3.11080600893108	-0.680676501158869\\
-3.10686308067396	-0.653325177135875\\
-3.10074036495277	-0.624127440491527\\
-3.09203500161607	-0.592701605607224\\
-3.08020554612993	-0.558585981669299\\
-3.06452402195011	-0.521216001401317\\
-3.04400726239405	-0.479894455357776\\
-3.01731796332176	-0.433752905048441\\
-2.9826213076251	-0.381702319818754\\
-2.93737677453569	-0.322371818288306\\
-2.87803728329398	-0.254037504262141\\
-2.79962226426849	-0.174551946279693\\
-2.69513849711839	-0.0813055615369774\\
-2.55487350140267	0.0287033971132761\\
-2.36575521837367	0.158525653958557\\
-2.11139719754735	0.310347034826338\\
-1.77424476755579	0.483627088348244\\
-1.34196332706271	0.672275560929426\\
-0.818726179635004	0.862306244954394\\
-0.23541171032341	1.03343880109232\\
0.354231974656627	1.1667926388925\\
0.895347405057608	1.25378950593271\\
1.35412989498003	1.298095646019\\
1.72274085757999	1.31026474427909\\
};
\addplot [color=mycolor1, forget plot]
  table[row sep=crcr]{%
1.28286346141528	3.19376085633426\\
1.45344992176763	3.18841316743345\\
1.61146233751843	3.17338414571857\\
1.75790426515861	3.15000377012599\\
1.89376274945308	3.11936411942584\\
2.01997584878284	3.08234393604552\\
2.13741318854943	3.03963409856881\\
2.24686552217705	2.99176163900756\\
2.34904020189213	2.93911108178861\\
2.44456026355682	2.88194260248964\\
2.53396546817392	2.82040693037988\\
2.61771412103994	2.75455715049177\\
2.69618483097944	2.68435767238165\\
2.76967760447452	2.60969067558258\\
2.83841381879587	2.53036035107147\\
2.90253470641246	2.44609525787348\\
2.96209802805208	2.35654912167942\\
3.01707262919853	2.26130043251898\\
3.06733057921585	2.15985126527585\\
3.11263660015073	2.05162586609647\\
3.15263452503901	1.93596973871599\\
3.18683061376612	1.8121502503604\\
3.21457374287693	1.67936018272564\\
3.23503283868411	1.53672620317594\\
3.24717252973688	1.38332493572243\\
3.24972896934425	1.21821014777914\\
3.24118924870029	1.04045544358652\\
3.21977988880917	0.849217546494955\\
3.1834725580871	0.643825333362289\\
3.13001814199517	0.423898551439897\\
3.05702282840415	0.189496614979834\\
2.96208046408279	-0.0587090367002419\\
2.84297176184823	-0.319256458036893\\
2.69793031738669	-0.589740078626076\\
2.52595624426856	-0.866727932818914\\
2.32713254218933	-1.1458000803693\\
2.10287500662487	-1.42174843885806\\
1.8560370038427	-1.68894570188325\\
1.59080879521382	-1.94184140203882\\
1.31239968144874	-2.17549463251815\\
1.02655471746895	-2.38602990610422\\
0.739008063001648	-2.57092037985953\\
0.454988272845025	-2.72905503920742\\
0.17886313749819	-2.86060861319044\\
-0.0860395711768357	-2.96677798302946\\
-0.33744884501812	-3.0494633517034\\
-0.574048963935673	-3.11096042749872\\
-0.795312543117972	-3.15370479378238\\
-1.00131324321407	-3.18008446073984\\
-1.19254879174875	-3.19231851186315\\
-1.36979058218974	-3.19239018933388\\
-1.53396526611143	-3.18201985703479\\
-1.68606707975293	-3.16266432210517\\
-1.82709638029388	-3.13553173401469\\
-1.95801885894119	-3.1016043038565\\
-2.07974014240425	-3.06166370936887\\
-2.19309129316794	-3.01631605259258\\
-2.29882165882822	-2.96601463753233\\
-2.3975963930293	-2.91107974953655\\
-2.48999669238132	-2.85171517738617\\
-2.57652134994271	-2.78802153705338\\
-2.65758863235173	-2.72000661991507\\
-2.73353777115685	-2.64759306006419\\
-2.80462954671893	-2.5706236378708\\
-2.87104555930817	-2.48886453900615\\
-2.93288584660048	-2.40200688999573\\
-2.99016453620524	-2.30966690859388\\
-3.04280323088502	-2.21138505388806\\
-3.09062182797518	-2.10662465187518\\
-3.13332649168657	-1.99477062473559\\
-3.17049455290271	-1.87512918724549\\
-3.20155624267972	-1.74692971627559\\
-3.22577342679063	-1.60933047455679\\
-3.24221597569766	-1.461430497007\\
-3.24973717941891	-1.30229072451559\\
-3.2469508233127	-1.13096834386344\\
-3.23221430086974	-0.946569114175136\\
-3.20362451552835	-0.748322915770596\\
-3.15903620107256	-0.535687288342782\\
-3.09611518507011	-0.308481479443459\\
-3.01244092568205	-0.0670484250269967\\
-2.90567144297647	0.187566820065364\\
-2.77377693185499	0.453446422817519\\
-2.61533349956532	0.727675555342042\\
-2.42984546794961	1.00631404478284\\
-2.21803827904919	1.28451198369756\\
-1.98204528315884	1.55679599914171\\
-1.72541491064481	1.81751033362336\\
-1.45289895432384	2.06134485117164\\
-1.17004172724966	2.28384340951517\\
-0.882650474864446	2.48178268065626\\
-0.596261176391486	2.65334877084781\\
-0.315705610257031	2.79809996170941\\
-0.0448430163955496	2.91676025414382\\
0.213533748640937	3.01091864183941\\
0.457649300011542	3.08270894153138\\
0.686605126289815	3.13452461130596\\
0.900197669369818	3.16879654796175\\
1.09873331760935	3.18783973202483\\
1.28286346141527	3.19376085633426\\
};
\addplot [color=mycolor2, forget plot]
  table[row sep=crcr]{%
1.89181837155654	1.47713763134783\\
2.13771553571714	1.46964939389994\\
2.33001012910751	1.45153211491801\\
2.48106192932662	1.42754820119153\\
2.60075866889671	1.40065512434169\\
2.69663740741938	1.3726112403047\\
2.77432332374972	1.34442024213581\\
2.83798709737352	1.31662450572021\\
2.89072519217482	1.28948916501309\\
2.93484951396149	1.26311474598279\\
2.97209959965753	1.23750518193163\\
3.00379547743106	1.21260852999437\\
3.03094720294237	1.18834111188663\\
3.05433340731742	1.16460159315882\\
3.07455783053367	1.14127892782587\\
3.0920901946036	1.11825652862231\\
3.10729586029809	1.0954140776154\\
3.12045735698308	1.07262781857304\\
3.13178992463875	1.04976982064082\\
3.141452537821	1.02670648217422\\
3.14955540266705	1.00329639964764\\
3.15616456487969	0.979387626220089\\
3.1613039916535	0.95481426681401\\
3.16495525780851	0.929392287372161\\
3.16705474608268	0.902914344806229\\
3.16748803510655	0.875143362056503\\
3.16608086437666	0.845804470789074\\
3.16258569349412	0.814574812976593\\
3.15666235788526	0.781070521381554\\
3.14785058585457	0.744829977260624\\
3.1355310666242	0.705292164790822\\
3.1188701800061	0.661768613519819\\
3.09674118814482	0.61340709041739\\
3.06761137190888	0.559145019828183\\
3.02938003899483	0.49765095619948\\
2.9791466974858	0.427254225025101\\
2.91288354275194	0.345868141978941\\
2.82498741508678	0.250925399102838\\
2.70770978260807	0.139373802808658\\
2.55054904516899	0.00784046109987624\\
2.33991897174694	-0.146824509493805\\
2.05988981018934	-0.326293957760779\\
1.69548212177068	-0.528509603760869\\
1.24004416179817	-0.744808228464494\\
0.70548881917923	-0.958493798790827\\
0.12783675969007	-1.14817471608847\\
-0.4416032325795	-1.29616971098341\\
-0.956740550574267	-1.39588042462719\\
-1.39228189895034	-1.45193873083072\\
-1.74441246300098	-1.47475135554755\\
-2.02232349506424	-1.47510894977225\\
-2.23973581762482	-1.46157145103386\\
-2.41001416968227	-1.44005415271611\\
-2.54430815062523	-1.41433334477799\\
-2.65128453981933	-1.38670069250552\\
-2.73746462382523	-1.35849200673027\\
-2.80769308911154	-1.33045148904568\\
-2.86556179250319	-1.30296510910786\\
-2.91374397688086	-1.27620490925257\\
-2.95424296342479	-1.25021663838917\\
-2.98857246774732	-1.22497243047476\\
-3.01788602285498	-1.20040220232462\\
-3.04306970540758	-1.17641214094034\\
-3.064808736466	-1.15289534086714\\
-3.08363552486989	-1.12973763674608\\
-3.09996447248869	-1.10682045932657\\
-3.11411724857011	-1.08402180768201\\
-3.12634110550271	-1.06121598213717\\
-3.1368220116024	-1.03827244474618\\
-3.14569381209809	-1.01505399705126\\
-3.15304422044083	-0.99141434594641\\
-3.15891813305506	-0.967195041532572\\
-3.16331851081537	-0.942221698869746\\
-3.16620484757075	-0.916299346538142\\
-3.16748902131662	-0.889206669386819\\
-3.16702806764512	-0.860688821992978\\
-3.16461309274635	-0.830448374032041\\
-3.15995310784245	-0.798133798868254\\
-3.15265195258697	-0.763324721426685\\
-3.14217558593179	-0.725512891445621\\
-3.12780572030924	-0.684077542100717\\
-3.10857386267558	-0.638253454802491\\
-3.08316704752093	-0.587089765742521\\
-3.04979262753941	-0.529397559975725\\
-3.0059843219301	-0.463685201190143\\
-2.94832597450879	-0.388083521184075\\
-2.87206625263878	-0.300271471378458\\
-2.77060661321309	-0.197432821785875\\
-2.63489222674009	-0.076317237101075\\
-2.4528799873791	0.0664401824931821\\
-2.20960200429706	0.233466482055127\\
-1.88896077904493	0.424913341965717\\
-1.47894830289929	0.635762259080663\\
-0.980951413796855	0.853286954305538\\
-0.41898401443042	1.05763974617386\\
0.161163149450842	1.22809747605534\\
0.708107111546855	1.35198310850992\\
1.18512109293991	1.42877601430675\\
1.57838917173825	1.46679056095548\\
1.89181837155654	1.47713763134783\\
};
\addplot [color=mycolor1, forget plot]
  table[row sep=crcr]{%
1.12337085347325	3.41944784996473\\
1.29798732958167	3.41396102716276\\
1.46196601803898	3.39835266926702\\
1.61598180314091	3.37375209585985\\
1.76072271522622	3.34109904968817\\
1.89686168659934	3.30115800815168\\
2.02503714076216	3.25453409588748\\
2.14584007538346	3.20168882359674\\
2.25980570258933	3.14295460284063\\
2.36740810261529	3.07854748545296\\
2.46905668700011	3.00857790509443\\
2.56509354175359	2.93305940788129\\
2.65579093071233	2.85191549030554\\
2.74134839309963	2.76498474793442\\
2.82188897843307	2.67202460225462\\
2.89745423864257	2.57271393484618\\
2.96799765449178	2.46665503396455\\
3.03337622540302	2.35337536315742\\
3.09334001561041	2.23232980883168\\
3.14751954699297	2.10290426749029\\
3.19541108883424	1.96442170598768\\
3.23636015534362	1.81615217714327\\
3.26954393191587	1.65732869446987\\
3.29395396865839	1.48717133646596\\
3.30838136349419	1.30492239406217\\
3.31140784697252	1.10989565819192\\
3.30140765839972	0.901542834665535\\
3.27656672623956	0.679539217503698\\
3.23492707925443	0.443888687781547\\
3.17446495544757	0.195044351376081\\
3.09320973934176	-0.0659645908731955\\
2.98940644267463	-0.337416511784161\\
2.86171601460507	-0.616815932541158\\
2.70943550282497	-0.900876370043321\\
2.53270626723075	-1.18559532876101\\
2.33266789204068	-1.46643785383327\\
2.1115144148106	-1.73862314181527\\
1.87242265834854	-1.99748102758807\\
1.61934908852423	-2.23882142629321\\
1.35672389053374	-2.45925068976833\\
1.08909666054117	-2.65637987051156\\
0.820797073047426	-2.82889748589837\\
0.555663663292206	-2.97651245966763\\
0.296870565317662	-3.09979918436642\\
0.0468560139137498	-3.19998884002555\\
-0.192663551417901	-3.27874919926749\\
-0.420619965366741	-3.33798418929181\\
-0.636487511273445	-3.3796707082659\\
-0.840167085322681	-3.40573820011202\\
-1.03187563778785	-3.41798833978931\\
-1.21204948593243	-3.41804793573803\\
-1.3812649278788	-3.40734686822138\\
-1.54017619937021	-3.38711338672114\\
-1.68946894170933	-3.35838042815215\\
-1.82982655532066	-3.32199816171223\\
-1.96190668838332	-3.27864937833709\\
-2.0863253406858	-3.22886548704013\\
-2.20364644186376	-3.17304173719398\\
-2.31437516862347	-3.11145089064738\\
-2.41895363492475	-3.04425497466228\\
-2.51775789681386	-2.97151501002056\\
-2.61109545485912	-2.89319877443323\\
-2.6992026176303	-2.8091867661675\\
-2.78224121963233	-2.7192766046861\\
-2.86029427845936	-2.62318616576988\\
-2.9333602412369	-2.52055581556418\\
-3.00134552312201	-2.41095019643518\\
-3.0640550961729	-2.29386014169077\\
-3.12118096441606	-2.16870547008987\\
-3.17228848545515	-2.03483964792995\\
-3.21680070377289	-1.8915576168803\\
-3.25398118948045	-1.73810847252857\\
-3.28291638252849	-1.57371512814135\\
-3.30249918624028	-1.39760356519063\\
-3.3114165882906	-1.20904465786337\\
-3.30814542905765	-1.00741168191251\\
-3.29096201643304	-0.792256187138982\\
-3.25797286635204	-0.563403515082209\\
-3.20717493015324	-0.321066389214163\\
-3.13655341623575	-0.0659702764881891\\
-3.04422258654089	0.200522420107314\\
-2.92860855717195	0.476310021275436\\
-2.7886626532819	0.758498504660572\\
-2.62408037933843	1.04342739569919\\
-2.43548813601527	1.32679813915937\\
-2.22455324917765	1.60391155894044\\
-1.99397861951217	1.86999526562393\\
-1.74736374460416	2.12057475325495\\
-1.48894469235357	2.35182434160235\\
-1.22325615775622	2.56083491987714\\
-0.954776947771264	2.74575579109426\\
-0.687619329080636	2.9057998932751\\
-0.425304744052455	3.0411326827088\\
-0.170642417405311	3.15268464004814\\
0.0742964523537774	3.24193210250864\\
0.308131369598009	3.31068389954316\\
0.530079690709056	3.36089818232796\\
0.739843429206208	3.39454053901598\\
0.937494593240998	3.41348430414743\\
1.12337085347325	3.41944784996473\\
};
\addplot [color=mycolor2, forget plot]
  table[row sep=crcr]{%
2.03026270353078	1.64575647587509\\
2.24201361397065	1.63929491819078\\
2.40988976796717	1.62346454151037\\
2.5439086404576	1.60217293817841\\
2.65192348194154	1.57789433711441\\
2.73991142031267	1.55215011587583\\
2.81236758836163	1.52585007471876\\
2.87266513834438	1.49951838258181\\
2.92334343033639	1.47343812077522\\
2.96632580056046	1.44774221252277\\
3.00307980332721	1.42247015209458\\
3.03473381366942	1.39760320969316\\
3.06216162049611	1.37308613323043\\
3.08604387326376	1.34884035107809\\
3.1069128492206	1.32477177949003\\
3.12518516010315	1.30077515426889\\
3.14118566099113	1.27673606856081\\
3.15516484960864	1.25253143538509\\
3.16731134706794	1.22802879750042\\
3.17776054716328	1.2030847121369\\
3.1866001477792	1.17754230256394\\
3.19387298775986	1.15122796591422\\
3.19957736951745	1.12394713932686\\
3.2036648216851	1.09547894128019\\
3.20603501902482	1.06556941121\\
3.2065272979622	1.03392295803928\\
3.20490784773303	1.00019148658358\\
3.20085116825551	0.963960488751789\\
3.19391369462968	0.924731153178388\\
3.18349649194873	0.88189725582094\\
3.16879247861024	0.834715253474649\\
3.14871154984905	0.782265661046414\\
3.12177402504919	0.723403600363448\\
3.08595887985921	0.65669674236961\\
3.03848848218452	0.580350616223418\\
2.97552755034422	0.492126403564453\\
2.89177588676445	0.389269007027596\\
2.77995604944656	0.268491300815408\\
2.63027155695706	0.126116785626424\\
2.43010358708565	-0.0414206862591693\\
2.16460932990471	-0.236405649579915\\
1.81944441184434	-0.457695773160881\\
1.38693674034512	-0.697839094321064\\
0.875038323293743	-0.941158665858655\\
0.313368717274706	-1.1659295814316\\
-0.252135552917663	-1.35184958146322\\
-0.776126199887151	-1.48819301966754\\
-1.22963821375558	-1.57605995670056\\
-1.60388412349005	-1.62425867750996\\
-1.9042365387354	-1.64371748521601\\
-2.1423340163231	-1.64401278743609\\
-2.33075772201044	-1.63226643854284\\
-2.48059210593731	-1.61331948699554\\
-2.60075362405107	-1.59029424094336\\
-2.69810967565792	-1.56513735762139\\
-2.77784760291292	-1.53902978813207\\
-2.84386080707416	-1.51266611354359\\
-2.89907416816592	-1.48643579068407\\
-2.94569574923572	-1.46053795843337\\
-2.98540399805985	-1.43505331283764\\
-3.01948454392525	-1.40998882547804\\
-3.04892951832121	-1.38530541457036\\
-3.07450962932375	-1.36093491127263\\
-3.09682658838307	-1.33679026387485\\
-3.11635136498169	-1.31277142156121\\
-3.13345215510356	-1.28876840528795\\
-3.14841479826139	-1.2646624896102\\
-3.16145755384025	-1.24032605003473\\
-3.17274155601308	-1.21562139138008\\
-3.18237783452297	-1.19039871137346\\
-3.19043146182922	-1.16449323731631\\
-3.19692312450592	-1.13772148055284\\
-3.20182818580579	-1.1098764686405\\
-3.20507307834332	-1.08072172691031\\
-3.20652861203835	-1.04998367925386\\
-3.20599946923685	-1.01734201226768\\
-3.20320874276366	-0.98241738645672\\
-3.19777579327309	-0.944755671769844\\
-3.18918487373995	-0.903807622805553\\
-3.17674076907669	-0.858902590472078\\
-3.15950595960833	-0.809214515661926\\
-3.13621132700539	-0.753718154746509\\
-3.10512897253988	-0.691133486978053\\
-3.06389128737052	-0.619857135842971\\
-3.00923570827716	-0.537882735252563\\
-2.9366524931571	-0.442720368567719\\
-2.83992194445915	-0.331344308054474\\
-2.71056965562503	-0.200238660869918\\
-2.53739135908002	-0.0456861773612729\\
-2.30648368509363	0.135441053243499\\
-2.00272025280522	0.344046356187461\\
-1.61408470738881	0.576195424583096\\
-1.13957746635919	0.820384013776223\\
-0.597773254972512	1.05727631209485\\
-0.0280177497212467	1.26470629064575\\
0.521625471572434	1.42639948348009\\
1.01267121040275	1.53774497708029\\
1.42656549795925	1.60443013244292\\
1.76263786922918	1.63692832306647\\
2.03026270353078	1.64575647587509\\
};
\addplot [color=mycolor1, forget plot]
  table[row sep=crcr]{%
1.01343528097655	3.63585201902066\\
1.19185337592755	3.63023677140222\\
1.36099770341136	3.61412819344363\\
1.52136225084341	3.58850545850642\\
1.67346151505693	3.55418471898578\\
1.81780658280804	3.51182884113409\\
1.95488752504785	3.4619585922676\\
2.08516054724753	3.40496396299583\\
2.20903855445196	3.34111480712293\\
2.32688401468736	3.27057034337028\\
2.43900320906891	3.19338731797567\\
2.54564112956951	3.10952680633696\\
2.64697642417426	3.0188597620073\\
2.74311589813743	2.92117152487511\\
2.83408816681036	2.81616559580879\\
2.9198361300294	2.70346708806976\\
3.00020801283658	2.58262638999081\\
3.07494680802332	2.45312373062445\\
3.14367808259371	2.31437553985711\\
3.20589629815401	2.16574374228541\\
3.26095007601892	2.00654941740021\\
3.30802724837493	1.83609258012509\\
3.3461411152807	1.6536801434531\\
3.37412010291755	1.45866433791852\\
3.39060399223019	1.25049384499662\\
3.39405099878186	1.02877944917842\\
3.38276106660261	0.793374855924018\\
3.35492146274524	0.544471159869154\\
3.30868060031537	0.28270004906654\\
3.24225428319321	0.00923621972990475\\
3.15406455666848	-0.274115814033056\\
3.04290471209723	-0.564870011927675\\
2.90811523997044	-0.859863697273204\\
2.74974649930657	-1.15533745256374\\
2.56867777004751	-1.44709857665007\\
2.36666285770699	-1.73076053886735\\
2.14628189544622	-2.00203209688494\\
1.91079657288267	-2.25701450242647\\
1.66392695151825	-2.49245974972126\\
1.40958524610206	-2.70595014654958\\
1.15160943172697	-2.89597743088331\\
0.893535292189428	-3.06192175638923\\
0.638432314978981	-3.20394967734436\\
0.388812268759267	-3.32286080230751\\
0.146604852015977	-3.41991411899592\\
-0.0868143097595534	-3.49665946835587\\
-0.310561319431775	-3.55479092968387\\
-0.524164133255402	-3.59603017651728\\
-0.727478342126641	-3.62204106531461\\
-0.920607476176883	-3.63437241270124\\
-1.10383281154742	-3.63442383470343\\
-1.2775546979489	-3.62342903307452\\
-1.44224545906707	-3.60245136822155\\
-1.59841278737773	-3.57238744537591\\
-1.74657203121279	-3.53397542842141\\
-1.8872256387729	-3.48780570303788\\
-2.02084811230786	-3.43433226123445\\
-2.14787501917248	-3.3738837591114\\
-2.26869483213253	-3.30667362660043\\
-2.38364258848127	-3.23280891210932\\
-2.49299454691819	-3.15229775837368\\
-2.59696317661036	-3.06505555737761\\
-2.69569193597136	-2.97090994650247\\
-2.7892493954591	-2.86960490543576\\
-2.87762233793308	-2.76080431079362\\
-2.96070754302555	-2.6440954174288\\
-3.03830204253425	-2.51899287473264\\
-3.11009173969587	-2.3849440637389\\
-3.17563843890095	-2.24133676433477\\
-3.23436556243217	-2.08751043345009\\
-3.28554317163611	-1.92277268588853\\
-3.32827339989067	-1.74642289138102\\
-3.36147807928184	-1.55778507495665\\
-3.38389122035015	-1.35625242644068\\
-3.3940600599788	-1.14134551988465\\
-3.39035951910219	-0.912785575902793\\
-3.37102586779912	-0.670582476113253\\
-3.33421575579759	-0.41513447650025\\
-3.27809591248885	-0.147332533287431\\
-3.20096602514468	0.131342905161217\\
-3.10141198482739	0.418749985897776\\
-2.97847885558869	0.712059402219527\\
-2.83184370894702	1.00779493438612\\
-2.6619604392095	1.30195494354937\\
-2.4701454483235	1.5902159480539\\
-2.25857794652784	1.86820131746095\\
-2.03020250578075	2.13178026081226\\
-1.78854156401486	2.37735127744792\\
-1.53744555744897	2.60206509794101\\
-1.28082125932745	2.8039553740791\\
-1.02238045525752	2.98196635995617\\
-0.765441767854556	3.13588812240772\\
-0.512802792854487	3.26622486838006\\
-0.266683626030244	3.37402778513509\\
-0.0287308580328474	3.46072126793759\\
0.199934997210314	3.52794382790228\\
0.418646483445373	3.57741593688789\\
0.627105777240416	3.61083916929073\\
0.825301884117939	3.62982544115539\\
1.01343528097654	3.63585201902066\\
};
\addplot [color=mycolor2, forget plot]
  table[row sep=crcr]{%
2.14516078266191	1.81183437647932\\
2.32810767202528	1.80624082044245\\
2.47498896724813	1.79237950465215\\
2.59392767073726	1.77347427500045\\
2.69121028702518	1.7516000183635\\
2.77162026450685	1.72806638687415\\
2.83877780008583	1.7036840880486\\
2.89542578857523	1.67894145181767\\
2.9436522231263	1.65411894328654\\
2.98505693235282	1.62936267187349\\
3.02087449597627	1.60473135783548\\
3.05206427187012	1.5802262386663\\
3.0793762728029	1.55580999860794\\
3.10339945256859	1.53141858573327\\
3.12459717608243	1.50696836088577\\
3.14333329276745	1.48236011877948\\
3.15989123768703	1.45748094377396\\
3.1744878629247	1.43220448859898\\
3.18728317556303	1.40639001567081\\
3.19838676732086	1.37988036802691\\
3.20786141780366	1.35249890818988\\
3.21572410207699	1.32404535679468\\
3.22194440374892	1.2942903628817\\
3.22644009903405	1.26296853170257\\
3.22906940570025	1.22976951211391\\
3.22961904749807	1.19432659237647\\
3.22778682211214	1.15620205798437\\
3.22315671266391	1.11486831569946\\
3.21516365575605	1.06968347569427\\
3.2030437427563	1.01985971386072\\
3.18576371422725	0.964422352420934\\
3.16192091548773	0.90215733690581\\
3.12960128344385	0.831545008064674\\
3.08617864417194	0.750679609819703\\
3.0280349472822	0.657178707510012\\
2.95018243305724	0.5480985257777\\
2.84578696967122	0.419897682430965\\
2.7056547387299	0.268545246798006\\
2.51790683589687	0.0899623815006545\\
2.26840203226445	-0.118893134541208\\
1.94296185496446	-0.357964425295355\\
1.53264287484255	-0.621143106950835\\
1.04185108331257	-0.893831953901765\\
0.494924609752802	-1.15403342101582\\
-0.066534518525441	-1.37894991663137\\
-0.597772602234644	-1.55378096771476\\
-1.06674192827408	-1.67591294678178\\
-1.46035076606654	-1.75222080015195\\
-1.78055759538858	-1.79347207767742\\
-2.03707648936046	-1.8100868393982\\
-2.24172853257791	-1.81033056486947\\
-2.40550246594102	-1.80010983010687\\
-2.53751880948657	-1.78340589518095\\
-2.64494597426942	-1.76281211984989\\
-2.73327449720316	-1.73998069484092\\
-2.80666650064496	-1.71594483032295\\
-2.8682718360826	-1.69133643129177\\
-2.92048189500717	-1.66652855324256\\
-2.96512276341988	-1.6417272257583\\
-3.00359856873776	-1.61703023471763\\
-3.03699671348769	-1.59246461859214\\
-3.0661648784366	-1.56801048741935\\
-3.09176740614943	-1.54361601670386\\
-3.11432667645284	-1.51920669087131\\
-3.13425352034404	-1.49469073780299\\
-3.15186955473061	-1.4699619740346\\
-3.16742347255512	-1.44490081626718\\
-3.18110270743932	-1.4193739110612\\
-3.19304143990479	-1.39323262886547\\
-3.20332557011713	-1.36631052109911\\
-3.21199500898494	-1.33841972365926\\
-3.21904340259761	-1.30934618874109\\
-3.22441517586908	-1.27884352534193\\
-3.22799953159585	-1.24662511534656\\
-3.22962073876998	-1.21235403508853\\
-3.22902364797926	-1.17563013964662\\
-3.2258528262045	-1.1359734462553\\
-3.21962292958437	-1.09280267290438\\
-3.20967681992079	-1.04540744492495\\
-3.1951263279035	-0.992912296007294\\
-3.17476828543654	-0.934230241279021\\
-3.14696530965509	-0.868003614028773\\
-3.10947680774259	-0.792530587011713\\
-3.05922140853847	-0.705678606652429\\
-2.99195001476793	-0.604793579562332\\
-2.90181625129863	-0.486631565879331\\
-2.78086674353476	-0.347377878770904\\
-2.61857755187232	-0.182890815216818\\
-2.40180375939587	0.0105804915569155\\
-2.11594162626837	0.234854610604258\\
-1.74855246596671	0.487240711876294\\
-1.29628392023436	0.757551460136748\\
-0.773087447884022	1.02700904971299\\
-0.213075259222931	1.27210066912647\\
0.338367057257413	1.47307141584576\\
0.841292482717871	1.62116207087161\\
1.27311059132485	1.71915089955838\\
1.62912389573152	1.77653729265007\\
1.91604217687101	1.80428448539298\\
2.14516078266191	1.81183437647932\\
};
\addplot [color=mycolor1, forget plot]
  table[row sep=crcr]{%
0.940529402382294	3.81671254728581\\
1.12221882568828	3.81098838439057\\
1.29553828688765	3.79447642475729\\
1.46088553715868	3.76805200798769\\
1.61867883079513	3.73244107004308\\
1.76933593247377	3.68822784529482\\
1.91325803670642	3.63586375726737\\
2.05081740222796	3.57567646725338\\
2.18234765030009	3.50787843672007\\
2.30813583155767	3.43257464778903\\
2.42841551097784	3.34976933939613\\
2.5433602463567	3.25937177792064\\
2.65307694136993	3.16120121065812\\
2.75759864377023	3.05499126627416\\
2.85687643993119	2.94039418355026\\
2.95077017923965	2.81698538133578\\
3.03903785950927	2.68426903942937\\
3.12132363527079	2.54168554995017\\
3.19714459664589	2.38862192459141\\
3.26587673431111	2.22442649953105\\
3.32674088588706	2.04842954703031\\
3.37878998011524	1.85997163851086\\
3.4208995765126	1.65844173226941\\
3.45176453222614	1.44332685920611\\
3.46990555407134	1.21427477949425\\
3.4736902636639	0.971169864968868\\
3.46137394331136	0.714220501283853\\
3.43116492062658	0.444053352368563\\
3.38131805752097	0.161805959730513\\
3.31025651689468	-0.130795159074461\\
3.21671664761649	-0.43138744706829\\
3.09990385796391	-0.736974296429673\\
2.95964010967504	-1.0439915601454\\
2.79647850519018	-1.3484462432788\\
2.61176009205239	-1.64612228743937\\
2.40759437810898	-1.93283329431207\\
2.18675786731902	-2.20468916462333\\
1.95252113303128	-2.45833775129871\\
1.70842956190959	-2.69114648965576\\
1.45807115625702	-2.90130167934605\\
1.20486434177597	-3.08782037619996\\
0.951890631527259	-3.25048609763232\\
0.701784809477116	-3.38973034605408\\
0.456683192328822	-3.50648564063056\\
0.218221550805198	-3.6020333187657\\
-0.0124304054884634	-3.67786331299615\\
-0.234511883913556	-3.73555603985411\\
-0.447603321986191	-3.77669032071695\\
-0.65155727176105	-3.8027767759739\\
-0.846433809638309	-3.81521349508384\\
-1.03244372230256	-3.81525964852368\\
-1.20990070786174	-3.80402259260859\\
-1.37918250494462	-3.78245448218111\\
-1.54070009742496	-3.75135511975611\\
-1.69487378229385	-3.71137852399006\\
-1.84211479056816	-3.66304138352832\\
-1.98281120528697	-3.6067321287988\\
-2.11731704942275	-3.54271979797231\\
-2.24594357092373	-3.47116220805626\\
-2.36895190407261	-3.392113189972\\
-2.48654642223561	-3.30552883098262\\
-2.59886821259585	-3.21127281082805\\
-2.70598820032301	-3.10912103854377\\
-2.80789953358164	-2.99876591164773\\
-2.90450892075347	-2.87982064227445\\
-2.99562669928613	-2.75182423785062\\
-3.08095552728283	-2.61424789649424\\
-3.16007774414687	-2.46650378520177\\
-3.23244166981062	-2.30795741139842\\
-3.29734743204461	-2.13794506389383\\
-3.35393335846368	-1.95579805791037\\
-3.40116456981079	-1.7608757129587\\
-3.43782617230095	-1.55260902490024\\
-3.4625243391138	-1.33055671819476\\
-3.47369949481505	-1.09447458398215\\
-3.4696565645729	-0.844397496185794\\
-3.4486174762433	-0.580731050508035\\
-3.40880032255423	-0.304346330489577\\
-3.3485272533785	-0.01666712280979\\
-3.26635885456223	0.280265268716509\\
-3.16124651288204	0.583767278568167\\
-3.03268692450557	0.890545754527654\\
-2.88085636606772	1.1968003018717\\
-2.70669926960498	1.49839709064195\\
-2.511948523168	1.79110154521217\\
-2.29906472086288	2.07084273240482\\
-2.0710966388664	2.33397236601865\\
-1.83148129387468	2.5774802038505\\
-1.58381386251731	2.79913625554401\\
-1.33162172825651	2.99754589709511\\
-1.07817232772824	3.17212143192531\\
-0.826333794313452	3.32298755848107\\
-0.578494768481339	3.45084547052904\\
-0.336538935920287	3.55682069245203\\
-0.101862995942506	3.6423151625029\\
0.124575853936526	3.7088772221073\\
0.342196161687663	3.75809635556959\\
0.550722883442389	3.79152412281218\\
0.750119970262207	3.81061919190203\\
0.940529402382293	3.81671254728581\\
};
\addplot [color=mycolor2, forget plot]
  table[row sep=crcr]{%
2.23886372154498	1.97054596626372\\
2.3971405473277	1.96569800633572\\
2.52565451777639	1.95356181822508\\
2.63101472075458	1.9368076774117\\
2.71829225436106	1.9171769431276\\
2.79134646283389	1.89579091537222\\
2.85311275745035	1.87336152224805\\
2.90583199922238	1.85033104357419\\
2.95122497346438	1.82696357625248\\
2.99062220768506	1.80340459198751\\
3.02505965735058	1.77971963312046\\
3.05534904636493	1.75591938705399\\
3.08212961471371	1.7319758194069\\
3.10590625968392	1.70783237231697\\
3.12707767721752	1.68341015198366\\
3.14595708030931	1.65861132931121\\
3.16278731704588	1.63332051994798\\
3.17775165866286	1.60740460463824\\
3.19098111743425	1.58071123920809\\
3.20255883808743	1.55306614754639\\
3.21252184778417	1.52426916483789\\
3.22086021804516	1.49408888349644\\
3.22751345957729	1.46225563596542\\
3.23236370895152	1.42845241350026\\
3.23522494030351	1.39230315482345\\
3.23582700102603	1.353357628782\\
3.23379266602203	1.31107186583288\\
3.22860504328548	1.26478275150053\\
3.21956142251217	1.21367497880448\\
3.20570787578924	1.15673809634543\\
3.18574640394614	1.09271099770805\\
3.15790302204014	1.02001118671122\\
3.11974100782698	0.93664727675023\\
3.06789961820395	0.840117152408607\\
2.99773845519081	0.727304690745195\\
2.90288113640783	0.594412126820719\\
2.77470249200837	0.437015500958592\\
2.60193895795226	0.250422457776838\\
2.37089293593623	0.0306421785226072\\
2.0671652834548	-0.22364673721209\\
1.68015815458778	-0.50804165113269\\
1.21061935927683	-0.809365687088379\\
0.677943225041865	-1.10554365855548\\
0.119994514024516	-1.37122039316945\\
-0.418603686536259	-1.58717161051174\\
-0.902562091961035	-1.74656914995138\\
-1.31454939143297	-1.85392677460347\\
-1.65327120545997	-1.91961872031168\\
-1.92668956910225	-1.95484601531965\\
-2.14600505565856	-1.96904561234524\\
-2.32220434948555	-1.96924705464741\\
-2.46466491470551	-1.96034786893389\\
-2.58087975605076	-1.94563555192347\\
-2.67664785983346	-1.92727015027268\\
-2.75639513382362	-1.9066511722555\\
-2.82348655693236	-1.88467393820216\\
-2.88048523384333	-1.86190155832319\\
-2.92935322944049	-1.83867809557897\\
-2.97160233631346	-1.81520240074319\\
-3.00840560076461	-1.79157612972125\\
-3.04067936767815	-1.76783490847551\\
-3.06914359976889	-1.74396847328966\\
-3.09436629393814	-1.71993353935428\\
-3.11679624346869	-1.6956618039765\\
-3.1367871984734	-1.67106462072325\\
-3.15461559476158	-1.64603531582131\\
-3.17049337645814	-1.62044974563793\\
-3.18457696299681	-1.59416544138832\\
-3.19697305294373	-1.56701950728727\\
-3.2077416738476	-1.53882530000431\\
-3.2168966457057	-1.50936779884364\\
-3.22440339703895	-1.47839746123413\\
-3.23017382918471	-1.44562223308809\\
-3.23405763531408	-1.4106972351741\\
-3.23582910716819	-1.37321146101976\\
-3.23516795230254	-1.33267058417853\\
-3.2316319243834	-1.28847466843858\\
-3.22461803537015	-1.23988919461343\\
-3.21330762930232	-1.18600737204386\\
-3.1965884725218	-1.12570125453857\\
-3.17294406804385	-1.05755892086571\\
-3.14029656852784	-0.979805398423894\\
-3.09578538824318	-0.890207250560821\\
-3.03546088218197	-0.785967333430171\\
-2.95387728181653	-0.663632354103263\\
-2.84359689196043	-0.5190712343952\\
-2.6947027328728	-0.347651706824467\\
-2.49462112019136	-0.144857159474346\\
-2.22894202869952	0.092288706523193\\
-1.88439266240653	0.362673847196198\\
-1.45498408819297	0.657793680364402\\
-0.950154585058804	0.959712068055632\\
-0.399297508665178	1.24364687671789\\
0.154306540059556	1.48615161318593\\
0.668936484537602	1.67386686455298\\
1.1179353871564	1.80617203399682\\
1.49270534993371	1.89125774695739\\
1.79746627586354	1.9403949720658\\
2.042394554795	1.96407939077264\\
2.23886372154498	1.97054596626372\\
};
\addplot [color=mycolor1, forget plot]
  table[row sep=crcr]{%
0.896441690959812	3.9441923617009\\
1.08051075693992	3.93838963927558\\
1.2567522619063	3.92159579500097\\
1.42551554186611	3.89462204154981\\
1.58716888950602	3.85813663492713\\
1.74208022106361	3.81267167666549\\
1.89060191874816	3.758630911541\\
2.03305881683847	3.69629765064671\\
2.16973841785594	3.62584228168578\\
2.30088255142231	3.54732908093015\\
2.42667980715507	3.46072223444294\\
2.54725817812176	3.36589112949966\\
2.66267744312509	3.26261510845942\\
2.77292089898392	3.15058800222867\\
2.87788613609048	3.02942289234418\\
2.97737464328587	2.8986577007422\\
3.07108014656033	2.75776238280855\\
3.15857574937347	2.60614870633934\\
3.23930017381126	2.44318383344733\\
3.31254372793733	2.26820916924529\\
3.3774350723429	2.08056616661775\\
3.43293044815691	1.87963091917116\\
3.47780776033545	1.66485933575901\\
3.51066874471785	1.4358443267926\\
3.52995327764365	1.19238556079154\\
3.53397050503074	0.934570769130519\\
3.520951542599	0.662865133062121\\
3.48912758734578	0.378201980646154\\
3.43683490925928	0.0820641606591533\\
3.36264404373427	-0.22345818841623\\
3.26550470313232	-0.53564992420845\\
3.14489133868984	-0.851208802312372\\
3.00092865089166	-1.16634927276106\\
2.83447398519056	-1.47697249275014\\
2.64713652401332	-1.77888975621129\\
2.4412221560901	-2.06807363835451\\
2.21960628443296	-2.34090274714581\\
1.98555095319482	-2.59436541308887\\
1.7424931426409	-2.82619562441075\\
1.49383473629253	-3.03492853999866\\
1.24276094183342	-3.21987843739294\\
0.99210489703689	-3.38105444633402\\
0.744265247213911	-3.51903616312019\\
0.501173856815033	-3.63483195176959\\
0.264304426877602	-3.72973898211346\\
0.0347100233669113	-3.80521812041878\\
-0.186922315934679	-3.86279069643553\\
-0.400209775873058	-3.90395918544841\\
-0.605011972928268	-3.93015044620767\\
-0.801373771193119	-3.9426782744144\\
-0.989475129968325	-3.94272132193596\\
-1.16958882528395	-3.93131248815339\\
-1.34204585640571	-3.90933636257011\\
-1.50720775825661	-3.87753193646439\\
-1.66544476555644	-3.83649845119894\\
-1.81711870285525	-3.78670283131044\\
-1.96256952083242	-3.72848763083526\\
-2.10210450369508	-3.66207879984237\\
-2.23598929741258	-3.58759286802539\\
-2.36444003225977	-3.50504336254576\\
-2.48761592541276	-3.41434644820779\\
-2.60561184747482	-3.31532591828493\\
-2.71845042333053	-3.20771779060812\\
-2.82607331901439	-3.09117489037264\\
-2.92833145207355	-2.96527194084203\\
-3.02497396639078	-2.82951184559114\\
-3.11563595065332	-2.68333403757056\\
-3.19982507385023	-2.52612599199289\\
-3.2769075865186	-2.35723924349641\\
-3.34609452020288	-2.17601148989445\\
-3.40642943436212	-1.98179655873834\\
-3.45677972219486	-1.77400407998747\\
-3.49583427838497	-1.55215052784587\\
-3.52211118323165	-1.31592270230808\\
-3.53397981522779	-1.06525351898234\\
-3.52970219987871	-0.800407976357051\\
-3.50749804443063	-0.522074279067044\\
-3.46563632196283	-0.231451448007957\\
-3.4025530281778	0.0696791348611318\\
-3.31698970315709	0.378913893937647\\
-3.20814095193239	0.693235425230204\\
-3.07579282076281	1.00908258119744\\
-2.92042958963605	1.32248811271824\\
-2.74328665053001	1.62927785135783\\
-2.54633316001776	1.9253118886953\\
-2.33217967750365	2.20673703088098\\
-2.10392029761606	2.4702150397221\\
-1.86493159586839	2.71309499905096\\
-1.61865801975323	2.93350962852924\\
-1.36841321339716	3.13039076137595\\
-1.1172199965142	3.30341369522077\\
-0.867701270448185	3.45288993022916\\
-0.622023503989817	3.5796314323213\\
-0.381886304879468	3.68480776815644\\
-0.148547037878136	3.76981232605801\\
0.0771317224625102	3.83614761341639\\
0.294622911237203	3.88533397713145\\
0.503672444507914	3.91884188328714\\
0.704239306475423	3.93804528486142\\
0.896441690959811	3.9441923617009\\
};
\addplot [color=mycolor2, forget plot]
  table[row sep=crcr]{%
2.31219753178786	2.11733153574279\\
2.44922341097804	2.11312775477941\\
2.56158996350191	2.10251021132567\\
2.65470000455284	2.08769859129554\\
2.73267510535881	2.07015547350546\\
2.79865320194544	2.05083686318459\\
2.85503036596741	2.0303609255361\\
2.90364646842595	2.00911982122122\\
2.94592401264342	1.98735336420384\\
2.98297073132288	1.96519745049244\\
3.01565511133232	1.94271587522494\\
3.04466200341872	1.9199211706665\\
3.07053364776049	1.89678811750632\\
3.09369999080958	1.8732622887459\\
3.11450107276432	1.84926514335361\\
3.13320345912556	1.8246966344936\\
3.15001210013918	1.79943592837321\\
3.16507856566243	1.77334057550395\\
3.17850627055307	1.74624429114838\\
3.19035303839031	1.71795335539022\\
3.20063111778286	1.68824151382186\\
3.20930453781797	1.65684313019624\\
3.21628343921874	1.62344419736214\\
3.22141471297521	1.58767063712821\\
3.22446787669025	1.54907309668509\\
3.22511456290194	1.50710716029286\\
3.22289920129302	1.4611075212188\\
3.21719733181251	1.41025418735661\\
3.20715632890983	1.35352823453838\\
3.19161094778317	1.28965404943067\\
3.16896282325786	1.21702465758599\\
3.13700882893513	1.13360723270606\\
3.09269866370287	1.03682870343218\\
3.03179975076724	0.923449784702938\\
2.94845514644759	0.789456527236652\\
2.83465694514752	0.630044135083091\\
2.67976544666188	0.439856708125415\\
2.470453671707	0.213788759349444\\
2.19190030976773	-0.0512092357059949\\
1.83149757056718	-0.353020390318954\\
1.38586064895721	-0.680636258290168\\
0.868979301433518	-1.01254253787914\\
0.314775059181264	-1.32092149657512\\
-0.231992170738203	-1.58148225610957\\
-0.732142184200613	-1.78216604951326\\
-1.16346924919793	-1.92431373598648\\
-1.52110294016983	-2.01754639030242\\
-1.81123723337378	-2.07382755476637\\
-2.04459157333144	-2.10389266620177\\
-2.23231835496362	-2.11604172030406\\
-2.38418957407001	-2.11620865195714\\
-2.50811759304301	-2.10846060918443\\
-2.61026815515519	-2.09552290855396\\
-2.69536337117685	-2.07919915628878\\
-2.76699906822492	-2.06067308728935\\
-2.82791570154822	-2.04071476579902\\
-2.88021121612825	-2.01981812605767\\
-2.92550194722388	-1.99829177037317\\
-2.96504207156239	-1.97631869958058\\
-2.99981164768789	-1.95399557306985\\
-3.03058141070106	-1.9313584754723\\
-3.05796052354452	-1.90839972850473\\
-3.08243183998244	-1.88507868464425\\
-3.10437796560589	-1.86132839610284\\
-3.12410046178291	-1.83705937156592\\
-3.14183384840193	-1.81216118317854\\
-3.15775555529696	-1.786502381649\\
-3.17199259332608	-1.75992896227229\\
-3.18462542050323	-1.73226146215457\\
-3.19568923178027	-1.70329063338424\\
-3.20517267385935	-1.67277150931253\\
-3.21301375125914	-1.64041554563228\\
-3.21909241698553	-1.6058803595666\\
-3.22321899405013	-1.5687563933435\\
-3.22511710383538	-1.52854957457582\\
-3.22439911527329	-1.48465871706912\\
-3.22053117725034	-1.43634598373198\\
-3.21278351860096	-1.38269821389076\\
-3.20015971378189	-1.32257633521977\\
-3.18129580814329	-1.25454957478905\\
-3.15431642558814	-1.17681114515829\\
-3.11663044452128	-1.08707347508828\\
-3.06464488273198	-0.982446092645308\\
-2.99337682270279	-0.859312703859403\\
-2.89596188345167	-0.713255221179613\\
-2.76312292999702	-0.539137347331856\\
-2.5828311924434	-0.331576630228325\\
-2.34073848640261	-0.0861895684435461\\
-2.02245974210641	0.197953775757708\\
-1.61894476666473	0.514714918158474\\
-1.1346272030851	0.847742484176509\\
-0.593813487202299	1.17140343647957\\
-0.037695774543649	1.45827588621514\\
0.489734952359442	1.68949633550006\\
0.95705966773936	1.86007193292024\\
1.35128361218501	1.97629591348342\\
1.67398237851932	2.04958304062953\\
1.93429493771153	2.0915584657179\\
2.14350538852695	2.1117855131171\\
2.31219753178786	2.11733153574279\\
};
\addplot [color=mycolor1, forget plot]
  table[row sep=crcr]{%
0.8754575855778	4.0104177677322\\
1.06078896386434	4.0045735522546\\
1.23854922283576	3.98763331937388\\
1.40906731527161	3.96037745627087\\
1.57269006917172	3.92344595010289\\
1.72976359699493	3.87734483702673\\
1.88061855825207	3.82245355026012\\
2.02555831002059	3.75903236969278\\
2.16484908783362	3.68722948504239\\
2.29871147429479	3.60708742245828\\
2.42731252133389	3.51854876960353\\
2.55075798927877	3.42146128496499\\
2.66908425269244	3.31558261015845\\
2.78224950425876	3.20058493399301\\
2.89012397248969	3.07606009659427\\
2.99247896873398	2.94152578111821\\
3.08897470971137	2.79643362639668\\
3.17914704345333	2.64018030796597\\
3.26239346283231	2.47212287057045\\
3.33795914786553	2.2915998319141\\
3.40492425999905	2.0979597736456\\
3.46219433283419	1.89059921987429\\
3.50849635386469	1.66901146600368\\
3.54238395366623	1.43284750920726\\
3.56225588019801	1.18198916153694\\
3.56639239984929	0.916632614519469\\
3.55301407953746	0.637378062993985\\
3.52036612351583	0.34531757374038\\
3.46682864144034	0.042109632465084\\
3.39104871394099	-0.269974385645804\\
3.29208421282323	-0.588047866480912\\
3.16954310200086	-0.908664677082647\\
3.02369724810034	-1.22794039218397\\
2.85554881414298	-1.5417357521964\\
2.66683175586405	-1.84588617090004\\
2.45994085201191	-2.13644959609351\\
2.23779398117415	-2.40993886934328\\
2.00364623041742	-2.66350649919343\\
1.76088292059494	-2.8950592542108\\
1.51282035604323	-3.10329417726252\\
1.26253818584371	-3.28766205539108\\
1.0127580451136	-3.44827510923211\\
0.765772890696933	-3.58578070169582\\
0.523422924806344	-3.70122243120848\\
0.28710867212055	-3.79590578646691\\
0.0578297533518808	-3.87127974547553\\
-0.16376158403674	-3.92884006165661\\
-0.377301125842935	-3.97005551774127\\
-0.582653973083151	-3.99631547460025\\
-0.779860478794216	-4.00889546709449\\
-0.969088921892505	-4.00893706080468\\
-1.15059559138525	-3.99743830836119\\
-1.32469204290745	-3.97525162052005\\
-1.49171877168303	-3.94308647587504\\
-1.65202430719974	-3.90151500087944\\
-1.80594867567897	-3.85097899069579\\
-1.95381022095887	-3.79179738648444\\
-2.09589487016749	-3.72417357645836\\
-2.23244704428052	-3.64820215943935\\
-2.36366152645962	-3.56387501880103\\
-2.48967570419104	-3.47108672039994\\
-2.61056169286983	-3.36963938799942\\
-2.72631793166803	-3.25924733949355\\
-2.83685992415061	-3.13954190057296\\
-2.94200988646712	-3.0100769607635\\
-3.04148517894587	-2.87033600869235\\
-3.13488555037737	-2.71974158361321\\
-3.22167944008226	-2.5576683066545\\
-3.30118988622717	-2.38346089509554\\
-3.37258100569672	-2.19645878626191\\
-3.43484656180295	-1.99602914905429\\
-3.48680282540288	-1.78161005091959\\
-3.52708873210587	-1.5527652451021\\
-3.5541771522628	-1.3092512763151\\
-3.56640173996894	-1.05109618213473\\
-3.56200401681776	-0.778686835306311\\
-3.53920466986621	-0.492858901596589\\
-3.49630104919356	-0.194979719730555\\
-3.43178919645336	0.112989202526469\\
-3.3445034434412	0.428465808248053\\
-3.2337603623416	0.748272731161792\\
-3.09948811511522	1.06872542460145\\
-2.94231915600336	1.38578570460051\\
-2.76362585630468	1.69527105654625\\
-2.56548592378343	1.99309733314492\\
-2.35057647791853	2.27552318459625\\
-2.12200924627566	2.53936219982267\\
-1.88313046792046	2.78213458674895\\
-1.63731435100178	3.00214257596618\\
-1.38777715533139	3.19846861629568\\
-1.13743151443992	3.37090840357856\\
-0.888790469792121	3.51985876913608\\
-0.643921028663922	3.64618260939846\\
-0.404440041536871	3.75107046116877\\
-0.171541574434479	3.83591307758739\\
0.0539556738293125	3.90219346612614\\
0.271550971989046	3.95140171852103\\
0.481001881327635	3.98497224848479\\
0.682267495359916	4.00424082343988\\
0.8754575855778	4.0104177677322\\
};
\addplot [color=mycolor2, forget plot]
  table[row sep=crcr]{%
2.36607513206161	2.24876922711939\\
2.48479706281508	2.2451218328169\\
2.58299896795016	2.23583800156676\\
2.66512219853627	2.22276998478633\\
2.73454184645255	2.2071481048122\\
2.79382894849366	2.18978550625748\\
2.84495288427685	2.17121476259944\\
2.88943286094029	2.15177833760319\\
2.92844974236369	2.13168839648718\\
2.96292821583002	2.11106637597001\\
2.99359717000535	2.08996915444847\\
3.0210341646257	2.06840627620922\\
3.04569826787349	2.04635111826338\\
3.06795432697702	2.02374786766238\\
3.088090849694	2.00051550832921\\
3.10633302831232	1.97654957172304\\
3.12285196427214	1.9517221016921\\
3.13777079597541	1.92588006614125\\
3.15116815280601	1.89884228016008\\
3.16307912154825	1.87039476134352\\
3.17349368874399	1.84028429845796\\
3.18235238645018	1.80820986219087\\
3.1895385885222	1.77381130435142\\
3.1948665410602	1.73665456055376\\
3.19806371075895	1.69621226889143\\
3.19874532252459	1.651838317919\\
3.1963779229547	1.60273431584277\\
3.19022729282418	1.54790531486466\\
3.17928382873872	1.48610135786744\\
3.16215537235939	1.41574068525492\\
3.13691319543502	1.33481019147983\\
3.1008716536058	1.24074009733246\\
3.0502774481935	1.13025550889201\\
2.97988581083231	0.999223520956103\\
2.88242297065914	0.842552786578786\\
2.74801189806255	0.654283612795641\\
2.56383956356074	0.42815369958251\\
2.31475351698567	0.159119005239815\\
1.98603619888372	-0.153648071589967\\
1.56964637155804	-0.502450673511533\\
1.07307920943061	-0.867685243885396\\
0.525045715892561	-1.21982070756272\\
-0.0300686417610128	-1.52892864853262\\
-0.54848079440757	-1.77614915002699\\
-1.00181111163593	-1.95815382094919\\
-1.38061631647323	-2.08304703639134\\
-1.68892329885739	-2.16344268568353\\
-1.93698816546921	-2.21156852013282\\
-2.13631241828859	-2.23724727348235\\
-2.29726209591901	-2.24765874981433\\
-2.42832874967536	-2.24779758371498\\
-2.53615694693746	-2.24105114662836\\
-2.62583786134551	-2.22968834031408\\
-2.70124308342754	-2.21521955092914\\
-2.7653167740668	-2.19864577489946\\
-2.82030728836154	-2.18062611348689\\
-2.86794271978042	-2.16158898882639\\
-2.90956118911193	-2.14180569951998\\
-2.94620672239702	-2.12143906531823\\
-2.97869967160048	-2.10057561443679\\
-3.00768851348651	-2.07924683740924\\
-3.03368805283886	-2.0574430954663\\
-3.05710765533064	-2.03512250741799\\
-3.07827209651661	-2.01221631374512\\
-3.09743685605133	-1.98863167260534\\
-3.1147991341797	-1.96425247643176\\
-3.13050545896176	-1.93893852243978\\
-3.14465643976465	-1.91252318125205\\
-3.157308968138	-1.88480955464968\\
-3.16847594086442	-1.85556497395101\\
-3.17812335416495	-1.8245135466138\\
-3.18616436406186	-1.79132629348262\\
-3.19244959157946	-1.7556082148057\\
-3.19675252762576	-1.71688135921992\\
-3.19874829759734	-1.67456262237005\\
-3.19798318852437	-1.62793454443151\\
-3.19383109034958	-1.57610678635818\\
-3.18543117438541	-1.51796524405418\\
-3.17159849099963	-1.45210498074569\\
-3.15069547410926	-1.37674258308107\\
-3.12044753174917	-1.28960389793679\\
-3.07768067158824	-1.18778615883497\\
-3.01795644245956	-1.06760331224818\\
-2.93508840278829	-0.924448472815183\\
-2.8205679718414	-0.752764040392522\\
-2.66305620321969	-0.546322006812653\\
-2.44839501627363	-0.299196532781962\\
-2.16110678567916	-0.00797435955711197\\
-1.78879993103978	0.324476345462676\\
-1.33009207596989	0.684705987266555\\
-0.802928758313889	1.04740134737392\\
-0.24540594588675	1.38129334496077\\
0.296099945092465	1.66082990681729\\
0.784276385241216	1.8749810740011\\
1.20050216946353	2.0269837269649\\
1.54301420737431	2.12799812665259\\
1.81975778976501	2.19086075125827\\
2.04205424924148	2.22670699051684\\
2.22100759983092	2.24400500522965\\
2.36607513206161	2.24876922711939\\
};
\addplot [color=mycolor1, forget plot]
  table[row sep=crcr]{%
0.873456719862334	4.01693373145583\\
1.05891327869839	4.01108540720015\\
1.23682310230941	3.9941307624143\\
1.40751329045177	3.96684723559204\\
1.5713287044198	3.92987209111072\\
1.72861344520434	3.88370883688959\\
1.8796961552078	3.8287345317985\\
2.02487818403272	3.76520719212756\\
2.16442376584047	3.69327281387803\\
2.29855146967883	3.61297176422278\\
2.42742629184089	3.52424447990469\\
2.55115185588764	3.42693656085289\\
2.66976227232088	3.32080348051212\\
2.78321329113623	3.20551526489463\\
2.89137246526386	3.08066163265594\\
2.99400814334819	2.94575824861939\\
3.09077724232353	2.80025492986278\\
3.18121193375922	2.64354685821892\\
3.26470563673946	2.47499008868385\\
3.34049907010301	2.29392287872273\\
3.40766760223331	2.09969455651876\\
3.46511176086235	1.89170372440407\\
3.51155351700004	1.66944744605708\\
3.54554177723419	1.4325825396255\\
3.56547127237876	1.18099900892871\\
3.56961947845612	0.914903810388281\\
3.55620599223771	0.634910472204568\\
3.52347746660291	0.342126651254164\\
3.46981837323584	0.0382279755197088\\
3.39388331735628	-0.274496619328825\\
3.29474071424938	-0.593144051856046\\
3.1720114463944	-0.914254534674148\\
3.02598151755879	-1.23393444664973\\
2.85766690030261	-1.54804101814983\\
2.66881335507843	-1.85241233716958\\
2.46182399016865	-2.14311481241083\\
2.23962059159806	-2.4166742889777\\
2.00545749895331	-2.67025899425647\\
1.76271511630104	-2.90179211307992\\
1.5147016951166	-3.10998598831279\\
1.26448700290808	-3.29430426487939\\
1.01478226487975	-3.45486885847563\\
0.767870580179866	-3.59233351054688\\
0.525583595809768	-3.70774515597869\\
0.289314989276039	-3.8024101063091\\
0.0600593545353311	-3.87777627277717\\
-0.161534365867956	-3.9353370560545\\
-0.375103765549384	-3.97655811533715\\
-0.580514536320754	-4.00282531612021\\
-0.777806689788507	-4.01541060842645\\
-0.967147489666844	-4.01545206439023\\
-1.1487917526378	-4.00394443509903\\
-1.32304927507435	-3.9817370633439\\
-1.49025863160819	-3.949536594988\\
-1.65076635603593	-3.90791253566546\\
-1.8049104570209	-3.85730423461502\\
-1.95300726548201	-3.79802831936665\\
-2.09534070557109	-3.73028595429559\\
-2.232153193876	-3.65416956580092\\
-2.36363748332244	-3.56966888501637\\
-2.48992887056574	-3.47667632423253\\
-2.6110972767274	-3.37499184311909\\
-2.72713879426336	-3.26432759100626\\
-2.8379663744662	-3.14431274546656\\
-2.94339942097207	-3.01449911656582\\
-3.04315216863044	-2.87436825898505\\
-3.13682088210797	-2.72334103518177\\
-3.22387012658132	-2.56079079956444\\
-3.30361866903396	-2.38606161298636\\
-3.37522598876934	-2.19849311809292\\
-3.43768093001322	-1.99745385297501\\
-3.489794721229	-1.78238476237182\\
-3.53020138305169	-1.55285435020383\\
-3.55736935648446	-1.30862613309598\\
-3.56962882114418	-1.04973761227755\\
-3.5652193417877	-0.776587727073905\\
-3.54236177337171	-0.490026666871447\\
-3.49935632124874	-0.191438251483621\\
-3.43470496054046	0.117198492103618\\
-3.34725110462667	0.43328428570102\\
-3.23632317040156	0.753626484098471\\
-3.10186302120267	1.07452892770211\\
-2.94451729347739	1.39194696563502\\
-2.76567137060142	1.70169765748539\\
-2.56741318493844	1.99970253001784\\
-2.35242605060152	2.28223116587682\\
-2.12382325714924	2.54611176537777\\
-1.88494811092087	2.78888085186346\\
-1.63916818199947	3.00885671507353\\
-1.38969059392881	3.20513601141079\\
-1.13941768184485	3.37752576543612\\
-0.890852237628319	3.52643083375097\\
-0.646051990474897	3.65271891601133\\
-0.406626059034493	3.75758255335449\\
-0.173762573571543	3.84241229809124\\
0.0517239108538937	3.90868937694286\\
0.269335186361097	3.95790108610335\\
0.478829977292244	3.99147848747363\\
0.680167467369668	4.01075377987807\\
0.873456719862334	4.01693373145583\\
};
\addplot [color=mycolor2, forget plot]
  table[row sep=crcr]{%
2.40210553937661	2.3630934927815\\
2.50510161403136	2.3599252946844\\
2.59093986907906	2.35180674191589\\
2.66329357058539	2.34029018666113\\
2.72494730336123	2.32641316944438\\
2.77802380521816	2.31086697042658\\
2.82415333884121	2.29410829369208\\
2.86459819850988	2.2764331562321\\
2.90034375025032	2.25802585522687\\
2.93216501324467	2.2389914814089\\
2.96067549452266	2.21937748795096\\
2.98636313825409	2.19918788625439\\
3.00961686072481	2.178392380323\\
3.03074612968263	2.15693193032603\\
3.04999531717496	2.13472169426589\\
3.06755402708116	2.11165193088673\\
3.08356420926532	2.08758718990546\\
3.0981245740805	2.06236392291898\\
3.11129257730319	2.03578648931227\\
3.12308402653721	2.00762138329789\\
3.13347013824599	1.9775893515926\\
3.1423716207288	1.94535488741487\\
3.1496490370355	1.91051235464126\\
3.1550882650705	1.87256769100378\\
3.15837925131487	1.83091423017277\\
3.15908534884755	1.78480063269962\\
3.15659919152362	1.73328818764031\\
3.15007907223557	1.67519382034874\\
3.13835688076933	1.60901406613714\\
3.11980449825242	1.53282429113761\\
3.09213997322834	1.44414731633822\\
3.05214840971314	1.33978830226637\\
2.99528834615863	1.21564299799959\\
2.915162391427	1.0665146585686\\
2.80287740920904	0.886041612979106\\
2.64646409737929	0.666974814346141\\
2.43087006477885	0.40227371715175\\
2.13964864643347	0.0877097586390446\\
1.75998295128699	-0.273604994523119\\
1.29162468022708	-0.666084542994651\\
0.755735709369115	-1.06045141329421\\
0.194381811289358	-1.42137835431777\\
-0.343993528887985	-1.72136328187118\\
-0.8230959741329	-1.94997123111311\\
-1.22706646442504	-2.11223255958094\\
-1.55674573595149	-2.22096273838767\\
-1.82167218307478	-2.29005839159569\\
-2.03382776263391	-2.33121944287373\\
-2.20441075259616	-2.35319302273665\\
-2.3427115930779	-2.36213552675717\\
-2.45602015570107	-2.36225150184665\\
-2.54991195846396	-2.35637325432495\\
-2.62861039286575	-2.34639861192798\\
-2.69531198456748	-2.33359694759486\\
-2.75244592225422	-2.3188156891715\\
-2.80187063105124	-2.30261759563426\\
-2.84501915808969	-2.28537159093286\\
-2.88300566451951	-2.26731289744304\\
-2.91670326267435	-2.24858293136569\\
-2.94680101222194	-2.22925579567656\\
-2.97384580138051	-2.20935580740452\\
-2.99827322614566	-2.18886893315524\\
-3.02043039130044	-2.16774999072742\\
-3.04059269726404	-2.14592680917192\\
-3.05897605708338	-2.12330209619639\\
-3.07574553608595	-2.09975345711912\\
-3.09102106812584	-2.07513178920378\\
-3.10488063548612	-2.04925810255761\\
-3.11736107164653	-2.0219186677147\\
-3.12845642910063	-1.99285823992896\\
-3.13811362046281	-1.96177094239724\\
-3.14622475857645	-1.92828818562114\\
-3.15261524936247	-1.89196273519221\\
-3.15702617264633	-1.85224768755824\\
-3.15908873808585	-1.80846863872752\\
-3.15828750318021	-1.75978669611459\\
-3.15390741000263	-1.70514915648678\\
-3.14495728852243	-1.64322365788887\\
-3.13005897651501	-1.57231053418962\\
-3.10728633676627	-1.49022740200106\\
-3.07393230130358	-1.39416092882715\\
-3.02617613796275	-1.28048634955713\\
-2.9586230951157	-1.14457270076291\\
-2.86371092077442	-0.980635813944171\\
-2.73106306753107	-0.781798681809504\\
-2.54709840982175	-0.540702677403615\\
-2.29568726130443	-0.251266722847516\\
-1.96130476006145	0.0877368265485396\\
-1.53613415024095	0.467498290295421\\
-1.02987232282991	0.865253094815708\\
-0.475123546990699	1.24715643189577\\
0.0803998746625715	1.58007232845334\\
0.592422884350118	1.84455614988675\\
1.03471808848556	2.03868190768345\\
1.40071126286749	2.17239124480139\\
1.69656031103102	2.25966486586898\\
1.93360764642683	2.31351636318777\\
2.12368640087881	2.34416651870353\\
2.27709645851434	2.35899206933955\\
2.40210553937661	2.3630934927815\\
};
\addplot [color=mycolor1, forget plot]
  table[row sep=crcr]{%
0.88748083090382	3.97200858962911\\
1.0720778919661	3.96618849790435\\
1.24895705497356	3.94933318180974\\
1.41845873996219	3.92224071305184\\
1.58094188710687	3.88556733999441\\
1.73676494485547	3.83983413351367\\
1.88627089384102	3.78543458741033\\
2.02977530398858	3.72264233455396\\
2.16755653586359	3.65161846206454\\
2.29984731849397	3.57241815496482\\
2.42682705012332	3.48499658733104\\
2.54861426999769	3.38921413209142\\
2.66525883878065	3.28484109258449\\
2.77673344728929	3.17156228599146\\
2.88292415622928	3.04898194384937\\
2.98361976516508	2.91662954878138\\
3.07849993246268	2.77396740708429\\
3.16712213882791	2.62040096697476\\
3.24890782883676	2.45529312748884\\
3.32312840391552	2.27798402600529\\
3.38889220239087	2.08781800617625\\
3.44513420522414	1.88417958710035\\
3.49061094602754	1.66654017533366\\
3.52390393434153	1.43451683673792\\
3.5434357040016	1.18794349033633\\
3.54750315362382	0.926953207658468\\
3.53433281244008	0.652067762950401\\
3.50216159819999	0.364287218810285\\
3.44934408098872	0.0651685085542933\\
3.37448295939045	-0.243121487807968\\
3.27657359760087	-0.557794681664479\\
3.15514702279179	-0.875487236086551\\
3.01039053558596	-1.19237084085232\\
2.84322332138081	-1.50432859235038\\
2.65530806361779	-1.80718120739027\\
2.44898895673042	-2.09693696378953\\
2.22715988255978	-2.37003130859836\\
1.9930801181445	-2.62352252760346\\
1.75016456792847	-2.85521852822322\\
1.50177832216072	-3.06372391945442\\
1.25106108174094	-3.24841164825857\\
1.00079784722329	-3.40933518809728\\
0.753341613742977	-3.54710327268561\\
0.510584667334669	-3.6627393695639\\
0.273969153648827	-3.75754413326148\\
0.0445251531774694	-3.83297319760561\\
-0.177075181045537	-3.89053676919391\\
-0.390456895440626	-3.93172272686753\\
-0.595481787898095	-3.95794172913841\\
-0.792192586781656	-3.97049108501902\\
-0.9807640315538	-3.97053350968948\\
-1.16146162361766	-3.95908697293618\\
-1.3346078374386	-3.93702232266239\\
-1.50055502356405	-3.90506599125295\\
-1.6596639758734	-3.86380572390053\\
-1.81228706908935	-3.81369783019103\\
-1.95875491793273	-3.7550749251003\\
-2.09936560985006	-3.68815349230441\\
-2.23437568316647	-3.61304088447291\\
-2.36399214139935	-3.52974159061903\\
-2.48836490267482	-3.43816276914898\\
-2.60757917840563	-3.338119185261\\
-2.72164736035928	-3.22933781906081\\
-2.83050007654208	-3.11146254031263\\
-2.93397616391571	-2.98405938907472\\
-3.03181141332334	-2.84662316795648\\
-3.12362608650336	-2.69858624707172\\
-3.20891140823143	-2.53933070666704\\
-3.28701552366608	-2.36820518468506\\
-3.35712980862818	-2.18454803109935\\
-3.41827695185381	-1.98771854765969\\
-3.46930290230872	-1.77713812720729\\
-3.50887556937882	-1.55234287588744\\
-3.53549400081329	-1.31304863658416\\
-3.54751247725296	-1.05922803781881\\
-3.54318427306414	-0.791197093373195\\
-3.52072934240834	-0.509705907386554\\
-3.47842843077109	-0.216024376266373\\
-3.41474269343404	0.0879899996174577\\
-3.32845275249314	0.399857903219787\\
-3.2188047962198	0.71649368801053\\
-3.08564520760163	1.03428315833578\\
-2.92952141661873	1.34922808352798\\
-2.7517274258466	1.65714986611883\\
-2.55427904466183	1.95393157086205\\
-2.3398156170525	2.23576713284225\\
-2.11143905370944	2.49938282954807\\
-1.87251307579609	2.74220082153557\\
-1.62645200166476	2.96242645934868\\
-1.37652754064401	3.15905625761151\\
-1.12571497113936	3.33181728725355\\
-0.876589749548109	3.48105775263239\\
-0.631275386710711	3.60761149658494\\
-0.391435775371032	3.71265703174771\\
-0.158301014886573	3.79758650411687\\
0.0672851495639999	3.86389391027349\\
0.284806868002934	3.91308646802211\\
0.494014904291655	3.94661904697569\\
0.694868100710875	3.96584912424045\\
0.887480830903819	3.97200858962911\\
};
\addplot [color=mycolor2, forget plot]
  table[row sep=crcr]{%
2.4225402142989	2.46018632662555\\
2.51205643688159	2.45742975989243\\
2.58715310402205	2.45032443935752\\
2.65088726236523	2.44017747558856\\
2.70557307792621	2.4278666981761\\
2.75297624258242	2.41398036768896\\
2.79445592127621	2.39890929225061\\
2.83106778148485	2.38290771886248\\
2.86363880157304	2.36613373705743\\
2.89282178268163	2.34867615047899\\
2.91913526192633	2.33057230571732\\
2.94299286602794	2.31181977576321\\
2.9647249495008	2.29238376667649\\
2.98459451106734	2.27220144467588\\
3.00280877580309	2.25118393410117\\
3.01952739229601	2.22921642937336\\
3.03486786757133	2.20615664079546\\
3.04890860593242	2.18183161751885\\
3.06168969865664	2.15603283440837\\
3.07321140170315	2.12850927126795\\
3.08343001054312	2.0989580327608\\
3.09225056289839	2.06701183348295\\
3.09951542959146	2.03222237847955\\
3.10498733118484	1.99403827093061\\
3.10832455512295	1.95177553164861\\
3.10904501145278	1.90457806492188\\
3.10647405557169	1.85136439391878\\
3.09966843494282	1.79075568108779\\
3.08730489620387	1.72097851095623\\
3.06751649839952	1.63973451932123\\
3.03765237194324	1.54402890082052\\
2.99392870120542	1.42995433931523\\
2.93093549525602	1.29244386400949\\
2.84098235827299	1.12505312272658\\
2.71335191405367	0.919943702626728\\
2.53377842768651	0.668462503356907\\
2.28502042078931	0.363048770166131\\
1.95020895826046	0.00136628808190746\\
1.52075402939387	-0.407433008979272\\
1.00742697991319	-0.837777316427742\\
0.446142849925326	-1.25106537212028\\
-0.111707244002825	-1.60996241837187\\
-0.620392649297839	-1.89356900100245\\
-1.05487621394852	-2.10098517856431\\
-1.41084644802164	-2.24401564095483\\
-1.69636065488845	-2.33820047087083\\
-1.92385592987469	-2.39753995321719\\
-2.10560952422855	-2.43280258185524\\
-2.25198836060394	-2.45165586010096\\
-2.37115351288484	-2.45935790680001\\
-2.46932748905164	-2.45945525895173\\
-2.55119603512821	-2.45432688174761\\
-2.62028102094775	-2.44556815644518\\
-2.67923988611994	-2.43425027528856\\
-2.73009188392266	-2.42109225869469\\
-2.77438441764409	-2.40657438532166\\
-2.81331400459103	-2.39101305145171\\
-2.84781404185429	-2.37461034956898\\
-2.8786186291562	-2.35748701566959\\
-2.90630918515096	-2.33970433405003\\
-2.93134866004977	-2.32127860593696\\
-2.95410673594858	-2.30219050947069\\
-2.97487839752585	-2.2823908489408\\
-2.99389753846539	-2.26180364522069\\
-3.0113467549276	-2.2403271507598\\
-3.02736410100977	-2.21783311276133\\
-3.0420472940977	-2.19416441213581\\
-3.0554556237838	-2.16913104219236\\
-3.0676096069072	-2.1425042361312\\
-3.07848821597365	-2.11400838576039\\
-3.08802325950052	-2.08331019481765\\
-3.09609017451912	-2.05000425497905\\
-3.10249405386319	-2.01359389111321\\
-3.1069491015482	-1.97346565571904\\
-3.10904877997025	-1.92885521122919\\
-3.10822251951847	-1.87880146499304\\
-3.10367276308135	-1.8220846630784\\
-3.09428297552815	-1.75714270577492\\
-3.07848264028545	-1.68195840127306\\
-3.05404884139697	-1.59390941537601\\
-3.01781608985064	-1.48957429658994\\
-2.96525931255143	-1.36449725176504\\
-2.88991926847882	-1.21294324366504\\
-2.78268341530166	-1.02774855362424\\
-2.63108638333475	-0.800533386004224\\
-2.41917822405782	-0.5228320884181\\
-2.12922877244577	-0.189017812821461\\
-1.7472170033862	0.198338271800076\\
-1.27289457392451	0.622144237570358\\
-0.729699692373591	1.04912960977973\\
-0.16354907825766	1.43911799032485\\
0.374313533081504	1.76164762221984\\
0.847581194730859	2.00624311026009\\
1.24235269675514	2.17958156815997\\
1.56166558886184	2.29626968605072\\
1.816559003061	2.37147371815747\\
2.01974994150058	2.4176362919499\\
2.18266083639468	2.44390408090305\\
2.31454294060768	2.45664623494165\\
2.4225402142989	2.46018632662555\\
};
\addplot [color=mycolor1, forget plot]
  table[row sep=crcr]{%
0.915533679617734	3.88713310961483\\
1.09852957586721	3.8813657567245\\
1.27346356547106	3.86469800603206\\
1.44070502679724	3.83796895114571\\
1.60064309186958	3.80187211215185\\
1.75366660945792	3.75696259991024\\
1.90014859260987	3.70366536152442\\
2.04043405159741	3.64228356855361\\
2.1748302425219	3.57300656400907\\
2.30359849977727	3.49591705144124\\
2.42694694982747	3.4109974107844\\
2.54502351717723	3.31813518200184\\
2.65790873062758	3.21712788786888\\
2.76560792312149	3.10768748793501\\
2.86804249941612	2.9894448810189\\
2.96504003340893	2.86195501551079\\
3.05632306563367	2.72470333476827\\
3.14149661968498	2.57711448481228\\
3.22003466697259	2.41856444310393\\
3.29126606904786	2.24839747987543\\
3.35436094428586	2.06594961004835\\
3.40831896547524	1.87058038140072\\
3.45196180628364	1.66171488462126\\
3.48393279307768	1.43889762943759\\
3.50270769798084	1.20185922764917\\
3.50662134749873	0.950595448240291\\
3.49391501100189	0.685455972113881\\
3.46280893822414	0.407237005046545\\
3.41160242848525	0.117268009256084\\
3.3388000232035	-0.182521169850204\\
3.24325677681965	-0.489569388934429\\
3.12432874896352	-0.800706281557046\\
2.98200847083346	-1.11223998463782\\
2.81702153450226	-1.42011406428428\\
2.63086202041295	-1.72012409561288\\
2.42575242199516	-2.0081703800757\\
2.20452699361871	-2.28051288937255\\
1.97045256789499	-2.53399172221552\\
1.72701316031923	-2.76618270743753\\
1.47769023896395	-2.97547140947847\\
1.22576810427838	-3.16104517817101\\
0.974185079409573	-3.32281700200052\\
0.725439687995681	-3.46130334842394\\
0.481550323893453	-3.57748007529576\\
0.244059467334431	-3.67263724172967\\
0.014069969409949	-3.74824762771832\\
-0.207699290829558	-3.80585726085608\\
-0.420849176249404	-3.84700075283503\\
-0.625235479465544	-3.87314040851484\\
-0.82090867815256	-3.88562588125512\\
-1.00806102171395	-3.8856702655477\\
-1.18698195644625	-3.8743385047252\\
-1.35802174485055	-3.85254446054366\\
-1.52156247283532	-3.82105366033698\\
-1.67799532803701	-3.78048942963511\\
-1.82770294948821	-3.73134074045006\\
-1.97104569765303	-3.67397062098299\\
-2.10835080773941	-3.60862437772012\\
-2.23990352580227	-3.53543718918442\\
-2.36593946219093	-3.45444086222352\\
-2.48663751863826	-3.36556971807446\\
-2.60211285033001	-3.26866571656902\\
-2.71240941474592	-3.16348305042861\\
-2.81749174097168	-3.04969256296791\\
-2.91723563586649	-2.92688647478541\\
-3.01141763956755	-2.79458405898928\\
-3.09970316870044	-2.65223908805077\\
-3.18163346197881	-2.49925009167618\\
-3.25661169457858	-2.3349747094033\\
-3.32388898286324	-2.15874967602557\\
-3.3825514872278	-1.96991820363688\\
-3.43151045682256	-1.76786665163131\\
-3.46949784105365	-1.55207229500711\\
-3.49507096764294	-1.32216355405645\\
-3.50663062600307	-1.07799303572134\\
-3.5024574558465	-0.819721949421519\\
-3.4807714497124	-0.547911752619291\\
-3.43981815001576	-0.263615298723845\\
-3.37798226858461	0.0315442947739052\\
-3.29392471123411	0.335322455088922\\
-3.18673262583612	0.644846778872989\\
-3.05606523351181	0.956670804307634\\
-2.90227288426035	1.26689664545942\\
-2.72646550439925	1.57136383997961\\
-2.53051133047817	1.86588772527822\\
-2.31695772449058	2.1465178955073\\
-2.08888061230601	2.40978030367758\\
-1.84968336974496	2.65286836215824\\
-1.60287524639162	2.87375881821858\\
-1.35186094488384	3.07124383893494\\
-1.0997670197766	3.24488656013633\\
-0.849320137476856	3.39491889635324\\
-0.602780744122213	3.52210551505497\\
-0.361926436947026	3.62759693510459\\
-0.128073852783262	3.71278975837447\\
0.0978738583301783	3.77920553430087\\
0.315365906003484	3.82839362228991\\
0.524138523843202	3.86185871657206\\
0.724151960402994	3.88101071234976\\
0.915533679617733	3.88713310961483\\
};
\addplot [color=mycolor2, forget plot]
  table[row sep=crcr]{%
2.4299081702385	2.54116882702592\\
2.50787494271559	2.53876558816946\\
2.57366262941401	2.53253896978367\\
2.62983071977584	2.52359475287463\\
2.67831546206082	2.51267833276714\\
2.72059573800434	2.50029126138849\\
2.75781232015285	2.48676781509864\\
2.79085365342427	2.47232557052543\\
2.82041782125844	2.45709897920762\\
2.84705757465151	2.44116170019426\\
2.87121325252423	2.42454138381536\\
2.89323696401346	2.40722927966503\\
2.91341038049567	2.389186191406\\
2.93195776796347	2.37034574422184\\
2.94905538221762	2.35061555615369\\
2.96483798082337	2.32987664103611\\
2.97940292712109	2.30798117168998\\
2.9928121358354	2.28474856519831\\
3.00509190739402	2.25995969292994\\
3.01623049197373	2.23334884492343\\
3.02617298565802	2.20459286898918\\
3.03481285392039	2.17329663289134\\
3.04197895192468	2.13897358873028\\
3.04741629281838	2.10101970552392\\
3.05075789123247	2.05867831603404\\
3.05148360533599	2.01099241587134\\
3.04885975085121	1.95673956443917\\
3.04184997123927	1.89434269996271\\
3.02898288471224	1.82174796300119\\
3.00815480790283	1.73625854312748\\
2.97633622607303	1.63431340182052\\
2.9291405632774	1.51120649421977\\
2.86021201401299	1.36076848408523\\
2.76042381261721	1.17510775670447\\
2.61702285655303	0.944686228955867\\
2.41326229334755	0.659356991790123\\
2.12990038377572	0.311457516252769\\
1.75089642057547	-0.0980187258714269\\
1.27463368388592	-0.551510957104492\\
0.725603479611226	-1.01200187265938\\
0.153423684980919	-1.43355393844155\\
-0.386895878096306	-1.781373752186\\
-0.857807998042484	-2.04405169615064\\
-1.24655035727424	-2.22970197490458\\
-1.55802675628121	-2.35488668960367\\
-1.80474892698442	-2.4362870199259\\
-2.00026827854185	-2.48728888963669\\
-2.15635972917608	-2.51757183527422\\
-2.28235123026508	-2.53379703128516\\
-2.3853325708139	-2.54045052172632\\
-2.47060400784085	-2.54053264189128\\
-2.54211426479498	-2.53605092303616\\
-2.60281571015797	-2.52835313116987\\
-2.65493194885777	-2.51834705799954\\
-2.70015307733974	-2.50664450911494\\
-2.7397763721719	-2.49365571111631\\
-2.77480739784914	-2.47965149699101\\
-2.80603286524599	-2.46480449290115\\
-2.83407341333794	-2.44921650261795\\
-2.85942208340816	-2.43293670021977\\
-2.8824725214897	-2.4159735910477\\
-2.90353972275058	-2.39830264290297\\
-2.92287527559821	-2.37987080314453\\
-2.94067846079959	-2.36059866261591\\
-2.95710412985306	-2.34038071534023\\
-2.97226796853908	-2.31908393637231\\
-2.98624950341748	-2.29654472067452\\
-2.99909299838002	-2.27256406567429\\
-3.01080618748486	-2.24690071671462\\
-3.02135657185462	-2.21926180642577\\
-3.03066474070404	-2.18929028214403\\
-3.03859381758708	-2.15654809985281\\
-3.04493362220341	-2.12049372869982\\
-3.0493773843031	-2.08045190262606\\
-3.05148770902473	-2.03557270286934\\
-3.05064675651646	-1.98477586828778\\
-3.04598293737177	-1.92667462167759\\
-3.03626237354422	-1.859471247099\\
-3.01972734727122	-1.78081438888453\\
-2.99385548589297	-1.68760656457908\\
-2.95500305891746	-1.57575272729209\\
-2.89788792633696	-1.43985494402061\\
-2.81487854563123	-1.27290378155379\\
-2.6951315525663	-1.06613497952922\\
-2.52387147871502	-0.809478334678318\\
-2.2827155102493	-0.493463210432633\\
-1.95295399038066	-0.113790440084776\\
-1.52408987804222	0.321166530106526\\
-1.0065467764534	0.783769472801461\\
-0.438781182587107	1.23030599286926\\
0.12365660080417	1.61796478898512\\
0.632364709248575	1.92317878057828\\
1.06242978314545	2.14554424480842\\
1.41124442532834	2.29875214256913\\
1.68861198838555	2.40013228843616\\
1.9081169061554	2.46490185274609\\
2.08260241895416	2.50454347441469\\
2.22262737845324	2.52711937482361\\
2.33635043410676	2.5381045734392\\
2.4299081702385	2.54116882702592\\
};
\addplot [color=mycolor1, forget plot]
  table[row sep=crcr]{%
0.956492163373761	3.77412205831772\\
1.13740076407384	3.76842378481052\\
1.30974206413514	3.75200626229156\\
1.47393318508464	3.72576781510803\\
1.6304120425485	3.69045468628328\\
1.77961572647687	3.64666912332949\\
1.9219640807902	3.59487871776386\\
2.05784721601137	3.53542590681148\\
2.18761584721043	3.46853695485647\\
2.3115735180329	3.39433003513442\\
2.42996992777755	3.31282225379333\\
2.54299471316223	3.22393562312583\\
2.65077114829901	3.12750212021302\\
2.75334931928612	3.02326808008332\\
2.85069841099577	2.91089828469517\\
2.94269782332709	2.78998023422623\\
3.02912692577029	2.66002923679982\\
3.10965337973529	2.52049513581231\\
3.18382012942104	2.37077171518533\\
3.25103141066373	2.21021007956218\\
3.31053848396032	2.03813758337323\\
3.36142629425119	1.85388414361534\\
3.40260292041893	1.65681794814161\\
3.43279450382072	1.44639255212235\\
3.45054929220922	1.22220697421679\\
3.45425537515231	0.984079443041352\\
3.44217736834696	0.732133662143365\\
3.41251732322385	0.466893666768205\\
3.36350395342279	0.189379534222988\\
3.29351130938284	-0.0988082424650076\\
3.20120293909346	-0.395432053992662\\
3.08569057875153	-0.697606752326291\\
2.94668872432899	-1.00185257784537\\
2.78464033521322	-1.30422162394327\\
2.60078732071264	-1.60049568740128\\
2.3971646973498	-1.88643777475042\\
2.17650964895159	-2.15806518450049\\
1.94209347366312	-2.41190418402993\\
1.69750048694484	-2.64518837669219\\
1.44638803650607	-2.85597474469135\\
1.19226280410582	-3.04316905185557\\
0.938301050414134	-3.20646988028905\\
0.687227916835214	-3.34625299598594\\
0.441257847553226	-3.46342270785044\\
0.202088153572837	-3.55925509628672\\
-0.0290677594005788	-3.63525195926824\\
-0.251422224679322	-3.69301689447977\\
-0.464543240306113	-3.73415822607492\\
-0.668282385146043	-3.76021855937426\\
-0.862707241732954	-3.77262779087261\\
-1.04804191430491	-3.77267508472103\\
-1.22461703881608	-3.76149514011564\\
-1.39282923260587	-3.74006453499351\\
-1.55310909471179	-3.70920467702781\\
-1.70589647410159	-3.66958869119952\\
-1.85162161649841	-3.62175029923513\\
-1.99069085949854	-3.56609334832289\\
-2.12347568568543	-3.50290111685884\\
-2.25030411081334	-3.43234487799468\\
-2.37145354827313	-3.35449146089099\\
-2.48714443705732	-3.26930973983096\\
-2.59753404351027	-3.17667612588477\\
-2.70270994881875	-3.07637925481466\\
-2.80268282012914	-2.96812417560813\\
-2.89737814214477	-2.85153646114373\\
-2.98662666980141	-2.72616679860698\\
-3.07015346630092	-2.59149678270892\\
-3.14756553372927	-2.44694683662406\\
-3.21833824985063	-2.29188742545607\\
-3.2818011233653	-2.12565499683434\\
-3.33712380274189	-1.94757435838472\\
-3.38330385000719	-1.75698943172239\\
-3.41915853623843	-1.55330441875053\\
-3.44332381209519	-1.33603724016791\\
-3.45426457288477	-1.10488646486432\\
-3.45030118873616	-0.859811606680412\\
-3.42965768085947	-0.601124392539394\\
-3.3905364155057	-0.329585284462418\\
-3.33122218023402	-0.0464953272455659\\
-3.25021449801282	0.24623106449233\\
-3.14638090878713	0.546030203512032\\
-3.01911638353627	0.849708542167321\\
-2.86848679485848	1.15353175153547\\
-2.69533015530887	1.45338808905218\\
-2.50129095670289	1.74501628182254\\
-2.28877187934261	2.02427253179527\\
-2.0608022087767	2.28739951598753\\
-1.82083942887876	2.53125710237021\\
-1.57253412315359	2.75348176980381\\
-1.31949403672393	2.95255719857406\\
-1.06507961242214	3.12779689068709\\
-0.812252728524991	3.27925516934623\\
-0.563487029195763	3.40759168605377\\
-0.320736368861388	3.51391591728976\\
-0.0854500579454544	3.59963385723916\\
0.141379654705287	3.66631205839245\\
0.359152010999243	3.71556690270327\\
0.567585643337569	3.74898109802236\\
0.766648128212963	3.76804547197061\\
0.95649216337376	3.77412205831772\\
};
\addplot [color=mycolor2, forget plot]
  table[row sep=crcr]{%
2.42663064814023	2.60785811664781\\
2.49468174008971	2.60575870030039\\
2.55239939390668	2.60029427736913\\
2.60193795001227	2.59240432776687\\
2.64492702754717	2.58272400429261\\
2.68261278465861	2.57168187990689\\
2.7159583805664	2.55956400668035\\
2.74571577241066	2.54655620235487\\
2.7724773872352	2.53277211909779\\
2.79671359330112	2.51827189344838\\
2.81880006068362	2.50307443871288\\
2.83903783407902	2.48716533789187\\
2.85766806721563	2.47050158520723\\
2.87488276140725	2.4530139569977\\
2.89083242168568	2.43460747402141\\
2.90563123071895	2.41516018695922\\
2.91936009944686	2.3945203352042\\
2.93206775108925	2.37250176715401\\
2.94376980445504	2.34887734481953\\
2.95444561662831	2.32336986413215\\
2.96403239338554	2.29563977935101\\
2.97241573727883	2.26526869274245\\
2.97941531880948	2.23173711437162\\
2.98476363463382	2.19439434819838\\
2.98807471584772	2.15241743103571\\
2.98879794433029	2.10475472069131\\
2.9861494745245	2.05004785320334\\
2.97900961012992	1.98652323679204\\
2.96576810702092	1.91184105925449\\
2.94408993137816	1.82288661116335\\
2.91056126141298	1.71548807070542\\
2.86016234749081	1.58405422656099\\
2.7855138056449	1.42116462835929\\
2.67590028700184	1.21725945971396\\
2.51630693803723	0.96085555711886\\
2.28733937387612	0.640253754606333\\
1.96811113829447	0.248309634107519\\
1.54507742635003	-0.208816149717553\\
1.02660002081288	-0.702678714632209\\
0.452566596117432	-1.18438280496429\\
-0.116845507547197	-1.60413285410775\\
-0.629111139445965	-1.93406654496179\\
-1.05822398949492	-2.17352895352723\\
-1.40273388675657	-2.33810426546856\\
-1.67410244944011	-2.44718986880552\\
-1.88715984915978	-2.51749020682078\\
-2.05545301269843	-2.56139108298409\\
-2.18985564718869	-2.58746499239997\\
-2.29861963965018	-2.60146966362994\\
-2.3878659616402	-2.60723369569702\\
-2.46210796987436	-2.60730327728648\\
-2.52468330161363	-2.60337981840676\\
-2.57807879986109	-2.59660701049621\\
-2.62416564382979	-2.58775721560065\\
-2.66436694812378	-2.57735252840745\\
-2.69977684240484	-2.56574383339848\\
-2.73124533318201	-2.553162806889\\
-2.75943915518276	-2.53975635217672\\
-2.7848857327831	-2.52560948615007\\
-2.80800517492384	-2.51076050824391\\
-2.82913370106633	-2.49521089982419\\
-2.84854084410645	-2.47893151898856\\
-2.86644204919181	-2.4618660817546\\
-2.88300777859459	-2.44393253682758\\
-2.89836986805199	-2.42502267264894\\
-2.91262560739631	-2.4050000935758\\
-2.92583980018095	-2.3836965333914\\
-2.93804486378162	-2.36090631348358\\
-2.94923883663002	-2.33637857752089\\
-2.95938093476062	-2.30980672078445\\
-2.96838401080529	-2.28081415192535\\
-2.97610286612645	-2.24893513973113\\
-2.98231677778854	-2.21358895431813\\
-2.98670371295673	-2.17404473597005\\
-2.98880233513725	-2.12937341183341\\
-2.9879557777621	-2.07838139629176\\
-2.98322783606726	-2.01951860794754\\
-2.97327707002324	-1.95075043755831\\
-2.95616649574139	-1.86937996465697\\
-2.92907538160684	-1.77180428933348\\
-2.88786595675086	-1.65319172469644\\
-2.8264483041122	-1.50708707923459\\
-2.73590657652846	-1.32502100773348\\
-2.60347279280227	-1.09638260105197\\
-2.41183305744973	-0.809216968254309\\
-2.1401807378859	-0.453250634669411\\
-1.76974184690201	-0.0267037097425988\\
-1.29601116923746	0.453883472638125\\
-0.742939856829561	0.94845834101123\\
-0.163493109629213	1.40443132492638\\
0.382433223343916	1.78091893864817\\
0.854599636609413	2.06434377873827\\
1.24050348687811	2.26394711433531\\
1.54663701409861	2.39844076975961\\
1.78701637795491	2.48631381412283\\
1.976156548392	2.5421269616579\\
2.12631849151321	2.57624209357592\\
2.24701642358105	2.59570022049178\\
2.34536835067793	2.60519857167455\\
2.42663064814023	2.60785811664781\\
};
\addplot [color=mycolor1, forget plot]
  table[row sep=crcr]{%
1.01008260529783	3.64344252016072\\
1.18863581493154	3.63782274867468\\
1.35795691941703	3.62169707538814\\
1.51853505039252	3.596039966336\\
1.67087992678567	3.56166356816446\\
1.81549807204968	3.51922733307875\\
1.95287527571584	3.46924908623552\\
2.08346376072164	3.41211623186198\\
2.20767273034483	3.34809628854992\\
2.32586118968549	3.27734630195263\\
2.43833213811751	3.19992093592348\\
2.54532739923415	3.11577922133519\\
2.64702249179135	3.02479007188445\\
2.74352105303856	2.92673678023746\\
2.83484841210704	2.82132080427773\\
2.92094398579676	2.70816525751865\\
3.0016522446899	2.58681864342859\\
3.07671208976739	2.45675953219305\\
3.14574460866834	2.31740307964802\\
3.20823937192019	2.16811053679873\\
3.26353971445634	2.00820319128623\\
3.31082786321983	1.83698250119168\\
3.34911135512656	1.6537584832036\\
3.37721296910294	1.45788861870122\\
3.39376736983027	1.2488295048429\\
3.39722876537074	1.02620299783167\\
3.38589494341176	0.78987740221178\\
3.35795373804042	0.540062057875569\\
3.31155776316144	0.277410246225679\\
3.24493144135332	0.00312072118151988\\
3.15651027593232	-0.280977070906215\\
3.04510563913755	-0.572373575603527\\
2.9100796338738	-0.86788681152822\\
2.75150571136334	-1.16374511301673\\
2.57028488793736	-1.45575285888894\\
2.36818821165584	-1.73953096665872\\
2.14780582530322	-2.01080537910998\\
1.91240056029732	-2.26570195682995\\
1.66568466521956	-2.50100121158749\\
1.41155505050136	-2.71431398248399\\
1.15382947508208	-2.90415714629118\\
0.896021605662616	-3.0699303069577\\
0.641179677015408	-3.2118128162806\\
0.391797123675682	-3.33061064331799\\
0.149789393445159	-3.42758371932104\\
-0.0834777132379029	-3.50427881045662\\
-0.307125970182486	-3.56238432952652\\
-0.520685487747275	-3.60361492375731\\
-0.724011280286682	-3.62962700154336\\
-0.9172043512485	-3.64196214467776\\
-1.10054218441131	-3.64201332125204\\
-1.27442062080241	-3.63100834590335\\
-1.43930716846449	-3.6100054885638\\
-1.59570468074986	-3.57989701029662\\
-1.7441238231301	-3.54141737984803\\
-1.88506261657972	-3.49515381977946\\
-2.01899143174947	-3.44155757175398\\
-2.14634199782038	-3.38095484324088\\
-2.26749921159166	-3.31355682009303\\
-2.3827947459713	-3.23946843078057\\
-2.49250164354792	-3.15869575998736\\
-2.59682923424688	-3.07115216037838\\
-2.69591783775042	-2.97666322589441\\
-2.78983280734277	-2.87497088806156\\
-2.87855755091879	-2.76573699535562\\
-2.96198523827919	-2.64854684907848\\
-3.03990898554357	-2.52291331009226\\
-3.11201041529539	-2.38828226978606\\
-3.17784664758319	-2.24404050335959\\
-3.23683601067323	-2.08952719546202\\
-3.28824310589124	-1.92405073777965\\
-3.33116435594089	-1.74691271601883\\
-3.36451584528456	-1.5574412687282\\
-3.38702614099515	-1.35503610420793\\
-3.39723783527421	-1.13922723364808\\
-3.39352266483723	-0.909748678869892\\
-3.37411599074839	-0.666626750771519\\
-3.33717673598115	-0.410279694791303\\
-3.28087795806842	-0.141621445498978\\
-3.20353036334593	0.1378428090178\\
-3.1037356896824	0.425946667307661\\
-2.98055904483278	0.71983904773537\\
-2.83370016585106	1.01602761806062\\
-2.66363574932995	1.31050293427032\\
-2.47170207179493	1.59894374716538\\
-2.2600922576455	1.87698593269571\\
-2.03175656787866	2.14051997763506\\
-1.79021401324247	2.38597142907599\\
-1.53930320431057	2.6105200170064\\
-1.28291278747925	2.81222659483883\\
-1.02473300661721	2.99005791525733\\
-0.768060490634631	3.14382020870676\\
-0.515672839772723	3.27402717373248\\
-0.269773766994071	3.3817334889122\\
-0.0319978264617681	3.46836230050207\\
0.196542153408091	3.5355475791765\\
0.415182535501067	3.58500331369573\\
0.623626184417784	3.61842374351327\\
0.821860461073582	3.63741338152024\\
1.01008260529783	3.64344252016072\\
};
\addplot [color=mycolor2, forget plot]
  table[row sep=crcr]{%
2.41476033068263	2.66229893813066\\
2.47426051877808	2.66046188142822\\
2.52496086636225	2.65566055223636\\
2.56868246992092	2.64869594138342\\
2.60680336812849	2.64011084184641\\
2.64037873090607	2.63027218455783\\
2.67022567673547	2.61942490430711\\
2.69698361012018	2.60772747837609\\
2.72115753202921	2.59527550289921\\
2.74314940113939	2.58211732487685\\
2.76328100985426	2.56826428139133\\
2.78181074564567	2.55369717031087\\
2.79894586309751	2.53836997930367\\
2.81485137716571	2.52221150387158\\
2.82965632375841	2.50512521041166\\
2.84345786519026	2.48698749574624\\
2.85632350665693	2.46764432527621\\
2.86829150595093	2.44690607167524\\
2.87936937632061	2.42454020182558\\
2.88953017480314	2.40026124729676\\
2.89870600245254	2.37371721339243\\
2.90677777235215	2.34447119331895\\
2.91355975624061	2.3119763996539\\
2.91877659097508	2.27554202075635\\
2.92202913523956	2.23428613260507\\
2.92274352947618	2.18707017433945\\
2.92009457053138	2.13240700466926\\
2.91288935500511	2.06833106433985\\
2.89938904279321	1.99221462611844\\
2.87703433081253	1.90050926551449\\
2.84202334551324	1.7883898674507\\
2.78867310206417	1.64929041000972\\
2.70849742357398	1.47437617373549\\
2.58902294051886	1.25216810687731\\
2.41272157600675	0.968961378919757\\
2.15740379551078	0.611488822489571\\
1.80110489897901	0.174007585924341\\
1.33491064908747	-0.329868142581518\\
0.780220632640025	-0.858437119498023\\
0.192569509838792	-1.35182905347143\\
-0.362109379256176	-1.76094095913193\\
-0.839173355705796	-2.06834567290477\\
-1.22537900019186	-2.28393895297461\\
-1.52855377354098	-2.42880161679264\\
-1.7643048112528	-2.52358245666273\\
-1.94827238565339	-2.58428805072776\\
-2.09334646337523	-2.62213200702853\\
-2.20933414641352	-2.6446321006958\\
-2.30345686127722	-2.65674983100523\\
-2.38097929309355	-2.66175499013238\\
-2.44574577706084	-2.66181416277486\\
-2.50058475857612	-2.65837441759815\\
-2.54759911312288	-2.65240981638321\\
-2.58837062757489	-2.6445796382357\\
-2.62410339759821	-2.63533052752175\\
-2.65572470420992	-2.62496301407602\\
-2.68395647334647	-2.61367522554978\\
-2.70936633781276	-2.60159182182315\\
-2.73240445392251	-2.58878320443428\\
-2.75343026534433	-2.57527820089079\\
-2.7727320785939	-2.5610722598607\\
-2.79054141304474	-2.54613245103991\\
-2.80704347033075	-2.53040007914688\\
-2.82238463652207	-2.51379139393238\\
-2.83667761962244	-2.49619664361787\\
-2.85000458909476	-2.47747753592046\\
-2.86241848981296	-2.45746300903009\\
-2.8739425229001	-2.43594305053212\\
-2.8845675945778	-2.41266011222196\\
-2.89424730217836	-2.38729742666149\\
-2.90288971521271	-2.35946320289597\\
-2.91034476235518	-2.32866921595343\\
-2.91638536501737	-2.29430163791519\\
-2.92067942563984	-2.25558098569283\\
-2.92274815937726	-2.2115066359581\\
-2.92190368942666	-2.16077928146949\\
-2.91715473495383	-2.10169173654886\\
-2.90706273852554	-2.03197446090965\\
-2.88952074197548	-1.94857728422476\\
-2.86141267148656	-1.84736477030459\\
-2.81809240408791	-1.72270557611584\\
-2.75260969633187	-1.56696416314431\\
-2.65464170389844	-1.37000355082238\\
-2.50927856523459	-1.11908505286392\\
-2.29642420195864	-0.800165548559755\\
-1.99294888695119	-0.402505596279578\\
-1.58128810879217	0.0715688300337349\\
-1.06580695917224	0.594675098125325\\
-0.486211284158695	1.11321633820615\\
0.0924395664717935	1.56881374821226\\
0.611840162879801	1.92719254817372\\
1.0434617323806	2.18638558471255\\
1.38643068898744	2.36383239917408\\
1.65383033355687	2.4813303299899\\
1.86188078388727	2.55739255487381\\
2.02499011504009	2.60552571919054\\
2.1544717399875	2.63494165601582\\
2.25876065635701	2.65175283414999\\
2.34402491715779	2.65998552639311\\
2.41476033068263	2.66229893813066\\
};
\addplot [color=mycolor1, forget plot]
  table[row sep=crcr]{%
1.07692127260879	3.503666468077\\
1.25300918281728	3.4981296553274\\
1.41903449901791	3.48232295184664\\
1.57559066828564	3.45731328419297\\
1.72328858336827	3.42399002181016\\
1.86273010346507	3.38307711742861\\
1.99448928975369	3.33514683294116\\
2.11909938364559	3.28063347005673\\
2.23704386587854	3.21984614712431\\
2.3487502458256	3.15298010112185\\
2.45458550625696	3.0801262906107\\
2.55485235559185	3.00127927236677\\
2.64978561667084	2.91634345581436\\
2.7395482141668	2.82513793286865\\
2.82422632106993	2.72740015821513\\
2.9038232990936	2.62278883421946\\
2.97825213075303	2.51088645081344\\
3.04732610688787	2.3912020576363\\
3.11074762068764	2.26317501621477\\
3.1680950512474	2.12618070521954\\
3.21880792679199	1.97953943919817\\
3.26217087868375	1.82253020947976\\
3.29729737848994	1.65441124709938\\
3.3231149407673	1.47444979335821\\
3.33835441273393	1.28196374684973\\
3.34154716145888	1.07637786772928\\
3.33103533186351	0.857296700760661\\
3.30500166178559	0.624594976529345\\
3.26152616108178	0.378523569503026\\
3.19867657908521	0.119824829489567\\
3.11463706764755	-0.150154690727738\\
3.00787389465825	-0.429372646421058\\
2.87732813737469	-0.715044489263038\\
2.72261396138454	-1.00366786439617\\
2.5441902657782	-1.29113798111643\\
2.34346776726496	-1.57295980618841\\
2.12281774905249	-1.84454133989462\\
1.88546479782922	-2.1015292855393\\
1.63527087347818	-2.3401328052854\\
1.37644393510446	-2.55738011708881\\
1.11322143137332	-2.7512680245816\\
0.84958107671147	-2.92079049449001\\
0.589018967382472	-3.06585923936307\\
0.334414531303806	-3.18714802934969\\
0.0879812010941502	-3.28589915315193\\
-0.148712951538626	-3.36372648548615\\
-0.374677009102123	-3.42243953622834\\
-0.589403634093858	-3.46390144626262\\
-0.792768060091346	-3.48992439196961\\
-0.984931880726181	-3.50219956417983\\
-1.16625854347781	-3.50225568023596\\
-1.33724335614839	-3.49143907338669\\
-1.49845810192408	-3.47090887109396\\
-1.65050882723953	-3.44164188608603\\
-1.7940046862125	-3.40444311702556\\
-1.92953558294803	-3.35995892643898\\
-2.05765650621834	-3.30869092286519\\
-2.17887673587479	-3.25100930325602\\
-2.29365241756244	-3.18716493616498\\
-2.40238129872309	-3.11729982803371\\
-2.50539867091555	-3.04145585700177\\
-2.60297376494856	-2.95958181897214\\
-2.69530599936835	-2.87153894005791\\
-2.7825205972781	-2.77710509254552\\
-2.86466317138413	-2.67597802777069\\
-2.94169294415755	-2.56777802516605\\
-3.01347433245063	-2.45205046662173\\
-3.07976670000611	-2.32826899249109\\
-3.14021218759816	-2.1958400922185\\
-3.19432169577425	-2.05411023834092\\
-3.24145935384523	-1.90237699139921\\
-3.28082620369488	-1.73990587678817\\
-3.31144440682784	-1.5659552316216\\
-3.33214409411165	-1.37981157025044\\
-3.34155604505053	-1.18083819109787\\
-3.33811467655641	-0.968539532429108\\
-3.32007719888221	-0.74264287264097\\
-3.28556593487212	-0.503196977428685\\
-3.23264112331153	-0.250683848211803\\
-3.15941020074127	0.0138653435980458\\
-3.06417561474539	0.28876538350546\\
-2.94561596149961	0.571604778973339\\
-2.80298488238984	0.859231691711481\\
-2.63630059119198	1.14782198589606\\
-2.44648999609513	1.43304350653639\\
-2.23545010493017	1.71031228038472\\
-2.00599954803418	1.97511307778672\\
-1.76171434840376	2.22333637142549\\
-1.50666874172212	2.45157480806186\\
-1.24512443377309	2.65732977749459\\
-0.981221667747762	2.83910042158708\\
-0.718719854451677	2.99635515654093\\
-0.46081804338262	3.12940944708454\\
-0.21006397215261	3.23924643367439\\
0.0316577094427846	3.32731793722812\\
0.263078101244961	3.39535570033236\\
0.483460592181638	3.44521145172643\\
0.692501675998655	3.47873364607262\\
0.890231168036171	3.49768077510133\\
1.07692127260879	3.503666468077\\
};
\addplot [color=mycolor2, forget plot]
  table[row sep=crcr]{%
2.39586003108064	2.70645250921799\\
2.44793921405715	2.7048434128308\\
2.4925056321078	2.7006219481352\\
2.53110320230635	2.69447266868954\\
2.56490106181823	2.68686034672239\\
2.59479573023351	2.67809952494233\\
2.62148280844775	2.66839996721084\\
2.64550781899207	2.65789663089041\\
2.66730263361219	2.64666953175667\\
2.68721182411929	2.63475687548339\\
2.70551186863031	2.6221635898773\\
2.72242520658815	2.6088666094499\\
2.73813050107971	2.59481775820078\\
2.75277002945031	2.57994473802309\\
2.76645481248171	2.56415049078378\\
2.77926786061602	2.54731101771482\\
2.79126572870291	2.52927157937977\\
2.80247840160814	2.50984103858069\\
2.81290735701722	2.48878392383097\\
2.82252144152823	2.46580955589587\\
2.83124991591435	2.44055725967451\\
2.83897162304683	2.41257622901504\\
2.84549862605467	2.38129794978588\\
2.85055172359163	2.34599810652856\\
2.85372375788986	2.30574343486008\\
2.85442423280694	2.25931679290446\\
2.85179486635119	2.20511047637028\\
2.84457937343576	2.140973103094\\
2.83092059535225	2.06398901685406\\
2.80804228925792	1.97016184831735\\
2.77175053613937	1.85396984851581\\
2.71566587753892	1.70777473990774\\
2.63010038706129	1.52114193105368\\
2.50062498549661	1.28037916993507\\
2.30690789344056	0.969241119782617\\
2.02384246864085	0.572942103193158\\
1.62923691139703	0.0883858147587356\\
1.12147624928202	-0.460565639479011\\
0.537853457168564	-1.0169569262796\\
-0.0519105584188836	-1.51239220851406\\
-0.581938954267458	-1.90352395875888\\
-1.01946440750957	-2.1855663139474\\
-1.36351965659421	-2.37768436448737\\
-1.62880917406411	-2.50446722807857\\
-1.83313334167097	-2.5866211214845\\
-1.99194137580707	-2.63902632364079\\
-2.11711539293686	-2.67167858606987\\
-2.21735901291246	-2.69112320755616\\
-2.29894325902303	-2.70162521662839\\
-2.36638328199093	-2.70597802925916\\
-2.42295348268433	-2.70602847514658\\
-2.47105477977271	-2.70301025935254\\
-2.51247004307102	-2.69775504555296\\
-2.5485406354134	-2.69082683301651\\
-2.5802887900039	-2.68260834336394\\
-2.608503090363	-2.6733571722273\\
-2.6337987661433	-2.66324266322134\\
-2.65666067324506	-2.65237030238575\\
-2.67747424204231	-2.64079788561504\\
-2.6965479567738	-2.6285461408226\\
-2.71412978233766	-2.61560550394926\\
-2.73041918415139	-2.60194012103131\\
-2.74557586123964	-2.58748973704813\\
-2.75972594573443	-2.57216985006503\\
-2.77296615566563	-2.555870301796\\
-2.78536618221272	-2.53845230665008\\
-2.7969694176343	-2.51974376357569\\
-2.80779196065837	-2.49953252520825\\
-2.81781964678537	-2.47755709259567\\
-2.82700261099537	-2.453493931178\\
-2.83524655702526	-2.42694022364257\\
-2.84239941590435	-2.397390327888\\
-2.84823132379159	-2.36420340393172\\
-2.85240466733144	-2.32655847699991\\
-2.85442905636686	-2.28339141396671\\
-2.85359303053655	-2.23330561876357\\
-2.84885934088247	-2.17444432624126\\
-2.83870260145156	-2.10430683505434\\
-2.82085533591204	-2.01948395812656\\
-2.79190930965565	-1.91528136423395\\
-2.74669449271135	-1.78520149587765\\
-2.67734125288136	-1.62029135042173\\
-2.57197791015913	-1.40850591963233\\
-2.41329867603964	-1.13464870014876\\
-2.17815769848986	-0.78237514111604\\
-1.84133030932303	-0.341010065890242\\
-1.38819237899538	0.180920934736513\\
-0.835208929213798	0.742291470905776\\
-0.239044017925271	1.27593051799716\\
0.32733697181048	1.72210942956557\\
0.812891499624408	2.05729133431892\\
1.20243775255801	2.29129848572901\\
1.50487719019694	2.44781159112029\\
1.73754441727249	2.55006142206413\\
1.91740163105295	2.61582044512544\\
2.05812274414373	2.65734719705302\\
2.16991322967182	2.68274295031821\\
2.26016704717165	2.69729025276353\\
2.33420250887109	2.70443730605496\\
2.39586003108064	2.70645250921799\\
};
\addplot [color=mycolor1, forget plot]
  table[row sep=crcr]{%
1.15864607231793	3.36165970522797\\
1.33225134403357	3.35620747412833\\
1.49478440075304	3.34073934225922\\
1.64698656841843	3.3164308911847\\
1.78960780066664	3.28425830398293\\
1.9233772092579	3.24501456805803\\
2.04898332032296	3.19932723382943\\
2.16706140026213	3.14767580996821\\
2.27818568495898	3.09040768498839\\
2.38286481120998	3.02775201263799\\
2.48153914528356	2.95983135156412\\
2.57457901793167	2.88667106749531\\
2.662283112096	2.80820663612774\\
2.74487642026642	2.72428906298787\\
2.82250730652801	2.63468868961239\\
2.89524328755055	2.53909770391505\\
2.96306520076115	2.43713173325252\\
3.0258594709254	2.32833098635054\\
3.08340823466972	2.21216153951883\\
3.1353771566508	2.08801754849827\\
3.18130089851176	1.95522542486019\\
3.22056641959446	1.81305135780606\\
3.25239464595539	1.66071399346844\\
3.27582160359392	1.49740459137472\\
3.28968094303244	1.32231751320041\\
3.29259094842626	1.13469435216573\\
3.28295064710913	0.933885175825142\\
3.25895143343967	0.719429892043736\\
3.21861241279621	0.491161172627005\\
3.15984885736914	0.249327100252536\\
3.08058273494598	-0.00527375221765975\\
2.97890085895925	-0.271159770129723\\
2.85325846718984	-0.546060096078085\\
2.70271349409731	-0.826866041703794\\
2.52716107812122	-1.10967315780278\\
2.32752360217438	-1.38993850604152\\
2.10584598766703	-1.66275658210955\\
1.86525579730255	-1.92322686588386\\
1.60977454229429	-2.16685595700952\\
1.34400348706045	-2.38992078133626\\
1.07274033607123	-2.58972555591923\\
0.800598780380851	-2.76471295661684\\
0.531695425490565	-2.91442778511639\\
0.269443497665235	-3.03936427842029\\
0.0164620121028157	-3.14074562674795\\
-0.225415782718549	-3.22028458620438\\
-0.455064137494796	-3.27996256296887\\
-0.671948205727318	-3.32184873132998\\
-0.875995320702246	-3.34796651731514\\
-1.0674721369507	-3.36020494960387\\
-1.24687767699094	-3.36026720629691\\
-1.41485619350969	-3.34964708085678\\
-1.57212977585483	-3.3296246480048\\
-1.71944847164187	-3.30127396546065\\
-1.85755482158101	-3.26547743932403\\
-1.98715961416919	-3.22294310732679\\
-2.10892597843005	-3.17422239821988\\
-2.22345940187905	-3.11972688940498\\
-2.33130174685321	-3.05974325486975\\
-2.43292777207422	-2.99444603774209\\
-2.52874302175775	-2.92390816047626\\
-2.61908221912585	-2.84810925437041\\
-2.70420750336871	-2.76694199026823\\
-2.78430599176549	-2.6802166549814\\
-2.85948624561622	-2.58766426650185\\
-2.92977328352482	-2.48893857372896\\
-2.99510183219263	-2.38361735876521\\
-3.05530754827947	-2.27120356644535\\
-3.11011600320062	-2.1511269414496\\
-3.15912931956567	-2.02274707328312\\
-3.20181051523691	-1.88535904787861\\
-3.23746589169837	-1.73820329135964\\
-3.26522625343941	-1.580481664241\\
-3.28402843150976	-1.41138239503501\\
-3.29259957665535	-1.23011695549039\\
-3.2894480336971	-1.03597232183191\\
-3.27286628917489	-0.828381968701454\\
-3.24095333148943	-0.607017992352561\\
-3.19166535112224	-0.371904411158706\\
-3.12290423878371	-0.123547372835805\\
-3.0326516010854	0.136928612253899\\
-2.91915057597159	0.407657775624189\\
-2.78112754703252	0.685951486808953\\
-2.61803134182142	0.968291107210478\\
-2.43025173714252	1.25042562042498\\
-2.21926821359736	1.52758956887821\\
-1.98768139377993	1.79483020803747\\
-1.73909823593737	2.04740097255614\\
-1.47787533741939	2.28115352612145\\
-1.20876162797556	2.49285493698564\\
-0.936507428543648	2.68037437802348\\
-0.665510906127175	2.84271842506794\\
-0.39955540906315	2.97993113850398\\
-0.141661611193392	3.09290102385607\\
0.105950536635203	3.18312551744094\\
0.341814414927393	3.25247713342869\\
0.565115580825522	3.303000844366\\
0.775566323367475	3.33675669571624\\
0.973278152500676	3.35570945870127\\
1.15864607231793	3.36165970522797\\
};
\addplot [color=mycolor2, forget plot]
  table[row sep=crcr]{%
2.37097245534212	2.7420316362117\\
2.41656189004161	2.74062210000145\\
2.45573036679667	2.73691111095205\\
2.48978854289773	2.73148430336595\\
2.5197299463119	2.72473991858471\\
2.54631766025098	2.71694762527161\\
2.57014492375801	2.70828693151035\\
2.59167800844056	2.69887250480756\\
2.61128690506718	2.68877093192561\\
2.6292675082278	2.67801175474508\\
2.64585777686604	2.66659456961682\\
2.66124954648495	2.65449331573003\\
2.67559712901856	2.64165844829721\\
2.68902346449365	2.62801740138216\\
2.70162432302637	2.61347353472187\\
2.71347085468897	2.59790359057999\\
2.72461061815595	2.58115353271565\\
2.73506706218434	2.56303247644746\\
2.7448372630099	2.54330422245611\\
2.7538875075488	2.52167564799242\\
2.76214601829786	2.49778084702536\\
2.76949168257558	2.47115938582597\\
2.77573698267025	2.4412262613363\\
2.78060227007522	2.40722997517465\\
2.7836768267455	2.36819334751416\\
2.7843593695785	2.32282895919441\\
2.78176603826495	2.26941694331938\\
2.77458623091866	2.20562662620142\\
2.76085399122843	2.12825471098869\\
2.73758242954271	2.03284186876362\\
2.70017911364511	1.91312183239206\\
2.64152756712036	1.76027240469776\\
2.5506233526731	1.56203952020427\\
2.41084116723899	1.30216273183048\\
2.19869982976661	0.961482342718005\\
1.88609986295266	0.523853307612389\\
1.45188103584108	-0.00941184988821951\\
0.905011412639907	-0.600841322450176\\
0.300984103015729	-1.17697437633149\\
-0.279639614986057	-1.66500002627316\\
-0.777141108438632	-2.03230438061452\\
-1.17291841174885	-2.28752439101081\\
-1.4766410052214	-2.45715990617252\\
-1.70755387179492	-2.56752878073292\\
-1.88417489769587	-2.63854841219933\\
-2.02113006664282	-2.68374289252458\\
-2.12912778760137	-2.71191375145814\\
-2.21579746346988	-2.72872411662436\\
-2.28654930912442	-2.73783043742404\\
-2.34524379108104	-2.74161762034185\\
-2.39466718845133	-2.74166067136464\\
-2.43685815704226	-2.73901242010449\\
-2.4733298827699	-2.73438371425431\\
-2.50522156088711	-2.72825746861326\\
-2.53340252940198	-2.7209617754234\\
-2.55854463909447	-2.71271737191783\\
-2.58117315185587	-2.70366879825636\\
-2.60170296423541	-2.69390499614592\\
-2.62046466944398	-2.68347292916575\\
-2.637723477896	-2.67238647518524\\
-2.65369303268425	-2.66063201051763\\
-2.66854549995222	-2.64817157435036\\
-2.68241886767288	-2.63494415067055\\
-2.69542207384251	-2.62086535994409\\
-2.70763835613721	-2.60582566737345\\
-2.71912703456596	-2.58968705650954\\
-2.72992377983627	-2.57227796133157\\
-2.74003925935967	-2.55338607341383\\
-2.74945586473644	-2.5327484173563\\
-2.75812197675783	-2.51003778313961\\
-2.76594286938916	-2.48484416956563\\
-2.77276681926595	-2.4566492548512\\
-2.77836415205913	-2.42479095535871\\
-2.78239562171843	-2.38841368543216\\
-2.78436434544	-2.34639771944883\\
-2.78354193410604	-2.297257676396\\
-2.77885350496149	-2.23899503538068\\
-2.76869638454963	-2.16888211395647\\
-2.75065120711101	-2.08314489972321\\
-2.72101924431481	-1.976501637464\\
-2.67408675857107	-1.8415136263038\\
-2.60099339294028	-1.66775062116819\\
-2.48814654562482	-1.44097153743184\\
-2.31553640753194	-1.14312435524285\\
-2.0566699437989	-0.755346434435791\\
-1.68465037679736	-0.267851466002901\\
-1.19012994655386	0.301873451191688\\
-0.60514456145037	0.895985809499205\\
-0.00307335409498148	1.43520501048875\\
0.540852901816076	1.86392021265926\\
0.987553308803798	2.1724095677462\\
1.3351415266425	2.38127196158624\\
1.59996349938777	2.51834233826423\\
1.80163750250165	2.60698012197524\\
1.95685829931324	2.66373366513709\\
2.07821045594123	2.69954427228426\\
2.17474736069969	2.72147367820844\\
2.25289193509988	2.73406788805195\\
2.31720883710598	2.74027552804658\\
2.37097245534212	2.7420316362117\\
};
\addplot [color=mycolor1, forget plot]
  table[row sep=crcr]{%
1.25819306153321	3.22316713074098\\
1.42932448156702	3.21780042499708\\
1.58817579819832	3.2026898549361\\
1.73569527099036	3.17913584564961\\
1.87282081932095	3.14820898628053\\
2.00044794120658	3.11077272316979\\
2.11940999752583	3.06750715761604\\
2.23046714131272	3.01893166092257\\
2.33430099158643	2.96542509142497\\
2.43151287638883	2.90724308576105\\
2.52262405474737	2.84453231054009\\
2.60807677090703	2.77734179199042\\
2.68823531434906	2.70563155834315\\
2.76338647949024	2.62927888099351\\
2.83373896250298	2.54808241930898\\
2.89942131942143	2.46176458372459\\
2.96047815656954	2.36997244992145\\
3.01686424639106	2.2722775981427\\
3.06843627388874	2.16817533053999\\
3.11494193831615	2.05708385197457\\
3.15600618418592	1.93834420480503\\
3.19111444712788	1.81122204761352\\
3.21959301982294	1.67491278418725\\
3.24058703633076	1.52855210026402\\
3.2530372271061	1.37123465250037\\
3.25565761918672	1.20204443800585\\
3.24691785729109	1.02010113697681\\
3.22503587243533	0.824627215083776\\
3.18798916524724	0.615040344528044\\
3.13355565748483	0.391074048366338\\
3.05939705864913	0.152925467254844\\
2.96319746704448	-0.0985781135962522\\
2.84286528066663	-0.361813147851201\\
2.69679515327729	-0.634226331552334\\
2.52416781458882	-0.912277211147875\\
2.32524184475379	-1.19150302866321\\
2.10157109850985	-1.46673833034004\\
1.85607672581616	-1.73248861197861\\
1.5929240733955	-1.98341096065366\\
1.31720186022244	-2.21481305199938\\
1.0344580862182	-2.42306622844925\\
0.750189036321172	-2.60584971916751\\
0.469385229877587	-2.76219307362953\\
0.196210292324188	-2.89233948726711\\
-0.0661579410990571	-2.99749091688168\\
-0.315538485073338	-3.07950661708148\\
-0.550651292773326	-3.14061457602759\\
-0.770962052237112	-3.18317238106405\\
-0.976508330981463	-3.20949148705748\\
-1.1677347744734	-3.22172275137178\\
-1.34535225314538	-3.22179255714557\\
-1.51022612358149	-3.21137621559538\\
-1.66329270619869	-3.19189623469262\\
-1.8055000413986	-3.16453548237547\\
-1.93776797815316	-3.13025800138574\\
-2.06096279273744	-3.08983262586188\\
-2.17588220806786	-3.0438563962646\\
-2.28324750945659	-2.9927760767148\\
-2.38370023495592	-2.93690694400996\\
-2.47780157576205	-2.87644855549136\\
-2.56603313501092	-2.81149751565592\\
-2.6487980723826	-2.74205742852956\\
-2.72642192920186	-2.6680463020594\\
-2.79915260797772	-2.58930170239171\\
-2.86715909316413	-2.50558396773426\\
-2.93052856475804	-2.41657780352222\\
-2.98926158898672	-2.32189260853916\\
-3.04326508547837	-2.22106193981792\\
-3.09234278368039	-2.11354262766093\\
-3.13618291239866	-1.99871421838422\\
-3.17434294200452	-1.87587967180223\\
-3.20623135758636	-1.74426859563808\\
-3.23108673821157	-1.60304478125932\\
-3.24795492865539	-1.45132042588722\\
-3.2556659137754	-1.28818017046075\\
-3.25281325462513	-1.11271887779823\\
-3.2377407180028	-0.924097747129222\\
-3.2085430476891	-0.721623565866524\\
-3.16309050763878	-0.504855062760248\\
-3.09908931177837	-0.273737620112226\\
-3.01419117581832	-0.0287620629975018\\
-2.90616309346768	0.228865828376097\\
-2.77312069140666	0.497070671546244\\
-2.61381332780608	0.772806139286641\\
-2.42792711446531	1.05205331312857\\
-2.21634862057458	1.32996047962264\\
-1.9813178241369	1.60114259875737\\
-1.72640633608714	1.86011730169536\\
-1.45629215031779	2.10180758900895\\
-1.1763571087301	2.32201043807557\\
-0.892185922484802	2.5177331366009\\
-0.609071712238378	2.68733707063445\\
-0.331621664611324	2.83048477424765\\
-0.0635163824766542	2.94793521683809\\
0.192571151958296	3.04125702685223\\
0.434927199860419	3.11252737579308\\
0.662665959751348	3.16406508076278\\
0.875560713711963	3.19822261541064\\
1.07387188834348	3.21724197812296\\
1.25819306153321	3.22316713074098\\
};
\addplot [color=mycolor2, forget plot]
  table[row sep=crcr]{%
2.34062091623357	2.77043391311226\\
2.38048810855438	2.76920049765774\\
2.41487063113053	2.76594225496986\\
2.44488053164076	2.76115988508774\\
2.47136191107579	2.75519433348111\\
2.49496423907643	2.74827649567593\\
2.51619347442919	2.74055966732415\\
2.53544818577876	2.73214092589803\\
2.55304538806354	2.72307526061283\\
2.56923921638794	2.71338483192463\\
2.58423452433188	2.70306485611385\\
2.59819681231562	2.69208705049761\\
2.61125943422324	2.680401208964\\
2.62352871490474	2.66793522629686\\
2.63508738407729	2.65459370329513\\
2.64599655732029	2.64025510948017\\
2.65629634560386	2.62476733086357\\
2.66600502845203	2.60794126391278\\
2.6751165591453	2.58954190826651\\
2.68359595434742	2.56927612719677\\
2.69137181364272	2.54677583926005\\
2.69832475175538	2.52157480550302\\
2.70426980120371	2.49307627195009\\
2.70892967619744	2.46050733885834\\
2.71189387130658	2.42285377253979\\
2.71255536591029	2.37876560454941\\
2.71001128895158	2.32641859378053\\
2.70290467892945	2.26330850689729\\
2.68916885789367	2.18594318845285\\
2.66561022175146	2.08938165487635\\
2.62722628319647	1.966555716939\\
2.56611036138608	1.80732423281948\\
2.46979588526204	1.59734325436237\\
2.3191579275618	1.31734250648625\\
2.08714632199668	0.944806770699193\\
1.74268782805652	0.462587824528167\\
1.26752298933528	-0.121065315080577\\
0.685174320015884	-0.751117827291476\\
0.0705983045612169	-1.33763520231911\\
-0.490095687381608	-1.80916846000086\\
-0.949033123034393	-2.14814942489748\\
-1.30235718224383	-2.37606131082363\\
-1.56808187960351	-2.52450123376009\\
-1.7679299976156	-2.62003143352293\\
-1.92007065852317	-2.68120975165639\\
-2.03792748199427	-2.72010163125482\\
-2.13097838217791	-2.74437261532731\\
-2.20583863974892	-2.75889124251424\\
-2.26714487892474	-2.76678071574231\\
-2.31818449186105	-2.7700729861553\\
-2.36132268812799	-2.77010970246252\\
-2.3982881179035	-2.7677887039666\\
-2.43036411232187	-2.76371721875338\\
-2.45851781959879	-2.75830844297451\\
-2.48348845348508	-2.75184334400764\\
-2.50584840985986	-2.7445107643716\\
-2.52604616859021	-2.73643374038363\\
-2.54443679848662	-2.72768688696962\\
-2.56130389714171	-2.71830785975126\\
-2.57687551520985	-2.70830478091108\\
-2.59133577674062	-2.69766081344432\\
-2.60483335046394	-2.68633661801923\\
-2.61748754855663	-2.67427112582108\\
-2.62939256346133	-2.66138084688549\\
-2.64062015615187	-2.64755776611475\\
-2.65122095017126	-2.63266572980567\\
-2.66122434073354	-2.61653507058767\\
-2.67063687479661	-2.59895503469838\\
-2.67943877086722	-2.57966333424822\\
-2.68757799217844	-2.55833180968797\\
-2.69496091224232	-2.53454669638471\\
-2.70143803483985	-2.50778125474477\\
-2.70678231334399	-2.47735740461383\\
-2.71065612191777	-2.44239127686218\\
-2.71256045774346	-2.4017149007247\\
-2.71175579375927	-2.35376202792122\\
-2.70713693658239	-2.29639953334494\\
-2.69703222541958	-2.22667588514445\\
-2.67887725147416	-2.14044412457706\\
-2.64868109923095	-2.0318006139283\\
-2.60015857138027	-1.89227550575164\\
-2.52336726231127	-1.70976638353818\\
-2.40277592972888	-1.46747869035947\\
-2.21528573365125	-1.14401469626111\\
-1.93073081033608	-0.717795814247614\\
-1.52128427408323	-0.181220092662584\\
-0.986029825337172	0.435612739739464\\
-0.376064190330708	1.0554004523323\\
0.220679833419737	1.59015249851199\\
0.733329590066657	1.99441882961407\\
1.1380104421674	2.27399244198054\\
1.44477007677968	2.45836496946707\\
1.67499107537872	2.57754226425367\\
1.84901930754608	2.65403439145904\\
1.98261292073439	2.7028810520088\\
2.08708479775557	2.73370957408162\\
2.17035470254002	2.75262407212757\\
2.23795475939865	2.76351769633521\\
2.29378247987082	2.76890493021635\\
2.34062091623356	2.77043391311226\\
};
\addplot [color=mycolor1, forget plot]
  table[row sep=crcr]{%
1.3803158105586	3.09362620152166\\
1.54893729354649	3.08834757461492\\
1.70385067175428	3.0736199824606\\
1.84629109518523	3.05088443207606\\
1.97744700638335	3.02131067122778\\
2.09842739246933	2.98583011929795\\
2.21024459821208	2.94516856333983\\
2.31380737873498	2.89987603999062\\
2.40992028466746	2.85035273838091\\
2.499286622764	2.79687059829109\\
2.58251310245332	2.7395907336281\\
2.6601148985556	2.67857702789487\\
2.73252028326624	2.61380632502745\\
2.8000742552139	2.545175640604\\
2.86304076030295	2.47250678588776\\
2.92160318910353	2.39554875609676\\
2.97586287170976	2.31397820155143\\
3.02583529045299	2.22739828867977\\
3.07144370765178	2.13533627934926\\
3.1125098733779	2.03724022582315\\
3.14874145372062	1.93247531338628\\
3.17971582729964	1.82032060830986\\
3.20485997404229	1.69996731790985\\
3.22342638361572	1.57052018167982\\
3.23446532983979	1.43100432997874\\
3.23679462022262	1.28038090100622\\
3.22896920905005	1.11757588981802\\
3.20925506229208	0.941528012539268\\
3.1756145649968	0.751262515788463\\
3.12571459433532	0.545998238420895\\
3.05697275890688	0.325293773351387\\
2.96666102541377	0.0892337103734397\\
2.85208654475064	-0.161354125139463\\
2.71086306896909	-0.42467583761746\\
2.54126855163561	-0.69778941448721\\
2.34265335777059	-0.976528947597679\\
2.11582407343387	-1.25560544923608\\
1.86329611123416	-1.52893099931784\\
1.58930778114802	-1.79015539430015\\
1.29953650539806	-2.033327807896\\
1.00054660757384	-2.25353593622805\\
0.699089460599829	-2.44736860510669\\
0.401423196147943	-2.61310444429223\\
0.112797733141857	-2.75062047786383\\
-0.162820833709833	-2.86109388236568\\
-0.422792905911817	-2.94660581462791\\
-0.66567587136833	-3.00974651590204\\
-0.891000432494973	-3.05328563427414\\
-1.09902013707136	-3.07993341126121\\
-1.29047876097123	-3.09219047278403\\
-1.46641776589206	-3.09226950456981\\
-1.6280297223777	-3.08206810833061\\
-1.77655415559569	-3.06317411338522\\
-1.91320817811311	-3.03688892090978\\
-2.03914348297347	-3.00425893086493\\
-2.15542213748432	-2.96610879480698\\
-2.26300508755094	-2.92307292285301\\
-2.36274878631352	-2.87562345824462\\
-2.45540665108207	-2.82409403305481\\
-2.54163305914944	-2.76869924496107\\
-2.62198833070582	-2.70955011621246\\
-2.69694366193025	-2.64666593257216\\
-2.76688531448405	-2.57998289280504\\
-2.83211758399588	-2.50935997984026\\
-2.8928641952833	-2.43458242540496\\
-2.94926783270839	-2.35536310126249\\
-3.00138752998868	-2.27134214650044\\
-3.04919363025445	-2.18208514363919\\
-3.09255999762963	-2.08708019949427\\
-3.13125313083228	-1.98573438583329\\
-3.16491781700242	-1.87737017147393\\
-3.19305899997689	-1.76122276026232\\
-3.21501966812754	-1.63643967473304\\
-3.22995486468682	-1.50208453506649\\
-3.23680249553661	-1.35714781644299\\
-3.23425260632015	-1.20056844175191\\
-3.22071841420947	-1.03127133462783\\
-3.19431481332988	-0.848227338451973\\
-3.15285345483355	-0.650542770593227\\
-3.09386769577832	-0.437585506196712\\
-3.01468497997496	-0.209151545640447\\
-2.91256677147678	0.0343312360156933\\
-2.78493380284614	0.291578628037772\\
-2.62968278173373	0.560240488330013\\
-2.44557614400683	0.836762832348514\\
-2.23264986763644	1.11639222886098\\
-1.99254607838426	1.3933852947127\\
-1.72865801087491	1.66144513354078\\
-1.44599765322294	1.91433606894612\\
-1.15076782531539	2.14655445462601\\
-0.849716631807568	2.35389657663218\\
-0.54942625744986	2.53379103644462\\
-0.255700867017199	2.68534239882346\\
0.0268326092836768	2.8091237895267\\
0.294878551857073	2.90681579248229\\
0.546419730517061	2.9808004118877\\
0.780531158675736	3.0337933296798\\
0.997135915631996	3.06855852011688\\
1.19675952955286	3.08771522176552\\
1.3803158105586	3.09362620152166\\
};
\addplot [color=mycolor2, forget plot]
  table[row sep=crcr]{%
2.30478646398007	2.79272230997285\\
2.33956694171741	2.79164558255829\\
2.36967416089367	2.78879189377679\\
2.39604915169522	2.78458826116053\\
2.41940738784067	2.77932580505079\\
2.44030047088818	2.773201622012\\
2.45915907372101	2.76634612415256\\
2.47632326503792	2.75884104672134\\
2.49206419846869	2.75073132446467\\
2.50659979070406	2.74203282882589\\
2.52010613681443	2.73273721255329\\
2.53272583618772	2.72281463555961\\
2.54457401700489	2.71221483445628\\
2.55574258032479	2.70086678072266\\
2.56630299095393	2.68867700627798\\
2.57630779005883	2.67552653084561\\
2.58579087030686	2.66126617950051\\
2.59476641722134	2.64570990850852\\
2.60322625758609	2.62862553629016\\
2.6111351372104	2.60972196762729\\
2.61842313156106	2.58863154881195\\
2.62497390203963	2.56488551457264\\
2.63060672793107	2.53787944840306\\
2.63504896247673	2.50682405511139\\
2.63789341837391	2.47067397545903\\
2.63853153897078	2.4280232661205\\
2.63604690669952	2.37694958087758\\
2.62904264638929	2.31477861271425\\
2.61535713588221	2.23772424957448\\
2.59158987092791	2.14033740317269\\
2.55230801849377	2.01467364838278\\
2.48874098609742	1.8491005504418\\
2.38676701709457	1.626837450619\\
2.22436620901246	1.32503913598231\\
1.97040697847379	0.917321547969514\\
1.59106365529649	0.386261610312481\\
1.07379118803747	-0.249267906989404\\
0.461322633711947	-0.912228345945979\\
-0.152439094359707	-1.49831803336745\\
-0.682985972155761	-1.94473859744379\\
-1.09888941523765	-2.252054648784\\
-1.41006759865393	-2.4528303265095\\
-1.64031122406817	-2.58146771345832\\
-1.81210247322955	-2.66359170219609\\
-1.9425173390933	-2.71603447836118\\
-2.04356549189356	-2.74937887236392\\
-2.12349951943009	-2.77022737719967\\
-2.18799236562903	-2.78273419074224\\
-2.24098691524362	-2.78955303187048\\
-2.28526666845532	-2.79240839789825\\
-2.32283069563633	-2.79243963015844\\
-2.35513972102892	-2.79041036313633\\
-2.38327901438151	-2.78683801838073\\
-2.40806762524164	-2.78207523978057\\
-2.43013272765714	-2.77636196288903\\
-2.44996099328966	-2.76985920138283\\
-2.46793461658927	-2.76267121917085\\
-2.48435692268607	-2.75486016166862\\
-2.49947078649536	-2.74645566828574\\
-2.51347200233559	-2.73746104186232\\
-2.52651903542685	-2.72785695947072\\
-2.53874011744027	-2.71760332751641\\
-2.5502383290549	-2.70663962616922\\
-2.5610950867276	-2.69488390063282\\
-2.57137228112513	-2.68223040464321\\
-2.58111317415563	-2.66854575916828\\
-2.59034202852945	-2.65366333410884\\
-2.59906229695985	-2.63737536881697\\
-2.60725301186582	-2.61942208769188\\
-2.61486275467235	-2.59947669586444\\
-2.62180019037119	-2.5771245895206\\
-2.62791953552688	-2.55183427868937\\
-2.63299832922247	-2.52291622416527\\
-2.63670322317505	-2.48946375149657\\
-2.63853671651824	-2.45026695988795\\
-2.63775297178264	-2.40368534204549\\
-2.63322252863105	-2.34745650548621\\
-2.62321120715757	-2.27840530799946\\
-2.60501339474685	-2.1919983213213\\
-2.57433840211125	-2.08166418718521\\
-2.52428849911235	-1.93778670175437\\
-2.44371639744693	-1.74634206498435\\
-2.31486598594063	-1.48752300588703\\
-2.11105659036845	-1.1359716084557\\
-1.79811766488291	-0.667275034316672\\
-1.34858501413813	-0.0780971683956526\\
-0.774238049795059	0.584032329267382\\
-0.148286129954118	1.22042159931633\\
0.43138878657625	1.74018191686149\\
0.905258255977768	2.11404264728545\\
1.26615647323833	2.36344830444132\\
1.53378736915116	2.52433371153796\\
1.7323137872215	2.62711401014371\\
1.88162743328096	2.69274559445242\\
1.996123787551	2.734609393255\\
2.08576827798602	2.76106149465897\\
2.15739694930724	2.77733056834636\\
2.21573094769686	2.78672991307819\\
2.26407619548579	2.79139417326984\\
2.30478646398007	2.79272230997285\\
};
\addplot [color=mycolor1, forget plot]
  table[row sep=crcr]{%
1.53253678575503	2.97922984659636\\
1.69848220691898	2.97404617197318\\
1.84905338789904	2.95974112755087\\
1.98587942337696	2.93791019857066\\
2.11047743894977	2.90982256838876\\
2.22422441894082	2.87646992265854\\
2.32834778659313	2.83861196188184\\
2.42392717619353	2.79681609187841\\
2.5119022713053	2.75149055169265\\
2.59308338133733	2.70291117802793\\
2.66816267662567	2.65124242894247\\
2.73772482749179	2.59655342193481\\
2.80225631409169	2.53882972272346\\
2.86215298532536	2.4779815333501\\
2.91772561499132	2.41384881780184\\
2.96920327876322	2.34620379492374\\
3.01673438752842	2.2747511352303\\
3.06038518124038	2.19912612837183\\
3.10013542583642	2.11889104851962\\
3.13587097367162	2.0335299450084\\
3.16737275505032	1.94244213940555\\
3.19430167865837	1.84493483995977\\
3.21617885567919	1.74021552379761\\
3.23236056803254	1.62738513522178\\
3.24200754825425	1.50543377229184\\
3.24404854847912	1.37324146874722\\
3.23713904166488	1.22958801733666\\
3.21961750629181	1.07317759260802\\
3.18946448203592	0.902686185291078\\
3.14427388092552	0.716842275389181\\
3.0812521866997	0.514552958629766\\
2.99726887340254	0.295087250864118\\
2.88898890014322	0.0583227105842752\\
2.75312121191095	-0.194952944280957\\
2.5868082044987	-0.462720065356516\\
2.38815062632492	-0.741457191260336\\
2.15680525510023	-1.02603128135291\\
1.89451953418333	-1.30986628440396\\
1.60541388803629	-1.58546199907443\\
1.29584000417588	-1.84522328545909\\
0.97376055486984	-2.08242099143707\\
0.647775025750236	-2.29202113725685\\
0.326058846345277	-2.47115338695656\\
0.0155010949859296	-2.61913164993093\\
-0.278783070476973	-2.73710304913087\\
-0.55354668355016	-2.82749834587453\\
-0.807211255219125	-2.89345983110686\\
-1.03951336708477	-2.93836411768581\\
-1.25111782291838	-2.96548646552457\\
-1.44327236002039	-2.97780172214899\\
-1.61753615327626	-2.97789196897808\\
-1.77558563503421	-2.96792588011471\\
-1.91908666014927	-2.94968000538996\\
-2.04961740702522	-2.92458054500986\\
-2.16862706998544	-2.89375195075539\\
-2.27741818531878	-2.858064557318\\
-2.37714358505926	-2.81817736357363\\
-2.46881171621756	-2.7745744601786\\
-2.55329617472792	-2.7275949154007\\
-2.63134681126098	-2.67745657822587\\
-2.70360078663693	-2.62427451367744\\
-2.77059261446044	-2.56807482875103\\
-2.83276263492102	-2.5088045865118\\
-2.8904635973038	-2.44633840269961\\
-2.94396514665196	-2.38048220756323\\
-2.99345605060642	-2.31097455363744\\
-3.0390439907467	-2.23748576780782\\
-3.08075269491288	-2.15961519038613\\
-3.11851611400772	-2.07688672281537\\
-3.15216925792774	-1.98874293037763\\
-3.18143521176219	-1.89453803467972\\
-3.20590777313985	-1.79353031021597\\
-3.22502911610769	-1.68487471079595\\
-3.23806195164112	-1.56761705291988\\
-3.24405591432524	-1.44069185098044\\
-3.24180851303236	-1.30292702486047\\
-3.22982217800392	-1.15306027114507\\
-3.20626105545628	-0.98977393705402\\
-3.16891466549225	-0.811757630175587\\
-3.11518074780994	-0.617810041760324\\
-3.04208664511558	-0.406992384778554\\
-2.94637653364231	-0.17884323886986\\
-2.82469784762952	0.0663449341509858\\
-2.67391843856611	0.327207594583538\\
-2.4915874111442	0.601000775136923\\
-2.27650873051731	0.883396115860219\\
-2.02932874848718	1.16849676839643\\
-1.75296947956007	1.44917774918503\\
-1.45271532510242	1.71777384233659\\
-1.13582816674967	1.96700459311098\\
-0.810723475202644	2.19090326270296\\
-0.485914798356431	2.38548715216578\\
-0.169021024921071	2.54900209472586\\
0.133920662571018	2.68173940921952\\
0.418738978475628	2.78556175444595\\
0.683059030323465	2.86332311555987\\
0.926008456463667	2.91833410460932\\
1.14783490818052	2.953953549418\\
1.3495335469166	2.97332388388875\\
1.53253678575503	2.97922984659636\\
};
\addplot [color=mycolor2, forget plot]
  table[row sep=crcr]{%
2.26280257423888	2.80960815564716\\
2.29302831305614	2.80867183074462\\
2.31929080247629	2.80618204287693\\
2.3423822844014	2.80250127951935\\
2.36290633502203	2.79787694028509\\
2.3813293543842	2.79247641197041\\
2.39801639079062	2.78640997886444\\
2.41325644043243	2.77974590780946\\
2.42728055074426	2.77252037269508\\
2.44027491297271	2.76474387369191\\
2.45239039602751	2.75640518306304\\
2.46374949358338	2.74747345215825\\
2.47445133409025	2.73789884971652\\
2.48457517910172	2.72761191317346\\
2.49418267040591	2.71652164580413\\
2.5033189539957	2.7045122571966\\
2.51201268803542	2.69143830082525\\
2.52027481314658	2.67711778764795\\
2.52809580404399	2.66132262019722\\
2.53544090084505	2.64376535696691\\
2.54224248849254	2.62408081970194\\
2.54838827505016	2.60180029685615\\
2.55370307958217	2.57631491169239\\
2.55792064149776	2.5468228411808\\
2.56063948195111	2.51225203425696\\
2.56125271337673	2.47114511402165\\
2.55883439128627	2.42148497715885\\
2.55195194722146	2.36042619470558\\
2.53835084075819	2.28387586047843\\
2.5144163927739	2.18583579277575\\
2.47425016722987	2.05738165567431\\
2.40811017730113	1.88515666246544\\
2.29995179407256	1.64947860721804\\
2.12432871806517	1.32318517052677\\
1.84541309187049	0.875462695164259\\
1.42728821133862	0.290077031820641\\
0.867424953745529	-0.3980128423087\\
0.232854072756388	-1.08529691844234\\
-0.366964603365829	-1.65843732329594\\
-0.857779150263658	-2.07163959142021\\
-1.22747996915798	-2.34491019091256\\
-1.4974781299389	-2.51915094733346\\
-1.69475300086089	-2.62937975121338\\
-1.84116903062546	-2.69937576089385\\
-1.95220622288503	-2.74402602797072\\
-2.03835733829138	-2.77245352777192\\
-2.10668852572652	-2.79027457647737\\
-2.16200402841573	-2.80100059504363\\
-2.2076240161277	-2.80686962964739\\
-2.24588648643235	-2.80933620065568\\
-2.27846958141515	-2.80936263900506\\
-2.30660028634828	-2.80759524637279\\
-2.33119146968984	-2.80447285739189\\
-2.35293335492057	-2.80029503494\\
-2.3723556209046	-2.79526567578926\\
-2.38987026671953	-2.78952131869115\\
-2.40580166404906	-2.7831497243389\\
-2.42040792605227	-2.77620212170264\\
-2.43389628536998	-2.76870121870028\\
-2.44643426042814	-2.76064628454099\\
-2.45815779743023	-2.75201611564983\\
-2.46917718349013	-2.74277037511218\\
-2.47958125896757	-2.73284957465974\\
-2.48944026641433	-2.7221738033032\\
-2.49880752772091	-2.71064016748868\\
-2.5077200168184	-2.69811877080593\\
-2.51619777295085	-2.6844469050665\\
-2.52424195866059	-2.66942092406263\\
-2.5318311813046	-2.65278499291078\\
-2.5389154290978	-2.63421549961047\\
-2.54540656145442	-2.61329930282569\\
-2.55116363630205	-2.58950304368931\\
-2.55597027620966	-2.5621292591992\\
-2.55949945480315	-2.53025264788471\\
-2.56125795447763	-2.49262596042441\\
-2.56049726233623	-2.44753862177759\\
-2.55606791828385	-2.39260070946658\\
-2.54617683975173	-2.32440785933653\\
-2.52797596916795	-2.23801615992016\\
-2.49685708013057	-2.12612016964453\\
-2.44524644183476	-1.97779984761483\\
-2.36061832539141	-1.77677482876997\\
-2.22260043664195	-1.4996131001525\\
-2.00028857914076	-1.11622173396557\\
-1.65524651627194	-0.59947096514045\\
-1.16265575560169	0.0462647387570705\\
-0.55270247609633	0.749777113952609\\
0.0775982456260898	1.39099568483034\\
0.627983461948721	1.8847859809947\\
1.05679567880607	2.22324449171281\\
1.37319110253725	2.4419540065758\\
1.60368096486637	2.58053233579598\\
1.77321390764975	2.66830794416033\\
1.90035718559736	2.72419513280603\\
1.99788638089036	2.75985431610817\\
2.07440788424436	2.78243293029504\\
2.13573790410978	2.79636172181024\\
2.18586127967366	2.80443709500679\\
2.22755746886345	2.80845902429587\\
2.26280257423888	2.80960815564716\\
};
\addplot [color=mycolor1, forget plot]
  table[row sep=crcr]{%
1.72691890166584	2.88847723179633\\
1.88977999741713	2.8834031973613\\
2.03536220504261	2.86958333143687\\
2.16582777796012	2.84877669185259\\
2.28311545746725	2.82234504184872\\
2.38892852063427	2.79132564942209\\
2.48474300968047	2.75649464867069\\
2.57182593624988	2.718419395954\\
2.65125727235224	2.67750026980201\\
2.72395218959132	2.63400319819502\\
2.79068166269082	2.58808440644126\\
2.85209053609323	2.53980879057863\\
2.90871270340877	2.4891631136742\\
2.96098333195722	2.43606498605994\\
3.00924818368783	2.38036836691099\\
3.05377010415965	2.32186612963826\\
3.09473271118193	2.26029007125134\\
3.13224123531534	2.19530861546202\\
3.1663203563447	2.12652235981508\\
3.19690874753616	2.05345754998075\\
3.2238498846893	1.97555753587078\\
3.2468785019026	1.89217228867236\\
3.26560188770085	1.80254616152805\\
3.27947503274202	1.70580430389986\\
3.28776850659022	1.60093856189606\\
3.28952794292936	1.48679442299365\\
3.28352431398367	1.36206175158604\\
3.26819506653325	1.22527392271579\\
3.24157815775366	1.07482273819487\\
3.20124481153415	0.909000389538981\\
3.14424341568583	0.726084617565809\\
3.06707742927126	0.524488235265126\\
2.96575475049153	0.302996821707431\\
2.83596244229785	0.0611134075880491\\
2.67343114881857	-0.200492162412919\\
2.4745413993529	-0.479478188643106\\
2.23716580664829	-0.77139298741713\\
1.96162194139772	-1.06950375809132\\
1.65145678651216	-1.36511457787493\\
1.31368992162837	-1.6484886224626\\
0.958235973900378	-1.91024059688417\\
0.596541749872809	-2.14279440193373\\
0.239848108588855	-2.3414111709203\\
-0.102343401094471	-2.50448188054882\\
-0.42317189979988	-2.633118866783\\
-0.718594149747866	-2.73033687201901\\
-0.987022555610887	-2.80016265072688\\
-1.2287068586837	-2.84690338107628\\
-1.44509216148004	-2.87465850432765\\
-1.63828267686112	-2.88705726953517\\
-1.81065220664195	-2.88716101599237\\
-1.96459092140898	-2.87746633983676\\
-2.10235789055556	-2.85995982776086\\
-2.22600690882601	-2.83619235396913\\
-2.33735892084216	-2.80735485227582\\
-2.4380017496072	-2.77434678579574\\
-2.52930430162641	-2.73783403942497\\
-2.61243724478097	-2.69829586975579\\
-2.68839544316936	-2.65606188520638\\
-2.75801954004429	-2.61134049812941\\
-2.82201536321552	-2.56424032138803\\
-2.8809705693721	-2.51478581828564\\
-2.93536834419476	-2.46292828524935\\
-2.98559816563916	-2.40855301349939\\
-3.03196370132581	-2.35148326563642\\
-3.07468789760506	-2.29148152437925\\
-3.11391525625967	-2.22824832451758\\
-3.14971120020528	-2.16141886424223\\
-3.18205830904091	-2.0905575082707\\
-3.21084906159126	-2.01515024616841\\
-3.23587455717597	-1.93459516483031\\
-3.25680850423628	-1.84819105386823\\
-3.27318557608884	-1.7551244211205\\
-3.28437306798325	-1.65445550928868\\
-3.28953470899622	-1.54510446245682\\
-3.28758560690962	-1.42583972519049\\
-3.27713785400533	-1.29527225199449\\
-3.25643767503956	-1.15186139365136\\
-3.2232977623087	-0.993941642129349\\
-3.17503349703523	-0.81978384387707\\
-3.1084201706057	-0.627709629035622\\
-3.01970090069941	-0.41628209193809\\
-2.9046910010042	-0.1845954521154\\
-2.75903948093035	0.0673254023635762\\
-2.57871015733816	0.338037514991398\\
-2.36071268601663	0.624187684126594\\
-2.10402531729576	0.920179220424164\\
-1.81050727508198	1.21822058644483\\
-1.48545896971827	1.50894404467598\\
-1.13747436990128	1.78259850877738\\
-0.777441183093496	2.03053725527135\\
-0.416920991136378	2.24651693100535\\
-0.0664435258140606	2.42737556252555\\
0.265744470379608	2.57294994399036\\
0.574208918267981	2.68541759064244\\
0.856202200836089	2.76840422224328\\
1.11113844155475	2.82615351570793\\
1.33994052889959	2.86291451272814\\
1.54444115644465	2.88257247184433\\
1.72691890166584	2.88847723179633\\
};
\addplot [color=mycolor2, forget plot]
  table[row sep=crcr]{%
2.21306861028515	2.82138468809461\\
2.23919421653483	2.82057483010069\\
2.26198296571115	2.81841389866629\\
2.28209648632592	2.81520741079904\\
2.30004012557273	2.81116410695579\\
2.31620543889458	2.80642506981829\\
2.33089976349982	2.80108275966494\\
2.34436712119354	2.79519354364591\\
2.35680319112508	2.78878591359092\\
2.36836614840949	2.78186575343349\\
2.37918455994482	2.77441950095214\\
2.3893631330347	2.76641571723579\\
2.39898684605211	2.75780535367829\\
2.40812380367975	2.74852084306517\\
2.41682701984414	2.73847400734123\\
2.42513521646558	2.72755264680445\\
2.43307261707491	2.71561553299119\\
2.44064759293219	2.70248534729842\\
2.44784986320374	2.6879388588166\\
2.45464572789424	2.67169327259211\\
2.46097047148896	2.65338713218998\\
2.46671653053907	2.63255331130523\\
2.47171511937966	2.60858028298763\\
2.4757074853344	2.58065567904366\\
2.47829932467659	2.54768257197588\\
2.47888721346337	2.50815292966782\\
2.47653745718111	2.45995258613374\\
2.46978225740399	2.40005495988158\\
2.45626945328734	2.32403227607062\\
2.43214990124387	2.22526860338397\\
2.39099721123487	2.09370287188537\\
2.32193164605888	1.91391676251592\\
2.20659445887431	1.66267101758591\\
2.01537655551612	1.30749113866626\\
1.70707307373937	0.81265146938813\\
1.24532057573136	0.166096340870068\\
0.644241666550335	-0.572997261016063\\
-0.000258562820945922	-1.27152011839544\\
-0.570947604965193	-1.81719050216002\\
-1.01313964274256	-2.18963851724364\\
-1.33455067673994	-2.42728370461455\\
-1.56475863857915	-2.57586851466407\\
-1.73148478174332	-2.6690337084946\\
-1.85490534393418	-2.72803682939888\\
-1.94857853884193	-2.76570357759088\\
-2.02144687313672	-2.78974676871039\\
-2.07944622325466	-2.80487199459892\\
-2.12658297327417	-2.81401105855275\\
-2.16561701663072	-2.81903194698809\\
-2.19849027082329	-2.82115039387458\\
-2.22659750116732	-2.82117260665784\\
-2.25096007245838	-2.8196414537099\\
-2.27233945660591	-2.81692643655401\\
-2.29131279918541	-2.81328021918748\\
-2.30832416989408	-2.80887480881956\\
-2.32371993871405	-2.80382506834013\\
-2.33777359401511	-2.79820415446371\\
-2.35070340680101	-2.79205367855204\\
-2.36268515580077	-2.78539031690628\\
-2.37386137428234	-2.77820994370734\\
-2.38434809181079	-2.77048994828268\\
-2.39423972083171	-2.76219012830864\\
-2.40361251632825	-2.75325236179538\\
-2.41252687705079	-2.74359911533113\\
-2.42102863209388	-2.73313071778383\\
-2.42914934676731	-2.72172119612913\\
-2.43690556889085	-2.70921231194465\\
-2.44429680113426	-2.69540522725624\\
-2.45130180048283	-2.68004892996989\\
-2.45787253219382	-2.66282410550147\\
-2.46392467675033	-2.64332046147067\\
-2.46932289096235	-2.62100444588785\\
-2.47385785781808	-2.59517259170852\\
-2.47721016019479	-2.56488293533537\\
-2.47889250869963	-2.52885233674563\\
-2.47815558144282	-2.48529976084126\\
-2.47383130451668	-2.43170242087853\\
-2.46406632740831	-2.36440952035556\\
-2.44585964082178	-2.27802228655166\\
-2.41424909041632	-2.16439669527417\\
-2.36088161752589	-2.01107690277439\\
-2.27159316119037	-1.79904741422875\\
-2.12283613807592	-1.50040313581385\\
-1.87864430103352	-1.07935997199548\\
-1.49635590393171	-0.506829834323151\\
-0.957923725922759	0.19921414448768\\
-0.31930233989347	0.936241581559927\\
0.300140911852266	1.56686264606333\\
0.8085892828741	2.02329470130406\\
1.18720792635716	2.32224961244474\\
1.45915746964589	2.51027569508556\\
1.65461761218222	2.62780476318544\\
1.79762993314633	2.70185160839854\\
1.90481495801368	2.74896531345433\\
1.98718679399244	2.779081264406\\
2.05201909311113	2.79820951733387\\
2.10417669754535	2.81005399951602\\
2.14697605630563	2.81694845203879\\
2.18272614636253	2.82039605103031\\
2.21306861028515	2.82138468809461\\
};
\addplot [color=mycolor1, forget plot]
  table[row sep=crcr]{%
1.98350190093304	2.83481109413524\\
2.14248056964784	2.82987367299531\\
2.28208118893736	2.81663430426115\\
2.40516899039759	2.79701459731132\\
2.51420403841414	2.7724512279593\\
2.61126839996874	2.7440035972167\\
2.69810891910516	2.71244079897124\\
2.77618358885115	2.67830928457756\\
2.8467054939859	2.64198422970282\\
2.9106817063769	2.60370782662248\\
2.96894633693247	2.56361736515773\\
3.02218784567284	2.5217654316552\\
3.07097109315635	2.47813402637803\\
3.11575472258553	2.43264394117089\\
3.15690442723473	2.38516036500903\\
3.19470255464972	2.33549538741588\\
3.22935436337866	2.2834078343478\\
3.2609910950986	2.22860068335032\\
3.28966985832646	2.1707161524134\\
3.31537013590001	2.10932843153631\\
3.33798651930749	2.0439339238606\\
3.35731702858033	1.97393878711947\\
3.37304608638084	1.89864352854998\\
3.38472087205928	1.81722443533096\\
3.39171938981744	1.72871177170863\\
3.39320817539114	1.63196503927921\\
3.38808722604025	1.52564634298641\\
3.37491967274232	1.40819430322313\\
3.35184434348169	1.27780342807982\\
3.31647150223608	1.132418022608\\
3.26576712772136	0.969756328464007\\
3.19594142625119	0.787390297779356\\
3.10237596899037	0.582918868903819\\
2.9796536297165	0.35428450928417\\
2.82179468067958	0.100284240976438\\
2.62283595679328	-0.178703088485548\\
2.37787712128811	-0.47984112537456\\
2.08459072102887	-0.797045633861037\\
1.74490067134976	-1.12070472977889\\
1.36616010011684	-1.4383845799873\\
0.961011440557227	-1.73669058147411\\
0.545540466365693	-2.00381031164518\\
0.136253891683127	-2.23172729541117\\
-0.252879637056443	-2.41719964416383\\
-0.612215416707421	-2.56131528530312\\
-0.936727653380476	-2.66814611224094\\
-1.22522905296017	-2.74323004872262\\
-1.47916627067631	-2.79237206539434\\
-1.70147892947065	-2.82091381775971\\
-1.89574084990639	-2.83340285743232\\
-2.06560850664664	-2.83352251146187\\
-2.21451442915048	-2.82415886677544\\
-2.34552614361502	-2.80752219717156\\
-2.46130458258079	-2.78527693510509\\
-2.56411606576263	-2.75865895339753\\
-2.65586928113545	-2.72857289609268\\
-2.73816091103867	-2.6956691735887\\
-2.81232128279681	-2.66040306853896\\
-2.87945597318512	-2.62307917768607\\
-2.94048180960857	-2.58388427271685\\
-2.99615700593859	-2.54291118335312\\
-3.04710577173586	-2.50017576111775\\
-3.0938379552933	-2.45562848446283\\
-3.13676430446855	-2.40916184966583\\
-3.17620785403866	-2.3606143575557\\
-3.21241182593122	-2.30977164153326\\
-3.24554428337353	-2.25636507247526\\
-3.27569962030643	-2.20006800733758\\
-3.30289679282638	-2.14048971016388\\
-3.32707400402481	-2.07716686053251\\
-3.34807932820653	-2.00955247449009\\
-3.36565649445862	-1.93700200381204\\
-3.3794247335105	-1.85875637028413\\
-3.38885122290928	-1.77392177183919\\
-3.39321425839128	-1.68144633904356\\
-3.39155488820183	-1.58009424993202\\
-3.38261450833846	-1.46841894084292\\
-3.36475612660729	-1.34473892226158\\
-3.33586825206194	-1.20712293827287\\
-3.29325374438244	-1.05339649906144\\
-3.2335133149152	-0.881189930367348\\
-3.1524474686113	-0.68805933509838\\
-3.04502470951272	-0.4717247658697\\
-2.90549908589009	-0.2304784019155\\
-2.72779972292002	0.0361951825770366\\
-2.50633242227956	0.326800387005326\\
-2.23727259392387	0.636955616243555\\
-1.92022039873202	0.958795871949097\\
-1.55973200992121	1.28113532185731\\
-1.16592980680383	1.59076497852406\\
-0.753506095799707	1.8747576300706\\
-0.33916317715618	2.122984674891\\
0.061546409408701	2.32978928870104\\
0.436689095220135	2.49422395791177\\
0.778975116482342	2.61906324192193\\
1.08543654134701	2.7092890691649\\
1.35635780230378	2.77069358498586\\
1.59405931410446	2.80891338869378\\
1.80188505015079	2.82891482601894\\
1.98350190093304	2.83481109413524\\
};
\addplot [color=mycolor2, forget plot]
  table[row sep=crcr]{%
2.15235170053101	2.82771016179676\\
2.17478115168786	2.8270143638408\\
2.19442963274587	2.82515076697225\\
2.21184331301689	2.82237430294861\\
2.22744071667035	2.81885934274927\\
2.24154721596132	2.81472355735665\\
2.25441907581603	2.81004353652318\\
2.26626047910532	2.80486506527101\\
2.27723574388257	2.79920983718986\\
2.287478180893	2.79307970635794\\
2.2970965514435	2.78645915794991\\
2.30617976561089	2.77931640431305\\
2.31480024420055	2.77160332548791\\
2.32301621447722	2.76325433179324\\
2.33087309285995	2.75418410478504\\
2.33840400822576	2.74428405092516\\
2.34562942082719	2.73341715966002\\
2.3525556768675	2.72141077046633\\
2.35917218516503	2.70804648849882\\
2.36544667654291	2.69304609484333\\
2.37131765367985	2.67605169284216\\
2.37668256454787	2.65659737944596\\
2.38137926771617	2.63406819591809\\
2.38515669445102	2.60763958760633\\
2.38762767356599	2.57618636589477\\
2.38819156745344	2.53814292479036\\
2.3859045295297	2.49128390585528\\
2.37925664472764	2.43237256993924\\
2.36577987128437	2.35658621879794\\
2.341343944372	2.25656583621438\\
2.29887835004151	2.12085217775264\\
2.22608704851716	1.93143401185242\\
2.10168592547686	1.66053185726397\\
1.89085754402466	1.26902361700542\\
1.54649503515728	0.716340999550197\\
1.03571087172451	0.000929336424112065\\
0.399348919740479	-0.782068038128535\\
-0.236420855862073	-1.47169274277777\\
-0.760509941703801	-1.9731442651855\\
-1.14599792097037	-2.29796994256754\\
-1.41795736062666	-2.49909633877957\\
-1.61004980181856	-2.6230915771769\\
-1.74853215088807	-2.7004757486327\\
-1.8510856437997	-2.74950170551074\\
-1.92914976045597	-2.78089014176923\\
-1.99012781130407	-2.80100848790145\\
-2.03888953927628	-2.81372339082726\\
-2.07870967972725	-2.82144283220785\\
-2.11184268359863	-2.82570383909892\\
-2.13987658965057	-2.82750974300448\\
-2.16395440534203	-2.82752820836332\\
-2.1849153903894	-2.82621036536572\\
-2.20338713933759	-2.82386418602627\\
-2.21984685321108	-2.82070066621658\\
-2.23466292048751	-2.81686344476888\\
-2.24812365619123	-2.81244808564602\\
-2.26045749776887	-2.80751474654026\\
-2.27184740571056	-2.80209650138798\\
-2.28244125532504	-2.79620471606388\\
-2.29235939745704	-2.78983234430576\\
-2.30170017193125	-2.78295567340977\\
-2.31054389541644	-2.77553482472919\\
-2.3189556644266	-2.76751315331924\\
-2.32698718181477	-2.75881556255123\\
-2.33467770906455	-2.74934563031601\\
-2.3420541496028	-2.7389813137606\\
-2.34913016391621	-2.72756883803061\\
-2.35590408628134	-2.71491415348089\\
-2.36235522843535	-2.70077102445467\\
-2.36843787494733	-2.68482432663472\\
-2.37407182667224	-2.66666637335596\\
-2.37912760666702	-2.64576288546655\\
-2.38340318009172	-2.62140325599939\\
-2.38658683527468	-2.59262649835452\\
-2.38819693135162	-2.55810873989928\\
-2.38748200409132	-2.51598860125734\\
-2.38325123982084	-2.4635902058119\\
-2.37357973587011	-2.39697463704155\\
-2.35528424240865	-2.31020156485598\\
-2.32297480791804	-2.19410716723213\\
-2.26733796727471	-2.03432483854576\\
-2.17215165550712	-1.8083671327232\\
-2.00985192138508	-1.48263115360921\\
-1.73835691289407	-1.01459181792908\\
-1.31180442026624	-0.375710526844209\\
-0.726543819406856	0.392110669737228\\
-0.0726552518913369	1.1473486329682\\
0.515994265558174	1.74708442268336\\
0.969581768645549	2.15449219604956\\
1.29396436142171	2.41070322055379\\
1.52212487005618	2.56847718214997\\
1.68471060334354	2.66624400461973\\
1.80346802395086	2.72773203965389\\
1.89264247285927	2.76692741400909\\
1.96142401905006	2.7920729024311\\
2.01580175575232	2.80811516354905\\
2.05975745345351	2.81809589920693\\
2.09600028640691	2.82393323818825\\
2.12641710976445	2.8268657627066\\
2.15235170053101	2.82771016179676\\
};
\addplot [color=mycolor1, forget plot]
  table[row sep=crcr]{%
2.33734640818134	2.84173773499851\\
2.49109306502754	2.83698081417114\\
2.62330022544378	2.82445653845541\\
2.73771631127657	2.80622994061618\\
2.83740682018188	2.78378031472703\\
2.92486031715751	2.75815635844025\\
3.00209113109873	2.73009192852672\\
3.07072932765729	2.70009040270244\\
3.13209557814148	2.66848528261135\\
3.18726167237742	2.63548328666928\\
3.23709857129096	2.60119468959842\\
3.28231411075317	2.56565438791066\\
3.32348228736911	2.52883617258919\\
3.36106573814055	2.49066194334983\\
3.39543268532002	2.45100704965754\\
3.42686930034134	2.40970253972035\\
3.45558815745043	2.36653479655119\\
3.48173319375229	2.32124280621342\\
3.50538135569285	2.27351311212424\\
3.52654087650361	2.22297234172502\\
3.54514587585504	2.169177033847\\
3.5610466794569	2.11160033706129\\
3.57399489574042	2.04961498688047\\
3.58362182697132	1.98247180702595\\
3.58940819663856	1.90927283688475\\
3.59064240796475	1.82893811365253\\
3.58636359083918	1.74016524118358\\
3.57528458424016	1.64138137476734\\
3.55568892486679	1.53068855771114\\
3.52529540174499	1.40580622725746\\
3.48108505982604	1.26402050241484\\
3.41909141725667	1.10216073546208\\
3.33417040425861	0.916642753393567\\
3.21980119327106	0.703648025466911\\
3.06803476626328	0.459547051737497\\
2.86981036427367	0.181706580406626\\
2.6159739292958	-0.130208155401798\\
2.29934194311722	-0.472513120888338\\
1.91782040438414	-0.835884415541505\\
1.4777126078581	-1.20491866160028\\
0.995289546876273	-1.56004776980447\\
0.494706089324506	-1.88187020607477\\
0.00238449151309759	-2.1560540933375\\
-0.459355597332624	-2.3761892341717\\
-0.876199631010727	-2.54343586600995\\
-1.24220440088729	-2.66399026487476\\
-1.55780494100006	-2.74618201507608\\
-1.8272057959172	-2.79836113551832\\
-2.05624254163932	-2.82780100467006\\
-2.25101548100488	-2.84034970404085\\
-2.41717577691034	-2.84048729330591\\
-2.5596376113602	-2.83154468445753\\
-2.68252577371285	-2.81595187433117\\
-2.78923445865596	-2.79545890038434\\
-2.88252731183531	-2.77131300475877\\
-2.96464362547795	-2.7443931490121\\
-3.03739528832324	-2.71530905464699\\
-3.1022492448699	-2.68447280866225\\
-3.16039499015794	-2.65215002671948\\
-3.21279858825657	-2.61849606107907\\
-3.26024529251385	-2.58358133702245\\
-3.30337281756184	-2.54740876374752\\
-3.34269704208471	-2.50992529935037\\
-3.37863158278195	-2.47102910840988\\
-3.41150234802259	-2.43057328017037\\
-3.44155787923642	-2.38836672707065\\
-3.46897602068764	-2.34417261896214\\
-3.49386721477555	-2.29770449848986\\
-3.51627448610095	-2.24862004575401\\
-3.53616993572033	-2.19651229908184\\
-3.55344729667083	-2.14089798164329\\
-3.56790977846952	-2.08120242353426\\
-3.57925202243482	-2.01674040514939\\
-3.58703446675794	-1.94669209089562\\
-3.59064774299446	-1.87007310494929\\
-3.58926386356524	-1.78569779700986\\
-3.58176991497527	-1.69213500956189\\
-3.56667883962573	-1.58765649062854\\
-3.54201099524978	-1.470180064693\\
-3.50514035916986	-1.3372137993762\\
-3.45260237602551	-1.18581540912352\\
-3.37987038390187	-1.01259563476494\\
-3.28113136886937	-0.813818484650238\\
-3.14914043914852	-0.585686455627477\\
-2.97531818455351	-0.324937813493696\\
-2.75037108609346	-0.0298939993925478\\
-2.46579913118171	0.297998214986816\\
-2.11652784717784	0.652394548293221\\
-1.70430281127496	1.02086053043047\\
-1.24041247959621	1.38549935232378\\
-0.745589454425837	1.72618573406833\\
-0.245957959688292	2.02551419754205\\
0.233410818537544	2.27295902153481\\
0.673940830231836	2.46611819797747\\
1.06564802589559	2.60904900756689\\
1.40610199972537	2.70934266685266\\
1.69793410542082	2.77553649714771\\
1.9463836188929	2.81552398599753\\
2.15754334599019	2.8358768856879\\
2.33734640818134	2.84173773499851\\
};
\addplot [color=mycolor2, forget plot]
  table[row sep=crcr]{%
2.07403866260917	2.8269543349669\\
2.09315924916392	2.82636066585801\\
2.10999318190359	2.82476358091409\\
2.12498416604991	2.82237301346102\\
2.1384737236779	2.81933272783761\\
2.15072854314413	2.81573951332741\\
2.16195960392293	2.81165577361651\\
2.17233576107662	2.80711780571264\\
2.18199352214924	2.80214117749728\\
2.19104415278104	2.79672407548457\\
2.19957886469313	2.79084915661269\\
2.20767258761167	2.78448421578705\\
2.21538665469627	2.77758182436031\\
2.2227706072954	2.7700779720829\\
2.22986322800465	2.76188963392558\\
2.23669282495772	2.75291106475939\\
2.24327670047541	2.74300847945363\\
2.24961962713554	2.73201257827272\\
2.25571100127946	2.71970809059483\\
2.26152011235544	2.70581907565178\\
2.26698859638195	2.68998804099299\\
2.27201852764331	2.67174585401969\\
2.27645355260126	2.6504676438978\\
2.28004862539231	2.62530691233002\\
2.28242057834992	2.59509496991008\\
2.28296560694686	2.55818389163297\\
2.28071807308664	2.51219529283697\\
2.27410238642777	2.45360860632651\\
2.26048518241586	2.37707127134637\\
2.23534788704536	2.27422554125036\\
2.19073871869562	2.13172032968132\\
2.11242614059813	1.92801461437831\\
1.97516213207316	1.62920486636794\\
1.73745884434234	1.18788523244802\\
1.34691053847526	0.561027491721929\\
0.783496643137142	-0.228530815809744\\
0.128514551779342	-1.03519522066827\\
-0.470022307285884	-1.68503886015666\\
-0.927825138946958	-2.12334570222652\\
-1.24955556419985	-2.39453726327172\\
-1.47174325929099	-2.55887721488673\\
-1.6276173784582	-2.65949588556988\\
-1.74007013041079	-2.72233230901795\\
-1.82370459021303	-2.76231110217522\\
-1.88773858188048	-2.78805580223589\\
-1.93807543546071	-2.80466143548468\\
-1.97858565442676	-2.81522326828103\\
-2.01187345695523	-2.82167525731679\\
-2.0397362836796	-2.82525763629486\\
-2.06344483303272	-2.82678421414579\\
-2.08391748886742	-2.82679934778336\\
-2.10183152825442	-2.82567259474096\\
-2.11769567143009	-2.82365721079403\\
-2.13189845770502	-2.820927108104\\
-2.14474116501868	-2.81760064446228\\
-2.15646063252991	-2.81375615329919\\
-2.16724534973486	-2.80944215691259\\
-2.17724696413185	-2.80468405816212\\
-2.18658860818404	-2.7994884187659\\
-2.1953709700189	-2.79384550823411\\
-2.203676722874	-2.78773053545745\\
-2.2115737211562	-2.78110379048918\\
-2.21911722628461	-2.77390978768314\\
-2.22635131743657	-2.7660753869192\\
-2.2333095527058	-2.75750675718095\\
-2.24001486011258	-2.74808491752273\\
-2.24647854030522	-2.73765942262243\\
-2.25269813468516	-2.72603952351195\\
-2.25865372597533	-2.71298178264141\\
-2.2643019470651	-2.69817258161117\\
-2.26956649912897	-2.68120310461785\\
-2.27432317971486	-2.66153299540644\\
-2.27837603427681	-2.63843659050964\\
-2.28141877576086	-2.61092174376639\\
-2.28297110720314	-2.57760452688373\\
-2.28227113129426	-2.53651120578387\\
-2.27808881038903	-2.48475757987664\\
-2.26839371171135	-2.41801737554461\\
-2.24974780590293	-2.32962364145465\\
-2.21617410986802	-2.209037813379\\
-2.15704627256193	-2.03929834557288\\
-2.05333502049323	-1.79319604265062\\
-1.87212667571019	-1.42961807547341\\
-1.56418794388763	-0.898801961181344\\
-1.08459312721087	-0.180259328262583\\
-0.458142264779495	0.642240804383108\\
0.185645057171374	1.3865479411452\\
0.717664498691321	1.92902099233589\\
1.10355717766559	2.27579000289122\\
1.37077553186421	2.48689678677427\\
1.55630717272349	2.61520166763647\\
1.68820173657598	2.69451217766848\\
1.78481611440694	2.74453254832799\\
1.85774317453226	2.77658392136132\\
1.91434049111072	2.79727288927779\\
1.95937293265292	2.81055649618859\\
1.99600504333981	2.81887305061057\\
2.02639365982054	2.82376652078023\\
2.05204572476894	2.82623889617718\\
2.07403866260917	2.82695433496691\\
};
\addplot [color=mycolor1, forget plot]
  table[row sep=crcr]{%
2.85378684355605	2.95375934774863\\
3.00036271091603	2.94924405278867\\
3.12344126758871	2.93759902399836\\
3.22778799871929	2.92098727686308\\
3.3170997188036	2.90088308953318\\
3.39424650393384	2.87828537605597\\
3.46146503151574	2.85386429021913\\
3.5205077004599	2.82806100926026\\
3.57275564791737	2.80115539161355\\
3.61930372218224	2.77331176433672\\
3.66102418476661	2.74460980903458\\
3.69861443570776	2.71506521735019\\
3.73263274860664	2.6846432222397\\
3.76352495212441	2.6532670549698\\
3.79164418608587	2.62082266241941\\
3.81726524580055	2.58716052734958\\
3.84059455760578	2.55209508554567\\
3.86177645753552	2.51540197328519\\
3.8808961356679	2.4768131280646\\
3.89797932807475	2.43600957663164\\
3.91298855500801	2.39261155440344\\
3.92581538456694	2.34616538903257\\
3.93626780564246	2.29612632827557\\
3.94405127046831	2.24183617786611\\
3.94874124478657	2.18249421862109\\
3.94974408471304	2.11711937884328\\
3.94624161194274	2.04450105336132\\
3.93711271718952	1.96313534372365\\
3.92082251357158	1.87114303712362\\
3.89526591422912	1.76616583920769\\
3.85754835703905	1.64523938113454\\
3.80368328564037	1.50464785205062\\
3.72818855730828	1.33978092653307\\
3.62358394602286	1.14504866135667\\
3.4798554152207	0.91397973008484\\
3.28410704277114	0.639748655323926\\
3.02092802527427	0.316534364168515\\
2.67443814677328	-0.0578283946622168\\
2.23315886346491	-0.477872185184208\\
1.6977294122587	-0.926613010670904\\
1.08809389975987	-1.37524073796466\\
0.443423744619842	-1.789656253505\\
-0.188919701103301	-2.1418787995415\\
-0.769190184836531	-2.41863570498246\\
-1.27496003117096	-2.62168455721952\\
-1.70082019331029	-2.76206254379989\\
-2.05243324705879	-2.85371812142474\\
-2.34036479104677	-2.90954929474476\\
-2.5760484738883	-2.939888827454\\
-2.76983699388432	-2.95240645345205\\
-2.93035907357098	-2.95256259839635\\
-3.06451310398694	-2.94415836419008\\
-3.17770308589254	-2.92980858807872\\
-3.2741269781873	-2.91130015429449\\
-3.35704103530817	-2.88984769035182\\
-3.42897740464765	-2.86627073504579\\
-3.49191444035656	-2.84111477549386\\
-3.5474067123776	-2.81473343561453\\
-3.59668306227496	-2.78734414827436\\
-3.6407201882803	-2.75906578463492\\
-3.680297784754	-2.72994395305715\\
-3.71603984576179	-2.69996777990496\\
-3.74844556194064	-2.66908069854847\\
-3.77791231563064	-2.63718690458173\\
-3.80475257411681	-2.60415454368636\\
-3.82920594423958	-2.5698162864352\\
-3.85144723589105	-2.53396764537538\\
-3.87159104604926	-2.49636315828672\\
-3.88969308395758	-2.45671036490989\\
-3.90574818005476	-2.41466131763696\\
-3.91968462439406	-2.369801168491\\
-3.93135412847148	-2.32163314528318\\
-3.94051625242322	-2.26955894890809\\
-3.94681552699829	-2.2128532506457\\
-3.94974864331101	-2.15063052461213\\
-3.94861786898256	-2.08180190868407\\
-3.94246512699939	-2.00501917332367\\
-3.92997876793897	-1.91860230349286\\
-3.90936183301247	-1.82044697762058\\
-3.87814661617138	-1.70790912168691\\
-3.83293637790775	-1.57766744265031\\
-3.76905389077023	-1.42557504030993\\
-3.68008561837665	-1.24653503414642\\
-3.55734757160428	-1.03448524106805\\
-3.38940075654997	-0.782670607277383\\
-3.16196877730482	-0.484526804717714\\
-2.8589974547303	-0.135636159747718\\
-2.46599260243726	0.262899932561106\\
-1.97647101926526	0.700218704365431\\
-1.40011050779229	1.15307365468656\\
-0.767179368858455	1.58875278686164\\
-0.122883733916398	1.9747623873091\\
0.487411681846658	2.28988111309565\\
1.03206660126965	2.52881822967954\\
1.49770385090727	2.69884339905605\\
1.88530306237435	2.81312253182019\\
2.2036337251602	2.88540035185023\\
2.46405326257749	2.92736787416042\\
2.67759798181907	2.94798892820503\\
2.85378684355605	2.95375934774863\\
};
\addplot [color=mycolor2, forget plot]
  table[row sep=crcr]{%
1.96331297825472	2.81411434692614\\
1.97955568460529	2.81360945989346\\
1.99394894145568	2.81224344495505\\
2.00684563657167	2.81018642729994\\
2.01851919858992	2.80755505797425\\
2.02918453054378	2.80442755317506\\
2.03901273452669	2.80085358909671\\
2.04814162857186	2.79686080122021\\
2.05668335343318	2.79245896553525\\
2.06472992354038	2.78764252681038\\
2.07235728995282	2.78239187592512\\
2.07962829257673	2.77667360100596\\
2.08659474688109	2.7704398071925\\
2.09329881290044	2.76362649201651\\
2.09977371479861	2.75615085852422\\
2.10604380432763	2.7479073287507\\
2.1121238779384	2.73876186553627\\
2.11801754870306	2.72854399284722\\
2.12371431648853	2.71703558071442\\
2.12918473361584	2.70395496073472\\
2.13437266031943	2.68893414383281\\
2.13918292172501	2.67148561889073\\
2.1434614872898	2.6509530546004\\
2.14696315807193	2.62643654424968\\
2.14929780981425	2.59667659095788\\
2.14983877505048	2.55986949608556\\
2.14756240313816	2.51336573228205\\
2.14075880220387	2.45316383993336\\
2.12649489728472	2.37304040554418\\
2.09959243129066	2.26303053274972\\
2.05066101712678	2.10679271522806\\
1.96241744398336	1.87735297786831\\
1.80373134755954	1.5320193803186\\
1.5249370184216	1.01443872907873\\
1.07434285272362	0.290876216853278\\
0.466845707927361	-0.561366935479028\\
-0.165834173365223	-1.34157313650627\\
-0.685139205075571	-1.90593960632484\\
-1.05566763729559	-2.26086599763237\\
-1.30803943883671	-2.4736311669978\\
-1.48098569504271	-2.60155014104102\\
-1.60277403995252	-2.68015961549638\\
-1.69139540992165	-2.72967353157353\\
-1.75798342375903	-2.76149930187783\\
-1.8095001932274	-2.78220824653105\\
-1.8504053136757	-2.79570011885863\\
-1.8836363370713	-2.80436241989564\\
-1.91118293182293	-2.80970035122978\\
-1.93442860552846	-2.81268811575503\\
-1.95435888882385	-2.81397064027515\\
-1.97169141749957	-2.81398282422971\\
-1.98695927691459	-2.81302198532718\\
-2.00056568939737	-2.81129297433242\\
-2.01282070314002	-2.80893688317258\\
-2.0239663129008	-2.80604963840826\\
-2.0341939807038	-2.8026941931011\\
-2.04365705814167	-2.79890855504601\\
-2.0524797183299	-2.79471102260153\\
-2.06076344870776	-2.79010347533507\\
-2.06859180064954	-2.78507323912646\\
-2.0760338591876	-2.77959383119857\\
-2.0831467385059	-2.77362474073185\\
-2.08997729655123	-2.76711028448372\\
-2.09656317542501	-2.7599774726838\\
-2.10293319858159	-2.7521327105695\\
-2.10910707836365	-2.74345702675666\\
-2.11509429371756	-2.73379933765209\\
-2.12089186851221	-2.72296699287375\\
-2.12648058522473	-2.71071244562911\\
-2.13181885532999	-2.69671426376555\\
-2.13683294532342	-2.68054968683557\\
-2.14140135889619	-2.66165426968956\\
-2.14532958657962	-2.63926134362759\\
-2.14830854448596	-2.61230916988259\\
-2.14984462122842	-2.57929506405428\\
-2.14913886684121	-2.53804021148533\\
-2.14487236280542	-2.48530023035206\\
-2.13481351048029	-2.41610339883878\\
-2.11507917243741	-2.322602075395\\
-2.07871661488327	-2.19206490860275\\
-2.01299166199431	-2.00347325007772\\
-1.89455311878832	-1.72253206278974\\
-1.68304146370885	-1.29824361601324\\
-1.32288788987185	-0.677317393127129\\
-0.784256047327998	0.130299107663984\\
-0.143052910286758	0.973226891250416\\
0.444335511621189	1.6531311090341\\
0.887744684682392	2.1055808885141\\
1.19402089518443	2.38089192911604\\
1.40239527567296	2.54552300786836\\
1.54695012512996	2.64548551289316\\
1.65041113146462	2.70769190997727\\
1.72693445679873	2.74730523772255\\
1.78530237107653	2.77295405043041\\
1.83106812686669	2.78968083596682\\
1.8678383551793	2.80052528515005\\
1.89802243479918	2.80737648435111\\
1.92327427133273	2.81144166910494\\
1.94475819050041	2.813511447244\\
1.96331297825472	2.81411434692614\\
};
\addplot [color=mycolor1, forget plot]
  table[row sep=crcr]{%
3.66143984979885	3.26030969750488\\
3.79889567075761	3.25609522771153\\
3.91145095592088	3.24545959918808\\
4.00488405111818	3.23059503687368\\
4.08344387248664	3.21291823573724\\
4.15028586095019	3.1933443182594\\
4.20777948300575	3.1724604713876\\
4.25772412466527	3.15063650206084\\
4.30150107266993	3.12809570627973\\
4.34018118660345	3.10496058427243\\
4.37460178831773	3.08128242180113\\
4.40542201274555	3.05706035845976\\
4.43316292657453	3.0322534630426\\
4.45823672388085	3.00678802538314\\
4.4809679471203	2.98056144301995\\
4.50160874510828	2.95344354072357\\
4.52034952145912	2.92527579532119\\
4.53732584811405	2.89586867450269\\
4.55262214951887	2.86499708846314\\
4.56627235198397	2.8323937631021\\
4.57825739685561	2.79774014511581\\
4.5884991936492	2.76065421558516\\
4.59685019208507	2.72067428857635\\
4.60307721600122	2.67723746494548\\
4.60683743426471	2.62965084430114\\
4.60764320176406	2.5770527932945\\
4.60481076479983	2.51836041969331\\
4.59738513955396	2.45219776821199\\
4.58402929444862	2.37679697110114\\
4.56285927616657	2.28986152667899\\
4.53119697906703	2.18837715999582\\
4.48519762792757	2.06835230804267\\
4.41928951640905	1.92447060357796\\
4.32534401180416	1.74965179478889\\
4.1914966368282	1.5345712889852\\
4.00063747017681	1.26734206152893\\
3.72897141283549	0.933929753391442\\
3.34608977710739	0.520567391559537\\
2.81996545962316	0.0201741504485785\\
2.13164049836243	-0.556265694742713\\
1.29875894103909	-1.16884591759495\\
0.38957392165057	-1.75320489136707\\
-0.498186453022081	-2.24783582138081\\
-1.28370138150545	-2.62273274160032\\
-1.93109871617218	-2.88288433040544\\
-2.44307834405449	-3.05183851426923\\
-2.84090750293457	-3.15567109193353\\
-3.14945119067209	-3.21558432071315\\
-3.39047657724607	-3.24666685292685\\
-3.5810051094907	-3.25901012604828\\
-3.73371646281758	-3.25918281434069\\
-3.85788271398113	-3.2514207524235\\
-3.96026221202168	-3.23845303017948\\
-4.04580440875587	-3.22204163116993\\
-4.11816598291268	-3.2033255661376\\
-4.18007743256084	-3.18303890969571\\
-4.23360073644704	-3.16164924418725\\
-4.28031038105244	-3.13944609773853\\
-4.32142116360366	-3.11659782607227\\
-4.35787908656229	-3.09318838386657\\
-4.39042653213797	-3.06924110152558\\
-4.41964935154084	-3.04473391305451\\
-4.44601108058334	-3.01960882400614\\
-4.4698778459518	-2.99377736739554\\
-4.4915363995599	-2.96712312797632\\
-4.51120693577764	-2.9395019735828\\
-4.52905178805701	-2.91074032479108\\
-4.5451806848625	-2.88063156228785\\
-4.55965291049511	-2.84893047509702\\
-4.57247641804878	-2.81534546188049\\
-4.58360363797199	-2.77952798499219\\
-4.59292337262107	-2.74105851436077\\
-4.60024771005255	-2.69942785082767\\
-4.60529225253164	-2.65401223980981\\
-4.60764702219832	-2.60404001140737\\
-4.60673400038473	-2.54854652175554\\
-4.60174509902098	-2.48631279899729\\
-4.59155101434331	-2.4157813597352\\
-4.5745662013086	-2.33493999874156\\
-4.54854715071333	-2.2411609192126\\
-4.51028900271617	-2.13097878721842\\
-4.45516831287353	-1.99978905489944\\
-4.37645900239336	-1.84145315801323\\
-4.26433536047132	-1.64782626160644\\
-4.10451173349945	-1.40831628899047\\
-3.87667368401192	-1.10982577643923\\
-3.55351055876737	-0.737953295795312\\
-3.10267774522643	-0.281144605569967\\
-2.49611814370087	0.260290711235291\\
-1.7301505261287	0.861715239181722\\
-0.84776523000926	1.46888239758364\\
0.0629472957477424	2.0145391317581\\
0.907080160091894	2.45060873194818\\
1.62516746798835	2.76588732943137\\
2.20290107764375	2.9770626761349\\
2.65467947580856	3.11042204052515\\
3.00487204212609	3.19003972564353\\
3.27722841285825	3.23399961853859\\
3.49116912628829	3.25470361386584\\
3.66143984979885	3.26030969750488\\
};
\addplot [color=mycolor2, forget plot]
  table[row sep=crcr]{%
1.781882940209	2.77322293973926\\
1.79588260529314	2.77278702308228\\
1.80841097659117	2.77159736069738\\
1.81974138446683	2.76978961057895\\
1.83008806950595	2.76745684099947\\
1.83962129977994	2.76466086069903\\
1.8484781286801	2.7614397057913\\
1.8567701602796	2.75781253548015\\
1.86458921862207	2.75378271177296\\
1.87201151627499	2.7493395386105\\
1.87910071970149	2.74445893837388\\
1.88591017453311	2.73910320409754\\
1.89248445759564	2.73321985568351\\
1.89886034778057	2.72673952788851\\
1.90506724197861	2.71957270969144\\
1.91112697391478	2.71160502049679\\
1.91705290988569	2.70269052456319\\
1.92284807805368	2.69264231552977\\
1.92850190848564	2.68121919108762\\
1.93398487063834	2.66810658861676\\
1.93923980603726	2.65288890189017\\
1.94416790475438	2.63500855607246\\
1.94860575705665	2.61370425569063\\
1.95228712340011	2.58791565836191\\
1.95477779378775	2.55613250846652\\
1.95536164810143	2.51614941394148\\
1.95283551157929	2.46465606258949\\
1.94512845155244	2.39653378330288\\
1.92857473641034	2.30362121407213\\
1.89649698060454	2.17253292220582\\
1.83645882975982	1.98093216469277\\
1.72533668269092	1.69210819871046\\
1.52302781961484	1.25184668083432\\
1.17606665822847	0.607304844084989\\
0.663584658507134	-0.216879808826287\\
0.0724679910453545	-1.04778855701782\\
-0.450405259793972	-1.69358161491012\\
-0.83620668374121	-2.11316810285577\\
-1.10040794624059	-2.36628461285566\\
-1.28021043293575	-2.51785112424728\\
-1.40550257695371	-2.61050165599441\\
-1.49571971739991	-2.6687177443404\\
-1.56287639808906	-2.7062287971572\\
-1.61442271385558	-2.73085860808695\\
-1.65508183507587	-2.74719838369212\\
-1.68793327545703	-2.75803071783548\\
-1.71504330725383	-2.76509519539211\\
-1.73783668099559	-2.76951037359158\\
-1.75732083822314	-2.77201338351128\\
-1.77422466624992	-2.77310013824509\\
-1.78908660957274	-2.77310975999451\\
-1.80231201968687	-2.77227676672347\\
-1.81421135362129	-2.77076408710107\\
-1.82502616337916	-2.76868436034322\\
-1.83494712521408	-2.76611388550085\\
-1.84412676764258	-2.76310183373794\\
-1.85268859658263	-2.75967631826468\\
-1.86073372247673	-2.75584830696964\\
-1.8683457190156	-2.75161398652662\\
-1.87559419994801	-2.74695594516419\\
-1.88253743820329	-2.74184337704484\\
-1.88922423869972	-2.73623138924254\\
-1.89569519247438	-2.73005938925365\\
-1.90198337088657	-2.72324842875773\\
-1.90811445297349	-2.71569726071395\\
-1.91410620476902	-2.70727671131428\\
-1.91996713155655	-2.69782174700842\\
-1.92569398016566	-2.68712028512562\\
-1.93126754139454	-2.67489728134176\\
-1.93664582735851	-2.66079180389336\\
-1.94175305655896	-2.64432345531191\\
-1.94646174838001	-2.62484223712471\\
-1.95056317919443	-2.6014520545482\\
-1.9537176319937	-2.57289118093351\\
-1.95536854304165	-2.53734057205639\\
-1.95459018715304	-2.49210796544402\\
-1.94980926975556	-2.43309269516865\\
-1.93828054844667	-2.35385588598758\\
-1.91507345139156	-2.24397836189112\\
-1.87109236299913	-2.08618419617105\\
-1.78932984678071	-1.85168026808045\\
-1.63883898093533	-1.49478489388947\\
-1.37068276603309	-0.956723276792841\\
-0.938084355572819	-0.210099538651129\\
-0.368170161346866	0.646002603077406\\
0.204498185789562	1.40023954540039\\
0.660626177511073	1.92880903266141\\
0.981154247279146	2.2560043438161\\
1.19872262954689	2.45157412015697\\
1.34824136987672	2.56968263421268\\
1.45410936810425	2.64287391127373\\
1.53163409357617	2.68947336578912\\
1.59025726812783	2.71981211549013\\
1.63589119605128	2.73985964858191\\
1.672335905501	2.75317592808163\\
1.7021050431826	2.76195291278141\\
1.72690876815852	2.76758092987106\\
1.74794154747409	2.77096544098006\\
1.76605796463212	2.77270965210016\\
1.781882940209	2.77322293973926\\
};
\addplot [color=mycolor1, forget plot]
  table[row sep=crcr]{%
4.99561704614432	3.93093778001022\\
5.12441715718723	3.92700594018682\\
5.22749359751082	3.91727727481723\\
5.31147431550237	3.90392417940075\\
5.38100855098295	3.88828361280772\\
5.4394192891387	3.87118259644813\\
5.48912451149793	3.853130710441\\
5.53191316501664	3.83443589164418\\
5.56912894315978	3.81527522357547\\
5.60179472704873	3.79573879287683\\
5.63069819801756	3.77585708918583\\
5.65645159570325	3.75561812131712\\
5.67953394460403	3.73497793935292\\
5.70032116110036	3.71386678908289\\
5.71910759992558	3.69219224168065\\
5.73612139531802	3.6698400907359\\
5.75153514953909	3.64667344917968\\
5.7654729662991	3.62253022618498\\
5.77801442013848	3.59721896598886\\
5.78919572749572	3.57051285010664\\
5.79900808959564	3.54214147165365\\
5.80739286480359	3.51177975532787\\
5.81423284683676	3.4790330822595\\
5.81933840613643	3.44341723417179\\
5.82242649126247	3.40433111923476\\
5.82308931925251	3.36101926251882\\
5.82074773116208	3.31251954517394\\
5.81458117655192	3.25758934794285\\
5.80342129046852	3.19459959207491\\
5.7855875771488	3.12138036097956\\
5.75862922647238	3.034992555466\\
5.71891201929251	2.93138554196882\\
5.66094598091564	2.80487890431387\\
5.57627654710481	2.64737706978255\\
5.45164906545852	2.44719972533898\\
5.26602389169874	2.18744287372611\\
4.98604568915978	1.8440724380056\\
4.56051953116549	1.385080343331\\
3.91861389651914	0.775217544816685\\
2.98722249841715	-0.0039109304312002\\
1.75058309585665	-0.912657960913425\\
0.328315571224359	-1.82653552024029\\
-1.04986979985048	-2.59477729602087\\
-2.19910607754332	-3.14384399650946\\
-3.06935198955974	-3.49401988568186\\
-3.7005382255588	-3.70261632449116\\
-4.15499533889883	-3.82140591624197\\
-4.48601544731098	-3.88578484184051\\
-4.7319291316338	-3.91755636433422\\
-4.91871783456797	-3.92969230616816\\
-5.06374868767354	-3.92987792432078\\
-5.1787004733027	-3.92270577353361\\
-5.27154301172934	-3.91095528868635\\
-5.34781428514036	-3.89632892311543\\
-5.4114358091086	-3.87987800734067\\
-5.46523604475778	-3.86225253481791\\
-5.51129068081925	-3.84385014702867\\
-5.55114736942836	-3.82490650236635\\
-5.58597664418855	-3.80555091596297\\
-5.61667493956786	-3.78584100285091\\
-5.64393599650679	-3.76578435055823\\
-5.66830102740274	-3.74535198599743\\
-5.69019434122226	-3.72448650066768\\
-5.70994881233312	-3.70310656504833\\
-5.72782408732873	-3.68110886870713\\
-5.7440194449165	-3.6583680807747\\
-5.75868256018706	-3.63473512752232\\
-5.77191495372025	-3.61003386380834\\
-5.78377454721926	-3.58405602983503\\
-5.79427544279993	-3.556554201649\\
-5.80338474493174	-3.52723223388266\\
-5.81101590439257	-3.49573242376901\\
-5.81701762371723	-3.46161825359495\\
-5.82115673986917	-3.42435103214597\\
-5.82309256192751	-3.38325795858695\\
-5.82233867563938	-3.3374879231185\\
-5.81820587077501	-3.28594949383599\\
-5.80971597542463	-3.22722262281281\\
-5.79546989489136	-3.15943099493215\\
-5.77344210597716	-3.08005461689224\\
-5.74065481114691	-2.98565064564301\\
-5.6926519686067	-2.87143251714126\\
-5.62263692343476	-2.73063152630738\\
-5.52004510004058	-2.55353455835752\\
-5.36819308466233	-2.32608261267201\\
-5.14054916221915	-2.02803325817305\\
-4.79549424832186	-1.63128596818774\\
-4.27157764880956	-1.10094716298679\\
-3.49246762041912	-0.406261979656713\\
-2.40332781996221	0.448023197706848\\
-1.04933145655836	1.37913204945238\\
0.380773929650926	2.23607011864938\\
1.65928387635527	2.89706136864684\\
2.66777140341069	3.34038976153233\\
3.41090951800739	3.61241114041432\\
3.94617389039209	3.77064798880819\\
4.33320444411984	3.85877485260488\\
4.61774349464813	3.90477787832598\\
4.83146612136199	3.92550583087053\\
4.99561704614432	3.93093778001022\\
};
\addplot [color=mycolor2, forget plot]
  table[row sep=crcr]{%
1.41336186657461	2.6452476547433\\
1.42661771092012	2.6448335028727\\
1.438711252913	2.6436839279649\\
1.44984844049261	2.64190594243159\\
1.46019496974537	2.6395722439554\\
1.46988592532474	2.63672910925978\\
1.47903283490991	2.63340160243155\\
1.48772889170151	2.62959686997006\\
1.49605285324855	2.62530599884741\\
1.50407196073714	2.62050471464515\\
1.51184410942881	2.61515305224684\\
1.51941941835957	2.60919401406851\\
1.52684128198338	2.60255111996953\\
1.53414692742136	2.59512463047509\\
1.54136743812141	2.58678607029672\\
1.54852712580497	2.57737046498136\\
1.55564201997255	2.56666538876986\\
1.56271706952095	2.55439544060589\\
1.56974136514771	2.54020000811172\\
1.57668020608856	2.52360095861468\\
1.58346198732936	2.50395488348667\\
1.58995636420672	2.48038113089167\\
1.59593735844811	2.45165102966057\\
1.60101980835639	2.41601348883995\\
1.60454742630167	2.37091401188447\\
1.60539083148236	2.31253186959236\\
1.60157464345816	2.23500407079215\\
1.58957687332405	2.12911610833666\\
1.56301143478508	1.9801432277843\\
1.5102532766994	1.76463817519863\\
1.41081821954095	1.44727205128183\\
1.23288596616825	0.984345860164834\\
0.943081411046606	0.352273078517919\\
0.54428566688769	-0.391140171268018\\
0.107904047733924	-1.09554150842607\\
-0.276044760851313	-1.63657136712392\\
-0.569184196744573	-1.99892328105359\\
-0.77995959606827	-2.22811087339792\\
-0.930346758973861	-2.37209991875548\\
-1.03947457874975	-2.46402444534223\\
-1.12072825956013	-2.52406730683613\\
-1.18290548049581	-2.56416308963738\\
-1.23174035972648	-2.59142346698548\\
-1.27101822274564	-2.61018011835085\\
-1.30329024585911	-2.62314182793299\\
-1.3303159546295	-2.63204789657639\\
-1.3533370722357	-2.63804298620778\\
-1.37324926014167	-2.64189713696169\\
-1.39071186785787	-2.64413816644854\\
-1.40621953027101	-2.64513332474643\\
-1.42014985394465	-2.64514082425715\\
-1.43279581960538	-2.64434303531236\\
-1.44438821946604	-2.64286824669118\\
-1.45511147030128	-2.64080511765368\\
-1.46511494034323	-2.63821233899968\\
-1.47452118274717	-2.63512506370108\\
-1.48343199755177	-2.63155908346106\\
-1.49193294054518	-2.62751335986359\\
-1.50009669731347	-2.62297127778391\\
-1.50798560497077	-2.61790082096134\\
-1.51565350807117	-2.61225374145689\\
-1.52314706266683	-2.6059636828706\\
-1.53050654150842	-2.59894310284286\\
-1.5377661337854	-2.59107870464897\\
-1.54495366368046	-2.5822249073791\\
-1.55208955895927	-2.57219462609524\\
-1.55918476169539	-2.56074624579452\\
-1.56623705092564	-2.54756507140195\\
-1.573224876085	-2.53223657740225\\
-1.58009716105139	-2.51420721748646\\
-1.5867564075857	-2.49272594887438\\
-1.59303037348538	-2.46675518971048\\
-1.59862377892076	-2.43483222422706\\
-1.60303420978513	-2.39484846990763\\
-1.60540220859596	-2.34368979276673\\
-1.60423757597317	-2.27663824466107\\
-1.59690900318884	-2.18636364362759\\
-1.57868173742978	-2.06123237273783\\
-1.54093102719399	-1.88261546525284\\
-1.46810249276303	-1.62137238292075\\
-1.33400160379627	-1.23656406937439\\
-1.10326361017691	-0.688505376792297\\
-0.754348062540462	0.0136382320891431\\
-0.324477087888203	0.758366136414363\\
0.0945266002362541	1.38980765530617\\
0.434093811825414	1.83775491680209\\
0.683522071725052	2.12686100309367\\
0.8613696880509	2.30832381252633\\
0.989104532384684	2.42306408013362\\
1.08294774530413	2.49713991499516\\
1.15378656170739	2.54608051781412\\
1.20871846719522	2.57907857516757\\
1.25239176239966	2.60166685117974\\
1.28790527516386	2.61725928393709\\
1.3173715371126	2.62801946673208\\
1.34226451525509	2.63535427264229\\
1.36363601743509	2.64020015530037\\
1.3822526363932	2.64319329085787\\
1.39868416267362	2.64477324354987\\
1.41336186657461	2.6452476547433\\
};
\addplot [color=mycolor1, forget plot]
  table[row sep=crcr]{%
6.84088886676008	4.99913208029814\\
6.96761917857306	4.99527446465504\\
7.06755331315122	4.98584932276291\\
7.14802261607397	4.97305911339979\\
7.21401975286968	4.95821728261124\\
7.26902978728066	4.94211411572269\\
7.31554052669344	4.92522402903217\\
7.35536390507287	4.90782600632662\\
7.38984331799043	4.89007512858044\\
7.41999042362753	4.87204581526907\\
7.4465772713189	4.85375826900637\\
7.47019948081834	4.83519467923873\\
7.49132023875733	4.81630899908268\\
7.51030129726572	4.79703254482266\\
7.52742495531667	4.77727674831561\\
7.54290961255317	4.75693383296865\\
7.5569205823831	4.73587582594179\\
7.56957724233974	4.71395207172416\\
7.58095716552604	4.69098521945959\\
7.59109753889791	4.66676548027988\\
7.5999938747872	4.64104276040482\\
7.60759571140593	4.61351603914356\\
7.61379862175913	4.58381903830862\\
7.6184313363036	4.55150076352845\\
7.62123602336253	4.51599879927524\\
7.62183858316341	4.47660216315016\\
7.61970388775799	4.43239882863183\\
7.61406769191097	4.38220029407415\\
7.60383144924698	4.32443108425763\\
7.58739662284487	4.25696352991992\\
7.56239769899336	4.1768652550648\\
7.52526100347198	4.08000429228032\\
7.47045574746095	3.96041707443767\\
7.38918722760005	3.80927485345324\\
7.26705835616366	3.61316623865749\\
7.07981294985811	3.35124052435176\\
6.78562883880548	2.99063248318109\\
6.31206390072326	2.4801833081198\\
5.53922240998824	1.74661149592949\\
4.30165628022422	0.712525630456547\\
2.48557351533814	-0.620688304855937\\
0.26402085414709	-2.04765566869058\\
-1.8687359569853	-3.2372039172557\\
-3.52815139697121	-4.03095814698191\\
-4.67630145924028	-4.49358687649038\\
-5.44305045173526	-4.7473157681752\\
-5.96001472258145	-4.8826087065026\\
-6.31831910812099	-4.95237714613542\\
-6.57479182115228	-4.98555661624868\\
-6.76422057514554	-4.99788833910659\\
-6.90818573778395	-4.99808676492652\\
-7.02040997793232	-4.99109349776442\\
-7.10986595001333	-4.97977722865751\\
-7.18258420021252	-4.96583597628711\\
-7.24272375925819	-4.95028803020279\\
-7.29322133832692	-4.93374643429515\\
-7.33619506303057	-4.91657652308681\\
-7.37320185482429	-4.8989885047452\\
-7.4054053934111	-4.88109297144779\\
-7.43368810964642	-4.86293468270704\\
-7.45872731796898	-4.84451327772963\\
-7.48104784848783	-4.82579590645338\\
-7.50105893333677	-4.80672470492933\\
-7.51908030042781	-4.78722084610942\\
-7.53536068244905	-4.76718618350808\\
-7.55009083275759	-4.74650306099718\\
-7.56341240252416	-4.72503256788109\\
-7.57542352380168	-4.70261130382071\\
-7.58618156519308	-4.67904653784841\\
-7.59570321451892	-4.65410946616483\\
-7.60396174412403	-4.62752606413669\\
-7.61088097851963	-4.59896475411069\\
-7.61632504963485	-4.56801972508052\\
-7.62008240431654	-4.53418817181828\\
-7.62184158267811	-4.49683885655334\\
-7.62115478012173	-4.4551680487666\\
-7.61738273052627	-4.40813675180086\\
-7.60961026522076	-4.35437963155908\\
-7.59651464829314	-4.29207025773525\\
-7.57615588237226	-4.21871742268348\\
-7.54563462676158	-4.13085028486\\
-7.50051934440824	-4.02352022392346\\
-7.43386033241791	-3.88949461361287\\
-7.33444657504917	-3.71792630922736\\
-7.18365457895993	-3.4921354354706\\
-6.94970302787466	-3.18596166290699\\
-6.57746887226027	-2.75822063891058\\
-5.97271943598497	-2.14656008491193\\
-4.98915068767605	-1.27049336483544\\
-3.4647769938628	-0.0761671675852215\\
-1.39846942708153	1.34372974891551\\
0.845794119376638	2.6886884672571\\
2.76685927831793	3.68281754683751\\
4.15924238449614	4.29571110156509\\
5.09825681476418	4.63989472459928\\
5.72610420536212	4.82573521306198\\
6.15479889280274	4.92346550355785\\
6.45671715430494	4.97233824045081\\
6.6763049251593	4.99366738199783\\
6.84088886676008	4.99913208029814\\
};
\addplot [color=mycolor2, forget plot]
  table[row sep=crcr]{%
0.523495985188513	2.21095812992688\\
0.543228512121635	2.21033588094888\\
0.562220859464411	2.20852524916379\\
0.58062348639751	2.20558242663241\\
0.59857107624324	2.20152953227212\\
0.616186173211566	2.19635696258384\\
0.6335821876191	2.19002390927179\\
0.650865900163185	2.18245712819836\\
0.668139552271565	2.1735478870409\\
0.685502566909632	2.16314685563153\\
0.703052896168851	2.15105651310548\\
0.720887929723092	2.13702040601345\\
0.739104808628203	2.12070827115511\\
0.757799850973994	2.10169559548784\\
0.777066575245817	2.07943556864067\\
0.796991447022335	2.05322052138817\\
0.817645880480452	2.02212875644123\\
0.839072043534282	1.98495110624118\\
0.861258399072199	1.94008964755294\\
0.884098302558147	1.88541917667427\\
0.907320920341571	1.81810169992184\\
0.930377965797729	1.73434916367601\\
0.952263194491732	1.62914930722371\\
0.971239355789409	1.4960241572878\\
0.984465785527444	1.3270165243409\\
0.987598438751026	1.11333546856789\\
0.974634416241003	0.847385341338023\\
0.938591949871787	0.526853328745189\\
0.873710716112844	0.160179721753477\\
0.778896374237475	-0.230310041134964\\
0.660066987294904	-0.612784015588797\\
0.528630790965447	-0.957467547825973\\
0.396777493547414	-1.24689744957127\\
0.273413498440531	-1.47776602135199\\
0.163031486303281	-1.6561500367302\\
0.0667094469319491	-1.79176889067705\\
-0.0163980118246023	-1.89431828124055\\
-0.0879334239880009	-1.97194566097251\\
-0.149699133735178	-2.0309668534891\\
-0.203365362938089	-2.07609011987583\\
-0.250369196107044	-2.11076539943113\\
-0.291903635828865	-2.13750761633118\\
-0.328942641967419	-2.15815283636643\\
-0.362276439623747	-2.17404814008715\\
-0.392546170745644	-2.18618812408863\\
-0.420273998952583	-2.19531176521621\\
-0.445887865276266	-2.20197089826355\\
-0.469741307298921	-2.20657867192898\\
-0.492129130424326	-2.20944393400595\\
-0.51329974773488	-2.21079569114687\\
-0.533464906173632	-2.21080050167315\\
-0.552807387145015	-2.20957476137215\\
-0.571487146611918	-2.2071932176451\\
-0.589646255366341	-2.20369460979581\\
-0.607412915652489	-2.19908502012294\\
-0.624904762922485	-2.19333928744042\\
-0.642231607068395	-2.18640064970422\\
-0.6594977214769	-2.1781786212345\\
-0.676803745849656	-2.16854495220916\\
-0.69424822429159	-2.15732734415148\\
-0.711928746334305	-2.14430038351473\\
-0.729942584674714	-2.12917287962482\\
-0.748386612912459	-2.11157041820197\\
-0.767356112690995	-2.09101142034797\\
-0.786941798226588	-2.06687426702275\\
-0.807223923936087	-2.0383520342906\\
-0.828261576687304	-2.00439000913586\\
-0.850073992278704	-1.96359939714325\\
-0.872608674221088	-1.91413867617342\\
-0.895687814770749	-1.85355268365995\\
-0.918919604020422	-1.77856110763088\\
-0.941554581125384	-1.6847986646771\\
-0.962261669610689	-1.56654291960342\\
-0.978803465833006	-1.41655046193693\\
-0.987632129793905	-1.22629992368928\\
-0.983559095423338	-0.987224987260853\\
-0.95992427292537	-0.693728548593595\\
-0.9099752585336	-0.348209238467667\\
-0.829844399657837	0.0339699757827997\\
-0.721851236154917	0.424610843801142\\
-0.595120868130539	0.791297011438935\\
-0.462094678343811	1.10958947459289\\
-0.333655905694098	1.3694178283083\\
-0.216470558877756	1.57294120700516\\
-0.113136514165607	1.72867233978889\\
-0.0235974610654696	1.84662458847378\\
0.0534986437817769	1.93581096326671\\
0.119928196544504	2.00345410619619\\
0.177449005002451	2.05502469963258\\
0.22761999764002	2.09455801098953\\
0.271754484623795	2.12499999714405\\
0.310931796261443	2.14849852256881\\
0.346029505257319	2.16662510211742\\
0.377759205625238	2.18053628765935\\
0.4066991226322	2.1910887605286\\
0.433321539256857	2.19892078291678\\
0.458014996438643	2.2045097844662\\
0.481101933251353	2.20821316547787\\
0.502852595133849	2.21029729226656\\
0.523495985188513	2.21095812992688\\
};
\addplot [color=mycolor1, forget plot]
  table[row sep=crcr]{%
6.86250796792293	5.01216327431788\\
6.98926231728329	5.00830501201048\\
7.0892040971588	4.99887920200613\\
7.1696723709199	4.98608919075086\\
7.2356639362686	4.97124863638639\\
7.29066611921641	4.95514778446123\\
7.33716798431924	4.93826093255989\\
7.37698216855827	4.92086693575256\\
7.41145245879218	4.90312076171995\\
7.44159072664735	4.8850967394826\\
7.46816913277388	4.86681500433133\\
7.49178334982416	4.84825769934232\\
7.51289658462578	4.82937874950955\\
7.53187059220708	4.8101094587785\\
7.5489876665659	4.790361260487\\
7.56446620043385	4.7700263919257\\
7.57847150229701	4.74897690563372\\
7.59112294956573	4.72706218251218\\
7.60249812227665	4.70410491905285\\
7.61263422345491	4.67989538484688\\
7.62152679279647	4.65418355600433\\
7.62912540959612	4.62666849341812\\
7.63532570446324	4.5969840120384\\
7.63995648517121	4.56467922102285\\
7.64276002044572	4.5291918154402\\
7.64336233688809	4.4898109229457\\
7.6412284595908	4.44562461062275\\
7.63559431755128	4.39544442207369\\
7.62536153911076	4.33769481597923\\
7.60893170603898	4.27024782131662\\
7.58393922779236	4.19017027676805\\
7.54680982482963	4.09332844236527\\
7.49201078226915	3.97375494172735\\
7.41074223820117	3.82261292741202\\
7.28859398844889	3.62647362156677\\
7.10127677227243	3.364448227552\\
6.80688429362408	3.0035862582056\\
6.33275976369098	2.49253691535988\\
5.55848013628344	1.75760613116161\\
4.31749294271232	0.720672658969268\\
2.49461507568364	-0.617516179954371\\
0.263345568688976	-2.05071979992666\\
-1.8785328447942	-3.245362799024\\
-3.5438740067486	-4.04196088661383\\
-4.69511885692381	-4.50584250495883\\
-5.46335414679059	-4.76006617664413\\
-5.98102449960256	-4.89554530075052\\
-6.33966919335148	-4.96538068726368\\
-6.59630802588342	-4.99858199844547\\
-6.78581731532883	-5.01091915477763\\
-6.92981960632722	-5.01111774157195\\
-7.04205839289073	-5.00412363463622\\
-7.1315169872655	-4.99280707632652\\
-7.20423156477816	-4.97886655640214\\
-7.26436420644563	-4.96332041832582\\
-7.31485330757953	-4.94678161368626\\
-7.35781793374445	-4.92961534825615\\
-7.3948155316173	-4.9120317075125\\
-7.42701007098371	-4.8941411814341\\
-7.45528413804713	-4.87598845082904\\
-7.48031512524895	-4.85757309838982\\
-7.50262789680656	-4.83886223714095\\
-7.52263169492403	-4.81979798318208\\
-7.54064624555851	-4.8003015042472\\
-7.55692027486454	-4.78027466167686\\
-7.57164452987791	-4.75959981905813\\
-7.58496065894079	-4.73813809662397\\
-7.59696679724758	-4.71572613590784\\
-7.60772032464288	-4.69217125881124\\
-7.617237950381	-4.66724472566727\\
-7.62549298070531	-4.64067258756152\\
-7.63240928909447	-4.61212335426371\\
-7.63785107451925	-4.58119131334615\\
-7.64160687227093	-4.54737376726125\\
-7.64336533556924	-4.51003958990812\\
-7.64267879968886	-4.46838515534314\\
-7.63890816347927	-4.42137154178832\\
-7.63113843688252	-4.3676334154589\\
-7.61804704135371	-4.30534418336554\\
-7.59769402623181	-4.23201213696242\\
-7.5671797709553	-4.1441652432129\\
-7.52207161887795	-4.03685227518945\\
-7.45541668020137	-3.90283505324939\\
-7.35599568951142	-3.73125459015708\\
-7.2051643343779	-3.50540534944489\\
-6.97108861704537	-3.19907014224744\\
-6.59851096802524	-2.77093651021558\\
-5.99285900701924	-2.15836741926386\\
-5.00703994044869	-1.28030472689602\\
-3.4776900948767	-0.0820928732758431\\
-1.40284278372905	1.34366140761687\\
0.85134661085044	2.69457031415574\\
2.78004453523949	3.69265884948226\\
4.17678575758048	4.3074785408114\\
5.11795459925128	4.65245605656726\\
5.74682533859377	4.83860148115811\\
6.17600915738681	4.93644425192202\\
6.47816500031136	4.98535593273646\\
6.69786876188027	5.00669659737774\\
6.86250796792293	5.01216327431788\\
};
\addplot [color=mycolor2, forget plot]
  table[row sep=crcr]{%
-0.733575921492504	1.31460374341902\\
-0.632596108039127	1.3113410779402\\
-0.520663702370314	1.3005890110674\\
-0.397263074514887	1.2807737843542\\
-0.262341623792816	1.25022656161158\\
-0.11652851656645	1.20733710372774\\
0.0386574440584269	1.15078286280803\\
0.200679110745031	1.07981345555107\\
0.366013660074528	0.994532769103677\\
0.530406661832258	0.896088323134107\\
0.689338991169707	0.786677146174762\\
0.838605846799504	0.6693274655713\\
0.974845201924293	0.547500688560829\\
1.095867236989	0.424630275227481\\
1.20072115263929	0.303728526329512\\
1.28953654070531	0.187146049008567\\
1.36323727351282	0.076498504484224\\
1.42322986832842	-0.0272778744932675\\
1.47113557370203	-0.12379976936684\\
1.50859537167969	-0.213089685086319\\
1.53714783701748	-0.295440389365299\\
1.5581653382286	-0.371305494710742\\
1.57283062677761	-0.441218943712308\\
1.582138147318	-0.505740022815961\\
1.58690854061136	-0.565418369279737\\
1.58780877458732	-0.620773443737429\\
1.58537339560882	-0.672283856281868\\
1.58002447744766	-0.720383054726347\\
1.57208914862459	-0.765458892081875\\
1.56181433196347	-0.807855384239133\\
1.54937872858848	-0.847875547553941\\
1.53490226248587	-0.88578460612647\\
1.51845326346976	-0.921813123485746\\
1.50005366306014	-0.956159781172203\\
1.47968244310377	-0.988993626619511\\
1.45727753006904	-1.02045566565407\\
1.43273627943341	-1.05065969493691\\
1.40591465056035	-1.0796922657676\\
1.37662513722201	-1.10761164827412\\
1.34463349691367	-1.13444562723217\\
1.30965431967807	-1.16018790954467\\
1.27134550445155	-1.18479286088385\\
1.22930178423368	-1.20816821926737\\
1.18304758612331	-1.23016536542467\\
1.13202976680905	-1.25056668239289\\
1.07561118349157	-1.26906954588727\\
1.01306671797197	-1.28526661767614\\
0.943584354520575	-1.29862247668178\\
0.866275297135439	-1.30844739031735\\
0.780198905747261	-1.31387044835238\\
0.684410242848602	-1.31381664823958\\
0.57803963763987	-1.3069960618567\\
0.460413542584466	-1.29191779453278\\
0.331221736278588	-1.26694601494493\\
0.190724547621685	-1.23041717488425\\
0.0399730394022008	-1.18083190374211\\
-0.119012709766546	-1.11711650257582\\
-0.283185610920674	-1.03891590754923\\
-0.448610699077059	-0.946841825316007\\
-0.610832441756026	-0.842579774836903\\
-0.765415585653945	-0.728782995388376\\
-0.908520369919128	-0.608753247524889\\
-1.03734588454356	-0.485994123255349\\
-1.15032942969175	-0.363769276058191\\
-1.2470907949759	-0.244779334817512\\
-1.32819617988291	-0.131007247923902\\
-1.39484812893842	-0.023716225666711\\
-1.44858972346348	0.0764520407182007\\
-1.49107166189482	0.169335335020153\\
-1.52389492325475	0.255106777306546\\
-1.54852001943411	0.334151396951494\\
-1.56622557067635	0.406971321366467\\
-1.57809896662283	0.47411865229053\\
-1.58504544890288	0.536151200579106\\
-1.58780615864932	0.593605370382616\\
-1.58697923972185	0.646981064609402\\
-1.5830406446606	0.696734560951497\\
-1.57636295534998	0.743276395173477\\
-1.567231530928	0.786972193951703\\
-1.5558578502988	0.828145082622651\\
-1.54239019348571	0.867078777152041\\
-1.52692191983323	0.904020796901718\\
-1.50949762482587	0.939185446898803\\
-1.49011743515793	0.972756349052911\\
-1.46873965915509	1.00488837603656\\
-1.4452819609737	1.03570887660166\\
-1.41962118013877	1.06531808828617\\
-1.39159187784586	1.09378861980928\\
-1.36098366219232	1.12116385507696\\
-1.32753733130181	1.14745508600307\\
-1.2909398841505	1.17263712414442\\
-1.25081849666718	1.19664207428242\\
-1.20673366580052	1.21935088267\\
-1.15817191813318	1.24058221179928\\
-1.1045388085493	1.2600781683996\\
-1.04515346275685	1.27748647074099\\
-0.979246726695719	1.29233887058039\\
-0.905966162524658	1.30402618397681\\
-0.824392727841595	1.31177134385085\\
-0.733575921492505	1.31460374341902\\
};
\addplot [color=mycolor1, forget plot]
  table[row sep=crcr]{%
4.82273228002968	3.83720887290188\\
4.95224699965661	3.83325358762484\\
5.05611778584741	3.82344889459213\\
5.14089177706792	3.80996895800022\\
5.21118152874184	3.79415795626632\\
5.27029535935223	3.77685073784405\\
5.32064745275308	3.75856365681821\\
5.36402814716335	3.73960996602969\\
5.40178480740073	3.72017066332652\\
5.43494478132041	3.70033854357982\\
5.46430026698535	3.68014580834553\\
5.49046771058867	3.65958136247208\\
5.51392987214387	3.63860147624619\\
5.53506587201003	3.61713604058954\\
5.55417272393345	3.59509176286132\\
5.57148068066435	3.57235310028877\\
5.58716392840006	3.54878136750211\\
5.60134761773018	3.52421220091616\\
5.61411181537904	3.49845136332545\\
5.62549263681689	3.47126869055441\\
5.63548052402979	3.4423897885073\\
5.64401531919005	3.41148485356697\\
5.65097740203965	3.3781536756374\\
5.65617363775665	3.34190544085221\\
5.65931612011444	3.30213130495984\\
5.65999052751267	3.2580667421089\\
5.65760906390394	3.20873920256256\\
5.65133996804031	3.15289434124797\\
5.64000063885971	3.0888905337548\\
5.62189313791285	3.01454582992094\\
5.59454673765195	2.92691277479322\\
5.55430807172166	2.82194311606146\\
5.49567845866715	2.69398488798758\\
5.41023066111837	2.5350299931434\\
5.28483775577398	2.33361496287792\\
5.09884485169681	2.07332964712349\\
4.81990247355855	1.73120544840765\\
4.39920671023066	1.27738149278302\\
3.77091717341521	0.680385231150418\\
2.86995021215103	-0.0733906543835908\\
1.68696680769167	-0.942802236466377\\
0.335242594647828	-1.8113829036668\\
-0.975850751363337	-2.54218196592073\\
-2.07722399358059	-3.06831338957807\\
-2.91971033690261	-3.40726568298498\\
-3.53681121128057	-3.61117424376579\\
-3.98482032351319	-3.72825990601219\\
-4.31328463216064	-3.79213151854898\\
-4.55853705626038	-3.82381182226538\\
-4.74555312990996	-3.83595915425882\\
-4.89120350145804	-3.8361435067725\\
-5.00692397817158	-3.82892209014364\\
-5.10056696603678	-3.81706943662609\\
-5.17761549046581	-3.80229342988045\\
-5.24196729344291	-3.7856532643666\\
-5.29644256945701	-3.76780633590028\\
-5.34311629806293	-3.74915634157748\\
-5.38353891570215	-3.72994353477927\\
-5.41888515330459	-3.71030051495013\\
-5.4500560003726	-3.69028708231215\\
-5.4777495889861	-3.66991210828325\\
-5.50251111476878	-3.649147162936\\
-5.52476836016115	-3.62793476012899\\
-5.54485713109196	-3.60619295469908\\
-5.56303946198115	-3.58381733326525\\
-5.5795164823624	-3.56068099779297\\
-5.5944371838535	-3.53663284196546\\
-5.60790385994208	-3.5114941991508\\
-5.61997463426678	-3.48505375427219\\
-5.63066318889856	-3.45706042796672\\
-5.63993550533676	-3.42721373096009\\
-5.64770308990145	-3.39515081727613\\
-5.65381171345032	-3.36042909447345\\
-5.65802406972734	-3.32250271645461\\
-5.65999381797042	-3.28069049628806\\
-5.65922701260224	-3.23413158650851\\
-5.65502458055458	-3.18172344909521\\
-5.64639567479162	-3.12203380256001\\
-5.63192534913398	-3.0531737927659\\
-5.6095692067967	-2.97261265973941\\
-5.57632924476756	-2.87690331670947\\
-5.52773361511053	-2.76127190280889\\
-5.45699012819906	-2.61900185992461\\
-5.35359971688748	-2.44051986601403\\
-5.20110634735667	-2.21209661788149\\
-4.97360823102457	-1.91421992438641\\
-4.63105640072421	-1.52031875631979\\
-4.11552958912839	-0.998417660583009\\
-3.35732869804741	-0.322290494927053\\
-2.30994690716635	0.499343820936391\\
-1.01998239562171	1.38648817003772\\
0.338534070191457	2.20051889899191\\
1.55857214384513	2.83121730922997\\
2.52985081867727	3.25812564683877\\
3.25294571500252	3.52276759202078\\
3.7785533816478	3.67812488620809\\
4.16142077655795	3.76529004425864\\
4.44452431404327	3.81105331235671\\
4.65811579031653	3.83176414287705\\
4.82273228002968	3.83720887290188\\
};
\addplot [color=mycolor2, forget plot]
  table[row sep=crcr]{%
-0.409091368008181	0.847807163764681\\
0.0363872777318144	0.833551248542955\\
0.4817250507643	0.791124827733238\\
0.888249773018939	0.726348923044037\\
1.23188738207708	0.649090012089854\\
1.50689824214324	0.568693910692423\\
1.71996212285585	0.491453974327295\\
1.88260397700221	0.420525765118745\\
2.00639290543533	0.356905955764647\\
2.10100336756632	0.300416597818771\\
2.17388392469238	0.250362608748835\\
2.2305535888983	0.205892398119981\\
2.27503824215836	0.166169537837721\\
2.31026475499303	0.130441927877661\\
2.33836800502053	0.0980607706484964\\
2.36091562510899	0.0684772729283638\\
2.37906807968813	0.0412308829428764\\
2.39369148912165	0.0159354581143135\\
2.40543701739583	-0.00773391989835962\\
2.41479689131733	-0.0300525119747493\\
2.42214411567697	-0.0512556852703417\\
2.42776075688613	-0.071547013686774\\
2.43185813181773	-0.0911049652894258\\
2.43459118001494	-0.110088443142424\\
2.43606856940276	-0.128641425045538\\
2.43635958053514	-0.146896915618108\\
2.43549845661457	-0.164980392485317\\
2.4334866433849	-0.183012901412597\\
2.43029313715	-0.201113934326957\\
2.42585298260436	-0.219404209021938\\
2.4200637916992	-0.238008458984596\\
2.41277996830448	-0.25705833460852\\
2.40380409637995	-0.276695510518368\\
2.39287465074495	-0.297075083466383\\
2.37964877707909	-0.318369323368185\\
2.36367830171839	-0.34077179180409\\
2.34437628718835	-0.364501740005801\\
2.32097022868314	-0.389808489635322\\
2.29243623898783	-0.416975085835033\\
2.25740613658626	-0.446319707981462\\
2.21403617770271	-0.478191786707268\\
2.15982264227314	-0.512956889308037\\
2.09134735282723	-0.550959152568429\\
2.00394095050149	-0.592440809000024\\
1.89127791599731	-0.637383660783571\\
1.74499861443862	-0.685218755545888\\
1.5546537323195	-0.73434281979572\\
1.30865514096727	-0.781435534003174\\
0.99738778007668	-0.820806041810157\\
0.619400593165738	-0.84450103591156\\
0.188952459819122	-0.844293417387755\\
-0.261696494489053	-0.815644863467068\\
-0.69179486850701	-0.760951163274546\\
-1.0686755863563	-0.688639413498638\\
-1.37774703986944	-0.608809307923861\\
-1.6204938645077	-0.529418433597329\\
-1.80683005131328	-0.455100938365687\\
-1.94869725624527	-0.387795309093099\\
-2.05682977985848	-0.327806702640065\\
-2.13977481288561	-0.27463742208824\\
-2.20396352356178	-0.227483314092969\\
-2.25411439088667	-0.18548691140633\\
-2.29365976976601	-0.147849323331971\\
-2.32509766189118	-0.113869568607998\\
-2.35025571162444	-0.0829500057831151\\
-2.37048128221355	-0.0545876561121953\\
-2.38677592984353	-0.0283608978100306\\
-2.39989003031857	-0.00391575783331045\\
-2.41038943186043	0.0190465212089434\\
-2.418702593801	0.0407801420528644\\
-2.42515408608277	0.0615036573149588\\
-2.42998848383307	0.0814073298086471\\
-2.43338741492735	0.100659214206848\\
-2.43548164095655	0.119410218566654\\
-2.43635944668375	0.137798375621783\\
-2.4360721894412	0.15595252101642\\
-2.43463755492039	0.173995546059013\\
-2.43204083528246	0.192047368630618\\
-2.42823435741328	0.210227747942568\\
-2.42313501838075	0.228659056246865\\
-2.4166197095217	0.24746911208251\\
-2.40851820720412	0.266794173188343\\
-2.39860284992443	0.286782179523592\\
-2.38657397207145	0.307596322070028\\
-2.37203957427972	0.329418980482539\\
-2.35448700745897	0.352456002023506\\
-2.33324343183192	0.376941147473959\\
-2.30742034857198	0.403140234953994\\
-2.2758354289806	0.431353934491819\\
-2.23690205457826	0.46191705234338\\
-2.18847352812088	0.495190032571688\\
-2.12762571293684	0.53153448224751\\
-2.05036208875032	0.571257487717239\\
-1.95123794787654	0.614497646681818\\
-1.8229482566059	0.661008445474991\\
-1.65605455441742	0.709778029129612\\
-1.43931539355192	0.758438100532923\\
-1.16155468723619	0.802541670170139\\
-0.816285878889351	0.835160767177662\\
-0.409091368008183	0.84780716376468\\
};
\addplot [color=mycolor1, forget plot]
  table[row sep=crcr]{%
3.32646336156982	3.11950096423544\\
3.4673413537824	3.11517424948534\\
3.58374016942398	3.10417038463928\\
3.68108791118803	3.08867944763077\\
3.76345208436014	3.07014401802894\\
3.83390080001395	3.04951196054361\\
3.89476794811634	3.02740122541234\\
3.94784519226144	3.00420724774397\\
3.994520472178	2.98017315169741\\
4.03587809210986	2.95543579797172\\
4.07277136817759	2.93005601422601\\
4.1058756325731	2.90403833369911\\
4.13572707628554	2.87734364507239\\
4.16275126496685	2.84989692604319\\
4.18728400030374	2.82159143566068\\
4.20958637530507	2.79229021225691\\
4.22985527769723	2.76182536074797\\
4.24823015242909	2.72999534805555\\
4.26479648414853	2.6965603133532\\
4.27958615827755	2.6612352082174\\
4.29257456593334	2.6236803840576\\
4.3036739942792	2.58348901526293\\
4.31272244412135	2.54017045864858\\
4.31946648004315	2.49312826914687\\
4.32353595839411	2.44163107470658\\
4.3244073661675	2.38477380401515\\
4.32135084517486	2.32142579094374\\
4.31335348121521	2.2501609803709\\
4.2990076892852	2.16916380032119\\
4.27634797012147	2.07610236547164\\
4.24261137132037	1.9679591029565\\
4.19388654870193	1.84080937778487\\
4.12460537108111	1.68954583103729\\
4.02682820492829	1.50757102235393\\
3.88930626934119	1.28654804716912\\
3.69644156479011	1.01645458273428\\
3.42765412937966	0.686491934124628\\
3.05850682441724	0.287838803306029\\
2.5661047161497	-0.180634561274986\\
1.94114780808001	-0.704170530007748\\
1.20371230481806	-1.2466740267169\\
0.4093959035994	-1.75723902570244\\
-0.367691869692907	-2.19015566537185\\
-1.06586467664211	-2.52327489742262\\
-1.65471167076941	-2.75980784569078\\
-2.13231136223699	-2.91734668343342\\
-2.51239515758807	-3.01649960533499\\
-2.81340129671657	-3.07491758745073\\
-3.05271283203895	-3.1057586216582\\
-3.24466165286656	-3.11818053282271\\
-3.40036685358521	-3.11834771041054\\
-3.52822452426209	-3.11034882452227\\
-3.63451407501058	-3.09688160317952\\
-3.72393128262089	-3.07972373122979\\
-3.80000535223279	-3.06004519381678\\
-3.86540922520523	-3.03861246415104\\
-3.92218542946281	-3.01592151010757\\
-3.97190915567376	-2.99228458614822\\
-4.01580594520563	-2.96788707091709\\
-4.05483690688043	-2.94282478363867\\
-4.08976073081524	-2.91712844319707\\
-4.12117904297665	-2.8907795265247\\
-4.14956968723582	-2.86372024646818\\
-4.17531113792668	-2.8358593807395\\
-4.19870026860253	-2.80707503683381\\
-4.21996500508263	-2.7772150024272\\
-4.23927288051749	-2.74609502346451\\
-4.25673611981222	-2.71349511828296\\
-4.2724135596746	-2.67915383784852\\
-4.28630941748958	-2.64276019064817\\
-4.29836861884402	-2.60394274067567\\
-4.30846803824527	-2.5622551323156\\
-4.31640254833021	-2.51715696644361\\
-4.32186413763414	-2.46798850942894\\
-4.32441144040951	-2.41393711139262\\
-4.32342566597318	-2.35399237957271\\
-4.31804688193366	-2.28688602589242\\
-4.30708154515252	-2.21101082677981\\
-4.28886759794017	-2.12431132081434\\
-4.26107676054016	-2.02413701363562\\
-4.2204244039374	-1.90704797673904\\
-4.16224618983675	-1.76856567553848\\
-4.07989195483255	-1.60287585146393\\
-3.96389676347797	-1.40253200632372\\
-3.80096093637043	-1.15831208656481\\
-3.57300992369185	-0.859605020688629\\
-3.25719868151819	-0.496092458045106\\
-2.82880022614507	-0.0618785465303164\\
-2.26979422053179	0.437272257365636\\
-1.58377242826251	0.976075726774439\\
-0.809107763453254	1.50920352669317\\
-0.0142167475127106	1.98545362547987\\
0.729332634606697	2.3694845312444\\
1.37460171844811	2.65269650078429\\
1.90677228309312	2.8471362738584\\
2.33340934238541	2.97301567219907\\
2.67163537044435	3.04987357854242\\
2.93979454170461	3.09313057854113\\
3.1538417162047	3.11382835567295\\
3.32646336156982	3.11950096423544\\
};
\addplot [color=mycolor2, forget plot]
  table[row sep=crcr]{%
0.560167015110667	0.853244053900496\\
1.06707393089475	0.837614568300724\\
1.47679824178267	0.799019827265189\\
1.78965952386821	0.749429636354734\\
2.02226091251516	0.697266211987803\\
2.1941524994748	0.647072601752934\\
2.32194042341777	0.600766346427937\\
2.41808474955201	0.558840228174232\\
2.49146704461956	0.521121780938159\\
2.54830649589018	0.487177503668587\\
2.59295308266754	0.456506966732256\\
2.62847138587656	0.428627544501524\\
2.65704463224663	0.403106100218799\\
2.68024800218523	0.379566373857723\\
2.69923259434205	0.357686050860287\\
2.71484974774595	0.337190063938117\\
2.72773576375632	0.317843067151704\\
2.73837023202117	0.299442302209988\\
2.74711659991187	0.281811278308091\\
2.75425064661735	0.264794329660224\\
2.75998059101263	0.24825196618472\\
2.76446130416895	0.232056879954383\\
2.76780426913872	0.216090458887398\\
2.77008437581847	0.200239663848504\\
2.77134425673718	0.184394133581673\\
2.77159659503874	0.168443387858922\\
2.77082462449488	0.152273999962742\\
2.76898086106356	0.135766603317522\\
2.76598393012495	0.118792582172281\\
2.76171315775648	0.101210270872707\\
2.75600034892523	0.0828604480786408\\
2.74861784118104	0.0635608583348252\\
2.73926144240195	0.0430994204588267\\
2.72752614849146	0.0212256878847664\\
2.71287145602787	-0.00235998808074396\\
2.69457142269093	-0.0280202621363476\\
2.67164205333901	-0.0561983212534008\\
2.64273460263223	-0.0874398238642967\\
2.60597729363336	-0.122419446196449\\
2.55873899524786	-0.16197063710764\\
2.49727644824871	-0.207112973876854\\
2.41621472705903	-0.25906134742367\\
2.30781294906669	-0.319178239043526\\
2.16103090376812	-0.388782584104714\\
1.96066602867023	-0.46864379618366\\
1.68754270561763	-0.557892575256645\\
1.32214543016994	-0.652168158065221\\
0.855139672833289	-0.741662557912\\
0.304000364149641	-0.811679733978347\\
-0.278731448373542	-0.848741899301714\\
-0.824668087485252	-0.849099418132795\\
-1.28465122185451	-0.820383506369671\\
-1.6444858391152	-0.77497166528901\\
-1.9147387932445	-0.72330745184811\\
-2.11466040306014	-0.671757944162537\\
-2.26268638490528	-0.623380385091793\\
-2.37333988344336	-0.579257040242056\\
-2.45718036768567	-0.539479019780047\\
-2.52164678111486	-0.503708930612821\\
-2.57193772497335	-0.471464138077679\\
-2.61169988858062	-0.442246757546901\\
-2.64351594898488	-0.415596919174394\\
-2.66923735914705	-0.391109861495715\\
-2.69020855922654	-0.368436997682488\\
-2.70741807953273	-0.347280568397312\\
-2.72160104643071	-0.327386299517498\\
-2.73330938581511	-0.308535984071788\\
-2.74296040698015	-0.290540731779852\\
-2.75087075690752	-0.273235092187402\\
-2.75728033616711	-0.256472024068654\\
-2.76236920991791	-0.240118592446481\\
-2.76626952842005	-0.224052247118856\\
-2.76907379540857	-0.208157535512617\\
-2.77084036426975	-0.192323110178293\\
-2.77159672029497	-0.1764388988405\\
-2.77134086887585	-0.160393308534797\\
-2.77004095713812	-0.144070332773498\\
-2.76763308176857	-0.127346420280503\\
-2.76401705350883	-0.110086943955474\\
-2.75904967228526	-0.092142077288357\\
-2.75253478283033	-0.0733418397797098\\
-2.74420898212786	-0.0534900098683465\\
-2.73372126704024	-0.032356520394106\\
-2.7206040346805	-0.00966784671947868\\
-2.70423150947638	0.014905222825905\\
-2.68375960308018	0.0417631560078521\\
-2.65803800694986	0.0713973388592425\\
-2.62548037540859	0.104414356051744\\
-2.58387101713977	0.141564491099818\\
-2.53007597315801	0.183771420326977\\
-2.45961364608305	0.232153476510912\\
-2.36603228770019	0.288011450466867\\
-2.24006518469893	0.352724342015756\\
-2.06866914692298	0.427428950372729\\
-1.83449282910597	0.512259192042037\\
-1.51738192768083	0.604872089801674\\
-1.10107594364602	0.698358248788268\\
-0.587556093100141	0.780086596003453\\
-0.0122557582588863	0.834865234224473\\
0.560167015110665	0.853244053900496\\
};
\addplot [color=mycolor1, forget plot]
  table[row sep=crcr]{%
2.4470975329268	2.85734050088096\\
2.59925956234538	2.85263748443431\\
2.72936239773121	2.84031623671661\\
2.84139874347278	2.82247155051008\\
2.93859222736187	2.80058642786175\\
3.0235313743472	2.77570091105713\\
3.09829197054803	2.74853551350831\\
3.16454044419819	2.71957967116439\\
3.22361771846211	2.68915438376271\\
3.27660568747186	2.6574562163269\\
3.32437921061262	2.62458795306691\\
3.36764642225098	2.59057968585606\\
3.40697974908862	2.55540298570755\\
3.44283955508847	2.51897998110674\\
3.47559189073129	2.48118857401167\\
3.50552143839477	2.44186459580276\\
3.53284041838165	2.40080138931511\\
3.5576939377613	2.35774706040887\\
3.58016200911911	2.31239944452605\\
3.60025821866181	2.2643986587764\\
3.61792476088008	2.21331694180605\\
3.63302325477194	2.15864530947419\\
3.64532038396865	2.09977636572004\\
3.65446692232179	2.03598240275462\\
3.65996807166613	1.96638771281183\\
3.66114219736428	1.88993384992122\\
3.6570639521016	1.80533650974888\\
3.64648641782961	1.7110329239271\\
3.62773537508888	1.60511957138689\\
3.59856753534355	1.48528233243545\\
3.55598465216144	1.34872632808139\\
3.49599945873118	1.19212300117946\\
3.41336285143547	1.01161127682071\\
3.3012949774795	0.802922570523745\\
3.15133200538093	0.561748137098855\\
2.95352110334314	0.284520029275175\\
2.6973552050403	-0.0302170952383606\\
2.37392335641798	-0.379827544265013\\
1.97946724983572	-0.755471413831684\\
1.51952588658696	-1.14109638579653\\
1.01136727122743	-1.51514517126524\\
0.482032224901594	-1.85544525334606\\
-0.0382838063201061	-2.14522947273086\\
-0.52410154924132	-2.37686303097922\\
-0.959488772986293	-2.55157186200705\\
-1.33839105970505	-2.67639513915615\\
-1.66205132699853	-2.76070307072847\\
-1.93580024564179	-2.81373776109643\\
-2.16654578772536	-2.84340740970685\\
-2.36125505045569	-2.85595956618366\\
-2.52621914730902	-2.8561017997265\\
-2.66679922833549	-2.84728154219588\\
-2.787420973919	-2.83197952277816\\
-2.89167569415667	-2.81196030141443\\
-2.98245314892876	-2.7884673805791\\
-3.06207131943148	-2.76236803217295\\
-3.13238964293222	-2.73425797963794\\
-3.19490246471729	-2.70453592010907\\
-3.25081385812493	-2.67345607439244\\
-3.30109649242935	-2.64116495686063\\
-3.34653745708976	-2.60772685290374\\
-3.38777365579463	-2.57314117569786\\
-3.42531892653787	-2.53735390476143\\
-3.45958457962228	-2.50026460922663\\
-3.49089463019454	-2.46173005540939\\
-3.51949664689044	-2.42156503154378\\
-3.54556883535683	-2.37954074729201\\
-3.56922370906653	-2.33538094838593\\
-3.59050845108607	-2.28875570263873\\
-3.60940181843321	-2.2392726437686\\
-3.6258071618823	-2.18646528955512\\
-3.63954080086355	-2.12977787016306\\
-3.65031457159793	-2.06854590523487\\
-3.65771081504082	-2.00197155739246\\
-3.66114733989249	-1.92909258526504\\
-3.65982893264023	-1.84874357670581\\
-3.65268075561092	-1.75950818944266\\
-3.63825750750274	-1.65966162877518\\
-3.61462074197262	-1.54710408232652\\
-3.57917593635714	-1.41928932570068\\
-3.52846249420908	-1.27316004705987\\
-3.45789767683273	-1.10511567818159\\
-3.36149714823051	-0.911064057090691\\
-3.23164370107576	-0.686649269870069\\
-3.0590694227229	-0.427801906604648\\
-2.83336215057245	-0.131797507300495\\
-2.54445295612327	0.201048934514697\\
-2.18548932838733	0.565232839046605\\
-1.756881214354	0.948298435946715\\
-1.26995300623213	1.33101269062124\\
-0.747435352721523	1.6907508679559\\
-0.218964618192301	2.00735905571867\\
0.286748605912422	2.26841802747823\\
0.748686704214283	2.47098556427373\\
1.15605475094327	2.61965277121247\\
1.50685469940157	2.72301328980851\\
1.80474754969108	2.7905970960028\\
2.05610271647609	2.83106393347599\\
2.26799126297232	2.85149584067307\\
2.4470975329268	2.85734050088096\\
};
\addplot [color=mycolor2, forget plot]
  table[row sep=crcr]{%
1.13400451818224	0.976808589286248\\
1.55591156704315	0.963967632646999\\
1.87844575220844	0.933648692297892\\
2.11936805604402	0.895471980855908\\
2.29861648452528	0.855264674972638\\
2.43291852387781	0.816033640430765\\
2.53480545668349	0.77909964354237\\
2.61323550840971	0.744886767314028\\
2.67451202995887	0.713381203705023\\
2.72306878728849	0.684375563701702\\
2.76204775962919	0.657592066074585\\
2.79370039348197	0.632741553186792\\
2.81966033819558	0.609549681145767\\
2.84112800312704	0.587766765388304\\
2.85899595219203	0.567169748121874\\
2.87393480527273	0.547560569004644\\
2.88645267285243	0.528763065252748\\
2.89693669235387	0.510619424756789\\
2.90568230974701	0.4929866544925\\
2.91291404181462	0.475733242668805\\
2.91880020475392	0.45873605083536\\
2.92346326744079	0.441877401398409\\
2.9269869300111	0.425042290128983\\
2.9294206417286	0.408115632972522\\
2.93078199156806	0.390979441224458\\
2.93105718626686	0.373509802692034\\
2.93019964118205	0.355573524530472\\
2.92812652187264	0.337024262310264\\
2.92471286181861	0.317697915544286\\
2.91978261199468	0.297407007473874\\
2.91309560767229	0.275933680108139\\
2.90432890298117	0.253020816484529\\
2.89305012649987	0.2283606419443\\
2.87867929825578	0.201579947253414\\
2.86043367930716	0.172220819462771\\
2.83724732590173	0.139715489795419\\
2.80765253239014	0.103353715016184\\
2.76960351029923	0.0622412937740589\\
2.72021268256465	0.0152496400797256\\
2.65535696720453	-0.0390393611120289\\
2.56909954942	-0.102377421932274\\
2.45287988771856	-0.176838945573836\\
2.2945089602085	-0.264649236867091\\
2.07732101696771	-0.367628681657688\\
1.78067588590787	-0.48587920660265\\
1.38454964462808	-0.615400748165417\\
0.881673507817823	-0.74534824221461\\
0.294918328976953	-0.858145393319198\\
-0.317126387879956	-0.936341921800144\\
-0.883847296346128	-0.972771788836004\\
-1.35810775058867	-0.973322344140018\\
-1.72865102211688	-0.950297867995204\\
-2.00781965414307	-0.915097530776598\\
-2.21555633660112	-0.875382765679109\\
-2.37050823287822	-0.835416192301556\\
-2.48728119065286	-0.797238697197828\\
-2.57650663409757	-0.761647304777937\\
-2.64570478272576	-0.728805858007367\\
-2.70015917313327	-0.698582512499448\\
-2.74359762199151	-0.6707242684433\\
-2.77867579716957	-0.64494284878883\\
-2.80730847419449	-0.620954518293437\\
-2.83089386719239	-0.598496660609891\\
-2.85046562760151	-0.57733295164336\\
-2.86679650790334	-0.557253161273629\\
-2.88046972309956	-0.538070608957517\\
-2.89192857699452	-0.519618756232101\\
-2.90151130391482	-0.501747631914064\\
-2.9094757140015	-0.484320386239284\\
-2.91601668742886	-0.467210069019795\\
-2.92127854785938	-0.450296626550061\\
-2.92536366807315	-0.433464061900805\\
-2.92833819837386	-0.416597676898128\\
-2.93023548133117	-0.399581297302019\\
-2.93105747144582	-0.382294367470811\\
-2.93077427789105	-0.364608782085508\\
-2.9293217633479	-0.346385296441374\\
-2.9265969355594	-0.3274693196282\\
-2.92245063167818	-0.307685842231661\\
-2.91667668191573	-0.286833176453542\\
-2.90899629612566	-0.264675084706324\\
-2.89903576598859	-0.240930734263808\\
-2.8862945941022	-0.215261731664662\\
-2.87009965741047	-0.187255256459951\\
-2.84953868532166	-0.156402040528551\\
-2.82336272243743	-0.122067683553979\\
-2.78984168857616	-0.0834557348325103\\
-2.74654882533352	-0.0395615824687034\\
-2.69003818636869	0.0108813368567756\\
-2.61536585856166	0.0694543294176981\\
-2.51539944447989	0.138078327170258\\
-2.37989458894041	0.218941624283059\\
-2.19448862572723	0.314173655445411\\
-1.94029548841821	0.424963981921096\\
-1.59600269760179	0.549723930217708\\
-1.14591807084824	0.681304947299176\\
-0.595828970644992	0.805112369640538\\
0.012534643673176	0.902349173444445\\
0.609958617098545	0.959665970740262\\
1.13400451818224	0.976808589286248\\
};
\addplot [color=mycolor1, forget plot]
  table[row sep=crcr]{%
1.90545647109768	2.84473020387397\\
2.06560614281006	2.83975180880271\\
2.20696629698615	2.82634185272677\\
2.3321915951789	2.80637841785645\\
2.44358977384658	2.7812802081649\\
2.54313495484306	2.75210344657922\\
2.63249879955004	2.71962180533369\\
2.71308776287006	2.68438966472592\\
2.78608015084079	2.64679082867193\\
2.85245993353386	2.60707528779914\\
2.91304609217289	2.56538646417426\\
2.96851724174839	2.52178098738162\\
3.01943172859427	2.47624262550686\\
3.06624357723105	2.42869160436426\\
3.10931468151508	2.37899021861514\\
3.14892357416526	2.32694537000167\\
3.18527100428818	2.27230845243878\\
3.21848242258461	2.21477283089051\\
3.24860732457302	2.15396902195747\\
3.27561523244261	2.08945757331174\\
3.29938790103738	2.02071955571186\\
3.31970710623613	1.94714453143744\\
3.33623710895942	1.86801586331425\\
3.34850058536363	1.78249331255182\\
3.35584648861182	1.68959310181112\\
3.35740800761322	1.58816609549383\\
3.3520486262168	1.47687564143641\\
3.33829449887801	1.35417819222828\\
3.3142524026042	1.2183124702892\\
3.27751522297935	1.06730717305853\\
3.22506265000043	0.899023567903594\\
3.15317550326068	0.71125798313182\\
3.05740019708648	0.501939051865713\\
2.9326263925163	0.269461325311117\\
2.77337167138756	0.0131902830470721\\
2.57438512912176	-0.265864250884355\\
2.33164968949332	-0.564299363001459\\
2.04372935410427	-0.875730608718146\\
1.71315341533049	-1.19073305175328\\
1.34725514404524	-1.49766199545704\\
0.957840581159799	-1.7843954061448\\
0.559476469129818	-2.04051966377913\\
0.166909292354912	-2.2591216772568\\
-0.20738281673389	-2.43751067242486\\
-0.554658638939878	-2.57677777831348\\
-0.870174254687224	-2.68063492454041\\
-1.15256142971836	-2.75411665693453\\
-1.40283481017664	-2.80254022426169\\
-1.62342322812576	-2.83085278857385\\
-1.81741597084171	-2.84331816002108\\
-1.98805916965749	-2.84343322803351\\
-2.13846090256022	-2.83397138670082\\
-2.27144346885171	-2.81708109928046\\
-2.38948818663578	-2.79439767208476\\
-2.49473284788786	-2.76714748807323\\
-2.58899587024454	-2.73623658309891\\
-2.67381156646907	-2.70232204470638\\
-2.75046784293595	-2.66586768325229\\
-2.82004187602089	-2.62718644856711\\
-2.88343176937095	-2.58647215117961\\
-2.94138353379713	-2.54382274343107\\
-2.99451340673076	-2.49925699574864\\
-3.04332582379704	-2.45272598968585\\
-3.0882274409002	-2.40412048827574\\
-3.12953757859625	-2.35327494589072\\
-3.16749537455486	-2.29996867912253\\
-3.20226381135857	-2.24392452729567\\
-3.23393064701117	-2.18480517625679\\
-3.26250611642261	-2.12220719462947\\
-3.28791709050908	-2.05565273435872\\
-3.30999716922801	-1.98457877940319\\
-3.32847193922572	-1.90832379869763\\
-3.34293834225436	-1.82611169612514\\
-3.35283678321713	-1.73703309658577\\
-3.35741428508422	-1.64002434273836\\
-3.35567674689123	-1.53384523447966\\
-3.34632834584585	-1.41705773592997\\
-3.32769667673192	-1.28800992744541\\
-3.29764396001492	-1.1448328491852\\
-3.25346864714091	-0.985463117106764\\
-3.19180969536969	-0.807711723896657\\
-3.10857992988814	-0.609408970823442\\
-2.99897728294295	-0.388664619236554\\
-2.85765247301142	-0.144284203070108\\
-2.6791393917441	0.123637165840428\\
-2.4586534840674	0.412984096921417\\
-2.19328667710712	0.718912984736189\\
-1.88343065048257	1.03347769843591\\
-1.53397275642332	1.34597848929993\\
-1.15461344762183	1.64426910082164\\
-0.758827839321173	1.91681242572131\\
-0.361611913312704	2.15477791315486\\
0.0231636257632512	2.35335165025311\\
0.384783706451366	2.51184807879158\\
0.716540098658869	2.63283491704342\\
1.01548918488673	2.7208374714998\\
1.28158359749607	2.78113776271335\\
1.51665497280379	2.81892605629222\\
1.72354052603863	2.83882994316678\\
1.90545647109768	2.84473020387397\\
};
\addplot [color=mycolor2, forget plot]
  table[row sep=crcr]{%
1.46855410137581	1.12707832297397\\
1.81746172861383	1.11648449734565\\
2.0832734524898	1.09148900817058\\
2.28420219676519	1.05963121359182\\
2.43667502329634	1.02541141897586\\
2.55355487106084	0.991253931079652\\
2.64431809237863	0.958339784161561\\
2.71577816800775	0.927157533362092\\
2.77280526976218	0.897829051554277\\
2.81889337956571	0.8702918262286\\
2.85657061328148	0.844397665828629\\
2.88768591502819	0.819964710924385\\
2.91360694503597	0.796803898388846\\
2.93535655415903	0.774731620325453\\
2.95370737147097	0.7535750041178\\
2.96924786678498	0.733173305273046\\
2.98242888329376	0.71337729747401\\
2.99359666860525	0.694047664479373\\
3.00301644870154	0.675052915197177\\
3.01088926630408	0.656267075791459\\
3.01736391707627	0.637567262531493\\
3.02254521232267	0.618831152111924\\
3.02649937460473	0.59993431215182\\
3.02925706552155	0.580747316169379\\
3.03081430616027	0.561132533969967\\
3.03113134592148	0.540940452919435\\
3.03012933592254	0.520005342248276\\
3.02768444097797	0.49814001572594\\
3.02361874677748	0.475129371377977\\
3.01768694324708	0.450722282586946\\
3.00955722904638	0.424621273347811\\
2.99878409207805	0.3964692207245\\
2.98476943299641	0.365832079889668\\
2.96670669226752	0.332176320531016\\
2.9434998840396	0.294839426073746\\
2.91364525751147	0.252991547337343\\
2.87505710623026	0.205586530339498\\
2.82481056570843	0.151301851642755\\
2.7587637319781	0.0884714203679338\\
2.6710140094552	0.015027152988835\\
2.55315732117441	-0.0715059378832148\\
2.39340687238096	-0.173853501639095\\
2.1759204274396	-0.294449287699812\\
1.8814106856495	-0.434129080814116\\
1.49134503928733	-0.589718938010167\\
0.998483596223561	-0.75106799166535\\
0.421858296720502	-0.900377107469046\\
-0.187265258270534	-1.01781985727513\\
-0.763463752921409	-1.09171592863421\\
-1.25816076574939	-1.12367349964627\\
-1.65447551481276	-1.12419096032058\\
-1.9596185164593	-1.10523445409195\\
-2.19074418828945	-1.07607638440074\\
-2.36560531342012	-1.04262754170864\\
-2.49890021152678	-1.00822970036258\\
-2.60172186414097	-0.974599376616324\\
-2.68211856458525	-0.942518623253538\\
-2.74585071655244	-0.91226261074259\\
-2.79703992585854	-0.883844558714942\\
-2.83865455548789	-0.857150338876948\\
-2.87285359067348	-0.832010381280273\\
-2.90122497915828	-0.808236977352645\\
-2.92494991100248	-0.785642778172455\\
-2.94491633958941	-0.764049184780701\\
-2.96179794748695	-0.743289371013107\\
-2.97610953260883	-0.723208506594891\\
-2.98824617924996	-0.703662559393909\\
-2.99851115012501	-0.684516404077932\\
-3.00713581596548	-0.66564160531801\\
-3.01429385668226	-0.646914043782802\\
-3.02011123698435	-0.62821143953346\\
-3.02467295600509	-0.60941075971465\\
-3.02802721267342	-0.59038545286311\\
-3.03018736051096	-0.571002417318382\\
-3.03113180758053	-0.551118577579125\\
-3.03080181858543	-0.530576903738905\\
-3.02909696891869	-0.509201659800999\\
-3.02586775522894	-0.486792600809011\\
-3.02090454690351	-0.463117749314465\\
-3.01392161641998	-0.43790426005694\\
-3.00453433777658	-0.41082671747549\\
-2.99222667527461	-0.381491992925512\\
-2.97630462145942	-0.349419510354504\\
-2.95582901110058	-0.314015440440706\\
-2.92951773672268	-0.274539022594774\\
-2.89560227628824	-0.23005908867203\\
-2.85161602955057	-0.179399434070764\\
-2.79408213382729	-0.121074198907511\\
-2.71805834339307	-0.0532218198776744\\
-2.61649598903557	0.026435103293064\\
-2.47941066892704	0.120533485531534\\
-2.29302953539182	0.231758171538483\\
-2.0395571501463	0.361972183384229\\
-1.69921188392751	0.510376156685171\\
-1.25735371882417	0.670635146156023\\
-0.718111284055277	0.828535287891957\\
-0.117225357787478	0.964113108564206\\
0.483218576030838	1.06040872070266\\
1.02278616318477	1.11239456361227\\
1.4685541013758	1.12707832297397\\
};
\addplot [color=mycolor1, forget plot]
  table[row sep=crcr]{%
1.5511141993765	2.96816970333686\\
1.71675358438934	2.96299691921139\\
1.86682539418795	2.94874046419969\\
2.00300869880171	2.92701307354291\\
2.1268612119055	2.89909435579508\\
2.23979214975311	2.86598173523314\\
2.34305412892067	2.82843762111414\\
2.43774626769991	2.78703031878428\\
2.52482323250256	2.74216802431532\\
2.6051068522611	2.69412618643373\\
2.67929821669458	2.64306893140237\\
2.74798902104091	2.58906536319621\\
2.81167144936764	2.53210151744128\\
2.87074620122304	2.47208864713597\\
2.92552843417927	2.40886839739986\\
2.97625146671294	2.34221530971119\\
3.02306809386465	2.27183699588217\\
3.06604933256557	2.19737224570959\\
3.10518034740554	2.11838728616227\\
3.14035322017697	2.03437040262416\\
3.17135612668665	1.94472517759859\\
3.19785838475057	1.84876272096299\\
3.21939075997375	1.74569349161575\\
3.23532040059592	1.63461969228791\\
3.24481988950191	1.51452982616827\\
3.24683026945662	1.38429792520045\\
3.24001870889995	1.24269129770411\\
3.22273302150037	1.08839248387585\\
3.19295793562874	0.920043451259204\\
3.14828230299302	0.736322673246078\\
3.08589270849307	0.536067880169691\\
3.00261702269694	0.318457344540068\\
2.8950497572107	0.0832576972436309\\
2.75979544183791	-0.168867570731735\\
2.59385881672152	-0.436021360781201\\
2.39518107659023	-0.714779306849007\\
2.16326296217152	-1.0000507108857\\
1.89973724771263	-1.2852211083055\\
1.60869142726366	-1.56266105395898\\
1.29655212742886	-1.82457122668823\\
0.971461435453503	-2.06398451866564\\
0.64226687823649	-2.27564747825153\\
0.317405648969006	-2.45653164756393\\
0.00399029116635881	-2.60587315260145\\
-0.292710800285464	-2.72481554129322\\
-0.569374827774472	-2.81583834360932\\
-0.824413429264136	-2.88215937495653\\
-1.05759927921737	-2.92723658466493\\
-1.2696590252546	-2.954419170429\\
-1.46191201754143	-2.96674238900522\\
-1.63598822105908	-2.96683397124905\\
-1.79362814199161	-2.95689494329898\\
-1.93655247395617	-2.93872345847997\\
-2.06638458222437	-2.91375925764997\\
-2.18460991573799	-2.88313463759206\\
-2.29255955355281	-2.84772397858109\\
-2.39140850588118	-2.80818795523389\\
-2.48218231230374	-2.76501099685318\\
-2.56576770136256	-2.7185318953432\\
-2.64292464567118	-2.66896809950429\\
-2.71429819888329	-2.61643447582328\\
-2.78042917454843	-2.56095734362065\\
-2.84176313728666	-2.50248451756251\\
-2.89865740951947	-2.44089197581842\\
-2.95138591191257	-2.37598765124789\\
-3.00014169247095	-2.30751273337605\\
-3.04503698345508	-2.2351407798849\\
-3.08610057314396	-2.15847487429311\\
-3.12327220164251	-2.0770430384803\\
-3.1563935950482	-1.99029212536997\\
-3.18519565057231	-1.89758049553535\\
-3.20928119314775	-1.79816994866221\\
-3.2281026712299	-1.69121767735085\\
-3.2409341989296	-1.5757694952191\\
-3.24683757518568	-1.45075634298661\\
-3.24462247064948	-1.31499719288961\\
-3.23280210809785	-1.16721305337147\\
-3.20954781988667	-1.00605887830981\\
-3.17264929489326	-0.83018271505137\\
-3.11949258355707	-0.638323940630935\\
-3.04707518900768	-0.429463817566223\\
-2.95208610020282	-0.203039647718152\\
-2.83108570188368	0.0407750039789764\\
-2.68082016772555	0.300741426695085\\
-2.49868774249454	0.574228932152976\\
-2.28333051547243	0.856982607028298\\
-2.03525425987047	1.14311012993192\\
-1.75730260272994	1.42540246962225\\
-1.45477926837949	1.69602394718267\\
-1.13507590084788	1.94746673630532\\
-0.806827848963815	2.17352897098816\\
-0.478810766230246	2.37003509442909\\
-0.158891672379683	2.53511231340366\\
0.146701500614249	2.66901330900869\\
0.433683425640174	2.7736265607983\\
0.699638558134577	2.85187124026548\\
0.943710310761243	2.90713854170722\\
1.166197624607	2.94286609588378\\
1.36816404860046	2.96226391390951\\
1.5511141993765	2.96816970333686\\
};
\addplot [color=mycolor2, forget plot]
  table[row sep=crcr]{%
1.69631441633007	1.28768207007386\\
1.99032662854603	1.27874738696554\\
2.2166701408415	1.2574447719624\\
2.39088211468516	1.22980388087852\\
2.52592030888927	1.19948059308081\\
2.6317285425257	1.16854555802824\\
2.71566106839321	1.13809809930942\\
2.78308250493548	1.10866994684214\\
2.83790007966455	1.08047124194359\\
2.88297368997499	1.0535348197561\\
2.92041396595002	1.02779904998385\\
2.95179366805963	1.00315466733232\\
2.97829637996918	0.979470777011036\\
3.00082110987031	0.956608880037873\\
3.02005620276326	0.934430016464237\\
3.03653189746237	0.912797952928076\\
3.0506579371935	0.891580089193716\\
3.06275061316424	0.870647035321154\\
3.07305223115758	0.849871390220485\\
3.0817450407718	0.829126003760703\\
3.0889610107669	0.808281853879328\\
3.09478837182346	0.787205574173226\\
3.09927551078203	0.765756600727119\\
3.10243253949991	0.743783853697696\\
3.10423064069486	0.721121818891428\\
3.10459908279546	0.697585839253715\\
3.10341956749125	0.672966358648663\\
3.10051729544469	0.647021773042724\\
3.09564776637187	0.619469428583003\\
3.08847781162818	0.589974151989877\\
3.07855860551277	0.558133495200779\\
3.06528728841541	0.523458614364247\\
3.04785217094134	0.485349384665519\\
3.02515399881607	0.44306201098606\\
2.99569207555459	0.395667149125735\\
2.95739875970142	0.341996730740528\\
2.90739882293268	0.280579102362855\\
2.84166243842304	0.209566654956785\\
2.75451725805986	0.126672070571172\\
2.63800260305594	0.0291573292462813\\
2.48113177255788	-0.0860210416188368\\
2.26937740142539	-0.22169921476775\\
1.98527052826225	-0.379277231781463\\
1.61192350068489	-0.55644225831266\\
1.14158841186434	-0.744225259582789\\
0.588057722722855	-0.925696180260258\\
-0.00675744185391754	-1.0800113283142\\
-0.584623958529031	-1.19167406956994\\
-1.09642539040285	-1.25745740352105\\
-1.51896416693009	-1.2848102027044\\
-1.8528008268099	-1.28525153808844\\
-2.11094151451397	-1.26919996062677\\
-2.30941040545633	-1.24414223451508\\
-2.4626089024133	-1.21481927141176\\
-2.58196490385806	-1.18400372538223\\
-2.67605415977655	-1.15321784723463\\
-2.75116272174692	-1.12323800969317\\
-2.81186746537437	-1.09441190091988\\
-2.86150824186208	-1.06684756239571\\
-2.90253908805109	-1.04052294570266\\
-2.93677969678039	-1.01534831669648\\
-2.96559277957666	-0.991201247053671\\
-2.9900087356399	-0.967945805144239\\
-3.01081351667094	-0.945442666432577\\
-3.02861090182525	-0.923554007263144\\
-3.04386692502897	-0.902145396894908\\
-3.05694175382007	-0.881085951917333\\
-3.06811263841632	-0.860247466272762\\
-3.07759040122891	-0.839502907414268\\
-3.08553114872227	-0.818724476404097\\
-3.09204433866097	-0.797781310465758\\
-3.09719794322897	-0.776536827601692\\
-3.10102115455803	-0.754845654470515\\
-3.10350484234504	-0.732550028067738\\
-3.10459976120126	-0.709475509870231\\
-3.10421228968941	-0.685425790570829\\
-3.10219723414302	-0.660176287121552\\
-3.09834691259617	-0.633466133528818\\
-3.09237529912219	-0.604988033344248\\
-3.08389538674237	-0.574375264372061\\
-3.07238701384697	-0.541184894215979\\
-3.05715103904924	-0.504875973594996\\
-3.03724371305329	-0.464781136540198\\
-3.01138206111722	-0.420069719438527\\
-2.97780665437066	-0.369700415319409\\
-2.93408197189791	-0.312362127070219\\
-2.87680689778822	-0.246404329456668\\
-2.80120128949956	-0.16976576083321\\
-2.7005384078293	-0.079928782593947\\
-2.56543493883457	0.0260318956802744\\
-2.3831591312352	0.15115536363008\\
-2.13750906636878	0.297774100114308\\
-1.8105897569183	0.465783528152239\\
-1.38866466012649	0.649891413810413\\
-0.87323150796197	0.837055758569039\\
-0.292272852787305	1.00745919807065\\
0.30138542793791	1.14168765120112\\
0.850896019481423	1.2300133132159\\
1.31925086546401	1.27523515772891\\
1.69631441633007	1.28768207007386\\
};
\addplot [color=mycolor1, forget plot]
  table[row sep=crcr]{%
1.30964269727806	3.1637962565165\\
1.4796617479317	3.15846843373279\\
1.63679061427442	3.14352532559141\\
1.78209449164426	3.12032833944566\\
1.91661469139635	3.08999203623042\\
2.04133587734847	3.05341084022803\\
2.15716704436191	3.01128640334179\\
2.26493187823667	2.96415318583326\\
2.36536518025283	2.91240103174314\\
2.45911293062631	2.85629427339921\\
2.54673426377869	2.79598733673527\\
2.62870414512695	2.73153704934776\\
2.70541590354361	2.66291195707374\\
2.77718301859967	2.58999898766238\\
2.84423971673922	2.51260779861417\\
2.90674001989998	2.4304731350626\\
2.96475493309473	2.34325552009698\\
3.01826746994878	2.25054061827051\\
3.06716521160309	2.1518376669181\\
3.11123009040846	2.04657747494767\\
3.15012510558999	1.93411066423918\\
3.18337774235946	1.81370709861271\\
3.21036002162971	1.68455783764647\\
3.23026541778701	1.54578149644089\\
3.24208343692729	1.39643760860752\\
3.24457356752618	1.23555047198094\\
3.23624174246731	1.06214793587486\\
3.21532452017037	0.875320475251067\\
3.17978895210455	0.674306296938357\\
3.12735936579723	0.458607445326076\\
3.05558538675182	0.228138886152453\\
2.96196699821337	-0.0165939040116348\\
2.84414987595434	-0.274304198788235\\
2.70019456494705	-0.542750437351112\\
2.52890392387698	-0.818625887665045\\
2.33016567717077	-1.09756710070662\\
2.10523835090943	-1.37432977635652\\
1.85689379064363	-1.64314957924953\\
1.58934378464163	-1.89825275383867\\
1.3079278520424	-2.13442513615714\\
1.01860977322162	-2.34751617881986\\
0.727390781441582	-2.53476738101936\\
0.439767930045209	-2.69490895567937\\
0.160339199500988	-2.82803826492504\\
-0.107399608192614	-2.93534672986096\\
-0.361078183550312	-3.0187811774478\\
-0.599343551116539	-3.08071390197956\\
-0.82167924151625	-3.12366813311058\\
-1.02820229942608	-3.15011728731852\\
-1.21947223478175	-3.16235593533549\\
-1.39632974112117	-3.16242964145298\\
-1.55977082001897	-3.15210762402562\\
-1.71085457163499	-3.1328834234634\\
-1.85063940386854	-3.10599186652311\\
-1.98014142579909	-3.07243399096331\\
-2.10030916868881	-3.03300448554837\\
-2.21200973246586	-2.98831837870147\\
-2.31602253244997	-2.93883521303982\\
-2.41303779837131	-2.88487991020164\\
-2.50365777336167	-2.82666011113997\\
-2.58839916531416	-2.76428009887243\\
-2.66769583949962	-2.69775156936529\\
-2.7419010420427	-2.62700157896064\\
-2.81128864048502	-2.5518780086988\\
-2.87605298716733	-2.47215287701608\\
-2.93630707519116	-2.38752382306851\\
-2.99207868262204	-2.29761408868848\\
-3.04330420322828	-2.20197136130826\\
-3.08981985611297	-2.10006591749712\\
-3.13134996953366	-1.99128864432507\\
-3.16749206989299	-1.87494973512688\\
-3.19769861039845	-1.7502791836144\\
-3.22125539743398	-1.61643066493288\\
-3.23725719218883	-1.47249102074309\\
-3.24458168556316	-1.31749836824648\\
-3.24186419916032	-1.15047279834493\\
-3.22747719906101	-0.970464595351572\\
-3.19952112736246	-0.77662561958305\\
-3.15583611401251	-0.568309413072392\\
-3.09404743511042	-0.345203854359003\\
-3.01166010965617	-0.107495624874144\\
-2.90621784361654	0.143942848398775\\
-2.77553580150244	0.407366412040965\\
-2.61800243985426	0.680008613328076\\
-2.432921803668	0.958023744307871\\
-2.22083819760022	1.23657425973828\\
-1.98376120673494	1.51009984708879\\
-1.72520699853086	1.77276127972635\\
-1.45000416550279	2.01899454967052\\
-1.16387551442911	2.24406316113598\\
-0.872876987209391	2.44448553178769\\
-0.582818051995086	2.61825016010458\\
-0.298783922300724	2.76479746668511\\
-0.0248348027380357	2.88481207518776\\
0.236102926428582	2.97990634957964\\
0.482187030483545	3.05227836659459\\
0.712507782295484	3.10440573556148\\
0.926890690971778	3.13880717167437\\
1.12569630509008	3.15787875025747\\
1.30964269727806	3.1637962565165\\
};
\addplot [color=mycolor2, forget plot]
  table[row sep=crcr]{%
1.87066321805417	1.45402756039734\\
2.12177093449327	1.44638275984957\\
2.317761693533	1.42791955610156\\
2.47135347756998	1.40353440201243\\
2.59275799155613	1.37625932425357\\
2.6897624708477	1.34788753229756\\
2.76817229032354	1.319434934294\\
2.83228382782909	1.29144458778211\\
2.88528007826317	1.26417714528765\\
2.92953233347676	1.23772686605507\\
2.96682099594705	1.21209130233767\\
2.99849438692525	1.1872127710152\\
3.025582317517	1.16300278176544\\
3.04887736169328	1.13935617858333\\
3.06899323549112	1.11615904872294\\
3.08640693070161	1.09329282421922\\
3.10148924371942	1.07063602288273\\
3.11452692052714	1.04806448560686\\
3.12573864511145	1.02545060684543\\
3.13528640133144	1.00266183085243\\
3.14328324184147	0.97955854127504\\
3.14979813366696	0.955991371688167\\
3.15485826910997	0.931797888189099\\
3.15844899574587	0.90679852796303\\
3.16051129998121	0.88079160930123\\
3.16093654653781	0.853547150047979\\
3.15955790023812	0.8247991342263\\
3.15613749773549	0.794235741298259\\
3.15034794215717	0.761486888763146\\
3.14174598657415	0.726108225906463\\
3.12973524113668	0.687560446781515\\
3.1135132220467	0.64518246836988\\
3.09199583371357	0.598156684667283\\
3.06370915644131	0.545464286952159\\
3.0266339377191	0.485828872524742\\
2.97798250492584	0.417648064197987\\
2.91388218623718	0.338917487750234\\
2.82893862932173	0.247163201907554\\
2.71567109994163	0.139425480376497\\
2.56388651868539	0.0123921444109677\\
2.36026722654807	-0.137122249369144\\
2.08890423110962	-0.311030939692938\\
1.73420883848114	-0.507843467702896\\
1.28787824980463	-0.719793581163217\\
0.759259933385833	-0.931073702832379\\
0.182177988575539	-1.12053266150978\\
-0.392171517511089	-1.26977533751479\\
-0.915575840684575	-1.37106932892531\\
-1.36004790937376	-1.42827021685356\\
-1.7199956170801	-1.45158832747825\\
-2.00398372858132	-1.45195535619048\\
-2.22580640137757	-1.43814554765677\\
-2.39915840499094	-1.41624205918801\\
-2.53554018619301	-1.39012326080988\\
-2.64390675165217	-1.36213304062674\\
-2.73099292835528	-1.33362899055591\\
-2.80179428767293	-1.30536071197632\\
-2.86000707011612	-1.27771170209651\\
-2.90837620006094	-1.2508483373094\\
-2.94895423287702	-1.22480990474502\\
-2.9832888558787	-1.19956242009131\\
-3.01255721582719	-1.17503050616074\\
-3.03766196113174	-1.15111603355987\\
-3.05930008373523	-1.12770876023663\\
-3.07801248582485	-1.10469210625135\\
-3.09421983150112	-1.08194593790484\\
-3.10824855146559	-1.05934747633484\\
-3.12034968056339	-1.03677098569893\\
-3.13071237649265	-1.01408661289765\\
-3.13947338120028	-0.991158571549518\\
-3.14672326345681	-0.967842743744772\\
-3.15250996389637	-0.9439836870027\\
-3.15683990998776	-0.919410963503353\\
-3.15967674486952	-0.893934642015034\\
-3.16093749189277	-0.867339750538167\\
-3.16048572690583	-0.839379370929042\\
-3.15812101853101	-0.809765956741145\\
-3.1535634778409	-0.77816031235226\\
-3.14643166941203	-0.744157484395111\\
-3.1362112833808	-0.707268575723716\\
-3.12221071816805	-0.666897194254896\\
-3.10349788402036	-0.622308911711488\\
-3.07880985193452	-0.572591804392801\\
-3.04642315041378	-0.516606090574266\\
-3.00396738607712	-0.452921596262271\\
-2.94815890970766	-0.37974451149767\\
-2.87442707678991	-0.294842389227502\\
-2.77641185225325	-0.195494294437964\\
-2.64535151344877	-0.0785320608051692\\
-2.46950778725791	0.0593858575997357\\
-2.23409419163594	0.221008435394047\\
-1.92277618699022	0.406878897255613\\
-1.52245210617972	0.612727365398179\\
-1.03231897555632	0.826789077549283\\
-0.473780599181381	1.02986037347607\\
0.108686675359043	1.20096726029463\\
0.662567557702798	1.32640137592083\\
1.14847974851045	1.40461533226047\\
1.55026489179016	1.44344976208378\\
1.87066321805417	1.45402756039734\\
};
\addplot [color=mycolor1, forget plot]
  table[row sep=crcr]{%
1.14198747225695	3.38832402736194\\
1.31605992222682	3.38285578958292\\
1.47926499672352	3.3673224594284\\
1.63231216590224	3.34287789641818\\
1.77592179996291	3.31048127019128\\
1.91079626334644	3.27091234371535\\
2.03760030921143	3.22478834037688\\
2.15694827157711	3.17258054332779\\
2.26939599959159	3.11462954399408\\
2.37543590802034	3.05115858110924\\
2.47549388692006	2.98228475333172\\
2.56992710848601	2.9080281024882\\
2.65902199295158	2.82831869522321\\
2.74299175804852	2.74300191243686\\
2.82197309033541	2.65184221413475\\
2.89602155468672	2.55452570213947\\
2.96510541366527	2.45066187094384\\
3.02909757560089	2.33978503259681\\
3.08776544561469	2.22135603930962\\
3.14075853869006	2.09476512195938\\
3.18759385630825	1.95933692768926\\
3.22763926585268	1.81433918576067\\
3.26009550419467	1.65899685866019\\
3.28397801391087	1.4925141257177\\
3.29810067759518	1.31410704126207\\
3.30106469525668	1.12305008611112\\
3.29125735525567	0.918739870152637\\
3.26686717093224	0.7007785980871\\
3.22592347601145	0.469078103296655\\
3.16636945965635	0.223981748356033\\
3.08617676437358	-0.0336041103607149\\
2.98350587120135	-0.302084954015504\\
2.85690839958484	-0.579083389716836\\
2.70555499087658	-0.861405365663716\\
2.52945757552058	-1.14509799818222\\
2.32964231475143	-1.42561972089645\\
2.10822613845917	-1.69812186340872\\
1.86836133044444	-1.957811514879\\
1.61403947829286	-2.20033847192607\\
1.34978085508433	-2.42213618717118\\
1.08026481735875	-2.62065529577265\\
0.809969103738292	-2.79445610767395\\
0.542877020464996	-2.94316206126457\\
0.282287201830255	-3.06730577664165\\
0.0307321921675561	-3.16811417651256\\
-0.210010359277831	-3.24727833304284\\
-0.438841755820347	-3.30674243380652\\
-0.655230912051859	-3.34853143793361\\
-0.85909190141726	-3.37462385965159\\
-1.05066706316558	-3.38686711003996\\
-1.23042500857762	-3.38692810863447\\
-1.39897720956761	-3.37627042460475\\
-1.55701316096008	-3.35614973643604\\
-1.70525208644774	-3.32762084659481\\
-1.84440832162054	-3.29155115765214\\
-1.97516739762051	-3.24863703808408\\
-2.09817011972688	-3.1994207324695\\
-2.21400235954319	-3.14430638303466\\
-2.32318872587667	-3.08357436830736\\
-2.42618868176004	-3.01739359026803\\
-2.52339400739865	-2.94583161297398\\
-2.61512676734915	-2.86886272311823\\
-2.70163713203072	-2.78637408546898\\
-2.78310054027091	-2.69817023325833\\
-2.85961378384395	-2.60397618789948\\
-2.93118965997596	-2.50343956182795\\
-2.99774988716902	-2.39613207818106\\
-3.05911602864777	-2.28155105592556\\
-3.11499823485627	-2.15912157337876\\
-3.16498172601717	-2.02820025121288\\
-3.20851112040985	-1.88808190062256\\
-3.24487301706631	-1.73801067006497\\
-3.27317771812207	-1.57719778735318\\
-3.29234169013923	-1.4048484969593\\
-3.30107337735163	-1.22020124874507\\
-3.29786632705124	-1.02258243707351\\
-3.28100522506141	-0.811479737863473\\
-3.2485921674592	-0.586635931031317\\
-3.19860185149968	-0.348162504129153\\
-3.12897453324035	-0.0966678067712452\\
-3.03775338926328	0.166612124818355\\
-2.92326702955321	0.43970012338347\\
-2.78434754524817	0.7198096240344\\
-2.62056039573655	1.00335319673104\\
-2.43240798435225	1.28605503841691\\
-2.22145995280244	1.56317868851008\\
-1.99036681856816	1.82985490562083\\
-1.74273320138674	2.08146498692662\\
-1.48285909721788	2.31401346585324\\
-1.2153915502866	2.52442163704885\\
-0.944951100570337	2.7106925546306\\
-0.675798932141263	2.8719316217413\\
-0.411592829482475	3.00824106676635\\
-0.155252198347039	3.12052944746041\\
0.0910740234581681	3.21028396949869\\
0.325960018222231	3.27934651147373\\
0.548605990898327	3.32972039217949\\
0.758718693172089	3.36342045582833\\
0.956390154658264	3.38236784241906\\
1.14198747225695	3.38832402736193\\
};
\addplot [color=mycolor2, forget plot]
  table[row sep=crcr]{%
2.01273186082157	1.62264188057517\\
2.22882063757858	1.61604978981954\\
2.39982292437263	1.5999264589979\\
2.53605024869709	1.57828560186922\\
2.64560455641858	1.55366230710075\\
2.73465289221244	1.52760891831352\\
2.80782868963228	1.50104856263404\\
2.86860364562702	1.47450913150948\\
2.91958676862407	1.44827262140604\\
2.96275059777069	1.42246876899394\\
2.99959758866234	1.39713323996872\\
3.0312810461154	1.37224358546903\\
3.05869273021849	1.34774130683866\\
3.08252639381533	1.32354521601587\\
3.10332400638802	1.2995592980853\\
3.12150948490404	1.27567705258614\\
3.13741333335669	1.25178352695749\\
3.15129057492953	1.22775577834639\\
3.16333363422859	1.20346219676488\\
3.17368130395596	1.17876092409924\\
3.18242454469853	1.15349746678444\\
3.18960956917868	1.12750149774532\\
3.19523841599993	1.10058275728146\\
3.19926699199864	1.0725258797971\\
3.20160032852801	1.04308388296492\\
3.20208452520021	1.01196994813214\\
3.20049450803311	0.978846985268764\\
3.19651625838312	0.943314301531696\\
3.18972150520314	0.9048904686354\\
3.17953191683756	0.862991203193945\\
3.16516843942157	0.816900741376229\\
3.14557941789191	0.765734845680667\\
3.11933827732454	0.708393357162769\\
3.08449765619282	0.64350043767493\\
3.03838211135335	0.569332132271484\\
2.97729712019097	0.483735409182865\\
2.89613254086358	0.384054181754214\\
2.78785645755072	0.267103344500391\\
2.64296078709071	0.129283819862817\\
2.44909508235137	-0.0329769279253369\\
2.19149803292758	-0.222156640087914\\
1.85540284097846	-0.437620556075269\\
1.43183220207253	-0.672781570954555\\
0.926529254475946	-0.91293714136645\\
0.366905965032531	-1.13685603092568\\
-0.201788153687179	-1.3237955378641\\
-0.732802230074641	-1.46194690177102\\
-1.19478592849004	-1.55144546914079\\
-1.57703266179045	-1.60067155688414\\
-1.88399309881306	-1.62055881991123\\
-2.12714244300369	-1.62086201768036\\
-2.3192602556512	-1.60888726103068\\
-2.47172727407397	-1.5896091371629\\
-2.59373537863074	-1.56623151723462\\
-2.69237111564201	-1.54074516975253\\
-2.77298453280887	-1.51435193789272\\
-2.83958597030085	-1.48775415463792\\
-2.89518319062102	-1.46134215062032\\
-2.94204278026816	-1.43531268450704\\
-2.98188459701558	-1.40974281616709\\
-3.01602373734729	-1.38463567926112\\
-3.04547347217946	-1.35994867978886\\
-3.07101983077187	-1.33561070585615\\
-3.09327577122147	-1.31153242886764\\
-3.11272065473482	-1.28761221449429\\
-3.12972907723809	-1.26373919332839\\
-3.14459190832569	-1.23979443853937\\
-3.15753152786689	-1.21565081849316\\
-3.16871263534396	-1.19117184813658\\
-3.1782495596713	-1.16620969956304\\
-3.18621066100641	-1.14060241545111\\
-3.19262014838458	-1.11417027688287\\
-3.19745740447247	-1.0867111941397\\
-3.20065368243093	-1.05799490378562\\
-3.20208579114637	-1.02775565754519\\
-3.20156608111895	-0.995682968167048\\
-3.19882764192543	-0.961409824006352\\
-3.19350306524589	-0.924497586282473\\
-3.1850943320687	-0.884416530974658\\
-3.17293023047192	-0.840520688268477\\
-3.15610603691663	-0.792015285376631\\
-3.13339778795653	-0.737914789304898\\
-3.10314010978443	-0.676989486867996\\
-3.0630521846915	-0.607699252841271\\
-3.00999156462662	-0.528115846321772\\
-2.93961266425308	-0.435842359447583\\
-2.8459134889737	-0.327955664745927\\
-2.72069030724748	-0.201034638583436\\
-2.55303012238377	-0.0514075447247277\\
-2.32923428621965	0.124137739729845\\
-2.03405456577611	0.326840235647845\\
-1.65465028027992	0.55345949443634\\
-1.18821921518954	0.793467136619876\\
-0.650950165933849	1.02834457263954\\
-0.0805694209150354	1.23597001051807\\
0.474456620282551	1.399221940702\\
0.973544504074251	1.51237661528221\\
1.39585274626457	1.58041146158093\\
1.73928748626154	1.61362058532941\\
2.01273186082157	1.62264188057517\\
};
\addplot [color=mycolor1, forget plot]
  table[row sep=crcr]{%
1.02611651957107	3.60772225884434\\
1.20403531250379	3.60212375025056\\
1.37252272406886	3.58607871275405\\
1.53209147433507	3.5605840611805\\
1.68327442490991	3.52647097361666\\
1.82660014241576	3.48441505698273\\
1.96257504267002	3.43494800021525\\
2.09167048293983	3.37846934578765\\
2.21431340202062	3.31525753296105\\
2.3308793479758	3.24547974201835\\
2.44168695081149	3.16920033169254\\
2.54699307979453	3.08638784461037\\
2.64698807085481	2.99692068599959\\
2.74179052285167	2.9005916824586\\
2.83144125016096	2.79711181942329\\
2.91589605313904	2.68611355417884\\
2.99501703994363	2.56715421982348\\
3.06856231834773	2.4397201868288\\
3.13617399414738	2.30323264312533\\
3.19736458859026	2.15705609778928\\
3.25150225242221	2.00051100716612\\
3.29779554621413	1.83289225205801\\
3.33527911587741	1.65349552358913\\
3.36280235276438	1.4616539294472\\
3.37902409905939	1.2567871829688\\
3.38241759486048	1.03846538671013\\
3.3712910142648	0.806488398293082\\
3.34382979549314	0.560979753679769\\
3.29816702173673	0.302490850827981\\
3.23248664958135	0.0321065160951329\\
3.14516065270606	-0.248462376543206\\
3.03491467713805	-0.536818690225567\\
2.90100793301208	-0.82987341863068\\
2.74340347468625	-1.12391471873249\\
2.56289793654726	-1.41476251714717\\
2.36117911643895	-1.6980036617741\\
2.14078838582012	-1.96928306110288\\
1.90498241059215	-2.22460940016504\\
1.65751055078008	-2.46062665077033\\
1.4023432160411	-2.67480853992983\\
1.14339564859588	-2.86555083124844\\
0.884288382218877	-3.03215927806954\\
0.628172427659431	-3.1747514530428\\
0.377629863914206	-3.29410266310233\\
0.134644947708159	-3.39146838897052\\
-0.0993692244464525	-3.46841037969752\\
-0.323507312636681	-3.52664453446868\\
-0.537289374400536	-3.56791950464855\\
-0.740573585424378	-3.59392766774999\\
-0.933473511872671	-3.60624546586133\\
-1.11628526502292	-3.60629781860137\\
-1.28942670714867	-3.59534075499257\\
-1.45338877731054	-3.5744568633044\\
-1.60869780134301	-3.54455908355186\\
-1.75588709590617	-3.50639940520567\\
-1.89547603925685	-3.46057998697967\\
-2.02795487972215	-3.40756500385319\\
-2.15377376160685	-3.34769213374202\\
-2.2733346897906	-3.28118304140237\\
-2.38698538549539	-3.20815253229395\\
-2.49501418637923	-3.12861626808435\\
-2.5976453079511	-3.04249708878699\\
-2.69503391204524	-2.94963009988326\\
-2.78726052797591	-2.84976677729476\\
-2.87432445198552	-2.74257843623574\\
-2.95613582189578	-2.62765951672197\\
-3.0325061402555	-2.50453127197162\\
-3.10313711807964	-2.37264661756504\\
-3.16760785464923	-2.23139711792507\\
-3.22536058499219	-2.08012335631199\\
-3.27568555028962	-1.91813024917339\\
-3.31770601716222	-1.74470920163683\\
-3.35036512836541	-1.55916930470951\\
-3.37241713470611	-1.36087994730133\\
-3.38242662245726	-1.14932709569401\\
-3.37878052110083	-0.924184843670482\\
-3.35971873119469	-0.685402357949884\\
-3.3233897443763	-0.43330372642119\\
-3.26793702021244	-0.168694272966524\\
-3.19161937754449	0.107038210307673\\
-3.09296357987329	0.391844357044387\\
-2.97093950589242	0.682977531095793\\
-2.82513876334528	0.977023428823052\\
-2.65592880016096	1.27001138997337\\
-2.46455021158412	1.55761110463077\\
-2.25312869633061	1.83540002987176\\
-2.02458646705947	2.09916774465816\\
-1.78245842128309	2.34521061155878\\
-1.53063974954406	2.57056923249898\\
-1.2731063612472	2.77317355736937\\
-1.01365249973888	2.95188184958125\\
-0.755681123047297	3.10642231749431\\
-0.502066530535087	3.2372628547284\\
-0.255091604879366	3.34544136045897\\
-0.0164489460183681	3.43238713735545\\
0.212711718933483	3.49975621592722\\
0.431708626818227	3.54929399415121\\
0.640241689089839	3.58273017202159\\
0.838306582515661	3.60170497518489\\
1.02611651957107	3.60772225884434\\
};
\addplot [color=mycolor2, forget plot]
  table[row sep=crcr]{%
2.13066155564074	1.78934952646239\\
2.31729935703494	1.78364460719312\\
2.46689620124448	1.76952844019794\\
2.58781105164417	1.75031033791726\\
2.68652254186893	1.72811583414305\\
2.76796031070307	1.70428226501363\\
2.83585227040171	1.67963405118568\\
2.89301957020744	1.65466520604145\\
2.94160681957125	1.6296575131894\\
2.98325487595012	1.6047562069412\\
3.01922820624035	1.58001819006828\\
3.05050810394904	1.55544264470556\\
3.07786083388366	1.53099034984978\\
3.10188752944395	1.50659570797689\\
3.12306081329798	1.4821740068146\\
3.14175170109824	1.45762550489228\\
3.15824931183566	1.43283733227823\\
3.17277515750889	1.4076838126804\\
3.18549323891532	1.38202555853736\\
3.19651677012285	1.35570751532808\\
3.20591204311225	1.32855600187292\\
3.21369968872289	1.30037468766754\\
3.21985335966386	1.27093935036681\\
3.22429562795569	1.23999115375056\\
3.22689062345455	1.20722806743009\\
3.22743260712858	1.17229390269361\\
3.22562922655149	1.13476425208792\\
3.22107757786229	1.09412838138774\\
3.21323030795922	1.04976582242581\\
3.20134770676748	1.00091605681835\\
3.18442989421847	0.946639300923022\\
3.16112060377793	0.885766123130951\\
3.12957055354315	0.816833764968077\\
3.08724411929405	0.73800835618114\\
3.03064910810781	0.64699640659248\\
2.95496974575675	0.540959600987511\\
2.85359820694491	0.416470990893481\\
2.71761568164062	0.269599915374646\\
2.53542181057784	0.0963008308050649\\
2.29302638863571	-0.106599668422552\\
1.97600821968529	-0.339474794327379\\
1.5744278205967	-0.597031272628531\\
1.09079846615224	-0.865714438789585\\
0.547271679730867	-1.12426677793354\\
-0.0156987973804935	-1.34975858100649\\
-0.552609514395057	-1.5264343025763\\
-1.02938785202724	-1.65058730346741\\
-1.43097456897238	-1.72843647162033\\
-1.75816612094634	-1.77058643378522\\
-2.02028675676074	-1.78756478900962\\
-2.22921263119039	-1.78781501291504\\
-2.39615822324344	-1.77739781919804\\
-2.53049340461053	-1.76040181520039\\
-2.63960205598709	-1.73948683427999\\
-2.72914325726933	-1.71634290147778\\
-2.80340494859105	-1.69202300667229\\
-2.86562888501474	-1.66716817309768\\
-2.9182730255947	-1.64215460857281\\
-2.96321163045467	-1.6171883613051\\
-3.00188381624089	-1.5923657554321\\
-3.0354025628448	-1.56771182882603\\
-3.06463440778332	-1.54320467631437\\
-3.09025774070329	-1.51879072927095\\
-3.11280553904297	-1.49439415253789\\
-3.13269675775084	-1.46992236302528\\
-3.15025937346884	-1.44526892684105\\
-3.16574720105731	-1.42031461319411\\
-3.17935196092714	-1.39492707139712\\
-3.19121160749489	-1.3689593874057\\
-3.20141557668629	-1.34224762723294\\
-3.21000733136552	-1.3146073592814\\
-3.21698434428276	-1.28582904753639\\
-3.2222954297344	-1.25567210867758\\
-3.22583508947928	-1.22385731664626\\
-3.22743424402313	-1.19005710671384\\
-3.22684633746756	-1.15388316582293\\
-3.22372727881572	-1.11487048454915\\
-3.21760693908985	-1.07245677716519\\
-3.20784885476026	-1.02595584522598\\
-3.19359324689341	-0.974523082798222\\
-3.17367626606242	-0.91711096815095\\
-3.14651532584494	-0.852412257800675\\
-3.10994643917659	-0.778789190780974\\
-3.06099512366542	-0.694189465974422\\
-2.99555986643644	-0.596056555476247\\
-2.90799288715264	-0.48125811924112\\
-2.79059380934311	-0.346091178417871\\
-2.63312516162728	-0.186489932883004\\
-2.42268081912405	0.00133018142641184\\
-2.14465351030925	0.219451212471439\\
-1.78602806334448	0.465803890576448\\
-1.34199072149221	0.731173209142785\\
-0.824317574584908	0.997756503334357\\
-0.265278147565503	1.24239048713434\\
0.28996029303649	1.44471801699729\\
0.799904615435649	1.59485812742332\\
1.23981820678715	1.69467540015643\\
1.60340305424167	1.75327938401769\\
1.8966275701755	1.78163647045096\\
2.13066155564074	1.78934952646239\\
};
\addplot [color=mycolor1]
  table[row sep=crcr]{%
0.948683298050514	3.79473319220206\\
1.12996891013208	3.78902241071535\\
1.3027838103604	3.77255915923418\\
1.46753552344581	3.74623053454947\\
1.62465224482371	3.71077288416293\\
1.77456152903207	3.66677970030574\\
1.91767403561802	3.61471073626391\\
2.05437110063153	3.55490128334521\\
2.18499505421831	3.4875709448019\\
2.30984136685525	3.41283153872394\\
2.42915185734506	3.33069397942336\\
2.54310832592638	3.24107414976038\\
2.65182608463458	3.14379790634541\\
2.75534694820212	3.03860547381077\\
2.8536313296431	2.92515559896292\\
2.9465491655584	2.80302996392973\\
3.0338694906698	2.67173851056894\\
3.11524860652315	2.53072651479627\\
3.19021696761039	2.3793844730251\\
3.25816516606007	2.21706211963439\\
3.31832976399147	2.0430881668871\\
3.36978023105031	1.85679760784185\\
3.41140891541445	1.65756857671014\\
3.44192680709927	1.44487070344488\\
3.45986879009691	1.21832646083672\\
3.46361298678702	0.977785965288651\\
3.45141941180798	0.723413822791541\\
3.4214930614047	0.455783738589644\\
3.37207522993717	0.175972765268488\\
3.30156372141992	-0.114357323644244\\
3.20865740947072	-0.412907981729085\\
3.0925136091427	-0.716739496073036\\
2.95289924526556	-1.02233055985344\\
2.79031115637498	-1.3257107709346\\
2.60603988731933	-1.6226624623078\\
2.40215712665582	-1.90897292279738\\
2.18141953804948	-2.18070442957845\\
1.94709823579504	-2.43444260737754\\
1.70275852491129	-2.66748659951055\\
1.45202370644265	-2.877956852001\\
1.19835703486066	-3.0648138241024\\
0.944888096380122	-3.22779787311511\\
0.694297500839398	-3.36731218002033\\
0.448761194715288	-3.48427490957016\\
0.209946190700892	-3.57996468520442\\
-0.0209555786883461	-3.65587741213202\\
-0.243169602253506	-3.71360522906444\\
-0.456269901983927	-3.7547419037936\\
-0.660108423391693	-3.78081428495334\\
-0.854748940791145	-3.79323662458319\\
-1.04040889805107	-3.79328335955738\\
-1.21741050351997	-3.78207578896275\\
-1.38614100991934	-3.76057854694004\\
-1.54702130890241	-3.72960250007616\\
-1.70048159286015	-3.6898114747465\\
-1.84694273429099	-3.64173092446581\\
-1.98680208954437	-3.58575723170233\\
-2.1204225680783	-3.52216679568137\\
-2.24812396922559	-3.45112440165093\\
-2.37017574651009	-3.37269062063008\\
-2.48679050050333	-3.2868281760587\\
-2.59811762050387	-3.19340735751618\\
-2.70423659456836	-3.09221068144532\\
-2.80514959232286	-2.98293711145668\\
-2.90077300437504	-2.86520627071399\\
-2.99092770791458	-2.73856321838213\\
-3.07532793559953	-2.60248453106578\\
-3.1535687736533	-2.45638663503084\\
-3.22511252958097	-2.29963757630341\\
-3.28927451889835	-2.13157368366421\\
-3.34520925490937	-1.95152284702656\\
-3.39189861348128	-1.7588363466938\\
-3.42814429814894	-1.5529312348605\\
-3.45256782598978	-1.33334504716437\\
-3.46362220116165	-1.0998039137315\\
-3.45962024522194	-0.852303713618117\\
-3.43878487541956	-0.591201557879526\\
-3.3993259814164	-0.317311504170883\\
-3.3395463824176	-0.0319942042762583\\
-3.25797518618734	0.262774139257094\\
-3.15352067565559	0.564371272065384\\
-3.02562738470585	0.869554764908886\\
-2.87441511977517	1.17455760801215\\
-2.70077402394131	1.47525676358082\\
-2.50639202484484	1.767403513355\\
-2.29370038309167	2.04688929885519\\
-2.06573813670639	2.31000991273335\\
-1.8259528645897	2.5536887433386\\
-1.57796799985389	2.77562773795799\\
-1.32535178123783	2.9743703475303\\
-1.07141885643386	3.14927865398809\\
-0.819084909373598	3.30044159903781\\
-0.570781691446472	3.42853925898355\\
-0.328428492520302	3.53468897189284\\
-0.093448740680559	3.62029468577783\\
0.13318251344032	3.68691393725669\\
0.350873927242831	3.73614982594868\\
0.55934713845833	3.76956970364108\\
0.758567824726602	3.78864856747597\\
0.948683298050513	3.79473319220206\\
};
\addlegendentry{Внешние аппроксимации}

\addplot [color=mycolor2, forget plot]
  table[row sep=crcr]{%
2.22721472973942	1.94935886896179\\
2.3886648425121	1.94441487511101\\
2.5195634483927	1.93205457572182\\
2.62670779849127	1.91501766628131\\
2.71531853364	1.89508786650117\\
2.78936864949369	1.87341097537522\\
2.85187814440773	1.85071228241933\\
2.90515032165589	1.82744075521499\\
2.9509520177711	1.80386332654994\\
2.99064783647547	1.78012618777659\\
3.02529909977035	1.7562945351756\\
3.0557365623221	1.73237827445755\\
3.08261387510984	1.7083485310185\\
3.10644696879361	1.68414807630787\\
3.12764310033082	1.65969765855822\\
3.14652223914943	1.63489950162782\\
3.16333268711427	1.60963876367151\\
3.17826225469757	1.5837834336332\\
3.19144589188063	1.55718292750329\\
3.20297034763997	1.52966548843736\\
3.2128761683912	1.50103436869378\\
3.2211571124149	1.47106265802835\\
3.22775682617807	1.43948650786522\\
3.23256237104755	1.40599637010381\\
3.23539387180292	1.3702257107414\\
3.23598913837185	1.33173645766061\\
3.23398152983158	1.29000018434938\\
3.22886850103885	1.24437370419161\\
3.21996707976175	1.19406735000614\\
3.20635080827997	1.13810376656035\\
3.18676025898309	1.07526465092859\\
3.15947593702972	1.0040228169715\\
3.12213828317303	0.922457920675385\\
3.07149545967596	0.828157721349833\\
3.00305879148097	0.718116213117772\\
2.91065699973037	0.588662002756959\\
2.785925244978	0.435496623074383\\
2.61788823849866	0.254008232649376\\
2.39306798253518	0.0401523940333445\\
2.09699730361163	-0.20771900183087\\
1.7183819635995	-0.485932899271695\\
1.25639683089422	-0.782386182967809\\
0.728319451607428	-1.07597695395358\\
0.17048677629427	-1.34156752261761\\
-0.372344302750219	-1.55919115860543\\
-0.863264272549092	-1.72086595365401\\
-1.28299919742863	-1.8302347441912\\
-1.62890152595063	-1.89731652670458\\
-1.90833376158261	-1.93331843320936\\
-2.13240654410545	-1.94782687646357\\
-2.3122537087605	-1.94803363323748\\
-2.45746747168844	-1.93896359662991\\
-2.57574556593285	-1.92399109460735\\
-2.67305603610964	-1.90533078490532\\
-2.75395564612754	-1.88441460443591\\
-2.82190758462804	-1.86215611801615\\
-2.87954766726586	-1.83912801979308\\
-2.92889157233296	-1.81567886095776\\
-2.97149068204595	-1.79200910352774\\
-3.00854741762708	-1.76822049003994\\
-3.04100007190525	-1.74434802267285\\
-3.0695851473103	-1.7203805908067\\
-3.09488323143276	-1.69627413213334\\
-3.11735281839452	-1.67195981527391\\
-3.13735524534064	-1.64734883071173\\
-3.15517299859553	-1.62233479306363\\
-3.17102297575804	-1.59679437416259\\
-3.18506579905015	-1.570586527561\\
-3.19741190629232	-1.54355048208521\\
-3.20812485602917	-1.51550254281508\\
-3.21722203851139	-1.48623162017347\\
-3.22467275544594	-1.45549329519237\\
-3.23039339092638	-1.42300210774249\\
-3.23423911335776	-1.38842161161915\\
-3.23599118590349	-1.35135156242766\\
-3.23533847055249	-1.31131137681683\\
-3.23185101774822	-1.2677187105326\\
-3.22494264006069	-1.21986163848029\\
-3.2138179373885	-1.16686249014231\\
-3.19739719631547	-1.10763095340298\\
-3.17420973909784	-1.0408037811618\\
-3.142242561071	-0.964668760862181\\
-3.09872682936675	-0.877072578179989\\
-3.0398417586358	-0.77531812311728\\
-2.96031892266926	-0.656071422287709\\
-2.85295425309884	-0.515330753928937\\
-2.70811161177175	-0.348574764250833\\
-2.51348980394932	-0.151314840167112\\
-2.25479111797749	0.0795960527623154\\
-1.91840618384291	0.343563906305597\\
-1.49721915275415	0.633014609293368\\
-0.998726638487032	0.931115827109969\\
-0.450328181715587	1.21375185121395\\
0.105465468340197	1.45718721810745\\
0.625945116714116	1.64701594620063\\
1.08251666630864	1.78154116777267\\
1.46486936946524	1.86834353319716\\
1.77626331377619	1.91854897142989\\
2.02656829443923	1.94275375393752\\
2.22721472973941	1.94935886896179\\
};
\addplot [color=mycolor2]
  table[row sep=crcr]{%
2.12132034355964	2.82842712474619\\
2.18582575000323	2.82640660874952\\
2.2459761483502	2.82067996094826\\
2.30316600770785	2.81153894842004\\
2.35833530419819	2.79908319427389\\
2.4121407651283	2.78328504568152\\
2.46505313939319	2.76402339892432\\
2.51741350921551	2.74110157112695\\
2.56946568278775	2.71425682483196\\
2.62137377470791	2.68316571665979\\
2.67322998558975	2.64744786912504\\
2.72505542741334	2.60667010622203\\
2.77679571750944	2.56035271147659\\
2.82831256449113	2.50797963523731\\
2.87937250576692	2.44901464186941\\
2.92963424895255	2.38292549191834\\
2.97863667364875	2.30921809960124\\
3.02579039162557	2.22748193169334\\
3.0703766721051	2.13744642550811\\
3.11155818595542	2.03904567139316\\
3.14840590189045	1.93248505458784\\
3.1799449834646	1.81829951655162\\
3.20521925970046	1.69738980361013\\
3.2233689191928	1.57102231442397\\
3.23371058284854	1.44078163681543\\
3.23580480244767	1.30847305653245\\
3.22949541157146	1.1759834831\\
3.21490900216393	1.04511946165368\\
3.19240995273442	0.917445713323391\\
3.16251366135309	0.794144173493542\\
3.12576385298257	0.675902054710668\\
3.08257536776515	0.562820516214273\\
3.03302813706062	0.454314775018543\\
2.97656426645917	0.348948817128694\\
2.91146921429886	0.244099783305823\\
2.83385680006128	0.135243682019644\\
2.73548220960228	0.0144130439331602\\
2.59871334341581	-0.133168683665202\\
2.38479638365891	-0.336180712113444\\
2.01057175466146	-0.648371525543042\\
1.34210041814469	-1.13791067535239\\
0.375352321166521	-1.75822148716338\\
-0.527448468220416	-2.26262801632552\\
-1.11365847778129	-2.5437869612038\\
-1.4567327767317	-2.68228003389134\\
-1.6691296256022	-2.75259678373195\\
-1.81428800439327	-2.79054335642282\\
-1.92320462744895	-2.81168897715777\\
-2.01132392805712	-2.82302365363059\\
-2.08680248126988	-2.82787464184343\\
-2.15422943553058	-2.82790871121207\\
-2.21634478307291	-2.8239827357917\\
-2.27487422572476	-2.8165264847195\\
-2.33095693947289	-2.80572465930185\\
-2.3853755403757	-2.79160756168836\\
-2.43868469976231	-2.77409782070174\\
-2.49128507004909	-2.75303495626884\\
-2.54346612600745	-2.7281884019911\\
-2.59543032955127	-2.69926454755406\\
-2.64730535029083	-2.66591103001887\\
-2.69914810041563	-2.62772046423944\\
-2.75094277189846	-2.58423541950896\\
-2.80259428647749	-2.53495641420315\\
-2.85391830194335	-2.47935483422118\\
-2.90462903903618	-2.41689283661657\\
-2.95432664759112	-2.34705230341292\\
-3.00248656709011	-2.26937453210355\\
-3.04845424185856	-2.18351130413452\\
-3.0914493816545	-2.08928598065547\\
-3.13058429431929	-1.98676020448289\\
-3.16490008683964	-1.87629788656397\\
-3.19342217916266	-1.75861429449935\\
-3.21523240499265	-1.63479579561046\\
-3.22954956680171	-1.50627700312196\\
-3.23580523135335	-1.37476797309193\\
-3.23369896357047	-1.2421341494958\\
-3.22321880671614	-1.11024302689198\\
-3.20461860896113	-0.980799507713277\\
-3.17835148049538	-0.85519274756288\\
-3.14496439092874	-0.734369602871579\\
-3.10495870881183	-0.618735254776386\\
-3.05861192744495	-0.508062722468019\\
-3.0057323937692	-0.401369642978784\\
-2.94527025004781	-0.296685462718974\\
-2.87460204254734	-0.190562899149911\\
-2.78805571026974	-0.0770301658026245\\
-2.67361421522119	0.0546866392302208\\
-2.50520482475694	0.225105040795738\\
-2.22541867988909	0.474087715816021\\
-1.7209533613293	0.868461990547897\\
-0.880186083660502	1.44500069250654\\
0.110323684998138	2.03883159526995\\
0.859259206598599	2.42737283484743\\
1.30741523004191	2.62511520505649\\
1.57426953542748	2.72304263504847\\
1.74766615441039	2.77435113512036\\
1.87207484559155	2.80265686669549\\
1.96924085114733	2.81832059141065\\
2.0502990590364	2.82612977760851\\
2.12132034355964	2.82842712474619\\
};
\addlegendentry{Внутренние аппроксимации}

\end{axis}

\begin{axis}[%
width=0.798\linewidth,
height=0.578\linewidth,
at={(-0.104\linewidth,-0.064\linewidth)},
scale only axis,
xmin=0,
xmax=1,
ymin=0,
ymax=1,
axis line style={draw=none},
ticks=none,
axis x line*=bottom,
axis y line*=left,
legend style={legend cell align=left, align=left, draw=white!15!black}
]
\end{axis}
\end{tikzpicture}%
        \caption{Эллипсоидальные аппроксимации для 50 направлений.}
\end{figure}
%%%%%%%%%%%%%%%%%%%%%%%%%%%%%%%%%%%%%%%%%%%%%%%%%%%%%%%%%%%%%%%%%%%%%%%%%%%%%%%%


%% Document end


%%%%%%%%%%%%%%%%%%%%%%%%%%%%%%%%%%%%%%%%%%%%%%%%%%%%%%%%%%%%%%%%%%%%%%%%%%%%%%%%
\clearpage
\begin{thebibliography}{9}
% Библиография
\bibitem{kurzh} Kurzhanski A. B., Varaiya P. \textit{Dynamics and Control of Trajectory Tubes.} Birkhauser, 2014.
\end{thebibliography}
\end{document}
\tableofcontents
\clearpage
%%%%%%%%%%%%%%%%%%%%%%%%%%%%%%%%%%%%%%%%%%%%%%%%%%%%%%%%%%%%%%%%%%%%%%%%%%%%%%%%        


%%  Document start


%%%%%%%%%%%%%%%%%%%%%%%%%%%%%%%%%%%%%%%%%%%%%%%%%%%%%%%%%%%%%%%%%%%%%%%%%%%%%%%%

\section{Об эллипсоидах и сумме Минковского}

\begin{definition}
        Назовём \textit{эллипсоидом} множество
$$
        \mathcal{E}(q,\,Q) = \{
x \in \setR^n \,:\, \langle x-q,\,Q^{-1}(x-q) \rangle \leqslant 1
        \},
        \qquad \mbox{где } Q= Q\T > 0.
$$
\end{definition}

\begin{assertion}
        Опорная функция и опорный вектор эллипсоида имеют вид:
$$
        \rho(l\,|\,\Varepsilon(q,\,Q)) =
        \langle l,\,q \rangle + 
        \langle l,\,Ql \rangle^{\nicefrac{1}{2}},
$$
$$
        x(l) = q + \frac{Ql}{\langle l,\,Ql\rangle^{\nicefrac12}}.
$$
\end{assertion}
\begin{proof}

Будем доказывать для случая $q = 0$. Иначе --- аналогично.

Так как по определению
$
        \rho(l\,|\,A) = \sup_{x\in A}\langle l,\,x\rangle,
$
то мы должны решать задачу максимизации скалярного произведения
$
        \langle l,\,x \rangle
$
при условии, что
$
        \langle x,\,Q^{-1}x \rangle = 1.
$
Запишем функцию Лагранжа для этой задачи:
$$
        \mathcal{L}(l,\,x,\,\lambda)
        =
        \langle l,\,x \rangle
        +
        \lambda(\langle x,\,Q^{-1}x \rangle - 1).
$$
Тогда
$$
        \frac{\partial \mathcal{L}}{\partial x}
        =
        l + 2\lambda Q^{-1} x 
        = 0
        \quad
        \Longrightarrow
        \quad
        x(l) = -\frac{1}{2\lambda}Ql.
$$
Подставим получившееся выражение для опорного вектора в условие:
$$
        \left\langle
-\frac{1}{2\lambda}Ql
,\,
-\frac{1}{2\lambda}Q^{-1}Ql
        \right\rangle
        = 1
        \;\Longrightarrow\;
        \lambda
        =
        -\frac12 \langle l,\,Ql \rangle^{\nicefrac12}
        \,\Longrightarrow\,
        x(l) = \frac{Ql}{\langle l,\,Ql\rangle^{\nicefrac12}}.
$$
В таком случае опорная функция в направлении $l \neq 0$ равна
$$
        \rho(l\,|\,\Varepsilon(0,\,Q))
        =
        \left\langle
l
,\,
\frac{Ql}{\langle l,\,Ql\rangle^{\nicefrac12}}
        \right\rangle
        =
        \langle l,\,Ql \rangle^{\nicefrac{1}{2}}.
$$
\end{proof}

\begin{definition}
        \textit{Суммой Минковского} множеств $A$ и $B$ называется множество
$$
        A + B
        =
        \{\,
x = a + b \::\: a \in A,\,b \in B 
        \,\}.
$$
\end{definition}

\begin{remark}
        Сумма эллипсоидов, вообще говоря, не является эллипсоидом.
        В этом можно убедиться на следующем примере:
$$
        \Varepsilon\left(
                0,\,
                \begin{pmatrix}
                        1 & 0 \\
                        0 & 0
                \end{pmatrix}
        \right)
        +
        \Varepsilon\left(
                0,\,
                \begin{pmatrix}
                        0 & 0 \\
                        0 & 1
                \end{pmatrix}
        \right)
        =
        [0 ,\, 1] \times [0 ,\, 1].
$$
\end{remark}

\begin{assertion}
        Опорная функция суммы Минковского равна сумме опорных функций каждого из множеств, то есть
$$
        \rho\left(l\,\left|\,\sum_{i=1}^n A_i\right.\right) = \sum_{i=1}^n \rho\,(l\,|\,A_i).
$$
\end{assertion}

\clearpage
\begin{figure}[t]
        \centering
        % This file was created by matlab2tikz.
%
%The latest updates can be retrieved from
%  http://www.mathworks.com/matlabcentral/fileexchange/22022-matlab2tikz-matlab2tikz
%where you can also make suggestions and rate matlab2tikz.
%
\definecolor{mycolor1}{rgb}{0.00000,0.44700,0.74100}%
%
\begin{tikzpicture}

\begin{axis}[%
width=0.618\linewidth,
height=0.471\linewidth,
at={(0\linewidth,0\linewidth)},
scale only axis,
xmin=-2,
xmax=4,
xlabel style={font=\color{white!15!black}},
xlabel={$x_1$},
ymin=0.5,
ymax=3.5,
ylabel style={font=\color{white!15!black}},
ylabel={$x_2$},
axis background/.style={fill=white},
axis x line*=bottom,
axis y line*=left,
xmajorgrids,
ymajorgrids,
legend style={legend cell align=left, align=left, draw=white!15!black}
]
\addplot [color=mycolor1]
  table[row sep=crcr]{%
3.12132034355964	3.41421356237309\\
3.14093309737706	3.41361883293874\\
3.15610096472476	3.41218966892045\\
3.16813059909176	3.41027848839136\\
3.17787846987421	3.40808691034947\\
3.18592472730893	3.40573192203561\\
3.19267367435337	3.40328137559688\\
3.19841408644868	3.40077372446074\\
3.20335666517915	3.3982293272274\\
3.20765799352175	3.39565707404995\\
3.21143626106343	3.3930583315506\\
3.2147818220386	3.39042930139934\\
3.217764420063	3.38776240921808\\
3.22043820622192	3.38504707754847\\
3.22284525774146	3.38227008706717\\
3.2250180482594	3.37941564172179\\
3.22698115917429	3.37646519817436\\
3.22875241567428	3.37339708240557\\
3.23034355770318	3.37018588734886\\
3.23176050074785	3.3668016188298\\
3.23300319375062	3.36320852788002\\
3.23406503318587	3.35936353046923\\
3.23493173445265	3.35521406411806\\
3.2355794824484	3.35069515469921\\
3.23597206441911	3.34572534988779\\
3.23605649958035	3.34020099103578\\
3.23575636783317	3.33398799537805\\
3.23496150515886	3.32690982030174\\
3.23351178703941	3.31872942300207\\
3.23117098985666	3.3091215094024\\
3.22758343607311	3.29762858411142\\
3.22219964903172	3.28358902636704\\
3.21414389031876	3.26601495419557\\
3.20196759721768	3.24337600812861\\
3.18316711789886	3.21319839168308\\
3.15318699112585	3.17128288128172\\
3.10323675000489	3.11010066257133\\
3.01525609198129	3.01537616591489\\
2.85016840053085	2.85895352273754\\
2.52458200959445	2.58760093991863\\
1.90440137584336	2.13363920441521\\
0.985501334090361	1.54409660598879\\
0.130059671215813	1.06605503500581\\
-0.409288268560127	0.807214909116197\\
-0.70597634609565	0.687302123652136\\
-0.872419406680929	0.632086042700855\\
-0.97199649313969	0.605972330977308\\
-1.03563236326191	0.593558335040314\\
-1.07869104959228	0.587977176231719\\
-1.10924316645516	0.585982787778395\\
-1.13178875034531	0.58594871840975\\
-1.14897766153775	0.587018094070618\\
-1.1624464899118	0.588720827478583\\
-1.17324845072647	0.59079102809827\\
-1.18208575929697	0.593075264744444\\
-1.18944113039835	0.595484336841729\\
-1.19565527927026	0.597966914051153\\
-1.20097426544368	0.600494649340231\\
-1.20557934247511	0.603053545598106\\
-1.20960630798951	0.605638849748727\\
-1.21315835534862	0.608252001222227\\
-1.2163147887704	0.610898815342025\\
-1.21913703504296	0.613588435376933\\
-1.22167284234597	0.616332784416996\\
-1.22395923010075	0.619146362645509\\
-1.22602455119113	0.622046304964338\\
-1.22788989784863	0.625052658896365\\
-1.2295699951167	0.628188875042512\\
-1.23107366292106	0.631482529497297\\
-1.23240387740984	0.634966324917864\\
-1.23355741534321	0.638679449344891\\
-1.23452401346765	0.642669415179332\\
-1.23528490798758	0.646994563037611\\
-1.23581052306342	0.651727509156087\\
-1.23605692848604	0.656959961411399\\
-1.23596044514103	0.662809563576559\\
-1.23542936997175	0.669429813383256\\
-1.23433108371332	0.677024752794653\\
-1.23247153021665	0.685871266354724\\
-1.22956168278891	0.696353870420598\\
-1.22516102360426	0.709020692660253\\
-1.21857881304302	0.724676739372734\\
-1.2086944178179	0.744545507733446\\
-1.19361473557335	0.770561635854139\\
-1.16998584185419	0.805927250653545\\
-1.13152846224427	0.856225279395141\\
-1.06573985695198	0.931754081541382\\
-0.946173527310501	1.05251473853723\\
-0.715145074514351	1.25784691099334\\
-0.25915963680298	1.61407245297346\\
0.533529697525656	2.15749204989669\\
1.47643270260674	2.72280409766538\\
2.17825862936543	3.08704498818123\\
2.57979469508172	3.26437318172905\\
2.80048557324685	3.34548951140256\\
2.92812298873437	3.38335485119594\\
3.00711025420764	3.40139640666682\\
3.05911603477277	3.40983029938284\\
3.09519171166257	3.41334200179773\\
3.12132034355964	3.41421356237309\\
};

\end{axis}

\begin{axis}[%
width=0.798\linewidth,
height=0.578\linewidth,
at={(-0.104\linewidth,-0.064\linewidth)},
scale only axis,
xmin=0,
xmax=1,
ymin=0,
ymax=1,
axis line style={draw=none},
ticks=none,
axis x line*=bottom,
axis y line*=left,
legend style={legend cell align=left, align=left, draw=white!15!black}
]
\end{axis}
\end{tikzpicture}%
        \caption{Эллипсоид с центром $q = \protect\begin{bmatrix}1\\2\protect\end{bmatrix}$ и матрицей $Q = \protect\begin{bmatrix}5&3\\3&2\protect\end{bmatrix}.$}
\end{figure}
\begin{figure}[b]
        \centering
        % This file was created by matlab2tikz.
%
%The latest updates can be retrieved from
%  http://www.mathworks.com/matlabcentral/fileexchange/22022-matlab2tikz-matlab2tikz
%where you can also make suggestions and rate matlab2tikz.
%
\definecolor{mycolor1}{rgb}{0.00000,0.44700,0.74100}%
\definecolor{mycolor2}{rgb}{0.85000,0.32500,0.09800}%
%
\begin{tikzpicture}

\begin{axis}[%
width=0.618\linewidth,
height=0.487\linewidth,
at={(0\linewidth,0\linewidth)},
scale only axis,
xmin=-4,
xmax=4,
xlabel style={font=\color{white!15!black}},
xlabel={$x_1$},
ymin=-3,
ymax=3,
ylabel style={font=\color{white!15!black}},
ylabel={$x_2$},
axis background/.style={fill=white},
axis x line*=bottom,
axis y line*=left,
xmajorgrids,
ymajorgrids,
legend style={at={(0.03,0.97)}, anchor=north west, legend cell align=left, align=left, draw=white!15!black}
]
\addplot [color=mycolor1]
  table[row sep=crcr]{%
0	1.41421356237309\\
0.0448926526261715	1.41278777581078\\
0.0898751836254369	1.40849029202781\\
0.135035408616089	1.40126046002867\\
0.180456834323975	1.39099628392442\\
0.226216037819367	1.37755312364591\\
0.272379465039817	1.36074202332744\\
0.31899942276683	1.3403278466662\\
0.366109017608597	1.31602749760457\\
0.413715781186158	1.28750864260985\\
0.461793724526318	1.25438953757444\\
0.510273605374737	1.21624080482269\\
0.559031297446442	1.17259030225851\\
0.607874358269208	1.12293255768884\\
0.656527248025466	1.06674455480223\\
0.704616200693154	1.00350985019654\\
0.751655514474458	0.932752901426886\\
0.797037975951299	0.854084849287771\\
0.840033114401921	0.767260538159248\\
0.879797685207568	0.672244052563361\\
0.915402708139832	0.569276526707822\\
0.945879950278728	0.458935986082397\\
0.970287525247814	0.342175739492072\\
0.987789436744396	0.22032715972476\\
0.997738518429436	0.0950562869276414\\
0.999748302867315	-0.0317279345033261\\
0.993739043738288	-0.15800451227805\\
0.979947497005068	-0.281790358648065\\
0.958898165695011	-0.401283709678683\\
0.931342671496431	-0.514977335908855\\
0.898180416909459	-0.62172652940075\\
0.860375718733436	-0.720768510152771\\
0.818884246741863	-0.81170017917703\\
0.774596669241484	-0.894427190999916\\
0.728302096399996	-0.969098608377261\\
0.68066980893543	-1.03603919926208\\
0.632245459597382	-1.09568761863817\\
0.583457251434522	-1.14854484958009\\
0.534627983128064	-1.19513423485098\\
0.485989745067013	-1.23597246546167\\
0.437699042301328	-1.2715498797676\\
0.38985098707616	-1.30231809315216\\
0.342491860563771	-1.32868305133133\\
0.295629790778837	-1.35100187031999\\
0.249243569363955	-1.36958215754347\\
0.203289781078724	-1.3846828264328\\
0.157708488746422	-1.39651568740012\\
0.112427735812955	-1.40524731219809\\
0.0673671215351546	-1.41100082986231\\
0.0224406851852784	-1.41385742962182\\
-0.0224406851852783	-1.41385742962182\\
-0.0673671215351544	-1.41100082986231\\
-0.112427735812955	-1.40524731219809\\
-0.157708488746422	-1.39651568740012\\
-0.203289781078724	-1.3846828264328\\
-0.249243569363954	-1.36958215754347\\
-0.295629790778837	-1.35100187031999\\
-0.342491860563771	-1.32868305133133\\
-0.389850987076159	-1.30231809315216\\
-0.437699042301328	-1.2715498797676\\
-0.485989745067013	-1.23597246546167\\
-0.534627983128064	-1.19513423485098\\
-0.583457251434522	-1.14854484958009\\
-0.632245459597382	-1.09568761863817\\
-0.68066980893543	-1.03603919926208\\
-0.728302096399996	-0.969098608377261\\
-0.774596669241483	-0.894427190999916\\
-0.818884246741864	-0.811700179177029\\
-0.860375718733436	-0.720768510152772\\
-0.898180416909458	-0.621726529400751\\
-0.931342671496431	-0.514977335908856\\
-0.958898165695011	-0.401283709678683\\
-0.979947497005068	-0.281790358648066\\
-0.993739043738288	-0.15800451227805\\
-0.999748302867315	-0.0317279345033253\\
-0.997738518429436	0.0950562869276414\\
-0.987789436744396	0.22032715972476\\
-0.970287525247814	0.34217573949207\\
-0.945879950278728	0.458935986082396\\
-0.915402708139832	0.569276526707822\\
-0.879797685207568	0.672244052563361\\
-0.840033114401921	0.767260538159247\\
-0.797037975951299	0.85408484928777\\
-0.751655514474458	0.932752901426886\\
-0.704616200693154	1.00350985019654\\
-0.656527248025466	1.06674455480223\\
-0.607874358269208	1.12293255768884\\
-0.559031297446443	1.17259030225851\\
-0.510273605374738	1.21624080482269\\
-0.461793724526318	1.25438953757444\\
-0.413715781186158	1.28750864260985\\
-0.366109017608597	1.31602749760457\\
-0.318999422766831	1.3403278466662\\
-0.272379465039818	1.36074202332744\\
-0.226216037819367	1.37755312364591\\
-0.180456834323975	1.39099628392442\\
-0.135035408616089	1.40126046002867\\
-0.0898751836254374	1.40849029202781\\
-0.0448926526261721	1.41278777581078\\
-1.73191211247099e-16	1.41421356237309\\
};
\addlegendentry{Эллипсоиды}

\addplot [color=mycolor1, forget plot]
  table[row sep=crcr]{%
2.12132034355964	1.41421356237309\\
2.14093309737706	1.41361883293874\\
2.15610096472476	1.41218966892045\\
2.16813059909176	1.41027848839136\\
2.17787846987421	1.40808691034947\\
2.18592472730893	1.40573192203561\\
2.19267367435337	1.40328137559688\\
2.19841408644868	1.40077372446074\\
2.20335666517915	1.39822932722739\\
2.20765799352175	1.39565707404995\\
2.21143626106343	1.3930583315506\\
2.2147818220386	1.39042930139934\\
2.217764420063	1.38776240921808\\
2.22043820622192	1.38504707754847\\
2.22284525774146	1.38227008706717\\
2.2250180482594	1.37941564172179\\
2.22698115917429	1.37646519817436\\
2.22875241567428	1.37339708240557\\
2.23034355770318	1.37018588734886\\
2.23176050074785	1.36680161882979\\
2.23300319375062	1.36320852788002\\
2.23406503318587	1.35936353046923\\
2.23493173445265	1.35521406411806\\
2.2355794824484	1.35069515469921\\
2.23597206441911	1.34572534988779\\
2.23605649958035	1.34020099103578\\
2.23575636783317	1.33398799537805\\
2.23496150515886	1.32690982030174\\
2.23351178703941	1.31872942300207\\
2.23117098985666	1.3091215094024\\
2.22758343607311	1.29762858411142\\
2.22219964903172	1.28358902636704\\
2.21414389031876	1.26601495419557\\
2.20196759721768	1.24337600812861\\
2.18316711789886	1.21319839168308\\
2.15318699112585	1.17128288128172\\
2.10323675000489	1.11010066257133\\
2.01525609198129	1.01537616591488\\
1.85016840053085	0.858953522737536\\
1.52458200959445	0.587600939918628\\
0.904401375843364	0.133639204415214\\
-0.0144986659096387	-0.455903394011212\\
-0.869940328784187	-0.933944964994194\\
-1.40928826856013	-1.1927850908838\\
-1.70597634609565	-1.31269787634786\\
-1.87241940668093	-1.36791395729915\\
-1.97199649313969	-1.39402766902269\\
-2.03563236326191	-1.40644166495969\\
-2.07869104959228	-1.41202282376828\\
-2.10924316645516	-1.4140172122216\\
-2.13178875034531	-1.41405128159025\\
-2.14897766153775	-1.41298190592938\\
-2.1624464899118	-1.41127917252142\\
-2.17324845072647	-1.40920897190173\\
-2.18208575929697	-1.40692473525556\\
-2.18944113039835	-1.40451566315827\\
-2.19565527927026	-1.40203308594885\\
-2.20097426544368	-1.39950535065977\\
-2.20557934247511	-1.39694645440189\\
-2.20960630798951	-1.39436115025127\\
-2.21315835534862	-1.39174799877777\\
-2.2163147887704	-1.38910118465798\\
-2.21913703504296	-1.38641156462307\\
-2.22167284234597	-1.383667215583\\
-2.22395923010075	-1.38085363735449\\
-2.22602455119113	-1.37795369503566\\
-2.22788989784863	-1.37494734110363\\
-2.2295699951167	-1.37181112495749\\
-2.23107366292106	-1.3685174705027\\
-2.23240387740984	-1.36503367508214\\
-2.23355741534321	-1.36132055065511\\
-2.23452401346765	-1.35733058482067\\
-2.23528490798758	-1.35300543696239\\
-2.23581052306342	-1.34827249084391\\
-2.23605692848604	-1.3430400385886\\
-2.23596044514103	-1.33719043642344\\
-2.23542936997175	-1.33057018661674\\
-2.23433108371332	-1.32297524720535\\
-2.23247153021665	-1.31412873364528\\
-2.22956168278891	-1.3036461295794\\
-2.22516102360426	-1.29097930733975\\
-2.21857881304302	-1.27532326062727\\
-2.2086944178179	-1.25545449226655\\
-2.19361473557335	-1.22943836414586\\
-2.16998584185419	-1.19407274934646\\
-2.13152846224427	-1.14377472060486\\
-2.06573985695198	-1.06824591845862\\
-1.9461735273105	-0.947485261462767\\
-1.71514507451435	-0.742153089006664\\
-1.25915963680298	-0.385927547026545\\
-0.466470302474344	0.15749204989669\\
0.476432702606736	0.72280409766538\\
1.17825862936543	1.08704498818123\\
1.57979469508172	1.26437318172905\\
1.80048557324685	1.34548951140256\\
1.92812298873437	1.38335485119594\\
2.00711025420764	1.40139640666682\\
2.05911603477277	1.40983029938284\\
2.09519171166257	1.41334200179773\\
2.12132034355964	1.4142135623731\\
};
\addplot [color=mycolor2]
  table[row sep=crcr]{%
2.12132034355964	2.82842712474619\\
2.18582575000323	2.82640660874952\\
2.2459761483502	2.82067996094826\\
2.30316600770785	2.81153894842004\\
2.35833530419819	2.79908319427389\\
2.4121407651283	2.78328504568152\\
2.46505313939319	2.76402339892432\\
2.51741350921551	2.74110157112695\\
2.56946568278775	2.71425682483196\\
2.62137377470791	2.68316571665979\\
2.67322998558975	2.64744786912504\\
2.72505542741334	2.60667010622203\\
2.77679571750944	2.56035271147659\\
2.82831256449113	2.50797963523731\\
2.87937250576692	2.44901464186941\\
2.92963424895255	2.38292549191834\\
2.97863667364875	2.30921809960124\\
3.02579039162557	2.22748193169334\\
3.0703766721051	2.13744642550811\\
3.11155818595542	2.03904567139316\\
3.14840590189045	1.93248505458784\\
3.1799449834646	1.81829951655162\\
3.20521925970046	1.69738980361013\\
3.2233689191928	1.57102231442397\\
3.23371058284854	1.44078163681543\\
3.23580480244767	1.30847305653245\\
3.22949541157146	1.1759834831\\
3.21490900216393	1.04511946165368\\
3.19240995273442	0.917445713323391\\
3.16251366135309	0.794144173493542\\
3.12576385298257	0.675902054710668\\
3.08257536776515	0.562820516214273\\
3.03302813706062	0.454314775018543\\
2.97656426645917	0.348948817128694\\
2.91146921429886	0.244099783305823\\
2.83385680006128	0.135243682019644\\
2.73548220960228	0.0144130439331602\\
2.59871334341581	-0.133168683665202\\
2.38479638365891	-0.336180712113444\\
2.01057175466146	-0.648371525543042\\
1.34210041814469	-1.13791067535239\\
0.375352321166521	-1.75822148716338\\
-0.527448468220416	-2.26262801632552\\
-1.11365847778129	-2.5437869612038\\
-1.4567327767317	-2.68228003389134\\
-1.6691296256022	-2.75259678373195\\
-1.81428800439327	-2.79054335642282\\
-1.92320462744895	-2.81168897715777\\
-2.01132392805712	-2.82302365363059\\
-2.08680248126988	-2.82787464184343\\
-2.15422943553058	-2.82790871121207\\
-2.21634478307291	-2.8239827357917\\
-2.27487422572476	-2.8165264847195\\
-2.33095693947289	-2.80572465930185\\
-2.3853755403757	-2.79160756168836\\
-2.43868469976231	-2.77409782070174\\
-2.49128507004909	-2.75303495626884\\
-2.54346612600745	-2.7281884019911\\
-2.59543032955127	-2.69926454755406\\
-2.64730535029083	-2.66591103001887\\
-2.69914810041563	-2.62772046423944\\
-2.75094277189846	-2.58423541950896\\
-2.80259428647749	-2.53495641420315\\
-2.85391830194335	-2.47935483422118\\
-2.90462903903618	-2.41689283661657\\
-2.95432664759112	-2.34705230341292\\
-3.00248656709011	-2.26937453210355\\
-3.04845424185856	-2.18351130413452\\
-3.0914493816545	-2.08928598065547\\
-3.13058429431929	-1.98676020448289\\
-3.16490008683964	-1.87629788656397\\
-3.19342217916266	-1.75861429449935\\
-3.21523240499265	-1.63479579561046\\
-3.22954956680171	-1.50627700312196\\
-3.23580523135335	-1.37476797309193\\
-3.23369896357047	-1.2421341494958\\
-3.22321880671614	-1.11024302689198\\
-3.20461860896113	-0.980799507713277\\
-3.17835148049538	-0.85519274756288\\
-3.14496439092874	-0.734369602871579\\
-3.10495870881183	-0.618735254776386\\
-3.05861192744495	-0.508062722468019\\
-3.0057323937692	-0.401369642978784\\
-2.94527025004781	-0.296685462718974\\
-2.87460204254734	-0.190562899149911\\
-2.78805571026974	-0.0770301658026245\\
-2.67361421522119	0.0546866392302208\\
-2.50520482475694	0.225105040795738\\
-2.22541867988909	0.474087715816021\\
-1.7209533613293	0.868461990547897\\
-0.880186083660502	1.44500069250654\\
0.110323684998138	2.03883159526995\\
0.859259206598599	2.42737283484743\\
1.30741523004191	2.62511520505649\\
1.57426953542748	2.72304263504847\\
1.74766615441039	2.77435113512036\\
1.87207484559155	2.80265686669549\\
1.96924085114733	2.81832059141065\\
2.0502990590364	2.82612977760851\\
2.12132034355964	2.82842712474619\\
};
\addlegendentry{Сумма Минковского}

\end{axis}

\begin{axis}[%
width=0.798\linewidth,
height=0.597\linewidth,
at={(-0.104\linewidth,-0.066\linewidth)},
scale only axis,
xmin=0,
xmax=1,
ymin=0,
ymax=1,
axis line style={draw=none},
ticks=none,
axis x line*=bottom,
axis y line*=left,
legend style={legend cell align=left, align=left, draw=white!15!black}
]
\end{axis}
\end{tikzpicture}%
        \caption{Сумма двух эллипсоидов.}
\end{figure}

%%%%%%%%%%%%%%%%%%%%%%%%%%%%%%%%%%%%%%%%%%%%%%%%%%%%%%%%%%%%%%%%%%%%%%%%%%%%%%%%
\clearpage
\section{Внешняя оценка суммы эллипсоидов}

\begin{theorem}
        Для суммы Минковского эллипсоидов справедлива следующая внешняя оценка
$$
        \sum\limits_{i=1}^{n} \Varepsilon(q_i,\,Q_i)
        =
        \bigcap\limits_{\| l \| = 1} \Varepsilon(q_+(l),\,Q_+(l)),
$$
где
$$
        \begin{aligned}
q_+(l) &= \sum_{i=1}^{n} q_i,
\\
Q_+(l) &= \sum_{i=1}^n p_i \cdot \sum_{i=1}^{n} \frac{Q_i}{p_i},
\quad
\mbox{где }
p_i = \langle l,\,Q_i l \rangle^{\nicefrac12}.
        \end{aligned}
$$
\end{theorem}

\begin{proof}

Будем доказывать для случая $q_i = 0,$ $i = \overline{1,\,n}$. Случай с произвольными центрами~--- аналогично.

Распишем квадрат опорной функции эллипсоида $\Varepsilon(0,\,Q_+(l))$:
\begin{multline*}
        \rho^2(l\,|\,\Varepsilon(0,\,Q_+(l)))
        =
        \sum_{i=1}^{n}
        \langle
        l,\,Q_il
        \rangle
        +
        \sum_{i < j}
        \left\langle
l,\,\left(
\frac{p_i}{p_j}Q_j + \frac{p_j}{p_i}Q_i
\right)l
        \right\rangle
        \geqslant\\\geqslant
        \left\{
\frac{a+b}{2} \geqslant \sqrt{ab}
        \right\}
        \geqslant
        \sum_{i=1}^{n}\langle l,\,Ql \rangle
        +
        2\sum_{i < j}
        \langle l,\,Q_il \rangle^{\nicefrac12}
        \langle l,\,Q_jl \rangle^{\nicefrac12}
        =\\=
        \left(
\sum_{i=1}^n\langle l,\,Q_il\rangle^{\nicefrac12}
        \right)^2
        =
        \rho^2\left(
l\left|
        \sum_{i=1}^n\Varepsilon(0, Q_i)
\right.
        \right).
\end{multline*}
Таким образом, получили, что для любого $l \neq 0$
$$
        \sum_{i=1}^n \Varepsilon(0,\,Q_i) \subseteq \Varepsilon(0,\,Q_+(l)),
$$
причем, так как равенство опорных функций достигается при
$
        p_i = \langle l,\,Q_i l \rangle^{\nicefrac12},
$
то в направлении $l \neq 0$ эллипсоид $\Varepsilon(0,\,Q_+)$ касается суммы $\sum_{i = 0}^{n} \Varepsilon(0,\,Q_i).$

\end{proof}

\clearpage
\begin{figure}[t]
        \centering
        % This file was created by matlab2tikz.
%
%The latest updates can be retrieved from
%  http://www.mathworks.com/matlabcentral/fileexchange/22022-matlab2tikz-matlab2tikz
%where you can also make suggestions and rate matlab2tikz.
%
\definecolor{mycolor1}{rgb}{0.00000,0.44700,0.74100}%
\definecolor{mycolor2}{rgb}{0.85000,0.32500,0.09800}%
%
\begin{tikzpicture}

\begin{axis}[%
width=0.618\linewidth,
height=0.471\linewidth,
at={(0\linewidth,0\linewidth)},
scale only axis,
xmin=-8,
xmax=8,
xlabel style={font=\color{white!15!black}},
xlabel={$x_1$},
ymin=-5,
ymax=5,
ylabel style={font=\color{white!15!black}},
ylabel={$x_2$},
axis background/.style={fill=white},
axis x line*=bottom,
axis y line*=left,
xmajorgrids,
ymajorgrids,
legend style={at={(0.03,0.97)}, anchor=north west, legend cell align=left, align=left, draw=white!15!black}
]
\addplot [color=mycolor1, forget plot]
  table[row sep=crcr]{%
1.1711787112841	3.34230777773576\\
1.34444389618673	3.33686722689733\\
1.50648318534277	3.32144701348819\\
1.65806242150429	3.29723890999844\\
1.79995419924852	3.26523166869947\\
1.93290799118999	3.22622794766654\\
2.05763032437368	3.18086276334402\\
2.17477223041259	3.12962150018857\\
2.28492171716205	3.07285634836238\\
2.3885995032216	3.01080060567632\\
2.48625667387058	2.94358064140543\\
2.57827324651315	2.87122553999274\\
2.66495688065443	2.79367457178694\\
2.74654114431786	2.71078271324095\\
2.82318287015103	2.62232448808267\\
2.8949582146523	2.52799644473262\\
2.96185708693385	2.42741864049226\\
3.02377565347494	2.3201355847161\\
3.0805066685131	2.20561721602438\\
3.13172744617751	2.08326066796132\\
3.17698540702193	1.95239382914097\\
3.21568133490354	1.81228204209558\\
3.24705081945703	1.66213971753183\\
3.27014489780722	1.50114915910447\\
3.28381172002077	1.32848945666784\\
3.28668221628109	1.14337881243048\\
3.2771642749897	0.945133918403077\\
3.25345178932588	0.733249674250213\\
3.21355684151227	0.507501133889645\\
3.15537469570401	0.268066487797808\\
3.07679115926015	0.015664562009573\\
2.97583883725379	-0.248307401701393\\
2.85090136682411	-0.521659161442773\\
2.70095216778915	-0.801347823435408\\
2.52579793500058	-1.08350783899399\\
2.32628156602984	-1.3635980584876\\
2.10439191407115	-1.63667268179468\\
1.86323617293213	-1.89775167737354\\
1.60685736899554	-2.14223401489281\\
1.33991789027605	-2.3662777508081\\
1.06730585933085	-2.56707514274059\\
0.793739425298697	-2.74297851870391\\
0.523437853608312	-2.89347215049445\\
0.259902596754298	-3.0190207516634\\
0.00581903236219554	-3.1208447467499\\
-0.2369372555121	-3.20067372983419\\
-0.467220354507	-3.2605178574641\\
-0.684493265340969	-3.30248032509504\\
-0.888694234930048	-3.32861897022119\\
-1.0801092196493	-3.34085455680139\\
-1.25926106809304	-3.34091776153004\\
-1.42681950330378	-3.33032515794113\\
-1.58353177449541	-3.31037507736533\\
-1.73017158467123	-3.2821558685242\\
-1.86750300244094	-3.24656096722397\\
-1.99625599202063	-3.20430689704777\\
-2.11711054093165	-3.15595168681459\\
-2.23068687011842	-3.10191219445658\\
-2.33753972875556	-3.04247952074268\\
-2.43815523449964	-2.97783215102365\\
-2.5329490934295	-2.90804674702345\\
-2.6222653208492	-2.83310668006404\\
-2.70637479428206	-2.75290849531896\\
-2.78547311726505	-2.66726655590999\\
-2.85967737154948	-2.5759161597878\\
-2.92902140016315	-2.47851546990414\\
-2.99344930844609	-2.37464666482809\\
-3.05280690958282	-2.26381681735669\\
-3.10683089311493	-2.14545915779927\\
-3.15513558252945	-2.01893559215682\\
-3.19719730262603	-1.88354163878489\\
-3.23233664163192	-1.73851533219976\\
-3.25969932318791	-1.58305212075492\\
-3.27823706849968	-1.41632833369867\\
-3.28669080350863	-1.23753634540448\\
-3.28357990805066	-1.04593497683649\\
-3.26720291066679	-0.840918687724783\\
-3.23565696131938	-0.622108318461704\\
-3.18688516768166	-0.389463978954261\\
-3.11876167995196	-0.143416533926881\\
-3.02922302993614	0.11499241610004\\
-2.91644915653916	0.383980836121862\\
-2.77908754849337	0.660934801326638\\
-2.61649915361415	0.942389492277687\\
-2.42898799999709	1.22411526429499\\
-2.21796402213223	1.50132757866017\\
-1.9859883878281	1.76901289011117\\
-1.7366683581316	2.02232922526914\\
-1.47440268220478	2.25701269019761\\
-1.20401775520347	2.46971278707818\\
-0.930363383198716	2.65819606075749\\
-0.657943143023997	2.82139307813531\\
-0.390637102663735	2.95930311865273\\
-0.131543811712393	3.07279931782965\\
0.117062449314586	3.1633871238911\\
0.353684227850852	3.23296276733506\\
0.577496523037587	3.28360333410188\\
0.788216624026908	3.3174035782658\\
0.985971690059736	3.33636162404406\\
1.1711787112841	3.34230777773576\\
};
\addplot [color=mycolor1, forget plot]
  table[row sep=crcr]{%
2.17430603473841	2.82928793073052\\
2.33044535777815	2.82444897168392\\
2.46594297362775	2.81160678704818\\
2.58415716708103	2.79277030050891\\
2.68789254387122	2.76940595677011\\
2.77946699484521	2.7425714670914\\
2.86078573906653	2.71301902411803\\
2.93341104171975	2.68127265050854\\
2.9986230753596	2.64768508851882\\
3.05747086442469	2.61247907686435\\
3.11081384046105	2.57577692251115\\
3.15935513715665	2.53762134677768\\
3.20366787001532	2.49798979992626\\
3.24421554049886	2.45680381827628\\
3.28136751247959	2.41393452310087\\
3.31541029254594	2.36920500026929\\
3.3465551303234	2.32239002374647\\
3.37494224630124	2.27321336944935\\
3.40064178827692	2.22134278883848\\
3.42365140325324	2.16638255933344\\
3.44389007534897	2.10786339134407\\
3.46118760463784	2.04522934489396\\
3.47526876652081	1.97782129527289\\
3.48573077423729	1.90485640152273\\
3.49201214853071	1.82540300920684\\
3.49335047065297	1.73835053234855\\
3.48872578597209	1.64237424725429\\
3.47678575017332	1.53589584679112\\
3.45574827300754	1.41704249382237\\
3.42327810810487	1.28361072542021\\
3.37633699194203	1.13304806285973\\
3.31101526097114	0.962476172308744\\
3.22237097655495	0.768796474817825\\
3.10433695290387	0.548942043211334\\
2.9498123210435	0.300361978709517\\
2.75112793000477	0.0218247814546395\\
2.50112454582504	-0.285440449235098\\
2.19500744060268	-0.616443644580409\\
1.83279348396173	-0.961489600474761\\
1.42152136958137	-1.3063970008595\\
0.975845648074582	-1.63450693799913\\
0.515964639544959	-1.93017020234628\\
0.0633211215265111	-2.18224450162264\\
-0.363949894265142	-2.38592186002566\\
-0.753796552421497	-2.54230740615597\\
-1.10060884720806	-2.65651208684665\\
-1.40389862561362	-2.73547347665808\\
-1.66643431787696	-2.78630322905355\\
-1.89260401811819	-2.81535915112738\\
-2.08728476369964	-2.82789001873507\\
-2.25518878499693	-2.82801992101541\\
-2.40055085490442	-2.81888822167606\\
-2.52702294235802	-2.80283518923491\\
-2.63767930270744	-2.78157977328078\\
-2.73507199115711	-2.75636932685319\\
-2.82130341222054	-2.72809764441867\\
-2.89809909341037	-2.69739456722559\\
-2.96687326596811	-2.66469244542933\\
-3.02878476984102	-2.63027467163603\\
-3.08478318424195	-2.5943106773748\\
-3.13564609696506	-2.55688082262324\\
-3.18200873955121	-2.51799374438073\\
-3.22438719765583	-2.47759802849779\\
-3.26319624623718	-2.43558952510388\\
-3.29876265010897	-2.39181521440426\\
-3.33133455291616	-2.34607421503113\\
-3.36108736562728	-2.29811628370925\\
-3.38812635908966	-2.24763796026885\\
-3.4124859564092	-2.1942763488638\\
-3.43412549761435	-2.13760038238197\\
-3.45292099547021	-2.0770992851559\\
-3.4686520985225	-2.01216782796055\\
-3.48098310405704	-1.94208786685168\\
-3.48943639793565	-1.86600559716315\\
-3.49335612455861	-1.7829039871982\\
-3.49185921454444	-1.69157008370894\\
-3.4837701826888	-1.59055749156644\\
-3.46753554642006	-1.47814566124369\\
-3.44111377096973	-1.35230025189567\\
-3.40183832761321	-1.21064372066799\\
-3.34625674726662	-1.05045381725873\\
-3.26996105481103	-0.868721532579536\\
-3.16745032441434	-0.662320310385892\\
-3.03211101926641	-0.428362470131523\\
-2.85646733273374	-0.164834179270465\\
-2.63292320715374	0.128425970177643\\
-2.35522203042063	0.448465239159584\\
-2.02065569486654	0.788006644772256\\
-1.63254200363741	1.13497995420302\\
-1.20180828192215	1.47359901929041\\
-0.746311741639337	1.78722909211069\\
-0.287482595077459	2.06211021974623\\
0.15439697822222	2.29018412674959\\
0.56404269287998	2.46977369323667\\
0.932710877487586	2.60426859699022\\
1.2575759170864	2.69994355321456\\
1.54000858568913	2.76398340384577\\
1.78376187044044	2.80319756768445\\
1.99357716381973	2.82340732367622\\
2.17430603473841	2.82928793073052\\
};
\addplot [color=mycolor1, forget plot]
  table[row sep=crcr]{%
6.81839938705513	4.98558555137348\\
6.9451056567848	4.98172857837679\\
7.04503269745632	4.97230404998413\\
7.12550382874078	4.95951351377152\\
7.19150742751882	4.9446702051191\\
7.24652621591938	4.92856445823852\\
7.29304670485787	4.91167081831979\\
7.33288010703728	4.89426840679286\\
7.36736941959477	4.87651242497647\\
7.39752608589151	4.8584773879902\\
7.42412204545158	4.84018356931731\\
7.44775286792895	4.82161320692916\\
7.46888172297768	4.802720283049\\
7.48787036298415	4.78343612628583\\
7.50500109471872	4.76367216612502\\
7.52049232706941	4.74332061039937\\
7.53450938011309	4.72225345843491\\
7.54717163259053	4.70032001509007\\
7.5585566510167	4.67734287814631\\
7.56870160579294	4.65311219540491\\
7.57760198044974	4.62737779738012\\
7.5852072695652	4.59983857492861\\
7.59141298466069	4.57012814868072\\
7.59604777339064	4.53779541139837\\
7.59885369623389	4.5022778262537\\
7.59945651723111	4.46286428877193\\
7.59732094276784	4.41864266581239\\
7.59168253759535	4.36842439739232\\
7.58144256176967	4.31063406338187\\
7.56500233907657	4.243144292049\\
7.5399964148077	4.16302350528493\\
7.50285171749803	4.0661415588715\\
7.44803942928042	3.9465388293111\\
7.36677017908548	3.79539498589725\\
7.24466050473369	3.59931675028097\\
7.0574886748104	3.33749314960869\\
6.76352025145179	2.97714793958209\\
6.29053702849613	2.46732264221061\\
5.51919290384651	1.7351657602429\\
4.28518881238173	0.704044860696217\\
2.47617548846006	-0.62399326502574\\
0.26472488379459	-2.04447711678708\\
-1.85854763216566	-3.22872807539079\\
-3.51179701572037	-4.01952314056496\\
-4.65672460540954	-4.48084729701715\\
-5.42192597149296	-4.73406098849309\\
-5.93815571883776	-4.86916023516302\\
-6.29610684807415	-4.93885917516955\\
-6.55240778900606	-4.97201605639051\\
-6.74175384000817	-4.98434219551015\\
-6.88568139038962	-4.98454045434875\\
-6.99789141017824	-4.97754800336823\\
-7.08734546190676	-4.96623193256517\\
-7.16006824322224	-4.95228978176482\\
-7.22021562299617	-4.93673979314651\\
-7.27072257215752	-4.92019511301591\\
-7.31370624925992	-4.90302121434563\\
-7.3507230387261	-4.88542843600515\\
-7.3829363263936	-4.86752747846997\\
-7.41122838809586	-4.84936318429306\\
-7.43627646256458	-4.8309352520156\\
-7.45860534888868	-4.81221087003346\\
-7.47862427236261	-4.79313219481558\\
-7.49665296586647	-4.77362040406707\\
-7.51294017121452	-4.75357734217863\\
-7.52767665013585	-4.73288533123189\\
-7.54100405805473	-4.71140542676794\\
-7.55302052460794	-4.68897418296435\\
-7.56378340709999	-4.66539881155098\\
-7.57330937097375	-4.64045043926517\\
-7.58157165270455	-4.61385495942758\\
-7.58849402465152	-4.58528069952081\\
-7.5939405470587	-4.55432174128212\\
-7.59769957191291	-4.52047516178699\\
-7.59945951764889	-4.48310960021399\\
-7.59877242863057	-4.44142120888268\\
-7.59499885959066	-4.39437090303387\\
-7.58722344586879	-4.34059333582325\\
-7.5741232741605	-4.27826223117822\\
-7.55375828026879	-4.20488688476054\\
-7.52322939077674	-4.11699767437352\\
-7.4781062033147	-4.00964867067367\\
-7.4114422958755	-3.87561300911386\\
-7.31203520475341	-3.70405586877619\\
-7.16128309678768	-3.47832412658061\\
-6.9274595685231	-3.17231678512378\\
-6.55558188850301	-2.7449831897496\\
-5.95177163129806	-2.13426805552596\\
-4.97054626054604	-1.26027984798751\\
-3.45135195503113	-0.0699980423053237\\
-1.39392514759134	1.34380755585939\\
0.840019649925826	2.68258056595531\\
2.75314581466553	3.67259137928405\\
4.14099226701179	4.28347990938291\\
5.07776315797388	4.62683679623043\\
5.70454533578041	4.81235986820016\\
6.13273164742407	4.90997324701889\\
6.4344036995944	4.95880562065331\\
6.65387188453432	4.98012287766274\\
6.81839938705513	4.98558555137348\\
};
\addplot [color=mycolor1, forget plot]
  table[row sep=crcr]{%
1.42632528730357	3.05419055216814\\
1.59409885697327	3.04894192904819\\
1.74764607607782	3.03434727439337\\
1.88831768585849	3.01189675254528\\
2.01740026430351	2.98279289019926\\
2.13608416558777	2.94798796969752\\
2.24544811665552	2.90822043063386\\
2.34645448710049	2.86404765904994\\
2.43995094101877	2.81587408211711\\
2.52667551525986	2.76397436887494\\
2.60726315074524	2.7085119950182\\
2.68225238936669	2.64955362911531\\
2.75209140658278	2.58707985082535\\
2.81714284020386	2.52099268940941\\
2.8776870472672	2.45112041621127\\
2.93392350905193	2.37721996369354\\
2.98597013463019	2.29897729217642\\
3.03386020403657	2.2160059952028\\
3.07753665694582	2.12784443619997\\
3.11684338412892	2.0339517557063\\
3.15151313143606	1.93370319745457\\
3.18115160055983	1.82638539774865\\
3.20521736059578	1.71119259976722\\
3.22299732316753	1.58722523687895\\
3.23357786670449	1.45349302920453\\
3.23581235158743	1.3089257071312\\
3.2282869316483	1.15239574159971\\
3.20928847715888	0.982758977142026\\
3.176781336107	0.798920617843454\\
3.12840373601196	0.59993508496799\\
3.06149968552777	0.385147822534311\\
2.97320730405123	0.154383494569105\\
2.86062722894035	-0.0918240253086883\\
2.72109091958337	-0.351980889021321\\
2.55253274510375	-0.623406293831111\\
2.3539375047955	-0.902099020098742\\
2.12578916758129	-1.1827809605491\\
1.87040277410249	-1.45918525535049\\
1.59200788692453	-1.72459900904301\\
1.29649516293625	-1.97258135199058\\
0.990837210384203	-2.197696081303\\
0.682309937938043	-2.39607365215453\\
0.377709629271056	-2.56567186636937\\
0.0827467276445912	-2.70621088488075\\
-0.198285197101545	-2.81885870156143\\
-0.462566309044838	-2.90579312858928\\
-0.70860369444456	-2.96975905169869\\
-0.935974346036652	-3.01369849093327\\
-1.14503867672096	-3.04048466100225\\
-1.33667724745256	-3.05275737054036\\
-1.51207595388432	-3.05283983637206\\
-1.67256535682076	-3.04271255194964\\
-1.81950894294363	-3.02402253127754\\
-1.95423070636569	-2.99811154608017\\
-2.07797191745059	-2.96605230356283\\
-2.19186825631256	-2.92868580072754\\
-2.29694038051205	-2.8866561327032\\
-2.3940928301155	-2.84044100242064\\
-2.48411769306366	-2.7903773569584\\
-2.56770060761846	-2.73668222109411\\
-2.64542750422993	-2.67946910987757\\
-2.71779105258215	-2.61876051726236\\
-2.7851961466965	-2.55449698645495\\
-2.84796398665464	-2.48654322508971\\
-2.9063344415271	-2.4146916680026\\
-2.96046643468488	-2.33866383248366\\
-3.01043610126227	-2.25810976873771\\
-3.05623244371903	-2.1726058926298\\
-3.09775016794598	-2.08165151011889\\
-3.13477933218787	-1.98466441800741\\
-3.16699140132554	-1.88097611459704\\
-3.19392129598201	-1.76982740563066\\
-3.21494510178343	-1.65036558457997\\
-3.22925332680441	-1.52164495079917\\
-3.23582007028983	-1.38263325815509\\
-3.2333693489984	-1.23222780416891\\
-3.22034133184322	-1.06928627425662\\
-3.19486361413493	-0.892679032711539\\
-3.15473614964176	-0.701370955594937\\
-3.09744308405396	-0.494541380754748\\
-3.02020998960036	-0.271748964553961\\
-2.92012930110728	-0.0331421889446313\\
-2.79437683989129	0.220296299061972\\
-2.64053313743929	0.486503595447298\\
-2.45699943943893	0.762146269718005\\
-2.24345818717511	1.04256505003492\\
-2.00128002875728	1.3219347994068\\
-1.73374676631364	1.59368386301881\\
-1.44597222977026	1.85114025281027\\
-1.14447704690877	2.08828044420864\\
-0.836487911591588	2.30039820704417\\
-0.529131242108844	2.48452621841593\\
-0.228720064349343	2.63952985558765\\
0.0597204337400604	2.76590334036294\\
0.332641313986094	2.86537701210694\\
0.58791360706013	2.94046425247022\\
0.824614911507097	2.99404850797217\\
1.0427491736426	3.02906396360623\\
1.24296707789219	3.04828205038319\\
1.42632528730357	3.05419055216814\\
};
\addplot [color=mycolor1, forget plot]
  table[row sep=crcr]{%
0.878355311988031	4.00104451440901\\
1.06350692625045	3.99520620111796\\
1.24105208545559	3.97828669634494\\
1.41132245401614	3.95107065481952\\
1.57466773303343	3.91420199851153\\
1.73143696912751	3.86819041177193\\
1.88196376005733	3.81341874894101\\
2.02655438296571	3.750150547048\\
2.16547798030495	3.67853714869157\\
2.29895805486653	3.59862418003358\\
2.42716463522509	3.51035731500992\\
2.55020657118865	3.41358740789436\\
2.668123506243	3.30807520909464\\
2.78087715546415	3.19349600831664\\
2.88834160146377	3.06944468768286\\
2.99029241959652	2.93544182533574\\
3.08639457260405	2.79094167459111\\
3.17618919386929	2.63534305689275\\
3.25907963118477	2.46800444241896\\
3.33431747555457	2.28826473051858\\
3.4009897767428	2.09547144282054\\
3.45800926415817	1.88901813486057\\
3.50411013946209	1.6683927090189\\
3.53785283151403	1.43323782151323\\
3.55764187603864	1.18342353411735\\
3.5617615690935	0.919130581462076\\
3.54843389392288	0.64093999291339\\
3.51590199238786	0.349921399335943\\
3.46253971391698	0.0477085835925056\\
3.38698331236318	-0.263452424287396\\
3.28827546184747	-0.580698823062855\\
3.16600547552711	-0.900604275852195\\
3.02042477853998	-1.21929774783176\\
2.85251553485022	-1.53264516950835\\
2.66399457594665	-1.83647819729149\\
2.45724457303916	-2.12684262395052\\
2.23517770285111	-2.4002325211065\\
2.00105011016354	-2.65377765196434\\
1.7582542402097	-2.88536099227669\\
1.51011809336617	-3.09365738536341\\
1.25973568625162	-3.27809894883646\\
1.00984379818778	-3.43878382182336\\
0.762749729660402	-3.57635010468598\\
0.520306129482393	-3.69183655722441\\
0.283923478606458	-3.78654748590043\\
0.0546086990229475	-3.86193343438586\\
-0.166981171541117	-3.91949359001469\\
-0.380479311033542	-3.96070128664519\\
-0.585749985503425	-3.986950972031\\
-0.782834069573656	-3.99952339235443\\
-0.971901342639129	-3.99956518582877\\
-1.15321025667837	-3.98807919312425\\
-1.32707494440725	-3.96592226831757\\
-1.49383870905093	-3.93380798674879\\
-1.65385299334717	-3.89231225949612\\
-1.80746076448412	-3.84188040852438\\
-1.95498329637778	-3.78283470631474\\
-2.09670942756051	-3.71538173904544\\
-2.23288648811713	-3.63961922624244\\
-2.36371220332813	-3.55554214043776\\
-2.48932698599055	-3.46304813674846\\
-2.60980612184367	-3.36194244219558\\
-2.7251514361164	-3.25194248386318\\
-2.83528211072562	-3.1326826674037\\
-2.9400244113109	-3.00371986447725\\
-3.03910019488404	-2.86454033832207\\
-3.13211422011443	-2.71456903567785\\
-3.21854049496589	-2.55318239915177\\
-3.29770819576308	-2.37972609464072\\
-3.3687881039304	-2.19353927459823\\
-3.43078105286486	-1.99398715573653\\
-3.4825105630904	-1.78050369093681\\
-3.52262264058648	-1.55264583001233\\
-3.54959653315705	-1.31016012225037\\
-3.56177090540339	-1.05306102369323\\
-3.55739011243071	-0.781718072012295\\
-3.53467462163135	-0.496946044577514\\
-3.49191769492277	-0.200088546569051\\
-3.42760684601089	0.106918179217524\\
-3.34056332736365	0.421516953800008\\
-3.23008662410292	0.74055253109619\\
-3.09608510202537	1.06035723112041\\
-2.93917069281702	1.37690232801292\\
-2.76069690826706	1.68600603833541\\
-2.56272661017962	1.98357610732913\\
-2.34792790075655	2.26585543441595\\
-2.11941020973985	2.52963648855715\\
-1.88052401217249	2.77241584827913\\
-1.63465314874106	2.9924724548048\\
-1.38502715711598	3.18886813413911\\
-1.13457364935907	3.3613821295424\\
-0.885820579981077	3.51039961632549\\
-0.640848456119384	3.63677651896027\\
-0.40128537454523	3.7417004733795\\
-0.16833402832988	3.82656253933673\\
0.0571808702239901	3.89284832757713\\
0.274754939025249	3.94205200602846\\
0.48414407094491	3.975612869128\\
0.685307253194799	3.99487187487964\\
0.87835531198803	4.00104451440901\\
};
\addplot [color=mycolor1, forget plot]
  table[row sep=crcr]{%
1.02375040531617	3.61290204042887\\
1.20176097639549	3.60730045239441\\
1.37036957674768	3.59124369131131\\
1.53008538748134	3.56572536946192\\
1.68143780878129	3.53157387649472\\
1.82495211979655	3.48946246279246\\
1.96113166891982	3.43992080417065\\
2.09044497609687	3.38334668835468\\
2.2133163581027	3.32001698205598\\
2.33011892490929	3.25009741061438\\
2.44116901027864	3.17365094361483\\
2.54672128027864	3.09064476183925\\
2.64696390782588	3.0009559112872\\
2.74201331386758	2.90437585191723\\
2.83190806411647	2.8006142012692\\
2.91660158441303	2.68930207223581\\
2.99595343022608	2.56999552386096\\
3.06971893197026	2.44217979639812\\
3.13753715736318	2.30527519718943\\
3.19891731008915	2.15864574879307\\
3.25322395208505	2.00161200452337\\
3.29966183225684	1.83346976493728\\
3.33726166727844	1.65351675400746\\
3.36486898352328	1.46108956040721\\
3.38113910113713	1.25561318766688\\
3.38454247282657	1.03666518752547\\
3.3733857285037	0.804055302209506\\
3.34585461031475	0.55791950056818\\
3.30008499547509	0.29882399643962\\
3.23426669333464	0.027870249673398\\
3.14678092088188	-0.253213474226809\\
3.03636585661394	-0.542013456154021\\
2.90229582008168	-0.835426867441589\\
2.74455015014256	-1.12973284012124\\
2.56394095281318	-1.42074875525535\\
2.36216843685569	-1.70406624141488\\
2.14178131502429	-1.97534198104143\\
1.90603727640003	-2.23060186778484\\
1.65868024595527	-2.46651004688063\\
1.40366973892964	-2.68056058438075\\
1.14490648466575	-2.87116725842797\\
0.885995072210844	-3.03764980487017\\
0.63007116427807	-3.18013500544318\\
0.379703604907356	-3.29940273110009\\
0.136866392078723	-3.3967091096898\\
-0.0970344061660403	-3.47361363503207\\
-0.3210974496125	-3.53182808856116\\
-0.534844239226391	-3.57309603977806\\
-0.738132419461264	-3.5991045027063\\
-0.931073626071126	-3.61142473381268\\
-1.11396114323217	-3.61147691260704\\
-1.28720951056091	-3.60051289518032\\
-1.45130614583367	-3.57961168548513\\
-1.60677385886146	-3.54968318739327\\
-1.75414258331887	-3.51147682884627\\
-1.89392851680262	-3.46559259474307\\
-2.02661895578985	-3.41249278629328\\
-2.15266131796345	-3.35251342660259\\
-2.27245508285067	-3.28587467397293\\
-2.38634561026136	-3.21268991747198\\
-2.49461899453625	-3.13297344723352\\
-2.59749727486108	-3.04664674493076\\
-2.69513344957367	-2.95354355339843\\
-2.78760584167458	-2.85341397942644\\
-2.87491144258746	-2.74592797770453\\
-2.95695793279452	-2.63067867162488\\
-3.0335541551112	-2.5071861012065\\
-3.10439891645093	-2.37490216114041\\
-3.16906813919749	-2.23321771157066\\
-3.22700060200421	-2.0814731142732\\
-3.2774828366435	-1.91897376091784\\
-3.31963422182456	-1.74501249358455\\
-3.35239397487088	-1.55890111606585\\
-3.37451261146532	-1.36001335729562\\
-3.38455150677088	-1.14784151232587\\
-3.38089535402736	-0.922068315058348\\
-3.36178335378977	-0.682654091339456\\
-3.32536546776439	-0.429936596356688\\
-3.26978941824833	-0.164736978685911\\
-3.1933215530012	0.111539791837043\\
-3.09449957081791	0.396827137223658\\
-2.97230730339444	0.688362960794408\\
-2.82635227231545	0.982721291257162\\
-2.65701807848947	1.27592551447855\\
-2.46555953737673	1.5636464007621\\
-2.25411243852087	1.84146980592806\\
-2.02560326847436	2.10520006333872\\
-1.78356466168654	2.35115257139234\\
-1.53188345629107	2.57638853279399\\
-1.27452259365677	2.77885734455166\\
-1.01526081426038	2.95743342231848\\
-0.757485202333998	3.11185660361474\\
-0.504055616196481	3.24260160764969\\
-0.257243124584251	3.35070882318175\\
-0.0187316848425002	3.43760661534103\\
0.210334443627698	3.50494770426823\\
0.429276150050858	3.55447278851852\\
0.637793727556865	3.58790627445286\\
0.835881553665213	3.6068830687125\\
1.02375040531617	3.61290204042887\\
};
\addplot [color=mycolor1, forget plot]
  table[row sep=crcr]{%
1.65642263207936	2.91541373390173\\
1.82037784583777	2.91030086390922\\
1.9677141607887	2.89631048846228\\
2.10039858637834	2.87514661502634\\
2.22021829502882	2.84814149164205\\
2.32876139289674	2.81631933318551\\
2.42741778412894	2.78045311346391\\
2.51739080453339	2.74111237919629\\
2.5997137273383	2.69870202037562\\
2.67526759617419	2.65349283915425\\
2.74479836712207	2.60564506995236\\
2.80893228619176	2.55522600914459\\
2.86818898383772	2.50222278100859\\
2.92299207190501	2.44655108756695\\
2.97367717316179	2.38806060833533\\
3.02049735848801	2.32653755200115\\
3.06362594816039	2.26170472357171\\
3.1031565727605	2.19321936030627\\
3.13910029784126	2.12066890980134\\
3.17137950106921	2.04356487823619\\
3.19981805563009	1.96133487528602\\
3.22412722545868	1.87331304194841\\
3.24388652917448	1.77872919811096\\
3.25851870767061	1.67669733555935\\
3.26725788924465	1.56620458170098\\
3.26911018723029	1.44610257455019\\
3.26280646739801	1.31510446242669\\
3.24674818526298	1.17179264275761\\
3.21894948375011	1.01464503971421\\
3.17698281835873	0.842091215225841\\
3.11794201723986	0.652613547633569\\
3.03844644930004	0.444911865172638\\
2.93472234736788	0.218149481119256\\
2.8028090465519	-0.0277104952854745\\
2.63894062157472	-0.291494298734734\\
2.44013251083137	-0.570393290038137\\
2.20494150359352	-0.859649139467996\\
1.93426251361171	-1.1525223052927\\
1.63191099575297	-1.44070778788785\\
1.30469689836056	-1.71524417896233\\
0.961809545903872	-1.96775111774133\\
0.613598070559529	-2.19163828849998\\
0.270106034005761	-2.3829008079898\\
-0.0601842625425428	-2.54029310737039\\
-0.371081514574716	-2.66493929156151\\
-0.658830666399524	-2.75962282515189\\
-0.921821623365125	-2.82802513033402\\
-1.16008213079777	-2.87409551910905\\
-1.37473982056242	-2.90162188662688\\
-1.56756118680114	-2.91399083853101\\
-1.740606418096	-2.91408981741843\\
-1.89599662931821	-2.90429937494905\\
-2.03577159460937	-2.88653403208306\\
-2.16181254714747	-2.86230368386329\\
-2.27580800711821	-2.83277894819268\\
-2.37924602595903	-2.79885186349478\\
-2.47342135895941	-2.76118829145959\\
-2.55945010266752	-2.72027114215685\\
-2.63828719565115	-2.67643490915497\\
-2.71074408814828	-2.6298925648727\\
-2.77750509407517	-2.58075599703199\\
-2.83914166821686	-2.52905108890512\\
-2.89612426683683	-2.47472838286705\\
-2.94883166501475	-2.41767008273216\\
-2.99755769310528	-2.35769397601212\\
-3.04251536440935	-2.2945547054575\\
-3.08383832442057	-2.22794269467252\\
-3.12157947480033	-2.15748093717595\\
-3.15570652122286	-2.08271979484268\\
-3.18609406848149	-2.00312992664664\\
-3.21251174361749	-1.91809349492053\\
-3.23460767710368	-1.82689389684362\\
-3.25188653209408	-1.72870448128615\\
-3.26368118112669	-1.62257709444266\\
-3.26911716301003	-1.50743193969673\\
-3.2670693455275	-1.38205126127052\\
-3.2561110053532	-1.24508092609879\\
-3.23445719275566	-1.09504625615184\\
-3.19990733869133	-0.930391566459728\\
-3.14979732421527	-0.749556666920957\\
-3.08097940364488	-0.5511073661317\\
-2.98985962964534	-0.333938851596169\\
-2.87253514113668	-0.0975668422736795\\
-2.72508233437513	0.157494529622942\\
-2.54404001527801	0.429303881850831\\
-2.32709234919699	0.714103779021443\\
-2.07387200165371	1.00612432375832\\
-1.78668666913353	1.29775933191409\\
-1.47088122335579	1.58023484821087\\
-1.1345756241589	1.84471730760219\\
-0.787717671493854	2.083588282538\\
-0.440683840723205	2.2914878559795\\
-0.102865668160095	2.46580854778831\\
0.218346993157528	2.60656514302512\\
0.517995089490965	2.71580904103266\\
0.793451069451776	2.79686253352453\\
1.04399255122457	2.85360765026289\\
1.27026126265922	2.88995393049057\\
1.47375493321376	2.90950846059762\\
1.65642263207936	2.91541373390173\\
};
\addplot [color=mycolor1, forget plot]
  table[row sep=crcr]{%
5.18712583895256	4.03643886090506\\
5.31525725204707	4.03252906981332\\
5.41757620024171	4.02287294925828\\
5.50079469346955	4.00964175117464\\
5.5696003253642	3.99416555895309\\
5.62733170440595	3.97726378963646\\
5.67641122072661	3.95943940624426\\
5.71862685754273	3.94099514264964\\
5.75531895681641	3.92210424785202\\
5.78750618287741	3.90285415339594\\
5.81597190070615	3.88327366803006\\
5.84132430607785	3.86334991685115\\
5.8640388204713	3.84303872581938\\
5.88448826356221	3.82227067715433\\
5.90296441651425	3.80095417392495\\
5.91969336093515	3.77897630111696\\
5.9348461628453	3.75620191182305\\
5.94854590917154	3.73247111600344\\
5.96087169458886	3.70759515234812\\
5.97185983006285	3.68135044432882\\
5.98150224897495	3.65347044950186\\
5.98974177522711	3.6236346761894\\
5.99646353807814	3.59145392663468\\
6.00148130126427	3.55645037850838\\
6.00451671496219	3.51803045859987\\
6.00516833049201	3.4754474699895\\
6.00286535774335	3.42774940781586\\
5.99679810904926	3.37370501352264\\
5.98581200843584	3.31169733754317\\
5.96824343302717	3.23956802578995\\
5.94166076599006	3.1543857977506\\
5.90244799924475	3.05209698328066\\
5.84512255933808	2.92699173597258\\
5.76120034149497	2.77088497318124\\
5.63729389255025	2.57187402037809\\
5.45196389052451	2.31254430268633\\
5.170800179951	1.96774488397845\\
4.74005681044081	1.50317000806675\\
4.08343851340225	0.879405714822638\\
3.11873275843801	0.0725212336633277\\
1.82240158153292	-0.879979006481791\\
0.3208772432074	-1.84474570715855\\
-1.13257542859024	-2.65499655952802\\
-2.33491184292571	-3.22951294326548\\
-3.23551674549088	-3.5919659263711\\
-3.8819331864792	-3.80563197939625\\
-4.34333653284971	-3.92625706921822\\
-4.67714451079244	-3.99118894206398\\
-4.92384191511931	-4.02306768443975\\
-5.1104782906461	-4.03519717120396\\
-5.25494229682946	-4.03538413725133\\
-5.36916620850968	-4.02825870448253\\
-5.46124197072875	-4.01660612024233\\
-5.53676482188745	-4.00212386009999\\
-5.59968129721263	-3.98585566318484\\
-5.65282887921241	-3.96844430635513\\
-5.69828447180154	-3.95028150863634\\
-5.73759331009454	-3.93159842884888\\
-5.77192201193149	-3.91252116361816\\
-5.80216268002389	-3.89310518437901\\
-5.82900484455859	-3.8733568160797\\
-5.8529858815526	-3.85324654641236\\
-5.87452674587498	-3.83271703316872\\
-5.89395747634478	-3.81168753720132\\
-5.91153540716701	-3.79005581256549\\
-5.92745802288954	-3.7676980440536\\
-5.94187172094541	-3.74446712558521\\
-5.95487727028482	-3.72018935421026\\
-5.9665323937902	-3.69465943029842\\
-5.97685159715728	-3.66763347237637\\
-5.98580306939653	-3.63881954565161\\
-5.99330214181357	-3.60786493364055\\
-5.99920035424942	-3.57433900902988\\
-6.00326855541121	-3.53771001937137\\
-6.00517152686637	-3.49731329719506\\
-6.00443015115258	-3.45230717604644\\
-6.00036477491604	-3.40161098959502\\
-5.99200950644002	-3.34381653301228\\
-5.9779805973346	-3.27705958700445\\
-5.95627075586493	-3.19883042356599\\
-5.92392157032137	-3.10568985900584\\
-5.87649171664733	-2.99283785316287\\
-5.80717843048081	-2.85345220038719\\
-5.70534840422258	-2.67767671923539\\
-5.5540823371721	-2.45111316334733\\
-5.32619170660676	-2.15275947247519\\
-4.97839825424011	-1.75289683024479\\
-4.44544133267055	-1.21346613054153\\
-3.64360737117788	-0.498615041134729\\
-2.50836858312705	0.391708787828822\\
-1.08269748007528	1.37202389763564\\
0.427899320710524	2.27720604166756\\
1.77169676503285	2.97202628862457\\
2.82116844859093	3.43344452188135\\
3.58608777976602	3.71348668836244\\
4.131761062643	3.87482760873836\\
4.52329054454194	3.96399345419352\\
4.80943067044692	4.01026330629437\\
5.02337853892604	4.03101761132268\\
5.18712583895256	4.03643886090506\\
};
\addplot [color=mycolor1, forget plot]
  table[row sep=crcr]{%
2.11655100682591	2.82843428941261\\
2.27354611471052	2.82356578471402\\
2.41025665966359	2.81060627378123\\
2.52989546066145	2.79154091646479\\
2.63516721135814	2.76782903818278\\
2.72832261743308	2.74053005988702\\
2.81122263331923	2.71040195994312\\
2.88540101761024	2.67797585950846\\
2.95212010046617	2.64361138065595\\
3.01241816078134	2.60753712579265\\
3.06714850265728	2.56987987342384\\
3.11701103132388	2.530685279376\\
3.16257733553198	2.48993216479353\\
3.20431024779221	2.44754189978502\\
3.2425787120241	2.40338394581532\\
3.27766860712889	2.35727827727643\\
3.30978998395011	2.30899513789073\\
3.33908098109372	2.25825237864351\\
3.36560849027125	2.20471045325069\\
3.38936543605966	2.14796500184279\\
3.41026430598958	2.08753682547208\\
3.42812629933144	2.02285894077584\\
3.44266513926722	1.95326031123863\\
3.4534641954334	1.87794579759002\\
3.4599450789137	1.79597189595953\\
3.46132530302978	1.70621801868249\\
3.45656199668418	1.60735356718522\\
3.44427815481841	1.49780211181684\\
3.4226678564112	1.37570607723411\\
3.38937801254489	1.23889915632012\\
3.34136798234704	1.08490031045564\\
3.27475744922555	0.910953996822627\\
3.18469151120958	0.71415720554415\\
3.06528542191618	0.491733786396986\\
2.90976311744834	0.241532573386507\\
2.71096460636707	-0.0371839138929645\\
2.4624261097362	-0.342670722307781\\
2.16014017496843	-0.6695545814815\\
1.80476531683354	-1.00810768327461\\
1.40349383662699	-1.34464591286619\\
0.97038036758317	-1.66351810247334\\
0.524318870350239	-1.95029936328905\\
0.085159888355619	-2.19486019379004\\
-0.330303447862874	-2.39290064579353\\
-0.710776629391997	-2.54551613559592\\
-1.05080767266494	-2.65747804245967\\
-1.34965578360068	-2.7352745046498\\
-1.60964878916396	-2.78560491585113\\
-1.83470724848301	-2.81451245994852\\
-2.02929746385226	-2.82703310171971\\
-2.19780572461383	-2.82716004636509\\
-2.34422376368673	-2.81795933766786\\
-2.47202961915923	-2.80173490909645\\
-2.58417664940125	-2.78019148732509\\
-2.68313477180989	-2.75457447721281\\
-2.77095171368762	-2.72578186937574\\
-2.84931736926041	-2.69445020124993\\
-2.91962332499729	-2.66101894855914\\
-2.98301449583141	-2.625777943375\\
-3.0404322803138	-2.5889018140239\\
-3.0926497659045	-2.55047463225179\\
-3.14029993083631	-2.51050718663972\\
-3.18389785141002	-2.46894866039765\\
-3.22385782311156	-2.42569398502503\\
-3.26050613704066	-2.38058775033286\\
-3.29409006498849	-2.3334252506136\\
-3.32478341468164	-2.28395101236323\\
-3.35268882400787	-2.23185496103844\\
-3.37783676405674	-2.17676622764621\\
-3.4001810051745	-2.11824445995762\\
-3.41959005405378	-2.05576838234722\\
-3.43583377655189	-1.98872124382334\\
-3.44856406235342	-1.91637271697959\\
-3.45728794729811	-1.83785678961846\\
-3.46133108014034	-1.75214528356065\\
-3.45978882279539	-1.65801695123747\\
-3.45146169164846	-1.55402283821256\\
-3.43477150758744	-1.43845010349195\\
-3.40765503359	-1.3092893324724\\
-3.36743408852029	-1.16421545119853\\
-3.31066711786081	-1.00060088864916\\
-3.23300037864963	-0.815592941179428\\
-3.12906223935289	-0.606305635533384\\
-2.99248669002533	-0.370196004944904\\
-2.81621104383475	-0.105701784839631\\
-2.59324586139854	0.186819618571972\\
-2.31809609238649	0.503941380582688\\
-1.98880682986173	0.838150307963187\\
-1.60913736983443	1.17759478204281\\
-1.18981009540952	1.5072611728684\\
-0.747698446788933	1.81168182170006\\
-0.30271621218734	2.07826646569063\\
0.126382115899771	2.29973691404187\\
0.525382985478411	2.47465037794311\\
0.885985247779934	2.60619273420778\\
1.20528688360715	2.70022002649834\\
1.48428693993919	2.7634737488266\\
1.72626932379742	2.80239670394494\\
1.93552996856662	2.8225480577959\\
2.11655100682591	2.82843428941261\\
};
\addplot [color=mycolor1]
  table[row sep=crcr]{%
0.948683298050514	3.79473319220206\\
1.12996891013208	3.78902241071535\\
1.3027838103604	3.77255915923418\\
1.46753552344581	3.74623053454947\\
1.62465224482371	3.71077288416293\\
1.77456152903207	3.66677970030574\\
1.91767403561802	3.61471073626391\\
2.05437110063153	3.55490128334521\\
2.18499505421831	3.4875709448019\\
2.30984136685525	3.41283153872394\\
2.42915185734506	3.33069397942336\\
2.54310832592638	3.24107414976038\\
2.65182608463458	3.14379790634541\\
2.75534694820212	3.03860547381077\\
2.8536313296431	2.92515559896292\\
2.9465491655584	2.80302996392973\\
3.0338694906698	2.67173851056894\\
3.11524860652315	2.53072651479627\\
3.19021696761039	2.3793844730251\\
3.25816516606007	2.21706211963439\\
3.31832976399147	2.0430881668871\\
3.36978023105031	1.85679760784185\\
3.41140891541445	1.65756857671014\\
3.44192680709927	1.44487070344488\\
3.45986879009691	1.21832646083672\\
3.46361298678702	0.977785965288651\\
3.45141941180798	0.723413822791541\\
3.4214930614047	0.455783738589644\\
3.37207522993717	0.175972765268488\\
3.30156372141992	-0.114357323644244\\
3.20865740947072	-0.412907981729085\\
3.0925136091427	-0.716739496073036\\
2.95289924526556	-1.02233055985344\\
2.79031115637498	-1.3257107709346\\
2.60603988731933	-1.6226624623078\\
2.40215712665582	-1.90897292279738\\
2.18141953804948	-2.18070442957845\\
1.94709823579504	-2.43444260737754\\
1.70275852491129	-2.66748659951055\\
1.45202370644265	-2.877956852001\\
1.19835703486066	-3.0648138241024\\
0.944888096380122	-3.22779787311511\\
0.694297500839398	-3.36731218002033\\
0.448761194715288	-3.48427490957016\\
0.209946190700892	-3.57996468520442\\
-0.0209555786883461	-3.65587741213202\\
-0.243169602253506	-3.71360522906444\\
-0.456269901983927	-3.7547419037936\\
-0.660108423391693	-3.78081428495334\\
-0.854748940791145	-3.79323662458319\\
-1.04040889805107	-3.79328335955738\\
-1.21741050351997	-3.78207578896275\\
-1.38614100991934	-3.76057854694004\\
-1.54702130890241	-3.72960250007616\\
-1.70048159286015	-3.6898114747465\\
-1.84694273429099	-3.64173092446581\\
-1.98680208954437	-3.58575723170233\\
-2.1204225680783	-3.52216679568137\\
-2.24812396922559	-3.45112440165093\\
-2.37017574651009	-3.37269062063008\\
-2.48679050050333	-3.2868281760587\\
-2.59811762050387	-3.19340735751618\\
-2.70423659456836	-3.09221068144532\\
-2.80514959232286	-2.98293711145668\\
-2.90077300437504	-2.86520627071399\\
-2.99092770791458	-2.73856321838213\\
-3.07532793559953	-2.60248453106578\\
-3.1535687736533	-2.45638663503084\\
-3.22511252958097	-2.29963757630341\\
-3.28927451889835	-2.13157368366421\\
-3.34520925490937	-1.95152284702656\\
-3.39189861348128	-1.7588363466938\\
-3.42814429814894	-1.5529312348605\\
-3.45256782598978	-1.33334504716437\\
-3.46362220116165	-1.0998039137315\\
-3.45962024522194	-0.852303713618117\\
-3.43878487541956	-0.591201557879526\\
-3.3993259814164	-0.317311504170883\\
-3.3395463824176	-0.0319942042762583\\
-3.25797518618734	0.262774139257094\\
-3.15352067565559	0.564371272065384\\
-3.02562738470585	0.869554764908886\\
-2.87441511977517	1.17455760801215\\
-2.70077402394131	1.47525676358082\\
-2.50639202484484	1.767403513355\\
-2.29370038309167	2.04688929885519\\
-2.06573813670639	2.31000991273335\\
-1.8259528645897	2.5536887433386\\
-1.57796799985389	2.77562773795799\\
-1.32535178123783	2.9743703475303\\
-1.07141885643386	3.14927865398809\\
-0.819084909373598	3.30044159903781\\
-0.570781691446472	3.42853925898355\\
-0.328428492520302	3.53468897189284\\
-0.093448740680559	3.62029468577783\\
0.13318251344032	3.68691393725669\\
0.350873927242831	3.73614982594868\\
0.55934713845833	3.76956970364108\\
0.758567824726602	3.78864856747597\\
0.948683298050513	3.79473319220206\\
};
\addlegendentry{Аппроксимации}

\addplot [color=mycolor2]
  table[row sep=crcr]{%
2.12132034355964	2.82842712474619\\
2.18582575000323	2.82640660874952\\
2.2459761483502	2.82067996094826\\
2.30316600770785	2.81153894842004\\
2.35833530419819	2.79908319427389\\
2.4121407651283	2.78328504568152\\
2.46505313939319	2.76402339892432\\
2.51741350921551	2.74110157112695\\
2.56946568278775	2.71425682483196\\
2.62137377470791	2.68316571665979\\
2.67322998558975	2.64744786912504\\
2.72505542741334	2.60667010622203\\
2.77679571750944	2.56035271147659\\
2.82831256449113	2.50797963523731\\
2.87937250576692	2.44901464186941\\
2.92963424895255	2.38292549191834\\
2.97863667364875	2.30921809960124\\
3.02579039162557	2.22748193169334\\
3.0703766721051	2.13744642550811\\
3.11155818595542	2.03904567139316\\
3.14840590189045	1.93248505458784\\
3.1799449834646	1.81829951655162\\
3.20521925970046	1.69738980361013\\
3.2233689191928	1.57102231442397\\
3.23371058284854	1.44078163681543\\
3.23580480244767	1.30847305653245\\
3.22949541157146	1.1759834831\\
3.21490900216393	1.04511946165368\\
3.19240995273442	0.917445713323391\\
3.16251366135309	0.794144173493542\\
3.12576385298257	0.675902054710668\\
3.08257536776515	0.562820516214273\\
3.03302813706062	0.454314775018543\\
2.97656426645917	0.348948817128694\\
2.91146921429886	0.244099783305823\\
2.83385680006128	0.135243682019644\\
2.73548220960228	0.0144130439331602\\
2.59871334341581	-0.133168683665202\\
2.38479638365891	-0.336180712113444\\
2.01057175466146	-0.648371525543042\\
1.34210041814469	-1.13791067535239\\
0.375352321166521	-1.75822148716338\\
-0.527448468220416	-2.26262801632552\\
-1.11365847778129	-2.5437869612038\\
-1.4567327767317	-2.68228003389134\\
-1.6691296256022	-2.75259678373195\\
-1.81428800439327	-2.79054335642282\\
-1.92320462744895	-2.81168897715777\\
-2.01132392805712	-2.82302365363059\\
-2.08680248126988	-2.82787464184343\\
-2.15422943553058	-2.82790871121207\\
-2.21634478307291	-2.8239827357917\\
-2.27487422572476	-2.8165264847195\\
-2.33095693947289	-2.80572465930185\\
-2.3853755403757	-2.79160756168836\\
-2.43868469976231	-2.77409782070174\\
-2.49128507004909	-2.75303495626884\\
-2.54346612600745	-2.7281884019911\\
-2.59543032955127	-2.69926454755406\\
-2.64730535029083	-2.66591103001887\\
-2.69914810041563	-2.62772046423944\\
-2.75094277189846	-2.58423541950896\\
-2.80259428647749	-2.53495641420315\\
-2.85391830194335	-2.47935483422118\\
-2.90462903903618	-2.41689283661657\\
-2.95432664759112	-2.34705230341292\\
-3.00248656709011	-2.26937453210355\\
-3.04845424185856	-2.18351130413452\\
-3.0914493816545	-2.08928598065547\\
-3.13058429431929	-1.98676020448289\\
-3.16490008683964	-1.87629788656397\\
-3.19342217916266	-1.75861429449935\\
-3.21523240499265	-1.63479579561046\\
-3.22954956680171	-1.50627700312196\\
-3.23580523135335	-1.37476797309193\\
-3.23369896357047	-1.2421341494958\\
-3.22321880671614	-1.11024302689198\\
-3.20461860896113	-0.980799507713277\\
-3.17835148049538	-0.85519274756288\\
-3.14496439092874	-0.734369602871579\\
-3.10495870881183	-0.618735254776386\\
-3.05861192744495	-0.508062722468019\\
-3.0057323937692	-0.401369642978784\\
-2.94527025004781	-0.296685462718974\\
-2.87460204254734	-0.190562899149911\\
-2.78805571026974	-0.0770301658026245\\
-2.67361421522119	0.0546866392302208\\
-2.50520482475694	0.225105040795738\\
-2.22541867988909	0.474087715816021\\
-1.7209533613293	0.868461990547897\\
-0.880186083660502	1.44500069250654\\
0.110323684998138	2.03883159526995\\
0.859259206598599	2.42737283484743\\
1.30741523004191	2.62511520505649\\
1.57426953542748	2.72304263504847\\
1.74766615441039	2.77435113512036\\
1.87207484559155	2.80265686669549\\
1.96924085114733	2.81832059141065\\
2.0502990590364	2.82612977760851\\
2.12132034355964	2.82842712474619\\
};
\addlegendentry{Сумма Минковского}

\end{axis}

\begin{axis}[%
width=0.798\linewidth,
height=0.578\linewidth,
at={(-0.104\linewidth,-0.064\linewidth)},
scale only axis,
xmin=0,
xmax=1,
ymin=0,
ymax=1,
axis line style={draw=none},
ticks=none,
axis x line*=bottom,
axis y line*=left,
legend style={legend cell align=left, align=left, draw=white!15!black}
]
\end{axis}
\end{tikzpicture}%
        \caption{Эллипсоидальные аппроксимации для 10 направлений.}
\end{figure}
\begin{figure}[b]
        \centering
        % This file was created by matlab2tikz.
%
%The latest updates can be retrieved from
%  http://www.mathworks.com/matlabcentral/fileexchange/22022-matlab2tikz-matlab2tikz
%where you can also make suggestions and rate matlab2tikz.
%
\definecolor{mycolor1}{rgb}{0.00000,0.44700,0.74100}%
\definecolor{mycolor2}{rgb}{0.85000,0.32500,0.09800}%
%
\begin{tikzpicture}

\begin{axis}[%
width=0.618\linewidth,
height=0.487\linewidth,
at={(0\linewidth,0\linewidth)},
scale only axis,
xmin=-10,
xmax=10,
xlabel style={font=\color{white!15!black}},
xlabel={$x_1$},
ymin=-6,
ymax=6,
ylabel style={font=\color{white!15!black}},
ylabel={$x_2$},
axis background/.style={fill=white},
axis x line*=bottom,
axis y line*=left,
xmajorgrids,
ymajorgrids,
legend style={at={(0.03,0.97)}, anchor=north west, legend cell align=left, align=left, draw=white!15!black}
]
\addplot [color=mycolor1, forget plot]
  table[row sep=crcr]{%
1.1711787112841	3.34230777773576\\
1.34444389618673	3.33686722689733\\
1.50648318534277	3.32144701348819\\
1.65806242150429	3.29723890999844\\
1.79995419924852	3.26523166869947\\
1.93290799118999	3.22622794766654\\
2.05763032437368	3.18086276334402\\
2.17477223041259	3.12962150018857\\
2.28492171716205	3.07285634836238\\
2.3885995032216	3.01080060567632\\
2.48625667387058	2.94358064140543\\
2.57827324651315	2.87122553999274\\
2.66495688065443	2.79367457178694\\
2.74654114431786	2.71078271324095\\
2.82318287015103	2.62232448808267\\
2.8949582146523	2.52799644473262\\
2.96185708693385	2.42741864049226\\
3.02377565347494	2.3201355847161\\
3.0805066685131	2.20561721602438\\
3.13172744617751	2.08326066796132\\
3.17698540702193	1.95239382914097\\
3.21568133490354	1.81228204209558\\
3.24705081945703	1.66213971753183\\
3.27014489780722	1.50114915910447\\
3.28381172002077	1.32848945666784\\
3.28668221628109	1.14337881243048\\
3.2771642749897	0.945133918403077\\
3.25345178932588	0.733249674250213\\
3.21355684151227	0.507501133889645\\
3.15537469570401	0.268066487797808\\
3.07679115926015	0.015664562009573\\
2.97583883725379	-0.248307401701393\\
2.85090136682411	-0.521659161442773\\
2.70095216778915	-0.801347823435408\\
2.52579793500058	-1.08350783899399\\
2.32628156602984	-1.3635980584876\\
2.10439191407115	-1.63667268179468\\
1.86323617293213	-1.89775167737354\\
1.60685736899554	-2.14223401489281\\
1.33991789027605	-2.3662777508081\\
1.06730585933085	-2.56707514274059\\
0.793739425298697	-2.74297851870391\\
0.523437853608312	-2.89347215049445\\
0.259902596754298	-3.0190207516634\\
0.00581903236219554	-3.1208447467499\\
-0.2369372555121	-3.20067372983419\\
-0.467220354507	-3.2605178574641\\
-0.684493265340969	-3.30248032509504\\
-0.888694234930048	-3.32861897022119\\
-1.0801092196493	-3.34085455680139\\
-1.25926106809304	-3.34091776153004\\
-1.42681950330378	-3.33032515794113\\
-1.58353177449541	-3.31037507736533\\
-1.73017158467123	-3.2821558685242\\
-1.86750300244094	-3.24656096722397\\
-1.99625599202063	-3.20430689704777\\
-2.11711054093165	-3.15595168681459\\
-2.23068687011842	-3.10191219445658\\
-2.33753972875556	-3.04247952074268\\
-2.43815523449964	-2.97783215102365\\
-2.5329490934295	-2.90804674702345\\
-2.6222653208492	-2.83310668006404\\
-2.70637479428206	-2.75290849531896\\
-2.78547311726505	-2.66726655590999\\
-2.85967737154948	-2.5759161597878\\
-2.92902140016315	-2.47851546990414\\
-2.99344930844609	-2.37464666482809\\
-3.05280690958282	-2.26381681735669\\
-3.10683089311493	-2.14545915779927\\
-3.15513558252945	-2.01893559215682\\
-3.19719730262603	-1.88354163878489\\
-3.23233664163192	-1.73851533219976\\
-3.25969932318791	-1.58305212075492\\
-3.27823706849968	-1.41632833369867\\
-3.28669080350863	-1.23753634540448\\
-3.28357990805066	-1.04593497683649\\
-3.26720291066679	-0.840918687724783\\
-3.23565696131938	-0.622108318461704\\
-3.18688516768166	-0.389463978954261\\
-3.11876167995196	-0.143416533926881\\
-3.02922302993614	0.11499241610004\\
-2.91644915653916	0.383980836121862\\
-2.77908754849337	0.660934801326638\\
-2.61649915361415	0.942389492277687\\
-2.42898799999709	1.22411526429499\\
-2.21796402213223	1.50132757866017\\
-1.9859883878281	1.76901289011117\\
-1.7366683581316	2.02232922526914\\
-1.47440268220478	2.25701269019761\\
-1.20401775520347	2.46971278707818\\
-0.930363383198716	2.65819606075749\\
-0.657943143023997	2.82139307813531\\
-0.390637102663735	2.95930311865273\\
-0.131543811712393	3.07279931782965\\
0.117062449314586	3.1633871238911\\
0.353684227850852	3.23296276733506\\
0.577496523037587	3.28360333410188\\
0.788216624026908	3.3174035782658\\
0.985971690059736	3.33636162404406\\
1.1711787112841	3.34230777773576\\
};
\addplot [color=mycolor1, forget plot]
  table[row sep=crcr]{%
1.21926508344108	3.27334196367036\\
1.39130630412839	3.26794365173349\\
1.55153598090393	3.25269915059364\\
1.70081822737307	3.22886110922975\\
1.84001535490823	3.19746467488806\\
1.96995663540462	3.15934744514424\\
2.09141838442226	3.11517076291109\\
2.20511208961792	3.06544020144585\\
2.31167798429415	3.01052405216791\\
2.41168207975524	2.95066926128589\\
2.50561517604179	2.88601465321008\\
2.59389276130835	2.81660150697173\\
2.67685499694353	2.74238167588435\\
2.75476618649088	2.66322350375262\\
2.82781326027155	2.57891582390079\\
2.89610289157071	2.489170352485\\
2.95965690978143	2.39362282222959\\
3.0184057055742	2.29183326184168\\
3.07217934834312	2.18328592468469\\
3.12069617535093	2.0673895238837\\
3.16354868972977	1.94347865783223\\
3.20018675510097	1.81081762911891\\
3.22989834610506	1.66860828812922\\
3.25178857180767	1.51600407734397\\
3.26475841566826	1.35213309692555\\
3.26748572490109	1.17613369082119\\
3.25841251405898	0.987206612265718\\
3.23574463765734	0.784687977211719\\
3.19747218836968	0.568146474359487\\
3.14142113656216	0.337505988781863\\
3.06534780932005	0.0931901220106009\\
2.96708629627689	-0.163722532576174\\
2.84475284042693	-0.431353836986617\\
2.69699912472874	-0.706924876445341\\
2.52328828199352	-0.986738253877743\\
2.32414719016614	-1.26628199746165\\
2.1013342676447	-1.54047573935053\\
1.85786382933463	-1.80404664367384\\
1.59785273337257	-2.05198218219746\\
1.32619884624731	-2.27997573381088\\
1.04814809460687	-2.48477539672332\\
0.768837071389945	-2.66437165107218\\
0.492898365330516	-2.81800515148501\\
0.224188427726118	-2.94602190329025\\
-0.0343421486742361	-3.04963199342895\\
-0.280651418753673	-3.13063387286929\\
-0.513517918101939	-3.19115408703827\\
-0.732402138426053	-3.23343245666966\\
-0.937291976094646	-3.25966380617091\\
-1.12855566644943	-3.27189398375029\\
-1.30681506667752	-3.27196083427941\\
-1.47284399601866	-3.26146857518149\\
-1.62749116596371	-3.24178474838236\\
-1.77162454720074	-3.21405096074922\\
-1.9060930438003	-3.1792009422838\\
-2.03170136752072	-3.13798151469425\\
-2.14919451031851	-3.09097368125037\\
-2.25924887975171	-3.03861221534132\\
-2.36246781585791	-2.9812029127911\\
-2.45937977071431	-2.91893717481279\\
-2.55043787954686	-2.85190389003105\\
-2.63601998899356	-2.78009875378493\\
-2.71642844964106	-2.70343125192797\\
-2.79188914508071	-2.62172958112562\\
-2.86254933659929	-2.53474380432833\\
-2.92847396752118	-2.44214756807608\\
-2.98964010899711	-2.34353875337005\\
-3.04592925441029	-2.23843950866587\\
-3.09711719913452	-2.12629623737968\\
-3.14286129724365	-2.00648030014085\\
-3.18268499588943	-1.87829046258472\\
-3.21595975211415	-1.74095849136247\\
-3.24188479216075	-1.59365978768918\\
-3.25946575421678	-1.43553154684198\\
-3.26749415178163	-1.26570160623257\\
-3.26453089777461	-1.0833317902911\\
-3.24889889583809	-0.887679962191258\\
-3.21869188468844	-0.678184770435691\\
-3.17180903470924	-0.454575648787598\\
-3.10602656951161	-0.217007231912446\\
-3.01911769523831	0.0337887596613096\\
-2.90902859638195	0.296350206020716\\
-2.77410929927924	0.568356995772149\\
-2.6133828822338	0.846566357802756\\
-2.42681658880744	1.12685188938682\\
-2.21553984909621	1.4043777730154\\
-1.98194673587055	1.67391391702028\\
-1.72963332752344	1.93025930170482\\
-1.4631559553519	2.16870275608031\\
-1.18764436113277	2.38543046078226\\
-0.908344887791558	2.57779963933086\\
-0.630184750931516	2.74443551343009\\
-0.357433580063127	2.88515703813491\\
-0.0935024920494292	3.00077594658033\\
0.15911854673327	3.09283095590222\\
0.398812914783312	3.163314718129\\
0.624719128954104	3.21443380598977\\
0.836580651685356	3.24842176843127\\
1.03459285132093	3.26740886591368\\
1.21926508344108	3.27334196367036\\
};
\addplot [color=mycolor1, forget plot]
  table[row sep=crcr]{%
1.27235524758866	3.20606889606907\\
1.44317107064595	3.20071319568392\\
1.60153774787983	3.18564973372891\\
1.74843510276585	3.16219597058339\\
1.88482914833813	3.13143492326719\\
2.01163976581522	3.09423891990115\\
2.12972112934266	3.05129435069619\\
2.23985098586104	3.00312508327947\\
2.34272577688468	2.95011332159164\\
2.43895935773047	2.89251739452003\\
2.52908368509872	2.8304863814774\\
2.61355030737859	2.76407171387989\\
2.69273182444275	2.69323600549816\\
2.76692271100404	2.61785941119026\\
2.83633904449915	2.53774382694731\\
2.9011167660551	2.45261524824953\\
2.96130814896022	2.36212461590775\\
3.01687616843364	2.26584751343373\\
3.06768647413494	2.16328315185123\\
3.11349667956523	2.05385320276079\\
3.15394272225835	1.93690123748885\\
3.18852214664256	1.81169382176359\\
3.21657436300479	1.67742472593303\\
3.23725830595004	1.53322426152362\\
3.24952854266518	1.37817645273807\\
3.25211187660099	1.21134756681076\\
3.24348797643739	1.03183035661999\\
3.22187962263265	0.838808978297762\\
3.18526077526704	0.631649497785332\\
3.13139352539076	0.410019487872334\\
3.05790730118758	0.174036479049343\\
2.96243394314779	-0.0755620455059427\\
2.84280816668981	-0.337245310360568\\
2.69733198476799	-0.608544401674519\\
2.52508258389448	-0.885980316488405\\
2.32621826608086	-1.16511378557182\\
2.10221443243537	-1.44075380334538\\
1.85595416779634	-1.70732887684435\\
1.59161769905144	-1.95937674660199\\
1.31436302269743	-2.19206276802241\\
1.02985073295835	-2.40161735788381\\
0.743712717332933	-2.58560229945124\\
0.46107506075	-2.74296710880307\\
0.186217675282633	-2.87391600087286\\
-0.0775960985361585	-2.97964801275228\\
-0.328133772499326	-3.06204567327999\\
-0.564094671509743	-3.12337551626293\\
-0.784948102500155	-3.16603958225219\\
-0.990751904935546	-3.19239302335658\\
-1.18198074976573	-3.20462570620811\\
-1.35937986129559	-3.20469658668081\\
-1.52384947555703	-3.19430684199975\\
-1.67635893937152	-3.17489872235242\\
-1.81788618426142	-3.14766969123728\\
-1.94937729908481	-3.11359431862355\\
-2.07172112440674	-3.07344891393212\\
-2.18573453476242	-3.02783582028248\\
-2.29215496469956	-2.9772056520116\\
-2.39163756765905	-2.92187665091016\\
-2.48475509068474	-2.86205088696231\\
-2.57199908552444	-2.79782734529923\\
-2.65378147146904	-2.72921210646848\\
-2.73043574186991	-2.65612590224109\\
-2.80221729049535	-2.57840935563713\\
-2.86930244872764	-2.49582622010879\\
-2.93178588931728	-2.40806493900825\\
-2.98967608320614	-2.31473886823903\\
-3.04288850753444	-2.21538555650091\\
-3.09123631078617	-2.10946557375074\\
-3.13441816403804	-1.99636153685547\\
-3.17200309148303	-1.87537822273494\\
-3.20341221640447	-1.74574500735886\\
-3.22789763501203	-1.60662234752231\\
-3.24451911637134	-1.45711464736294\\
-3.25212012271858	-1.29629261534778\\
-3.24930586973196	-1.12322905938835\\
-3.23442791317944	-0.937052827376804\\
-3.20558210147849	-0.737025951103928\\
-3.16062953150577	-0.522648426972396\\
-3.09725286895748	-0.293792623174874\\
-3.01306191087989	-0.0508640093726965\\
-2.90576066218136	0.205024084380883\\
-2.77338096163089	0.471886516761707\\
-2.61457267711254	0.746751816909297\\
-2.42891789191419	1.0256452840926\\
-2.21721135786237	1.30371511604229\\
-1.98163270338355	1.57552461067301\\
-1.72574098366183	1.83549137118179\\
-1.45425689642249	2.0784046669254\\
-1.17265530960846	2.29991677896501\\
-0.886647911507271	2.49690345800275\\
-0.601666181623873	2.66762625184561\\
-0.322445205446415	2.81168834633155\\
-0.0527673783806232	2.92982878700886\\
0.204625871021466	3.02362775785022\\
0.447985415526008	3.09519465143329\\
0.676417888047168	3.14689079713794\\
0.889709449657395	3.1811133975024\\
1.08814645535659	3.20014614291162\\
1.27235524758866	3.20606889606907\\
};
\addplot [color=mycolor1, forget plot]
  table[row sep=crcr]{%
1.331122526514	3.1411519504109\\
1.50070265021491	3.13583953954657\\
1.65714137817264	3.12096355391375\\
1.801554491142	3.09791011124431\\
1.93502746499673	3.06781117661229\\
2.05858258751093	3.03157307872426\\
2.1731603825899	2.98990544455629\\
2.27961069804984	2.94334806244517\\
2.37868996212446	2.89229446064749\\
2.47106207995261	2.83701177144793\\
2.55730118991524	2.7776568956931\\
2.63789504769877	2.71428921024775\\
2.71324818860979	2.64688015759524\\
2.78368427326967	2.57532008083767\\
2.849447181013	2.49942265679744\\
2.9107005052992	2.41892725977025\\
2.96752514659502	2.33349957604737\\
3.01991470673442	2.24273079836124\\
3.06776837901516	2.14613577319474\\
3.11088101529588	2.04315056769814\\
3.14893005419316	1.93313008597277\\
3.18145904053263	1.81534662115847\\
3.20785759670683	1.68899060993897\\
3.22733798348076	1.5531753925286\\
3.23890890115111	1.40694850199169\\
3.24134805532164	1.24931291790325\\
3.23317639837383	1.07926277119951\\
3.2126390158541	0.895839016149664\\
3.177700448064	0.69821122718357\\
3.1260656987857	0.48579126058317\\
3.05524169695212	0.258381986555632\\
2.96265615170331	0.0163582778932624\\
2.84584911120226	-0.239133300220675\\
2.70274374574876	-0.505985173879308\\
2.53198394173171	-0.780996403060881\\
2.33329738953371	-1.05985624741079\\
2.10781090040156	-1.33729898614381\\
1.85822480693181	-1.60745595567134\\
1.58876376753239	-1.86437606022391\\
1.30487103163859	-2.10262351755484\\
1.01268938097015	-2.31782176529302\\
0.718441030482152	-2.50702035799616\\
0.427846180930232	-2.66881735136476\\
0.145694042649061	-2.80324567414949\\
-0.124380790790276	-2.91149239657131\\
-0.379927228454498	-2.99554341560088\\
-0.619563635809872	-3.05783481070723\\
-0.842784978801955	-3.10096237588935\\
-1.04974635506385	-3.12746975963674\\
-1.24106000163502	-3.13971312950987\\
-1.41762489959441	-3.13978845934957\\
-1.58049474935395	-3.12950408936179\\
-1.73078211578909	-3.11038263397954\\
-1.86959284659487	-3.0836797413193\\
-1.99798391664057	-3.0504108901298\\
-2.11693835508858	-3.01138053033187\\
-2.22735200674953	-2.96721019972459\\
-2.33002807564449	-2.9183638378905\\
-2.42567646575971	-2.86516952714163\\
-2.51491579139968	-2.80783748683269\\
-2.59827657406231	-2.74647447073001\\
-2.67620460331178	-2.68109487045736\\
-2.74906375329484	-2.61162888258028\\
-2.81713774976441	-2.53792809983284\\
-2.88063050418508	-2.45976886910422\\
-2.93966469482067	-2.37685374065141\\
-2.99427829770052	-2.28881132968064\\
-3.04441876799606	-2.19519493611852\\
-3.08993455892186	-2.09548033507462\\
-3.13056365740489	-1.98906327620029\\
-3.1659188360687	-1.87525743658402\\
-3.19546940308262	-1.75329388623641\\
-3.21851942578628	-1.6223235794633\\
-3.23418278584247	-1.48142501135315\\
-3.24135609946731	-1.32961999626986\\
-3.23869164592522	-1.16590151836407\\
-3.22457415197589	-0.989278675191041\\
-3.19710771688668	-0.798844633232678\\
-3.15412234119223	-0.593873733391723\\
-3.09321312554347	-0.373952564769022\\
-3.01182828769036	-0.139145727453714\\
-2.90742279527531	0.109811261632613\\
-2.77768964509709	0.371312723440716\\
-2.62086721620327	0.642715147198648\\
-2.43609684116487	0.920255137140879\\
-2.22377291482778	1.19911285221477\\
-1.9857996409858	1.4736651303669\\
-1.72566175604194	1.73792940542773\\
-1.44824671829608	1.98613763849639\\
-1.15942195970072	2.2133243081796\\
-0.865448289650424	2.41579463270279\\
-0.572362115157663	2.59137293020573\\
-0.285459358621382	2.73940143022293\\
-0.0089668275124522	2.8605320594217\\
0.254079449410444	2.95639689740252\\
0.50178376295928	3.02924769275413\\
0.73322946004012	3.08163193886222\\
0.948268430835706	3.11614082010529\\
1.14730794379744	3.13523684905876\\
1.331122526514	3.1411519504109\\
};
\addplot [color=mycolor1, forget plot]
  table[row sep=crcr]{%
1.3964102868975	3.07930591740507\\
1.56473037248086	3.07403794379094\\
1.71915971789631	3.05935744921679\\
1.86097350375265	3.03672287758242\\
1.99139432743895	3.00731572039262\\
2.1115596313916	2.9720749730111\\
2.22250508572583	2.93173110760911\\
2.32515837474101	2.88683696403915\\
2.42033934512417	2.83779441843637\\
2.50876368745401	2.78487654478706\\
2.59104822992295	2.72824544210295\\
2.66771656628559	2.66796611120257\\
2.73920417543268	2.60401683410806\\
2.8058624705604	2.53629650265557\\
2.86796138470457	2.46462930281241\\
2.92569018902551	2.38876711316769\\
2.97915627449035	2.30838993674501\\
3.0283816240354	2.2231046670328\\
3.07329667483836	2.13244250358555\\
3.11373123222949	2.03585539346854\\
3.14940206395745	1.93271200046933\\
3.17989679896267	1.82229391931966\\
3.20465381435363	1.70379319033865\\
3.22293797458151	1.57631267238219\\
3.2338124753162	1.4388715448039\\
3.23610777024232	1.29041917122538\\
3.22838979925308	1.12986177408304\\
3.20893170831586	0.956107756545554\\
3.17569616377437	0.768138804989481\\
3.12633929237158	0.565114530656926\\
3.05825190772653	0.346517306048897\\
2.96865788840053	0.112339502298826\\
2.85479090633259	-0.136694414172483\\
2.71416515318386	-0.398894887802234\\
2.54493847995357	-0.671409304753301\\
2.34633470428385	-0.950126228446477\\
2.11905007755647	-1.22975688596338\\
1.86553297853278	-1.50414781259021\\
1.59002111748132	-1.76682066303184\\
1.29826723729962	-2.01165403512721\\
0.996976978743987	-2.23355486779611\\
0.693081911192092	-2.42895471339356\\
0.39302449262342	-2.59602245879899\\
0.10221242031689	-2.73458150354873\\
-0.175275587441964	-2.84580580466324\\
-0.436737284630285	-2.93180948107383\\
-0.680711669681773	-2.99523570280546\\
-0.90674417217064	-3.03891333463364\\
-1.11512406441372	-3.0656088453726\\
-1.30664144946907	-3.07787111261669\\
-1.48238709584989	-3.07795134920049\\
-1.64360097670177	-3.06777622825643\\
-1.79156543558501	-3.04895448590184\\
-1.92753469255854	-3.02280191024239\\
-2.05269169243189	-2.99037438471845\\
-2.16812430727995	-2.95250254880701\\
-2.27481451443136	-2.90982444674297\\
-2.37363578578764	-2.862814384836\\
-2.46535529269494	-2.81180734627586\\
-2.5506385879597	-2.75701894637208\\
-2.63005519501394	-2.69856122981593\\
-2.70408406639507	-2.6364547413632\\
-2.77311822563175	-2.57063732609075\\
-2.83746812660903	-2.50097008809051\\
-2.89736339022059	-2.42724088988245\\
-2.95295263753188	-2.34916572943973\\
-3.00430115212401	-2.26638830150595\\
-3.05138608715343	-2.17847804645976\\
-3.09408889814955	-2.08492702572269\\
-3.13218464480151	-1.98514605333694\\
-3.16532778277964	-1.87846068021922\\
-3.19303408855744	-1.76410789964738\\
-3.21465847136983	-1.64123485723529\\
-3.22936869755953	-1.50890145057942\\
-3.23611559058606	-1.36608953760508\\
-3.23360122780567	-1.21172256498737\\
-3.22024823272812	-1.04470074878479\\
-3.19417568159924	-0.863958330476281\\
-3.15319057074906	-0.668550491360247\\
-3.09480814848275	-0.457777436782653\\
-3.01631904392544	-0.231350619837756\\
-2.91492429939803	0.0104007841530797\\
-2.78795789248472	0.266298052535966\\
-2.63320550966211	0.534090230068579\\
-2.44930392305496	0.81029787181832\\
-2.23616740772478	1.0901969999926\\
-1.99534603198641	1.36801221596138\\
-1.73019696643741	1.63734829722721\\
-1.44576953465053	1.89181673044\\
-1.14837734553415	2.12573379650668\\
-0.844933408739004	2.33472294072841\\
-0.542206513864935	2.51607716502315\\
-0.246175073832186	2.66881927116423\\
0.0383973476672991	2.79349537142056\\
0.308127665128874	2.89180296092848\\
0.560959304573922	2.96616893137416\\
0.795966987499965	3.01936648760927\\
1.01310046780864	3.05421819894996\\
1.21292743436782	3.07339593495123\\
1.39641028689749	3.07930591740507\\
};
\addplot [color=mycolor1, forget plot]
  table[row sep=crcr]{%
1.46927735143059	3.02133831480419\\
1.63629310035011	3.01611658260144\\
1.7886098202728	3.00164167558635\\
1.9276891077562	2.97944772464369\\
2.05491029732378	2.95076569259784\\
2.17153957212107	2.91656520547484\\
2.27871666322874	2.8775945552278\\
2.37745251949422	2.8344162564902\\
2.46863330682164	2.78743718747514\\
2.55302761411155	2.73693325892181\\
2.63129483361326	2.68306900537496\\
2.70399342855486	2.62591267004484\\
2.77158828747314	2.56544738150251\\
2.83445666632126	2.5015789733937\\
2.89289239235212	2.43414092191058\\
2.94710808822543	2.36289679566997\\
2.99723519856857	2.28754054406744\\
3.04332158332015	2.2076949033788\\
3.08532639615547	2.12290818421282\\
3.1231119034702	2.03264973086929\\
3.15643183233097	1.93630442907388\\
3.18491578251768	1.83316680739732\\
3.20804922766571	1.72243556373827\\
3.22514871272181	1.60320979852764\\
3.23533211017806	1.47448891063472\\
3.23748435336648	1.33517907761217\\
3.23022011209626	1.18411055554444\\
3.21184667676275	1.02007169815298\\
3.18033318732	0.841867471807365\\
3.13329656201969	0.648411901057839\\
3.06802005852571	0.43886436819431\\
2.9815266205418	0.212817298608521\\
2.87073386701779	-0.0294650388365179\\
2.73271642082416	-0.286772864596821\\
2.56508773964061	-0.556683859649172\\
2.36648135624935	-0.835374737329632\\
2.13706031610444	-1.11760600143702\\
1.87892800807281	-1.39696779352388\\
1.59628548819868	-1.66641963371706\\
1.29521584343495	-1.91905712943754\\
0.98308441078536	-2.14893516111894\\
0.667683177910749	-2.35173150036812\\
0.356342600548117	-2.52508420102095\\
0.0552298920758284	-2.6685567979865\\
-0.231040238055092	-2.78330877698785\\
-0.499476806935808	-2.87161508239866\\
-0.748546728386371	-2.93637438496716\\
-0.977882288979001	-2.98069824932831\\
-1.18795708828636	-3.00761818760242\\
-1.37979150940508	-3.01990728770175\\
-1.55471582553872	-3.01999292769046\\
-1.71419607931498	-3.00993230317797\\
-1.85971553954364	-2.99142602962007\\
-1.99269995310453	-2.96585147839687\\
-2.11447465948285	-2.93430374220686\\
-2.22624346296074	-2.89763701989019\\
-2.32908151045466	-2.85650259944695\\
-2.42393660051112	-2.81138175736593\\
-2.51163509895577	-2.76261313414686\\
-2.59288993246338	-2.71041479938398\\
-2.66830903791877	-2.65490151430437\\
-2.73840325102495	-2.59609778961128\\
-2.80359300360421	-2.53394731925019\\
-2.86421343036985	-2.46831930479252\\
-2.92051761046401	-2.39901210446869\\
-2.9726777203323	-2.32575456506696\\
-3.02078387548822	-2.24820533607854\\
-3.06484040527515	-2.16595043299995\\
-3.10475924890094	-2.07849932066814\\
-3.14035009427886	-1.98527984148405\\
-3.17130681823211	-1.88563243717909\\
-3.19718975024394	-1.77880433519281\\
-3.21740331018235	-1.66394473219286\\
-3.23116872653173	-1.54010256139314\\
-3.23749192767323	-1.40622924192809\\
-3.23512747185839	-1.26118994304902\\
-3.22254076830271	-1.10378839066771\\
-3.19787313688991	-0.932812045608432\\
-3.1589177725706	-0.747106330769343\\
-3.10311960877118	-0.545687819133067\\
-3.02761813373429	-0.327905611498936\\
-2.92935806841843	-0.0936553921639545\\
-2.80529528362486	0.156360933207452\\
-2.6527188844722	0.420357817462845\\
-2.46968808889038	0.695227527519055\\
-2.25554013643435	0.976425937428937\\
-2.01136913889677	1.25807902966462\\
-1.74032911995783	1.53337706969685\\
-1.44761450798646	1.7952432948003\\
-1.14004500386212	2.03715506205079\\
-0.825313975647545	2.25391343942017\\
-0.511086370392609	2.44215796957082\\
-0.204180704567995	2.6005151780265\\
0.089984434113084	2.72940082914642\\
0.367614630809603	2.83059573675472\\
0.626479062175471	2.90674459359535\\
0.865667727030579	2.96089679720825\\
1.08527323667551	2.99615293675265\\
1.28607612949814	3.0154312519322\\
1.46927735143059	3.02133831480419\\
};
\addplot [color=mycolor1, forget plot]
  table[row sep=crcr]{%
1.55106012395173	2.96820108237492\\
1.71670039509252	2.96302826671365\\
1.866773650437	2.94877167105386\\
2.00295881409983	2.92704398092938\\
2.12681348158948	2.89912477506173\\
2.23974677558376	2.86601146149039\\
2.34301123972137	2.82846644198549\\
2.43770593604595	2.78705801960205\\
2.52478548784315	2.74219439080212\\
2.60507169099129	2.69415100565756\\
2.67926560970047	2.64309199157726\\
2.74795891918622	2.5890864527319\\
2.81164378731278	2.53212042352181\\
2.87072089988808	2.47210515402108\\
2.92550540208963	2.40888228458754\\
2.97623060040506	2.34222635006831\\
3.02304927747868	2.27184495383945\\
3.06603243671452	2.19737687562467\\
3.10516522738603	2.11838833098142\\
3.14033971357704	2.03436759302091\\
3.17134405045729	1.94471823184124\\
3.19784753184586	1.8487513458686\\
3.21938089574431	1.74567738509342\\
3.23531125925981	1.63459854816662\\
3.24481117124365	1.5145033422472\\
3.24682163905332	1.38426581574507\\
3.2400097972802	1.24265331190513\\
3.22272343233275	1.0883484320068\\
3.19294725989111	0.919993240239861\\
3.14827014535604	0.736266352737939\\
3.0858787298984	0.536005698720147\\
3.00260100298442	0.318389812049431\\
2.89503167884312	0.0831856446549773\\
2.75977558920631	-0.168942950911744\\
2.59383787432449	-0.436098516778841\\
2.39516019521871	-0.714856398931109\\
2.16324375154413	-1.00012576884616\\
1.89972165804939	-1.28529226677395\\
1.60868151094332	-1.56272681971187\\
1.29654972154919	-1.82463070149011\\
0.971467852367241	-2.06403750195299\\
0.642282701040046	-2.27569441529627\\
0.317430689649746	-2.4565734498626\\
0.00402370353224809	-2.60591096109272\\
-0.292670312621192	-2.72485050727559\\
-0.569328782256929	-2.81587147438732\\
-0.824363369645803	-2.88219145532937\\
-1.05754664046762	-2.92726816036025\\
-1.26960505851823	-2.95445057042329\\
-1.46185776232636	-2.96677376570594\\
-1.63593450851817	-2.96686534407497\\
-1.79357561533683	-2.95692623777213\\
-1.93650161604717	-2.9387545377024\\
-2.06633574417822	-2.9137899458279\\
-2.18456334326995	-2.88316473659085\\
-2.29251540962804	-2.84775327892353\\
-2.3913668893472	-2.80821624292687\\
-2.48214327273579	-2.76503805721257\\
-2.56573125059349	-2.7185575147283\\
-2.6428907665372	-2.66899206563575\\
-2.71426685168592	-2.61645657716661\\
-2.78040030166351	-2.5609773681715\\
-2.84173666627119	-2.50250225126656\\
-2.89863325499027	-2.44090720078849\\
-2.95136397640314	-2.3760001438983\\
-3.00012186641745	-2.30752226257338\\
-3.04501914441794	-2.23514710521278\\
-3.08608458434066	-2.15847774454589\\
-3.12325790984976	-2.07704219053697\\
-3.15638082793169	-1.99028728365994\\
-3.18518421352558	-1.89757137239653\\
-3.20927086577889	-1.79815624601589\\
-3.22809320374143	-1.69119909028635\\
-3.24092530881614	-1.57574571829855\\
-3.24682894495659	-1.45072708034643\\
-3.24461374770692	-1.31496217351149\\
-3.23279290849798	-1.16717205316684\\
-3.20953773849238	-1.0060117507856\\
-3.17263792532288	-0.830129432068425\\
-3.11947955211359	-0.638264643741304\\
-3.04706020690436	-0.429398878067323\\
-2.95206903691371	-0.202969729290941\\
-2.83106667794031	0.0408488926212685\\
-2.68079965641057	0.300817909394284\\
-2.49866665674673	0.574306299043266\\
-2.2833102427322	0.857058927982916\\
-2.03523660271512	1.14318345421905\\
-1.75728959879587	1.42547108532208\\
-1.45477290398947	1.69608663675106\\
-1.13507778488407	1.94752294706826\\
-0.806838945162423	2.17357884107625\\
-0.478831266741026	2.37007933016134\\
-0.158921038277483	2.53515197102043\\
0.14666436956898	2.66904955862832\\
0.433639963808184	2.77366049616377\\
0.699590316977137	2.85190376286616\\
0.943658792568718	2.90717031589308\\
1.16614417848768	2.94289755467673\\
1.36810982104501	2.9622952925113\\
1.55106012395173	2.96820108237492\\
};
\addplot [color=mycolor1, forget plot]
  table[row sep=crcr]{%
1.64345852363055	2.92105776339129\\
1.80761732204454	2.9159376568551\\
1.95528246259752	2.90191530665662\\
2.08838519009964	2.88068407505593\\
2.20868456958566	2.85357029991398\\
2.31774709165881	2.82159539265171\\
2.41694629827407	2.78553142969586\\
2.50747327914833	2.74594812184625\\
2.59035220798831	2.70325101153669\\
2.66645738332475	2.65771166454735\\
2.7365297390966	2.60949094897271\\
2.80119172506743	2.55865651553041\\
2.86096001123987	2.50519547473524\\
2.91625577678492	2.44902309758856\\
2.96741249260937	2.38998819246268\\
3.01468115556098	2.32787565263928\\
3.05823291728477	2.26240653502024\\
3.09815899310691	2.19323592423288\\
3.13446764802029	2.11994876026648\\
3.16707794468208	2.04205376673256\\
3.19580980735125	1.95897562065497\\
3.22036981287264	1.87004557129875\\
3.24033197908955	1.77449087509657\\
3.25511271117103	1.67142371408152\\
3.26393904345144	1.55983077873778\\
3.26580948222163	1.4385655265798\\
3.25944729491222	1.30634641319336\\
3.24324730307265	1.16176629450234\\
3.21521958208466	1.00332085639153\\
3.1729375897116	0.829467334288683\\
3.11350486983964	0.638728528193248\\
3.03356407943737	0.429859907970045\\
2.92938400748147	0.202096599130512\\
2.79707107094353	-0.0445127188891665\\
2.63295314192722	-0.308703031259661\\
2.43416129871963	-0.58758430600421\\
2.19937376746793	-0.876349005913475\\
1.92958437026846	-1.1682643309048\\
1.62865018731794	-1.45510280891482\\
1.30333820181883	-1.72804614070483\\
0.962706283169578	-1.97889373554604\\
0.616909840592004	-2.20122852165362\\
0.275783131818425	-2.39117340478182\\
-0.0523698660120398	-2.54754598280501\\
-0.361476459509195	-2.67147265521413\\
-0.647834840576828	-2.76569685549902\\
-0.909834199478792	-2.8338396057494\\
-1.14746606281625	-2.8797869376726\\
-1.36180400226045	-2.90727098253691\\
-1.55455572035072	-2.91963433147983\\
-1.72772588709667	-2.91973241682908\\
-1.88338728987833	-2.90992407129012\\
-2.02353978634332	-2.89211005318921\\
-2.15003280356409	-2.86779221172766\\
-2.26453015630305	-2.83813698352557\\
-2.36850105812257	-2.80403468242865\\
-2.46322609254644	-2.76615089014963\\
-2.54981079323019	-2.72496898281352\\
-2.62920226484629	-2.68082419945641\\
-2.7022061438885	-2.63393023468613\\
-2.76950238990311	-2.58439948452057\\
-2.83165912241808	-2.53225801075129\\
-2.88914413537482	-2.47745613762484\\
-2.94233393935166	-2.41987541944139\\
-2.991520274945	-2.35933254995108\\
-3.03691405420067	-2.29558063773091\\
-3.07864664870852	-2.22830815131749\\
-3.11676836891342	-2.15713574628113\\
-3.15124387830048	-2.08161112701245\\
-3.18194416388457	-2.00120207572529\\
-3.20863454636729	-1.915287813493\\
-3.23095806911326	-1.82314896648223\\
-3.24841347485075	-1.72395663315592\\
-3.26032690319308	-1.61676144428311\\
-3.2658164975705	-1.50048416440397\\
-3.2637494350632	-1.37391042204906\\
-3.25269172194939	-1.23569372924949\\
-3.23085280738982	-1.08437321816084\\
-3.19603021020376	-0.918415572441364\\
-3.14556463730224	-0.736294298285524\\
-3.0763241683432	-0.536622987018219\\
-2.984747049609	-0.31836059341864\\
-2.86698469390163	-0.0811021282462965\\
-2.71919399936644	0.174548415407669\\
-2.53801971706134	0.446560906457461\\
-2.32126736186821	0.731109536982981\\
-2.06868395448563	1.02240047830783\\
-1.78265173563995	1.31286886198096\\
-1.4685171815477	1.5938532425014\\
-1.13431162016923	1.85668637703656\\
-0.789815291758204	2.09393194184063\\
-0.445196810880552	2.3003844084033\\
-0.109648904470953	2.47353264858748\\
0.209589152140003	2.61342251208492\\
0.507643648353963	2.72208375164278\\
0.781911203897705	2.80278585825734\\
1.03164857637783	2.85934727107597\\
1.25745038849592	2.89561713910808\\
1.46075663711784	2.91515243254314\\
1.64345852363055	2.92105776339129\\
};
\addplot [color=mycolor1, forget plot]
  table[row sep=crcr]{%
1.74865638965985	2.88137384176368\\
1.9111836710517	2.87631163289344\\
2.05623578848918	2.8625432733052\\
2.18603530742935	2.84184384607306\\
2.30256716606395	2.81558335823543\\
2.40756928437339	2.78480240242788\\
2.50254327140952	2.75027755034177\\
2.58877477830147	2.71257508773336\\
2.66735724515472	2.67209371664647\\
2.73921552628005	2.62909765426629\\
2.80512756595767	2.58374173106424\\
2.86574328648835	2.53608997072186\\
2.92160039629944	2.48612890195269\\
2.97313709926905	2.43377659779317\\
3.0207017959453	2.37888820127053\\
3.06455987921545	2.32125849205571\\
3.10489767976282	2.26062187935964\\
3.14182353134469	2.19665007013035\\
3.17536581272197	2.12894755657744\\
3.20546768582882	2.05704499373496\\
3.23197808934203	1.98039050162045\\
3.25463836423213	1.89833894078851\\
3.27306368856977	1.81013929990314\\
3.28671829938547	1.71492054271635\\
3.29488331912033	1.61167665999203\\
3.29661596543324	1.49925236882212\\
3.29069916429219	1.37633205880754\\
3.27558139366417	1.24143642297809\\
3.24930844435586	1.09293399751025\\
3.20945245739726	0.929078802553234\\
3.15305014452637	0.748090410689812\\
3.07657268871799	0.548298317047832\\
2.97596499191847	0.328376066452439\\
2.84680977757094	0.0876869626864989\\
2.68468495329439	-0.17325657939976\\
2.48577362462204	-0.452264497626377\\
2.24772972631639	-0.744992585231754\\
1.97067943599865	-1.04472501879821\\
1.65807220993474	-1.34265634595364\\
1.31698343105269	-1.62881221959049\\
0.957554514883364	-1.89348839583847\\
0.591585930754306	-2.12878970482027\\
0.230708359684339	-2.32973727022693\\
-0.115245097273801	-2.49460300242127\\
-0.439198726834851	-2.62449590283836\\
-0.73702122592129	-2.72250681712574\\
-1.0071386753611	-2.7927748429065\\
-1.24987806314248	-2.83972219698163\\
-1.46679045121205	-2.86754714735139\\
-1.66009022809612	-2.87995479744129\\
-1.83225170660669	-2.88005998154867\\
-1.98574995545417	-2.87039434462988\\
-2.12291225381591	-2.85296575291794\\
-2.24584534304923	-2.82933679992632\\
-2.35641026393456	-2.80070390154217\\
-2.45622465215348	-2.76796819319801\\
-2.54667927319293	-2.7317951045581\\
-2.62896065345727	-2.69266242953424\\
-2.70407508028692	-2.6508980274472\\
-2.77287140616133	-2.60670872452633\\
-2.8360613932267	-2.56020198090403\\
-2.89423707595181	-2.51140169625282\\
-2.9478850124543	-2.46025927635858\\
-2.99739747596487	-2.40666083432898\\
-3.04308069233051	-2.3504311788191\\
-3.08516020825859	-2.29133505490838\\
-3.12378340691881	-2.22907595079856\\
-3.15901908747461	-2.16329266303052\\
-3.19085389967043	-2.0935537234282\\
-3.21918527572757	-2.01934973515528\\
-3.24381033001434	-1.94008365202977\\
-3.26441000484119	-1.85505908316613\\
-3.28052753819491	-1.76346684744097\\
-3.29154014235314	-1.66437029480382\\
-3.29662266873162	-1.55669044235931\\
-3.29470210917163	-1.4391928762441\\
-3.28440226177932	-1.31047983674861\\
-3.26397914640617	-1.16899318217089\\
-3.23125040615276	-1.01303728294905\\
-3.18352689318838	-0.840835481318406\\
-3.11756307229887	-0.650639262158281\\
-3.02955578265117	-0.440914290992723\\
-2.91523790632757	-0.210628333221629\\
-2.77013032758312	0.0403443027574127\\
-2.59002029256205	0.310719099018591\\
-2.37170493501425	0.597277998955034\\
-2.11394990123552	0.894492241963369\\
-1.81846128678003	1.19452690062365\\
-1.49051303930177	1.48783794592996\\
-1.13884197202703	1.76438739211616\\
-0.774635244305867	2.01519847780868\\
-0.409838890092488	2.23374014663527\\
-0.055353551683624	2.4166687307997\\
0.280296632940744	2.5637629907927\\
0.591527576299578	2.67724232125039\\
0.875558910164217	2.76083170511072\\
1.13185505571184	2.8188917889414\\
1.36143427945912	2.85578004516305\\
1.56623969593983	2.87546934518756\\
1.74865638965985	2.88137384176368\\
};
\addplot [color=mycolor1, forget plot]
  table[row sep=crcr]{%
1.86949259182206	2.851041894416\\
2.03018408192628	2.84604446976994\\
2.17237182230191	2.83255422748916\\
2.29861218587825	2.81242751649097\\
2.41114071012953	2.78707343834094\\
2.51187956915807	2.75754580885987\\
2.60246366961384	2.72461978989047\\
2.68427383880935	2.68885303783657\\
2.75847075137008	2.65063311987221\\
2.82602639165477	2.61021350747583\\
2.88775166141764	2.56774038583839\\
2.94431972289547	2.52327219864455\\
2.99628515378765	2.47679346786944\\
3.0440991929524	2.42822406951643\\
3.0881213996135	2.37742483755954\\
3.12862800618561	2.32420011403568\\
3.16581715426406	2.2682976576057\\
3.19981108386009	2.20940615768815\\
3.2306552047691	2.14715046917842\\
3.2583138162204	2.08108457955932\\
3.28266205296936	2.01068224645031\\
3.30347341769176	1.9353252070536\\
3.32040200762152	1.85428888022718\\
3.3329582615423	1.766725592993\\
3.3404767620074	1.67164562940943\\
3.34207438265767	1.56789692668809\\
3.33659699577093	1.45414520093235\\
3.32255330074806	1.3288579322199\\
3.29803556898633	1.19029834444202\\
3.26063004462054	1.0365397390164\\
3.20732570918268	0.865516700806141\\
3.13444095158236	0.675137754701349\\
3.03760531141148	0.463492569463746\\
2.91185824557558	0.229191149003763\\
2.751953616705	-0.0281375813465989\\
2.55296987173708	-0.307201442265731\\
2.31128645067285	-0.604357222621918\\
2.02585325470708	-0.913112149382156\\
1.69944506937969	-1.22415553322009\\
1.33936031707961	-1.52621725105204\\
0.957020192028156	-1.8077470468533\\
0.566323354814297	-2.05894306961403\\
0.181253089060065	-2.27336840108622\\
-0.186359555833534	-2.44856971763358\\
-0.528173197105508	-2.58564108209243\\
-0.839579847232978	-2.68814030184989\\
-1.11914634816221	-2.76088306328742\\
-1.36770876357337	-2.80897125440314\\
-1.58747699763472	-2.83717491568213\\
-1.78132765176004	-2.84962817416705\\
-1.95232315093529	-2.84974105301399\\
-2.10342373972285	-2.84023327533267\\
-2.23733814474857	-2.8232230272202\\
-2.35646319745511	-2.80033068410671\\
-2.46287530736805	-2.77277712795278\\
-2.55834911181294	-2.7414682639622\\
-2.64438817914622	-2.70706376792262\\
-2.72225912478861	-2.67003109368689\\
-2.79302456833141	-2.63068688472884\\
-2.85757276629238	-2.58922811280272\\
-2.91664309970992	-2.54575503873085\\
-2.97084729611412	-2.50028772566202\\
-3.02068658959441	-2.45277745981182\\
-3.0665651341151	-2.40311409839735\\
-3.10879997935108	-2.35113008330869\\
-3.14762784831883	-2.29660163007258\\
-3.18320884957155	-2.23924741727532\\
-3.21562712602563	-2.17872495371844\\
-3.24488829079953	-2.11462468320623\\
-3.27091332569106	-2.04646179793107\\
-3.29352841538703	-1.97366567471089\\
-3.31244995552533	-1.89556683644361\\
-3.32726370464709	-1.8113814004094\\
-3.33739675935027	-1.72019315351482\\
-3.34208075361804	-1.62093377446207\\
-3.3403045017562	-1.5123624388513\\
-3.33075439751556	-1.39304730665817\\
-3.31174159508735	-1.2613535150509\\
-3.28111695661172	-1.11544570406821\\
-3.23617900766813	-0.953318255767754\\
-3.17358828797476	-0.772873598414232\\
-3.08931551102783	-0.572077543716153\\
-2.97867232411474	-0.349227927358609\\
-2.83650045946101	-0.103371682617185\\
-2.6576173160146	0.165117691616635\\
-2.43760740047999	0.453853738244781\\
-2.17396722216338	0.757806234807604\\
-1.86742258058554	1.06902256126122\\
-1.52298134004091	1.37704825677867\\
-1.15013513350289	1.67022508059427\\
-0.761815116812467	1.93763085392071\\
-0.372273819119186	2.17099814937115\\
0.00534641066062597	2.36587592682938\\
0.360863004955486	2.52169235164974\\
0.687831398489772	2.64092765973912\\
0.983332883262683	2.72791009701897\\
1.24718773674731	2.78769820053344\\
1.48102129027608	2.82528353001849\\
1.68745078808525	2.84514024371596\\
1.86949259182206	2.851041894416\\
};
\addplot [color=mycolor1, forget plot]
  table[row sep=crcr]{%
2.00970785227068	2.83255951582902\\
2.16829459506053	2.82763576051588\\
2.30731523094934	2.81445258780785\\
2.4297046947207	2.79494515119588\\
2.53797278103981	2.77055534078189\\
2.6342363110595	2.74234305753359\\
2.72026598535531	2.71107549953793\\
2.79753589144043	2.67729624992662\\
2.86726977473392	2.641377482169\\
2.93048162747608	2.60355872384335\\
2.98800995892579	2.56397518676149\\
3.04054597826645	2.52267808539351\\
3.0886562731758	2.47964880147187\\
3.13280064745708	2.43480827137261\\
3.1733457264768	2.38802258405018\\
3.21057482112467	2.33910547032397\\
3.24469439457135	2.2878181226942\\
3.27583731546928	2.23386659248464\\
3.30406290887543	2.17689685470301\\
3.3293536276346	2.11648750120062\\
3.35160795356739	2.05213991488855\\
3.37062888832095	1.98326569362213\\
3.38610709619862	1.90917104296972\\
3.39759740635128	1.8290378685784\\
3.40448696981579	1.74190142248968\\
3.40595292314895	1.64662468673838\\
3.40090701464311	1.54187037394823\\
3.3879244932748	1.42607275936865\\
3.36515505297591	1.29741396377849\\
3.33021556567402	1.15381342249475\\
3.28006917739702	0.992945935825488\\
3.21090547265683	0.812313688080073\\
3.11805519292571	0.609410871367224\\
2.9960036943785	0.382033114912436\\
2.83860917803556	0.128788191766161\\
2.63967063205985	-0.150161553980915\\
2.39398493840761	-0.452182928665707\\
2.09890926629675	-0.7713121847596\\
1.75614168558561	-1.09789400392915\\
1.37302381558774	-1.41923798734738\\
0.962478354853678	-1.72151318483291\\
0.541122414584741	-1.99241543448791\\
0.126086512136677	-2.22353565354694\\
-0.268133107704887	-2.41143559577134\\
-0.631577552867782	-2.55720324292457\\
-0.959131116260233	-2.66503948191669\\
-1.24968021080543	-2.74066012508232\\
-1.50483046760844	-2.79004009696308\\
-1.72770386342792	-2.81865646246515\\
-1.92204364433203	-2.83115261151491\\
-2.09164728934969	-2.83127375429125\\
-2.24005708169665	-2.82194264110306\\
-2.37042229740948	-2.80538913525812\\
-2.48546291352305	-2.78328649808138\\
-2.58748692100441	-2.75687309892156\\
-2.67843186444601	-2.72705265847209\\
-2.75991406419349	-2.69447306987477\\
-2.8332769777984	-2.65958660284275\\
-2.8996347939691	-2.62269497831406\\
-2.95990987145525	-2.58398257526998\\
-3.01486390614539	-2.54354048910545\\
-3.06512327954946	-2.50138357319334\\
-3.11119923529598	-2.45746206919823\\
-3.15350353100543	-2.41166899718123\\
-3.19236012037668	-2.36384413054485\\
-3.22801328528932	-2.3137751086922\\
-3.26063248329872	-2.26119602514444\\
-3.29031400960702	-2.20578365590005\\
-3.31707939300565	-2.1471513506101\\
-3.34087024559752	-2.08484049055914\\
-3.36153905617021	-2.01830932075474\\
-3.37883514501698	-1.94691889450658\\
-3.39238467255464	-1.86991584538964\\
-3.40166320955607	-1.78641176052254\\
-3.40595894338724	-1.69535914022167\\
-3.40432415968643	-1.59552441581267\\
-3.39551232876152	-1.48545947032062\\
-3.37789822385948	-1.36347491041274\\
-3.34937958137217	-1.22762150438802\\
-3.30726196979925	-1.07569147765284\\
-3.24813565478921	-0.905259609032158\\
-3.16776724190762	-0.7137958141245\\
-3.06105331900569	-0.498894981689488\\
-2.92212022887542	-0.258680560012784\\
-2.74469760922797	0.00756902553045162\\
-2.5229174330319	0.298574963947673\\
-2.25263627569165	0.610127618005103\\
-1.93316830403618	0.934409971105092\\
-1.5689363762796	1.26008813656112\\
-1.17019380909004	1.57359617920111\\
-0.752052372940286	1.86152328722302\\
-0.331816374923651	2.11328116118183\\
0.0743649903603743	2.32291239975837\\
0.454129450740461	2.48937682241393\\
0.799989615652643	2.61552393552942\\
1.10897926826103	2.70649812614107\\
1.38150841585277	2.76827057497105\\
1.62007426223507	2.80663226408797\\
1.82819971892363	2.82666489950488\\
2.00970785227068	2.83255951582902\\
};
\addplot [color=mycolor1, forget plot]
  table[row sep=crcr]{%
2.17430603473841	2.82928793073052\\
2.33044535777815	2.82444897168392\\
2.46594297362775	2.81160678704818\\
2.58415716708103	2.79277030050891\\
2.68789254387122	2.76940595677011\\
2.77946699484521	2.7425714670914\\
2.86078573906653	2.71301902411803\\
2.93341104171975	2.68127265050854\\
2.9986230753596	2.64768508851882\\
3.05747086442469	2.61247907686435\\
3.11081384046105	2.57577692251115\\
3.15935513715665	2.53762134677768\\
3.20366787001532	2.49798979992626\\
3.24421554049886	2.45680381827628\\
3.28136751247959	2.41393452310087\\
3.31541029254594	2.36920500026929\\
3.3465551303234	2.32239002374647\\
3.37494224630124	2.27321336944935\\
3.40064178827692	2.22134278883848\\
3.42365140325324	2.16638255933344\\
3.44389007534897	2.10786339134407\\
3.46118760463784	2.04522934489396\\
3.47526876652081	1.97782129527289\\
3.48573077423729	1.90485640152273\\
3.49201214853071	1.82540300920684\\
3.49335047065297	1.73835053234855\\
3.48872578597209	1.64237424725429\\
3.47678575017332	1.53589584679112\\
3.45574827300754	1.41704249382237\\
3.42327810810487	1.28361072542021\\
3.37633699194203	1.13304806285973\\
3.31101526097114	0.962476172308744\\
3.22237097655495	0.768796474817825\\
3.10433695290387	0.548942043211334\\
2.9498123210435	0.300361978709517\\
2.75112793000477	0.0218247814546395\\
2.50112454582504	-0.285440449235098\\
2.19500744060268	-0.616443644580409\\
1.83279348396173	-0.961489600474761\\
1.42152136958137	-1.3063970008595\\
0.975845648074582	-1.63450693799913\\
0.515964639544959	-1.93017020234628\\
0.0633211215265111	-2.18224450162264\\
-0.363949894265142	-2.38592186002566\\
-0.753796552421497	-2.54230740615597\\
-1.10060884720806	-2.65651208684665\\
-1.40389862561362	-2.73547347665808\\
-1.66643431787696	-2.78630322905355\\
-1.89260401811819	-2.81535915112738\\
-2.08728476369964	-2.82789001873507\\
-2.25518878499693	-2.82801992101541\\
-2.40055085490442	-2.81888822167606\\
-2.52702294235802	-2.80283518923491\\
-2.63767930270744	-2.78157977328078\\
-2.73507199115711	-2.75636932685319\\
-2.82130341222054	-2.72809764441867\\
-2.89809909341037	-2.69739456722559\\
-2.96687326596811	-2.66469244542933\\
-3.02878476984102	-2.63027467163603\\
-3.08478318424195	-2.5943106773748\\
-3.13564609696506	-2.55688082262324\\
-3.18200873955121	-2.51799374438073\\
-3.22438719765583	-2.47759802849779\\
-3.26319624623718	-2.43558952510388\\
-3.29876265010897	-2.39181521440426\\
-3.33133455291616	-2.34607421503113\\
-3.36108736562728	-2.29811628370925\\
-3.38812635908966	-2.24763796026885\\
-3.4124859564092	-2.1942763488638\\
-3.43412549761435	-2.13760038238197\\
-3.45292099547021	-2.0770992851559\\
-3.4686520985225	-2.01216782796055\\
-3.48098310405704	-1.94208786685168\\
-3.48943639793565	-1.86600559716315\\
-3.49335612455861	-1.7829039871982\\
-3.49185921454444	-1.69157008370894\\
-3.4837701826888	-1.59055749156644\\
-3.46753554642006	-1.47814566124369\\
-3.44111377096973	-1.35230025189567\\
-3.40183832761321	-1.21064372066799\\
-3.34625674726662	-1.05045381725873\\
-3.26996105481103	-0.868721532579536\\
-3.16745032441434	-0.662320310385892\\
-3.03211101926641	-0.428362470131523\\
-2.85646733273374	-0.164834179270465\\
-2.63292320715374	0.128425970177643\\
-2.35522203042063	0.448465239159584\\
-2.02065569486654	0.788006644772256\\
-1.63254200363741	1.13497995420302\\
-1.20180828192215	1.47359901929041\\
-0.746311741639337	1.78722909211069\\
-0.287482595077459	2.06211021974623\\
0.15439697822222	2.29018412674959\\
0.56404269287998	2.46977369323667\\
0.932710877487586	2.60426859699022\\
1.2575759170864	2.69994355321456\\
1.54000858568913	2.76398340384577\\
1.78376187044044	2.80319756768445\\
1.99357716381973	2.82340732367622\\
2.17430603473841	2.82928793073052\\
};
\addplot [color=mycolor1, forget plot]
  table[row sep=crcr]{%
2.37008966809432	2.84583467340354\\
2.52336101304862	2.84109395545007\\
2.65493105339172	2.82863117278086\\
2.76862267778487	2.81052085769044\\
2.86755001510834	2.7882437769117\\
2.95423266046411	2.76284622438936\\
3.03070422684652	2.73505813212284\\
3.09860634829873	2.70537870361567\\
3.15926627203745	2.6741376613985\\
3.2137591926997	2.64153863503644\\
3.26295751751183	2.60768960954319\\
3.30756937735905	2.57262400672931\\
3.34816845266683	2.53631493276348\\
3.38521681902747	2.49868435450491\\
3.41908214635702	2.45960840435614\\
3.45005024766046	2.41891960171226\\
3.47833367691025	2.37640647248396\\
3.50407681296056	2.33181081137422\\
3.5273576240679	2.28482263812596\\
3.54818606830286	2.23507272898644\\
3.56649882899361	2.18212244331108\\
3.58214978799726	2.12545040199865\\
3.59489527503244	2.06443540369702\\
3.60437266468315	1.99833478863803\\
3.6100702845173	1.92625729534473\\
3.6112858071673	1.84712934765145\\
3.60706929830707	1.7596537595386\\
3.59614590411288	1.66226026082249\\
3.57681194832122	1.55304843364005\\
3.54679746572038	1.42972636836512\\
3.50308914992433	1.28955395370645\\
3.44171303509782	1.1293104537239\\
3.35749134537993	0.945325142230009\\
3.24382230805877	0.733640633043858\\
3.09259867497051	0.490420643488118\\
2.89448951307269	0.212751666019624\\
2.63993722127874	-0.100030692797326\\
2.321252018958	-0.44454182182079\\
1.93586348388899	-0.811582534465944\\
1.48987037994912	-1.18554016734986\\
0.999850167481938	-1.54625462069504\\
0.490800302083166	-1.87351819056015\\
-0.00976760886881658	-2.15229736504357\\
-0.47862074310768	-2.3758291948706\\
-0.900969397609134	-2.54529082015793\\
-1.27082671096515	-2.66712019821288\\
-1.58885251183779	-2.74994862101959\\
-1.85957398047222	-2.80238751830318\\
-2.08913881445424	-2.83189831408148\\
-2.28390274429081	-2.84444872664284\\
-2.44970755124551	-2.84458774491333\\
-2.59160185077621	-2.83568206418269\\
-2.71380175301886	-2.82017758069955\\
-2.81976180515058	-2.79982915091515\\
-2.91228461517198	-2.77588316498957\\
-2.99363402781255	-2.74921520687656\\
-3.06563694687861	-2.72043083707034\\
-3.12976911307429	-2.68993811207581\\
-3.18722482786806	-2.65799919115049\\
-3.2389724585397	-2.62476673107416\\
-3.28579804751296	-2.59030927637789\\
-3.32833924397085	-2.55462866090615\\
-3.36711145022197	-2.51767153878045\\
-3.40252769945223	-2.47933650369906\\
-3.43491342463749	-2.43947777474164\\
-3.46451696149501	-2.39790607275902\\
-3.49151635019761	-2.35438704353505\\
-3.51602275007773	-2.30863737162493\\
-3.53808054304273	-2.2603185491965\\
-3.55766395638719	-2.20902810013721\\
-3.57466976262089	-2.15428789845941\\
-3.58890528732447	-2.09552905336005\\
-3.60007054506754	-2.03207265890987\\
-3.60773279122859	-1.96310553255902\\
-3.61129108299517	-1.8876499221355\\
-3.60992754867974	-1.80452611441937\\
-3.60254096123919	-1.71230707561059\\
-3.58765697617668	-1.60926498692926\\
-3.56330832071048	-1.49331136579171\\
-3.52687810256052	-1.36193640716485\\
-3.47490208222548	-1.21216100573125\\
-3.40283508129398	-1.04052939130352\\
-3.30480996090834	-0.843194970645823\\
-3.17346649956951	-0.61618907910191\\
-3.00001523271263	-0.356005884048084\\
-2.77482637545278	-0.060655994624172\\
-2.48893587499026	0.26874259104347\\
-2.13675360062321	0.626078821425667\\
-1.71965020354387	0.998892456188959\\
-1.24894356722181	1.36887963300529\\
-0.745958132188205	1.71518129242515\\
-0.237835878231387	2.01959717603921\\
0.249302172258792	2.27105681113961\\
0.696170609995157	2.46700138361849\\
1.09254058306598	2.61163994282778\\
1.43609585729356	2.71285279072919\\
1.72975951920885	2.77946657892918\\
1.97909740102934	2.81960054669284\\
2.19048887725582	2.83997842985783\\
2.37008966809432	2.84583467340354\\
};
\addplot [color=mycolor1, forget plot]
  table[row sep=crcr]{%
2.6064589828494	2.88862616640401\\
2.75637109414822	2.883999158983\\
2.88357674785248	2.87195708876679\\
2.99239608382493	2.85462841786627\\
3.08625831948619	2.83349617714752\\
3.16787772828815	2.80958545644448\\
3.23940444300088	2.7835968756565\\
3.30254662301602	2.75600014215947\\
3.35866654942507	2.72709908273117\\
3.40885502240302	2.69707659985484\\
3.4539884661676	2.66602556057904\\
3.4947725378337	2.63396979088962\\
3.53177528488427	2.60087803167415\\
3.56545219557076	2.56667278660369\\
3.59616489562369	2.53123534419147\\
3.62419476499703	2.49440779820407\\
3.64975236089754	2.4559925584995\\
3.67298321301724	2.41574959242231\\
3.69397027804979	2.37339143202941\\
3.71273307794191	2.32857580000093\\
3.72922327387838	2.28089552774105\\
3.74331611674247	2.22986524679513\\
3.7547968298526	2.17490411565226\\
3.76334047684362	2.11531358795446\\
3.76848318927127	2.05024893201245\\
3.76958170216264	1.9786828892658\\
3.76575688652169	1.89935956348535\\
3.7558152982188	1.81073649949425\\
3.73814067326371	1.71091325142572\\
3.71054500789473	1.59754622004815\\
3.67006719259592	1.46775350112306\\
3.61270839144689	1.31802251130379\\
3.53310287780543	1.14415182510341\\
3.42415232901896	0.941294042417582\\
3.27672169828017	0.704226331451283\\
3.07963497410675	0.428057684036333\\
2.82043165537474	0.109644908416964\\
2.48755709893795	-0.250105954072952\\
2.07450524984844	-0.643388213434865\\
1.58529271977403	-1.05349246283719\\
1.03845463107716	-1.455972350082\\
0.465526666727522	-1.82428685022561\\
-0.0971822994778088	-2.13769698101945\\
-0.619092708293362	-2.38657010343355\\
-1.0818079135853	-2.57227964755975\\
-1.47930913785865	-2.70326146416607\\
-1.81429707356905	-2.79054540480154\\
-2.09396044839245	-2.84474516538712\\
-2.32688763075065	-2.87470949979353\\
-2.52134887775033	-2.88725599524887\\
-2.68456326331782	-2.88740426406207\\
-2.82251915336173	-2.8787542171408\\
-2.94005166132089	-2.8638482597678\\
-3.04101234893106	-2.84446473367481\\
-3.12845234049856	-2.82183799344284\\
-3.20478679703347	-2.79681700764569\\
-3.27193131730941	-2.76997725891164\\
-3.3314105736454	-2.74169880644514\\
-3.38444298534991	-2.71222039102835\\
-3.43200595019206	-2.68167673487756\\
-3.47488577361502	-2.65012405459818\\
-3.51371571737217	-2.61755724492019\\
-3.54900485010229	-2.58392108518757\\
-3.58115973412663	-2.54911704627594\\
-3.61050044905449	-2.513006731903\\
-3.63727202218022	-2.47541259993116\\
-3.6616519849773	-2.43611632205393\\
-3.68375447858502	-2.39485491537756\\
-3.70363106361525	-2.35131458846271\\
-3.72126812544317	-2.3051220656886\\
-3.73658047820497	-2.25583296979466\\
-3.74940042748902	-2.20291663828086\\
-3.75946111404373	-2.14573651277383\\
-3.76637237832512	-2.08352496403139\\
-3.76958659341788	-2.01535110261679\\
-3.76835083198556	-1.94007980586714\\
-3.76164027613148	-1.85631995225746\\
-3.74806589202947	-1.76235990295578\\
-3.72574714829292	-1.65608906519836\\
-3.6921384105946	-1.53490686388163\\
-3.64379699476674	-1.3956265130413\\
-3.57608526073647	-1.23439409674593\\
-3.48281657465312	-1.04666947244277\\
-3.35590135639156	-0.827362343558006\\
-3.18515089500215	-0.571289645864595\\
-2.95857988607527	-0.274202847881349\\
-2.6637901042153	0.0653566355405698\\
-2.2911130842731	0.443383637526595\\
-1.83861873047926	0.847729220696273\\
-1.31734186160238	1.25738786218882\\
-0.752965148538522	1.64591896811454\\
-0.180742070492837	1.98874283474401\\
0.364702770050638	2.27033687843499\\
0.858482795179477	2.48690166765121\\
1.28869281512832	2.64393862627122\\
1.6542297268519	2.75166978874569\\
1.96051424731082	2.82117963527452\\
2.21572948658655	2.86228446899281\\
2.42844674742143	2.88280838763335\\
2.6064589828494	2.88862616640401\\
};
\addplot [color=mycolor1, forget plot]
  table[row sep=crcr]{%
2.89659432980286	2.96676193805005\\
3.04261530234076	2.96226512107439\\
3.16502460886113	2.95068440839662\\
3.26865741175649	2.93418704719984\\
3.35725109593054	2.91424503832653\\
3.43369849912562	2.89185260649569\\
3.50024831591357	2.86767479814833\\
3.55865857421896	2.84214816144364\\
3.61031232001395	2.81554874916724\\
3.65630421859564	2.78803798830021\\
3.69750524874502	2.7596935331198\\
3.7346110320036	2.73052984690876\\
3.76817793593031	2.70051165407789\\
3.7986499818713	2.66956232868485\\
3.82637874357441	2.63756856028971\\
3.85163778682918	2.60438214155117\\
3.87463271664979	2.56981937111724\\
3.89550751961209	2.53365830398773\\
3.91434757504091	2.49563387047219\\
3.93117942554675	2.45543069532303\\
3.94596711272854	2.41267325741374\\
3.95860456359512	2.36691281659495\\
3.96890311675038	2.31761027725621\\
3.97657275161087	2.26411383557913\\
3.98119485614645	2.20562984672729\\
3.98218333748952	2.14118483016385\\
3.97872940632414	2.06957590343883\\
3.96972327094597	1.9893062437335\\
3.95364306233142	1.89850159302278\\
3.9283974564574	1.79480380868164\\
3.89110391335995	1.675239124805\\
3.83778061354591	1.53606460590245\\
3.76293148666367	1.37261143308965\\
3.6590218454198	1.17917817429086\\
3.51590365944118	0.949097715740115\\
3.32040545205894	0.675227100275456\\
3.05662035949933	0.351281984348361\\
2.70790143678588	-0.0254718849548195\\
2.26183065135216	-0.45005683315673\\
1.71833479132094	-0.905539269966142\\
1.09751946508794	-1.36238128704534\\
0.439952295888921	-1.78508392229778\\
-0.204892158233907	-2.14427478985174\\
-0.795522309411559	-2.42598247372705\\
-1.30879486149514	-2.63205385457764\\
-1.73948386246382	-2.7740324841352\\
-2.09385275020469	-2.86641317063372\\
-2.38310831218506	-2.9225058993971\\
-2.61920035292473	-2.95290136813909\\
-2.81284514872881	-2.96541205374497\\
-2.97290932641795	-2.96556939715673\\
-3.10644015794655	-2.95720537493408\\
-3.2189319875517	-2.94294495808261\\
-3.31463614585934	-2.92457530690133\\
-3.39683944813556	-2.90330721565073\\
-3.46809074661824	-2.87995516111223\\
-3.53037679839086	-2.85505969363724\\
-3.58525568559304	-2.82897019776391\\
-3.63395695620088	-2.80190075278642\\
-3.67745648029622	-2.77396777398898\\
-3.71653236360908	-2.74521525198787\\
-3.75180672247609	-2.71563145415037\\
-3.78377686886483	-2.68515963854271\\
-3.81283848422717	-2.65370444876066\\
-3.83930262805466	-2.62113505968558\\
-3.86340787361186	-2.58728572894365\\
-3.88532843746716	-2.55195410850564\\
-3.90517882758428	-2.51489743876753\\
-3.92301524024991	-2.47582655031098\\
-3.93883365621706	-2.43439741085335\\
-3.95256428852306	-2.39019975498695\\
-3.96406168175765	-2.34274210155407\\
-3.97308930908288	-2.29143217652475\\
-3.9792968971866	-2.23555139539052\\
-3.98218784472244	-2.17422159649788\\
-3.98107286747296	-2.1063616421403\\
-3.97500424384076	-2.0306308336959\\
-3.96268255309857	-1.9453554158036\\
-3.94232441658273	-1.84843405923733\\
-3.91147547523812	-1.73721883569902\\
-3.86674834272132	-1.6083714994132\\
-3.80346313929779	-1.45770448476151\\
-3.71517576571336	-1.28003921712422\\
-3.59311423455181	-1.06916436795909\\
-3.42564401367706	-0.81807283846141\\
-3.1981127735367	-0.519810584526548\\
-2.89383572578272	-0.169431726839988\\
-2.49743742084217	0.232526788235511\\
-2.00153691664825	0.675524368527635\\
-1.41545564256883	1.1360006086817\\
-0.770243446554317	1.58012503409236\\
-0.11298263180075	1.9739034807845\\
0.508902205780387	2.29501428786976\\
1.06251140744219	2.53789012609105\\
1.53425404867556	2.71015448202869\\
1.92556178584946	2.82553485172154\\
2.24585827135187	2.89826474559884\\
2.50708931975046	2.9403670821331\\
2.72072963620371	2.9610001758028\\
2.89659432980286	2.96676193805005\\
};
\addplot [color=mycolor1, forget plot]
  table[row sep=crcr]{%
3.25912552532344	3.09324266531364\\
3.40075479465786	3.08889121951157\\
3.51801223395664	3.07780502924075\\
3.61624480213526	3.06217246485952\\
3.69947648861053	3.04344120313106\\
3.77075341827375	3.02256613476829\\
3.83239965728367	3.00017203474404\\
3.88620378798641	2.97666014161672\\
3.93355428317857	2.95227813953397\\
3.97553779282755	2.92716623650056\\
4.01301075784132	2.90138751585974\\
4.04665182531094	2.87494781248438\\
4.07700035875947	2.84780848553431\\
4.1044847677202	2.8198942489043\\
4.12944326369463	2.79109743186516\\
4.1521388515986	2.76127951769643\\
4.17276978659718	2.73027044613358\\
4.19147629157254	2.69786590048875\\
4.20834398499086	2.66382258803285\\
4.22340416900177	2.62785133053263\\
4.23663083508319	2.58960758460892\\
4.24793392093488	2.5486787841831\\
4.25714795241982	2.50456761286711\\
4.26401466840201	2.45666994036201\\
4.26815746931866	2.40424565290605\\
4.2690444268092	2.34637992245714\\
4.26593495471885	2.28193153358199\\
4.2578027961164	2.20946366815217\\
4.2432243386947	2.12715103461714\\
4.22021593992867	2.03265559398637\\
4.18599646687352	1.92296203525389\\
4.13664176158023	1.79416546599789\\
4.06658864013277	1.64121227545014\\
3.96794662741513	1.45762193816308\\
3.82961331960112	1.23528635760583\\
3.63633210909192	0.96459705871368\\
3.36821531474038	0.63543950469671\\
3.002045454138	0.239976599825101\\
2.51667707823438	-0.221836445673655\\
1.9044993812272	-0.734699889847041\\
1.18580916303899	-1.2634372751351\\
0.413743826998447	-1.7597067026252\\
-0.341863310285406	-2.18064657921402\\
-1.02281270170815	-2.50552986704684\\
-1.59982705916651	-2.73729161093054\\
-2.07028016133366	-2.89245873154349\\
-2.4465644340315	-2.99061032245346\\
-2.74589996190596	-3.04869727856406\\
-2.98479716449934	-3.07948042799385\\
-3.17703046646929	-3.09191777405463\\
-3.33338405234074	-3.09208364407099\\
-3.46206006354589	-3.08403218295096\\
-3.56922864873687	-3.07045261226997\\
-3.65952583873925	-3.05312518023861\\
-3.73644959712913	-3.03322632582248\\
-3.802657834974	-3.01152960911262\\
-3.86018713294046	-2.98853736603995\\
-3.91061169893487	-2.96456703941671\\
-3.95515868121247	-2.93980794958082\\
-3.99479201962409	-2.91435870000864\\
-4.03027367842802	-2.8882517715398\\
-4.06220855779067	-2.8614695133002\\
-4.09107752686464	-2.83395423114886\\
-4.11726169670518	-2.80561409923711\\
-4.14106010818103	-2.77632597920563\\
-4.16270233220879	-2.74593579823306\\
-4.18235698062868	-2.71425683029612\\
-4.20013674200496	-2.68106599098774\\
-4.21610023890133	-2.64609805777553\\
-4.23025071160572	-2.60903753620112\\
-4.24253123028556	-2.569507683344\\
-4.25281578214716	-2.52705594759409\\
-4.2608951209982	-2.48113475918942\\
-4.26645563286624	-2.43107617280046\\
-4.2690485600191	-2.37605827577271\\
-4.26804558369017	-2.31506047765258\\
-4.26257476548153	-2.24680372935623\\
-4.25142785944008	-2.16967034780763\\
-4.23292558549316	-2.08159650663929\\
-4.20472109273816	-1.97992895305211\\
-4.16351325898368	-1.86123725761094\\
-4.1046316041567	-1.72107692250579\\
-4.02144842120085	-1.55371432541402\\
-3.90458758851443	-1.35186824245691\\
-3.74097753180596	-1.106627367739\\
-3.51303771445789	-0.807919823449194\\
-3.19886052999548	-0.446266543352548\\
-2.77522862261228	-0.0168551754644445\\
-2.22596970505169	0.473625417843501\\
-1.55583841724168	0.99997826760293\\
-0.802144912027912	1.51868967653404\\
-0.0296675510206517	1.98150862124943\\
0.694232425793066	2.35537571235375\\
1.32497545153508	2.63219319355454\\
1.84780947670892	2.82320517706321\\
2.26914351855745	2.94750772659703\\
2.60476612527971	3.02376569855963\\
2.87197017953517	3.06686310061952\\
3.08600600689612	3.08755613156197\\
3.25912552532344	3.09324266531364\\
};
\addplot [color=mycolor1, forget plot]
  table[row sep=crcr]{%
3.72011336764885	3.28650261533118\\
3.85702413168351	3.28230597003519\\
3.96897610105512	3.27172810514178\\
4.06180092705263	3.25696084177994\\
4.13977420248845	3.23941639939732\\
4.20606357291101	3.22000459090589\\
4.26304295070732	3.19930775068244\\
4.31251211391209	3.17769171474465\\
4.35585073130903	3.15537674917136\\
4.39412715861489	3.13248318681217\\
4.42817594083798	3.10906089241272\\
4.45865349172613	3.08510821856777\\
4.4860783837653	3.06058398783418\\
4.51086062960552	3.03541471430641\\
4.5333229443513	3.00949844243116\\
4.55371602420306	2.98270603911308\\
4.57222920919513	2.95488040947898\\
4.58899741369647	2.92583384332133\\
4.60410483656924	2.89534348991946\\
4.61758565067858	2.86314476919085\\
4.62942157532395	2.82892232882523\\
4.63953591305895	2.79229792284285\\
4.6477832358126	2.75281428543453\\
4.65393336975607	2.70991366424717\\
4.65764755998012	2.66290910319778\\
4.65844355059981	2.61094574676467\\
4.65564456679094	2.55294826337533\\
4.64830447353098	2.48754880315085\\
4.63509714703282	2.412987529111\\
4.61415146693176	2.32697452337006\\
4.58280309607122	2.2264978124928\\
4.53721893257655	2.10755820649293\\
4.47182920549984	1.96481080674113\\
4.37847981232864	1.79110508686486\\
4.24521490284242	1.57696626547966\\
4.05468868211603	1.31021198654074\\
3.78258157057486	0.976271917354255\\
3.39746211800831	0.560513847144141\\
2.86564603019587	0.0547349303129109\\
2.16628095956492	-0.530918998666906\\
1.3163458188637	-1.15601715113801\\
0.386364271571009	-1.75373532161103\\
-0.521395033430425	-2.25951955673995\\
-1.32245507455195	-2.64185412289414\\
-1.98003247030003	-2.90611385282926\\
-2.49783304165879	-3.0770017070695\\
-2.89858578292377	-3.18160571774392\\
-3.20833386187208	-3.24175809995871\\
-3.44961337829407	-3.27287671472173\\
-3.63990028487811	-3.28520643597026\\
-3.79212961182057	-3.28537994521807\\
-3.9157126484899	-3.27765525513644\\
-4.01748172736179	-3.26476547925274\\
-4.10242429214688	-3.24846956642585\\
-4.17421535716773	-3.22990138708828\\
-4.23559313594463	-3.20978984403993\\
-4.28862168962215	-3.18859808518074\\
-4.3348747062288	-3.16661214278255\\
-4.37556482043913	-3.14399778752436\\
-4.41163533937276	-3.12083719473372\\
-4.44382586715106	-3.09715260383444\\
-4.4727196324573	-3.07292144245103\\
-4.49877782605937	-3.04808571249083\\
-4.52236456680204	-3.02255738720676\\
-4.54376496468835	-2.99622089810873\\
-4.56319795420717	-2.96893334830397\\
-4.5808250056787	-2.94052278162212\\
-4.59675540197106	-2.91078460550761\\
-4.61104843177575	-2.87947606989276\\
-4.62371255163821	-2.84630851359366\\
-4.63470126528994	-2.81093687707772\\
-4.64390511629923	-2.77294571693644\\
-4.6511387335686	-2.73183060765709\\
-4.65612123157329	-2.68697333269246\\
-4.65844733256287	-2.63760858249132\\
-4.65754516584234	-2.58277889731557\\
-4.65261452453361	-2.52127318585437\\
-4.64253597186815	-2.45154214564333\\
-4.62573588562742	-2.37158111951525\\
-4.59998426947178	-2.2787672414254\\
-4.56208956747593	-2.16963347584623\\
-4.5074365468492	-2.03955904725652\\
-4.4292905327985	-1.88235946304682\\
-4.31777414606879	-1.6897858888492\\
-4.15845144678625	-1.4510336699434\\
-3.93065071450564	-1.15260296148236\\
-3.60631549513373	-0.779398394422836\\
-3.15176018832689	-0.318841953790946\\
-2.53704200326684	0.229845489251054\\
-1.75695286026187	0.84232819540724\\
-0.855158528695751	1.46283364501566\\
0.0765427825210307	2.02106832642988\\
0.938738827138494	2.46648514783927\\
1.66964433031892	2.78741035581351\\
2.25519674352221	3.00145883703757\\
2.71117451008825	3.13606823051264\\
3.06331051275912	3.21613443636322\\
3.33632262894028	3.26020434931857\\
3.55022760940807	3.28090751767623\\
3.72011336764885	3.28650261533118\\
};
\addplot [color=mycolor1, forget plot]
  table[row sep=crcr]{%
4.31376385210612	3.5712909289112\\
4.44604796470272	3.56724523047818\\
4.55294229853712	3.55715130579748\\
4.64071758107381	3.54319152926674\\
4.71386006446686	3.52673700116767\\
4.77562771826157	3.50865147131496\\
4.82842228145861	3.48947632770434\\
4.8740398886074	3.46954452840045\\
4.91384175266373	3.44905163095831\\
4.94887186762898	3.42810053785964\\
4.97993921391307	3.40672986044454\\
5.00767585842429	3.38493186964506\\
5.03257843421039	3.36266366895516\\
5.05503796211655	3.33985381874632\\
5.07536132941632	3.31640577504021\\
5.09378664609597	3.29219895681292\\
5.11049395508259	3.26708789225794\\
5.12561224768768	3.24089963630617\\
5.13922334338563	3.21342944799483\\
5.15136287276803	3.18443453110304\\
5.16201830606558	3.15362544550125\\
5.17112365225563	3.12065456076874\\
5.17855006487048	3.08510061308643\\
5.18409106277195	3.04644799415585\\
5.1874403073849	3.00405877844646\\
5.18815871459962	2.95713457915007\\
5.18562586284048	2.90466395578317\\
5.17896776172049	2.84534903574844\\
5.16694835218675	2.77750189542916\\
5.14780441920348	2.69889655486705\\
5.11899093564329	2.6065555205818\\
5.07678312764305	2.4964401358647\\
5.01564854053493	2.36300223925064\\
4.92725373028048	2.1985462606597\\
4.79891330831509	1.99236703555086\\
4.61127879316853	1.72973582654781\\
4.33531671753969	1.39118557929181\\
3.92978196480634	0.953580933677038\\
3.3437375136301	0.396514247226397\\
2.53427804952791	-0.280980476147483\\
1.50745677047191	-1.03587163166222\\
0.357068633485824	-1.7751575057098\\
-0.761980801568648	-2.39880021333496\\
-1.72327572086905	-2.85783412550436\\
-2.48205251320246	-3.16295654865162\\
-3.05554184320146	-3.35235574201659\\
-3.48329478537669	-3.46408909594419\\
-3.80382171578637	-3.52638354906937\\
-4.04728934136811	-3.55781365822627\\
-4.23545614256618	-3.57002395842887\\
-4.38356160756193	-3.57020410987645\\
-4.50222697748644	-3.56279421365319\\
-4.59890521232152	-3.55055423665639\\
-4.67889059104945	-3.53521283874192\\
-4.74599897837135	-3.51785831702342\\
-4.80302249625893	-3.49917539843617\\
-4.85203452262181	-3.47959019054209\\
-4.89459620783079	-3.45936000860568\\
-4.93189798638166	-3.4386296892856\\
-4.9648577892612	-3.41746720159537\\
-4.99419005699258	-3.39588623180777\\
-5.02045477689393	-3.37386039493518\\
-5.04409263240669	-3.35133191832602\\
-5.06545031740138	-3.32821654269527\\
-5.08479872897419	-3.30440569935721\\
-5.1023458529469	-3.27976657861271\\
-5.11824553461315	-3.25414040114044\\
-5.1326028775606	-3.22733897851624\\
-5.14547666374989	-3.19913945845492\\
-5.1568788852583	-3.16927696331457\\
-5.16677117692064	-3.13743461832273\\
-5.17505759304233	-3.1032301976982\\
-5.18157272299476	-3.06619825240833\\
-5.18606350855219	-3.02576606630659\\
-5.18816218546643	-2.98122103397904\\
-5.1873463224102	-2.93166593613698\\
-5.18287964143246	-2.87595691119277\\
-5.17372362286784	-2.81261638748794\\
-5.15840389644149	-2.73970941162479\\
-5.13480555360966	-2.65466608790162\\
-5.09985528731257	-2.55402457940664\\
-5.04902196155985	-2.43305815523723\\
-4.97552653095822	-2.28523843610815\\
-4.86909697091442	-2.1014864433282\\
-4.71405748155353	-1.86921149932266\\
-4.48661682319537	-1.57134726483217\\
-4.1518034654525	-1.18623914751836\\
-3.66256272570685	-0.690777535072004\\
-2.96831088898074	-0.0714274511682935\\
-2.04438402146404	0.653640890152905\\
-0.938508424831904	1.41435936861618\\
0.21598448538672	2.10611182803394\\
1.26708430107625	2.64931391731234\\
2.1278446066936	3.02747233197286\\
2.78963433717024	3.26955278862576\\
3.28506386419243	3.41591365444406\\
3.65486208112682	3.50005904778578\\
3.93365070187683	3.54509915839146\\
4.14720426646698	3.56579116524099\\
4.31376385210611	3.5712909289112\\
};
\addplot [color=mycolor1, forget plot]
  table[row sep=crcr]{%
5.07461298996009	3.9742534137051\\
5.20312201465524	3.97033115483588\\
5.30587092362685	3.96063385017361\\
5.38952296477647	3.94733331450053\\
5.45874344806197	3.93176352930561\\
5.51686185932693	3.91474824761516\\
5.56629796059966	3.89679421172553\\
5.60884022191911	3.8782071291593\\
5.64583086106727	3.85916244128577\\
5.67829092663376	3.83974909822166\\
5.70700623307129	3.81999687063234\\
5.7325872750885	3.79989339005687\\
5.75551152455045	3.77939461209223\\
5.77615356449002	3.75843092844573\\
5.79480664213944	3.73691026877469\\
5.81169800874334	3.71471898236091\\
5.82699960588411	3.69172093049256\\
5.84083510008146	3.66775496851056\\
5.85328385958774	3.6426307988632\\
5.86438214150385	3.61612299648009\\
5.87412146180624	3.58796281533438\\
5.88244380882355	3.55782714999972\\
5.88923298014295	3.52532371137192\\
5.89430080465441	3.48997102991825\\
5.89736625162972	3.45117124520797\\
5.89802426051472	3.40817265543968\\
5.8956992693481	3.36001749023562\\
5.88957539716316	3.30546801732987\\
5.87849020807931	3.24290038135406\\
5.86077046633799	3.17014965928821\\
5.8339736340321	3.08428016763865\\
5.79447338218908	2.98124009511028\\
5.73678308902399	2.85533667481716\\
5.65243596636619	2.69843657867155\\
5.52812160933873	2.49876569105719\\
5.34263239614403	2.23920506114815\\
5.0621707267555	1.89525219867204\\
4.63447728325225	1.4339410702098\\
3.98645789817976	0.818301459391054\\
3.04127562108193	0.0276824728203648\\
1.78004297196272	-0.899091935467874\\
0.325218964179175	-1.83387368454933\\
-1.08389811375681	-2.61937894138237\\
-2.2550301337491	-3.17893916732926\\
-3.13784810473174	-3.53419885722625\\
-3.77536201908882	-3.74490148296817\\
-4.23270744586419	-3.86445431115576\\
-4.56488181034869	-3.92906223056567\\
-4.81111173084384	-3.96087712458478\\
-4.99782490940931	-3.97300961597196\\
-5.14260703237204	-3.97319579578444\\
-5.25724330095103	-3.96604388570524\\
-5.34975483269134	-3.95433565851878\\
-5.42570363473993	-3.93977138106006\\
-5.48902167942809	-3.92339911305248\\
-5.54254114935149	-3.90586575046038\\
-5.58833817113287	-3.88756639621381\\
-5.62795931121304	-3.86873478129887\\
-5.66257339361963	-3.84949884281418\\
-5.69307497569185	-3.82991527850158\\
-5.72015597962227	-3.8099911362125\\
-5.74435596365189	-3.78969721327136\\
-5.76609779204375	-3.76897613123127\\
-5.78571311862739	-3.74774681542691\\
-5.80346059509227	-3.72590641372493\\
-5.81953872845096	-3.70333024722295\\
-5.83409464438102	-3.6798700882352\\
-5.84722954031131	-3.65535084147328\\
-5.85900125249416	-3.62956551949584\\
-5.86942405656107	-3.60226822087717\\
-5.87846552319394	-3.57316460983039\\
-5.88603991148949	-3.54189912649133\\
-5.89199714349145	-3.50803778459862\\
-5.89610578039421	-3.47104487502588\\
-5.89802748328929	-3.43025109272021\\
-5.89727897404888	-3.38480938741684\\
-5.89317515025065	-3.33363295680044\\
-5.88474311936696	-3.27530684970659\\
-5.87059038716489	-3.20795996555619\\
-5.84869928000546	-3.1290767595092\\
-5.81610037050964	-3.03521604508351\\
-5.76834404664527	-2.92158566067602\\
-5.69863131208933	-2.78139436731751\\
-5.59636883937504	-2.60486868121386\\
-5.44477406956813	-2.37780660366194\\
-5.21703948257889	-2.07964643193155\\
-4.87085181653827	-1.68161072853984\\
-4.34317610222974	-1.1474913730259\\
-3.55463859109555	-0.444439617948892\\
-2.44646178577775	0.424728851617969\\
-1.06299018604471	1.37607382544733\\
0.400171324256592	2.25282459691835\\
1.70555026237845	2.92773748712853\\
2.73097404485681	3.37854084526397\\
3.48314877041734	3.65388972130317\\
4.02274125320431	3.81341724852021\\
4.41164031661734	3.90197566701296\\
4.69683755618252	3.94808845837674\\
4.91064267037425	3.96882631493131\\
5.07461298996009	3.9742534137051\\
};
\addplot [color=mycolor1, forget plot]
  table[row sep=crcr]{%
5.99541349123009	4.49732482894193\\
6.12198373456767	4.49346810979022\\
6.22232276431917	4.48400228275453\\
6.30345828935967	4.47110453976137\\
6.37022710499521	4.45608804678951\\
6.4260340040194	4.43975081575307\\
6.47332606041965	4.42257641347924\\
6.51389546554197	4.40485202361184\\
6.54907708213389	4.38673928808308\\
6.57987997837295	4.36831750635314\\
6.60707667311733	4.34961025530518\\
6.63126471753926	4.33060181272328\\
6.65290979885519	4.31124713321961\\
6.67237623661826	4.29147760548794\\
6.68994867800905	4.27120391812361\\
6.70584748329541	4.25031680783956\\
6.72023943094676	4.2286861073336\\
6.73324478583477	4.20615826222468\\
6.74494135221224	4.18255229293414\\
6.7553658016758	4.15765400075927\\
6.7645122701438	4.13120802717579\\
6.77232790850611	4.10290714059211\\
6.77870469585667	4.07237780676337\\
6.7834663114413	4.03916064288895\\
6.78634810483478	4.0026836765805\\
6.7869670294593	3.96222529387516\\
6.7847765162843	3.9168621418823\\
6.77899814289203	3.86539467577519\\
6.76851666415351	3.80623886471428\\
6.7517158001861	3.73726568768493\\
6.72621592257996	3.65555853467875\\
6.68844539018337	3.55703918497539\\
6.63292334288572	3.43588027230273\\
6.55103235852934	3.28356844963774\\
6.42888013525645	3.0874023791601\\
6.24355374627446	2.82812556715442\\
5.95672477318787	2.47646662632\\
5.50486682757952	1.98928488869036\\
4.78954650692568	1.31006109479641\\
3.68844777342875	0.389582967520768\\
2.13844260427236	-0.748782620037423\\
0.291673312658474	-1.93519244771774\\
-1.48840841691493	-2.92778854675826\\
-2.91487958197734	-3.60977739845915\\
-3.93942991406759	-4.02237595996972\\
-4.64665575647416	-4.2562875645141\\
-5.13582834199775	-4.38424753063528\\
-5.48132614055454	-4.45149222266906\\
-5.73208249096043	-4.48391637079128\\
-5.91920627616184	-4.49608929107534\\
-6.06253170237649	-4.49628172862736\\
-6.17493002609912	-4.4892744801395\\
-6.26494796344672	-4.47788510802943\\
-6.33839878081503	-4.46380206027429\\
-6.39932959310621	-4.44804860908601\\
-6.45061973895545	-4.43124671036897\\
-6.4943587403697	-4.41377052971911\\
-6.53209025416744	-4.3958376866492\\
-6.56497287364746	-4.37756447947498\\
-6.5938882314731	-4.35899976779666\\
-6.61951498853594	-4.34014589512689\\
-6.64238027283283	-4.32097153382069\\
-6.6628958968002	-4.30141933996902\\
-6.68138407285814	-4.28141013959875\\
-6.69809570437103	-4.26084466461084\\
-6.71322326879524	-4.23960341595104\\
-6.72690960312022	-4.21754493771128\\
-6.73925340887719	-4.19450257061826\\
-6.7503119251	-4.17027957233026\\
-6.76010090999522	-4.14464231222923\\
-6.76859177539645	-4.11731104005872\\
-6.77570538291145	-4.08794745695008\\
-6.78130157695893	-4.05613793861392\\
-6.7851629118082	-4.0213707060733\\
-6.7869700919149	-3.98300440271407\\
-6.78626516134385	-3.94022424412106\\
-6.78239605797812	-3.89197987008238\\
-6.77443209693362	-3.83689575601213\\
-6.76103300288552	-3.77313969088716\\
-6.74024193420233	-3.69822593956794\\
-6.7091511345993	-3.60871475364321\\
-6.66334908680843	-3.49974462133492\\
-6.59598477240038	-3.36429144842522\\
-6.49615044266738	-3.19198199938406\\
-6.34604973540744	-2.96719982612318\\
-6.11607156316345	-2.66617766747122\\
-5.7566940389327	-2.25311741435295\\
-5.18768700672088	-1.67742587824769\\
-4.2942798350915	-0.881332086703665\\
-2.966737669024	0.159264763485881\\
-1.23176837138047	1.35185575181467\\
0.630117465771918	2.46760048475806\\
2.25388472598717	3.30754348402831\\
3.47340789742028	3.84405640500079\\
4.32607619065736	4.15642304228316\\
4.91317670196399	4.33011777856789\\
5.32296814448572	4.42349646802813\\
5.61627328687972	4.47095329859258\\
5.83215503416641	4.49191078227295\\
5.99541349123009	4.49732482894193\\
};
\addplot [color=mycolor1, forget plot]
  table[row sep=crcr]{%
6.89736886606275	5.03319437558776\\
7.02416387175193	5.02933501139698\\
7.12411966905462	5.01990796379786\\
7.20458777221079	5.0071180347061\\
7.27057166197707	4.9922792439007\\
7.32556233563215	4.97618178754688\\
7.37205090732628	4.9592997824019\\
7.4118511675274	4.94191188332185\\
7.44630755102136	4.924172880178\\
7.47643228783712	4.90615695916026\\
7.50299772990644	4.88788414863349\\
7.52659964487822	4.86933651746032\\
7.54770128007571	4.85046794499134\\
7.56666440163349	4.8312097142689\\
7.58377130067985	4.81147325938556\\
7.59924036284171	4.79115083779699\\
7.61323689171352	4.77011454015385\\
7.6258802666304	4.74821380263283\\
7.63724808003991	4.72527139393899\\
7.64737756120027	4.70107767299109\\
7.65626429327812	4.67538272270993\\
7.66385792005916	4.64788572851158\\
7.67005416198436	4.61822064710494\\
7.67468194685035	4.58593674444558\\
7.67748369861844	4.55047188167056\\
7.67808563840317	4.51111534899351\\
7.67595302539693	4.46695534627469\\
7.6703220523474	4.41680346793679\\
7.66009460612644	4.35908403918405\\
7.64367242880821	4.2916685701503\\
7.61868976481584	4.21162259560055\\
7.58157130548474	4.11480947629323\\
7.52678116568926	3.99525565555512\\
7.44551109794936	3.84411120881695\\
7.32332970899458	3.64791936884771\\
7.13589448163364	3.38573012706076\\
6.84116389135927	3.02445608149658\\
6.36613618695305	2.51243804049203\\
5.58954013189169	1.77531831863544\\
4.34304282563732	0.73379839849817\\
2.50920940090667	-0.612411235566368\\
0.262259739171652	-2.05567829798883\\
-1.89433615075502	-3.25854162383184\\
-3.56922904333947	-4.05972382111383\\
-4.72545819139513	-4.5256232886552\\
-5.49608724978678	-4.7806438240618\\
-6.01489616881318	-4.91642316290733\\
-6.37409111441823	-4.98636678638294\\
-6.6309998647971	-5.01960357672328\\
-6.82064113847222	-5.03194962969862\\
-6.96470528987139	-5.03214847701271\\
-7.07696932594897	-5.0251529030354\\
-7.16643373595251	-5.01383567684412\\
-7.23914378704781	-4.99989606987022\\
-7.29926650130696	-4.98435252965248\\
-7.34974301291436	-4.96781787150778\\
-7.39269392434695	-4.95065710246054\\
-7.42967754762933	-4.93308011617047\\
-7.46185833587123	-4.91519724170708\\
-7.49011913873529	-4.89705303525015\\
-7.51513748514114	-4.87864698942531\\
-7.5374383030822	-4.85994615784343\\
-7.55743085815116	-4.84089262376764\\
-7.57543487892197	-4.82140754519059\\
-7.59169908559255	-4.80139279414265\\
-7.60641421853607	-4.78073076348686\\
-7.6197219241111	-4.75928262020958\\
-7.63172034425345	-4.73688506958777\\
-7.64246687771582	-4.71334551425335\\
-7.65197826816	-4.68843531256141\\
-7.66022787530608	-4.66188063126278\\
-7.66713964921773	-4.63335011340417\\
-7.67257789316159	-4.60243819627687\\
-7.67633127945071	-4.568642344819\\
-7.67808863581077	-4.53133159948159\\
-7.67740251220767	-4.48970248650128\\
-7.67363405872692	-4.44271618478672\\
-7.66586855333436	-4.38900733663324\\
-7.65278364435701	-4.32674905720569\\
-7.6324394206162	-4.25344879525755\\
-7.60193576333479	-4.16563255672918\\
-7.55683815140872	-4.05834487081539\\
-7.49018848824136	-3.92433857229832\\
-7.3907541462704	-3.75273558519102\\
-7.23985724583102	-3.52678910568387\\
-7.00557905309622	-3.22019061556202\\
-6.63244583047343	-2.79142193565118\\
-6.02533913185054	-2.17738838513611\\
-5.03589622488247	-1.29611144069466\\
-3.49852803761304	-0.0916384912261362\\
-1.4099049355907	1.34356442438593\\
0.860303057242737	2.70407592263365\\
2.80131087132802	3.70855023314097\\
4.20507350080996	4.32647318144255\\
5.14971166906231	4.6727291307529\\
5.78023129590174	4.85936587721496\\
6.21020475115491	4.95739022037701\\
6.51274555585935	5.00636499346055\\
6.7326384758569	5.02772444162467\\
6.89736886606275	5.03319437558776\\
};
\addplot [color=mycolor1, forget plot]
  table[row sep=crcr]{%
7.29591158957958	5.27513025289535\\
7.42332987221212	5.27125335301717\\
7.52358643507382	5.26179883012366\\
7.60417597996328	5.2489901777062\\
7.67018025597365	5.23414719531595\\
7.72513429683464	5.21806073965346\\
7.77155458382699	5.20120373405925\\
7.81126978887302	5.1838531464475\\
7.8456331896464	5.16616213164784\\
7.8756623093767	5.14820348838183\\
7.90213267362373	5.12999615354589\\
7.9256419455298	5.11152139069478\\
7.94665449572159	5.09273253083439\\
7.96553275207919	5.07356053360328\\
7.98255940086198	5.05391670628169\\
7.99795308180025	5.03369335345425\\
8.01187929539751	5.01276276994595\\
8.02445761963065	4.99097474117082\\
8.03576589188746	4.96815252170294\\
8.04584166962416	4.9440870861721\\
8.05468098115947	4.91852925510848\\
8.06223406538229	4.8911790599331\\
8.06839742085364	4.86167138534677\\
8.07300096778685	4.8295564552579\\
8.07578835965867	4.79427301852752\\
8.07638728205741	4.75511099372161\\
8.07426462963591	4.71115859635049\\
8.06865819541017	4.66122616456551\\
8.05847090822105	4.60373425702524\\
8.04210376970811	4.5365457484206\\
8.01718571494063	4.4567080860975\\
7.98012522319577	4.3600479551804\\
7.92534468945407	4.24051775681551\\
7.84393410430332	4.08911620168565\\
7.72121595557031	3.89206964324699\\
7.53224428080095	3.62774426898471\\
7.23346459228597	3.26153245470666\\
6.74802758508566	2.73834698777122\\
5.94514374297721	1.97638429548909\\
4.63617708355683	0.882869012273487\\
2.67725454219386	-0.554912272002079\\
0.250106303551813	-2.11382512863947\\
-2.07544543435575	-3.41106365837185\\
-3.85922221824205	-4.26450734222697\\
-5.07193654273235	-4.7532753562224\\
-5.8697099806853	-5.01733122425154\\
-6.40155209947301	-5.15654607646189\\
-6.76717349051142	-5.22775289084373\\
-7.02734471168533	-5.26141775845063\\
-7.21867447336467	-5.2738769789914\\
-7.36361282690222	-5.27407889076835\\
-7.47631503422351	-5.26705713646384\\
-7.56597768604665	-5.25571554687432\\
-7.63875149643691	-5.24176419093408\\
-7.6988620035609	-5.22622413513902\\
-7.74928367714008	-5.2097076762349\\
-7.7921565112804	-5.19257827760396\\
-7.82905027501155	-5.1750441323721\\
-7.86113625917744	-5.15721404607183\\
-7.88930141267163	-5.13913133431034\\
-7.91422571902979	-5.12079454432989\\
-7.93643556700714	-5.10217005697309\\
-7.95634108450513	-5.08319952348469\\
-7.97426250931867	-5.06380387995354\\
-7.99044887366695	-5.0438849610277\\
-8.00509113481198	-5.02332528700023\\
-8.01833113061329	-5.00198630280175\\
-8.03026721982775	-4.97970513216102\\
-8.04095708344062	-4.9562897293411\\
-8.05041784766114	-4.93151213056845\\
-8.05862338820194	-4.90509929667592\\
-8.065498337632	-4.87672076222637\\
-8.07090788086842	-4.84597191630921\\
-8.07464179934598	-4.81235116353054\\
-8.07639027069993	-4.77522833395339\\
-8.07570740910741	-4.73380033467176\\
-8.07195602234769	-4.68702783425895\\
-8.06422280598895	-4.63354317114868\\
-8.05118578101769	-4.57151365771521\\
-8.03090251436506	-4.49843416256824\\
-8.00046327374921	-4.410804886061\\
-7.95540723953655	-4.30361828179464\\
-7.88871095532438	-4.16952157079232\\
-7.78898364202658	-3.99741843465297\\
-7.63716302068276	-3.77009842557475\\
-7.40037751408439	-3.46023676319357\\
-7.02074134537897	-3.02403193890216\\
-6.39703409109933	-2.39328418225762\\
-5.36654591193065	-1.47560069962743\\
-3.73799675272195	-0.199935111082237\\
-1.49146927378901	1.3435635047952\\
0.962926008148577	2.81449051151017\\
3.04480000395319	3.8920465842516\\
4.52832215317503	4.54522389310579\\
5.5122497122189	4.90595764984086\\
6.16153292543111	5.09818439883339\\
6.6006229495706	5.19830404295129\\
6.90772781078717	5.24802599683915\\
7.12996437498889	5.26961747647854\\
7.29591158957958	5.27513025289535\\
};
\addplot [color=mycolor1, forget plot]
  table[row sep=crcr]{%
6.81839938705513	4.98558555137348\\
6.9451056567848	4.98172857837679\\
7.04503269745632	4.97230404998413\\
7.12550382874078	4.95951351377152\\
7.19150742751882	4.9446702051191\\
7.24652621591938	4.92856445823852\\
7.29304670485787	4.91167081831979\\
7.33288010703728	4.89426840679286\\
7.36736941959477	4.87651242497647\\
7.39752608589151	4.8584773879902\\
7.42412204545158	4.84018356931731\\
7.44775286792895	4.82161320692916\\
7.46888172297768	4.802720283049\\
7.48787036298415	4.78343612628583\\
7.50500109471872	4.76367216612502\\
7.52049232706941	4.74332061039937\\
7.53450938011309	4.72225345843491\\
7.54717163259053	4.70032001509007\\
7.5585566510167	4.67734287814631\\
7.56870160579294	4.65311219540491\\
7.57760198044974	4.62737779738012\\
7.5852072695652	4.59983857492861\\
7.59141298466069	4.57012814868072\\
7.59604777339064	4.53779541139837\\
7.59885369623389	4.5022778262537\\
7.59945651723111	4.46286428877193\\
7.59732094276784	4.41864266581239\\
7.59168253759535	4.36842439739232\\
7.58144256176967	4.31063406338187\\
7.56500233907657	4.243144292049\\
7.5399964148077	4.16302350528493\\
7.50285171749803	4.0661415588715\\
7.44803942928042	3.9465388293111\\
7.36677017908548	3.79539498589725\\
7.24466050473369	3.59931675028097\\
7.0574886748104	3.33749314960869\\
6.76352025145179	2.97714793958209\\
6.29053702849613	2.46732264221061\\
5.51919290384651	1.7351657602429\\
4.28518881238173	0.704044860696217\\
2.47617548846006	-0.62399326502574\\
0.26472488379459	-2.04447711678708\\
-1.85854763216566	-3.22872807539079\\
-3.51179701572037	-4.01952314056496\\
-4.65672460540954	-4.48084729701715\\
-5.42192597149296	-4.73406098849309\\
-5.93815571883776	-4.86916023516302\\
-6.29610684807415	-4.93885917516955\\
-6.55240778900606	-4.97201605639051\\
-6.74175384000817	-4.98434219551015\\
-6.88568139038962	-4.98454045434875\\
-6.99789141017824	-4.97754800336823\\
-7.08734546190676	-4.96623193256517\\
-7.16006824322224	-4.95228978176482\\
-7.22021562299617	-4.93673979314651\\
-7.27072257215752	-4.92019511301591\\
-7.31370624925992	-4.90302121434563\\
-7.3507230387261	-4.88542843600515\\
-7.3829363263936	-4.86752747846997\\
-7.41122838809586	-4.84936318429306\\
-7.43627646256458	-4.8309352520156\\
-7.45860534888868	-4.81221087003346\\
-7.47862427236261	-4.79313219481558\\
-7.49665296586647	-4.77362040406707\\
-7.51294017121452	-4.75357734217863\\
-7.52767665013585	-4.73288533123189\\
-7.54100405805473	-4.71140542676794\\
-7.55302052460794	-4.68897418296435\\
-7.56378340709999	-4.66539881155098\\
-7.57330937097375	-4.64045043926517\\
-7.58157165270455	-4.61385495942758\\
-7.58849402465152	-4.58528069952081\\
-7.5939405470587	-4.55432174128212\\
-7.59769957191291	-4.52047516178699\\
-7.59945951764889	-4.48310960021399\\
-7.59877242863057	-4.44142120888268\\
-7.59499885959066	-4.39437090303387\\
-7.58722344586879	-4.34059333582325\\
-7.5741232741605	-4.27826223117822\\
-7.55375828026879	-4.20488688476054\\
-7.52322939077674	-4.11699767437352\\
-7.4781062033147	-4.00964867067367\\
-7.4114422958755	-3.87561300911386\\
-7.31203520475341	-3.70405586877619\\
-7.16128309678768	-3.47832412658061\\
-6.9274595685231	-3.17231678512378\\
-6.55558188850301	-2.7449831897496\\
-5.95177163129806	-2.13426805552596\\
-4.97054626054604	-1.26027984798751\\
-3.45135195503113	-0.0699980423053237\\
-1.39392514759134	1.34380755585939\\
0.840019649925826	2.68258056595531\\
2.75314581466553	3.67259137928405\\
4.14099226701179	4.28347990938291\\
5.07776315797388	4.62683679623043\\
5.70454533578041	4.81235986820016\\
6.13273164742407	4.90997324701889\\
6.4344036995944	4.95880562065331\\
6.65387188453432	4.98012287766274\\
6.81839938705513	4.98558555137348\\
};
\addplot [color=mycolor1, forget plot]
  table[row sep=crcr]{%
5.81106869281575	4.39030195368767\\
5.9378306742482	4.38643831889657\\
6.03846583405338	4.37694387713713\\
6.1199335771122	4.36399287535022\\
6.1870374509301	4.34890071949983\\
6.24316659297427	4.33246893432805\\
6.29076134204276	4.31518444620006\\
6.33161167052361	4.29733720124044\\
6.36705246181677	4.27909093792948\\
6.39809385319555	4.26052644680076\\
6.42550985478205	4.24166828271736\\
6.44989960323473	4.22250127616143\\
6.47173029631398	4.20298058067738\\
6.49136760427196	4.18303748327818\\
6.50909732396471	4.16258230635481\\
6.52514074436214	4.14150517702381\\
6.53966534011105	4.11967508309843\\
6.55279182890572	4.09693738654155\\
6.56459820933533	4.07310977119143\\
6.57512106561025	4.04797642449512\\
6.58435412975198	4.02128006254521\\
6.59224378227706	3.99271117302261\\
6.59868079617576	3.96189353339659\\
6.60348711570957	3.92836460753721\\
6.60639570497698	3.89154874944605\\
6.60702032865372	3.85072011495395\\
6.60481024568603	3.80495058240551\\
6.59898169346763	3.75303544587264\\
6.58841279772551	3.69338554863195\\
6.57147948699918	3.62386780739903\\
6.54579401358252	3.54156491815036\\
6.50777897262767	3.44240636935973\\
6.45195745624496	3.32059186998236\\
6.36974484521994	3.16767858486497\\
6.24735944954143	2.97113266238818\\
6.06220161960239	2.7120821831487\\
5.77677395539875	2.3621237276036\\
5.32965500631877	1.88001801110771\\
4.62732032063472	1.21306234745563\\
3.55676427257078	0.318016299685843\\
2.06476136417372	-0.777861068382086\\
0.298062509953041	-1.91287086600401\\
-1.40643183116124	-2.86326345521889\\
-2.78185360802044	-3.52076771016897\\
-3.77869138961688	-3.9221517335401\\
-4.47246549395942	-4.15158450527908\\
-4.95546408019367	-4.27791426206472\\
-5.2982738775063	-4.34462797714021\\
-5.54798757103047	-4.37691313728978\\
-5.73484389100408	-4.38906632495553\\
-5.87826329687165	-4.38925751520342\\
-5.99091736913433	-4.38223347418573\\
-6.08125526720893	-4.37080307140861\\
-6.15504250345002	-4.35665515142711\\
-6.21630322685912	-4.34081614461725\\
-6.26790632881065	-4.32391153902328\\
-6.31193723841499	-4.3063185848153\\
-6.34993871357164	-4.28825732814437\\
-6.38306999766929	-4.26984585027003\\
-6.41221405884161	-4.25113423359834\\
-6.43805112511992	-4.23212557690578\\
-6.46110988715076	-4.21278892058008\\
-6.48180359518745	-4.19306696380742\\
-6.5004557152727	-4.17288029514566\\
-6.51731819170392	-4.15212915680827\\
-6.53258431563611	-4.13069332223909\\
-6.54639750008312	-4.10843037237793\\
-6.55885677247197	-4.08517244024534\\
-6.5700194288572	-4.06072131195114\\
-6.5799009870112	-4.03484159215411\\
-6.58847227898074	-4.00725143358615\\
-6.59565318817315	-3.97761005983719\\
-6.60130210178703	-3.94550093310744\\
-6.60519953099088	-3.91040886721914\\
-6.60702341415294	-3.87168855594162\\
-6.60631213863073	-3.82852070749267\\
-6.60240890778542	-3.77984996581519\\
-6.5943770580994	-3.72429558219703\\
-6.58086905971169	-3.66002056526266\\
-6.55991993495031	-3.58453639081319\\
-6.52861445599441	-3.49440592415262\\
-6.48253879024475	-3.38478309582889\\
-6.41485667012296	-3.24868830269853\\
-6.31472279447289	-3.07585768979424\\
-6.16452915602238	-2.8509292437166\\
-5.93517781718945	-2.55071472514717\\
-5.57847827374839	-2.14070828646252\\
-5.01746429854773	-1.57305770883352\\
-4.14439668688545	-0.795006619829821\\
-2.86036399139633	0.211598326002595\\
-1.19674002935348	1.3552324973526\\
0.583584041481237	2.42208766959095\\
2.14300486862341	3.22866929909522\\
3.3241693684594	3.74823849458138\\
4.15735160444676	4.05342574908404\\
4.73528772043855	4.22438788118071\\
5.14097069607059	4.31681952704536\\
5.43256333691566	4.3639936027741\\
5.6478623113483	4.38489141310636\\
5.81106869281575	4.39030195368767\\
};
\addplot [color=mycolor1, forget plot]
  table[row sep=crcr]{%
4.79211818889952	3.82077342471787\\
4.92177086224927	3.81681362251261\\
5.02579382855331	3.80699436771221\\
5.11071934761554	3.79349020396659\\
5.18115322605272	3.77764668970029\\
5.24040110153725	3.76030015906999\\
5.29087648834215	3.74196825006895\\
5.334370015191	3.72296522291877\\
5.37222975797709	3.70347281736717\\
5.40548392042828	3.68358434168994\\
5.43492556274751	3.66333232196307\\
5.46117193263937	3.64270583274064\\
5.48470650120408	3.62166118565911\\
5.50590899835184	3.60012820367493\\
5.52507694347959	3.57801342936336\\
5.54244099201972	3.55520106522052\\
5.55817563103771	3.53155208318496\\
5.57240620970301	3.50690168662155\\
5.58521288765786	3.481055108402\\
5.59663176028572	3.453781546994\\
5.60665312406216	3.42480584883557\\
5.61521653142038	3.39379730983344\\
5.62220190131706	3.36035465530512\\
5.62741543027179	3.32398581588658\\
5.63056828640992	3.28408047206688\\
5.63124490194165	3.23987237622389\\
5.62885583532081	3.19038699505496\\
5.62256719023864	3.13436775439726\\
5.61119365519057	3.07017064517139\\
5.59303397095464	2.99561142824242\\
5.56561361540334	2.90774105071726\\
5.5252755607558	2.80251167617108\\
5.46651940192237	2.67427662899367\\
5.38092287164628	2.51504408972986\\
5.25538259842559	2.31339080669481\\
5.06931449840661	2.05299768168736\\
4.7905529760893	1.71109084870631\\
4.37072655265967	1.25819696264862\\
3.74488371386983	0.663512848732174\\
2.84933918050736	-0.0857440839287663\\
1.67583031501588	-0.948209188546892\\
0.336492051431472	-1.80883640449244\\
-0.962811281776992	-2.53305604756679\\
-2.05571937770858	-3.05513154500464\\
-2.89325465498746	-3.39208229568967\\
-3.50782571836996	-3.59514896413235\\
-3.95467324606576	-3.71192766817774\\
-4.28267967718778	-3.77570835930012\\
-4.52781894824707	-3.80737297697914\\
-4.71488412494648	-3.81952287066889\\
-4.86065518138394	-3.81970699278071\\
-4.97652334114615	-3.81247611664875\\
-5.07031940568377	-3.80060392687166\\
-5.14751628170736	-3.78579935958603\\
-5.2120073467488	-3.76912310516208\\
-5.26661129858299	-3.7512339630798\\
-5.31340301878347	-3.73253677799184\\
-5.35393349155295	-3.71327267409269\\
-5.38937825513544	-3.69357487338636\\
-5.42063916980341	-3.67350359035917\\
-5.44841520493515	-3.65306793974812\\
-5.47325231470367	-3.63223959432685\\
-5.49557894254391	-3.61096105269788\\
-5.51573145198247	-3.58915025213298\\
-5.53397233142163	-3.56670256895459\\
-5.55050306209345	-3.54349080645135\\
-5.5654728855542	-3.51936347099641\\
-5.57898424168559	-3.4941414155384\\
-5.59109529174929	-3.46761274293533\\
-5.60181963693514	-3.43952567750913\\
-5.61112304392667	-3.40957890264328\\
-5.61891664770849	-3.37740859296992\\
-5.6250456597327	-3.34257099954288\\
-5.62927198358847	-3.30451891450158\\
-5.63124820150209	-3.26256955533503\\
-5.63047893279206	-3.21586022250144\\
-5.62626322571677	-3.16328626637663\\
-5.6176078196875	-3.10341308010106\\
-5.60309475075204	-3.03434942572295\\
-5.58067602824568	-2.95356249238552\\
-5.54734979529474	-2.85760436722309\\
-5.49864116120367	-2.741703537316\\
-5.42775866109708	-2.59915315257045\\
-5.32421527535131	-2.42040586979525\\
-5.17159688773638	-2.19179337220612\\
-4.94411730786096	-1.89393757040694\\
-4.60201321517407	-1.50054537619279\\
-4.08799717243774	-0.980163705403888\\
-3.333539748464	-0.307359363168784\\
-2.29356464353244	0.508483010097138\\
-1.01486676146027	1.38789111961271\\
0.331085521457898	2.1943914998068\\
1.54082044438049	2.81975256886592\\
2.50549061640488	3.24374496810197\\
3.22499695397131	3.50706584023681\\
3.74886645174885	3.66190485532824\\
4.13098470470159	3.74889693772828\\
4.41383400597597	3.79461769413917\\
4.62740908872181	3.81532611953117\\
4.79211818889952	3.82077342471787\\
};
\addplot [color=mycolor1, forget plot]
  table[row sep=crcr]{%
3.95999323486845	3.39757254170735\\
4.09484250813319	3.39344321228675\\
4.2045297265259	3.38308210600781\\
4.29508387125841	3.36867800128559\\
4.37087714532624	3.35162544588445\\
4.43511993062576	3.33281395952028\\
4.49020046207001	3.31280761045635\\
4.53791827504927	3.29195743937878\\
4.57964576454014	3.27047251437929\\
4.61644106500604	3.24846521707147\\
4.64912774304262	3.22598024130648\\
4.67835162694857	3.20301311246613\\
4.70462167883089	3.17952181318685\\
4.72833955405471	3.15543373939353\\
4.74982098673809	3.13064935989219\\
4.76931112259044	3.10504340685299\\
4.78699521765181	3.07846405916893\\
4.80300561850361	3.05073031928915\\
4.81742555817475	3.02162757703464\\
4.83028998579573	2.99090116593392\\
4.84158335147927	2.95824752002809\\
4.85123394759627	2.92330230365497\\
4.8591040140495	2.88562458039237\\
4.86497428238551	2.84467566638736\\
4.86852086469722	2.79979071572883\\
4.86928123764474	2.75014022072918\\
4.86660428864334	2.69467734480257\\
4.85957658975848	2.63206514756602\\
4.84691261115477	2.56057504602007\\
4.82678946546552	2.47794396053811\\
4.79659542946469	2.38117226334323\\
4.75254376940376	2.26623814365281\\
4.68907718339365	2.12769852720028\\
4.59795460683918	1.95815040051454\\
4.46688732676326	1.74756400797157\\
4.27764254367676	1.48263691142847\\
4.00387072308215	1.14670604183995\\
3.61005153759665	0.721636574198892\\
3.05560765177234	0.194451774755332\\
2.31137241334251	-0.428642285968568\\
1.39083488530141	-1.10555625327907\\
0.373895154446771	-1.75913136062439\\
-0.617367064691563	-2.31148893173218\\
-1.48259799127861	-2.7245327663766\\
-2.18148886039909	-3.00546910573645\\
-2.72247859519357	-3.18406267255354\\
-3.1346724468484	-3.29168669210835\\
-3.44905165555013	-3.35275913373484\\
-3.69127710003747	-3.38401246048074\\
-3.88062555859937	-3.3962893159387\\
-4.03102237946802	-3.39646583053762\\
-4.15240860493939	-3.38888182308754\\
-4.25189304055679	-3.37628371252528\\
-4.33460233967622	-3.36041786217831\\
-4.40427697195088	-3.34239825578999\\
-4.46368138722626	-3.32293419830987\\
-4.51488543859349	-3.30247223687471\\
-4.55945837766816	-3.2812854333876\\
-4.59860372563644	-3.25953004009571\\
-4.63325399543124	-3.23728172962425\\
-4.66413790900644	-3.2145587924539\\
-4.69182854788939	-3.19133686150793\\
-4.71677809651419	-3.16755798725861\\
-4.7393429950025	-3.14313581339344\\
-4.75980208289272	-3.11795792443445\\
-4.7783694726755	-3.09188599300709\\
-4.7952033007698	-3.06475404861714\\
-4.81041106952656	-3.03636496063061\\
-4.8240519512485	-3.00648503446068\\
-4.83613612420751	-2.97483643062173\\
-4.84662090779453	-2.94108690350011\\
-4.85540311393893	-2.90483609045282\\
-4.8623065785354	-2.86559722460702\\
-4.86706320060209	-2.82277264472673\\
-4.86928487803478	-2.77562075794744\\
-4.86842229710538	-2.72321106614295\\
-4.86370430180861	-2.6643623334864\\
-4.85404804085803	-2.59755672352673\\
-4.83792446013161	-2.52081946986195\\
-4.81315470961862	-2.43154904828298\\
-4.77659879836072	-2.32627679894911\\
-4.72367605630488	-2.20032845352253\\
-4.64762643990108	-2.04735745318607\\
-4.53838863733007	-1.85873574998009\\
-4.38096984835562	-1.62286298531766\\
-4.1533355323116	-1.32469184340526\\
-3.8244975574566	-0.946371254917345\\
-3.35530815035159	-0.47108481288752\\
-2.7077897400668	0.106751713865016\\
-1.86973007851861	0.764620138831268\\
-0.887073673343907	1.44068914574646\\
0.132452543813392	2.05155563779595\\
1.06971589193757	2.5358237813157\\
1.85310797649221	2.87987586763252\\
2.47009835668971	3.10548017401548\\
2.94265469363239	3.24502618750772\\
3.30233078271589	3.32683325735637\\
3.57783719557537	3.37132200537939\\
3.79158168965869	3.39201965933028\\
3.95999323486845	3.39757254170735\\
};
\addplot [color=mycolor1, forget plot]
  table[row sep=crcr]{%
3.31964373338877	3.11680725665986\\
3.46059685762262	3.11247807026117\\
3.57708127662116	3.10146599881294\\
3.67451705414661	3.08596097142581\\
3.75696740880465	3.06740608789205\\
3.82749830257685	3.04674991857918\\
3.888442678605	3.02461109511954\\
3.94159191694989	3.001385629976\\
3.98833402720211	2.97731709974738\\
4.02975355906996	2.95254269648813\\
4.06670414816545	2.92712347062241\\
4.09986146259606	2.90106408349867\\
4.12976201336457	2.87432546960569\\
4.15683165226114	2.84683257984022\\
4.1814064231642	2.81847858065534\\
4.20374761128889	2.7891263560364\\
4.22405224219013	2.75860779629214\\
4.24245983997726	2.72672109262395\\
4.25905590451528	2.69322604436327\\
4.27387226537608	2.65783719413068\\
4.28688417700432	2.62021440850668\\
4.29800369589844	2.5799502930124\\
4.30706848081821	2.53655354267434\\
4.31382462055043	2.48942694947274\\
4.31790133406429	2.43783827225292\\
4.31877427644065	2.38088146767593\\
4.31571252742472	2.3174248153883\\
4.30770184951786	2.24604117926284\\
4.29333306429863	2.16491400132116\\
4.27063886204853	2.07171075141813\\
4.23685446381317	1.96341402908087\\
4.1880672114508	1.83610108349573\\
4.11870939430507	1.68466979149208\\
4.02084615921213	1.5025342124434\\
3.88324316513264	1.28138011224131\\
3.69033637688843	1.01122646403587\\
3.4216152440466	0.681343386361434\\
3.05276596181384	0.283009552357022\\
2.56107231208726	-0.184793005303621\\
1.9374093620001	-0.707248384379817\\
1.20187957948651	-1.24835241324659\\
0.40982998738374	-1.75746107812609\\
-0.365070108068133	-2.18915793181554\\
-1.06149298312255	-2.52144033937862\\
-1.64914225121043	-2.75749030492595\\
-2.12602172492128	-2.91479010698421\\
-2.50572442956788	-3.01384258513672\\
-2.80656397105716	-3.07222755005006\\
-3.04583480008297	-3.10306288904879\\
-3.2378125671698	-3.11548637709263\\
-3.39358276868888	-3.11565342567214\\
-3.5215221365268	-3.1076492923641\\
-3.62789922876217	-3.094170883302\\
-3.71740391909101	-3.07699615540657\\
-3.79356234576415	-3.05729574550448\\
-3.85904598931447	-3.03583683613959\\
-3.91589681262369	-3.01311602974166\\
-3.96568992832198	-2.98944609620787\\
-4.00965105193101	-2.96501280481785\\
-4.04874158413868	-2.93991225009947\\
-4.08372054805974	-2.91417532469041\\
-4.11518990057731	-2.88778359109191\\
-4.14362779027714	-2.86067927062269\\
-4.16941295659567	-2.83277108062239\\
-4.19284249089347	-2.80393700477406\\
-4.21414448452044	-2.77402464624158\\
-4.23348657969749	-2.74284950605136\\
-4.25098104927468	-2.71019129526367\\
-4.26668671068013	-2.67578819118601\\
-4.28060768645165	-2.63932875635692\\
-4.2926887203984	-2.60044102900901\\
-4.30280640314065	-2.55867803931154\\
-4.31075520152593	-2.51349867667858\\
-4.31622655132413	-2.46424239171588\\
-4.31877835652204	-2.4100956127698\\
-4.3177908838693	-2.35004692972493\\
-4.31240301138506	-2.28282697679007\\
-4.30141973612445	-2.20682747567492\\
-4.28317728538454	-2.11999210763607\\
-4.25534552110689	-2.01967006216038\\
-4.21463814624494	-1.90242229035895\\
-4.15639015745231	-1.76377354814919\\
-4.07395353425142	-1.59791747429286\\
-3.95787205311072	-1.39742388831325\\
-3.79486863777062	-1.15310163350004\\
-3.56691807683934	-0.854393647214405\\
-3.25126970548047	-0.491066383999079\\
-2.82334993058864	-0.057334736614125\\
-2.26532838218795	0.440940424271642\\
-1.58091915229243	0.978480466449083\\
-0.808389310351763	1.51014074160763\\
-0.0157820009785098	1.98502234731836\\
0.725768143862515	2.36801909836561\\
1.36956393520291	2.65058247727238\\
1.90079146402238	2.84467600208762\\
2.32689533909663	2.97039684409577\\
2.66486093238406	3.04719472764852\\
2.93292528781976	3.09043587806671\\
3.14697196975999	3.11113324325111\\
3.31964373338877	3.11680725665986\\
};
\addplot [color=mycolor1, forget plot]
  table[row sep=crcr]{%
2.8295327054462	2.9466008779749\\
2.97642610855213	2.94207499671692\\
3.09988970543158	2.93039295654103\\
3.20464856418559	2.91371517181489\\
3.29437580196887	2.89351712903362\\
3.37192808911391	2.87080038913761\\
3.43953494245618	2.84623802679657\\
3.4989453355168	2.82027388436162\\
3.55153916974506	2.79319002120546\\
3.59841129928026	2.76515244759292\\
3.64043465239194	2.73624202568855\\
3.67830760138359	2.70647516533049\\
3.71258947966441	2.6758173996928\\
3.74372712841474	2.64419188113415\\
3.77207456717094	2.61148412790016\\
3.79790728059861	2.5775438631922\\
3.82143215068827	2.54218444063185\\
3.84279369705492	2.50518009028548\\
3.86207698134672	2.46626100924553\\
3.87930725265861	2.42510613230112\\
3.89444612840362	2.38133322903582\\
3.90738378625799	2.3344857638394\\
3.91792624803269	2.28401570541228\\
3.92577631430403	2.22926116270718\\
3.93050598965004	2.16941733623002\\
3.93151722677788	2.10349879512476\\
3.92798638587754	2.03029053102864\\
3.91878579530399	1.94828466613264\\
3.90237305398243	1.855599309194\\
3.87663518959904	1.74987637235859\\
3.83867086386778	1.62815736097154\\
3.78449111775298	1.48674276649606\\
3.70862241492482	1.32105686703013\\
3.60361677607863	1.12557497438373\\
3.45953832854664	0.893939279584528\\
3.26365014721632	0.619506570978022\\
3.00082463472069	0.29671869953972\\
2.65561419860998	-0.0762714533761386\\
2.21706286502895	-0.493729627568006\\
1.68619671892557	-0.938656377115806\\
1.08285674531277	-1.38265831675536\\
0.445432578270779	-1.79241778806565\\
-0.179885110508808	-2.14072410941843\\
-0.754317926290556	-2.41469139112317\\
-1.25584181744111	-2.61602985214543\\
-1.67895470657398	-2.75549727534181\\
-2.02899033676413	-2.84673786287883\\
-2.31615813868755	-2.9024182014246\\
-2.55160205772157	-2.93272494003078\\
-2.74546895585324	-2.9452462777605\\
-2.90625184398168	-2.94540172336058\\
-3.0407635978437	-2.9369743970697\\
-3.15435598513096	-2.92257310831705\\
-3.25119590115092	-2.90398444924689\\
-3.33452168008777	-2.88242517854157\\
-3.40685557962428	-2.85871771507956\\
-3.47017090376185	-2.83341037990418\\
-3.52602004104249	-2.80685924361002\\
-3.57563130948749	-2.77928368215287\\
-3.6199817995631	-2.75080399443141\\
-3.65985205939119	-2.72146673256919\\
-3.69586712014066	-2.69126152679475\\
-3.72852722176749	-2.6601319168179\\
-3.7582307012757	-2.6279818414424\\
-3.78529081653201	-2.59467885057009\\
-3.80994775181099	-2.56005469350327\\
-3.83237664138479	-2.52390363935476\\
-3.85269211510605	-2.48597865435748\\
-3.87094958091437	-2.44598536461657\\
-3.88714318232659	-2.40357354660641\\
-3.9012000727569	-2.35832569061562\\
-3.91297029725293	-2.30974195490434\\
-3.92221112123775	-2.25722055102696\\
-3.92856403549743	-2.20003225407342\\
-3.93152181524005	-2.13728729900023\\
-3.9303818073913	-2.06789240167837\\
-3.92417992018739	-1.99049506249897\\
-3.91159742794709	-1.90341179254178\\
-3.89082955793276	-1.8045367757561\\
-3.85940100403848	-1.69122852929405\\
-3.81390986646538	-1.56017608957687\\
-3.74968089091811	-1.40725677726019\\
-3.66031887909048	-1.22742176886083\\
-3.53719150856458	-1.01469576302685\\
-3.3689732107208	-0.76246914035007\\
-3.14160337218006	-0.464400090228629\\
-2.83938580218212	-0.116368668956703\\
-2.44832025295945	0.280211398732743\\
-1.96241815215748	0.71430772072177\\
-1.39154278411818	1.16286197831489\\
-0.765517221975054	1.59379223854703\\
-0.1285047127005	1.97543774602584\\
0.475270323667959	2.28718535710219\\
1.01486948515288	2.52389873550162\\
1.47704271994541	2.69265356832579\\
1.86252585744045	2.80630441153345\\
2.17972734440079	2.87832260557126\\
2.43967596952195	2.92021193495773\\
2.65316087503536	2.9408255990419\\
2.8295327054462	2.9466008779749\\
};
\addplot [color=mycolor1, forget plot]
  table[row sep=crcr]{%
2.44984468888118	2.85779721664325\\
2.60196738533444	2.8530955341335\\
2.73201868937063	2.84077925552878\\
2.84399722427826	2.82294384498009\\
2.941130399199	2.80107235469133\\
3.026009095818	2.77620459012577\\
3.10071052533583	2.7490607249949\\
3.1669019408754	2.72012984881909\\
3.22592470970097	2.68973265485586\\
3.27886093326873	2.65806546150423\\
3.32658553570199	2.62523087231385\\
3.36980663514464	2.59125886400063\\
3.40909660163786	2.55612095481477\\
3.44491572902982	2.51973927917976\\
3.47763000266974	2.48199180044322\\
3.50752405799971	2.44271446448007\\
3.53481009681523	2.40170078042435\\
3.55963324491399	2.35869907193627\\
3.58207357934416	2.31340744425876\\
3.6021448055327	2.26546633725287\\
3.61978930212512	2.21444836608594\\
3.63486894898146	2.15984497665486\\
3.6471507808511	2.10104925358721\\
3.65628602800719	2.03733401243435\\
3.66178046940091	1.96782409416737\\
3.66295318110665	1.89146159371687\\
3.65887966452276	1.80696267972253\\
3.64831397236724	1.71276488449446\\
3.62958291927769	1.60696463932235\\
3.60044417174799	1.48724713914983\\
3.55790006148047	1.35081571892447\\
3.49796295032051	1.19433822014812\\
3.41538138482406	1.01394710682261\\
3.3033694664129	0.805363083825396\\
3.15345104039838	0.564260916986537\\
2.95565145013395	0.287049461849504\\
2.69942978217378	-0.0277551991169861\\
2.37583077534834	-0.377545202656914\\
1.98105202249717	-0.75349516448623\\
1.52061048003519	-1.13953849261922\\
1.01179793653135	-1.51406798747553\\
0.481729687576415	-1.85483926486565\\
-0.039299878752971	-2.14502112512604\\
-0.525728760766699	-2.37694655936211\\
-0.961583347612355	-2.55184349143257\\
-1.34080776834139	-2.67677340867116\\
-1.66466717479823	-2.76113363997588\\
-1.93852205206722	-2.81418918953832\\
-2.16930815024689	-2.84386430177315\\
-2.36401462859855	-2.85641646336314\\
-2.52894863991	-2.8565588079033\\
-2.66948238030248	-2.8477415602283\\
-2.7900488682189	-2.83244662824804\\
-2.89424415876269	-2.81243887826061\\
-2.98496100689268	-2.78896168880488\\
-3.06451923449325	-2.76288202714237\\
-3.13477937039516	-2.7347952654056\\
-3.19723637238164	-2.70509977050788\\
-3.25309462510342	-2.67404948529879\\
-3.30332692445318	-2.64179071027718\\
-3.34872037797878	-2.60838758321233\\
-3.38991184884979	-2.5738394337785\\
-3.42741510892757	-2.53809221853144\\
-3.46164139897658	-2.50104554062317\\
-3.49291467668763	-2.46255625454029\\
-3.52148247681545	-2.42243928901094\\
-3.54752300405808	-2.3804660457283\\
-3.57114881240569	-2.33636051412372\\
-3.59240717557897	-2.2897930581589\\
-3.61127700091922	-2.24037166112528\\
-3.62766186012688	-2.18763024423159\\
-3.64137837682092	-2.13101349361061\\
-3.65213878904816	-2.06985743237638\\
-3.65952595267066	-2.00336476214815\\
-3.66295831902879	-1.93057379188937\\
-3.66164145467123	-1.85031962616379\\
-3.65450143563783	-1.76118632626276\\
-3.64009397438109	-1.661449251915\\
-3.61648164448542	-1.54900826961948\\
-3.58107074079176	-1.42131599087942\\
-3.53040086531508	-1.27531251851639\\
-3.45988808475131	-1.10739240636343\\
-3.36354413203345	-0.913455098139927\\
-3.23374299194595	-0.689131260121456\\
-3.06120008839267	-0.430331666141548\\
-2.83547558989729	-0.134305568090806\\
-2.54646084629997	0.198661406724477\\
-2.18725706829663	0.563087783868987\\
-1.75823788126407	0.946519661247175\\
-1.2707262246671	1.329691649195\\
-0.747503230270968	1.68991502715918\\
-0.218296874824636	2.0069639827157\\
0.288086702015426	2.26836939515059\\
0.750566434261	2.47117499192935\\
1.15832749650522	2.61998617758171\\
1.50938451881729	2.72342291650602\\
1.80742617457819	2.7910408610157\\
2.05885144943164	2.83151927338772\\
2.27075651633564	2.85195298808745\\
2.44984468888117	2.85779721664325\\
};
\addplot [color=mycolor1, forget plot]
  table[row sep=crcr]{%
2.15083773120232	2.82869718267466\\
2.30732436468486	2.82384624621737\\
2.4433118420125	2.81095668645537\\
2.56209976980254	2.79202803374775\\
2.66645262443518	2.76852402002715\\
2.75866143124573	2.74150316040845\\
2.84061379406736	2.71172005533557\\
2.91386069740201	2.67970163413908\\
2.97967530357787	2.64580343658018\\
3.03910245765468	2.61025057614152\\
3.09299924612863	2.57316716626594\\
3.14206760332443	2.53459711237977\\
3.18688011313304	2.49451841871177\\
3.22790007742325	2.45285255800203\\
3.26549675106933	2.40946998904446\\
3.29995644163108	2.36419255373392\\
3.33148996622944	2.31679321381851\\
3.36023675623487	2.26699337395324\\
3.38626569865824	2.21445786304212\\
3.40957259224826	2.15878749628598\\
3.43007386289986	2.09950900667044\\
3.44759591049805	2.03606201314232\\
3.46185912854562	1.96778258734543\\
3.47245522847033	1.89388290774894\\
3.47881599515294	1.81342648696486\\
3.48017099572729	1.72529860068735\\
3.47549109541112	1.6281719775772\\
3.46341402797019	1.52046878617669\\
3.44214804512423	1.40032192500813\\
3.40935054063824	1.26554232467367\\
3.3619819544403	1.11360553644566\\
3.29614389481357	0.941681805088982\\
3.2069287312535	0.746750454613378\\
3.08834196896464	0.525861154298206\\
2.93341318106459	0.276624430611918\\
2.7346792123074	-0.00199005653599627\\
2.48526439126132	-0.308540843598514\\
2.18069789166611	-0.637876894190381\\
1.82126405509475	-0.980283585224052\\
1.41407042577293	-1.32177794963178\\
0.973529098553298	-1.64611233998187\\
0.519302179515138	-1.93814165635475\\
0.0721749089467226	-2.18714232225843\\
-0.350271619857278	-2.38851645064079\\
-0.736296205869651	-2.54336468492444\\
-1.08035082362415	-2.65665726702385\\
-1.3818401671288	-2.73514641622193\\
-1.64334934794763	-2.7857745520406\\
-1.86907351690406	-2.81477095858355\\
-2.0637212393411	-2.82729792441939\\
-2.23187175557143	-2.82742663992165\\
-2.37766114657535	-2.81826702500692\\
-2.50467106829572	-2.80214488610198\\
-2.61592711340404	-2.78077361439386\\
-2.71394841871074	-2.75539991373785\\
-2.80081557583602	-2.72691936427101\\
-2.87823996979546	-2.6959645656945\\
-2.94762690053014	-2.66297077414461\\
-3.01012976176274	-2.62822399478868\\
-3.06669497313819	-2.59189576090128\\
-3.11809842125759	-2.55406793083113\\
-3.16497452169875	-2.51475000986999\\
-3.20783902965818	-2.47389082687288\\
-3.24710659152563	-2.43138586657657\\
-3.28310383789386	-2.38708115401469\\
-3.31607861296981	-2.340774278353\\
-3.3462057315549	-2.29221290360805\\
-3.37358945379951	-2.2410909216505\\
-3.39826266307694	-2.18704224225232\\
-3.4201825120606	-2.12963207414236\\
-3.43922205140252	-2.06834542350988\\
-3.45515705631335	-2.00257242185593\\
-3.46764689873355	-1.93159000279873\\
-3.4762078570285	-1.85453940282288\\
-3.48017669902106	-1.7703990179856\\
-3.47866172938115	-1.6779524117519\\
-3.47047783053574	-1.5757519309959\\
-3.45406155302938	-1.4620797890869\\
-3.42736251064461	-1.33491119630422\\
-3.3877092317964	-1.19188908423251\\
-3.33165320346531	-1.03032851580001\\
-3.2548076381554	-0.847282536978453\\
-3.15172288291488	-0.639720740842309\\
-3.01588443066694	-0.404894147748275\\
-2.8399829976851	-0.140972001347189\\
-2.61666888523843	0.151994797623094\\
-2.33999732459845	0.470856797629973\\
-2.00757182915667	0.808234985349818\\
-1.62289674730295	1.15214255424824\\
-1.1968205700236	1.48710610175714\\
-0.746798289601564	1.7969696755341\\
-0.293634844758991	2.06845614829498\\
0.143018691654359	2.29382996891854\\
0.548319090065009	2.47151065383526\\
0.913702541845487	2.60480312130623\\
1.23630864336571	2.69980902609968\\
1.51735250612403	2.76353078512541\\
1.76039282780355	2.80262768372586\\
1.9699877541269	2.82281419935843\\
2.15083773120232	2.82869718267466\\
};
\addplot [color=mycolor1, forget plot]
  table[row sep=crcr]{%
1.91158019794265	2.84376903183464\\
2.07163776438853	2.83879386724661\\
2.21285807640053	2.82539747440454\\
2.33791246105122	2.80546152865478\\
2.44912082680197	2.7804062830365\\
2.54846600359532	2.75128830803495\\
2.63762577791934	2.71888098047477\\
2.71801084845853	2.68373809671895\\
2.79080242003554	2.64624280460785\\
2.85698641658264	2.60664449345616\\
2.91738312264264	2.5650861096821\\
2.9726720213868	2.52162396989751\\
3.02341205008208	2.47624170916187\\
3.07005766382338	2.42885960651774\\
3.11297111497144	2.37934019668158\\
3.15243129157618	2.32749080597833\\
3.18863935121423	2.27306343349177\\
3.22172125493937	2.21575222433704\\
3.25172715531842	2.15518864183414\\
3.27862742163927	2.0909343333344\\
3.30230488909748	2.02247159947225\\
3.32254269007925	1.94919132453424\\
3.33900675849467	1.87037822283741\\
3.35122179204887	1.78519333555837\\
3.35853912660402	1.69265393427859\\
3.3600946677081	1.59161145435502\\
3.35475484794567	1.48072896213269\\
3.34104876892245	1.35846122100487\\
3.31708569791958	1.22304305571737\\
3.28045974439537	1.07249594527396\\
3.22814921378437	0.904669156349769\\
3.15642885792177	0.717340480465908\\
3.06083139967709	0.508411707762838\\
2.93622152863079	0.276241139701024\\
2.7770770319997	0.020149459472978\\
2.57809096843142	-0.258902162895629\\
2.3351780468082	-0.557553080740128\\
2.04683497078149	-0.869439194448154\\
1.71554714297952	-1.18511784885378\\
1.34865220465666	-1.49288119946767\\
0.958021807609145	-1.78050890078937\\
0.558338454421943	-2.03748111201489\\
0.164481555358333	-2.25680166315308\\
-0.21095810717441	-2.43573832577295\\
-0.559170187615061	-2.57538181366409\\
-0.875387966569343	-2.67947102389358\\
-1.15825539247116	-2.75307858502423\\
-1.40881879363976	-2.80155900902423\\
-1.62954531860348	-2.82988991803198\\
-1.82356115063245	-2.84235727819136\\
-1.99414405767677	-2.8424727142264\\
-2.14442741054747	-2.8330186505085\\
-2.27725282196893	-2.81614859152382\\
-2.39511589422416	-2.79350028839508\\
-2.50016473838548	-2.76630098678658\\
-2.59422508977545	-2.73545669331649\\
-2.6788363688267	-2.70162402151971\\
-2.75528999541142	-2.66526614123614\\
-2.82466552950899	-2.62669536187234\\
-2.88786267206039	-2.58610494914208\\
-2.94562849673728	-2.54359245794157\\
-2.99857995368833	-2.49917643438195\\
-3.04722197671781	-2.45280791977808\\
-3.09196160662951	-2.404377823958\\
-3.13311851320763	-2.35372093401608\\
-3.17093220955621	-2.30061708200279\\
-3.20556613189311	-2.24478980073838\\
-3.23710861645723	-2.18590264080907\\
-3.26557064483797	-2.12355319619622\\
-3.29088004627111	-2.05726478724901\\
-3.31287163385656	-1.98647567989964\\
-3.33127250427905	-1.91052568967319\\
-3.34568144341957	-1.82864005190303\\
-3.35554105860475	-1.73991058053854\\
-3.3601009295058	-1.64327446592406\\
-3.35836980714755	-1.53749170939389\\
-3.34905485717965	-1.42112337286191\\
-3.33048646848317	-1.29251486138864\\
-3.30052884841376	-1.1497918177212\\
-3.25648057704918	-0.990881452334094\\
-3.19497719768095	-0.813579719624154\\
-3.11192207726117	-0.615694432781554\\
-3.00249429743133	-0.395303860048757\\
-2.8613125677913	-0.151172713601745\\
-2.68286279053364	0.11665152416492\\
-2.46229728022884	0.40610058150193\\
-2.19663792258006	0.712364312934084\\
-1.88621757371078	1.02749962503193\\
-1.53590108495125	1.34076629312927\\
-1.15542361646115	1.63993480818183\\
-0.758353994407265	1.91336173347405\\
-0.359817457823202	2.15211845573736\\
0.0261883777012717	2.35132769387831\\
0.38885595919089	2.51028408905373\\
0.721431866190724	2.63157073461392\\
1.02096906501805	2.71974731916771\\
1.28744399189816	2.7801346433684\\
1.52272459753807	2.81795725135526\\
1.72968641530442	2.83786903536155\\
1.91158019794265	2.84376903183464\\
};
\addplot [color=mycolor1, forget plot]
  table[row sep=crcr]{%
1.71745758679747	2.89173975132996\\
1.8804644346493	2.88666054980203\\
2.02627889269441	2.87281811389132\\
2.15703696678287	2.85196438812595\\
2.27465720289688	2.82545742453582\\
2.38082766665929	2.79433294485042\\
2.47701312392245	2.75936681771373\\
2.56447232708102	2.72112681113683\\
2.64427925055575	2.68001399481002\\
2.71734473025739	2.63629501379885\\
2.78443660076244	2.59012667981921\\
2.84619740298685	2.54157425144665\\
2.90315928776839	2.49062457766663\\
2.95575602681665	2.43719505069377\\
3.00433216544909	2.38113909602147\\
3.04914937549984	2.32224873683445\\
3.09039002973783	2.26025461070945\\
3.12815794227413	2.19482368873312\\
3.16247611357899	2.12555485016604\\
3.19328118862104	2.05197240139194\\
3.22041418435718	1.97351760285347\\
3.24360687108098	1.88953829664735\\
3.26246300883696	1.79927683719634\\
3.27643346532593	1.70185676287416\\
3.28478411983947	1.59626907917125\\
3.28655547768413	1.48135976265904\\
3.28051324744219	1.35582129585754\\
3.26509006168158	1.21819291246397\\
3.23832053313869	1.06687700254827\\
3.1977756659706	0.900182964836113\\
3.14050925870732	0.716414564032025\\
3.06303931277295	0.514021630017732\\
2.96140177152125	0.291839165060381\\
2.831329743771	0.049431359624268\\
2.66862073969067	-0.212463672502666\\
2.46974101058715	-0.491439304059168\\
2.23265724939228	-0.782998939691912\\
1.95776837223382	-1.08040459012058\\
1.64866160836398	-1.37500967383526\\
1.31233089204491	-1.65718103512082\\
0.958592565229468	-1.91767088860956\\
0.598741748159326	-2.14903977073193\\
0.243852707791889	-2.34665123377138\\
-0.0967147415893769	-2.50894703232191\\
-0.416191996672758	-2.63704100182664\\
-0.710574296148952	-2.73391547893574\\
-0.978269118005141	-2.80354919199455\\
-1.2194938979658	-2.85019993873972\\
-1.43564881129259	-2.87792454612846\\
-1.62879099653703	-2.89031939215795\\
-1.80125105253692	-2.8904225079943\\
-1.95538246568011	-2.88071512412685\\
-2.09341468931788	-2.86317442696131\\
-2.21737842377636	-2.839346057336\\
-2.32907704654099	-2.81041845139936\\
-2.43008526104953	-2.77729025756574\\
-2.52176231073087	-2.74062749478147\\
-2.60527181932884	-2.70091001046648\\
-2.68160355032981	-2.65846814186684\\
-2.75159446004058	-2.61351096804578\\
-2.81594769357519	-2.56614758502651\\
-2.87524891462383	-2.51640268507215\\
-2.92997976318596	-2.46422750077732\\
-2.98052842983881	-2.40950694815937\\
-3.02719740245919	-2.35206359744722\\
-3.07020843125574	-2.29165892510769\\
-3.10970469913453	-2.22799215727647\\
-3.14575009217691	-2.16069690234515\\
-3.17832534667139	-2.08933568928717\\
-3.20732070773014	-2.01339248233482\\
-3.23252457198269	-1.93226324211468\\
-3.25360740779271	-1.84524466839372\\
-3.27010006373279	-1.75152142513656\\
-3.28136541998481	-1.65015247165899\\
-3.2865622713529	-1.54005769300379\\
-3.28460047717549	-1.42000696964363\\
-3.27408699466891	-1.28861533321583\\
-3.25326380716356	-1.14435014645906\\
-3.21994156970062	-0.985559539435554\\
-3.17143788509773	-0.81053568915016\\
-3.1045375211571	-0.61763149825431\\
-3.01550430731523	-0.405453196239712\\
-2.90019008781384	-0.173150576686815\\
-2.7543001876026	0.0791858516391993\\
-2.57387541216259	0.350044816530449\\
-2.35601728856905	0.636015792033466\\
-2.09979503696684	0.931474644994217\\
-1.80713233468106	1.22865077950936\\
-1.48333918421578	1.51825424663493\\
-1.13694684041011	1.79065828945332\\
-0.778714076415862	2.03735792720416\\
-0.42003797068775	2.25223275015673\\
-0.0712897935506828	2.43219826408154\\
0.259402493638034	2.57711605372954\\
0.566669600263679	2.6891458577117\\
0.847778827249968	2.77187104282629\\
1.10212364312125	2.82948518216731\\
1.33058671380521	2.86619067025825\\
1.53495371890201	2.88583489506905\\
1.71745758679747	2.89173975132996\\
};
\addplot [color=mycolor1, forget plot]
  table[row sep=crcr]{%
1.55816621582833	2.96411764280911\\
1.72369026381853	2.9589489634012\\
1.87357395094473	2.94471081198243\\
2.00951539727676	2.92302238097717\\
2.13308772686911	2.89516714394286\\
2.24571234639	2.8621446196781\\
2.34865136405567	2.8247181742914\\
2.44301121000513	2.78345639595214\\
2.52975215041391	2.73876741695496\\
2.60970030106496	2.69092650013397\\
2.68356005581673	2.64009761362272\\
2.75192569999174	2.58634982836324\\
2.81529151105137	2.52966933299994\\
2.87405996181646	2.46996775540036\\
2.9285478083254	2.40708735530879\\
2.97898991488656	2.34080353268013\\
3.02554067531358	2.27082499337512\\
3.06827285217283	2.196791835172\\
3.1071735880505	2.11827176852309\\
3.14213725341473	2.03475467638285\\
3.17295469322562	1.94564575898457\\
3.19929833125135	1.85025762396374\\
3.22070250845181	1.74780190281111\\
3.23653840868177	1.63738135042231\\
3.24598303048299	1.51798398475037\\
3.24798201588985	1.38848173911886\\
3.24120693772994	1.24763743617153\\
3.22400916860176	1.09412574273013\\
3.19437511598683	0.926576139898226\\
3.14989189658064	0.74364862494752\\
3.087738838116	0.54415514122812\\
3.00472841444392	0.327240017128452\\
2.89742883428089	0.0926281100414999\\
2.76240535369208	-0.159064235610916\\
2.59661056566265	-0.42598689659216\\
2.39792476504199	-0.704753322834074\\
2.16578860862381	-0.990290191232043\\
1.90179012891402	-1.27596971649145\\
1.61000317878624	-1.55411412857669\\
1.2968823130403	-1.8168464843066\\
0.970638174303505	-2.0571084413693\\
0.640213756305188	-2.26956197274868\\
0.314147274549137	-2.45111751934481\\
-0.000362505488881228	-2.60098113264671\\
-0.297988270820512	-2.720295012052\\
-0.575378380735594	-2.81155756868797\\
-0.830941166451551	-2.87801576920823\\
-1.06446367709236	-2.92315885645935\\
-1.27669672435658	-2.95036437111437\\
-1.46898741728075	-2.96269063235406\\
-1.64299300668765	-2.96278271928243\\
-1.80047857881911	-2.95285388917795\\
-1.94318577414053	-2.93471041232882\\
-2.07275514816701	-2.90979707621904\\
-2.19068588816658	-2.87924906665681\\
-2.29831984755406	-2.843942221781\\
-2.39684037593059	-2.80453778741935\\
-2.48727941435114	-2.76152026863806\\
-2.57052858919337	-2.71522830970021\\
-2.64735163136145	-2.66587917238847\\
-2.71839651149636	-2.61358761803144\\
-2.78420636050438	-2.55838002010735\\
-2.84522865642899	-2.5002044539993\\
-2.90182239101521	-2.43893739131148\\
-2.95426304297833	-2.37438750172038\\
-3.00274522025381	-2.30629695283276\\
-3.04738281614693	-2.23434050698934\\
-3.08820647051653	-2.15812264950087\\
-3.12515804747891	-2.07717295221539\\
-3.15808174399543	-1.990939889959\\
-3.1867113389556	-1.89878340212204\\
-3.2106529959443	-1.79996665418033\\
-3.22936297363835	-1.69364774480912\\
-3.24211962779781	-1.57887258222754\\
-3.24798929893982	-1.45457089871028\\
-3.24578622146153	-1.31955848477936\\
-3.23402770229988	-1.17255030912029\\
-3.21088785082031	-1.01219131125092\\
-3.17415655487047	-0.837114232777981\\
-3.12121566819313	-0.646036467539452\\
-3.04905171671751	-0.437909464786748\\
-2.95433316852084	-0.212132517993081\\
-2.83358778416639	0.0311657486123974\\
-2.68351580771023	0.290794598039761\\
-2.50145812401623	0.564167055530497\\
-2.28599477919338	0.847057253091699\\
-2.03757711193584	1.13357592830069\\
-1.75901774854111	1.41648320142122\\
-1.45562685451183	1.68787901637399\\
-1.13484527909285	1.94016869125852\\
-0.805392760759659	2.16705995139283\\
-0.476146011004946	2.36430279892949\\
-0.155067833387058	2.52997857399797\\
0.151540333209896	2.66432503572871\\
0.439349481079671	2.76924068531352\\
0.705928858253586	2.84766989253353\\
0.950428454900413	2.90303490669397\\
1.17316738454906	2.93880362351463\\
1.37523579774972	2.95821190730456\\
1.55816621582833	2.96411764280911\\
};
\addplot [color=mycolor1, forget plot]
  table[row sep=crcr]{%
1.42632528730357	3.05419055216814\\
1.59409885697326	3.04894192904819\\
1.74764607607782	3.03434727439337\\
1.88831768585849	3.01189675254528\\
2.01740026430351	2.98279289019926\\
2.13608416558777	2.94798796969752\\
2.24544811665552	2.90822043063386\\
2.34645448710049	2.86404765904994\\
2.43995094101876	2.81587408211711\\
2.52667551525986	2.76397436887494\\
2.60726315074524	2.7085119950182\\
2.68225238936669	2.64955362911531\\
2.75209140658278	2.58707985082535\\
2.81714284020386	2.52099268940941\\
2.8776870472672	2.45112041621127\\
2.93392350905193	2.37721996369354\\
2.98597013463019	2.29897729217643\\
3.03386020403657	2.2160059952028\\
3.07753665694582	2.12784443619997\\
3.11684338412892	2.0339517557063\\
3.15151313143606	1.93370319745457\\
3.18115160055983	1.82638539774865\\
3.20521736059578	1.71119259976722\\
3.22299732316753	1.58722523687895\\
3.23357786670449	1.45349302920453\\
3.23581235158743	1.3089257071312\\
3.2282869316483	1.15239574159971\\
3.20928847715888	0.982758977142025\\
3.176781336107	0.798920617843453\\
3.12840373601196	0.599935084967989\\
3.06149968552777	0.38514782253431\\
2.97320730405123	0.154383494569103\\
2.86062722894035	-0.0918240253086897\\
2.72109091958337	-0.351980889021323\\
2.55253274510374	-0.623406293831113\\
2.35393750479549	-0.902099020098744\\
2.12578916758129	-1.1827809605491\\
1.87040277410249	-1.45918525535049\\
1.59200788692452	-1.72459900904301\\
1.29649516293625	-1.97258135199058\\
0.990837210384202	-2.197696081303\\
0.682309937938043	-2.39607365215453\\
0.377709629271056	-2.56567186636938\\
0.0827467276445914	-2.70621088488075\\
-0.198285197101544	-2.81885870156143\\
-0.462566309044838	-2.90579312858928\\
-0.708603694444559	-2.96975905169869\\
-0.935974346036651	-3.01369849093327\\
-1.14503867672095	-3.04048466100225\\
-1.33667724745256	-3.05275737054036\\
-1.51207595388432	-3.05283983637206\\
-1.67256535682076	-3.04271255194964\\
-1.81950894294362	-3.02402253127754\\
-1.95423070636569	-2.99811154608017\\
-2.07797191745059	-2.96605230356283\\
-2.19186825631255	-2.92868580072754\\
-2.29694038051205	-2.8866561327032\\
-2.3940928301155	-2.84044100242064\\
-2.48411769306366	-2.7903773569584\\
-2.56770060761846	-2.73668222109411\\
-2.64542750422993	-2.67946910987757\\
-2.71779105258215	-2.61876051726236\\
-2.7851961466965	-2.55449698645495\\
-2.84796398665464	-2.48654322508971\\
-2.9063344415271	-2.4146916680026\\
-2.96046643468488	-2.33866383248366\\
-3.01043610126227	-2.25810976873771\\
-3.05623244371903	-2.1726058926298\\
-3.09775016794598	-2.08165151011889\\
-3.13477933218787	-1.98466441800741\\
-3.16699140132554	-1.88097611459704\\
-3.19392129598201	-1.76982740563066\\
-3.21494510178343	-1.65036558457997\\
-3.2292533268044	-1.52164495079917\\
-3.23582007028983	-1.38263325815509\\
-3.2333693489984	-1.23222780416891\\
-3.22034133184322	-1.06928627425662\\
-3.19486361413493	-0.892679032711538\\
-3.15473614964176	-0.701370955594936\\
-3.09744308405396	-0.494541380754747\\
-3.02020998960036	-0.271748964553959\\
-2.92012930110728	-0.0331421889446298\\
-2.79437683989129	0.220296299061973\\
-2.64053313743929	0.4865035954473\\
-2.45699943943893	0.762146269718007\\
-2.24345818717511	1.04256505003492\\
-2.00128002875728	1.3219347994068\\
-1.73374676631364	1.59368386301882\\
-1.44597222977026	1.85114025281027\\
-1.14447704690877	2.08828044420864\\
-0.836487911591588	2.30039820704417\\
-0.529131242108844	2.48452621841593\\
-0.228720064349343	2.63952985558765\\
0.05972043374006	2.76590334036294\\
0.332641313986093	2.86537701210694\\
0.587913607060129	2.94046425247022\\
0.824614911507096	2.99404850797217\\
1.0427491736426	3.02906396360623\\
1.24296707789219	3.04828205038319\\
1.42632528730357	3.05419055216814\\
};
\addplot [color=mycolor1, forget plot]
  table[row sep=crcr]{%
1.31655348510211	3.15637921267897\\
1.48642982758109	3.15105639663561\\
1.6433349255677	3.13613505069415\\
1.78835014147596	3.11298458113963\\
1.92253093721614	3.08272520919187\\
2.04687406090608	3.04625525378925\\
2.1622986584819	3.00427899608015\\
2.2696368592467	2.95733267011823\\
2.36963046158222	2.9058073600355\\
2.46293125979115	2.84996834893613\\
2.55010326760306	2.78997090464124\\
2.63162562069553	2.72587271692795\\
2.70789531091329	2.65764330257214\\
2.7792291529804	2.5851707245444\\
2.8458645409221	2.50826596733315\\
2.90795864101281	2.42666529630212\\
2.96558571049066	2.34003092262652\\
3.01873224249287	2.24795031068524\\
3.06728963207688	2.14993451539684\\
3.11104405126304	2.04541603839254\\
3.14966323256332	1.93374686338077\\
3.18267991879898	1.81419759669302\\
3.20947188448626	1.68595902752834\\
3.22923873353934	1.54814796410214\\
3.24097621960527	1.39981992001385\\
3.24344974021805	1.23999211740366\\
3.23517007051758	1.06768127718753\\
3.21437646873718	0.881961600425187\\
3.17903506682906	0.682048823881183\\
3.12686378827857	0.467415571580784\\
3.05539826716717	0.237940380069017\\
2.9621149441898	-0.00591354243759519\\
2.8446252500433	-0.262904611596984\\
2.70094538850893	-0.530834152049075\\
2.52982713787425	-0.806428962524813\\
2.33110705889611	-1.08534184026105\\
2.10600184068736	-1.36232084982776\\
1.8572609218841	-1.63156753781995\\
1.58910066947964	-1.88725091001963\\
1.30689405569773	-2.12408573046757\\
1.01666210925465	-2.33784925911224\\
0.724476500123249	-2.52572184134781\\
0.435905312060046	-2.68639164379551\\
0.155607403891534	-2.819935532752\\
-0.112877160118432	-2.92754352702366\\
-0.367152187414235	-3.01117485959625\\
-0.605855357627645	-3.07322211607211\\
-0.828473614608814	-3.11623165194633\\
-1.03513592590833	-3.14269931239132\\
-1.22641845594262	-3.15493938537975\\
-1.40318038950259	-3.15501361429271\\
-1.56643609265679	-3.14470381180652\\
-1.71726173041519	-3.12551291097201\\
-1.85673089174345	-3.09868249350657\\
-1.98587279537864	-3.06521830682594\\
-2.10564706617362	-3.02591824464424\\
-2.21693006990254	-2.98139949136202\\
-2.32050890884273	-2.93212306146033\\
-2.41708018607659	-2.87841494577717\\
-2.50725146172323	-2.82048366252365\\
-2.59154394175497	-2.75843433336419\\
-2.67039538429476	-2.69227956197409\\
-2.7441625133768	-2.62194745266871\\
-2.81312242894797	-2.54728711579714\\
-2.87747262214401	-2.46807199484482\\
-2.93732926870396	-2.3840013380924\\
-2.99272349836275	-2.29470014050124\\
-3.04359533907467	-2.19971791269805\\
-3.08978502654407	-2.09852670780088\\
-3.13102136894168	-1.99051897056593\\
-3.16690688734356	-1.87500598855174\\
-3.19689954892627	-1.7512180482722\\
-3.22029112401107	-1.61830786078233\\
-3.2361826053834	-1.47535944913705\\
-3.24345783448158	-1.32140549834562\\
-3.24075761963484	-1.15545713000458\\
-3.22645835672939	-0.976551065658248\\
-3.1986615884	-0.783819914013145\\
-3.15520403655315	-0.576591333807192\\
-3.09370104580571	-0.354520223017647\\
-3.01163908232814	-0.117753668347263\\
-2.90653301678303	0.13288011034839\\
-2.77615849394102	0.39568075177871\\
-2.61885563696615	0.667920990033036\\
-2.43387633491964	0.94578097461004\\
-2.22171710630155	1.2244281019515\\
-1.98435422762312	1.49828116328787\\
-1.72529430245151	1.76145443214308\\
-1.4493851392521	2.00831828601761\\
-1.16239594134893	2.23406294420578\\
-0.870448039756891	2.4351388225132\\
-0.579423174651817	2.60948215525256\\
-0.294473622727724	2.7565021318834\\
-0.0197126607919893	2.87687296629366\\
0.241898300129466	2.97221327049281\\
0.488499802108194	3.04473818059727\\
0.719179682800395	3.09694755689515\\
0.933771636258873	3.1313832330886\\
1.13265092126358	3.15046252545726\\
1.31655348510211	3.15637921267897\\
};
\addplot [color=mycolor1, forget plot]
  table[row sep=crcr]{%
1.2248568044801	3.26583710005082\\
1.39676323735686	3.2604434592983\\
1.55679049023156	3.24521862229108\\
1.70581451982452	3.221422185182\\
1.84470834389625	3.19009450224251\\
1.97431066725544	3.15207701567812\\
2.09540597774779	3.10803389929256\\
2.20871277539915	3.05847284285746\\
2.31487729233147	3.00376378484091\\
2.41447069016377	2.94415504221871\\
2.50798823846962	2.87978668139907\\
2.59584937599353	2.81070120318579\\
2.67839784791313	2.73685173789504\\
2.75590131620158	2.65810800822904\\
2.82854997541634	2.57426034850786\\
2.89645379062483	2.48502209186076\\
2.95963802332905	2.3900306692737\\
3.01803673979666	2.28884782091383\\
3.07148401943925	2.18095941565358\\
3.11970261709477	2.06577552528384\\
3.16228990657625	1.94263162365297\\
3.19870107764335	1.81079209738987\\
3.2282298224622	1.66945768180885\\
3.24998719594592	1.51777898202722\\
3.26288005083467	1.35487889119504\\
3.26559152860871	1.17988741327663\\
3.25656761648345	0.991992989171771\\
3.23401578224707	0.790514626554382\\
3.19592403882969	0.57499846910328\\
3.140111030269	0.345340222336652\\
3.06431894751212	0.101930268413435\\
2.96635975428256	-0.154189284546743\\
2.84431936093831	-0.421176800253831\\
2.69681232712494	-0.696285159889052\\
2.52326143687634	-0.97583839918236\\
2.32415570313803	-1.25533021869393\\
2.10122515708008	-1.52966670070289\\
1.85747174494847	-1.79354229005829\\
1.59701994460964	-2.04189681903875\\
1.32479505155844	-2.27036877288047\\
1.04608570702848	-2.47565308395311\\
0.766079005791641	-2.65569655093843\\
0.489457644572723	-2.80971028876742\\
0.220121090286436	-2.9380259158805\\
-0.0389481193804397	-3.04185234037724\\
-0.285688301139277	-3.12299646489722\\
-0.518869419207026	-3.18359900414887\\
-0.737952297696529	-3.22591629846879\\
-0.942931433327034	-3.2521596106048\\
-1.1341854907516	-3.26438966961406\\
-1.31234862482834	-3.26445694460304\\
-1.47820740793431	-3.25397586124015\\
-1.63262284267026	-3.23432191795935\\
-1.77647420393349	-3.20664275108662\\
-1.91062047148805	-3.17187657033859\\
-2.03587515139584	-3.13077349442778\\
-2.15299081170959	-3.08391696630888\\
-2.26265034563407	-3.0317436138914\\
-2.36546264670328	-2.97456072077281\\
-2.46196095618675	-2.9125609786977\\
-2.55260259969955	-2.84583449679776\\
-2.63776917271905	-2.77437821224611\\
-2.71776647992242	-2.69810293469696\\
-2.79282370044644	-2.61683829985855\\
-2.86309135893114	-2.53033593213455\\
-2.92863774711611	-2.4382711419608\\
-2.98944347778396	-2.3402435260626\\
-3.04539387669787	-2.23577691297095\\
-3.0962689453209	-2.12431921703683\\
-3.14173067841532	-2.00524294895247\\
-3.1813076247535	-1.87784739843053\\
-3.214376776523	-1.74136387446161\\
-3.24014321977739	-1.59496587513795\\
-3.25761854918551	-1.43778666219383\\
-3.26559993659443	-1.26894740114535\\
-3.2626530391848	-1.0875996971195\\
-3.24710370237166	-0.892986799456058\\
-3.21704561516177	-0.684527589669214\\
-3.17037344658949	-0.461926123467113\\
-3.1048528720396	-0.225306196851926\\
-3.01823906938497	0.024635631298619\\
-2.90845193438096	0.286474238038438\\
-2.77380746201065	0.557924329770409\\
-2.61328940594513	0.835770476163347\\
-2.42682507743038	1.11590042265069\\
-2.21550988553163	1.39347465647061\\
-1.98171681985069	1.66323968274924\\
-1.72903932359748	1.91995352116603\\
-1.46205156141205	2.15885263362157\\
-1.18591914600036	2.37606806816528\\
-0.905936065855446	2.56890781903211\\
-0.62708081388796	2.73596014692679\\
-0.353670263989849	2.87702212913656\\
-0.0891533241573329	2.99289809453567\\
0.163954213030953	3.0851308914121\\
0.404021800262236	3.15572495189461\\
0.630184158086769	3.2069025595404\\
0.842188542214752	3.24091398383363\\
1.04023911444572	3.25990527538251\\
1.2248568044801	3.26583710005082\\
};
\addplot [color=mycolor1, forget plot]
  table[row sep=crcr]{%
1.14821861877544	3.37822194099113\\
1.32211422966463	3.37275975697398\\
1.48506577373199	3.35725102144531\\
1.63779453771928	3.33285774558015\\
1.78103188774723	3.30054550382355\\
1.91549014928952	3.26109905551922\\
2.04184294580084	3.21513954137717\\
2.16071243679431	3.1631413777036\\
2.27266135883972	3.10544775489936\\
2.37818821564684	3.04228417982872\\
2.47772434226298	2.97376984761565\\
2.57163187060185	2.89992684386039\\
2.66020185224264	2.82068730874675\\
2.74365196007394	2.73589877482412\\
2.82212330574859	2.64532794657016\\
2.89567598835189	2.54866324231546\\
2.96428304456024	2.44551648425036\\
3.02782251609899	2.33542421486215\\
3.08606740302247	2.21784925275971\\
3.13867335214308	2.09218329214815\\
3.18516406667791	1.9577516125178\\
3.22491465336956	1.81382130969792\\
3.25713349629051	1.65961488879327\\
3.28084382325812	1.49433155665004\\
3.29486697834341	1.31717906207993\\
3.29781058888808	1.12741933821346\\
3.28806632819085	0.924431288084098\\
3.2638237265217	0.707793477853093\\
3.22310816963416	0.477387781564748\\
3.16385222464998	0.233521602091018\\
3.08400873558892	-0.0229392581382749\\
2.98171041137278	-0.290442608907457\\
2.85547265898182	-0.566650817741802\\
2.70442391469964	-0.848401476834022\\
2.52853252816493	-1.13175937750553\\
2.32878609395836	-1.41218193840917\\
2.10727494205935	-1.68479879629618\\
1.86714241956969	-1.9447765374198\\
1.6123914789193	-2.18771138208102\\
1.34757265013792	-2.40997843292828\\
1.07740931147432	-2.60897387674382\\
0.806429669689267	-2.78321434775047\\
0.538666472900795	-2.93229411050522\\
0.277460863216762	-3.05673154096928\\
0.0253774981168844	-3.15775215772752\\
-0.215785267024473	-3.23705505106691\\
-0.444918584837936	-3.29659819762263\\
-0.661489845233607	-3.33842296110853\\
-0.865417916930842	-3.36452454422826\\
-1.05695405637369	-3.37676584910671\\
-1.23657808233503	-3.37682731799263\\
-1.40491358433021	-3.36618381742783\\
-1.56266213479399	-3.34610016813901\\
-1.71055440301189	-3.31763840952904\\
-1.84931521840959	-3.28167160219613\\
-1.97963952727928	-3.23890053206647\\
-2.10217647246083	-3.18987093478668\\
-2.21751926609763	-3.1349897903404\\
-2.3261989864738	-3.07453988823723\\
-2.42868084382842	-3.0086922954671\\
-2.52536180090137	-2.93751663369261\\
-2.61656869841968	-2.86098924012431\\
-2.70255623138843	-2.7789993881692\\
-2.78350426090654	-2.69135380942585\\
-2.8595140414808	-2.59777981075905\\
-2.9306030087679	-2.49792733707459\\
-2.996697820998	-2.39137040748995\\
-3.0576253941722	-2.2776084643912\\
-3.11310173490182	-2.15606833597676\\
-3.16271847950798	-2.02610773800262\\
-3.2059272260731	-1.88702154284204\\
-3.24202204076467	-1.73805243153392\\
-3.27012098635517	-1.57840801168737\\
-3.28914822470641	-1.40728699800874\\
-3.29781925092471	-1.22391753054802\\
-3.29463316440711	-1.02761098651421\\
-3.27787753672069	-0.817834448416722\\
-3.24565321045131	-0.594303910063707\\
-3.19592780817642	-0.357097800483401\\
-3.12662703308031	-0.106785954479507\\
-3.0357708082547	0.155437482067335\\
-2.92165557654961	0.427637018465533\\
-2.78307377857452	0.707062532913073\\
-2.61954727411969	0.990151953958752\\
-2.43153652126608	1.27263826333063\\
-2.22057774153498	1.54977366982352\\
-1.98930312676034	1.81665734322728\\
-1.74131842508857	2.06862261920746\\
-1.48094487598371	2.30161695110491\\
-1.21286748687648	2.51250421836134\\
-0.941755014781069	2.69923773288845\\
-0.671919466907365	2.86088622283792\\
-0.4070651975853	2.99753033745599\\
-0.150149193850236	3.11007118332234\\
0.0966532046284112	3.19999973697445\\
0.331900943995047	3.26916921812658\\
0.55478867041782	3.31959838882448\\
0.765025060922061	3.35331887615604\\
0.962709256903624	3.3722680462706\\
1.14821861877544	3.37822194099113\\
};
\addplot [color=mycolor1, forget plot]
  table[row sep=crcr]{%
1.08432098958412	3.48958906878031\\
1.26016263041394	3.48406059583351\\
1.42584987037089	3.46828664202598\\
1.58198865342118	3.44334418104502\\
1.72920180552377	3.41013078867507\\
1.86810226553408	3.36937710519764\\
1.99927420417125	3.32166089092418\\
2.12326000419845	3.26742106314437\\
2.2405513984182	3.20697074047129\\
2.35158338521673	3.14050876766226\\
2.45672982678386	3.06812949587693\\
2.55629986906307	2.98983079230571\\
2.65053450445137	2.90552038458823\\
2.73960273450841	2.81502073768703\\
2.82359688986924	2.71807273618282\\
2.90252673922986	2.6143385213636\\
2.97631208113869	2.50340392539018\\
3.04477357591937	2.38478106818833\\
3.10762165833728	2.25791184958972\\
3.16444349775839	2.12217329129565\\
3.21468817177836	1.97688596873151\\
3.25765053035063	1.82132712217\\
3.29245469720726	1.65475043399235\\
3.31803883456972	1.47641485992724\\
3.33314372828983	1.28562521396671\\
3.33630894286411	1.08178726562651\\
3.32588168137908	0.864479657741579\\
3.30004485434708	0.633543639329572\\
3.25687178373545	0.389189022073703\\
3.19441473769967	0.132110575790638\\
3.1108321596272	-0.136396726777785\\
3.00455406714634	-0.414341923537122\\
2.87447621685277	-0.698985909754783\\
2.72016211379556	-0.986859236316647\\
2.54202061056272	-1.27387130908329\\
2.34142036981858	-1.55551849505863\\
2.12070586008893	-1.82717689529902\\
1.8830952937858	-2.08444177624029\\
1.63246628524926	-2.32345876586698\\
1.37306191215871	-2.5411897987414\\
1.10916823294834	-2.73557159187355\\
0.844817383951105	-2.90555079510593\\
0.583558255283464	-3.05100778424301\\
0.32831573119089	-3.17260093950908\\
0.0813380657245664	-3.27157072675212\\
-0.155783455734006	-3.34953920535941\\
-0.382045527197197	-3.40833035585505\\
-0.596937432007435	-3.44982485664703\\
-0.800337824945961	-3.47585307111208\\
-0.992416369819534	-3.48812343909048\\
-1.17354739216084	-3.48818010731902\\
-1.34423845207748	-3.47738266288519\\
-1.50507393048241	-3.4569013050777\\
-1.6566721384329	-3.42772193538039\\
-1.7996537604388	-3.39065695771374\\
-1.93461930124317	-3.34635878733215\\
-2.06213336918966	-3.29533405368629\\
-2.18271392744576	-3.23795723094195\\
-2.29682497433818	-3.17448296668221\\
-2.40487142121653	-3.10505674818091\\
-2.50719519661724	-3.02972379116863\\
-2.60407181297479	-2.94843619727989\\
-2.69570679019512	-2.8610585351157\\
-2.78223144719621	-2.76737208114765\\
-2.86369765836433	-2.66707803048689\\
-2.94007123848177	-2.55980007038668\\
-3.0112236804157	-2.44508681580146\\
-3.07692204077251	-2.32241474987444\\
-3.1368168701386	-2.1911925053699\\
-3.19042824292162	-2.05076757615838\\
-3.2371301915545	-1.90043686597723\\
-3.27613423396141	-1.73946285848749\\
-3.30647325143702	-1.56709759995265\\
-3.32698777524983	-1.38261705772837\\
-3.33631780438843	-1.18536862902833\\
-3.33290458143599	-0.974834417079117\\
-3.3150081689874	-0.750712059173906\\
-3.28074788496511	-0.513012992199026\\
-3.22817310695987	-0.262174680466433\\
-3.1553708019304	0.00082170216403196\\
-3.06061238059654	0.274343052643648\\
-2.94253533527052	0.556027086251549\\
-2.80034470250325	0.842761973870159\\
-2.6340074807505	1.13074781218675\\
-2.44440357023709	1.41565556019632\\
-2.23339472179801	1.69288079372895\\
-2.00378251961062	1.95786588536709\\
-1.75914769036507	2.20644272094582\\
-1.50359042146601	2.43513784071188\\
-1.2414151827585	2.64138844283209\\
-0.976814721288001	2.82363934648277\\
-0.713602903748065	2.98131945678464\\
-0.455028473577867	3.11472105262322\\
-0.203679564761306	3.22481906642957\\
0.0385301600539593	3.31306897526129\\
0.270314178817421	3.38121429414528\\
0.490928261198941	3.43112311143594\\
0.700069202602977	3.46466199268161\\
0.897772817418163	3.4836072979851\\
1.08432098958412	3.48958906878031\\
};
\addplot [color=mycolor1, forget plot]
  table[row sep=crcr]{%
1.03135213633093	3.59637090839305\\
1.20907004504027	3.59077914425909\\
1.37729149238138	3.57475983814494\\
1.53653710018345	3.54931719978665\\
1.68734744553568	3.51528855270082\\
1.83025841110878	3.47335468467441\\
1.96578322832601	3.42405169926548\\
2.09439954870802	3.36778297747839\\
2.21654011806895	3.30483039105101\\
2.33258587487359	3.2353642907107\\
2.44286051696477	3.15945205909332\\
2.54762576750443	3.07706520209026\\
2.64707671968337	2.98808508293693\\
2.74133675496009	2.89230750409191\\
2.83045161912324	2.78944643230707\\
2.91438231446358	2.67913725852437\\
2.99299653716238	2.56094010047323\\
3.06605847196424	2.43434380459931\\
3.13321687072209	2.29877149590738\\
3.19399151237957	2.1535887667974\\
3.24775840081393	1.99811588966944\\
3.29373444147692	1.83164577081624\\
3.33096288903687	1.65346970017156\\
3.35830161214142	1.46291322151251\\
3.37441719098545	1.25938452546964\\
3.37778900764212	1.0424374575076\\
3.36672866651923	0.811850263579898\\
3.33942099285791	0.567719246035073\\
3.29399299369215	0.310563285224406\\
3.22861581888925	0.0414306205472954\\
3.14164114942691	-0.238006225302083\\
3.03176704475577	-0.525386795849033\\
2.89821937358575	-0.81765279173427\\
2.74092516593776	-1.11111257881753\\
2.56064673823679	-1.40159202394692\\
2.35904427078028	-1.68466770128535\\
2.13864271113069	-1.95595869568967\\
1.90269631859855	-2.21143571861846\\
1.65496642560124	-2.44769807494091\\
1.39944760586285	-2.66217435664607\\
1.14008737146357	-2.85322028953649\\
0.880541753683481	-3.02011052134225\\
0.623995949698562	-3.16294213169387\\
0.37306151489212	-3.2824802789669\\
0.12974553557614	-3.37997902331034\\
-0.104523279302319	-3.45700515846479\\
-0.328830742118102	-3.51528377119382\\
-0.542693766928892	-3.55657483654339\\
-0.745971785698193	-3.58258266842174\\
-0.938782731174552	-3.59489523402033\\
-1.1214290670792	-3.59494797211015\\
-1.29433609777269	-3.58400615651164\\
-1.45800263195444	-3.56316030107577\\
-1.61296283916448	-3.53333004490172\\
-1.75975757063237	-3.49527301601026\\
-1.89891327739221	-3.44959614620652\\
-2.03092676136875	-3.39676771434008\\
-2.15625421059277	-3.33712901411439\\
-2.27530321811131	-3.27090499549261\\
-2.38842672143102	-3.19821354850303\\
-2.49591800484421	-3.11907331953905\\
-2.59800607440349	-3.03341010428896\\
-2.69485084646195	-2.94106197440155\\
-2.78653769193132	-2.84178338833293\\
-2.87307095877536	-2.73524862823342\\
-2.9543661659553	-2.62105500931138\\
-3.03024063679595	-2.498726439141\\
-3.10040243574305	-2.36771807355573\\
-3.16443761172733	-2.22742303238647\\
-3.22179596191571	-2.07718240693662\\
-3.27177584626739	-1.91630010695898\\
-3.31350904610332	-1.74406443604791\\
-3.34594730888101	-1.5597785997573\\
-3.36785308394414	-1.36280254438075\\
-3.37779802106989	-1.15260843741755\\
-3.37417399067421	-0.92885149910718\\
-3.35522247816075	-0.691456475521483\\
-3.31908880564762	-0.440717488068642\\
-3.26390712736859	-0.177405091985249\\
-3.18791975818069	0.0971307505286315\\
-3.08962942159058	0.380878683043629\\
-2.96797524458198	0.671126334464415\\
-2.82251367488618	0.96448540313711\\
-2.65357638041745	1.25699868441593\\
-2.46237236697735	1.54433379386422\\
-2.25100483013732	1.82204987931364\\
-2.02238635691151	2.08590404271753\\
-1.7800567491228	2.33215056177113\\
-1.52792968019001	2.55778436909587\\
-1.27000986510104	2.76069222248298\\
-1.01012604337771	2.93969646714663\\
-0.751716524415859	3.09449943478099\\
-0.49768777443629	3.22555383260815\\
-0.250348979705656	3.33389203355759\\
-0.0114119588916324	3.42094545037425\\
0.217961434923565	3.48837750643975\\
0.437083598214068	3.53794407781902\\
0.645653665675828	3.57138665281924\\
0.843670278731899	3.59035728676248\\
1.03135213633093	3.59637090839305\\
};
\addplot [color=mycolor1, forget plot]
  table[row sep=crcr]{%
0.987871498742845	3.69541322260165\\
1.16735455738493	3.68976238204575\\
1.33788136567027	3.67352014927281\\
1.49990992033289	3.64762963289456\\
1.6539191227879	3.61287609205577\\
1.80038590003472	3.56989587477314\\
1.93976812181382	3.51918672685299\\
2.07249189596827	3.46111825705069\\
2.19894201178193	3.39594180113781\\
2.31945449842671	3.32379926151298\\
2.43431044689757	3.24473073832322\\
2.54373039816002	3.15868094038059\\
2.64786872602072	3.06550449381061\\
2.74680754427043	2.96497037394905\\
2.84054975122996	2.85676578883048\\
2.92901090095079	2.74049995545068\\
3.01200967177911	2.61570834557892\\
3.0892568060592	2.48185814693752\\
3.16034254016137	2.33835589572287\\
3.22472275782997	2.18455848992647\\
3.28170441353351	2.01978908193851\\
3.33043122107849	1.84335964832785\\
3.36987121904565	1.6546022912774\\
3.39880862692961	1.45291144505109\\
3.41584337586375	1.23779899173702\\
3.41940274611399	1.00896362031364\\
3.40777046478862	0.766374328897531\\
3.37913904597866	0.510365512535147\\
3.33169055130225	0.241737460153689\\
3.26370865971377	-0.0381485193979178\\
3.17372040069302	-0.327295984552708\\
3.06065902694148	-0.623039582008411\\
2.92403111682985	-0.922071362753897\\
2.76406319399621	-1.2205420859526\\
2.58179906418554	-1.51424124241489\\
2.37912182003006	-1.79884340078391\\
2.1586854567446	-2.07019150701435\\
1.9237584818392	-2.32457581984514\\
1.67800075380296	-2.5589652969476\\
1.42520869896009	-2.77115794478507\\
1.16906838814842	-2.95983473943839\\
0.912949989095865	-3.12452187550422\\
0.659764068320027	-3.26548192892061\\
0.411885248544655	-3.38356235916136\\
0.171136380764908	-3.48002953470973\\
-0.0611809602427222	-3.55641062829667\\
-0.284226117261211	-3.61435758803426\\
-0.497544137563553	-3.65553962958098\\
-0.700987315142112	-3.68156479488022\\
-0.894641229489225	-3.69392746799862\\
-1.07875959830396	-3.69397702581169\\
-1.25370967242688	-3.68290245745213\\
-1.41992818321254	-3.66172824258476\\
-1.57788686152857	-3.63131759631898\\
-1.72806608400622	-3.59238008450817\\
-1.87093507956281	-3.5454814332458\\
-2.0069372018364	-3.49105403593258\\
-2.13647893981229	-3.42940718912599\\
-2.25992153593609	-3.36073648046014\\
-2.37757427242732	-3.28513203497942\\
-2.48968865497345	-3.20258552877778\\
-2.59645286285159	-3.11299602701529\\
-2.69798594721313	-3.01617481964689\\
-2.79433135048951	-2.91184953141287\\
-2.88544939831458	-2.79966788876174\\
-2.97120849235176	-2.67920164911807\\
-3.05137482238792	-2.54995134904115\\
-3.12560053738497	-2.41135271672331\\
-3.19341049118051	-2.26278582631016\\
-3.25418793804369	-2.10358834462691\\
-3.30715992998904	-1.93307451972076\\
-3.35138369660574	-1.75056184753966\\
-3.38573599681602	-1.55540755584059\\
-3.40890832391732	-1.34705704199601\\
-3.41941187170032	-1.12510601210963\\
-3.41559718901055	-0.889377051198837\\
-3.39569418203983	-0.640009436839153\\
-3.3578781132558	-0.377557973486523\\
-3.30036588172902	-0.103092462559662\\
-3.22154350942313	0.181715443294424\\
-3.12012001649417	0.474535749374782\\
-2.99529505715178	0.772374077051107\\
-2.84691925271312	1.07163416394608\\
-2.67561978752376	1.36825898805647\\
-2.4828628388947	1.65794660849366\\
-2.27093128469621	1.93641959476136\\
-2.0428107979076	2.19971170207586\\
-1.80199636098332	2.44442807993632\\
-1.55224837432126	2.66793932616076\\
-1.29733702121988	2.86848425234937\\
-1.04081254973412	3.04517622282719\\
-0.785829042578658	3.19792659241119\\
-0.535034581828675	3.32731083218641\\
-0.29052660598202	3.43440654295955\\
-0.0538612669716209	3.52062911387546\\
0.173899109718393	3.58758339731566\\
0.392116549741104	3.63694155623125\\
0.600499387624035	3.67035030112556\\
0.799025320798523	3.68936596235326\\
0.987871498742845	3.69541322260165\\
};
\addplot [color=mycolor1, forget plot]
  table[row sep=crcr]{%
0.952714132411607	3.78404114737501\\
1.13380397064677	3.77833685920762\\
1.30637325103211	3.76189732263229\\
1.47083442317104	3.73561543393371\\
1.62762067567458	3.70023265778367\\
1.77716446579211	3.65634701927888\\
1.91988117948408	3.60442233459398\\
2.05615667234629	3.54479760321261\\
2.18633759746216	3.47769588836797\\
2.31072359156383	3.40323231173144\\
2.42956054432884	3.32142100801071\\
2.54303430814367	3.23218104894099\\
2.65126431602905	3.13534147560688\\
2.75429666745447	3.03064569156601\\
2.85209632277312	2.91775558261401\\
2.94453812726338	2.79625585571411\\
3.03139647872044	2.66565924095871\\
3.11233357545849	2.52541338509484\\
3.18688635625902	2.37491048747444\\
3.25445249690282	2.213500986042\\
3.31427619007051	2.04051287582365\\
3.36543493760153	1.85527849771146\\
3.40682924953868	1.65717080131025\\
3.43717797286668	1.44565104805399\\
3.45502291556962	1.22032951214688\\
3.45874735579992	0.9810397392692\\
3.44661367496507	0.72792509839229\\
3.41682531964459	0.461533528993223\\
3.36761704032181	0.182912558011353\\
3.29737431525226	-0.106307761422098\\
3.20477771495196	-0.403860382744142\\
3.08896096889273	-0.706833750006439\\
2.9496639046841	-1.01172793986564\\
2.78735554857161	-1.31458405779946\\
2.60330137337451	-1.61118403748095\\
2.39955419368833	-1.89730252117866\\
2.17886067707563	-2.16897849116513\\
1.94449207431254	-2.42276690311203\\
1.70002351197028	-2.65593308604974\\
1.44909583639802	-2.86656477541579\\
1.19519465816116	-3.05359425813677\\
0.941473578325101	-3.21674038215368\\
0.690636114193532	-3.3563922133266\\
0.444878017363584	-3.47346077781157\\
0.205881893719113	-3.56922337966759\\
-0.0251492633383242	-3.64517894479444\\
-0.24743415390647	-3.70292549841588\\
-0.460543675595968	-3.74406429249515\\
-0.664329558251052	-3.77013028278857\\
-0.858857509148538	-3.78254577658923\\
-1.04434837522275	-3.78259279981239\\
-1.22112868110194	-3.77139956280173\\
-1.38959048251925	-3.74993686590292\\
-1.55015965532614	-3.71902102360354\\
-1.70327135469562	-3.67932067358308\\
-1.84935127458584	-3.63136555260085\\
-1.98880139529637	-3.57555591462119\\
-2.12198904416514	-3.51217173044483\\
-2.24923825863018	-3.44138115671302\\
-2.37082260203892	-3.36324801856241\\
-2.48695872612035	-3.27773823906836\\
-2.59780009525661	-3.18472529276192\\
-2.70343038817182	-3.08399487987252\\
-2.80385617807281	-2.97524912958049\\
-2.89899857143374	-2.85811075903503\\
-2.98868357032124	-2.73212775263374\\
-3.07263102866869	-2.59677929319657\\
-3.15044221867331	-2.45148387999543\\
-3.22158623375341	-2.29561080920482\\
-3.28538575814983	-2.1284964612851\\
-3.34100316177998	-1.9494671113347\\
-3.38742846086942	-1.75787020032102\\
-3.42347143418947	-1.55311608678694\\
-3.44776108074811	-1.33473209991477\\
-3.45875656165306	-1.10243004134154\\
-3.45477459667135	-0.856186900710389\\
-3.4340386540989	-0.596336235620857\\
-3.39475469997954	-0.323664315135675\\
-3.33521618761754	-0.0395009185994836\\
-3.25393688541677	0.254209713866721\\
-3.14980398001913	0.554875578344262\\
-3.02223637374841	0.85927944122605\\
-2.87132601507902	1.16367107638014\\
-2.69793615489473	1.46393315962507\\
-2.50373234616036	1.75581037118509\\
-2.29113115679563	2.03517586494238\\
-2.0631666435215	2.29829796170304\\
-1.82329152448334	2.54206725605282\\
-1.57514323706355	2.76415193623645\\
-1.32231036646699	2.96306465758262\\
-1.06813112802755	3.13814248527204\\
-0.815544970130462	3.28945653769512\\
-0.567005194255521	3.41767637308333\\
-0.324448874733253	3.52391527433429\\
-0.0893127649146833	3.60957823091694\\
0.137419219282995	3.67622740769931\\
0.355150868005853	3.72547273073984\\
0.563602246342536	3.75888945002311\\
0.762739981900453	3.77796071070406\\
0.952714132411606	3.78404114737501\\
};
\addplot [color=mycolor1, forget plot]
  table[row sep=crcr]{%
0.924922571603334	3.86012979383972\\
1.10741517226221	3.85437906114762\\
1.28173030615979	3.83777101375592\\
1.44824772692772	3.81115839519889\\
1.6073672638724	3.77524698969403\\
1.75948843022384	3.73060297968216\\
1.90499467336231	3.67766142563773\\
2.04424113250968	3.61673489540518\\
2.17754490465196	3.5480216365273\\
2.30517696374031	3.47161295997233\\
2.42735501310555	3.38749970936446\\
2.54423666888277	3.29557784790108\\
2.6559124724642	3.19565332509177\\
2.76239831665528	3.08744650423698\\
2.86362695087446	2.97059655376326\\
2.95943831615204	2.84466634353788\\
3.04956856500963	2.70914855109376\\
3.13363776267843	2.56347387892083\\
3.21113646713818	2.40702251364427\\
3.28141167295689	2.23914021248628\\
3.34365300705965	2.05916065760012\\
3.39688060976846	1.8664359259029\\
3.43993683501123	1.66037699706607\\
3.4714847415984	1.44050603636683\\
3.49001724665632	1.20652156428516\\
3.49388160346017	0.958376347328325\\
3.48132425384463	0.696365710316007\\
3.45056065889319	0.42122087860642\\
3.39987291230763	0.13419806142575\\
3.32773433350307	-0.162850132918554\\
3.23295470419749	-0.467437565315586\\
3.11483292281914	-0.776459007045048\\
2.97329712335472	-1.08626991353548\\
2.8090081060877	-1.39283663973601\\
2.62340278296963	-1.69194920344484\\
2.41866172249558	-1.97947427848262\\
2.19759801761047	-2.25161475590067\\
1.96348025985427	-2.50513826980121\\
1.71981555962489	-2.73754254492922\\
1.47012509843908	-2.94713875607942\\
1.21774297344864	-3.13305084521394\\
0.96566055867123	-3.29514366021501\\
0.716426820712345	-3.43390207285486\\
0.472103823814196	-3.55028577351698\\
0.234268654575758	-3.64558147105733\\
0.00404905387826575	-3.7212681828598\\
-0.21782038296065	-3.77890458056491\\
-0.430932716495946	-3.82004159770682\\
-0.635142282220024	-3.84615943395138\\
-0.830502859775551	-3.85862573905873\\
-1.01721333322895	-3.85867078542238\\
-1.19557193450195	-3.84737538802329\\
-1.36593894731248	-3.82566779777615\\
-1.5287070495612	-3.79432647920966\\
-1.68427814337166	-3.75398639801371\\
-1.83304543344168	-3.70514708831453\\
-1.97537956488262	-3.64818130356677\\
-2.11161775084459	-3.58334347413291\\
-2.24205496316544	-3.51077751254264\\
-2.36693640038687	-3.43052374495339\\
-2.48645057430092	-3.34252492663097\\
-2.60072246497577	-3.24663144100706\\
-2.70980628690403	-3.14260590427168\\
-2.81367749143409	-3.03012751626585\\
-2.91222371213947	-2.90879662713068\\
-3.00523445264759	-2.77814013900335\\
-3.09238943676117	-2.63761854151964\\
-3.17324570879828	-2.48663559317992\\
-3.24722381276431	-2.32455190463207\\
-3.31359372064852	-2.15070393914319\\
-3.37146165151542	-1.96443018467537\\
-3.41975954570869	-1.76510640611786\\
-3.45723973290234	-1.5521918499892\\
-3.48247821581105	-1.3252878936088\\
-3.4938908649245	-1.08420970731213\\
-3.48976745481902	-0.829069814385136\\
-3.46832850481223	-0.560369821260565\\
-3.42780882966715	-0.279093054192476\\
-3.36656904634427	0.0132132762966951\\
-3.28323169253416	0.314381602606185\\
-3.17683227421202	0.621610598879662\\
-3.04696849954554	0.931510987921317\\
-2.89392516352604	1.24022063163617\\
-2.71875019385723	1.54358792361473\\
-2.52326142158784	1.83740829534198\\
-2.3099741650074	2.11768516038246\\
-2.08195461616956	2.38087853731976\\
-1.84261899636801	2.62410531352001\\
-1.59550869855381	2.8452649189445\\
-1.34407405027175	3.04307988961685\\
-1.09149384707052	3.21705724794761\\
-0.840547139155294	3.36738904976286\\
-0.593541831202277	3.49481633599458\\
-0.352294833901325	3.6004802503094\\
-0.118152527498448	3.68577925135342\\
0.107961758343429	3.75224470509597\\
0.325485658990213	3.80144075774801\\
0.534150980927119	3.83488943420737\\
0.733919086341772	3.85401872217122\\
0.924922571603333	3.86012979383972\\
};
\addplot [color=mycolor1, forget plot]
  table[row sep=crcr]{%
0.903698258268103	3.92215764008513\\
1.08735132504351	3.91636861736943\\
1.2630878670561	3.89962346304805\\
1.43126467080252	3.8727440060535\\
1.59225780420536	3.83640815616314\\
1.74644301896131	3.79115683907415\\
1.89418044405055	3.73740195984527\\
2.03580251378513	3.67543449883245\\
2.17160419604085	3.60543218458413\\
2.30183471682192	3.52746644617713\\
2.4266900999022	3.44150854364669\\
2.54630594847205	3.34743492958732\\
2.66074998944189	3.2450320258736\\
2.77001398473036	3.13400072270291\\
2.874004695425	3.01396103650767\\
2.97253367537735	2.88445751021317\\
3.06530578539447	2.74496611261371\\
3.15190647658124	2.5949035980394\\
3.231788114712	2.43364052099142\\
3.30425593346166	2.26051934980577\\
3.36845464039736	2.07487935748161\\
3.42335727770903	1.87609012852151\\
3.46775866388929	1.66359551304482\\
3.50027657951629	1.43696954395447\\
3.51936471072105	1.1959850265647\\
3.52334202990972	0.940694008201649\\
3.51044345244203	0.671516969034837\\
3.47889581887818	0.389334317916928\\
3.42702102801173	0.0955698956418503\\
3.35336413039243	-0.207747597431574\\
3.25683844568718	-0.517961781933052\\
3.13687308352426	-0.831820439695171\\
2.99354231586669	-1.14557322156846\\
2.82765341052591	-1.45513677753876\\
2.64077198130094	-1.7563157285933\\
2.43517250879938	-2.04505456852969\\
2.21371504654298	-2.31768641400818\\
1.97966364228776	-2.57114312091857\\
1.73647315857763	-2.80309866965773\\
1.48757553920183	-3.01203162287412\\
1.23619331302362	-3.19720832758851\\
0.985199174442602	-3.3586016427204\\
0.737029308745312	-3.49676734226096\\
0.493648113057673	-3.61270149016879\\
0.256555180764672	-3.70769850472153\\
0.0268223699175176	-3.78322365819072\\
-0.194851198997808	-3.8408075074894\\
-0.408076437012607	-3.88196457610608\\
-0.612711373450221	-3.90813504631275\\
-0.808802834085554	-3.92064622583874\\
-0.996535261072368	-3.92068977991008\\
-1.17618757834346	-3.90931075018121\\
-1.3480979305358	-3.88740485343349\\
-1.5126355067774	-3.85572120270058\\
-1.67017837235513	-3.81486825892942\\
-1.82109615533191	-3.76532141722852\\
-1.96573648222838	-3.7074311251094\\
-2.10441416474785	-3.64143081852443\\
-2.23740226869434	-3.56744425820473\\
-2.36492432410724	-3.48549207330571\\
-2.48714705137577	-3.39549749225424\\
-2.60417307874457	-3.2972913812052\\
-2.71603321453662	-3.19061683572763\\
-2.82267791884308	-3.07513369606113\\
-2.92396770392946	-2.95042349313354\\
-3.01966229315015	-2.81599549175349\\
-3.10940850153282	-2.67129468598716\\
-3.19272698840363	-2.51571282150219\\
-3.2689982985768	-2.3486037636964\\
-3.33744898135135	-2.16930477750443\\
-3.39713908163803	-1.9771654914995\\
-3.44695294995837	-1.77158641000755\\
-3.48559610618886	-1.55206869578192\\
-3.51160175340777	-1.318276409936\\
-3.5233513285425	-1.07011126691083\\
-3.51911393753557	-0.80779804351817\\
-3.49710926858435	-0.531975958161062\\
-3.45559712848183	-0.243787708484773\\
-3.39299365328631	0.0550461005874151\\
-3.30800931705898	0.362182708775455\\
-3.19979749246997	0.674659892579683\\
-3.06809574540893	0.988959834364201\\
-2.91333735231228	1.30114111617248\\
-2.73671011286166	1.60703409649741\\
-2.54014507190236	1.90248118283486\\
-2.32622907275073	2.18359171546618\\
-2.09804955338222	2.44697556903457\\
-1.85899337556969	2.68992265269857\\
-1.61252952337273	2.91050660041041\\
-1.36200597763492	3.10760646104273\\
-1.11048459324002	3.28085520275383\\
-0.860627278989088	3.43053430254579\\
-0.6146358272448	3.5574378624554\\
-0.374239182449423	3.66272820938912\\
-0.140717047647816	3.74779986315749\\
0.0850526535298473	3.81416242323357\\
0.302533810177599	3.86334709699072\\
0.511468552957436	3.89683720004627\\
0.711816190771982	3.91602021218667\\
0.903698258268102	3.92215764008513\\
};
\addplot [color=mycolor1, forget plot]
  table[row sep=crcr]{%
0.888367104169478	3.96922705110388\\
1.07291122390955	3.96340870005969\\
1.24972660934698	3.94655953164109\\
1.41915454831557	3.91947891877675\\
1.58155489367523	3.88282430170261\\
1.73728701994619	3.83711784890623\\
1.88669482939918	3.78275407669708\\
2.03009480218316	3.72000758549893\\
2.16776619914711	3.64904039405054\\
2.2999426476819	3.5699085994557\\
2.42680445556867	3.48256828096067\\
2.54847109980377	3.38688071757307\\
2.6649934270774	3.28261712152967\\
2.77634518475087	3.16946321640947\\
2.88241358406765	3.04702412345997\\
2.98298869260379	2.91483017324689\\
3.07775157595266	2.7723444398493\\
3.16626127875783	2.61897300468039\\
3.24794097593904	2.4540791921464\\
3.32206396272324	2.27700326274185\\
3.38774061280826	2.08708926419992\\
3.44390803562623	1.88372086279525\\
3.4893249028563	1.66636790177474\\
3.52257474521719	1.43464501549463\\
3.54208182626729	1.18838268134413\\
3.546144261553	0.927709422610597\\
3.53298902860703	0.653141346565991\\
3.50085246228754	0.365671846300655\\
3.44808729507601	0.0668504672978017\\
3.37329300996861	-0.241163543734945\\
3.2754604206344	-0.55558933153358\\
3.15411492963611	-0.873069141527941\\
3.0094376299469	-1.18977885737679\\
2.8423415906143	-1.50160330875996\\
2.65448421822659	-1.80436222226359\\
2.4482059418933	-2.0940602942511\\
2.22639883879704	-2.36712731776046\\
1.99232247360235	-2.6206146492307\\
1.74939393345407	-2.85232289414203\\
1.50098193132137	-3.06084981396987\\
1.25023064039342	-3.24556257819414\\
0.999929787427645	-3.40651029606547\\
0.752436848006519	-3.54429883294141\\
0.509647993557539	-3.65995016654044\\
0.273008470238771	-3.7547646023417\\
0.0435506210950787	-3.83019828138013\\
-0.178051930487216	-3.88776249702837\\
-0.391423447866059	-3.92894655792853\\
-0.596425502401252	-3.95516271243238\\
-0.793101019687933	-3.96770989035063\\
-0.981625239981682	-3.96775237649092\\
-1.1622643685979	-3.95630961403404\\
-1.33534170842412	-3.93425381000624\\
-1.50121050516149	-3.90231264186172\\
-1.66023247507985	-3.86107499730855\\
-1.8127609182453	-3.81099824347687\\
-1.95912736560332	-3.75241598788524\\
-2.09963080922159	-3.68554566155298\\
-2.23452868539888	-3.61049553716092\\
-2.36402889969611	-3.52727101104556\\
-2.48828229154454	-3.43578014660196\\
-2.60737503158122	-3.33583861669606\\
-2.72132052997725	-3.22717431025802\\
-2.83005051531157	-3.10943199750812\\
-2.93340503094563	-2.98217859123723\\
-3.03112120282013	-2.84490970765218\\
-3.12282077645936	-2.69705842527188\\
-3.20799662322782	-2.53800736405932\\
-3.28599870174384	-2.36710544939466\\
-3.35602035661871	-2.18369096083397\\
-3.41708636655713	-1.98712264415913\\
-3.46804482675066	-1.7768207039597\\
-3.50756574495878	-1.55231926827857\\
-3.53415006926062	-1.31333125914399\\
-3.54615358389633	-1.05982531792969\\
-3.54183042910619	-0.792112345344705\\
-3.51940052341957	-0.510936253052693\\
-3.4771434253147	-0.217559861614279\\
-3.41351776998375	0.0861669099851317\\
-3.32730027834201	0.397772267523406\\
-3.21773200376542	0.714177318684968\\
-3.08465333840251	1.03177309149074\\
-2.92860545811216	1.34656437398015\\
-2.75087657637592	1.65437294940559\\
-2.5534779083619	1.95107955455206\\
-2.3390459754472	2.23287342098895\\
-2.11068193469523	2.4964744238251\\
-1.87175078567116	2.7392974959319\\
-1.62566981795802	2.95954082059142\\
-1.37571486537172	3.15619453949885\\
-1.12486587587972	3.32898062712203\\
-0.87570296296231	3.4782436754916\\
-0.630353853834017	3.60481537243616\\
-0.390485948581672	3.70987334533868\\
-0.157332027587421	3.79480985536257\\
0.0682623078150467	3.86111972866702\\
0.285780096344525	3.91031146769596\\
0.494971613979965	3.94384147128421\\
0.695795704407208	3.96306883484295\\
0.888367104169477	3.96922705110388\\
};
\addplot [color=mycolor1, forget plot]
  table[row sep=crcr]{%
0.878355311988031	4.00104451440901\\
1.06350692625045	3.99520620111796\\
1.24105208545559	3.97828669634494\\
1.41132245401614	3.95107065481952\\
1.57466773303343	3.91420199851153\\
1.73143696912751	3.86819041177193\\
1.88196376005733	3.81341874894101\\
2.02655438296571	3.750150547048\\
2.16547798030495	3.67853714869157\\
2.29895805486653	3.59862418003358\\
2.42716463522509	3.51035731500992\\
2.55020657118865	3.41358740789436\\
2.668123506243	3.30807520909464\\
2.78087715546415	3.19349600831664\\
2.88834160146377	3.06944468768286\\
2.99029241959652	2.93544182533574\\
3.08639457260405	2.79094167459111\\
3.17618919386929	2.63534305689275\\
3.25907963118477	2.46800444241896\\
3.33431747555457	2.28826473051858\\
3.4009897767428	2.09547144282054\\
3.45800926415817	1.88901813486057\\
3.50411013946209	1.6683927090189\\
3.53785283151403	1.43323782151323\\
3.55764187603864	1.18342353411735\\
3.5617615690935	0.919130581462076\\
3.54843389392288	0.64093999291339\\
3.51590199238786	0.349921399335943\\
3.46253971391698	0.0477085835925056\\
3.38698331236318	-0.263452424287396\\
3.28827546184747	-0.580698823062855\\
3.16600547552711	-0.900604275852195\\
3.02042477853998	-1.21929774783176\\
2.85251553485022	-1.53264516950835\\
2.66399457594665	-1.83647819729149\\
2.45724457303916	-2.12684262395052\\
2.23517770285111	-2.4002325211065\\
2.00105011016354	-2.65377765196434\\
1.7582542402097	-2.88536099227669\\
1.51011809336617	-3.09365738536341\\
1.25973568625162	-3.27809894883646\\
1.00984379818778	-3.43878382182336\\
0.762749729660402	-3.57635010468598\\
0.520306129482393	-3.69183655722441\\
0.283923478606458	-3.78654748590043\\
0.0546086990229475	-3.86193343438586\\
-0.166981171541117	-3.91949359001469\\
-0.380479311033542	-3.96070128664519\\
-0.585749985503425	-3.986950972031\\
-0.782834069573656	-3.99952339235443\\
-0.971901342639129	-3.99956518582877\\
-1.15321025667837	-3.98807919312425\\
-1.32707494440725	-3.96592226831757\\
-1.49383870905093	-3.93380798674879\\
-1.65385299334717	-3.89231225949612\\
-1.80746076448412	-3.84188040852438\\
-1.95498329637778	-3.78283470631474\\
-2.09670942756051	-3.71538173904544\\
-2.23288648811713	-3.63961922624244\\
-2.36371220332813	-3.55554214043776\\
-2.48932698599055	-3.46304813674846\\
-2.60980612184367	-3.36194244219558\\
-2.7251514361164	-3.25194248386318\\
-2.83528211072562	-3.1326826674037\\
-2.9400244113109	-3.00371986447725\\
-3.03910019488404	-2.86454033832207\\
-3.13211422011443	-2.71456903567785\\
-3.21854049496589	-2.55318239915177\\
-3.29770819576308	-2.37972609464072\\
-3.3687881039304	-2.19353927459823\\
-3.43078105286486	-1.99398715573653\\
-3.4825105630904	-1.78050369093681\\
-3.52262264058648	-1.55264583001233\\
-3.54959653315705	-1.31016012225037\\
-3.56177090540339	-1.05306102369323\\
-3.55739011243071	-0.781718072012295\\
-3.53467462163135	-0.496946044577514\\
-3.49191769492277	-0.200088546569051\\
-3.42760684601089	0.106918179217524\\
-3.34056332736365	0.421516953800008\\
-3.23008662410292	0.74055253109619\\
-3.09608510202537	1.06035723112041\\
-2.93917069281702	1.37690232801292\\
-2.76069690826706	1.68600603833541\\
-2.56272661017962	1.98357610732913\\
-2.34792790075655	2.26585543441595\\
-2.11941020973985	2.52963648855715\\
-1.88052401217249	2.77241584827913\\
-1.63465314874106	2.9924724548048\\
-1.38502715711598	3.18886813413911\\
-1.13457364935907	3.3613821295424\\
-0.885820579981077	3.51039961632549\\
-0.640848456119384	3.63677651896027\\
-0.40128537454523	3.7417004733795\\
-0.16833402832988	3.82656253933673\\
0.0571808702239901	3.89284832757713\\
0.274754939025249	3.94205200602846\\
0.48414407094491	3.975612869128\\
0.685307253194799	3.99487187487964\\
0.87835531198803	4.00104451440901\\
};
\addplot [color=mycolor1, forget plot]
  table[row sep=crcr]{%
0.87317260289467	4.01786189580409\\
1.0586470073743	4.01201298588825\\
1.23657813808843	3.99505628803038\\
1.40729283295192	3.96776882189008\\
1.57113567513125	3.93078746483326\\
1.72845048096189	3.88461536516827\\
1.87956560720388	3.8296292438126\\
2.02478211899208	3.7660867946643\\
2.16436396668982	3.69413370111489\\
2.29852943361614	3.61381002254227\\
2.42744322415422	3.52505588894224\\
2.55120865822076	3.42771659234049\\
2.669859524372	3.32154729687035\\
2.78335122505298	3.20621771999007\\
2.8915509323161	3.08131727771277\\
2.99422657288481	2.94636134697276\\
3.09103459362408	2.80079948506522\\
3.18150664226152	2.64402666092272\\
3.26503555735054	2.47539878863308\\
3.34086142196184	2.29425408892726\\
3.40805892143134	2.09994199696792\\
3.46552787019314	1.89186141208578\\
3.51198952458983	1.66950993600597\\
3.54599211843127	1.43254521754385\\
3.565929810729	1.18085842894172\\
3.57007968075661	0.914658061332085\\
3.55666118804975	0.634559542850044\\
3.52392119282383	0.34167275065185\\
3.47024478877763	0.0376757525251807\\
3.39428765284131	-0.275140018359175\\
3.29511970073396	-0.593869140628072\\
3.17236365200774	-0.915049888989435\\
3.02630751934734	-1.23478733986051\\
2.85796923624723	-1.54893823008124\\
2.6690962381088	-1.85334103266034\\
2.46209281396263	-2.14406336221362\\
2.23988130272687	-2.41763291314587\\
2.00571593672854	-2.67122014574578\\
1.76297642204718	-2.90275057507449\\
1.51496986821874	-3.11093871465674\\
1.26476464219	-3.2952500330042\\
1.01507049410291	-3.45580782062897\\
0.76816912641704	-3.59326672357929\\
0.52589097390353	-3.70867415021093\\
0.289628743614559	-3.80333652933828\\
0.0603763166980937	-3.87870162167307\\
-0.161217833407163	-3.93626249501914\\
-0.37479155748923	-3.97748436629825\\
-0.580210632191441	-4.00375260580504\\
-0.777515021874765	-4.01633865542448\\
-0.966871847477634	-4.01638009184131\\
-1.14853571801903	-4.00487119763804\\
-1.32281618140011	-3.98266087997323\\
-1.49005154074905	-3.95045538220979\\
-1.65058804753554	-3.90882383733263\\
-1.80476342481098	-3.85820524655758\\
-1.95289371835076	-3.79891590799786\\
-2.09526256837972	-3.73115666920935\\
-2.23211210715962	-3.65501964694317\\
-2.36363479941683	-3.57049426544381\\
-2.48996564479346	-3.47747262983988\\
-2.61117425248971	-3.37575439108605\\
-2.72725638116091	-3.2650513891372\\
-2.83812461886204	-3.14499249511081\\
-2.94359796877304	-3.01512922244045\\
-3.04339022056795	-2.87494284998519\\
-3.13709714252516	-2.72385400113866\\
-3.22418274777113	-2.56123584983809\\
-3.30396519460744	-2.38643236363178\\
-3.37560330143226	-2.19878321489474\\
-3.43808521149454	-1.99765713755406\\
-3.49022143481789	-1.78249548733109\\
-3.53064529193214	-1.552867446259\\
-3.55782459317131	-1.30853752562824\\
-3.57008902380522	-1.04954457618762\\
-3.56567787070637	-0.776289257127412\\
-3.54281201375396	-0.489623826818346\\
-3.4997920658988	-0.19093445153481\\
-3.43512084778765	0.117797349155185\\
-3.34764306570541	0.43396984883628\\
-3.23668882152542	0.754388232410177\\
-3.10220192627757	1.07535469351025\\
-2.94483103100244	1.39282366721893\\
-2.7659633660714	1.70261215790338\\
-2.56768831396552	2.00064250305003\\
-2.35269006632684	2.28318584141813\\
-2.1240821328952	2.54707244995374\\
-1.88520740306866	2.78984116869868\\
-1.63943250522551	3.00981256628613\\
-1.38996326771129	3.20608531441715\\
-1.13970056624627	3.37846804420796\\
-0.891145740461128	3.52736675030405\\
-0.646355204967131	3.65364981369306\\
-0.406936982742872	3.75851006021307\\
-0.174078362593084	3.8433380143813\\
0.0514066882572753	3.90961465117128\\
0.26902032040407	3.95882687108028\\
0.478521424067281	3.99240526127457\\
0.679869196887858	4.0116815148737\\
0.873172602894669	4.01786189580409\\
};
\addplot [color=mycolor1, forget plot]
  table[row sep=crcr]{%
0.872400712460433	4.02038721270919\\
1.05792368884529	4.01453670886125\\
1.23591279430132	3.99757442485156\\
1.40669415698166	3.97027624228707\\
1.57061160586457	3.93327798661598\\
1.72800818514876	3.88708182838907\\
1.87921147654295	3.83206356936454\\
2.02452177307966	3.76848002634673\\
2.16420225459164	3.69647603203228\\
2.29847042837004	3.61609080803541\\
2.42749020571817	3.52726364931374\\
2.55136408132319	3.42983900962203\\
2.67012496850639	3.32357121094303\\
2.78372732461065	3.20812913063445\\
2.89203728571882	3.08310136072375\\
2.99482163073449	2.94800249437786\\
3.09173552753474	2.80228138172703\\
3.18230919841028	2.64533241228778\\
3.26593390212375	2.47651111686476\\
3.34184799159715	2.29515561666519\\
3.40912429337584	2.10061563872189\\
3.46666068087803	1.89229089164514\\
3.51317646588389	1.66968044266358\\
3.54721805189474	1.43244420253153\\
3.56717804270875	1.1804765244426\\
3.57133243904973	0.913990076760244\\
3.55790032826914	0.633605457730588\\
3.52512913593394	0.340438584799407\\
3.47140564893363	0.0361741613396976\\
3.39538845990468	-0.276889589273369\\
3.29615156509209	-0.595840886473753\\
3.17332267837953	-0.917212741306342\\
3.02719527215502	-1.23710669588607\\
2.8587926074176	-1.55137815237982\\
2.66986666846752	-1.85586663643602\\
2.46282494966837	-2.14664304453676\\
2.24059128960257	-2.42024009883123\\
2.0064196277991	-2.67383432706057\\
1.76368777428001	-2.9053575740163\\
1.51569973809642	-3.1135302481095\\
1.26552008439842	-3.29782276943657\\
1.01585455977801	-3.45836216129842\\
0.768981076019589	-3.59580552596122\\
0.526726777741583	-3.71120155773919\\
0.290481737747762	-3.8058570046132\\
0.061237901848247	-3.88121921953595\\
-0.160357530756129	-3.93878036750128\\
-0.373943110125655	-3.98000446543796\\
-0.579384844552284	-4.00627553969976\\
-0.77672257159037	-4.01886365278914\\
-0.96612302601515	-4.0189050361166\\
-1.14784025442594	-4.00739269981099\\
-1.32218313135468	-3.98517436751489\\
-1.48948922383381	-3.95295518874916\\
-1.65010401763265	-3.91130328345454\\
-1.80436446095217	-3.8606567064451\\
-1.95258582575122	-3.80133085975595\\
-2.09505098149806	-3.73352572889659\\
-2.23200128836394	-3.65733258789161\\
-2.36362842820677	-3.57274002561814\\
-2.4900665935909	-3.47963931098985\\
-2.61138454585985	-3.37782925445319\\
-2.72757713607705	-3.26702085362826\\
-2.83855596441736	-3.14684214525272\\
-2.94413894473072	-3.01684383515928\\
-3.04403865551544	-2.87650645132992\\
-3.13784951437789	-2.72524996645255\\
-3.22503403219608	-2.56244706340121\\
-3.30490871084637	-2.38744145611349\\
-3.37663057017367	-2.19957289845858\\
-3.43918584588259	-1.99821065819708\\
-3.49138309306751	-1.78279721065399\\
-3.53185372744502	-1.55290358461373\\
-3.55906384378357	-1.30829699957148\\
-3.57134178307268	-1.04901997985141\\
-3.56692607720377	-0.775477865301702\\
-3.54403767512824	-0.488528543722764\\
-3.50097829987952	-0.189564564061972\\
-3.4362530729949	0.1194257760861\\
-3.34871021754107	0.435834095364993\\
-3.23768441700587	0.756459683113159\\
-3.10312477660631	1.07760026181148\\
-2.94568542159817	1.39520778860554\\
-2.76675859942821	1.70509912442116\\
-2.56843763007815	2.00319881562071\\
-2.35340908664954	2.2857822333132\\
-2.12478707505283	2.5496852985695\\
-1.88591335131713	2.79245314539585\\
-1.64015198650535	3.01241253190674\\
-1.39070529283191	3.20866760230002\\
-1.1404701847406	3.38103134977479\\
-0.891944060345442	3.5299128589176\\
-0.647179766393561	3.65618236087359\\
-0.407782351929108	3.76103345507587\\
-0.174936821431493	3.84585659144264\\
0.0505444541971551	3.91213206239664\\
0.268164599667046	3.96134569498887\\
0.477682956765757	3.99492678814681\\
0.679058762338607	4.01420566237649\\
0.872400712460432	4.02038721270919\\
};
\addplot [color=mycolor1, forget plot]
  table[row sep=crcr]{%
0.875685557219478	4.00967763805254\\
1.06100272774934	4.00383388897912\\
1.23874599994084	3.98689529297496\\
1.40924453870376	3.95964257308244\\
1.57284539700775	3.92271602638168\\
1.72989491724704	3.87662197661192\\
1.88072398936467	3.82174012688877\\
2.02563619962134	3.75833101425244\\
2.16489801188383	3.68654307749908\\
2.29873023832086	3.60641908780634\\
2.42730016506097	3.51790187687297\\
2.55071379570723	3.42083944803591\\
2.66900776242509	3.3149896888231\\
2.78214053566533	3.20002503333925\\
2.88998264804146	3.07553756233089\\
2.99230574750499	2.94104518790835\\
3.08877042556289	2.7959997556108\\
3.17891294772867	2.6397981105391\\
3.26213126941631	2.47179740992005\\
3.33767107714397	2.29133620127919\\
3.40461307655313	2.09776298196715\\
3.46186336966171	1.89047404078754\\
3.5081495136968	1.66896224611083\\
3.54202567591895	1.43287793509279\\
3.56189106090967	1.18210199051418\\
3.56602625257259	0.916829382170032\\
3.55265192876722	0.637658790924856\\
3.52001312997204	0.345680512379708\\
3.46648947089461	0.0425510835278295\\
3.39072717737164	-0.269460118123823\\
3.29178292249394	-0.587468353417559\\
3.16926320037092	-0.908029045899134\\
3.0234382680544	-1.22725882065251\\
2.85530871517088	-1.54101882183472\\
2.66660715145907	-1.84514416267025\\
2.45972740504391	-2.13569183046976\\
2.23758690778079	-2.4091731874464\\
2.0034408310195	-2.66273895188084\\
1.76067505595249	-2.89429402127679\\
1.51260680563624	-3.10253369378862\\
1.2623168570532	-3.28690728975612\\
1.01252803382881	-3.4475259237388\\
0.765534418367372	-3.58503623101065\\
0.523177189657266	-3.70048142961837\\
0.286857653948686	-3.79516691484637\\
0.0575760054487964	-3.87054178698562\\
-0.164015131686133	-3.92810206847397\\
-0.377551337644567	-3.969316898912\\
-0.582897646867223	-3.99557603830755\\
-0.780094452261493	-4.00815543066842\\
-0.969310149493292	-4.00819704008281\\
-1.15080119641976	-3.99669929558502\\
-1.32487935016732	-3.97451495749124\\
-1.4918853257357	-3.94235382694221\\
-1.65216788081969	-3.90078832860645\\
-1.806067272198	-3.85026053577205\\
-1.95390207358847	-3.79108965251219\\
-2.09595844071301	-3.72347931960282\\
-2.23248102314662	-3.64752438246993\\
-2.36366483543669	-3.56321696870832\\
-2.48964750319551	-3.47045188848123\\
-2.6105013915586	-3.36903151101306\\
-2.72622520662556	-3.25867040014854\\
-2.83673474211532	-3.13900012521297\\
-2.94185253377562	-3.00957481160171\\
-3.04129629697352	-2.86987816734751\\
-3.13466617617788	-2.71933292200084\\
-3.22143105060679	-2.55731384054737\\
-3.3009144433641	-2.38316571498379\\
-3.37228099786505	-2.19622795969058\\
-3.43452503594903	-1.99586758867153\\
-3.48646340108576	-1.78152234342872\\
-3.52673558716778	-1.55275543868523\\
-3.55381496787173	-1.30932262844642\\
-3.56603559239636	-1.05125087648801\\
-3.56163920527467	-0.778925685763544\\
-3.53884647527735	-0.493181072480737\\
-3.49595442884873	-0.195382502407381\\
-3.43145843393958	0.112510508935222\\
-3.34419179023719	0.427917864513764\\
-3.23346972277805	0.747663938626615\\
-3.09921883410767	1.06806550948267\\
-2.94206996319232	1.38508513324789\\
-2.76339399824633	1.69454034792435\\
-2.56526748071004	1.99234636366853\\
-2.35036682262639	2.27476058800659\\
-2.12180357232166	2.53859494654633\\
-1.8829243026072	2.78136778825985\\
-1.63710397863557	3.00137951287317\\
-1.3875599028346	3.19771094877064\\
-1.13720588438874	3.3701564993994\\
-0.888556135040165	3.5191120805132\\
-0.643678721429679	3.64544004071904\\
-0.404191376720517	3.75033068852255\\
-0.171288844332672	3.8351748005113\\
0.0542097043526339	3.90145558792076\\
0.271803250693252	3.95066346198877\\
0.481249224027468	3.98423321927077\\
0.682506708461911	4.00350103458518\\
0.875685557219478	4.00967763805254\\
};
\addplot [color=mycolor1, forget plot]
  table[row sep=crcr]{%
0.882731909221739	3.98702827620123\\
1.06761539375866	3.98119877047761\\
1.24483900571829	3.96431025452675\\
1.41473856780442	3.93715382213604\\
1.57766817606962	3.90037931935314\\
1.73398135577779	3.85450191085997\\
1.88401618029681	3.79990957904154\\
2.02808336540157	3.73687073334937\\
2.16645646266481	3.66554142460836\\
2.29936339425423	3.58597190173672\\
2.42697868338947	3.4981124361362\\
2.54941583457895	3.40181849054501\\
2.66671940613247	3.29685544149948\\
2.77885639919958	3.18290319271153\\
2.88570667111642	3.05956115349769\\
2.98705217798458	2.92635421240694\\
3.08256497769965	2.78274051879899\\
3.17179409970626	2.62812209688069\\
3.25415163517833	2.46185955216491\\
3.32889874738757	2.28329237097139\\
3.39513277197684	2.09176652072532\\
3.45177718708876	1.88667116454702\\
3.49757697745115	1.66748620191229\\
3.53110274393243	1.4338418888766\\
3.55076769715874	1.18559079145782\\
3.55486219390545	0.922890594589315\\
3.54161038162041	0.646293701855839\\
3.50925236618872	0.356836174514016\\
3.45615266872313	0.0561147602165053\\
3.38093134812654	-0.253662597463924\\
3.28260828478366	-0.56966898255585\\
3.16074475307444	-0.888508003772609\\
3.01556137678958	-1.20632901239781\\
2.84801011498882	-1.51900601649008\\
2.6597818717229	-1.82236525663449\\
2.45324093105079	-2.11243438844237\\
2.23129076023667	-2.3856792747371\\
1.99718904460682	-2.63919536900434\\
1.75433899699584	-2.87082967070344\\
1.50608635773878	-3.07922340030803\\
1.25554697065881	-3.26378036902835\\
1.00548064154205	-3.42457735549414\\
0.758216486533735	-3.5622384135832\\
0.515626080765996	-3.67779497954975\\
0.279135032617231	-3.77254959507967\\
0.0497613395445611	-3.84795521492908\\
-0.171830716992709	-3.90551627241799\\
-0.385270399547987	-3.94671303691297\\
-0.590420759709649	-3.97294769465193\\
-0.787323520997589	-3.98550890671645\\
-0.976150797072024	-3.98555100251722\\
-1.15716437571668	-3.97408406911436\\
-1.33068234754739	-3.95197167189453\\
-1.49705231623919	-3.91993356087358\\
-1.65663017597268	-3.87855133874811\\
-1.80976337831088	-3.82827561962295\\
-1.95677765570536	-3.7694336642546\\
-2.09796626742514	-3.70223683836107\\
-2.23358095111042	-3.62678751815175\\
-2.36382387957633	-3.54308528011456\\
-2.48884002864075	-3.45103237950283\\
-2.60870945550432	-3.35043866190228\\
-2.72343907143695	-3.2410261807742\\
-2.83295357390829	-3.12243392489568\\
-2.93708529193436	-2.99422320488968\\
-3.03556280793928	-2.855884416668\\
-3.12799836739509	-2.70684609687665\\
-3.21387429557463	-2.54648741039426\\
-3.29252893414871	-2.37415545137323\\
-3.36314301552118	-2.18918896964803\\
-3.42472793185169	-1.99095030145124\\
-3.47611803587304	-1.77886730131582\\
-3.51596990654864	-1.5524868133086\\
-3.54277234109924	-1.31154051518176\\
-3.55487152426177	-1.05602262530621\\
-3.55051608963588	-0.78627681059057\\
-3.52792622494695	-0.503086622526053\\
-3.48538912189345	-0.207760120922984\\
-3.42137955892997	0.0978043714961962\\
-3.33469919677839	0.411087109991549\\
-3.22462186239684	0.728966240501443\\
-3.09102612672127	1.04779966508264\\
-2.93449296225586	1.363573149435\\
-2.75634736244044	1.67210627095784\\
-2.55862967980305	1.96929477663697\\
-2.34399429304059	2.25135796132634\\
-2.11554707999374	2.51505648762724\\
-1.87664488694667	2.7578512438258\\
-1.63068614503158	2.97798592410831\\
-1.38092054995231	3.17449110084244\\
-1.13029848048179	3.34712106410228\\
-0.881370570904034	3.49624330839168\\
-0.636237859826358	3.62270319198721\\
-0.396545539721535	3.72768397322806\\
-0.163509402534211	3.81257721008895\\
0.0620367292732862	3.87887249966095\\
0.279582967943426	3.9280702282238\\
0.488882722600083	3.96161712127999\\
0.689894874815934	3.98086202666136\\
0.882731909221738	3.98702827620123\\
};
\addplot [color=mycolor1, forget plot]
  table[row sep=crcr]{%
0.893299774619925	3.95386829659369\\
1.07755214373071	3.94805954177936\\
1.25401542773574	3.93124431583162\\
1.42303580063363	3.90422922774552\\
1.5849782444428	3.86767833420769\\
1.74020732941008	3.82211988601337\\
1.88907212038673	3.76795405568808\\
2.03189418792735	3.70546078719016\\
2.16895781927488	3.63480723708031\\
2.3005016488094	3.55605452683267\\
2.42671104466854	3.46916371784449\\
2.54771069215461	3.37400107359669\\
2.66355690548067	3.27034280494737\\
2.77422928199946	3.1578796201603\\
2.8796213954568	3.03622153427032\\
2.97953031852448	2.90490354378534\\
3.07364488506467	2.76339295064852\\
3.16153276848965	2.61109932753198\\
3.24262668758664	2.44738835125698\\
3.31621038180594	2.2716009766731\\
3.38140545071861	2.08307964485391\\
3.43716074643413	1.88120335412893\\
3.48224674236073	1.66543336983428\\
3.5152581350474	1.43537096426486\\
3.53462875686095	1.19082767771975\\
3.53866347378391	0.931906976742413\\
3.52559177989175	0.659093708520822\\
3.49364683462498	0.373344424108135\\
3.44117125418124	0.0761677941582254\\
3.36674676341693	-0.230319245427688\\
3.26933899140187	-0.543376023084219\\
3.14844215251458	-0.859678674650531\\
3.0042028551991	-1.17542656741677\\
2.83750012747688	-1.48651439077919\\
2.64996194821813	-1.78875659376243\\
2.44390769691871	-2.07813814119094\\
2.22221931485283	-2.35105748741888\\
1.98815791197158	-2.60452746429405\\
1.7451527266885	-2.83630800111545\\
1.49659270031397	-3.0449586627256\\
1.24564701169292	-3.22981436239356\\
0.99513183136759	-3.39089983342567\\
0.747429709962517	-3.52880492315626\\
0.504458560161213	-3.64454330167256\\
0.267680966437529	-3.73941334949917\\
0.0381419077695911	-3.81487407197817\\
-0.183476801654758	-3.87244286466293\\
-0.39679508897941	-3.91361704952985\\
-0.601673307903635	-3.93981777331237\\
-0.798155537809111	-3.95235302598206\\
-0.986419868182171	-3.95239585477658\\
-1.16673648453138	-3.94097391618187\\
-1.33943335949599	-3.91896698158723\\
-1.50486877396498	-3.88710964629557\\
-1.66340962350881	-3.8459971348178\\
-1.81541439567	-3.79609266937436\\
-1.96121974953256	-3.7377353433588\\
-2.10112973204507	-3.67114781585288\\
-2.23540678869332	-3.5964434302804\\
-2.36426384812058	-3.51363257887012\\
-2.48785687111275	-3.42262830465481\\
-2.60627735145271	-3.32325127287698\\
-2.71954434210345	-3.21523437045106\\
-2.82759566144855	-3.09822731991584\\
-2.93027802071645	-2.97180183529007\\
-3.02733591852908	-2.83545801111575\\
-3.11839928890021	-2.68863282891876\\
-3.20297008630592	-2.53071188782891\\
-3.28040827084369	-2.36104570920414\\
-3.34991804508014	-2.17897220451209\\
-3.41053571604363	-1.98384708380746\\
-3.46112122223544	-1.77508403838827\\
-3.50035615831711	-1.55220633353923\\
-3.52675197680331	-1.31491082965675\\
-3.53867278879889	-1.06314421601491\\
-3.53437755233595	-0.797189207100136\\
-3.51208603276816	-0.517755532335079\\
-3.47007127328522	-0.226066895661054\\
-3.40677801058063	0.0760687666748171\\
-3.32096139680575	0.386221402916465\\
-3.21183403917211	0.70134969368578\\
-3.07920308030691	1.01787383287239\\
-2.92357492114615	1.3318155924254\\
-2.74620552404545	1.63899910462536\\
-2.5490804545193	1.93529237337846\\
-2.33482042707061	2.21685862135462\\
-2.10652232685255	2.48038217619461\\
-1.86755824066811	2.72323774414946\\
-1.62136202998576	2.94358355286687\\
-1.37123257662162	3.14037417707846\\
-1.1201759479247	3.31330312954958\\
-0.870798319037104	3.46269482591756\\
-0.625251013653767	3.58936892524313\\
-0.385221052769517	3.69449812986298\\
-0.151956201110977	3.7794753728873\\
0.07368758457424	3.8458001464851\\
0.291187166213739	3.89498815865432\\
0.500290138046111	3.92850437284837\\
0.700955402138476	3.94771693646535\\
0.893299774619924	3.95386829659369\\
};
\addplot [color=mycolor1, forget plot]
  table[row sep=crcr]{%
0.907201964607069	3.91167245534824\\
1.09065777465157	3.90588993340582\\
1.266154057183	3.88916794883912\\
1.43405126957733	3.86233344813099\\
1.594729297961	3.82606897971694\\
1.74856773050552	3.78091969528162\\
1.8959304784888	3.72730139516052\\
2.03715367621373	3.66550870945552\\
2.17253591515043	3.59572285092159\\
2.3023300003203	3.51801863638448\\
2.42673554145718	3.43237067112468\\
2.54589180117484	3.33865874564094\\
2.65987031709079	3.23667262488993\\
2.76866689895426	3.12611653258124\\
2.87219268316559	3.00661376124556\\
2.97026401679408	2.87771198423091\\
3.06259105600252	2.73889001751568\\
3.1487651184107	2.58956698232377\\
3.22824504843374	2.42911505251825\\
3.30034316574832	2.2568772212265\\
3.36421179758033	2.07219175911517\\
3.41883196817198	1.87442520570766\\
3.4630065393329	1.66301574203399\\
3.49536093365565	1.43752849981995\\
3.51435543146834	1.1977235870137\\
3.51831372075252	0.943636145187155\\
3.50547257797936	0.675665425934568\\
3.47405682690306	0.394666641041049\\
3.42238156859522	0.102035452677137\\
3.34897972599436	-0.200228877409782\\
3.25274723000599	-0.509498242008303\\
3.13309144873207	-0.822544621750892\\
2.99006239042379	-1.13563484408611\\
2.82444314390421	-1.44469331517534\\
2.63777820796342	-1.74552174078721\\
2.43232677218777	-2.03405132380219\\
2.2109413525581	-2.30659339061513\\
1.97688688500037	-2.56005258739435\\
1.73362686517079	-2.79207387400802\\
1.48460783432562	-3.00110837186224\\
1.23307049154952	-3.18639914912145\\
0.981906815594165	-3.34790143399911\\
0.733571301996436	-3.48615942365957\\
0.490044211354484	-3.60216321962818\\
0.252837747946268	-3.69720593341672\\
0.0230329004807929	-3.77275501842284\\
-0.198665367807622	-3.8303455570256\\
-0.411864969177006	-3.87149796299146\\
-0.616423224962243	-3.89765891705149\\
-0.812387983253727	-3.91016230237642\\
-0.999945933157816	-3.91020610170607\\
-1.17937904953582	-3.89884123580373\\
-1.3510290044646	-3.8769687935728\\
-1.51526875180609	-3.8453427594907\\
-1.67248019647208	-3.80457601721365\\
-1.8230367819316	-3.75514801179825\\
-1.96728987702298	-3.69741295278498\\
-2.10555795271765	-3.63160783367408\\
-2.23811767079117	-3.55785984339636\\
-2.3651961362982	-3.47619297194968\\
-2.4869636832195	-3.38653378625872\\
-2.60352666456006	-3.28871649299009\\
-2.71491980680991	-3.18248752976885\\
-2.821097770212	-3.06751004995626\\
-2.92192564020713	-2.94336880162888\\
-3.01716817460706	-2.8095760590619\\
-3.10647776214747	-2.66557945213046\\
-3.1893812319927	-2.51077275782587\\
-3.2652659155717	-2.34451096228418\\
-3.33336572959656	-2.16613115111861\\
-3.39274854857961	-1.97498099842778\\
-3.44230678277244	-1.77045672708944\\
-3.48075386369101	-1.5520522900466\\
-3.50663020521454	-1.31942101315796\\
-3.51832301358775	-1.07244984603562\\
-3.51410481113174	-0.811344487469956\\
-3.49219533339304	-0.536720862648344\\
-3.45085007673776	-0.249694813293135\\
-3.38847575070051	0.0480421262739279\\
-3.30376801415931	0.354176392328224\\
-3.19586050306031	0.665772375441081\\
-3.06446749681585	0.97933329473128\\
-2.90999768909297	1.29093043938497\\
-2.73361585877175	1.59639665376012\\
-2.53723454146527	1.89156610234076\\
-2.32342899703125	2.17253026180334\\
-2.09528334141111	2.43587407085777\\
-1.85618936035988	2.67885886947717\\
-1.60962792701334	2.89952967674314\\
-1.35896372133843	3.096739924233\\
-1.10727761787898	3.27010201066058\\
-0.857250548598626	3.41988281855851\\
-0.611101534345176	3.54686777559801\\
-0.37057381810315	3.65221571353813\\
-0.13695797183735	3.73732173818946\\
0.0888606376441422	3.8036989372349\\
0.306341543883091	3.85288383501983\\
0.51522509565988	3.88636602159796\\
0.715470826708424	3.90553956693049\\
0.907201964607069	3.91167245534824\\
};
\addplot [color=mycolor1, forget plot]
  table[row sep=crcr]{%
0.924302577268678	3.86189211630072\\
1.10682793588482	3.85614030118489\\
1.28118346465662	3.8395283562989\\
1.44774822158637	3.8129081250885\\
1.6069213205598	3.77698458517328\\
1.75910156319047	3.73232319280708\\
1.90467170212362	3.67935834691924\\
2.04398620368068	3.61840200227287\\
2.1773615125963	3.54965182669744\\
2.30506796545077	3.47319857177777\\
2.42732263390715	3.38903253178664\\
2.54428249642295	3.29704912361095\\
2.65603743712263	3.19705375039243\\
2.76260265707262	3.08876623051121\\
2.86391016387759	2.97182519594288\\
2.95979909106812	2.84579300228258\\
3.05000470335532	2.71016185681574\\
3.13414608564841	2.56436206748747\\
3.21171271537749	2.40777354542344\\
3.28205040596633	2.23974194812529\\
3.34434751338946	2.05960110511928\\
3.39762284392349	1.86670357364382\\
3.44071740249287	1.66046124474937\\
3.47229295909028	1.4403977306503\\
3.49084130871891	1.20621363425014\\
3.49470888810416	0.957864517924979\\
3.482141794043	0.695649247545377\\
3.45135579210955	0.420303290690675\\
3.40063409282872	0.133087650260337\\
3.32845205279382	-0.164139997232879\\
3.23362241769647	-0.468888574874113\\
3.11544783921875	-0.778048547420278\\
2.97386069163747	-1.08797222096061\\
2.80952605372759	-1.39462440672324\\
2.62388462414642	-1.6937954453718\\
2.41911975881142	-1.98135419210385\\
2.19804596743955	-2.25350728117634\\
1.96393173817926	-2.50702712272158\\
1.7202826576272	-2.73941660557328\\
1.47061727341327	-2.94899183374906\\
1.21826637271547	-3.1348809568265\\
0.966217799225929	-3.2969520196137\\
0.717017161305549	-3.43569199190078\\
0.472723616438722	-3.55206163566387\\
0.234911949930583	-3.6473478781585\\
0.00470825077904838	-3.72302931723577\\
-0.217153924593788	-3.78066377945691\\
-0.430268147238923	-3.82180111000609\\
-0.63448884007801	-3.84792031745743\\
-0.829869553541633	-3.86038785584452\\
-1.01660872145817	-3.86043285839573\\
-1.19500398360006	-3.84913508976578\\
-1.36541495081342	-3.82742185085764\\
-1.52823359176441	-3.7960707542068\\
-1.68386109282635	-3.75571600097849\\
-1.83268995444608	-3.70685643347235\\
-1.97509013772311	-3.64986416981211\\
-2.11139819369793	-3.58499304573975\\
-2.24190845035496	-3.512386405709\\
-2.3668654730398	-3.43208402260333\\
-2.48645714046661	-3.34402810450295\\
-2.60080778702254	-3.24806848861867\\
-2.70997095461662	-3.14396724498773\\
-2.81392137979104	-3.03140303150048\\
-2.91254592337998	-2.90997567075147\\
-3.00563324307241	-2.77921156931496\\
-3.09286212990652	-2.63857077978506\\
-3.17378859834742	-2.48745671942205\\
-3.24783206100307	-2.32522980327834\\
-3.31426126159385	-2.15122650860271\\
-3.37218111214272	-1.96478562561834\\
-3.42052220388863	-1.76528360221187\\
-3.45803553631711	-1.55218085095605\\
-3.4832958913014	-1.32508050224315\\
-3.49471815072211	-1.08380015805423\\
-3.49059148637807	-0.828455510171989\\
-3.46913637521871	-0.559552068116832\\
-3.42858832468511	-0.278077702428641\\
-3.36730952046585	0.0144154025009713\\
-3.28392500818716	0.315754618250104\\
-3.1774736798184	0.623133930516661\\
-3.0475572882158	0.933160270630387\\
-2.89446495101418	1.24196910719445\\
-2.71924869608243	1.54540820624748\\
-2.52372969898562	1.83927427087464\\
-2.31042541409597	2.11957373071161\\
-2.08240269310558	2.38277094052677\\
-1.84307691646843	2.62598784012684\\
-1.59598734938198	2.84712897763138\\
-1.34458128952978	3.04492150669661\\
-1.09203405767341	3.21887617504392\\
-0.841121217041329	3.36918768566299\\
-0.594147512821357	3.49659864745824\\
-0.352927239901517	3.60225082355921\\
-0.118804802241516	3.68754254447156\\
0.107297807599128	3.75400451875631\\
0.324818987254881	3.80319990561513\\
0.533490828574192	3.83664957538831\\
0.733274610028959	3.85578031258039\\
0.924302577268677	3.86189211630072\\
};
\addplot [color=mycolor1, forget plot]
  table[row sep=crcr]{%
0.944516283357556	3.80590651146832\\
1.12600694416265	3.80018893313277\\
1.29907839514465	3.78370091403236\\
1.46413313464144	3.75732354672323\\
1.6215942509449	3.72178786797593\\
1.77188427403367	3.67768265364217\\
1.9154090101981	3.62546341875114\\
2.05254514551271	3.56546157697518\\
2.18363055333876	3.49789310579632\\
2.30895639976892	3.42286635562741\\
2.4287602885957	3.34038885649499\\
2.54321981533404	3.25037313791706\\
2.65244600700955	3.15264170713785\\
2.75647621470372	3.04693144590314\\
2.85526610658436	2.93289780187852\\
2.94868049073801	2.81011928079819\\
3.03648279320481	2.67810290045345\\
3.11832314470376	2.5362914557908\\
3.193725211668	2.38407366912169\\
3.26207217013619	2.22079855604496\\
3.3225925950385	2.04579560759144\\
3.37434755227592	1.85840263152925\\
3.41622085719777	1.65800323681731\\
3.44691529552266	1.44407586624896\\
3.46495853482702	1.2162558124227\\
3.46872334307913	0.974410573845452\\
3.45646730708205	0.718726991762843\\
3.42639709238593	0.449805694754777\\
3.37676087139271	0.168754519994192\\
3.30596933613776	-0.122731768557435\\
3.21274043621382	-0.422321960411291\\
3.09625600043273	-0.727047255616864\\
2.95631104282055	-1.03336438385475\\
2.7934311549163	-1.33729128191246\\
2.60893272668596	-1.63461097082836\\
2.40490683411936	-1.9211240646005\\
2.18412034337852	-2.1929170949095\\
1.9498441360456	-2.44660745683292\\
1.70563334591066	-2.67952921263771\\
1.45509320214317	-2.88983651530382\\
1.20166398360216	-3.07651881288057\\
0.948450624488358	-3.23933857772801\\
0.698110229976512	-3.37871350278845\\
0.452798421644794	-3.49556910274425\\
0.21416619587588	-3.59118538035426\\
-0.0166058971438045	-3.66705516544324\\
-0.238750359501091	-3.72476457513914\\
-0.451844476043196	-3.76589970664256\\
-0.655740473954405	-3.79197908652604\\
-0.850500202401142	-3.80440868486485\\
-1.03633766529251	-3.80445512238523\\
-1.21357068491732	-3.79323256638883\\
-1.38258161808189	-3.77169927444621\\
-1.54378626382428	-3.74066046616579\\
-1.69760973338298	-3.70077496828251\\
-1.84446795318847	-3.65256377189301\\
-1.98475352691003	-3.59641921584209\\
-2.11882481403556	-3.53261396042429\\
-2.24699723997037	-3.46130925477345\\
-2.3695360075457	-3.38256225194533\\
-2.48664951816332	-3.29633231161098\\
-2.59848292814463	-3.20248637363108\\
-2.70511136386072	-3.10080360597876\\
-2.80653240361764	-2.99097964416911\\
-2.90265751394226	-2.87263086079165\\
-2.99330221482033	-2.74529924502634\\
-3.07817485806523	-2.60845864283539\\
-3.15686405504991	-2.46152331494399\\
-3.22882500894723	-2.30386001167888\\
-3.29336532118994	-2.13480503046703\\
-3.34963128286441	-1.95368798485662\\
-3.39659625583826	-1.75986421743488\\
-3.43305350587418	-1.55275783808721\\
-3.45761674382456	-1.3319171192781\\
-3.46873256610547	-1.09708323512193\\
-3.46470976061236	-0.848271860648108\\
-3.44377071778722	-0.585864743501096\\
-3.40412947457992	-0.310704947477907\\
-3.34409866432677	-0.0241852744818788\\
-3.26222340618926	0.271684877588597\\
-3.15743393924866	0.574251940885052\\
-3.02920140236346	0.880247522848288\\
-2.87767444134446	1.18588745718983\\
-2.70377096196827	1.4870430624743\\
-2.50920191889443	1.77947271618314\\
-2.29641362803012	2.05908698282915\\
-2.06845016011193	2.32221017542621\\
-1.82875372350786	2.56579959516712\\
-1.58093329845906	2.7875920105064\\
-1.32853619394178	2.98616258070545\\
-1.07485285323216	3.16089911709276\\
-0.822774572692031	3.31190888410948\\
-0.574710991668611	3.43988277640292\\
-0.332563142706253	3.54594231826279\\
-0.0977407564306555	3.6314904143393\\
0.128790253767559	3.69807987412364\\
0.346443592488374	3.74730680680177\\
0.554942570560637	3.78073046282245\\
0.754251950804218	3.79981746875353\\
0.944516283357555	3.80590651146832\\
};
\addplot [color=mycolor1, forget plot]
  table[row sep=crcr]{%
0.967808437787213	3.74499261078458\\
1.14818689311289	3.73931195079239\\
1.31985791757415	3.72295916374077\\
1.48325294871124	3.69684878707962\\
1.63882433054309	3.66174127259313\\
1.78702324306083	3.61825136430489\\
1.92828304146005	3.56685777405959\\
2.06300668643723	3.50791301948369\\
2.19155711540948	3.44165271477233\\
2.31424958279424	3.36820391851422\\
2.43134516225289	3.28759237058299\\
2.54304474522393	3.19974861744842\\
2.6494829867059	3.1045131547088\\
2.75072174486073	3.00164082654933\\
2.84674264284519	2.89080483060273\\
2.9374384594573	2.77160079737071\\
3.02260314284291	2.64355155793964\\
3.1019203554607	2.50611339179397\\
3.17495062009433	2.35868476397706\\
3.24111737238403	2.20061881692842\\
3.29969256622918	2.03124116458242\\
3.34978295744406	1.84987481269934\\
3.39031883591177	1.65587423738532\\
3.42004779543775	1.4486706874312\\
3.43753708839786	1.22783047483062\\
3.44118909537926	0.993127164611371\\
3.4292752135284	0.744626918972085\\
3.39999364017396	0.482783573532793\\
3.35155555677723	0.208536270438194\\
3.28230149779729	-0.0766020530382409\\
3.19084478476633	-0.370481450901351\\
3.07623192928899	-0.670295958881667\\
2.93810193520605	-0.972626900863537\\
2.77681969261661	-1.27356044058473\\
2.5935561610029	-1.56887933595467\\
2.39029241707874	-1.85431300653857\\
2.16973656449054	-2.12581468117063\\
1.935159563803	-2.37982503876469\\
1.69017314560574	-2.61348247365794\\
1.43848440406202	-2.82475130603174\\
1.18366381336087	-3.01245716703307\\
0.928956384295275	-3.17623731692343\\
0.677152922916439	-3.31642727135342\\
0.430524619113761	-3.43391106100985\\
0.190813353526649	-3.52996117362653\\
-0.0407360190398877	-3.60608824569472\\
-0.263316741814673	-3.66391288558087\\
-0.476488298868083	-3.70506493338691\\
-0.680102117264968	-3.73111020158552\\
-0.874231767747263	-3.74350154438241\\
-1.0591115050854	-3.74354965140351\\
-1.23508466119144	-3.73240872042629\\
-1.40256186232585	-3.71107262287069\\
-1.56198815148524	-3.68037794588043\\
-1.71381767872152	-3.64101112654664\\
-1.85849450947297	-3.59351765219552\\
-1.99643816425042	-3.53831192954726\\
-2.12803265189593	-3.47568691584796\\
-2.25361793603441	-3.40582297170541\\
-2.37348294778796	-3.32879566284923\\
-2.48785941152722	-3.24458243257702\\
-2.59691587926138	-3.153068212415\\
-2.70075147467725	-3.05405015647705\\
-2.79938893547174	-2.94724179295157\\
-2.89276662118796	-2.83227699924786\\
-2.9807292348313	-2.70871433859584\\
-3.06301710498207	-2.57604245620105\\
-3.13925400991203	-2.4336874304032\\
-3.20893372022145	-2.28102321172531\\
-3.27140572068634	-2.1173865544282\\
-3.32586097793594	-1.9420981301299\\
-3.37131917984663	-1.75449176545173\\
-3.40661960543466	-1.5539538829359\\
-3.43041868081252	-1.33997511444028\\
-3.44119826803717	-1.11221550915271\\
-3.43728965053286	-0.870583535237989\\
-3.41691871122378	-0.615326928416679\\
-3.37827747551848	-0.347130215287896\\
-3.31962541754045	-0.0672095307517391\\
-3.23942014602516	0.222609279273533\\
-3.13647108052289	0.519847122444371\\
-3.01010204621649	0.821381790707592\\
-2.86030101476354	1.12352752006707\\
-2.6878302144381	1.42219058941629\\
-2.49427053077934	1.71309325270415\\
-2.28198236475224	1.99204203709281\\
-2.05398010648345	2.2552034280972\\
-1.81373520525708	2.49934532639181\\
-1.56493772823386	2.72200884185793\\
-1.31125332476364	2.92159021948065\\
-1.05610981641013	3.09733170608087\\
-0.802537173781546	3.24923679019982\\
-0.553070814292143	3.37793516318734\\
-0.309715500973054	3.48452486394514\\
-0.0739585488149666	3.57041506006226\\
0.153182698290488	3.6371857336636\\
0.371093942046129	3.68647293783938\\
0.579489767029112	3.71988203757484\\
0.778340934540154	3.73892713446705\\
0.967808437787212	3.74499261078458\\
};
\addplot [color=mycolor1, forget plot]
  table[row sep=crcr]{%
0.994196135033692	3.68031066273118\\
1.17340809817122	3.67466886909633\\
1.34358529454558	3.65846042759474\\
1.50519440257395	3.63263740579726\\
1.65872294953164	3.59799278088386\\
1.80465617574299	3.55516956627888\\
1.94345982046772	3.50467133034416\\
2.07556737826532	3.44687286752932\\
2.20137056907467	3.38203024968933\\
2.32121196908053	3.31028982587883\\
2.43537893622874	3.23169598202777\\
2.54409812292944	3.14619764586289\\
2.64752999734826	3.05365365219381\\
2.7457628975869	2.95383719024497\\
2.83880622732637	2.84643965569918\\
2.92658247713054	2.73107434051203\\
3.00891783529258	2.60728052632482\\
3.08553125196808	2.47452871344698\\
3.15602196091464	2.33222792507424\\
3.21985567036081	2.1797362787155\\
3.2763499398936	2.01637630730141\\
3.32465969933645	1.84145681803885\\
3.36376447264931	1.65430334738335\\
3.39245966613	1.45429941414018\\
3.40935525226816	1.24094064278145\\
3.41288624655233	1.01390321439251\\
3.40134033834497	0.773126737678835\\
3.37290854193795	0.518909246585072\\
3.3257642423025	0.252008462058861\\
3.25817385791012	-0.026261159208295\\
3.16863793304269	-0.313951047084765\\
3.05605464214126	-0.608440208278517\\
2.91988919546258	-0.906456213524509\\
2.7603245152051	-1.20417127277554\\
2.57836396307739	-1.4973783244928\\
2.37585911180771	-1.78173589999365\\
2.15544620475775	-2.05305307612738\\
1.92039246005548	-2.30757306327229\\
1.67437275902528	-2.54221124732962\\
1.42121197663355	-2.7547126655596\\
1.16463328497573	-2.94371199518092\\
0.908047180755667	-3.1086997798198\\
0.654402882025848	-3.2499151684208\\
0.406108377881043	-3.36819391757691\\
0.165012567418598	-3.46480052850891\\
-0.0675648393948347	-3.54126761569399\\
-0.290772489374853	-3.59925732074702\\
-0.504151061163646	-3.640451598681\\
-0.70755343146076	-3.66647209103852\\
-0.901069341070793	-3.67882649143326\\
-1.08495903597741	-3.6788765088859\\
-1.25959768192474	-3.66782215616716\\
-1.42543057397612	-3.64669754584442\\
-1.58293813815797	-3.61637421068568\\
-1.73260924409234	-3.57756888264478\\
-1.87492122205869	-3.53085350535999\\
-2.01032505403617	-3.47666595152244\\
-2.13923438109869	-3.41532045672481\\
-2.26201717324251	-3.34701718184587\\
-2.3789891052139	-3.27185060417864\\
-2.49040785532301	-3.18981664267737\\
-2.59646768779841	-3.10081857168896\\
-2.69729379438629	-3.00467189329604\\
-2.79293596339599	-2.90110844015344\\
-2.88336122293346	-2.78978008465047\\
-2.96844518093896	-2.67026255030781\\
-3.04796187215268	-2.54205996954056\\
-3.12157203946577	-2.40461101806195\\
-3.18880994740434	-2.25729768628089\\
-3.24906907751574	-2.09945802108844\\
-3.30158742310312	-1.93040447377617\\
-3.34543361998514	-1.74944978628918\\
-3.37949585047893	-1.55594256956421\\
-3.40247634696937	-1.34931475687185\\
-3.4128953566482	-1.12914277315562\\
-3.40910947746106	-0.895223306830814\\
-3.38935006572105	-0.647662725689275\\
-3.35178750025212	-0.386976208024451\\
-3.29462585142282	-0.114188522962037\\
-3.21622928049041	0.169076531734969\\
-3.11527584916467	0.460535709474934\\
-2.99092658770654	0.757235257041731\\
-2.84298902572669	1.05560799962333\\
-2.67204759944406	1.35160972666908\\
-2.47953180906159	1.64093219927022\\
-2.26769940087452	1.91927262561283\\
-2.03952636078154	2.1826235633948\\
-1.79851474015648	2.42753895198215\\
-1.54844718051117	2.65133526219359\\
-1.29312730776821	2.85220102774083\\
-1.036144754668	3.02920829215682\\
-0.780693638554805	3.1822388174508\\
-0.529458403888326	3.31185067720291\\
-0.284566353757399	3.41911497970214\\
-0.0475957291863381	3.50544923079226\\
0.18037683768094	3.57246640440727\\
0.398706003313474	3.6218503749631\\
0.60709838598585	3.65526119870713\\
0.805534219195617	3.6742687731098\\
0.994196135033692	3.68031066273118\\
};
\addplot [color=mycolor1, forget plot]
  table[row sep=crcr]{%
1.02375040531617	3.61290204042887\\
1.20176097639549	3.60730045239441\\
1.37036957674768	3.59124369131131\\
1.53008538748134	3.56572536946192\\
1.68143780878129	3.53157387649472\\
1.82495211979655	3.48946246279246\\
1.96113166891982	3.43992080417065\\
2.09044497609687	3.38334668835468\\
2.2133163581027	3.32001698205598\\
2.33011892490929	3.25009741061438\\
2.44116901027864	3.17365094361483\\
2.54672128027864	3.09064476183925\\
2.64696390782588	3.0009559112872\\
2.74201331386758	2.90437585191723\\
2.83190806411648	2.8006142012692\\
2.91660158441303	2.68930207223581\\
2.99595343022608	2.56999552386096\\
3.06971893197026	2.44217979639812\\
3.13753715736318	2.30527519718943\\
3.19891731008915	2.15864574879307\\
3.25322395208505	2.00161200452337\\
3.29966183225684	1.83346976493728\\
3.33726166727844	1.65351675400746\\
3.36486898352328	1.46108956040721\\
3.38113910113713	1.25561318766688\\
3.38454247282657	1.03666518752547\\
3.3733857285037	0.804055302209507\\
3.34585461031475	0.557919500568181\\
3.30008499547509	0.298823996439621\\
3.23426669333464	0.0278702496733986\\
3.14678092088188	-0.253213474226809\\
3.03636585661394	-0.54201345615402\\
2.90229582008168	-0.835426867441588\\
2.74455015014256	-1.12973284012124\\
2.56394095281318	-1.42074875525535\\
2.36216843685569	-1.70406624141488\\
2.14178131502429	-1.97534198104143\\
1.90603727640003	-2.23060186778484\\
1.65868024595527	-2.46651004688063\\
1.40366973892964	-2.68056058438075\\
1.14490648466575	-2.87116725842797\\
0.885995072210844	-3.03764980487017\\
0.63007116427807	-3.18013500544318\\
0.379703604907356	-3.29940273110009\\
0.136866392078723	-3.3967091096898\\
-0.0970344061660405	-3.47361363503207\\
-0.321097449612501	-3.53182808856116\\
-0.534844239226391	-3.57309603977806\\
-0.738132419461264	-3.5991045027063\\
-0.931073626071126	-3.61142473381267\\
-1.11396114323217	-3.61147691260704\\
-1.28720951056091	-3.60051289518032\\
-1.45130614583367	-3.57961168548513\\
-1.60677385886146	-3.54968318739327\\
-1.75414258331887	-3.51147682884627\\
-1.89392851680262	-3.46559259474307\\
-2.02661895578985	-3.41249278629328\\
-2.15266131796345	-3.35251342660259\\
-2.27245508285067	-3.28587467397293\\
-2.38634561026136	-3.21268991747198\\
-2.49461899453625	-3.13297344723351\\
-2.59749727486108	-3.04664674493076\\
-2.69513344957367	-2.95354355339843\\
-2.78760584167458	-2.85341397942644\\
-2.87491144258746	-2.74592797770453\\
-2.95695793279452	-2.63067867162487\\
-3.0335541551112	-2.5071861012065\\
-3.10439891645093	-2.37490216114041\\
-3.16906813919749	-2.23321771157066\\
-3.22700060200421	-2.0814731142732\\
-3.2774828366435	-1.91897376091784\\
-3.31963422182456	-1.74501249358455\\
-3.35239397487088	-1.55890111606585\\
-3.37451261146532	-1.36001335729562\\
-3.38455150677088	-1.14784151232587\\
-3.38089535402736	-0.922068315058349\\
-3.36178335378977	-0.682654091339456\\
-3.32536546776439	-0.429936596356689\\
-3.26978941824833	-0.164736978685912\\
-3.1933215530012	0.111539791837042\\
-3.09449957081791	0.396827137223658\\
-2.97230730339444	0.688362960794408\\
-2.82635227231545	0.982721291257161\\
-2.65701807848947	1.27592551447855\\
-2.46555953737673	1.5636464007621\\
-2.25411243852087	1.84146980592806\\
-2.02560326847436	2.10520006333872\\
-1.78356466168654	2.35115257139234\\
-1.53188345629107	2.57638853279399\\
-1.27452259365677	2.77885734455166\\
-1.01526081426038	2.95743342231848\\
-0.757485202333998	3.11185660361474\\
-0.504055616196481	3.24260160764968\\
-0.257243124584251	3.35070882318175\\
-0.0187316848424999	3.43760661534103\\
0.210334443627698	3.50494770426823\\
0.429276150050858	3.55447278851852\\
0.637793727556865	3.58790627445286\\
0.835881553665213	3.6068830687125\\
1.02375040531617	3.61290204042887\\
};
\addplot [color=mycolor1, forget plot]
  table[row sep=crcr]{%
1.05659982423101	3.5436962097753\\
1.23338953524892	3.5381356863494\\
1.40036900034433	3.52223658455641\\
1.55809830241556	3.49703803845282\\
1.70715629008817	3.46340652800132\\
1.84811490673	3.42204718070233\\
1.98152076859	3.37351662858223\\
2.10788216348064	3.31823592487422\\
2.22765991767207	3.25650260540557\\
2.34126085936369	3.18850139185262\\
2.44903285812511	3.11431331704303\\
2.55126062779082	3.03392324367229\\
2.64816164416347	2.94722587889718\\
2.73988165365188	2.85403048404157\\
2.82648934309907	2.7540645618524\\
2.90796981500203	2.64697689049896\\
2.98421657841733	2.53234037859128\\
3.05502183896928	2.4096553521991\\
3.12006497023276	2.27835406514218\\
3.17889919744524	2.1378074573444\\
3.23093675399576	1.98733547741162\\
3.27543311994892	1.82622262905518\\
3.31147146435916	1.65374077137479\\
3.33794913309673	1.46918154206162\\
3.3535689795292	1.27190097051325\\
3.35683951089544	1.06137872310933\\
3.34608911292561	0.837293708330638\\
3.31950076128171	0.599616122860203\\
3.27517415625753	0.348713088675428\\
3.21122140990007	0.0854606072336505\\
3.12589939848037	-0.188651157300746\\
3.01777592707658	-0.471438308117994\\
2.88591787080211	-0.759992854618088\\
2.73007874651697	-1.05072513220936\\
2.550853793361	-1.33949555086233\\
2.34976699558954	-1.62183705326857\\
2.12926057137599	-1.89324869827937\\
1.89257432820553	-2.14952018695187\\
1.64352619912048	-2.38703493485975\\
1.38622833938899	-2.60300137176016\\
1.12478689562998	-2.79557870125465\\
0.863033287022711	-2.96388830141094\\
0.604321973242455	-3.10792613548574\\
0.351410388033882	-3.22840742802162\\
0.106418236504346	-3.32657959408979\\
-0.129149325311288	-3.40403480820078\\
-0.354335519552628	-3.46254393948293\\
-0.568643112585792	-3.50392309778006\\
-0.771939191240562	-3.5299354786536\\
-0.964364609538357	-3.54222559917473\\
-1.14625434955744	-3.54228020380336\\
-1.3180713423304	-3.53140935223569\\
-1.48035384250214	-3.51074166070109\\
-1.63367505571405	-3.48122869904976\\
-1.77861308794007	-3.44365471792408\\
-1.91572914317403	-3.39864895889987\\
-2.04555202384444	-3.34669868759013\\
-2.16856723967643	-3.28816176780067\\
-2.28520931521275	-3.22327808645115\\
-2.39585615477706	-3.15217948163206\\
-2.50082455396636	-3.07489805892091\\
-2.60036613240358	-2.99137293864105\\
-2.69466310590541	-2.90145558780369\\
-2.78382342447307	-2.80491397812694\\
-2.86787488513555	-2.70143589453793\\
-2.94675789701699	-2.59063181274185\\
-3.02031664354309	-2.4720378837501\\
-3.08828847005424	-2.34511972036689\\
-3.15029144544934	-2.20927788655847\\
-3.20581023122645	-2.06385625298164\\
-3.25418067517839	-1.90815470103258\\
-3.29457397191594	-1.74144801953847\\
-3.32598184364106	-1.56301320296016\\
-3.34720503075629	-1.37216764621141\\
-3.35684845374742	-1.16832079568533\\
-3.35332766149975	-0.951041435213579\\
-3.33489244351797	-0.720141650946591\\
-3.29967439518837	-0.475776270276219\\
-3.24576518889597	-0.218552900262679\\
-3.17133050273471	0.0503574420655213\\
-3.07476013326221	0.329125177290405\\
-2.95484728379189	0.615204118239174\\
-2.81097990419437	0.905334800269474\\
-2.6433164241271	1.19563026176681\\
-2.45291113117283	1.48175420485149\\
-2.24175529051304	1.75918285353335\\
-2.01271173396278	2.02352008155767\\
-1.76934182203281	2.27081800201832\\
-1.51564830327697	2.49784970293208\\
-1.25577690828014	2.70229053142342\\
-0.993726460657979	2.88278613974326\\
-0.733110174569144	3.03891122275533\\
-0.476993794902001	3.17104359983949\\
-0.22781652556404	3.28018863279807\\
0.0126152938222836	3.36778859164235\\
0.243080823721761	3.4355438678002\\
0.462864883201927	3.48526240878233\\
0.671664208211566	3.51874399905155\\
0.86949350295471	3.53769889760678\\
1.05659982423101	3.5436962097753\\
};
\addplot [color=mycolor1, forget plot]
  table[row sep=crcr]{%
1.09293589616732	3.47352416739806\\
1.26849674134541	3.46800521628701\\
1.4337963726869	3.45226881742003\\
1.58945551384844	3.42740359121524\\
1.73611105723776	3.39431658087325\\
1.87438896106105	3.35374610094779\\
2.00488523863961	3.30627618560696\\
2.1281529475438	3.25235098875703\\
2.24469342574854	3.19228814442275\\
2.35495035945798	3.12629055407149\\
2.45930556407635	3.05445637538651\\
2.55807560223686	2.97678718834121\\
2.6515085506791	2.89319444595849\\
2.73978036775067	2.80350440791691\\
2.82299041518313	2.70746182798343\\
2.90115576282974	2.60473273944519\\
2.97420396577533	2.49490677185651\\
3.04196406409613	2.37749955163049\\
3.10415563430801	2.25195590161078\\
3.16037584105218	2.11765477296779\\
3.21008462779589	1.97391712583436\\
3.25258848496511	1.82001832512479\\
3.28702369044987	1.65520702208577\\
3.31234058338146	1.47873290964321\\
3.32729135394503	1.28988608325564\\
3.33042502619716	1.08805085086839\\
3.32009472087404	0.872776463495758\\
3.294483716037	0.643866026764014\\
3.25165786063622	0.401482380030563\\
3.18965183998718	0.14626562297628\\
3.10659467762955	-0.120548837136475\\
3.00087467000739	-0.397029733799231\\
2.87133514004594	-0.680491177724248\\
2.71748067453288	-0.967502756385573\\
2.53966159761159	-1.25399141701655\\
2.3391970671875	-1.53544458360951\\
2.11839966185456	-1.80720207873993\\
1.88047959687191	-2.06479972792575\\
1.6293324405293	-2.304309147934\\
1.36924230153405	-2.52261466175929\\
1.10455236597759	-2.71758238549025\\
0.839358910264985	-2.88810324984138\\
0.577273118980962	-3.03402069412792\\
0.321273502343611	-3.15597497606478\\
0.0736493239613713	-3.25520446646513\\
-0.163980901938123	-3.33334094448337\\
-0.39059948088467	-3.39222550860005\\
-0.605691971291168	-3.43375953298264\\
-0.809141323680796	-3.4597947909185\\
-1.00112695072322	-3.47205997306241\\
-1.18203618954599	-3.47211728526765\\
-1.35239117552056	-3.46134177342606\\
-1.51279121505116	-3.44091649946167\\
-1.66386909976731	-3.41183787617206\\
-1.80625908431625	-3.37492682851767\\
-1.94057411061837	-3.33084269798519\\
-2.06739003805033	-3.28009782656068\\
-2.18723495344339	-3.22307152820622\\
-2.30058198023905	-3.1600227071581\\
-2.40784432622723	-3.09110075955144\\
-2.50937157959258	-3.01635464438066\\
-2.60544647762741	-2.93574017215796\\
-2.69628153531545	-2.84912566758599\\
-2.78201504058541	-2.75629624183129\\
-2.86270601000842	-2.65695698090751\\
-2.93832776493263	-2.55073543601379\\
-3.00875984686818	-2.43718390414292\\
-3.07377805825877	-2.31578212656641\\
-3.1330425106563	-2.18594122180271\\
-3.18608371292067	-2.04700991947117\\
-3.23228697158644	-1.89828447842317\\
-3.27087574721622	-1.73902405258735\\
-3.30089516523149	-1.56847368625895\\
-3.32119766893653	-1.38589751586511\\
-3.3304338617257	-1.19062500815314\\
-3.32705290150082	-0.982112972744464\\
-3.30931826690162	-0.760025345617177\\
-3.27534601764221	-0.524330951425817\\
-3.2231732680184	-0.275416197277719\\
-3.15086363930317	-0.0142046695419724\\
-3.05665291922995	0.257730879547194\\
-2.93913116800754	0.538085541447675\\
-2.7974470431082	0.823794593647349\\
-2.6315078179007	1.11108723868343\\
-2.44213824832725	1.3956391351759\\
-2.23115840500554	1.67282304544363\\
-2.00134937353323	1.93803265140908\\
-1.75629691134506	2.18703182923845\\
-1.50013135080415	2.41626989391422\\
-1.2372073006177	2.62310873735567\\
-0.971779315032066	2.80592927048401\\
-0.707725490017997	2.96411384119377\\
-0.448353209970221	3.09792740271686\\
-0.196298235066255	3.20833524870649\\
0.0464916587598717	3.29679723218257\\
0.278709355828535	3.36507081509076\\
0.499601885991418	3.41504341430274\\
0.708866665431189	3.44860294222669\\
0.906546832754389	3.46754676618677\\
1.09293589616732	3.47352416739806\\
};
\addplot [color=mycolor1, forget plot]
  table[row sep=crcr]{%
1.13302068675688	3.40313637942729\\
1.30735216851495	3.39765928538817\\
1.47092677483942	3.38209011309542\\
1.62443718206064	3.3575709367022\\
1.7685880866297	3.32505162330178\\
1.90406761415	3.28530464075765\\
2.03152780257099	3.23894145950706\\
2.15157173658891	3.18642873356713\\
2.26474533572043	3.12810319314109\\
2.37153220990307	3.06418469332344\\
2.47235035129693	2.99478719894024\\
2.56754971585957	2.91992769738027\\
2.6574099652991	2.83953316231655\\
2.74213779835695	2.7534457746188\\
2.82186341190383	2.66142666773136\\
2.89663570968929	2.56315852299072\\
2.96641593285578	2.45824741206679\\
3.03106943574792	2.34622438362293\\
3.09035538994984	2.22654743366613\\
3.14391429031312	2.09860469807455\\
3.19125328747383	1.96171997457725\\
3.23172961998089	1.81516202908499\\
3.26453281477108	1.65815956638465\\
3.28866692652729	1.48992422469463\\
3.30293495665252	1.30968442431625\\
3.30592877774352	1.11673323346959\\
3.29602938262015	0.910493382839516\\
3.27142395306259	0.690601814115529\\
3.23014776659083	0.457014216525408\\
3.17015968197974	0.210126377221273\\
3.08945885590666	-0.0490964927735401\\
2.98624619175463	-0.318998769080705\\
2.85912575571581	-0.597146044523929\\
2.70732901701836	-0.880299346639518\\
2.53093039362746	-1.16448129438572\\
2.33101101524119	-1.44515283591611\\
2.10972539158081	-1.71749742055327\\
1.87023802609507	-1.97678093821641\\
1.61652380359922	-2.21873030170006\\
1.35305954468467	-2.43986251032453\\
1.08446177591869	-2.63770588575249\\
0.815136413441826	-2.81088291357719\\
0.548996485816688	-2.95905851749179\\
0.289280196144789	-3.08278558254861\\
0.0384743594774492	-3.18329306991735\\
-0.201672979692613	-3.26226070706034\\
-0.430077793684999	-3.32161311525602\\
-0.646211432851508	-3.36335192736009\\
-0.849981410527051	-3.38943187493426\\
-1.04161760406947	-3.40167823750354\\
-1.22157284792635	-3.40173856002849\\
-1.39044147947608	-3.39106017274081\\
-1.54889585570102	-3.37088556392582\\
-1.69763890928502	-3.34225904988895\\
-1.83736999643794	-3.30603979435721\\
-1.9687611709434	-3.2629176975686\\
-2.09244126968614	-3.21342986163187\\
-2.20898559685683	-3.1579762239611\\
-2.31890942029672	-3.09683357292348\\
-2.42266387964347	-3.03016757652347\\
-2.52063322625479	-2.95804272260561\\
-2.61313256500743	-2.88043023576772\\
-2.70040545437466	-2.79721413980557\\
-2.78262085456982	-2.7081957039486\\
-2.85986900638624	-2.6130965685346\\
-2.93215588849615	-2.51156090882488\\
-2.99939595191547	-2.40315707968205\\
-3.06140288241337	-2.28737930319375\\
-3.11787821370082	-2.16365013027391\\
-3.16839773091541	-2.03132463968469\\
-3.212395798176	-1.88969764544751\\
-3.24914805925617	-1.73801557117162\\
-3.27775345127651	-1.57549510699851\\
-3.29711719989476	-1.40135125129047\\
-3.30593748850052	-1.2148377624613\\
-3.302699839467	-1.01530323260925\\
-3.28568485651335	-0.802265657690554\\
-3.25299663615765	-0.575507104717066\\
-3.2026203827126	-0.3351873569578\\
-3.13251772553263	-0.0819707903550062\\
-3.04076577657179	0.182845918157791\\
-2.92573984772423	0.457225623874559\\
-2.7863293168359	0.73832973233008\\
-2.62216232277006	1.02253511756338\\
-2.43380124155044	1.30555436200757\\
-2.22286316939781	1.58266832264814\\
-1.99202429442627	1.84905394833601\\
-1.74488709228206	2.10016182470575\\
-1.48572084946671	2.33207840649466\\
-1.21911829403769	2.54180703771746\\
-0.949631265581961	2.72742165398387\\
-0.681448676680737	2.88807983140649\\
-0.41816212230173	3.02391448273191\\
-0.162637548043815	3.1358447873217\\
0.0830137649937929	3.22535264164017\\
0.317387867035559	3.29426385042997\\
0.539693124365956	3.34455979598504\\
0.749633818797704	3.3782314325697\\
0.947292108645063	3.3971767483328\\
1.13302068675687	3.40313637942729\\
};
\addplot [color=mycolor1, forget plot]
  table[row sep=crcr]{%
1.17719733975312	3.33322396506955\\
1.35030249818025	3.32778891865049\\
1.51210809846406	3.31239138794923\\
1.6633919221548	3.28823087405312\\
1.80493755041589	3.25630209514873\\
1.93750430270129	3.21741226946189\\
2.06180738253903	3.17219990803701\\
2.17850539478485	3.12115312410239\\
2.28819294015768	3.06462632145195\\
2.3913965003925	3.00285469798302\\
2.48857225515869	2.93596636593233\\
2.58010480873367	2.86399211193033\\
2.66630605624465	2.7868729486618\\
2.74741359916657	2.7044656838033\\
2.82358824267196	2.61654677904956\\
2.89491018808665	2.52281481350526\\
2.96137358622634	2.42289191838283\\
3.02287915605375	2.31632462878386\\
3.07922461390126	2.20258471810196\\
3.13009272134828	2.08107075675129\\
3.17503687128785	1.95111138561789\\
3.21346432814191	1.81197163079968\\
3.24461756879648	1.66286401890881\\
3.26755469912438	1.50296677519668\\
3.28113072177179	1.33145196182033\\
3.2839825776602	1.14752694432796\\
3.27452241765133	0.950492869956774\\
3.25094543127714	0.739823575020654\\
3.21126052595	0.515267021592148\\
3.15335365277612	0.276968374618114\\
3.07509361635764	0.0256085736204264\\
2.97448735595804	-0.237455475206265\\
2.84988440258159	-0.510072370381124\\
2.70021767939671	-0.789231277761984\\
2.52525123765674	-1.07108606918227\\
2.32578937578246	-1.35109731252547\\
2.1037934386845	-1.62430061525686\\
1.8623602868618	-1.88567821596643\\
1.60554296119306	-2.13057743858749\\
1.33803325211188	-2.35509891946982\\
1.06476312386227	-2.55638059451322\\
0.790501560057985	-2.73273083095373\\
0.519517860924475	-2.88360440588051\\
0.255356471029191	-3.00945165230741\\
0.000735039895534709	-3.11149167834557\\
-0.242450364521467	-3.19146232558214\\
-0.473044209789634	-3.25138778815474\\
-0.690508675550763	-3.29338783178077\\
-0.894787541396454	-3.3195370096192\\
-1.08617634430937	-3.33177145469118\\
-1.2652096649494	-3.33183511510098\\
-1.43256970375565	-3.32125551528476\\
-1.58901598471793	-3.301339723258\\
-1.73533370960581	-3.2731828908662\\
-1.87229737577021	-3.23768367258399\\
-2.00064620563421	-3.19556257903416\\
-2.1210682989065	-3.14738071624703\\
-2.23419094197467	-3.09355738515112\\
-2.34057504222673	-3.03438572111461\\
-2.44071212577514	-2.97004601408423\\
-2.53502271940783	-2.90061663612471\\
-2.6238552305647	-2.82608267277113\\
-2.70748465315566	-2.7463424518198\\
-2.78611057654608	-2.66121222069601\\
-2.85985407502429	-2.57042926552956\\
-2.92875311990893	-2.47365381018884\\
-2.99275620019862	-2.37047009744903\\
-3.05171387521637	-2.26038715189157\\
-3.10536803169253	-2.14283987010183\\
-3.15333870101788	-2.01719129415084\\
-3.19510844104602	-1.88273721512157\\
-3.23000454330044	-1.738714637411\\
-3.25717974709943	-1.58431611467405\\
-3.27559279693591	-1.41871252496896\\
-3.28399114506574	-1.24108742257173\\
-3.28089944041944	-1.05068654847698\\
-3.26461916262123	-0.846886146933993\\
-3.23324672514285	-0.629283014845219\\
-3.18471920149778	-0.397807139638174\\
-3.11689775481201	-0.152853719678765\\
-3.02769764519051	0.104575151663589\\
-2.91526879688354	0.372737614777059\\
-2.7782210069923	0.649055916963807\\
-2.61587298585537	0.930091699417381\\
-2.42848727696386	1.21162640459507\\
-2.21743989233247	1.48886716772943\\
-1.98527245900548	1.75677187288989\\
-1.73559192587236	2.01045298137057\\
-1.47281714099461	2.24559094462114\\
-1.20181194350925	2.4587783360375\\
-0.92747448166092	2.64773182655615\\
-0.654359680035902	2.81134496592553\\
-0.386394751691459	2.94959521024363\\
-0.126716165951527	3.06334822411246\\
0.122375429000432	3.15411339977998\\
0.359367735470225	3.22379856286806\\
0.583430829221697	3.27449645974263\\
0.794284661203453	3.30831872767862\\
0.992064213675017	3.32727966867312\\
1.17719733975312	3.33322396506955\\
};
\addplot [color=mycolor1, forget plot]
  table[row sep=crcr]{%
1.22590434247989	3.26444255553571\\
1.39778568519362	3.2590497847381\\
1.55777518892862	3.24382861499667\\
1.70675103836143	3.22003994081109\\
1.84558826292861	3.18872508758141\\
1.97512733776895	3.15072621278902\\
2.09615428269976	3.10670801538788\\
2.20938891075182	3.05717857651015\\
2.31547857593696	3.00250813821333\\
2.41499540189816	2.94294526954143\\
2.50843549202283	2.87863026509742\\
2.59621902113075	2.80960585033293\\
2.67869040132478	2.73582539079166\\
2.75611791888507	2.65715886373023\\
2.82869237462659	2.5733968811675\\
2.89652434460227	2.48425307600197\\
2.9596397271147	2.38936519468617\\
3.01797327033446	2.28829529598714\\
3.07135979769443	2.1805295503353\\
3.11952288389663	2.06547828426429\\
3.16206080708689	1.94247713761692\\
3.19842974646615	1.81079051711406\\
3.22792445709442	1.66961895656359\\
3.24965710022171	1.51811254070199\\
3.26253562201684	1.35539320285037\\
3.26524415210251	1.18058940520403\\
3.25622942186491	0.992887307773702\\
3.23369920606965	0.791602743226298\\
3.19564113848408	0.576277664196529\\
3.1398725068913	0.346802528699445\\
3.06413287471086	0.103561521425096\\
2.96623008252873	-0.152410070565631\\
2.84424437601905	-0.419277452491931\\
2.69678336750397	-0.694299485893786\\
2.52326227368228	-0.973804263907594\\
2.32416298853439	-1.25328660214706\\
2.10121018814479	-1.52765009372196\\
1.85740346435531	-1.79158308598752\\
1.59686869882091	-2.04001649265165\\
1.32453632558024	-2.26857849595714\\
1.04570301329607	-2.4739540329236\\
0.765565360812775	-2.65408168115743\\
0.488815500179686	-2.80816699289228\\
0.219360997648381	-2.93653887741222\\
-0.0398095987651686	-3.04040602242973\\
-0.286630899201536	-3.12157692434183\\
-0.519871267646057	-3.18219496718836\\
-0.738991605563387	-3.22451960078847\\
-0.943987665821066	-3.25076517916732\\
-1.13524008710841	-3.26299522676806\\
-1.3133853463107	-3.26306258127946\\
-1.47921241980635	-3.25258358098722\\
-1.63358462390113	-3.23293521265411\\
-1.77738335876483	-3.20526623833188\\
-1.9114694962231	-3.17051570223822\\
-2.03665819186413	-3.12943433562294\\
-2.15370343312798	-3.08260603329976\\
-2.26328932688233	-3.03046776553468\\
-2.36602580463572	-2.9733270897908\\
-2.4624470016689	-2.91137693497489\\
-2.55301102484149	-2.84470763450985\\
-2.63810016768621	-2.77331635409199\\
-2.71802087730646	-2.69711414748705\\
-2.79300294516599	-2.61593091635425\\
-2.86319750178541	-2.52951857428171\\
-2.92867346026797	-2.43755274047984\\
-2.98941209046259	-2.33963333071361\\
-3.04529942914146	-2.23528448666135\\
-3.09611625826001	-2.12395440523951\\
-3.14152543401717	-2.00501581362534\\
-3.18105645264084	-1.87776810279971\\
-3.21408733492589	-1.7414425017944\\
-3.23982425672907	-1.59521216135885\\
-3.25727992165527	-1.43820961964884\\
-3.26525255654388	-1.26955481037257\\
-3.26230870533485	-1.08839744737734\\
-3.24677476675413	-0.893978070280039\\
-3.21674442802203	-0.685711889067924\\
-3.17011152831026	-0.463298237743598\\
-3.10463978567584	-0.226855164446353\\
-3.01808102245287	0.0229273383947193\\
-2.9083502301115	0.284631072839721\\
-2.77375704158983	0.555977281428288\\
-2.61327784344003	0.833755689427438\\
-2.42683244890199	1.11385674149264\\
-2.2155098612699	1.3914402852622\\
-1.98167908157629	1.66124848376536\\
-1.72893302723864	1.91803170286154\\
-1.46184919114113	2.15701658680652\\
-1.18559991910093	2.37432382629505\\
-0.905488161095217	2.56725215472164\\
-0.626502063598137	2.73438287412226\\
-0.352967397025358	2.87550894157846\\
-0.0883401830366873	2.99143329249239\\
0.164858930307504	3.08369955520424\\
0.404996785704091	3.1543143820327\\
0.631207403458537	3.20550301505091\\
0.843238767849654	3.23951886961585\\
1.04129670838009	3.25851096417827\\
1.22590434247988	3.26444255553571\\
};
\addplot [color=mycolor1, forget plot]
  table[row sep=crcr]{%
1.27969473451899	3.19743891413949\\
1.45034994159423	3.19208882280089\\
1.608468718383	3.17704946054004\\
1.75504743713589	3.15364704231165\\
1.8910667842996	3.12297092862\\
2.01745932978799	3.08589794332656\\
2.13509003936657	3.04311762731975\\
2.24474574897865	2.99515607355915\\
2.34713052878886	2.9423971209665\\
2.44286465667115	2.88510040147726\\
2.53248555223779	2.82341615911886\\
2.61644949636217	2.75739699170234\\
2.6951332998723	2.68700677794949\\
2.76883531595009	2.61212709684994\\
2.8377753393484	2.53256145660749\\
2.90209302353936	2.44803765162519\\
2.96184449223774	2.35820857506451\\
3.01699683974581	2.2626518460952\\
3.06742021994134	2.16086867923668\\
3.11287723304133	2.05228254415422\\
3.15300935413325	1.93623835707118\\
3.18732023860967	1.81200323241128\\
3.21515593194735	1.67877023059018\\
3.23568236933861	1.53566708793343\\
3.24786116349761	1.38177261724521\\
3.25042565989046	1.21614429781399\\
3.24186071298692	1.03786143510928\\
3.22039170350251	0.846088937327736\\
3.18399096149264	0.640166798123835\\
3.13041270376352	0.41972908786561\\
3.05727006314035	0.184852660212058\\
2.9621682737045	-0.0637711837261371\\
2.84290427252501	-0.324659745035805\\
2.69773225686422	-0.595388289159807\\
2.5256756024321	-0.872510538970974\\
2.32684009585234	-1.15160055593432\\
2.10265963161525	-1.42745523771516\\
1.85599667089474	-1.69446398913854\\
1.59103854112513	-1.94710290900718\\
1.31297908263676	-2.18046319213452\\
1.02753779962426	-2.39070140666667\\
0.740417862665125	-2.57531756857995\\
0.456816757094041	-2.73321902000345\\
0.18107562986538	-2.8645894324229\\
-0.0834972651340051	-2.97062633628157\\
-0.33464256235706	-3.05322456841679\\
-0.571049054307578	-3.11467098787108\\
-0.79218830185694	-3.15739101298888\\
-0.998129124263412	-3.18376270669748\\
-1.18936222632091	-3.19599631795903\\
-1.36665104937862	-3.19606775514251\\
-1.53091423172332	-3.1856915971373\\
-1.68313847021417	-3.16632026979512\\
-1.82431733797418	-3.13915871316964\\
-1.95541060374083	-3.10518684935743\\
-2.07731882889597	-3.06518475312069\\
-2.19086879991828	-3.01975740852929\\
-2.296806278529	-2.96935732324368\\
-2.39579341215939	-2.91430418010914\\
-2.48840886072844	-2.85480126234141\\
-2.57514924628002	-2.79094870596757\\
-2.65643093499139	-2.72275379750876\\
-2.73259144246289	-2.65013860775138\\
-2.80388993994359	-2.57294527616556\\
-2.87050645497965	-2.4909392638613\\
-2.93253942459908	-2.40381089613441\\
-2.99000128898612	-2.31117553424823\\
-3.04281182340003	-2.21257276418519\\
-3.09078891113382	-2.10746508255407\\
-3.13363647922882	-1.99523671409207\\
-3.17092937706462	-1.87519343226531\\
-3.20209511295751	-1.74656459867964\\
-3.22639262956776	-1.60850911330203\\
-3.24288877161496	-1.46012759407702\\
-3.25043388083961	-1.30048387741061\\
-3.24763916545357	-1.12863979540267\\
-3.232860253651	-0.943707989005588\\
-3.20419371017733	-0.744927940467152\\
-3.15949614900126	-0.531769892733092\\
-3.09643841910547	-0.304069017672216\\
-3.01260905935427	-0.062187036124715\\
-2.90567988916881	0.192810438976774\\
-2.77363964444937	0.458985248496227\\
-2.61508666850671	0.733405387358675\\
-2.4295487824329	1.01212011812496\\
-2.21777243410176	1.29027878622215\\
-1.98190506393249	1.56241892651132\\
-1.72549842501278	1.82290682446117\\
-1.45329498813935	2.06646233416912\\
-1.17081808859522	2.28866210291941\\
-0.883846036343302	2.48631281300606\\
-0.597883133052734	2.65762344314827\\
-0.317731837641074	2.8021658944279\\
-0.0472280991005781	2.92066872817214\\
0.210850791495053	3.01471826027117\\
0.454737366582832	3.08644089581483\\
0.683534623061324	3.13822035187157\\
0.89703583690876	3.17247731940366\\
1.09554128372098	3.19151731174063\\
1.27969473451899	3.19743891413949\\
};
\addplot [color=mycolor1, forget plot]
  table[row sep=crcr]{%
1.3392619362044	3.13288106114262\\
1.5086791795128	3.12757437885239\\
1.6648604132956	3.11272344007181\\
1.80894084011687	3.08972360883537\\
1.94202280659676	3.05971330019967\\
2.06514290229219	3.02360320058587\\
2.1792535778672	2.98210580929765\\
2.28521452921726	2.93576279359985\\
2.3837902847826	2.88496894959092\\
2.47565142870222	2.82999235401166\\
2.56137766034903	2.77099073935933\\
2.64146145066806	2.70802435097411\\
2.71631144520293	2.6410656385359\\
2.78625502182879	2.5700061549625\\
2.85153957186433	2.49466102162765\\
2.91233216336754	2.41477129524368\\
2.96871728581953	2.33000455594079\\
3.02069238245767	2.23995404162502\\
3.06816086448863	2.14413669371357\\
3.11092228510539	2.04199056894032\\
3.14865934933596	1.93287223022889\\
3.18092147510529	1.81605498124403\\
3.2071047420893	1.6907291845516\\
3.22642832936629	1.55600643641962\\
3.23790804050797	1.41093009280062\\
3.24032837021464	1.2544955614675\\
3.23221593702453	1.08568485209026\\
3.21181915823305	0.903520955896151\\
3.17710188324506	0.707148356743944\\
3.12576223284096	0.495945692402419\\
3.05529155810392	0.269674225576345\\
2.96309086907791	0.0286599408519614\\
2.84666081340954	-0.226003578863136\\
2.70387285481623	-0.492260043094357\\
2.5333104918282	-0.766949791157519\\
2.33464000483891	-1.0457837492775\\
2.10893700227236	-1.3234898634351\\
1.85887326877483	-1.59416127410092\\
1.58867727353884	-1.85178014035832\\
1.30383136652625	-2.09082616151757\\
1.01054697329997	-2.30683587515119\\
0.715131634073073	-2.49678465263894\\
0.423391948519504	-2.6592193276381\\
0.140191149224581	-2.79414784538079\\
-0.13078263680191	-2.90275563289121\\
-0.387047456258826	-2.98704378778889\\
-0.627211121354136	-3.04947311479465\\
-0.850773576984367	-3.09266743431044\\
-1.05790511290341	-3.11919740795147\\
-1.24923868994443	-3.13144278304647\\
-1.42569600205082	-3.13151872715371\\
-1.58835308692518	-3.12124838514936\\
-1.73834309104347	-3.10216529525421\\
-1.87679002952083	-3.0755328613566\\
-2.00476644935078	-3.04237188331492\\
-2.12326846130743	-3.00349035732629\\
-2.23320275723049	-2.95951214245916\\
-2.33538147355553	-2.91090271021194\\
-2.43052186422163	-2.85799121830943\\
-2.51924862748709	-2.80098875199537\\
-2.60209739095236	-2.74000289964855\\
-2.67951832872522	-2.67504898055312\\
-2.75187920382206	-2.60605829388496\\
-2.81946733449019	-2.53288375738476\\
-2.88249010557909	-2.45530328282913\\
-2.94107370907848	-2.3730212138732\\
-2.9952598198123	-2.28566814510265\\
-3.04499990804278	-2.19279946220562\\
-3.09014687456381	-2.09389300588201\\
-3.13044368203036	-1.98834638334865\\
-3.16550867093336	-1.87547465280863\\
-3.19481732271165	-1.75450941559799\\
-3.21768041602521	-1.62460080063886\\
-3.23321888992378	-1.48482445001928\\
-3.24033638637087	-1.33419643647127\\
-3.23769153638757	-1.17170005268329\\
-3.22367374567674	-0.996329519905232\\
-3.196388676233	-0.807156629135423\\
-3.15366283961751	-0.603426658440192\\
-3.09308042758429	-0.384688747459116\\
-3.01206879249482	-0.150961996233866\\
-2.90804995608205	0.0970695671635621\\
-2.7786711381308	0.357853290762011\\
-2.62211397341698	0.628793108926508\\
-2.43745768365259	0.906158296836139\\
-2.22503879901498	1.1851375243136\\
-1.98672015936148	1.46008544100662\\
-1.72597326286571	1.72496609922759\\
-1.44770719319412	1.97393409200938\\
-1.15784449555035	2.20193615907363\\
-0.862724864969265	2.40519530789025\\
-0.5684724898963	2.58147229590912\\
-0.280464895163095	2.7300712736549\\
-0.0029932945470816	2.85163151150354\\
0.260864143836346	2.94779278933954\\
0.509191499615886	3.02082769660696\\
0.741070000077793	3.07331076844402\\
0.956362264850219	3.10786112278658\\
1.15549403945334	3.12696676719541\\
1.3392619362044	3.13288106114262\\
};
\addplot [color=mycolor1, forget plot]
  table[row sep=crcr]{%
1.4054745831124	3.07149337387774\\
1.57362727100623	3.06623132298505\\
1.72778699106429	3.05157706491179\\
1.8692516647565	3.02899874981702\\
1.99926321162977	2.99968435119239\\
2.11897510688696	2.96457699823885\\
2.22943611038759	2.92440967730864\\
2.33158448622435	2.87973669499959\\
2.42624859489868	2.83096077860544\\
2.51415099035098	2.77835555376001\\
2.59591408402611	2.72208359692615\\
2.67206609429429	2.66221046842641\\
2.74304644182516	2.59871519613369\\
2.80921003520959	2.53149766884962\\
2.87083006084297	2.46038335395687\\
2.92809898034762	2.38512570196443\\
2.98112747194105	2.30540655755974\\
3.02994104686825	2.22083487486481\\
3.07447404219802	2.13094404515007\\
3.1145606499368	2.03518820178653\\
3.14992260504253	1.93293798778814\\
3.1801531438731	1.82347648080454\\
3.20469689494574	1.70599630232067\\
3.22282553140833	1.57959943447058\\
3.23360938613093	1.44330197737494\\
3.23588593502282	1.29604704519222\\
3.22822727176805	1.1367302319388\\
3.20891065145785	0.964243506987811\\
3.17589909545075	0.777544777654567\\
3.12684302710255	0.575761118613226\\
3.05911867262978	0.358332765458058\\
2.96992343272682	0.125200762736179\\
2.85645018416743	-0.122968392686797\\
2.71615742706347	-0.384544218725802\\
2.54713532170004	-0.656725403355817\\
2.34853577548911	-0.935432686045723\\
2.12099170299253	-1.21537911606742\\
1.86691228769943	-1.49037570040727\\
1.59053375100839	-1.75387251044781\\
1.29765092469468	-1.99965166553425\\
0.995049756673487	-2.22251711528282\\
0.689764855128191	-2.41881039084985\\
0.388344336714448	-2.58663740439976\\
0.0962863203254775	-2.72579076860021\\
-0.182266134892198	-2.83744263071393\\
-0.444575305476755	-2.92372607293893\\
-0.689170200651332	-2.98731462456856\\
-0.915605230675081	-3.03107101158091\\
-1.124190568135	-3.05779374085243\\
-1.3157431266952	-3.07005907314908\\
-1.49138202257451	-3.07013998665209\\
-1.65237434032767	-3.05997949315592\\
-1.80002680080641	-3.04119800758308\\
-1.93561465998037	-3.0151192949233\\
-2.06033850655863	-2.98280444559781\\
-2.17530072283779	-2.94508734198385\\
-2.28149506482953	-2.90260795655498\\
-2.37980449691982	-2.85584170843552\\
-2.4710038297239	-2.80512424893919\\
-2.55576479613425	-2.75067168453415\\
-2.6346619863128	-2.69259656245031\\
-2.70817860414535	-2.63092006979291\\
-2.77671136436757	-2.56558091705725\\
-2.84057407142502	-2.4964413451172\\
-2.8999995470633	-2.42329064404795\\
-2.95513963227719	-2.34584652296658\\
-3.00606300125895	-2.26375463621684\\
-3.05275050582436	-2.17658656405072\\
-3.09508773151852	-2.08383657758257\\
-3.13285440499613	-1.9849176037046\\
-3.16571026463317	-1.87915696707727\\
-3.19317702051026	-1.76579275225261\\
-3.21461613019455	-1.64397203747039\\
-3.2292023730239	-1.51275284866754\\
-3.23589372448983	-1.37111251535963\\
-3.23339896779886	-1.217966211236\\
-3.22014603594558	-1.05220081037912\\
-3.19425648770722	-0.872730640045606\\
-3.15353496995275	-0.678582877771494\\
-3.09548696227414	-0.469020438077985\\
-3.01738292443531	-0.243707879537554\\
-2.91639048564559	-0.00291924116312545\\
-2.78979527494392	0.252226179261229\\
-2.63532064441046	0.519533917556674\\
-2.45153225954353	0.795567761858333\\
-2.23827491409184	1.07562197911916\\
-1.99704545399365	1.35390475240838\\
-1.73117936849501	1.62396651620491\\
-1.44574621356274	1.87933276613553\\
-1.14712218591709	2.11421751592803\\
-0.842314357326939	2.32414547803164\\
-0.538198389740975	2.50633192203147\\
-0.240852599292319	2.65975270557451\\
0.0448812356255614	2.78493845106198\\
0.315570224341204	2.88359640994968\\
0.569135749148115	2.9581792577659\\
0.804652925319656	3.01149314457779\\
1.02208728745503	3.04639408810963\\
1.22203103342413	3.06558388836308\\
1.4054745831124	3.07149337387774\\
};
\addplot [color=mycolor1, forget plot]
  table[row sep=crcr]{%
1.47942380987548	3.01409903169868\\
1.64626455039693	3.00888351732135\\
1.79829624689677	2.9944363478465\\
1.93700704846841	2.97230176776496\\
2.06379831826375	2.94371715953913\\
2.1799540989066	2.90965595947843\\
2.28662837119145	2.87086852888969\\
2.38484333686672	2.82791837021459\\
2.4754940038472	2.78121274986893\\
2.55935591460565	2.73102770910772\\
2.63709397460015	2.67752789168202\\
2.70927109620344	2.62078178779733\\
2.77635586641289	2.56077301411612\\
2.83872875041704	2.49740819628071\\
2.89668651596338	2.43052193867583\\
2.95044464676317	2.35987928153766\\
3.0001375352307	2.28517597293728\\
3.04581622485987	2.20603683256147\\
3.08744342398849	2.12201246457777\\
3.12488544666132	2.03257459934331\\
3.15790066486958	1.93711042432436\\
3.18612399696444	1.83491642705397\\
3.20904693798338	1.72519255157135\\
3.22599270701752	1.60703791200326\\
3.23608632413714	1.47944997405672\\
3.23821996159076	1.34133007776124\\
3.23101493250038	1.19149949585151\\
3.21278345253694	1.0287319156817\\
3.1814961642855	0.851810177806806\\
3.13476565527705	0.659616892903152\\
3.06986188544866	0.451269262738913\\
2.9837819155047	0.226306340358796\\
2.87340149427001	-0.0150705104270993\\
2.73573556174386	-0.271718997598551\\
2.56832184339207	-0.54127972878473\\
2.36970959446515	-0.819974718732148\\
2.1399833195045	-1.10257758631277\\
1.88119288598873	-1.38264819930625\\
1.59752959516814	-1.65307042307391\\
1.29512089242035	-1.90682964704465\\
0.981426447742972	-2.13785775176816\\
0.66436393646992	-2.34172200086941\\
0.351394605016278	-2.51598196277505\\
0.0487974547586737	-2.6601626822905\\
-0.23873474837032	-2.77542164655584\\
-0.508170243751502	-2.86405775042834\\
-0.757967507119653	-2.92900735035789\\
-0.98777367530101	-2.9734232879994\\
-1.1980910269562	-3.00037532671187\\
-1.38997469205245	-3.01266848879099\\
-1.56479036970873	-3.01275487347464\\
-1.72403700215334	-3.00270968394413\\
-1.86922667587088	-2.98424595906764\\
-2.00180940524786	-2.95874918876373\\
-2.12313041210999	-2.92731945832026\\
-2.23440947949695	-2.89081381148429\\
-2.33673442944079	-2.84988499437422\\
-2.43106303818853	-2.80501492086142\\
-2.5182295086471	-2.75654245540378\\
-2.59895294951886	-2.70468576332112\\
-2.67384623600738	-2.64955976796782\\
-2.74342424195536	-2.59118933749998\\
-2.80811082315821	-2.52951880014098\\
-2.86824416379404	-2.46441831516597\\
-2.92408022172711	-2.39568754127205\\
-2.97579405823449	-2.32305696396324\\
-3.02347883677504	-2.24618718088355\\
-3.06714223968626	-2.16466640771873\\
-3.10669999303777	-2.07800646707122\\
-3.14196611940757	-1.98563757183635\\
-3.17263947018963	-1.88690233258323\\
-3.19828604522639	-1.78104963392229\\
-3.21831662507747	-1.66722937843925\\
-3.23195938213613	-1.54448964299202\\
-3.2382275020994	-1.41177859804979\\
-3.23588259406218	-1.26795467637459\\
-3.22339602477535	-1.11180998860894\\
-3.198912583081	-0.94211383125201\\
-3.16022439810807	-0.757685076645846\\
-3.10476802094913	-0.557503640031831\\
-3.02966381164673	-0.340870796261394\\
-2.93182298174123	-0.107623705902376\\
-2.80815068243951	0.141601790059955\\
-2.65586777743815	0.405086766923669\\
-2.47295211775567	0.679779419408755\\
-2.25865735059234	0.961166577758489\\
-2.01400900809958	1.24336659938685\\
-1.7421274693308	1.51951626565132\\
-1.44822398926211	1.78244372089397\\
-1.13918912623464	2.02550654974069\\
-0.822829286010876	2.24338606724704\\
-0.506941025352666	2.43262553182077\\
-0.198466686990418	2.59179277046852\\
0.0970790994492219	2.72128431288835\\
0.375842686436026	2.82289348564613\\
0.635569619905188	2.89929725677801\\
0.87535433088247	2.95358556276458\\
1.09531230173884	2.98889935870533\\
1.29625660842899	3.00819221393258\\
1.47942380987548	3.01409903169868\\
};
\addplot [color=mycolor1, forget plot]
  table[row sep=crcr]{%
1.56248802181173	2.9616734001184\\
1.72794160058687	2.95650722867695\\
1.87771037776087	2.94228025725477\\
2.01350416846367	2.9206156108536\\
2.13690554077677	2.89279910679247\\
2.24934335930265	2.85983152504823\\
2.3520855292164	2.8224767991473\\
2.44624293856636	2.781303675091\\
2.5327792614432	2.73672023362352\\
2.61252321755089	2.68900161357266\\
2.68618120369028	2.63831167779904\\
2.75434907197669	2.5847194697673\\
2.81752236361814	2.52821126532338\\
2.87610462006534	2.46869891581417\\
2.93041355954928	2.40602505151122\\
2.98068497658418	2.33996559239673\\
3.02707422747102	2.27022990888279\\
3.06965512668631	2.19645889483955\\
3.10841601017396	2.1182211652924\\
3.14325263091391	2.03500757937128\\
3.17395744815529	1.94622432861669\\
3.20020476631742	1.85118494273159\\
3.22153109376454	1.74910178229258\\
3.23731006382553	1.63907795993805\\
3.24672135907247	1.52010122767342\\
3.24871342237838	1.3910422798955\\
3.24196051612137	1.25066125707407\\
3.22481619765923	1.09762809066295\\
3.19526792680432	0.930564723553199\\
3.15090180614554	0.748119965368382\\
3.08889279489712	0.549090102380487\\
3.00604403733272	0.332598796784509\\
2.89890774126572	0.0983453980010447\\
2.76402519326072	-0.153082636643624\\
2.59831705952781	-0.419864091461348\\
2.39962622519607	-0.698635848076194\\
2.16735629533455	-0.98433558391937\\
1.90306755215678	-1.27032769523082\\
1.6108251747217	-1.54890500107748\\
1.29710068781251	-1.81214296080079\\
0.970146867704863	-2.0529270739582\\
0.638965572146003	-2.26586713629577\\
0.312157381743099	-2.44783584723404\\
-0.00302571524577319	-2.59802067678299\\
-0.301220094054039	-2.71756300043051\\
-0.579056423883941	-2.80897290177942\\
-0.8349411321204	-2.87551534463542\\
-1.06867026275776	-2.92069886383828\\
-1.28100962445009	-2.94791844465975\\
-1.47332349419031	-2.96024657501622\\
-1.6472859563713	-2.96033897061529\\
-1.80467722973043	-2.95041637068623\\
-1.94725182593645	-2.93228999711988\\
-2.07666084446663	-2.90740770495249\\
-2.19441188468987	-2.87690642664304\\
-2.30185338601962	-2.84166287301235\\
-2.40017378689372	-2.80233862274819\\
-2.4904089277545	-2.75941821344266\\
-2.57345341291102	-2.7132401876885\\
-2.6500732538513	-2.66402168351149\\
-2.72091818762069	-2.61187739006263\\
-2.78653274547894	-2.55683370715729\\
-2.84736555949864	-2.49883886357561\\
-2.90377662674676	-2.43776962713736\\
-2.95604236355122	-2.37343511295737\\
-3.00435831666581	-2.30557808202369\\
-3.04883937980577	-2.23387402918455\\
-3.08951730923333	-2.15792829371994\\
-3.12633525131421	-2.07727139352432\\
-3.15913889648398	-1.99135279573921\\
-3.18766376747544	-1.89953340913903\\
-3.21151805059608	-1.80107724325244\\
-3.23016031570991	-1.69514296658646\\
-3.24287149488152	-1.58077657029018\\
-3.24872069156883	-1.45690708487812\\
-3.24652492274259	-1.32234840752889\\
-3.23480399458236	-1.17581188348314\\
-3.21173373099997	-1.01593641696128\\
-3.17510417730993	-0.841345494965563\\
-3.12229467922138	-0.650743181776849\\
-3.05028512889501	-0.443062800463219\\
-2.95573153358211	-0.217680470676936\\
-2.835141774389	0.0253026699225968\\
-2.68518801965275	0.284725337264344\\
-2.50317597893971	0.558027535137331\\
-2.28764751658583	0.841001509445815\\
-2.03902032112432	1.12776023726286\\
-1.76008767943735	1.41104522436188\\
-1.45616345687017	1.6829170560622\\
-1.13471863967304	1.93572767581246\\
-0.804524934801826	2.16312908782645\\
-0.474521414348072	2.36082533570033\\
-0.152730019317042	2.52686942081269\\
0.154502462670652	2.66148988749968\\
0.44282014922353	2.76659143926723\\
0.709783032080508	2.84513401290853\\
0.954545295336056	2.90055901688932\\
1.17743860785017	2.93635299009173\\
1.37956959771753	2.95576769492973\\
1.56248802181173	2.9616734001184\\
};
\addplot [color=mycolor1, forget plot]
  table[row sep=crcr]{%
1.65642263207936	2.91541373390173\\
1.82037784583778	2.91030086390921\\
1.9677141607887	2.89631048846228\\
2.10039858637834	2.87514661502634\\
2.22021829502882	2.84814149164205\\
2.32876139289674	2.81631933318551\\
2.42741778412894	2.78045311346391\\
2.51739080453339	2.74111237919629\\
2.5997137273383	2.69870202037562\\
2.67526759617419	2.65349283915425\\
2.74479836712207	2.60564506995236\\
2.80893228619176	2.55522600914459\\
2.86818898383772	2.50222278100859\\
2.92299207190501	2.44655108756695\\
2.97367717316179	2.38806060833533\\
3.02049735848801	2.32653755200114\\
3.06362594816039	2.26170472357171\\
3.1031565727605	2.19321936030627\\
3.13910029784126	2.12066890980134\\
3.17137950106921	2.04356487823619\\
3.19981805563009	1.96133487528602\\
3.22412722545868	1.87331304194841\\
3.24388652917448	1.77872919811096\\
3.25851870767061	1.67669733555935\\
3.26725788924465	1.56620458170098\\
3.26911018723029	1.44610257455019\\
3.26280646739801	1.3151044624267\\
3.24674818526298	1.17179264275761\\
3.21894948375011	1.01464503971421\\
3.17698281835873	0.842091215225842\\
3.11794201723986	0.652613547633571\\
3.03844644930004	0.444911865172639\\
2.93472234736788	0.218149481119258\\
2.8028090465519	-0.0277104952854724\\
2.63894062157472	-0.291494298734732\\
2.44013251083137	-0.570393290038135\\
2.20494150359352	-0.859649139467993\\
1.93426251361171	-1.1525223052927\\
1.63191099575297	-1.44070778788784\\
1.30469689836056	-1.71524417896233\\
0.961809545903871	-1.96775111774133\\
0.613598070559528	-2.19163828849998\\
0.27010603400576	-2.3829008079898\\
-0.0601842625425439	-2.54029310737039\\
-0.371081514574717	-2.66493929156151\\
-0.658830666399525	-2.75962282515189\\
-0.921821623365127	-2.82802513033402\\
-1.16008213079778	-2.87409551910905\\
-1.37473982056242	-2.90162188662688\\
-1.56756118680114	-2.91399083853101\\
-1.74060641809601	-2.91408981741843\\
-1.89599662931821	-2.90429937494904\\
-2.03577159460937	-2.88653403208306\\
-2.16181254714747	-2.86230368386328\\
-2.27580800711821	-2.83277894819267\\
-2.37924602595903	-2.79885186349478\\
-2.47342135895941	-2.76118829145959\\
-2.55945010266753	-2.72027114215685\\
-2.63828719565115	-2.67643490915497\\
-2.71074408814828	-2.6298925648727\\
-2.77750509407517	-2.58075599703199\\
-2.83914166821686	-2.52905108890512\\
-2.89612426683683	-2.47472838286705\\
-2.94883166501475	-2.41767008273216\\
-2.99755769310528	-2.35769397601212\\
-3.04251536440935	-2.2945547054575\\
-3.08383832442057	-2.22794269467252\\
-3.12157947480033	-2.15748093717595\\
-3.15570652122286	-2.08271979484268\\
-3.18609406848149	-2.00312992664664\\
-3.21251174361749	-1.91809349492053\\
-3.23460767710368	-1.82689389684362\\
-3.25188653209408	-1.72870448128615\\
-3.26368118112669	-1.62257709444266\\
-3.26911716301003	-1.50743193969673\\
-3.2670693455275	-1.38205126127052\\
-3.2561110053532	-1.24508092609879\\
-3.23445719275566	-1.09504625615185\\
-3.19990733869133	-0.930391566459729\\
-3.14979732421527	-0.749556666920958\\
-3.08097940364488	-0.551107366131702\\
-2.98985962964534	-0.333938851596171\\
-2.87253514113668	-0.0975668422736816\\
-2.72508233437513	0.15749452962294\\
-2.54404001527801	0.429303881850829\\
-2.32709234919699	0.714103779021441\\
-2.07387200165371	1.00612432375832\\
-1.78668666913353	1.29775933191409\\
-1.47088122335579	1.58023484821087\\
-1.1345756241589	1.84471730760218\\
-0.787717671493854	2.083588282538\\
-0.440683840723204	2.2914878559795\\
-0.102865668160094	2.46580854778831\\
0.218346993157529	2.60656514302512\\
0.517995089490967	2.71580904103266\\
0.793451069451778	2.79686253352453\\
1.04399255122457	2.85360765026289\\
1.27026126265922	2.88995393049057\\
1.47375493321376	2.90950846059762\\
1.65642263207936	2.91541373390173\\
};
\addplot [color=mycolor1, forget plot]
  table[row sep=crcr]{%
1.76348601971647	2.87683329369742\\
1.92578633299117	2.87177911807686\\
2.07047954639434	2.85804562529419\\
2.19982936085868	2.837418579363\\
2.3158520581634	2.81127338959993\\
2.42030894180586	2.78065273973874\\
2.51471878827486	2.74633336758708\\
2.60037969658333	2.70888073371149\\
2.67839404839091	2.66869233009578\\
2.74969308449871	2.62603116006154\\
2.81505931225171	2.58105106779765\\
2.87514595113622	2.53381545319279\\
2.93049316546575	2.48431065802551\\
2.98154109959006	2.43245504247299\\
3.02863983365929	2.3781045253445\\
3.07205638379692	2.32105515081392\\
3.11197881832259	2.26104307036364\\
3.14851747228343	2.19774218859202\\
3.18170312587335	2.13075961284405\\
3.21148187177012	2.05962896925004\\
3.23770623227883	1.98380160652951\\
3.2601218996854	1.90263571639291\\
3.27834926667941	1.8153834799699\\
3.2918587030151	1.7211765459532\\
3.29993835675765	1.61901052787068\\
3.30165319257391	1.5077298845134\\
3.29579418014286	1.38601568238664\\
3.28081729572738	1.25238055969162\\
3.25477378510873	1.10517799928342\\
3.21523672977476	0.94263703848744\\
3.15923545962907	0.762938836079086\\
3.08322002447145	0.56435741322315\\
2.98309348302936	0.345491101696972\\
2.85436853747411	0.105608547675051\\
2.692519632201	-0.154885635179985\\
2.49359485450611	-0.433906731071035\\
2.25509698278377	-0.727187237009924\\
1.97701804067147	-1.02802696071195\\
1.66273686318081	-1.32754885415733\\
1.31936433370759	-1.61561712920537\\
0.957197314032151	-1.88230756971129\\
0.588281686375872	-2.1195031919381\\
0.224519787448563	-2.32205758013215\\
-0.124024317679309	-2.4881594718107\\
-0.450126094146457	-2.61891568901138\\
-0.749594467833043	-2.71747033472558\\
-1.02086678096851	-2.78804076186645\\
-1.26432516705138	-2.83512893618276\\
-1.48159516891929	-2.86300127331954\\
-1.67496794109156	-2.87541488118393\\
-1.84698734463292	-2.87552103698405\\
-2.00018683612315	-2.86587509023157\\
-2.13694034232468	-2.84849917106608\\
-2.25939062132157	-2.82496362814055\\
-2.36942584813439	-2.79646841722182\\
-2.46868372951739	-2.7639156604951\\
-2.55856966932492	-2.72797036453293\\
-2.64028075426689	-2.68910924597426\\
-2.71483083160932	-2.64765891538533\\
-2.78307414907135	-2.60382507704837\\
-2.84572633842907	-2.55771437365284\\
-2.90338226450221	-2.50935029296217\\
-2.95653064787545	-2.45868428788424\\
-3.00556554356136	-2.40560300220309\\
-3.05079480570215	-2.34993226542908\\
-3.09244564168406	-2.29143832806084\\
-3.13066728644761	-2.22982665188924\\
-3.16553072400175	-2.1647384458182\\
-3.19702525437404	-2.09574504440137\\
-3.22505155192867	-2.02234016598363\\
-3.24941068500344	-1.94393006826753\\
-3.26978836880001	-1.85982165915236\\
-3.28573351194609	-1.76920875235269\\
-3.29662991623451	-1.67115693541895\\
-3.30165985357059	-1.56458803023886\\
-3.29975828568289	-1.44826600751467\\
-3.28955692138947	-1.32078766195834\\
-3.26931849695565	-1.18058362230308\\
-3.23686423711171	-1.02593864805765\\
-3.18950234636825	-0.85504484502446\\
-3.12397381588454	-0.666107183445228\\
-3.03644493809001	-0.457526194514793\\
-2.92259354736716	-0.228184376895331\\
-2.77785413884563	0.0221464900115145\\
-2.59789378652418	0.292291090070833\\
-2.37936353476511	0.579126249471295\\
-2.12088114755301	0.877173423796475\\
-1.8240442075674	1.17857189841392\\
-1.49410500651416	1.47365939389888\\
-1.13989674277931	1.75220125656733\\
-0.772813421464587	2.00499206780641\\
-0.405068595964772	2.22530026810592\\
-0.0478214785723391	2.40965525281262\\
0.290210933590949	2.55779532341061\\
0.603341316210892	2.67196929045974\\
0.888768223076226	2.75597143720336\\
1.14599270169202	2.81424369783125\\
1.37610047475835	2.85121851829582\\
1.58111190506633	2.870929013894\\
1.76348601971647	2.87683329369742\\
};
\addplot [color=mycolor1, forget plot]
  table[row sep=crcr]{%
1.88661887654611	2.84789220388966\\
2.04705214990411	2.8429038537427\\
2.18884443214041	2.82945197986441\\
2.31459898637225	2.80940341620734\\
2.42658588194401	2.78417194600725\\
2.52675220198436	2.75481261074247\\
2.61675052305756	2.72209991644837\\
2.69797402742826	2.68658999318215\\
2.77159191642958	2.64866863606242\\
2.83858199221868	2.60858767046551\\
2.89975909593142	2.56649197280371\\
2.95579906312272	2.52243912845706\\
3.00725833086018	2.47641330663042\\
3.05458952043593	2.42833455819933\\
3.0981533527615	2.37806442389918\\
3.13822720222123	2.32540847912831\\
3.17501049763042	2.27011623126171\\
3.20862705450401	2.21187861648089\\
3.23912427775346	2.15032320768242\\
3.26646900783157	2.08500713814001\\
3.29053959190897	2.0154076671603\\
3.31111353902201	1.94091027094747\\
3.32784985983523	1.86079415202803\\
3.34026489971346	1.77421515866877\\
3.34770016613014	1.68018635367781\\
3.34928037988859	1.57755697444958\\
3.34385986297503	1.46499145273588\\
3.32995565774689	1.34095177579936\\
3.30566691605795	1.20368914956853\\
3.268582925055	1.05125515563213\\
3.21568798943698	0.881548851538667\\
3.14328218917604	0.692424616921975\\
3.04695489446798	0.481894709881565\\
2.92167355656706	0.248465993719188\\
2.7620789756982	-0.00835812346349713\\
2.56309271743613	-0.287419179604642\\
2.32090628856909	-0.585186685965562\\
2.03428782115633	-0.895217074015581\\
1.70589780178838	-1.20814309277256\\
1.34305318204809	-1.51251547659411\\
0.957358301848464	-1.79651289090998\\
0.563027461849464	-2.05004471948193\\
0.17440368490501	-2.2664497757554\\
-0.196376909748047	-2.44316293361382\\
-0.540783844466607	-2.58127675633976\\
-0.854143981342034	-2.68442154501285\\
-1.13505368197173	-2.75751618959368\\
-1.38443246395149	-2.80576439146061\\
-1.6045931848711	-2.83402015795313\\
-1.79851286989534	-2.84647927607051\\
-1.9693409729063	-2.8465932033257\\
-2.1201079802089	-2.8371073567179\\
-2.25357672915568	-2.8201544840852\\
-2.37218440879142	-2.79736219431427\\
-2.47803682426109	-2.76995408295331\\
-2.57292963265337	-2.73883618320411\\
-2.65838120293001	-2.70466698118174\\
-2.73566843032449	-2.66791222186143\\
-2.80586098660883	-2.62888680496751\\
-2.8698519182274	-2.58778620568981\\
-2.92838384723987	-2.54470959153192\\
-2.98207071911846	-2.49967641573398\\
-3.03141535246996	-2.45263787417719\\
-3.07682314525913	-2.40348426509281\\
-3.11861227649762	-2.35204900150335\\
-3.15702066482802	-2.29810979171628\\
-3.19220983321692	-2.24138731469277\\
-3.2242656938994	-2.18154156579224\\
-3.253196112476	-2.11816592758243\\
-3.27892493199109	-2.05077892738059\\
-3.3012819315108	-1.97881358102903\\
-3.31998795355617	-1.90160420286395\\
-3.33463415867616	-1.81837061019379\\
-3.34465406183439	-1.72819981365247\\
-3.34928670602053	-1.63002564346745\\
-3.34752911347362	-1.52260744991088\\
-3.33807619606171	-1.40451024639017\\
-3.31924694264404	-1.27409075461728\\
-3.28889755562535	-1.12949720141945\\
-3.24432634438246	-0.968695913179166\\
-3.18218323873547	-0.789545104072829\\
-3.09841088038382	-0.589945317049548\\
-2.98826611939397	-0.368104162226417\\
-2.84649908086911	-0.122953273533075\\
-2.66779180783707	0.145266116939086\\
-2.4475534338266	0.434295408854172\\
-2.18308925931526	0.739191127217544\\
-1.87496839310639	1.05200132842453\\
-1.52814377342763	1.36215315008865\\
-1.15220751368633	1.65775617088943\\
-0.760348434732937	1.9275974139927\\
-0.367169777737965	2.16314400211131\\
0.0138428598720332	2.3597740592165\\
0.37225558768193	2.51686213953497\\
0.701498660578425	2.63692954475243\\
0.998640491179626	2.72439733159422\\
1.26356280992866	2.78442955309\\
1.4979877098556	2.82211183597127\\
1.70463610269356	2.84199118101627\\
1.88661887654611	2.84789220388966\\
};
\addplot [color=mycolor1, forget plot]
  table[row sep=crcr]{%
2.02970341638092	2.83118406656601\\
2.18799151583129	2.82627070826438\\
2.32657319718337	2.81313005572053\\
2.4484361969554	2.79370724827128\\
2.55612775499061	2.76944789457504\\
2.65179070273404	2.74141210242071\\
2.73721348154227	2.71036551716844\\
2.81388206297824	2.67684947612192\\
2.88302799853583	2.6412338361512\\
2.94567029440224	2.60375608343852\\
3.0026506013816	2.56454984358704\\
3.05466205328004	2.52366528406634\\
3.10227241407705	2.48108331006773\\
3.14594225428358	2.43672495627237\\
3.18603880658966	2.39045697724693\\
3.22284602136534	2.34209432522433\\
3.2565711878713	2.29139995779901\\
3.28734832050702	2.23808222232966\\
3.31523833276256	2.18178990450667\\
3.34022582962197	2.12210489554508\\
3.36221213263002	2.05853232050682\\
3.38100389863481	1.99048788022667\\
3.39629639027895	1.91728210122766\\
3.40765009260943	1.83810118658649\\
3.41445894347145	1.75198426532106\\
3.41590797658147	1.65779713944551\\
3.41091773876497	1.55420328624735\\
3.39807261966851	1.43963415894118\\
3.37553062049951	1.31226317838362\\
3.34091387918683	1.16999188138802\\
3.29118392127937	1.01046337054864\\
3.22251556622729	0.831128393084588\\
3.13020222573058	0.629403191943986\\
3.00865666611958	0.402973082376185\\
2.85161512835336	0.150302156920767\\
2.65269571955597	-0.128613728236073\\
2.40646243195864	-0.431300484550089\\
2.11002654386438	-0.751892842889294\\
1.7649061071668	-1.08070904660359\\
1.37842750844447	-1.40486610785543\\
0.963726805587811	-1.71019729738016\\
0.537833723209402	-1.98401574313531\\
0.118366070380709	-2.21760513799863\\
-0.279767988968165	-2.40737349899909\\
-0.646366232116422	-2.55440928108333\\
-0.976246163067648	-2.66301463115588\\
-1.26835500452852	-2.73904412880406\\
-1.52442446289068	-2.78860444565441\\
-1.74771927502018	-2.81727691382763\\
-1.94211413490662	-2.82977819743403\\
-2.11151509006866	-2.8299004587552\\
-2.2595483536233	-2.82059402181255\\
-2.38942531946137	-2.80410331288318\\
-2.50391055697638	-2.78210802442187\\
-2.60534333966609	-2.75584821494047\\
-2.69568274109955	-2.72622676262518\\
-2.77655963538476	-2.69388956052156\\
-2.84932713683109	-2.65928653977993\\
-2.91510570900307	-2.62271721150604\\
-2.97482169203059	-2.5843641256496\\
-3.02923924618228	-2.54431705456848\\
-3.07898625410271	-2.50259008809168\\
-3.12457489493291	-2.45913327921625\\
-3.16641758628866	-2.4138400312764\\
-3.20483888407323	-2.36655106269949\\
-3.24008378529377	-2.31705550766154\\
-3.27232271764382	-2.26508949195961\\
-3.30165332831405	-2.2103323474067\\
-3.32809900115148	-2.1524004828699\\
-3.35160382843044	-2.09083880791846\\
-3.37202353024845	-2.02510950346722\\
-3.38911153809806	-1.95457785774221\\
-3.40249912689535	-1.87849485180717\\
-3.41166808383927	-1.79597622221363\\
-3.41591394977336	-1.70597791654738\\
-3.41429740138974	-1.60726831190391\\
-3.40558097926308	-1.49839849557875\\
-3.38814838024225	-1.37767365741986\\
-3.35990448915265	-1.24313175674567\\
-3.31815730919366	-1.09254088240715\\
-3.25948987861662	-0.923435062714006\\
-3.17964414953854	-0.733220371008295\\
-3.07346350523359	-0.519398117047068\\
-2.93497869715061	-0.279964343057266\\
-2.75776845829473	-0.0140400219106444\\
-2.53575536604124	0.277264088159463\\
-2.26454919752634	0.5898750285627\\
-1.94323913720542	0.916019429230686\\
-1.5761386446494	1.24425583883913\\
-1.17359697122873	1.56074624024008\\
-0.751049416477331	1.85170522131172\\
-0.326269606485461	2.10618551936316\\
0.0841269569430639	2.31799429981231\\
0.467444213254643	2.48601907070544\\
0.816043340955449	2.61316846626642\\
1.12696299533261	2.70471399004408\\
1.40071392257016	2.76676605369003\\
1.63993250907036	2.80523492133321\\
1.84828108699536	2.82529081769113\\
2.02970341638092	2.83118406656601\\
};
\addplot [color=mycolor1, forget plot]
  table[row sep=crcr]{%
2.19794348574178	2.83020791239487\\
2.35373363816799	2.82538097677279\\
2.48874200304993	2.81258609601171\\
2.60638535099903	2.79384130177606\\
2.70950805314651	2.77061553326634\\
2.80045446322886	2.74396555105998\\
2.88114703858336	2.71464105191456\\
2.9531589992383	2.68316310588763\\
3.01777725847028	2.64988163896509\\
3.07605480264875	2.61501701171813\\
3.1288532349893	2.57868972809118\\
3.17687675042674	2.54094132969035\\
3.22069888491605	2.50174871427288\\
3.26078324797071	2.46103347785424\\
3.29749923425388	2.41866739360158\\
3.33113347916659	2.37447477349825\\
3.36189759804843	2.32823217865962\\
3.38993253319128	2.27966572467875\\
3.41530962187418	2.22844604886074\\
3.4380282810242	2.17418085132003\\
3.45800996505544	2.11640478115467\\
3.47508777476958	2.0545663069574\\
3.48899075557473	1.98801108946729\\
3.49932149906526	1.91596127624999\\
3.50552513068025	1.83749009698622\\
3.50684711571555	1.75149122253605\\
3.50227656741248	1.65664269587682\\
3.49047099763168	1.55136609748902\\
3.46965800078646	1.43378341593225\\
3.437509876863	1.30167761211687\\
3.39099009386595	1.15246929499121\\
3.32617849345752	0.983232960249207\\
3.23809999368771	0.790793683555821\\
3.1206161475555	0.571969256244199\\
2.96649683732306	0.324047654852989\\
2.76786675583152	0.0455943566014359\\
2.51727988560229	-0.262379028283973\\
2.2096098960078	-0.595051692367061\\
1.84459494234405	-0.942756585655447\\
1.42919527126162	-1.29111794491241\\
0.978307570666708	-1.62306038301653\\
0.512678782781185	-1.92241778446513\\
0.0544287312810153	-2.17761606270126\\
-0.377737554718123	-2.38363051745455\\
-0.771451366904411	-2.54157158742407\\
-1.12104353606049	-2.65669577496405\\
-1.42614017964794	-2.73613114161499\\
-1.68970031090223	-2.78716214054464\\
-1.91631039667646	-2.81627692693882\\
-2.11101924854228	-2.82881137685641\\
-2.27867372333326	-2.82894245377662\\
-2.42360785275526	-2.81983869614429\\
-2.54954389456128	-2.80385452952917\\
-2.65960446621492	-2.78271420765073\\
-2.75637414654071	-2.75766555114021\\
-2.84197675809744	-2.72960045336035\\
-2.91815161790832	-2.69914593213989\\
-2.98632159096612	-2.66673140099171\\
-3.04765072082669	-2.63263763015077\\
-3.10309155195078	-2.59703195032721\\
-3.15342321866503	-2.5599932276488\\
-3.19928164481707	-2.52152923413244\\
-3.24118314601982	-2.48158831164077\\
-3.27954254164904	-2.44006666857769\\
-3.31468665714298	-2.39681222612689\\
-3.34686386761021	-2.35162561087716\\
-3.37625011359407	-2.30425864380888\\
-3.40295160773395	-2.25441047833786\\
-3.42700423840378	-2.20172137452279\\
-3.44836945014586	-2.14576394973124\\
-3.46692612433319	-2.08603160999408\\
-3.48245767719466	-2.02192373879323\\
-3.49463321342869	-1.95272710773176\\
-3.50298109834676	-1.87759289802314\\
-3.50685272079441	-1.79550873151772\\
-3.50537351343399	-1.70526530223622\\
-3.49737752988289	-1.605417758054\\
-3.48132122789405	-1.49424324164873\\
-3.45517202082913	-1.3696985434041\\
-3.41626862149497	-1.22938661008532\\
-3.36115520018392	-1.07054914816206\\
-3.28540356604819	-0.890116606399041\\
-3.18346285077177	-0.684867785709255\\
-3.04862189585315	-0.451777235719651\\
-2.87323907188057	-0.188647424477684\\
-2.64947050202814	0.104898760550868\\
-2.37074235958796	0.426112149877728\\
-2.03402508173049	0.76782684360933\\
-1.64243859568701	1.11789622291151\\
-1.20698322528366	1.46022099380214\\
-0.745924559696358	1.77767795717415\\
-0.281335260344847	2.0560103396237\\
0.165852337997601	2.28682675361663\\
0.579901420918831	2.46835078353251\\
0.951887237654244	2.60406024107767\\
1.27902455765975	2.70040825839983\\
1.56284739585687	2.76476655539025\\
1.80730949491724	2.80409733637165\\
2.01733982975625	2.8243298222766\\
2.19794348574178	2.83020791239487\\
};
\addplot [color=mycolor1, forget plot]
  table[row sep=crcr]{%
2.39842872978907	2.84977139256201\\
2.55129038431519	2.84504461674373\\
2.68231539555117	2.83263442291828\\
2.79539001740285	2.81462313009287\\
2.89366946536777	2.79249252018157\\
2.97969935245457	2.76728667816391\\
3.05552932001567	2.73973209652949\\
3.12281051005268	2.71032437359527\\
3.18287548552932	2.67938999343328\\
3.23680211219273	2.64712995242337\\
3.28546384805857	2.61365029122884\\
3.3295689325236	2.57898318346077\\
3.36969066335952	2.54310115785129\\
3.40629054629358	2.50592623984897\\
3.43973570423691	2.46733522432841\\
3.47031157804237	2.42716187302171\\
3.49823064269921	2.3851965200136\\
3.523637592907	2.34118332956168\\
3.5466112048444	2.29481525526136\\
3.56716283849509	2.2457265776743\\
3.58523128640436	2.19348273351238\\
3.60067337611301	2.13756698200104\\
3.61324936578264	2.07736327643932\\
3.62260170039184	2.01213452177731\\
3.62822507741097	1.94099521647983\\
3.62942496089109	1.86287734047253\\
3.62526064607836	1.77648835582449\\
3.61446772309498	1.68026053263946\\
3.59535345920811	1.57229189592498\\
3.56565767806044	1.45028166408724\\
3.5223723437118	1.31146848252927\\
3.46151792285882	1.15259035574255\\
3.37788913343578	0.969904397791377\\
3.26481670291628	0.759336217353002\\
3.11405964046072	0.51687337923087\\
2.91605591056187	0.239360678696166\\
2.66089845199008	-0.074154951409486\\
2.34045290561423	-0.420557366703982\\
1.95172262498189	-0.790768853476641\\
1.50061335174333	-1.16900621117368\\
1.00396182899994	-1.53459576559659\\
0.487507470558486	-1.86661795835332\\
-0.0202704812606559	-2.14941502129083\\
-0.495328724565367	-2.37591010083617\\
-0.922461096163118	-2.54729676726565\\
-1.29564998398071	-2.67022881755567\\
-1.61576083610479	-2.75360467736886\\
-1.88760925020119	-2.80626531035516\\
-2.11761850849402	-2.83583585301951\\
-2.31236767058961	-2.84838727307023\\
-2.47786376359186	-2.848527498087\\
-2.61927104117742	-2.83965348790494\\
-2.74088359664343	-2.82422436669264\\
-2.84620715031017	-2.80399881931045\\
-2.93807710626956	-2.78022231052697\\
-3.01877781024656	-2.75376742051602\\
-3.09014858164517	-2.72523609470473\\
-3.15367233792879	-2.69503292377865\\
-3.21054721822161	-2.6634171134397\\
-3.26174334772701	-2.63053902736006\\
-3.30804728020505	-2.59646561565443\\
-3.35009648208242	-2.56119780348917\\
-3.38840584828985	-2.52468198992928\\
-3.42338783163099	-2.4868171327911\\
-3.45536738907869	-2.44745840594938\\
-3.48459261736701	-2.40641805663432\\
-3.51124166291223	-2.3634638195184\\
-3.53542623466593	-2.3183150301698\\
-3.55719180613415	-2.27063639909317\\
-3.57651434509614	-2.22002924128097\\
-3.59329313427756	-2.16601979160752\\
-3.607338917041	-2.10804406434049\\
-3.61835618706849	-2.04542853193148\\
-3.62591790130348	-1.97736571032573\\
-3.62943018671176	-1.90288357081217\\
-3.62808369083953	-1.82080761576549\\
-3.62078707436777	-1.72971459653246\\
-3.60607682103516	-1.6278774991293\\
-3.58199631503049	-1.51320312714434\\
-3.5459367630778	-1.38316739339852\\
-3.49443481646271	-1.2347610883137\\
-3.42293053187514	-1.0644733062168\\
-3.32551203519009	-0.868364747381815\\
-3.19472224053972	-0.642321746624063\\
-3.02159306783709	-0.38262930064068\\
-2.79620535592164	-0.0870280333253533\\
-2.50919050985094	0.243654883901909\\
-2.1545017961401	0.603522232975445\\
-1.73316922104735	0.980104720301319\\
-1.25652350100294	1.35475173726096\\
-0.746404409624323	1.70596083943897\\
-0.230855365319286	2.01482665889464\\
0.263069460758329	2.26979350025063\\
0.71545763148392	2.46816383184406\\
1.1158702518671	2.61428313674225\\
1.46209994357525	2.71628873977033\\
1.75733287011997	2.78326240843799\\
2.00742483932292	2.82352076593156\\
2.21900696162006	2.84391929265026\\
2.39842872978907	2.84977139256201\\
};
\addplot [color=mycolor1, forget plot]
  table[row sep=crcr]{%
2.64097641089213	2.89659444528762\\
2.79041062611745	2.8919835091847\\
2.91701342264999	2.8799994826167\\
3.02517161377208	2.86277682362462\\
3.11835537351822	2.84179789268256\\
3.1993034241174	2.8180842826645\\
3.27017992279921	2.79233229640008\\
3.33270056385315	2.76500748982383\\
3.38823118982876	2.73641014297704\\
3.43786378858543	2.7067203746822\\
3.48247461290933	2.67602905414433\\
3.52276843328127	2.64435875960573\\
3.55931210575014	2.61167768001073\\
3.59255988695773	2.57790840977447\\
3.62287230601135	2.54293292803773\\
3.65052990345537	2.50659459017085\\
3.67574274773775	2.46869762432826\\
3.69865631155523	2.42900437234479\\
3.71935400691495	2.3872303083034\\
3.73785641225198	2.34303668445357\\
3.75411695064979	2.29602047291274\\
3.76801346544322	2.24570107596631\\
3.77933475253769	2.19150305315141\\
3.78776060210338	2.13273384775694\\
3.79283321636605	2.06855518320083\\
3.79391692835111	1.99794645060503\\
3.79014185654502	1.9196580684176\\
3.78032540163229	1.83215258867449\\
3.76286329027734	1.73353154789707\\
3.73557937374125	1.62144735702884\\
3.69552135443048	1.4930032209677\\
3.63869022202073	1.34465276673472\\
3.5596997724219	1.1721294490384\\
3.45139077633049	0.970471418888199\\
3.3044941123845	0.73426928073975\\
3.10758058775252	0.458352527766974\\
2.84777069524612	0.139206421099586\\
2.51291985474115	-0.222666080392999\\
2.09587767575504	-0.619732372235386\\
1.60026445800111	-1.03518853022456\\
1.04485624976115	-1.44396712379425\\
0.462207844468628	-1.81852813787821\\
-0.109945569927732	-2.13720202313305\\
-0.639837694079034	-2.38988812122926\\
-1.1085225069396	-2.57800121726277\\
-1.51002341235624	-2.71030781968856\\
-1.84740273810277	-2.79822029892015\\
-2.12828543924942	-2.852660444426\\
-2.3616410819168	-2.88268286233461\\
-2.55602732535301	-2.89522665708\\
-2.71886286896576	-2.89537612907735\\
-2.85626797288029	-2.88676174916544\\
-2.9731620129911	-2.87193760572698\\
-3.07344897627211	-2.85268406443714\\
-3.16021175941515	-2.83023305489948\\
-3.23588415401786	-2.80542946599412\\
-3.3023921495358	-2.77884446518797\\
-3.3612657187078	-2.75085422974431\\
-3.41372549818282	-2.72169432633904\\
-3.46074929415268	-2.69149708850984\\
-3.50312281966151	-2.66031711633317\\
-3.54147825685777	-2.62814841171454\\
-3.57632343641523	-2.59493552952458\\
-3.60806373863459	-2.56058033584831\\
-3.63701826208773	-2.52494541321332\\
-3.66343135939488	-2.48785476032272\\
-3.687480279284	-2.44909214452755\\
-3.70927935164872	-2.40839723914093\\
-3.72888088108065	-2.36545948564737\\
-3.74627264792341	-2.31990944066658\\
-3.7613716263694	-2.27130718105428\\
-3.77401318424574	-2.21912713209767\\
-3.7839345887977	-2.16273843990084\\
-3.79075105493889	-2.10137972025599\\
-3.79392176922668	-2.03412668286445\\
-3.79270221906001	-1.95985077568743\\
-3.78607765707419	-1.87716669740717\\
-3.77267056209203	-1.78436658148074\\
-3.75061256297117	-1.67933930236004\\
-3.71736886968593	-1.55947562288976\\
-3.66950212780627	-1.42156566010487\\
-3.60236624409406	-1.26170792856632\\
-3.50973711671127	-1.07527511122444\\
-3.38343266281118	-0.857029400844818\\
-3.21307688276914	-0.601556618540798\\
-2.98635308825682	-0.304279960318684\\
-2.69035381736796	0.0366596101033578\\
-2.31476776430595	0.417622393762846\\
-1.85709586746691	0.826579657652298\\
-1.32825248141111	1.24217289287799\\
-0.754563161249114	1.63710924267664\\
-0.172579790492442	1.98578138140815\\
0.381690366566272	2.27193718950993\\
0.882472466847537	2.49158054055026\\
1.31763497668127	2.65043265612539\\
1.68631313491539	2.75909577113312\\
1.9943501924068	2.82900808155476\\
2.25034735327705	2.87024233480355\\
2.46321131062424	2.89078292609621\\
2.64097641089213	2.89659444528762\\
};
\addplot [color=mycolor1, forget plot]
  table[row sep=crcr]{%
2.93937327811104	2.98020791618488\\
3.0848469729724	2.97572929737168\\
3.20660057920926	2.96421158105103\\
3.30953681559854	2.94782581088244\\
3.39743205735299	2.92804154463718\\
3.47320078612098	2.90584830807318\\
3.53910295079154	2.88190610921701\\
3.59690170272777	2.85664696662268\\
3.64798167114937	2.83034322980455\\
3.69343713544351	2.80315351614966\\
3.73413767026049	2.77515352399524\\
3.77077704972821	2.74635653760725\\
3.80390969856187	2.71672679942007\\
3.83397781176107	2.68618783037745\\
3.8613313854702	2.65462704500914\\
3.88624274436105	2.62189750703973\\
3.9089166545751	2.58781731858184\\
3.92949672483139	2.5521668737934\\
3.94806848002368	2.51468399639427\\
3.9646592060098	2.47505679034342\\
3.97923437833746	2.43291384066253\\
3.99169016663867	2.38781118544366\\
4.00184110903122	2.33921521879601\\
4.00940152307047	2.28648035436313\\
4.01395848707556	2.22881985483185\\
4.0149331832171	2.16526769114884\\
4.01152589598762	2.09462862606547\\
4.00263781427355	2.01541295408357\\
3.98675976919101	1.92575162362794\\
3.96181398478126	1.82328724568355\\
3.92492998739264	1.70503781680153\\
3.87213126655526	1.56723527884344\\
3.79790934549623	1.40515550036124\\
3.69467810189503	1.21299017713548\\
3.55216050847545	0.983882265427233\\
3.35691612825008	0.71037699308149\\
3.09254740660601	0.385728441595871\\
2.74163772771271	0.00662483549291108\\
2.29080897278802	-0.422469058064585\\
1.73923812242842	-0.884699757276557\\
1.10716516270388	-1.34981282010608\\
0.436573322336864	-1.78088447824577\\
-0.220890687156659	-2.14710992718977\\
-0.821943562709433	-2.43379883759398\\
-1.34272331037627	-2.64289484339832\\
-1.7782116219387	-2.78646452955348\\
-2.13529852485433	-2.87956050057985\\
-2.42584722998334	-2.93590877128599\\
-2.6623291416762	-2.96635774621973\\
-2.85582398510105	-2.97886103559161\\
-3.01543395442862	-2.97901953234204\\
-3.14835247008457	-2.97069499888843\\
-3.26016183347867	-2.95652191845138\\
-3.35516490155065	-2.93828744324002\\
-3.43667771452072	-2.9171984584452\\
-3.507264902453	-2.89406441611592\\
-3.56892100214669	-2.86942101856488\\
-3.6232072007064	-2.84361351437841\\
-3.67135349433301	-2.81685272698547\\
-3.71433476709828	-2.78925269730199\\
-3.75292744395917	-2.7608558587592\\
-3.78775171277993	-2.73164965769039\\
-3.81930298044209	-2.70157719211839\\
-3.84797521331808	-2.67054354674246\\
-3.87407805236539	-2.63841889729279\\
-3.89784902335864	-2.60503903927504\\
-3.9194617271411	-2.57020369456018\\
-3.93903054716445	-2.53367271663447\\
-3.95661211382123	-2.49516011774674\\
-3.9722034834123	-2.45432565273035\\
-3.98573669073324	-2.41076349279229\\
-3.99706898085955	-2.36398728674946\\
-4.00596757094343	-2.31341061456523\\
-4.01208717353537	-2.25832146405978\\
-4.01493764077979	-2.19784888121129\\
-4.01383783984167	-2.13091933860427\\
-4.00785007513615	-2.05619964169997\\
-3.99568682003507	-1.97202242746716\\
-3.97557799561631	-1.87628976447428\\
-3.94508247899672	-1.76635072010641\\
-3.90082250269903	-1.63885163641798\\
-3.83811651857233	-1.48956683168124\\
-3.75049206492978	-1.31323991930616\\
-3.62909313531356	-1.10351573998024\\
-3.46209560616133	-0.853141210902829\\
-3.2344777859539	-0.554776905634855\\
-2.92892665923736	-0.202946176203328\\
-2.52917471338455	0.202394248066321\\
-2.02691275660062	0.651054593927904\\
-1.43106859787255	1.11918464768549\\
-0.773466768821263	1.57182874168429\\
-0.103109774127866	1.97345457244795\\
0.530458453168599	2.30060631642212\\
1.0930536622269	2.54743514594387\\
1.57088555125187	2.7219330138421\\
1.96586550132787	2.8384039978474\\
2.2880922624006	2.91157789220269\\
2.55010990667103	2.95381098597267\\
2.76383367615874	2.97445489318156\\
2.93937327811104	2.98020791618488\\
};
\addplot [color=mycolor1, forget plot]
  table[row sep=crcr]{%
3.31310804065758	3.11423285912486\\
3.45413337104492	3.10990129700141\\
3.57070011418171	3.09888133309348\\
3.66822059229077	3.08336274896067\\
3.75075389077549	3.06478914220523\\
3.82136389596575	3.04410976075739\\
3.88238263370365	3.02194389134628\\
3.93560120648659	2.99868810193405\\
3.98240768763913	2.97458640468319\\
4.02388686053869	2.94977631080212\\
4.06089266641142	2.92431908538716\\
4.09410109529664	2.8982195133334\\
4.12404896518883	2.87143857341009\\
4.15116240252091	2.84390119084381\\
4.17577768368916	2.81550044301811\\
4.19815627842688	2.78609906534202\\
4.21849534453176	2.75552874147828\\
4.23693448191354	2.72358739709812\\
4.25355920468701	2.69003450423134\\
4.26840128824111	2.65458421163407\\
4.2814358549939	2.61689591897394\\
4.29257473888013	2.57656168398216\\
4.30165526881013	2.53308956451831\\
4.30842307490455	2.48588161816088\\
4.31250676187696	2.43420476743595\\
4.31338118330189	2.37715203405867\\
4.31031439599432	2.31359068433281\\
4.30229088841038	2.24209254415909\\
4.28789994881676	2.16084011080941\\
4.2651725267102	2.06750024045705\\
4.23134208942419	1.95905570936229\\
4.18249472269963	1.83158559607804\\
4.11306313531554	1.67999283802569\\
4.01511709631591	1.49770258622065\\
3.87743618320911	1.27642243790266\\
3.68448903879646	1.00621107253835\\
3.41583178233321	0.676404659142812\\
3.04726885144551	0.278377622413228\\
2.55625498552266	-0.188781270660875\\
1.93383218683997	-0.710200982381025\\
1.20012725724235	-1.24996491816525\\
0.410247293308147	-1.75767968050705\\
-0.362558776364174	-2.1882089706272\\
-1.05730575048697	-2.51968996339351\\
-1.64380701589454	-2.75527700362356\\
-2.11999550471781	-2.91234747622804\\
-2.49933220685415	-3.01130348963465\\
-2.80001154620597	-3.06965670617956\\
-3.03924310506355	-3.10048655263343\\
-3.23124858898985	-3.11291154929754\\
-3.38708122729526	-3.11307847353079\\
-3.51509913996024	-3.10506929500743\\
-3.62156044024057	-3.09158012403946\\
-3.711149313784	-3.07438917578723\\
-3.78738893952191	-3.05466771209734\\
-3.8529493853219	-3.03318359710622\\
-3.90987206489963	-3.01043404356493\\
-3.95973201171652	-2.98673231744526\\
-4.00375510589819	-2.96226456418639\\
-4.04290302683582	-2.93712714354745\\
-4.07793511653314	-2.91135111566008\\
-4.10945364919197	-2.88491812560413\\
-4.13793706580435	-2.85777040301206\\
-4.1637643602983	-2.82981660720521\\
-4.18723283314462	-2.80093460271564\\
-4.20857073457294	-2.77097181506166\\
-4.22794581138425	-2.73974350941688\\
-4.245470382185	-2.70702910090965\\
-4.26120324543763	-2.67256640697519\\
-4.27514843193874	-2.63604356066886\\
-4.2872505100243	-2.59708809391366\\
-4.29738579648496	-2.55525244546419\\
-4.30534836692443	-2.50999481981321\\
-4.31082912432757	-2.46065388244215\\
-4.31338526900716	-2.40641517487225\\
-4.31239615973142	-2.34626630873118\\
-4.30699952897152	-2.27893688278491\\
-4.29599896864385	-2.20281760686899\\
-4.27772905629921	-2.11585134255531\\
-4.24985786814903	-2.01538698375836\\
-4.20909750788577	-1.89798634141617\\
-4.15078234346563	-1.75917735716389\\
-4.06826643508234	-1.59316129396303\\
-3.95210196693554	-1.39252369712639\\
-3.78903362871618	-1.14810314065818\\
-3.56108363900491	-0.84939444890248\\
-3.24559190290084	-0.486245395053664\\
-2.81813173129827	-0.0529767983154305\\
-2.26105422587032	0.444458656327441\\
-1.578189749031	0.980788387203884\\
-0.807703545356659	1.51104381240574\\
-0.0172819665077225	1.98461563751243\\
0.722354115713504	2.36662229951065\\
1.36473839717828	2.6485642939408\\
1.89506161310841	2.84232579337481\\
2.32065368498424	2.96789444026509\\
2.65836901131155	3.0446346460869\\
2.92634210140821	3.08786054169085\\
3.14038817811298	3.10855749810132\\
3.31310804065758	3.11423285912486\\
};
\addplot [color=mycolor1, forget plot]
  table[row sep=crcr]{%
3.78933340392954	3.31791824327719\\
3.9256218828151	3.31374193339205\\
4.03688754026066	3.30322976742082\\
4.12902223330395	3.28857288936556\\
4.2063313864434	3.27117830469308\\
4.27199608767686	3.25194973580832\\
4.32839497011113	3.23146399032309\\
4.37732801465156	3.21008240302836\\
4.42017287967763	3.18802181828246\\
4.45799495636751	3.16540012959222\\
4.49162555254062	3.1422656083244\\
4.52171793409172	3.1186157297649\\
4.54878780123903	3.09440904826234\\
4.57324266131504	3.06957233862602\\
4.59540313353684	3.04400438056539\\
4.61551824784775	3.01757721978827\\
4.63377612142125	2.9901353736854\\
4.65031090648788	2.96149318670523\\
4.6652065283807	2.93143033182382\\
4.67849741922023	2.89968526525472\\
4.69016615633992	2.86594624339662\\
4.70013759293096	2.82983927634141\\
4.70826867265188	2.79091208910161\\
4.71433258516435	2.74861274838639\\
4.71799515078679	2.70226103120105\\
4.71878017350119	2.6510097787119\\
4.71601874132723	2.59379227609881\\
4.70877471281688	2.52924996225507\\
4.69573432228688	2.45563229247299\\
4.67504105577271	2.37065713885981\\
4.64404636968432	2.27131567053681\\
4.59893080163012	2.15360087231286\\
4.53412752008188	2.01213665651605\\
4.4414546834988	1.83969417544177\\
4.30885398769578	1.62662912569426\\
4.11871132685225	1.36042186851634\\
3.84610274133577	1.0258819329934\\
3.4584010973348	0.607359779437747\\
2.91995060592895	0.0953039023489661\\
2.20759156509503	-0.501193725730767\\
1.33742359745292	-1.14114239634577\\
0.382662983474024	-1.75477769362568\\
-0.548910372727928	-2.27384341746798\\
-1.36839053415821	-2.66499197898418\\
-2.03794194892131	-2.93408438541059\\
-2.56253374214651	-3.10722854796537\\
-2.96667026485512	-3.21272556954055\\
-3.27780092330027	-3.27315255507966\\
-3.51936725246729	-3.30431196037223\\
-3.70937444249712	-3.31662595527043\\
-3.86105204278324	-3.31680038675033\\
-3.98397054689743	-3.30911826665799\\
-4.08504620548983	-3.29631702607449\\
-4.16930917359062	-3.28015199262873\\
-4.2404548367953	-3.26175110641758\\
-4.30122979639096	-3.24183736156445\\
-4.35370019480879	-3.22086886829122\\
-4.3994386079345	-3.1991277038773\\
-4.43965506957823	-3.17677672466477\\
-4.47528972515932	-3.15389610687864\\
-4.50707895446667	-3.13050686958774\\
-4.53560296016067	-3.10658587936471\\
-4.56132023664761	-3.08207514272936\\
-4.58459259871904	-3.05688713619861\\
-4.60570327346796	-3.03090725083102\\
-4.6248697490697	-3.0039939853017\\
-4.6422525006617	-2.97597721465299\\
-4.65796028890226	-2.94665463110278\\
-4.67205238859317	-2.91578625810213\\
-4.68453780508563	-2.8830867485093\\
-4.69537123272274	-2.84821496494332\\
-4.70444515773916	-2.81076007592616\\
-4.71157705238056	-2.77022304923311\\
-4.71648996968417	-2.72599193502021\\
-4.71878391210462	-2.67730863644428\\
-4.71789392855401	-2.6232238652005\\
-4.71302870163677	-2.5625355332369\\
-4.70307995437032	-2.49370375141885\\
-4.6864876004763	-2.41473266923938\\
-4.66103707452359	-2.32300543050104\\
-4.62355218403979	-2.21505373816029\\
-4.56942756657358	-2.08623940876509\\
-4.49191974942988	-1.93032717585299\\
-4.38109402030484	-1.73895152208022\\
-4.22234462466892	-1.50106645059843\\
-3.99459081237621	-1.20270969746124\\
-3.66891074591962	-0.827976910678309\\
-3.21003812908939	-0.363074319302198\\
-2.58576302165782	0.194107345369954\\
-1.78898086231441	0.819660721978775\\
-0.864096666789198	1.45603320419475\\
0.0926204481931285	2.029259224117\\
0.976282252524162	2.48578474467102\\
1.72232346382894	2.81337818018193\\
2.3170353021528	3.03079287264475\\
2.77789272839084	3.16685501765346\\
3.13226917842333	3.24743838858308\\
3.4060300034355	3.29163397229516\\
3.61988992172188	3.31233578974474\\
3.78933340392954	3.31791824327719\\
};
\addplot [color=mycolor1, forget plot]
  table[row sep=crcr]{%
4.40309459243992	3.61673537985275\\
4.53481677921992	3.61270800079255\\
4.64110043478882	3.60267248745111\\
4.72826995708325	3.58880955901003\\
4.80083637203317	3.57248498439558\\
4.86206764080882	3.55455676819874\\
4.91436804378816	3.53556130027745\\
4.95953263533969	3.51582758848991\\
4.99891992945788	3.49554826079772\\
5.03357064447642	3.47482417862802\\
5.06429044154428	3.45369265395223\\
5.09170828781157	3.43214527136778\\
5.11631805711692	3.410138959954\\
5.13850839996918	3.38760254418567\\
5.15858423577884	3.36444013421383\\
5.17678210917628	3.34053216635835\\
5.19328089884202	3.3157345415432\\
5.20820883754311	3.28987605209768\\
5.22164740765234	3.26275408446327\\
5.23363235533374	3.23412840086048\\
5.24415177023397	3.20371260744588\\
5.25314086087471	3.17116268062802\\
5.26047266817884	3.13606161193869\\
5.26594343298205	3.09789879743532\\
5.26925056740572	3.05604217020051\\
5.26996001559929	3.00970014813575\\
5.26745796636491	2.95786908053401\\
5.26087896394693	2.89925977346441\\
5.24899772008713	2.8321934711684\\
5.23006411986865	2.75445280529139\\
5.20154796627757	2.66306595096728\\
5.15973861448595	2.5539918319648\\
5.09911009413821	2.42166098728948\\
5.01131013473353	2.25831539476813\\
4.88356677732647	2.05310112519422\\
4.69627993517117	1.79096594601817\\
4.41979118172258	1.45178521240388\\
4.01147087775361	1.01120055205626\\
3.41775468130055	0.446881263255729\\
2.59202807503267	-0.244177181185543\\
1.53803684977493	-1.01899743473361\\
0.353058949636403	-1.780497840141\\
-0.79904577232246	-2.42258252442959\\
-1.78480006806738	-2.89332952841377\\
-2.55848306879809	-3.20447426270419\\
-3.13987999047351	-3.39650357026675\\
-3.57134976436999	-3.5092188842842\\
-3.89333155653455	-3.57180253210422\\
-4.13710369462435	-3.60327573004656\\
-4.32501861242621	-3.61547196053107\\
-4.47262257592968	-3.61565291919622\\
-4.59069228147971	-3.60828113370718\\
-4.68675783778689	-3.59611933784414\\
-4.76615035916567	-3.58089207619238\\
-4.83270185626321	-3.56368187291071\\
-4.88921007291105	-3.54516800989344\\
-4.93774878031748	-3.52577211107219\\
-4.97987706309158	-3.50574806491598\\
-5.01678227739226	-3.4852382414019\\
-5.04937901093761	-3.46430895694594\\
-5.07837847286748	-3.44297291939761\\
-5.10433771159524	-3.42120332425291\\
-5.12769484256259	-3.39894245124507\\
-5.14879439005106	-3.37610650483996\\
-5.16790548484653	-3.35258775419939\\
-5.18523474790841	-3.32825458453661\\
-5.20093506196988	-3.30294976941533\\
-5.21511097995922	-3.27648704866286\\
-5.22782116803274	-3.24864590680683\\
-5.23907797792926	-3.21916426047402\\
-5.24884394262677	-3.18772855137705\\
-5.2570246441864	-3.15396047294106\\
-5.26345695555513	-3.11739919286055\\
-5.26789102729119	-3.07747741355114\\
-5.26996344992716	-3.03348885180034\\
-5.26915757024225	-2.98454358648732\\
-5.26474463940321	-2.92950601583791\\
-5.25569575852446	-2.86690757081248\\
-5.24054850816678	-2.79482237995296\\
-5.21720209561772	-2.71068811292387\\
-5.1825981888099	-2.61104546079613\\
-5.13221728813398	-2.49115771374786\\
-5.05927746750375	-2.34445847477063\\
-4.95346176740859	-2.16177090388398\\
-4.79894273764576	-1.93028302696452\\
-4.57152676326354	-1.63246322477282\\
-4.23529796064144	-1.24574721441874\\
-3.7412462009014	-0.74544591316852\\
-3.03549773002775	-0.115886240823348\\
-2.08987122394223	0.626159587489338\\
-0.952219886815669	1.40870385076199\\
0.237291686538823	2.12144366640694\\
1.31764533683373	2.67979391491851\\
2.19789026299071	3.06654459157377\\
2.87070142789591	3.31267996601176\\
3.37164122166647	3.46068313324343\\
3.74384219945325	3.54538371592389\\
4.02341056241594	3.5905547199245\\
4.23693830634328	3.61124714534204\\
4.40309459243992	3.61673537985275\\
};
\addplot [color=mycolor1, forget plot]
  table[row sep=crcr]{%
5.18712583895256	4.03643886090506\\
5.31525725204707	4.03252906981332\\
5.41757620024171	4.02287294925828\\
5.50079469346955	4.00964175117464\\
5.5696003253642	3.99416555895309\\
5.62733170440595	3.97726378963646\\
5.67641122072661	3.95943940624426\\
5.71862685754273	3.94099514264964\\
5.75531895681641	3.92210424785202\\
5.78750618287741	3.90285415339594\\
5.81597190070615	3.88327366803006\\
5.84132430607785	3.86334991685115\\
5.8640388204713	3.84303872581938\\
5.88448826356221	3.82227067715433\\
5.90296441651425	3.80095417392495\\
5.91969336093515	3.77897630111696\\
5.9348461628453	3.75620191182305\\
5.94854590917154	3.73247111600344\\
5.96087169458886	3.70759515234812\\
5.97185983006285	3.68135044432882\\
5.98150224897495	3.65347044950186\\
5.98974177522711	3.6236346761894\\
5.99646353807814	3.59145392663468\\
6.00148130126427	3.55645037850838\\
6.00451671496219	3.51803045859987\\
6.00516833049201	3.4754474699895\\
6.00286535774335	3.42774940781586\\
5.99679810904926	3.37370501352264\\
5.98581200843584	3.31169733754317\\
5.96824343302717	3.23956802578995\\
5.94166076599006	3.1543857977506\\
5.90244799924475	3.05209698328066\\
5.84512255933808	2.92699173597258\\
5.76120034149497	2.77088497318124\\
5.63729389255025	2.57187402037809\\
5.45196389052451	2.31254430268633\\
5.170800179951	1.96774488397845\\
4.74005681044081	1.50317000806675\\
4.08343851340225	0.879405714822638\\
3.11873275843801	0.0725212336633277\\
1.82240158153292	-0.879979006481791\\
0.3208772432074	-1.84474570715855\\
-1.13257542859024	-2.65499655952802\\
-2.33491184292571	-3.22951294326548\\
-3.23551674549088	-3.5919659263711\\
-3.8819331864792	-3.80563197939625\\
-4.34333653284971	-3.92625706921822\\
-4.67714451079244	-3.99118894206398\\
-4.92384191511931	-4.02306768443975\\
-5.1104782906461	-4.03519717120396\\
-5.25494229682946	-4.03538413725133\\
-5.36916620850968	-4.02825870448253\\
-5.46124197072875	-4.01660612024233\\
-5.53676482188745	-4.00212386009999\\
-5.59968129721263	-3.98585566318484\\
-5.65282887921241	-3.96844430635513\\
-5.69828447180154	-3.95028150863634\\
-5.73759331009454	-3.93159842884888\\
-5.77192201193149	-3.91252116361816\\
-5.80216268002389	-3.89310518437901\\
-5.82900484455859	-3.8733568160797\\
-5.8529858815526	-3.85324654641236\\
-5.87452674587498	-3.83271703316872\\
-5.89395747634478	-3.81168753720132\\
-5.91153540716701	-3.79005581256549\\
-5.92745802288954	-3.7676980440536\\
-5.94187172094541	-3.74446712558521\\
-5.95487727028482	-3.72018935421026\\
-5.9665323937902	-3.69465943029842\\
-5.97685159715728	-3.66763347237637\\
-5.98580306939653	-3.63881954565161\\
-5.99330214181357	-3.60786493364055\\
-5.99920035424942	-3.57433900902988\\
-6.00326855541121	-3.53771001937137\\
-6.00517152686637	-3.49731329719506\\
-6.00443015115258	-3.45230717604644\\
-6.00036477491604	-3.40161098959502\\
-5.99200950644002	-3.34381653301228\\
-5.9779805973346	-3.27705958700445\\
-5.95627075586493	-3.19883042356599\\
-5.92392157032137	-3.10568985900584\\
-5.87649171664733	-2.99283785316287\\
-5.80717843048081	-2.85345220038719\\
-5.70534840422258	-2.67767671923539\\
-5.5540823371721	-2.45111316334733\\
-5.32619170660676	-2.15275947247519\\
-4.97839825424011	-1.75289683024479\\
-4.44544133267055	-1.21346613054153\\
-3.64360737117788	-0.498615041134729\\
-2.50836858312705	0.391708787828822\\
-1.08269748007528	1.37202389763564\\
0.427899320710524	2.27720604166756\\
1.77169676503285	2.97202628862457\\
2.82116844859093	3.43344452188135\\
3.58608777976602	3.71348668836244\\
4.131761062643	3.87482760873836\\
4.52329054454194	3.96399345419352\\
4.80943067044692	4.01026330629437\\
5.02337853892604	4.03101761132268\\
5.18712583895256	4.03643886090506\\
};
\addplot [color=mycolor1, forget plot]
  table[row sep=crcr]{%
6.12157382844716	4.57111927024128\\
6.24806332456751	4.56726569668653\\
6.34824649775386	4.55781500425311\\
6.4291970916542	4.54494694268638\\
6.49577467523898	4.52997365270056\\
6.55139508238138	4.51369115448788\\
6.59851042630908	4.49658102796028\\
6.63891485485238	4.47892879176142\\
6.67394362166934	4.46089480869968\\
6.70460542067214	4.44255745871458\\
6.7316720572432	4.42393970758168\\
6.75574025693286	4.40502547929216\\
6.77727488947267	4.38576958944698\\
6.79663952794232	4.36610347045196\\
6.814118176515	4.34593801528116\\
6.82993067301352	4.32516431250666\\
6.84424340439968	4.30365268873562\\
6.85717638380716	4.28125022700269\\
6.86880731417405	4.25777673640515\\
6.87917293117044	4.23301897186331\\
6.88826762162818	4.20672271277163\\
6.8960390043297	4.17858207435375\\
6.90237978446501	4.14822510695287\\
6.90711468046038	4.1151942809265\\
6.90998046502719	4.07891977302142\\
6.91059598638067	4.03868242701766\\
6.90841714258277	3.99356163051609\\
6.90266864778835	3.94236074878176\\
6.89223910917539	3.8834985298116\\
6.87551668756016	3.81484790313155\\
6.85012617833902	3.73349184818549\\
6.81249851632434	3.63534605823882\\
6.75714866376768	3.51456422659319\\
6.67543564422854	3.36258547306192\\
6.55339203348151	3.16659726668628\\
6.3678959605814	2.9070891579332\\
6.08006719213609	2.55421584701719\\
5.62497593235754	2.06357009602314\\
4.90087775492559	1.37605332592143\\
3.77905242794992	0.438318091531441\\
2.18933297264066	-0.729136310160695\\
0.287381237730008	-1.95096832528054\\
-1.5447403507541	-2.97262224633514\\
-3.00610436096118	-3.6713485534524\\
-4.04944884188627	-4.09155331359193\\
-4.76576569968722	-4.32849181891376\\
-5.25912890942235	-4.45755806731079\\
-5.60648327241671	-4.52516921820955\\
-5.85799505250904	-4.55769374925049\\
-6.0453526545804	-4.56988338160137\\
-6.18866536129684	-4.57007668099744\\
-6.30093745826316	-4.56307784238045\\
-6.39078107110039	-4.5517108744142\\
-6.46404187547533	-4.53766449240715\\
-6.52478291269304	-4.52196027003734\\
-6.57589106657402	-4.50521810715663\\
-6.6194591016604	-4.48781032519455\\
-6.65703171446965	-4.46995307151583\\
-6.68976743548591	-4.45176155084663\\
-6.71854731555456	-4.43328386418837\\
-6.74404922583356	-4.41452187854762\\
-6.76679946214588	-4.39544402454388\\
-6.78720904944798	-4.37599291397698\\
-6.80559950258139	-4.35608949909373\\
-6.82222114061885	-4.33563479105379\\
-6.83726598293421	-4.31450971391857\\
-6.85087654374139	-4.29257337685617\\
-6.86315134649674	-4.26965983229243\\
-6.87414760938744	-4.24557320696391\\
-6.88388124494256	-4.22008091326308\\
-6.89232401999751	-4.19290443992443\\
-6.89939738720861	-4.16370694997184\\
-6.90496206581973	-4.13207653417616\\
-6.90880183133517	-4.09750341183482\\
-6.91059903522101	-4.05934852983483\\
-6.90989788950252	-4.01679970999594\\
-6.90604912324757	-3.96880944003592\\
-6.89812554682906	-3.91400509544899\\
-6.88479106675031	-3.85055695492978\\
-6.86409340019097	-3.77598032518114\\
-6.8331286479579	-3.6868327962381\\
-6.78748542624817	-3.5782416184092\\
-6.72030123773543	-3.44315227413553\\
-6.62062508317728	-3.27111847402897\\
-6.47053413497078	-3.04635546174779\\
-6.24007622704454	-2.74471365640521\\
-5.87884444936142	-2.32953799591968\\
-5.3044219589401	-1.7483980287049\\
-4.3972539283441	-0.940097828098037\\
-3.04005847943889	0.123663394463265\\
-1.25604321060693	1.34990678306991\\
0.662081669560649	2.49936265935053\\
2.33001000851835	3.36220391793696\\
3.57564188357901	3.91025083801778\\
4.44149212444558	4.22747445667198\\
5.03479210309462	4.4030175728086\\
5.44739037766176	4.49704300301994\\
5.74189957370979	4.5446983443192\\
5.95822788276908	4.56570118070013\\
6.12157382844716	4.57111927024128\\
};
\addplot [color=mycolor1, forget plot]
  table[row sep=crcr]{%
6.9909859082623	5.08978129550532\\
7.11790148972651	5.08591861692123\\
7.21790506732202	5.07648728423415\\
7.29838163561349	5.06369615375623\\
7.36435274701211	5.04886033454228\\
7.41931940096902	5.03276997863885\\
7.46577835194663	5.01589878061026\\
7.50554660831771	4.99852490148967\\
7.53997044729718	4.98080268271892\\
7.57006314704963	4.96280594493406\\
7.59659764406972	4.94455443903197\\
7.62017001937801	4.92603003784696\\
7.6412436766496	4.90718649633422\\
7.66018045043824	4.88795503534393\\
7.6772626550723	4.86824708120501\\
7.69270868132481	4.84795493222628\\
7.70668383791504	4.8269507645884\\
7.71930752208815	4.80508414235582\\
7.73065736702177	4.78217800344967\\
7.74077067438982	4.75802291717117\\
7.74964314023736	4.73236921810162\\
7.75722457110038	4.70491638404746\\
7.76341091045485	4.67529870204613\\
7.7680313806868	4.64306579848476\\
7.77082878287473	4.60765590708681\\
7.77142980517594	4.56835866541107\\
7.7693002597077	4.52426252110813\\
7.76367694468062	4.47417907231269\\
7.75346230241894	4.41653212432005\\
7.73705831926242	4.34919159985885\\
7.71209854955049	4.26921930519493\\
7.67500460179675	4.17247056612995\\
7.62023168766171	4.05295499601583\\
7.53894862904738	3.90178743964219\\
7.41666688165914	3.70543622994144\\
7.22890136428397	3.44278820015857\\
6.93324989831278	3.08039162177707\\
6.4557914223663	2.56576625180694\\
5.67298969892936	1.82278227009393\\
4.41173338381147	0.768977579181146\\
2.54848954519755	-0.598763404959222\\
0.259362341041705	-2.06909945961282\\
-1.93680878161845	-3.29406765961223\\
-3.63733063802839	-4.10755022276248\\
-4.80690862843087	-4.57885393338226\\
-5.58394911087071	-4.8360087212446\\
-6.10581601941451	-4.97259431008842\\
-6.46649792202238	-5.04283035884027\\
-6.72414397385616	-5.07616400887524\\
-6.91415248836824	-5.08853476964782\\
-7.05839474350069	-5.08873432244207\\
-7.17073735575507	-5.08173413086652\\
-7.26022690093588	-5.07041390261868\\
-7.33293315759245	-5.0564751412908\\
-7.39303655546847	-5.04093667685731\\
-7.44348572540194	-5.02441103394238\\
-7.48640552596234	-5.00726273880557\\
-7.52335670437388	-4.98970120591377\\
-7.55550510552885	-4.97183635536423\\
-7.58373436459845	-4.95371242221127\\
-7.60872244132046	-4.93532866348408\\
-7.63099448766155	-4.91665197383152\\
-7.65095987493623	-4.89762434431278\\
-7.6689383733506	-4.87816689879263\\
-7.6851787145651	-4.85818152645098\\
-7.69987164223171	-4.83755068372668\\
-7.71315881284425	-4.81613564451565\\
-7.72513839650761	-4.79377326270991\\
-7.73586784746538	-4.77027113072694\\
-7.74536400090288	-4.74540083794309\\
-7.75360035311974	-4.71888882329563\\
-7.76050104570531	-4.69040404176945\\
-7.76593063930396	-4.65954127749152\\
-7.76967814082003	-4.62579836507921\\
-7.77143279958508	-4.58854471138716\\
-7.77074767729164	-4.54697715285237\\
-7.76698451038783	-4.50005701858685\\
-7.75922917705121	-4.44641874011509\\
-7.74615977391762	-4.38423447145167\\
-7.72583628097031	-4.31100918742543\\
-7.69535695951676	-4.22326338726391\\
-7.65028192517443	-4.11602995471153\\
-7.58363868665366	-3.98203740160641\\
-7.48415836494337	-3.81035641884321\\
-7.33307293563603	-3.58413002197802\\
-7.09823734409178	-3.27680652861004\\
-6.7236017538666	-2.84632035718348\\
-6.11259151482577	-2.22835698922053\\
-5.11344545089394	-1.3384735771218\\
-3.55457909703552	-0.117214279941999\\
-1.4289302365554	1.34338348366825\\
0.884372753194159	2.72972904867204\\
2.85844942495496	3.75135840219081\\
4.28103103102118	4.37759901212331\\
5.23495821793396	4.72727792235517\\
5.86989868459124	4.91523188368736\\
6.30199845709865	5.01374509012517\\
6.60558442554296	5.06289109773317\\
6.82599817175736	5.08430221422049\\
6.9909859082623	5.08978129550532\\
};
\addplot [color=mycolor1, forget plot]
  table[row sep=crcr]{%
7.2832163016226	5.26738346659666\\
7.41061043289889	5.263507259095\\
7.51085362238329	5.25405397104058\\
7.59143597922978	5.24124644397224\\
7.65743670387548	5.22640424861722\\
7.71238936855266	5.21031818762825\\
7.75880959249844	5.19346119896369\\
7.79852552685244	5.17611028823901\\
7.8328901282851	5.15841865182427\\
7.86292071882693	5.14045912617052\\
7.88939269615076	5.12225067952556\\
7.91290364110264	5.10377460001331\\
7.93391787068537	5.0849842368701\\
7.95279777727258	5.06581056233232\\
7.96982602300386	5.0461648913656\\
7.98522123048198	5.02593953196896\\
7.99914888712475	5.00500677865477\\
8.01172855981302	4.98321641326176\\
8.02303807533367	4.96039168395227\\
8.03311497988752	4.93632355656157\\
8.04195528894996	4.91076284102306\\
8.04950922619908	4.88340955722362\\
8.0556732719991	4.85389857879987\\
8.06027732503006	4.82178012145223\\
8.06306501409945	4.78649293281249\\
8.06366399836447	4.74732694439775\\
8.06154114814309	4.70337041156567\\
8.05593424328879	4.65343376196269\\
8.0457462307116	4.59593773479463\\
8.02937820478327	4.5287455533124\\
8.00445936209182	4.4489053281845\\
7.9673988192036	4.35224501095417\\
7.91262045065983	4.23271945852377\\
7.83121765478356	4.08133226464549\\
7.70852088001885	3.88431981373448\\
7.51960327357625	3.62006967991628\\
7.22095771012417	3.25402150257812\\
6.73585491905443	2.73119467365785\\
5.9338037090026	1.97001890949518\\
4.62681297446452	0.878148203435068\\
2.67187007308957	-0.556720013437796\\
0.250486289929001	-2.11193355209638\\
-2.0696655194179	-3.40615562364339\\
-3.8499829292279	-4.25793875217054\\
-5.06091117825139	-4.74598363350573\\
-5.85782546953629	-5.00975354786056\\
-6.38925156712959	-5.14885876736711\\
-6.75466431715758	-5.22002459554351\\
-7.01472669261033	-5.25367520056542\\
-7.20599774179902	-5.26613050162035\\
-7.3509037063389	-5.26633231327831\\
-7.46358791204496	-5.25931164731821\\
-7.5532406990084	-5.24797128446114\\
-7.62600937502927	-5.23402089881646\\
-7.68611755561483	-5.21848143480156\\
-7.73653859168646	-5.20196517777155\\
-7.77941180915661	-5.18483562082599\\
-7.81630656912174	-5.16730099817604\\
-7.84839390877197	-5.14947015553103\\
-7.87656061685162	-5.13138644318063\\
-7.90148657445999	-5.11304843630537\\
-7.92369810403299	-5.09442253706724\\
-7.94360528994299	-5.07545041203258\\
-7.96152834084078	-5.05605300734567\\
-7.97771626876498	-5.03613216313046\\
-7.99236001617379	-5.01557040116862\\
-8.00560140905129	-4.99422916439238\\
-8.0175387954673	-4.97194557148382\\
-8.02822984560691	-4.9485275690411\\
-8.03769167373634	-4.92374718350929\\
-8.04589814162964	-4.89733136452143\\
-8.05277386521658	-4.86894963511716\\
-8.05818400956813	-4.83819737436594\\
-8.06191833293965	-4.80457298148844\\
-8.0636669871316	-4.76744629123877\\
-8.0629840602184	-4.72601423503403\\
-8.05923233959752	-4.67923754237978\\
-8.05149851972958	-4.62574867987064\\
-8.03846066931219	-4.56371521155516\\
-8.01817651567367	-4.49063248713586\\
-7.98773673884338	-4.40300162235347\\
-7.94268149303391	-4.2958168297695\\
-7.875989565814	-4.16172878067136\\
-7.77627537215769	-3.9896481243096\\
-7.62448891865551	-3.76237899002894\\
-7.38778859683814	-3.45262826362175\\
-7.00836393412335	-3.01666537601575\\
-6.38518511700294	-2.38644975042891\\
-5.35599446820208	-1.46991717648909\\
-3.7303363840847	-0.19650874428884\\
-1.48884909752624	1.34353339141959\\
0.959651428398393	2.81092659973965\\
3.03703543232128	3.88615305828664\\
4.51803085418353	4.53821345645394\\
5.50071654952238	4.89848960522985\\
6.14940394979527	5.09053888006187\\
6.58820082916082	5.19059117671614\\
6.89515570020845	5.24028859755106\\
7.11731260873144	5.2618722081855\\
7.2832163016226	5.26738346659666\\
};
\addplot [color=mycolor1, forget plot]
  table[row sep=crcr]{%
6.70347763616719	4.91651399485236\\
6.83007664955833	4.91265981889838\\
6.92998138220201	4.90323710271026\\
7.01047416996457	4.89044293379464\\
7.07652162707673	4.87558963231008\\
7.1315946956581	4.85946790418777\\
7.1781734424047	4.84255304116001\\
7.21806555594753	4.82512492910375\\
7.2526121361442	4.80733942500051\\
7.28282364523882	4.78927155807402\\
7.30947155849757	4.77094197770443\\
7.33315126942091	4.75233317462754\\
7.35432592629523	4.73339927747518\\
7.37335733591397	4.71407166956962\\
7.39052788745482	4.69426175477585\\
7.40605607042567	4.67386164292077\\
7.42010726413257	4.65274316775044\\
7.43280087100144	4.63075540303822\\
7.44421443399252	4.60772064972786\\
7.45438504147412	4.58342869094532\\
7.46330802425303	4.55762892140207\\
7.47093263923078	4.53001972152165\\
7.47715405822289	4.50023412488345\\
7.48180046698014	4.46782036326869\\
7.4846133193766	4.43221517819991\\
7.48521760641887	4.39270671784476\\
7.48307708251151	4.34838215388948\\
7.47742619858199	4.29805244526697\\
7.46716503316036	4.24014223168562\\
7.45069394100136	4.17252539686212\\
7.42564742943805	4.09227413291477\\
7.38845507139868	3.99526728048557\\
7.33359759524606	3.87556506932507\\
7.25231233843863	3.72439006851256\\
7.13028518270596	3.52844199409026\\
6.94347119049367	3.2671147221298\\
6.65058787595554	2.90809150816575\\
6.18057119284037	2.40144748284787\\
5.41690092925881	1.67654181760295\\
4.20115170498602	0.660616150378107\\
2.42827484707059	-0.640964198887718\\
0.268347616781196	-2.02837983043246\\
-1.80653486008804	-3.18560397589839\\
-3.42824772437214	-3.96126484223808\\
-4.55665807253282	-4.41590235047736\\
-5.31392618188489	-4.66647502121664\\
-5.82640166433818	-4.80058418077178\\
-6.18255946006111	-4.8699302108425\\
-6.43799920851102	-4.90297376549035\\
-6.62693996389362	-4.91527247467651\\
-6.77069166466908	-4.91546988765032\\
-6.88284379882309	-4.90848067661172\\
-6.9723011509031	-4.89716395384352\\
-7.04505863936063	-4.88321499271203\\
-7.10525615208148	-4.86765193458418\\
-7.15581995336506	-4.85108855312973\\
-7.19886243218019	-4.83389110241934\\
-7.23593737895186	-4.81627063938727\\
-7.2682068052951	-4.79833845040109\\
-7.29655230128934	-4.78013982151605\\
-7.3216508068323	-4.7616747634758\\
-7.344027034824	-4.74291066225017\\
-7.36409023460512	-4.72378977323213\\
-7.38216021207903	-4.70423328747032\\
-7.398485793522	-4.6841429867534\\
-7.41325781261155	-4.66340106094978\\
-7.4266179677889	-4.64186836716448\\
-7.43866439015068	-4.6193811957718\\
-7.44945438564445	-4.59574642809872\\
-7.45900450396759	-4.57073479109756\\
-7.46728778834022	-4.54407170536698\\
-7.47422772475973	-4.51542494970146\\
-7.47968797536858	-4.48438798100889\\
-7.4834563607787	-4.4504571823829\\
-7.48522061203805	-4.41300045211077\\
-7.48453191140727	-4.37121320755525\\
-7.48074977535588	-4.32405574785452\\
-7.47295767181221	-4.27016245942286\\
-7.45983155827868	-4.20770761137826\\
-7.43943073289554	-4.13420278451995\\
-7.40885710156391	-4.04618426431852\\
-7.36368557622261	-3.9387195459219\\
-7.29698595053399	-3.80461097266097\\
-7.19759899571727	-3.63308679491217\\
-7.04703364595124	-3.40763157658516\\
-6.81384596300016	-3.10245055255353\\
-6.44377654815097	-2.67718339857415\\
-5.84477135816693	-2.07130501058155\\
-4.87556119954063	-1.20797396692885\\
-3.38288024296397	-0.0383960540040172\\
-1.37078772986577	1.3443138563634\\
0.810538102279413	2.65154390175893\\
2.68311428953206	3.62052057186414\\
4.04772863334151	4.22114173444051\\
4.97299565420324	4.56025948872589\\
5.59432313215858	4.74415705549701\\
6.01991840236974	4.84117447389263\\
6.32034721556549	4.88980295022609\\
6.53922157908623	4.91106112284334\\
6.70347763616719	4.91651399485236\\
};
\addplot [color=mycolor1, forget plot]
  table[row sep=crcr]{%
5.6711334783885	4.30975043106863\\
5.7981024301589	4.30587961160684\\
5.89901983552652	4.29635798587537\\
5.98079215866699	4.28335819729862\\
6.04819759579739	4.26819796487796\\
6.1046137173402	4.25168198630243\\
6.15247622562267	4.23430012596883\\
6.19357391517587	4.21634470956528\\
6.22924216395473	4.1979812635934\\
6.26049234894254	4.17929183890547\\
6.28809998345381	4.16030180769172\\
6.31266572321334	4.14099645246868\\
6.33465817685667	4.12133107574606\\
6.35444425796681	4.10123685616594\\
6.37231081268117	4.08062378112781\\
6.38847997395563	4.05938143380288\\
6.40311984915975	4.03737805543945\\
6.41635157079522	4.01445805505438\\
6.42825332307538	3.99043794404617\\
6.43886162772311	3.96510049580183\\
6.44816987671727	3.93818673968355\\
6.45612378990955	3.9093851641175\\
6.46261309868026	3.87831718679015\\
6.4674582429467	3.84451749726722\\
6.47039011205888	3.80740720632041\\
6.47101968873904	3.76625671562929\\
6.46879257848685	3.72013363666001\\
6.46292031781962	3.66782858083868\\
6.45227514912502	3.6077476111253\\
6.43522598444519	3.53775356181477\\
6.40937752556968	3.45492756017388\\
6.37114634207824	3.3552040438814\\
6.31505680508197	3.23280294416779\\
6.23254804360725	3.0793361534654\\
6.10992346565928	2.88240170329279\\
5.92482870183606	2.62343174964573\\
5.64041904469585	2.27470730679447\\
5.19691702437587	1.79647488054428\\
4.50458082029795	1.13896519617542\\
3.45742283844856	0.263403896699317\\
2.00941141207743	-0.800244905551356\\
0.303011623706775	-1.89654499842687\\
-1.34449683242487	-2.81512225579304\\
-2.68112629374548	-3.4540231187615\\
-3.65671878469359	-3.84681066198187\\
-4.34013422475085	-4.07279450366802\\
-4.81839601455809	-4.19787328446349\\
-5.15918042587159	-4.26418658504594\\
-5.4081503849663	-4.29637222498502\\
-5.5948639841194	-4.30851423607433\\
-5.7384171486743	-4.30870448503699\\
-5.85132502829035	-4.30166392516054\\
-5.94196078964505	-4.29019538445848\\
-6.01605318700399	-4.27598864975694\\
-6.07760908488791	-4.26007311282492\\
-6.12948986345481	-4.24307738928111\\
-6.17377834682678	-4.22538140368627\\
-6.21201710773861	-4.20720728139212\\
-6.24536633812097	-4.18867461752844\\
-6.27471041969153	-4.16983452283209\\
-6.30073111301828	-4.15069072193919\\
-6.32395858420191	-4.13121254900976\\
-6.34480741542836	-4.11134271879507\\
-6.36360222110575	-4.09100159519299\\
-6.38059589306671	-4.07008897948351\\
-6.39598246150051	-4.04848399971752\\
-6.40990586422851	-4.02604338814606\\
-6.42246543071431	-4.00259821730328\\
-6.43371852153588	-3.97794898337333\\
-6.44368045765436	-3.9518587450571\\
-6.4523215771323	-3.92404381761165\\
-6.4595609210172	-3.89416125156107\\
-6.46525561527158	-3.86179194902714\\
-6.46918439675131	-3.8264177215145\\
-6.47102279443057	-3.78738976774571\\
-6.47030599987355	-3.74388478137508\\
-6.46637306063012	-3.69484290960486\\
-6.45828203146807	-3.63887861096069\\
-6.44467890473077	-3.57415031720324\\
-6.42359128449403	-3.49816635320554\\
-6.39209674307987	-3.40749056189011\\
-6.34577796197681	-3.29728790107074\\
-6.2778083077063	-3.16061280682044\\
-6.17738955106327	-2.98728708213752\\
-6.02706024461078	-2.76214966984903\\
-5.79812536235465	-2.46246989105206\\
-5.44343817795874	-2.05475714247584\\
-4.88856704559779	-1.49328566815426\\
-4.03113780386764	-0.729107711971661\\
-2.78027450561968	0.251580606084027\\
-1.17052585641563	1.35824257218321\\
0.548410181789146	2.38830122955274\\
2.05915259483019	3.16964756291024\\
3.21103173494561	3.67628253450724\\
4.02923013302172	3.97594957924406\\
4.60011313937785	4.14480849271164\\
5.00266458119172	4.23651798591384\\
5.29299196057574	4.28348280355837\\
5.5079043851364	4.3043405849677\\
5.6711334783885	4.30975043106863\\
};
\addplot [color=mycolor1, forget plot]
  table[row sep=crcr]{%
4.67163867468412	3.75659873546539\\
4.80186866683076	3.75262005310516\\
4.90652567250872	3.74274013679767\\
4.99208151308878	3.72913519677954\\
5.06311456176333	3.71315652110489\\
5.12291958259539	3.69564659311496\\
5.17390741933042	3.67712836345684\\
5.21786995034725	3.65792026237507\\
5.2561582181402	3.63820710234224\\
5.28980398174832	3.61808431913829\\
5.31960387782812	3.59758578423176\\
5.34617847729379	3.57670127592357\\
5.3700141988918	3.55538727713922\\
5.39149329896913	3.53357332830043\\
5.41091539240697	3.51116528725187\\
5.4285128029399	3.48804629794914\\
5.44446126337847	3.46407590816154\\
5.45888694392339	3.43908752149046\\
5.47187038633885	3.41288416844244\\
5.48344759855503	3.3852323987665\\
5.49360826829224	3.35585390306671\\
5.50229073971236	3.32441423610317\\
5.50937301255037	3.29050770125132\\
5.51465850025144	3.25363701578832\\
5.51785452024647	3.21318573661633\\
5.51854032300997	3.16838047263847\\
5.51611962807314	3.11823846564039\\
5.50974966984763	3.06149390496678\\
5.49823388315572	2.99649290883732\\
5.47985722789589	2.92104177132475\\
5.45212943924673	2.83218483792799\\
5.41137828575746	2.72587598020497\\
5.35209610574454	2.5964902658839\\
5.26588076293174	2.43610259371712\\
5.13972473458458	2.23345390462815\\
4.95333122733118	1.97259485714987\\
4.67527629662551	1.63153661214883\\
4.2589173556774	1.18235232003381\\
3.64281353614946	0.596872128640378\\
2.76871086709487	-0.134514116493091\\
1.63239985388465	-0.969704806761522\\
0.341481053554475	-1.79923945066372\\
-0.911702846433598	-2.49772316798219\\
-1.97133552712673	-3.00385668656745\\
-2.78928181566971	-3.33288905651447\\
-3.39378724049444	-3.53260625340405\\
-3.83599982110015	-3.64816015023836\\
-4.1621849514401	-3.71157909184766\\
-4.40688862759836	-3.74318310171582\\
-4.59417261718687	-3.75534464528515\\
-4.74045144297392	-3.75552783829145\\
-4.85693590176527	-3.74825749999184\\
-4.9513691171685	-3.73630399961283\\
-5.0291828692432	-3.72138067376052\\
-5.09425280363193	-3.70455440951017\\
-5.14939157249457	-3.68648981444233\\
-5.19667372054293	-3.66759648240025\\
-5.23765253214277	-3.64811914323348\\
-5.27350690411595	-3.62819359810396\\
-5.30514231084921	-3.60788177858723\\
-5.33326119401197	-3.58719380834022\\
-5.35841265244685	-3.5661017859846\\
-5.3810278694075	-3.54454814645157\\
-5.40144551937356	-3.52245033846133\\
-5.41992997189453	-3.49970286447534\\
-5.43668416493028	-3.47617728657543\\
-5.4518583743395	-3.45172050149471\\
-5.46555564428253	-3.42615136553884\\
-5.47783428839028	-3.39925556254538\\
-5.48870756772893	-3.37077842323692\\
-5.49814035214634	-3.34041519337928\\
-5.50604222919097	-3.30779797902985\\
-5.51225608123179	-3.27247822817927\\
-5.51654052399648	-3.23390307935904\\
-5.51854366039925	-3.19138312850961\\
-5.51776414387268	-3.14404799408307\\
-5.51349321652105	-3.09078427355424\\
-5.50472759415137	-3.0301477285399\\
-5.49003678633989	-2.96023725677942\\
-5.46735789106889	-2.87851157009022\\
-5.43367306040936	-2.78151934276526\\
-5.38449473794708	-2.66449872084913\\
-5.31303423010647	-2.52078270542301\\
-5.20885353055482	-2.34093024024107\\
-5.05570892896308	-2.11152138409498\\
-4.82828202528647	-1.8137207970727\\
-4.4879570021524	-1.4223506632844\\
-3.97996628739933	-0.908028977357546\\
-3.24036481558319	-0.248412187594042\\
-2.22956888661681	0.544610566929457\\
-0.994985668439008	1.39372573796315\\
0.301866163761128	2.17079799240933\\
1.47121494724185	2.77523922020516\\
2.40982402831559	3.18773350054884\\
3.11508888893499	3.44581206508271\\
3.63202941986422	3.59858514417259\\
4.01115966441353	3.68488685491627\\
4.29300553502665	3.73043968840615\\
4.50653510037146	3.75114038016021\\
4.67163867468412	3.75659873546539\\
};
\addplot [color=mycolor1, forget plot]
  table[row sep=crcr]{%
3.86621004882867	3.35342706786038\\
4.00183357207157	3.34927247233028\\
4.1123685284783	3.33883024245586\\
4.20377044648071	3.32429056068696\\
4.28037595763327	3.30705474370343\\
4.34537997702568	3.28801997229603\\
4.40116579060115	3.26775716189122\\
4.44953332480904	3.24662287243617\\
4.49185790821677	3.22483033303829\\
4.52920164601461	3.20249486927488\\
4.56239231197835	3.17966308188443\\
4.59207976211996	3.15633153217353\\
4.61877659959918	3.13245850227588\\
4.64288763783899	3.10797105090717\\
4.66473124550015	3.08276873997757\\
4.68455466314305	3.05672486288503\\
4.7025446915998	3.02968563972196\\
4.71883465603935	3.00146758243057\\
4.73350817184596	2.97185302495721\\
4.74659992366957	2.9405836247656\\
4.7580933725364	2.90735144414468\\
4.76791498485876	2.87178698466784\\
4.77592418240633	2.83344324345249\\
4.7818976783785	2.79177444288712\\
4.78550609576224	2.74610749632804\\
4.78627961188332	2.69560342433257\\
4.78355760174661	2.63920470388718\\
4.77641448341741	2.57556273942764\\
4.7635495930259	2.50293705401208\\
4.74312197712771	2.41905414834294\\
4.71250005368641	2.32090912183603\\
4.66787928390382	2.20448757604381\\
4.60369678663558	2.06438162517247\\
4.51174254996496	1.89328081664084\\
4.37985077778534	1.68136173464953\\
4.19012139625965	1.41574376435977\\
3.91697709373688	1.08056317916047\\
3.52647748988007	0.659046311978013\\
2.98075927797431	0.140115068399494\\
2.25401139944964	-0.46838896206813\\
1.36123583027566	-1.1249298141095\\
0.378653360704582	-1.75643635154976\\
-0.579639605518917	-2.29040469240906\\
-1.41967280366853	-2.69138883814888\\
-2.10247401918624	-2.9658295509358\\
-2.63451282620022	-3.14144828765701\\
-3.04232770706418	-3.24791623306647\\
-3.3549485141355	-3.3086392593448\\
-3.59681921560905	-3.33984201416901\\
-3.78652373158095	-3.35213895732227\\
-3.93761030978875	-3.35231435881987\\
-4.05982011941077	-3.34467762409983\\
-4.16015857488049	-3.33197049824379\\
-4.24370077268982	-3.31594426507581\\
-4.31416325657116	-3.29772045747233\\
-4.37430112835846	-3.27801574968895\\
-4.42618242695195	-3.25728289561473\\
-4.47137831709626	-3.23579978582744\\
-4.51109590643123	-3.21372620312733\\
-4.54627186749189	-3.19114022325021\\
-4.5776390756402	-3.16806158910986\\
-4.60577446408777	-3.14446658671418\\
-4.63113362443092	-3.1202972373356\\
-4.65407589608681	-3.09546655614066\\
-4.67488248461812	-3.06986095178151\\
-4.69376932377597	-3.04334039813289\\
-4.71089581438933	-3.01573670290691\\
-4.72637014416942	-2.98684996782189\\
-4.74025155221982	-2.95644314047749\\
-4.7525496017215	-2.92423436818679\\
-4.76322022105016	-2.88988665116764\\
-4.77215792256101	-2.85299402715322\\
-4.77918315360532	-2.81306316485667\\
-4.78402309752955	-2.76948874955416\\
-4.78628330477081	-2.72152033843684\\
-4.78540610917088	-2.66821734199282\\
-4.78060957357864	-2.60838730064551\\
-4.77079723101516	-2.54050046645771\\
-4.7544233768589	-2.46257061045725\\
-4.72928994097639	-2.37198772160957\\
-4.69223734416034	-2.26528291109474\\
-4.63867132226216	-2.13780063274324\\
-4.56184012363635	-1.98325319303706\\
-4.45174957172697	-1.79315260240838\\
-4.29361423929005	-1.55619623735213\\
-4.06591436106207	-1.25792352950098\\
-3.73878545119878	-0.881544558184724\\
-3.27520921049671	-0.411907453801613\\
-2.64040735696895	0.154629335400547\\
-1.82504978681276	0.794725831716961\\
-0.874286234966954	1.44888122767422\\
0.110528333002808	2.0389454945846\\
1.01821992348271	2.50790754880763\\
1.7810827303388	2.84291284876407\\
2.38588406134361	3.06403611085734\\
2.8520687344207	3.20168452259337\\
3.20887045341322	3.28282783173843\\
3.48343404570952	3.32715821301171\\
3.69724199086767	3.34785822823937\\
3.86621004882867	3.35342706786038\\
};
\addplot [color=mycolor1, forget plot]
  table[row sep=crcr]{%
3.24802072526657	3.08898657669455\\
3.3897758927434	3.08463098294626\\
3.50717780454169	3.07353093809921\\
3.60555965346185	3.05787447765259\\
3.68893799237806	3.0391101092248\\
3.7603551517977	3.01819389413152\\
3.82213347265528	2.99575175340309\\
3.87606096630307	2.97218590476173\\
3.92352615805759	2.9477448044097\\
3.96561606674019	2.92256922878291\\
4.00318764128468	2.89672264545511\\
4.03692007441669	2.87021111198516\\
4.06735325513992	2.84299606912493\\
4.09491606564018	2.81500218748499\\
4.11994711853208	2.78612163966091\\
4.14270973683786	2.75621564574238\\
4.16340240243195	2.72511377835292\\
4.18216546548543	2.69261124842913\\
4.19908456276707	2.65846418064867\\
4.21419089312556	2.62238269572208\\
4.22745820611286	2.58402141962884\\
4.23879603612072	2.54296681275648\\
4.24803831449298	2.4987204280975\\
4.25492595624999	2.45067683506364\\
4.25908126159783	2.39809444370101\\
4.2599708704382	2.34005678287332\\
4.25685237487358	2.27542086782332\\
4.2486972575211	2.20274808738622\\
4.23407919947639	2.1202115526856\\
4.21101150918225	2.02547225932618\\
4.17671002598232	1.91551539532524\\
4.12724851717392	1.78643957729673\\
4.05706479291633	1.63320051076463\\
3.95827691705199	1.44933768874049\\
3.8198074555026	1.2267818260611\\
3.62645740478369	0.955994004666752\\
3.35845471980742	0.626973491313833\\
2.99278339242373	0.232044799412283\\
2.50858313759176	-0.228662052705863\\
1.89851323643394	-0.739765058077636\\
1.18289867440378	-1.26624364600204\\
0.414474409831214	-1.76017377325569\\
-0.337616378172249	-2.17915311760037\\
-1.0157375690553	-2.50268408552043\\
-1.59079943986477	-2.73365860485437\\
-2.06006652098701	-2.88843216535393\\
-2.43571632063607	-2.98641656949835\\
-2.73477077015239	-3.04444784505154\\
-2.97359690442859	-3.07522108427288\\
-3.16587677526447	-3.08766094281896\\
-3.32233866490942	-3.08782658981517\\
-3.45115212667339	-3.07976629512333\\
-3.5584688334254	-3.06616779092057\\
-3.6489146448141	-3.04881172004382\\
-3.7259821618184	-3.02887558897982\\
-3.7923266830378	-3.0071341439782\\
-3.84998372656887	-2.98409079310284\\
-3.90052729064399	-2.96006385616357\\
-3.94518476284576	-2.93524332188682\\
-3.98492053793448	-2.90972826726283\\
-4.02049711639897	-2.88355147382037\\
-4.05251993928572	-2.85669544050908\\
-4.0814703778279	-2.82910249060896\\
-4.10772998172893	-2.80068069715055\\
-4.13159815263183	-2.77130671125858\\
-4.15330473483526	-2.74082614489049\\
-4.17301851823932	-2.70905185267998\\
-4.19085226555785	-2.67576022356852\\
-4.20686455867674	-2.64068539442126\\
-4.22105846772521	-2.60351110647912\\
-4.23337674350934	-2.56385971649354\\
-4.24369287854316	-2.52127762252951\\
-4.25179692277218	-2.47521604069441\\
-4.25737430662315	-2.42500563752001\\
-4.25997501365807	-2.36982293811103\\
-4.2589691055279	-2.30864563778352\\
-4.25348260730029	-2.24019288784219\\
-4.24230478577604	-2.16284527211018\\
-4.22375345868195	-2.0745376090272\\
-4.19547866566049	-1.97261627352369\\
-4.1541765623212	-1.85365258466172\\
-4.0951757525708	-1.71320800031408\\
-4.01185252086737	-1.54556278064905\\
-3.8948460055357	-1.34346386421076\\
-3.73112332047878	-1.09805241778276\\
-3.50318686910388	-0.79934673966639\\
-3.18928479391729	-0.438006532001369\\
-2.7664475480644	-0.00939589426369803\\
-2.2188015897552	0.479649837256682\\
-1.551283722985	1.00395488422053\\
-0.801025192159766	1.52030488372181\\
-0.0322148105001834	1.98092643936069\\
0.688464523498724	2.35312769590979\\
1.3168168305411	2.62889283803365\\
1.83810593029394	2.81933767520858\\
2.2585572322313	2.94337777355454\\
2.59374360323828	3.0195352365236\\
2.86078600862169	3.06260564085484\\
3.07481850245181	3.08329773658049\\
3.24802072526657	3.08898657669455\\
};
\addplot [color=mycolor1, forget plot]
  table[row sep=crcr]{%
2.77439331404181	2.93091302742763\\
2.92201679669116	2.92636277086672\\
3.04637157876297	2.91459504312614\\
3.15208826828026	2.89776376165187\\
3.2427841339122	2.87734691491567\\
3.3212839387362	2.85435203999827\\
3.38979992259227	2.82945891805353\\
3.45007259499269	2.80311756223426\\
3.50347860095776	2.77561515658961\\
3.55111252006413	2.74712164781818\\
3.59384861636267	2.71772067013067\\
3.63238736330016	2.6874303278905\\
3.66729043889089	2.65621687058045\\
3.69900694701284	2.62400327690954\\
3.72789287992069	2.59067406893071\\
3.75422526400935	2.55607719458439\\
3.77821198551289	2.52002347276501\\
3.79999793695132	2.48228383665412\\
3.81966782478975	2.44258440184194\\
3.83724570330605	2.40059919840696\\
3.85269101930178	2.35594021887548\\
3.86589063488885	2.30814422775121\\
3.87664590262258	2.25665553496582\\
3.88465334892666	2.2008036375973\\
3.8894768110636	2.13977426643563\\
3.89050787933548	2.07257193041771\\
3.8869101016783	1.99797155235903\\
3.87754047298414	1.91445631756412\\
3.86083913397739	1.82013864395438\\
3.83467496285018	1.71266178182518\\
3.79613135228652	1.58908219207822\\
3.74121472048687	1.44574010822192\\
3.66447316001772	1.27814261737485\\
3.55853593268374	1.08091920886736\\
3.41365109965799	0.847977226098818\\
3.21745098500263	0.573094105004352\\
2.95545729312942	0.251310050265984\\
2.61320031683666	-0.118511197166621\\
2.18088500636221	-0.530058184913609\\
1.66037168352198	-0.966331273249687\\
1.07123681999009	-1.39989519993898\\
0.45011847127832	-1.79917692173613\\
-0.159383005627218	-2.13866734081843\\
-0.720633271641079	-2.40633549571523\\
-1.21252016464745	-2.60379224476195\\
-1.62935814917913	-2.74118014520894\\
-1.97576460223835	-2.83146612569576\\
-2.26115887990112	-2.8867963145686\\
-2.49603422430993	-2.91702541889677\\
-2.69007033677433	-2.9295545520092\\
-2.85144939310223	-2.92970835120029\\
-2.98678676772916	-2.92122770039677\\
-3.10131263512495	-2.90670689464467\\
-3.19912099480982	-2.88793146665785\\
-3.28340775081399	-2.86612289143199\\
-3.35667147783187	-2.84211015751369\\
-3.42087317426777	-2.81644812355084\\
-3.47755971760584	-2.78949854229469\\
-3.52795787563839	-2.76148532437063\\
-3.57304540203354	-2.73253211436099\\
-3.61360464511784	-2.70268768525832\\
-3.65026290958858	-2.6719428606411\\
-3.6835227709447	-2.64024144169147\\
-3.71378470478819	-2.60748677533737\\
-3.74136374106741	-2.57354502183569\\
-3.76650134914323	-2.53824577448667\\
-3.78937336346313	-2.50138038823889\\
-3.81009443551002	-2.4626981440497\\
-3.82871921349276	-2.42190018056349\\
-3.84524017711625	-2.37863093987127\\
-3.85958176020026	-2.33246667967303\\
-3.87159004412699	-2.28290038161987\\
-3.88101685660539	-2.2293221170647\\
-3.88749650543373	-2.17099360012175\\
-3.89051253781423	-2.10701525198191\\
-3.88935073818355	-2.03628362365249\\
-3.88303293072704	-1.95743652160941\\
-3.87022389757597	-1.8687827950562\\
-3.84910078373677	-1.76821383653104\\
-3.81717092845662	-1.65309525059751\\
-3.77102113456042	-1.52014164967862\\
-3.70598200889864	-1.36528878398867\\
-3.61570295942839	-1.1836020632259\\
-3.49167419272179	-0.969310283789998\\
-3.32283524579673	-0.716141548030639\\
-3.09562346027749	-0.418264181886093\\
-2.79515779611711	-0.0722301614851879\\
-2.40854550978567	0.319857789234741\\
-1.93088423023517	0.746616654536766\\
-1.37241539502882	1.18544283033955\\
-0.761944613841044	1.60567594363182\\
-0.141318421918017	1.97750271825346\\
0.447759648128476	2.28165213535834\\
0.975915141814334	2.51333247222445\\
1.43020225350868	2.67919557517685\\
1.81083524976017	2.79140654950542\\
2.12542789170596	2.8628250344706\\
2.38427515070782	2.90453159423087\\
2.59760801138153	2.92512684507289\\
2.77439331404181	2.93091302742763\\
};
\addplot [color=mycolor1, forget plot]
  table[row sep=crcr]{%
2.40672193448848	2.85099006036898\\
2.55946400860147	2.84626734990196\\
2.69033064347883	2.8338724347951\\
2.80322644467541	2.81588983864324\\
2.90131847641568	2.79380159691169\\
2.98715982076598	2.76865112711042\\
3.06280469611034	2.74116390876034\\
3.1299069409141	2.71183448715012\\
3.18980061012433	2.68098840470885\\
3.24356430475479	2.6488258944964\\
3.29207175671717	2.61545243483868\\
3.33603121277982	2.58089983906641\\
3.37601584075997	2.54514046966064\\
3.41248696586263	2.50809636937505\\
3.44581153967493	2.46964452440797\\
3.47627488404781	2.42961905478184\\
3.50408944080944	2.38781081578398\\
3.52939998615142	2.34396465460018\\
3.55228552002567	2.29777437053471\\
3.57275779753005	2.24887525477032\\
3.59075621010789	2.1968339208358\\
3.60613842512094	2.14113496828313\\
3.61866582373807	2.08116384252486\\
3.62798230340243	2.01618506341848\\
3.63358438976255	1.94531480750321\\
3.63477978748752	1.86748668408962\\
3.63063045232983	1.78140953722261\\
3.61987499455317	1.68551643113255\\
3.60082386156014	1.57790502980948\\
3.57121974908707	1.45627211371012\\
3.52805622590764	1.31785035715285\\
3.46735228172388	1.1593660453844\\
3.38389486753738	0.977055647018671\\
3.27099539636765	0.766811080918345\\
3.1203742964426	0.524568862670887\\
2.92240239086533	0.247103233007234\\
2.66707034700497	-0.0666239051903698\\
2.34611261882098	-0.413576567700192\\
1.9564054111336	-0.78471495934714\\
1.50379513917584	-1.16420787544064\\
1.00519331949073	-1.53123118839722\\
0.486558707776251	-1.86465454960816\\
-0.0233418928124208	-2.14863449799717\\
-0.500224441331953	-2.37600079694935\\
-0.92876000791341	-2.54795215360327\\
-1.30292325563112	-2.67120671237276\\
-1.62364176742573	-2.75474212836996\\
-1.89581714913953	-2.80746710454253\\
-2.12595426693129	-2.8370548462651\\
-2.32069791209789	-2.84960647951803\\
-2.48610350192749	-2.84974705246401\\
-2.62736899749821	-2.84088225899685\\
-2.74881112519193	-2.82547500293986\\
-2.85395039459688	-2.80528503157114\\
-2.94563164671655	-2.78155750762754\\
-3.02614509779226	-2.75516411941453\\
-3.09733360141841	-2.72670575288099\\
-3.16068210115339	-2.69658598894361\\
-3.21738980476314	-2.66506317591553\\
-3.26842731238166	-2.6322870135419\\
-3.314581298808	-2.59832399127148\\
-3.35648915611748	-2.56317477226966\\
-3.39466561513653	-2.52678568243184\\
-3.42952294651184	-2.48905578494707\\
-3.46138595739844	-2.44984052919454\\
-3.49050266464898	-2.40895260246941\\
-3.51705123538792	-2.36616034149742\\
-3.54114352770935	-2.32118384593142\\
-3.5628253207578	-2.27368875417122\\
-3.58207307499217	-2.22327747492459\\
-3.59878678753142	-2.16947750238967\\
-3.61277817756754	-2.11172626935765\\
-3.62375302059351	-2.04935180693829\\
-3.63128590837492	-1.98154828765504\\
-3.63478499888611	-1.90734535493057\\
-3.63344339233583	-1.82557004893318\\
-3.62617260376615	-1.73480026336293\\
-3.61151225487164	-1.63330929028357\\
-3.58750883853339	-1.5190026775697\\
-3.55155596273051	-1.38935235741425\\
-3.50019064181572	-1.2413406084902\\
-3.42884882418522	-1.07144080442315\\
-3.3316059040818	-0.875687035934242\\
-3.20097695900251	-0.649923754453545\\
-3.02794219877558	-0.390375127140654\\
-2.80249808934361	-0.094702634584102\\
-2.51515717546199	0.236352716673745\\
-2.15973732910717	0.596958360645233\\
-1.73716603044673	0.974644700231005\\
-1.25877548178621	1.35066068275713\\
-0.746556765376957	1.70331415643677\\
-0.228820724963049	2.01349036870507\\
0.267101046683619	2.26948919152698\\
0.721110320794781	2.46857197377776\\
1.12270709349533	2.61512503245354\\
1.46971763786224	2.71736211276299\\
1.76540696891737	2.78444046005549\\
2.01571702780733	2.82473480138731\\
2.22735322408389	2.84513919980508\\
2.40672193448848	2.85099006036898\\
};
\addplot [color=mycolor1, forget plot]
  table[row sep=crcr]{%
2.11655100682591	2.82843428941261\\
2.27354611471052	2.82356578471402\\
2.41025665966359	2.81060627378123\\
2.52989546066145	2.79154091646479\\
2.63516721135814	2.76782903818278\\
2.72832261743308	2.74053005988702\\
2.81122263331923	2.71040195994312\\
2.88540101761024	2.67797585950846\\
2.95212010046617	2.64361138065595\\
3.01241816078134	2.60753712579265\\
3.06714850265728	2.56987987342384\\
3.11701103132388	2.530685279376\\
3.16257733553198	2.48993216479353\\
3.20431024779221	2.44754189978502\\
3.2425787120241	2.40338394581532\\
3.27766860712889	2.35727827727643\\
3.30978998395011	2.30899513789073\\
3.33908098109372	2.25825237864351\\
3.36560849027125	2.20471045325069\\
3.38936543605966	2.14796500184279\\
3.41026430598958	2.08753682547208\\
3.42812629933144	2.02285894077584\\
3.44266513926722	1.95326031123863\\
3.4534641954334	1.87794579759002\\
3.4599450789137	1.79597189595953\\
3.46132530302978	1.70621801868249\\
3.45656199668418	1.60735356718522\\
3.44427815481841	1.49780211181684\\
3.4226678564112	1.37570607723411\\
3.38937801254489	1.23889915632012\\
3.34136798234704	1.08490031045564\\
3.27475744922555	0.910953996822627\\
3.18469151120958	0.71415720554415\\
3.06528542191618	0.491733786396986\\
2.90976311744834	0.241532573386507\\
2.71096460636707	-0.0371839138929645\\
2.4624261097362	-0.342670722307781\\
2.16014017496843	-0.6695545814815\\
1.80476531683354	-1.00810768327461\\
1.40349383662699	-1.34464591286619\\
0.97038036758317	-1.66351810247334\\
0.524318870350239	-1.95029936328905\\
0.085159888355619	-2.19486019379004\\
-0.330303447862874	-2.39290064579353\\
-0.710776629391997	-2.54551613559592\\
-1.05080767266494	-2.65747804245967\\
-1.34965578360068	-2.7352745046498\\
-1.60964878916396	-2.78560491585113\\
-1.83470724848301	-2.81451245994852\\
-2.02929746385226	-2.82703310171971\\
-2.19780572461383	-2.82716004636509\\
-2.34422376368673	-2.81795933766786\\
-2.47202961915923	-2.80173490909645\\
-2.58417664940125	-2.78019148732509\\
-2.68313477180989	-2.75457447721281\\
-2.77095171368762	-2.72578186937574\\
-2.84931736926041	-2.69445020124993\\
-2.91962332499729	-2.66101894855914\\
-2.98301449583141	-2.625777943375\\
-3.0404322803138	-2.5889018140239\\
-3.0926497659045	-2.55047463225179\\
-3.14029993083631	-2.51050718663972\\
-3.18389785141002	-2.46894866039765\\
-3.22385782311156	-2.42569398502503\\
-3.26050613704066	-2.38058775033286\\
-3.29409006498849	-2.3334252506136\\
-3.32478341468164	-2.28395101236323\\
-3.35268882400787	-2.23185496103844\\
-3.37783676405674	-2.17676622764621\\
-3.4001810051745	-2.11824445995762\\
-3.41959005405378	-2.05576838234722\\
-3.43583377655189	-1.98872124382334\\
-3.44856406235342	-1.91637271697959\\
-3.45728794729811	-1.83785678961846\\
-3.46133108014034	-1.75214528356065\\
-3.45978882279539	-1.65801695123747\\
-3.45146169164846	-1.55402283821256\\
-3.43477150758744	-1.43845010349195\\
-3.40765503359	-1.3092893324724\\
-3.36743408852029	-1.16421545119853\\
-3.31066711786081	-1.00060088864916\\
-3.23300037864963	-0.815592941179428\\
-3.12906223935289	-0.606305635533384\\
-2.99248669002533	-0.370196004944904\\
-2.81621104383475	-0.105701784839631\\
-2.59324586139854	0.186819618571972\\
-2.31809609238649	0.503941380582688\\
-1.98880682986173	0.838150307963187\\
-1.60913736983443	1.17759478204281\\
-1.18981009540952	1.5072611728684\\
-0.747698446788933	1.81168182170006\\
-0.30271621218734	2.07826646569063\\
0.126382115899771	2.29973691404187\\
0.525382985478411	2.47465037794311\\
0.885985247779934	2.60619273420778\\
1.20528688360715	2.70022002649834\\
1.48428693993919	2.7634737488266\\
1.72626932379742	2.80239670394494\\
1.93552996856662	2.8225480577959\\
2.11655100682591	2.82843428941261\\
};
\addplot [color=mycolor1, forget plot]
  table[row sep=crcr]{%
1.8839109593124	2.84837258818049\\
2.04438503565292	2.84338280475722\\
2.18623968056034	2.82992488077221\\
2.31207075292011	2.80986400883025\\
2.42414288002402	2.78461324524576\\
2.524399227141	2.75522744745394\\
2.6144895971911	2.72248123179932\\
2.69580523727765	2.68693097387905\\
2.76951401643886	2.64896275152078\\
2.83659283040392	2.60882865184404\\
2.89785590868346	2.56667375911347\\
2.95397867354868	2.52255579514719\\
3.00551727602874	2.47645898572703\\
3.05292412571317	2.42830335512052\\
3.09655976598816	2.37795033370079\\
3.13670139641996	2.32520530362994\\
3.17354824791263	2.26981749794226\\
3.20722389261628	2.21147750004641\\
3.23777542608711	2.14981245574515\\
3.26516929363055	2.08437900355741\\
3.28928334186076	2.01465385142123\\
3.30989445454165	1.94002188583742\\
3.32666087447042	1.85976171107942\\
3.33909802279223	1.77302861621705\\
3.34654632203126	1.67883521851811\\
3.3481292617274	1.57603053813278\\
3.34269983548856	1.46327919049815\\
3.32877377018421	1.33904400156975\\
3.30444912628243	1.20157803439204\\
3.26731469495041	1.04893624597627\\
3.21435548933653	0.879023236256821\\
3.14187443310187	0.689701858699571\\
3.04546717651756	0.478996523364139\\
2.92011247449501	0.245430336841415\\
2.7604689387904	-0.0114734525998377\\
2.5614829365389	-0.290535148120325\\
2.31937579911317	-0.588206229077695\\
2.03294464193934	-0.898035058199291\\
1.70486833862432	-1.21066306833375\\
1.34246114266892	-1.51466922851812\\
0.957298419245449	-1.7982752131383\\
0.563544258574439	-2.05143637377369\\
0.17548435714616	-2.26752728383446\\
-0.194793772703191	-2.44400064822539\\
-0.538789710898925	-2.58194925525039\\
-0.851840571636949	-2.68499183533562\\
-1.13253788137548	-2.75803083744598\\
-1.38178777086944	-2.80625377472262\\
-1.60188666155828	-2.83450133180103\\
-1.79579559551911	-2.84695953436325\\
-1.96665021039329	-2.84707329656296\\
-2.11746986643662	-2.83758398914435\\
-2.25100884799222	-2.82062207494757\\
-2.3696979777122	-2.79781403418103\\
-2.47563843265873	-2.77038304485132\\
-2.57062256708164	-2.73923512802462\\
-2.65616642914844	-2.70502896323499\\
-2.73354530585091	-2.66823056893121\\
-2.80382776856561	-2.62915512247521\\
-2.86790611868862	-2.58799833682481\\
-2.92652247610629	-2.5448595531361\\
-2.98029044423412	-2.49975832182142\\
-3.02971259851557	-2.45264585394876\\
-3.07519414669671	-2.40341237922406\\
-3.11705309513639	-2.35189115869908\\
-3.15552717907817	-2.29785966662167\\
-3.19077770350021	-2.24103826799102\\
-3.22289030677915	-2.18108656759356\\
-3.25187250472187	-2.11759748601923\\
-3.27764769496605	-2.05008902577164\\
-3.3000450960285	-1.97799362927374\\
-3.31878485592477	-1.90064501222353\\
-3.33345729045611	-1.81726240582165\\
-3.34349490965776	-1.72693230686436\\
-3.34813559491003	-1.62858819671687\\
-3.34637507943646	-1.52098938281574\\
-3.33690693424297	-1.40270135184946\\
-3.3180489099609	-1.27208212003948\\
-3.28765635604558	-1.12728245865763\\
-3.24302759373264	-0.966273063059091\\
-3.18081418964464	-0.786919056262009\\
-3.09696316231815	-0.587131211462139\\
-2.98673994755229	-0.365131324720503\\
-2.84490907694386	-0.119869220079494\\
-2.66617395415224	0.148392918344402\\
-2.44597148905526	0.437376123390119\\
-2.18163739288019	0.742122945956752\\
-1.87376584258502	1.05468104394214\\
-1.52731875729217	1.36449608541033\\
-1.15187248617585	1.65971438859468\\
-0.760574942281166	1.92916920291734\\
-0.367973512747125	2.16436994990962\\
0.0125009114487025	2.36072200208788\\
0.370454450605222	2.51760845619305\\
0.69933719935473	2.637544049508\\
0.996219485266728	2.7249350397717\\
1.2609731326353	2.7849286865175\\
1.49530475588198	2.82259567527393\\
1.70191874368275	2.84247146195521\\
1.8839109593124	2.84837258818049\\
};
\addplot [color=mycolor1, forget plot]
  table[row sep=crcr]{%
1.69485018192866	2.89996807890675\\
1.85820655715354	2.89487647765564\\
2.00457984397636	2.8809797270943\\
2.13604336602734	2.86001243183153\\
2.25446721546826	2.83332346706828\\
2.36150279435773	2.80194460779108\\
2.45858756385122	2.76665089554655\\
2.54696018123358	2.72801094037392\\
2.62767995343108	2.68642736149034\\
2.70164705308327	2.64216844335124\\
2.76962154304058	2.59539234248843\\
2.83224022351806	2.54616513680408\\
2.89003087130571	2.49447383688251\\
2.94342373414113	2.44023526897727\\
2.99276027509175	2.38330153487777\\
3.03829919395036	2.32346257290518\\
3.08021972265549	2.26044619262228\\
3.11862212089304	2.19391583437811\\
3.15352519751962	2.1234662139144\\
3.18486055866431	2.04861695467206\\
3.21246313728203	1.96880429383634\\
3.23605739555129	1.88337098822631\\
3.25523842050502	1.79155467072672\\
3.2694469773572	1.69247516304914\\
3.27793749214014	1.58512170826644\\
3.27973799910182	1.46834185451507\\
3.27360148014203	1.34083495111\\
3.25794903920057	1.20115510414749\\
3.23080747248973	1.04773118604595\\
3.18974772417294	0.878915217825194\\
3.13183736099263	0.693074924434428\\
3.05363038060174	0.488750457842159\\
2.95123130069541	0.264896523752804\\
2.82048480477904	0.0212242347237356\\
2.65734908593136	-0.241365893412844\\
2.45849463889622	-0.520314844299026\\
2.22211017793365	-0.811023372373281\\
1.94878415697773	-1.10674648214036\\
1.6421946299421	-1.39895948782619\\
1.30927104075118	-1.67827753077494\\
0.959596009127865	-1.9357781217093\\
0.604107705613491	-2.1643428143617\\
0.25348980666344	-2.35957492855925\\
-0.0832302884218608	-2.5200350160566\\
-0.399501438849518	-2.64684060282257\\
-0.691411116616583	-2.74289832883057\\
-0.957357474203116	-2.81207428785222\\
-1.19748394163742	-2.85850998165818\\
-1.41308626186256	-2.88616140914034\\
-1.60611127361413	-2.89854677647915\\
-1.77878771692437	-2.89864837370952\\
-1.93338211754963	-2.88891044962224\\
-2.07205337146284	-2.87128738842466\\
-2.19677689735299	-2.84731199527596\\
-2.30931382793077	-2.81816645646392\\
-2.41120717242651	-2.784747251869\\
-2.50379272061424	-2.74772055487814\\
-2.58821691441871	-2.70756750439545\\
-2.66545700910847	-2.66462009255071\\
-2.73634086543492	-2.61908892742578\\
-2.80156496636929	-2.57108420877264\\
-2.86170999100684	-2.52063113127606\\
-2.9172536873189	-2.46768073119723\\
-2.96858098836634	-2.41211698161585\\
-3.01599139254191	-2.35376074751745\\
-3.05970362599114	-2.29237104537152\\
-3.09985755302369	-2.2276439152803\\
-3.13651321368978	-2.1592091074652\\
-3.16964675459914	-2.08662471002761\\
-3.19914288327941	-2.00936980644851\\
-3.22478332091583	-1.92683526041964\\
-3.24623055931555	-1.83831280347719\\
-3.26300606042059	-1.7429827835832\\
-3.27446190405073	-1.63990127816714\\
-3.27974485941134	-1.52798787187569\\
-3.27775205581099	-1.40601637633116\\
-3.26707808478902	-1.27261230077595\\
-3.24595485914421	-1.12626317540111\\
-3.21218847434543	-0.965351063463567\\
-3.16310248510393	-0.788220765195193\\
-3.09550534797973	-0.593301764250026\\
-3.0057118108052	-0.379305157815972\\
-2.88966262082075	-0.145514809122397\\
-2.7431989937066	0.10782186103107\\
-2.5625459681075	0.379032043483445\\
-2.34502296408943	0.664572021693395\\
-2.08991266545217	0.958757334829479\\
-1.79928738607017	1.25387241352455\\
-1.47847542184976	1.54081565734261\\
-1.13585696179675	1.81025596046772\\
-0.781886175584397	2.05402234197167\\
-0.427573956485562	2.26628262034397\\
-0.0829201636283366	2.44413355176673\\
0.244226294998288	2.58749481622467\\
0.548649664549752	2.69848476061002\\
0.827654445755815	2.78058761635831\\
1.0805877294045	2.83787921004732\\
1.3082392155596	2.87445182827387\\
1.51228482589137	2.89406303166017\\
1.69485018192866	2.89996807890675\\
};
\addplot [color=mycolor1, forget plot]
  table[row sep=crcr]{%
1.5395140061444	2.97500957994766\\
1.70534410082583	2.96983000906519\\
1.85572707401358	2.95554327789407\\
1.9923107530755	2.93375139112282\\
2.11662750506519	2.90572748918844\\
2.23006643917711	2.87246545096575\\
2.33386447233891	2.83472602671762\\
2.42910859408973	2.79307698683249\\
2.51674415445317	2.74792657593276\\
2.59758582791122	2.69955050273766\\
2.67232917150208	2.64811311550725\\
2.74156152853298	2.59368353952608\\
2.80577155393159	2.53624752869216\\
2.8653569492562	2.47571569084652\\
2.92063016456976	2.41192863217861\\
2.97182189850544	2.34465945447924\\
3.01908223835145	2.27361394319377\\
3.06247924904098	2.19842871193207\\
3.10199475663549	2.11866752710578\\
3.13751698776756	2.03381603362686\\
3.16882963099094	1.94327515304639\\
3.19559679248079	1.84635355110563\\
3.21734325007464	1.74225980594756\\
3.23342940734364	1.63009530021964\\
3.2430204850141	1.50884947785317\\
3.24504988117975	1.37740003626802\\
3.2381774770256	1.23452196372132\\
3.22074524976044	1.07891115508892\\
3.19073527136989	0.909230628141407\\
3.14573946868521	0.724189852387416\\
3.08295671749822	0.522669626245742\\
2.99924068776572	0.303904659505936\\
2.89122968292588	0.0677307008671348\\
2.75559327064357	-0.185111188979598\\
2.58942210103305	-0.452647179329826\\
2.39075715043921	-0.731391841739698\\
2.15919697674414	-1.01622745388764\\
1.89644649909105	-1.30056297494084\\
1.60661428118054	-1.57684933779702\\
1.29608038339813	-1.83741477251027\\
0.972874658231894	-2.07544116347108\\
0.645689164156102	-2.2858126665623\\
0.32279690444423	-2.46560003908645\\
0.0111702639346649	-2.61408822075381\\
-0.284018382596925	-2.73242298820054\\
-0.559493616072948	-2.82305325505413\\
-0.813673044702863	-2.88914946235601\\
-1.04630650619837	-2.93411858390528\\
-1.25808172423759	-2.96126352480465\\
-1.45027307466865	-2.97358176151219\\
-1.62446613306611	-2.97367251087915\\
-1.78236127963077	-2.96371661956631\\
-1.92564490571912	-2.94549878857511\\
-2.05591214294895	-2.92045034564002\\
-2.17462581182947	-2.88969872908586\\
-2.28309919511411	-2.8541158283054\\
-2.38249348967667	-2.8143613010691\\
-2.47382360087399	-2.77091938932807\\
-2.55796809608302	-2.72412907714267\\
-2.63568066532851	-2.67420807934244\\
-2.70760146949569	-2.62127140011584\\
-2.77426742181663	-2.56534523839753\\
-2.83612085628008	-2.50637695086907\\
-2.89351626999702	-2.44424167588688\\
-2.94672494338748	-2.37874610655234\\
-2.99593728124691	-2.30962979626666\\
-3.04126270453835	-2.23656429522712\\
-3.08272687330215	-2.15915035826483\\
-3.12026594626481	-2.07691344069961\\
-3.15371749158923	-1.98929772057509\\
-3.18280756657785	-1.89565897027643\\
-3.20713339939221	-1.79525677540025\\
-3.22614106377603	-1.68724690463044\\
-3.23909759290579	-1.57067512940798\\
-3.24505722438492	-1.44447455408784\\
-3.24282205851402	-1.30746963988646\\
-3.23089858435342	-1.15839168124217\\
-3.20745362432094	-0.995912562443984\\
-3.17027670028752	-0.818706068731119\\
-3.11676105244137	-0.625548373771311\\
-3.04392265811155	-0.415470422042853\\
-2.94848477357015	-0.18797256088483\\
-2.82706195131493	0.0566974860419996\\
-2.67647622903883	0.317222338979823\\
-2.4942200968804	0.590900251243522\\
-2.27903699278498	0.8734299226913\\
-2.0315209511659	1.15891560518726\\
-1.75456510124819	1.44020027978298\\
-1.45346149726645	1.70955456555951\\
-1.13552077609128	1.95961285942646\\
-0.809240575052896	2.18432063074294\\
-0.48323218558479	2.37962329690684\\
-0.165206978506065	2.54372251001265\\
0.138726712936532	2.67689519113288\\
0.424354947668622	2.78101359892693\\
0.689287467491502	2.85895599889901\\
0.932657573845497	2.91406305971143\\
1.15473186815185	2.94972304820235\\
1.35653090954332	2.96910368814714\\
1.5395140061444	2.97500957994766\\
};
\addplot [color=mycolor1, forget plot]
  table[row sep=crcr]{%
1.41082729830732	3.06696346341668\\
1.57888188824117	3.06170488517268\\
1.73288335955502	3.04706602621272\\
1.87414312688596	3.02452072980028\\
2.00391454723165	2.99526075284613\\
2.12336056424584	2.96023162134034\\
2.2335376505108	2.92016776505852\\
2.33539028955245	2.87562431982674\\
2.42975183496754	2.82700448312965\\
2.51734885600281	2.77458217793819\\
2.59880702224616	2.71852023752744\\
2.67465724389923	2.65888453001565\\
2.74534123038429	2.59565450298443\\
2.81121591551885	2.52873061463361\\
2.87255636770134	2.45793907089664\\
2.92955689257045	2.38303423364092\\
2.98233006803905	2.30369901990077\\
3.0309034452899	2.21954358802503\\
3.07521361811412	2.13010261489511\\
3.1150973197344	2.03483152232268\\
3.15027916623443	1.93310212832522\\
3.1803556508972	1.82419840505262\\
3.20477503873642	1.70731335329514\\
3.22281297056105	1.58154849661164\\
3.23354394743114	1.44591820544066\\
3.23580955861631	1.29936202832266\\
3.22818552015256	1.14076944943835\\
3.20895153448483	0.969022943927854\\
3.17607089684429	0.783066626566984\\
3.12719077837387	0.582008631504728\\
3.05967895816599	0.365264576686387\\
2.9707174033871	0.132745404909382\\
2.85747509860654	-0.114916532506813\\
2.71737778747114	-0.376125734509712\\
2.54847564470256	-0.648111497870774\\
2.34987790098481	-0.92681406401126\\
2.12217966549545	-1.20694813391023\\
1.86776648471642	-1.48230420879851\\
1.59087330425521	-1.74629029409601\\
1.2973196913592	-1.99263140219444\\
0.993939257680404	-2.21607026299848\\
0.687827986807597	-2.41289475473447\\
0.385597020319771	-2.58117320154147\\
0.092798406562557	-2.72067983488085\\
-0.186386490639635	-2.83258573447756\\
-0.449198933908991	-2.91903531731147\\
-0.694162162043553	-2.98272022812905\\
-0.920836124759431	-3.0265233621924\\
-1.12954360328692	-3.05326227094516\\
-1.32111758772385	-3.06552945518219\\
-1.49669412922668	-3.06561076784676\\
-1.65755646084978	-3.05545885686464\\
-1.80502578545109	-3.03670100108363\\
-1.940389825552	-3.01066563361817\\
-2.0648596017813	-2.97841687486326\\
-2.17954605591969	-2.94079047915299\\
-2.28544987622582	-2.8984275184682\\
-2.38345960078487	-2.85180403442792\\
-2.47435451497848	-2.80125604253879\\
-2.55880996272381	-2.74699991262773\\
-2.63740348711924	-2.68914846490745\\
-2.7106207635212	-2.62772324420868\\
-2.77886064751763	-2.56266345208767\\
-2.84243888316447	-2.49383198198069\\
-2.90159014285708	-2.42101894941071\\
-2.95646812835036	-2.34394305783714\\
-3.00714347359018	-2.26225110474146\\
-3.05359916966103	-2.17551592317112\\
-3.09572319326319	-2.08323308317822\\
-3.13329797629251	-1.98481676073562\\
-3.16598632330039	-1.87959533992676\\
-3.19331339329362	-1.76680757652886\\
-3.21464445626887	-1.64560055585467\\
-3.22915838245868	-1.51503127165456\\
-3.23581732988064	-1.37407448527119\\
-3.23333401795761	-1.22164063012467\\
-3.22013951826568	-1.05660889000736\\
-3.19435689680321	-0.877882063447059\\
-3.15378950237545	-0.684471055268767\\
-3.09593718817678	-0.475617033070586\\
-3.01805869148634	-0.250957106870563\\
-2.91730211708451	-0.0107329121178832\\
-2.7909247167741	0.24397134333908\\
-2.63661310096762	0.510994758935142\\
-2.4528908367777	0.786927069310679\\
-2.23956137318445	1.06707387295504\\
-1.9980896606226	1.34563420623355\\
-1.73179791579767	1.61612676971752\\
-1.44576736588601	1.87202631518176\\
-1.14641122706676	2.10748617964671\\
-0.840792519535627	2.31797229868893\\
-0.535850622331257	2.50065357292545\\
-0.237723450808828	2.65447783145559\\
0.0487005982121677	2.77996645491535\\
0.319959039170133	2.87883253552236\\
0.573960279152549	2.95354414673502\\
0.8097798590146	3.00692708029006\\
1.02739289189082	3.04185725774207\\
1.22740630901176	3.06105425092449\\
1.41082729830732	3.06696346341668\\
};
\addplot [color=mycolor1, forget plot]
  table[row sep=crcr]{%
1.30361953236429	3.17036463029885\\
1.47376416606297	3.16503240495085\\
1.63108944317391	3.15007019639534\\
1.7766465540186	3.12683240291195\\
1.91146452429925	3.09642860492775\\
2.03651750336495	3.05974978176111\\
2.1527056347414	3.01749524388881\\
2.26084522591371	2.97019785848818\\
2.36166495027574	2.91824634527061\\
2.45580568463729	2.86190416805311\\
2.54382227054595	2.80132498247682\\
2.62618599578559	2.73656483097968\\
2.70328695170963	2.66759138181632\\
2.77543566424828	2.59429054405119\\
2.84286355022453	2.51647079143163\\
2.90572183974832	2.43386551933816\\
2.96407864894928	2.34613375805254\\
3.01791390089306	2.2528595865683\\
3.06711179070596	2.15355064791606\\
3.11145048955135	2.04763627519511\\
3.15058880159253	1.93446591646719\\
3.18404955764049	1.81330882011404\\
3.21119969208565	1.68335633779026\\
3.23122726951574	1.54372874757138\\
3.24311629403442	1.39348921404227\\
3.2456210665141	1.23166837496241\\
3.23724329235029	1.05730400136355\\
3.21621721229362	0.869501022271874\\
3.18051076818563	0.667517534924947\\
3.12785401697737	0.450881541738476\\
3.05580897864137	0.21954004163846\\
2.96189638125696	-0.0259644596040753\\
2.84379194918649	-0.284305861266947\\
2.69959498153774	-0.553205406163492\\
2.52815280101113	-0.829327537810989\\
2.32939746931219	-1.10829520263138\\
2.10462355751182	-1.38487132898864\\
1.85662197887327	-1.65332173850767\\
1.58960019902782	-1.90792269649361\\
1.30886852536418	-2.14352180023087\\
1.02034111000833	-2.35603101754621\\
0.729958305028817	-2.54274468527235\\
0.443155842501758	-2.70242930019275\\
0.164479140140104	-2.83519990861521\\
-0.10261427537527	-2.94224915593108\\
-0.355776568571634	-3.02551318243085\\
-0.593663001368655	-3.08734677455416\\
-0.815754331765168	-3.13025317655978\\
-1.022157458092	-3.15668638671042\\
-1.21341761347519	-3.16892386956687\\
-1.39035957818846	-3.16899711976838\\
-1.5539635035359	-3.15866437392518\\
-1.7052737242077	-3.13941095678695\\
-1.84533548233189	-3.11246576440986\\
-1.97515349264969	-3.07882568028455\\
-2.09566662361031	-3.03928254929481\\
-2.20773388699658	-2.99444947212902\\
-2.31212797355392	-2.94478466322395\\
-2.4095335250971	-2.89061207068761\\
-2.50054811227694	-2.83213853275303\\
-2.58568448087765	-2.76946756644291\\
-2.66537305925385	-2.70261004411595\\
-2.73996401639047	-2.63149207845245\\
-2.80972835476686	-2.55596045081979\\
-2.87485764102363	-2.47578591157516\\
-2.93546204161393	-2.39066467413014\\
-2.99156635732208	-2.30021843292352\\
-3.04310375458557	-2.20399327251553\\
-3.08990688777381	-2.10145791534584\\
-3.13169611264067	-1.99200189660985\\
-3.16806453130349	-1.87493447766496\\
-3.19845971881004	-1.74948544026469\\
-3.22216221323852	-1.61480937119357\\
-3.23826128142189	-1.46999567551543\\
-3.24562920526706	-1.3140873543616\\
-3.24289650054273	-1.14611251355545\\
-3.22843222084146	-0.965133505438952\\
-3.20033591008952	-0.770319259909192\\
-3.15645078689902	-0.561046193550949\\
-3.09441095916207	-0.337031235685504\\
-3.01173783361475	-0.098495821530246\\
-2.90600046968373	0.153648974685787\\
-2.77504864012543	0.417619031336771\\
-2.6173129607066	0.690614035340361\\
-2.43214277890123	0.968766232625099\\
-2.2201237012431	1.24723435609051\\
-1.98329388368493	1.52047682401108\\
-1.72517741052467	1.78269522605538\\
-1.4505859192436	2.0283827126093\\
-1.16520188757099	2.25286634840677\\
-0.875024671769417	2.45272336342021\\
-0.585801320529994	2.62598737981331\\
-0.302559157014001	2.7721257626261\\
-0.0293125372598301	2.89183212438409\\
0.231042498441196	2.9867134565383\\
0.476678738039672	3.05895312777444\\
0.7066887332391	3.11100953386938\\
0.920891090611341	3.14538139339833\\
1.11963378465583	3.16444637329992\\
1.30361953236428	3.17036463029885\\
};
\addplot [color=mycolor1, forget plot]
  table[row sep=crcr]{%
1.21404429523693	3.28044238815931\\
1.38621268985409	3.27503967250787\\
1.54663278389918	3.25977667517288\\
1.69615770955681	3.23589953360583\\
1.83563983114042	3.20443849817299\\
1.96589963992468	3.16622753638176\\
2.08770582995269	3.12192530571178\\
2.20176333666588	3.07203536120413\\
2.30870677728523	3.01692441575519\\
2.40909733174624	2.9568380968778\\
2.50342159874684	2.89191403251249\\
2.5920913452711	2.82219232605669\\
2.67544335036796	2.74762360551715\\
2.75373874229953	2.66807489620662\\
2.82716136067381	2.5833336011911\\
2.89581475873685	2.49310990092694\\
2.95971751088797	2.39703792042776\\
3.01879652122194	2.29467607384036\\
3.07287805593017	2.18550709727907\\
3.12167626436159	2.06893843718183\\
3.164779035279	1.944303891071\\
3.20163119093747	1.81086771903676\\
3.23151530030509	1.66783287357184\\
3.25353085916163	1.51435553835408\\
3.26657332103525	1.34956880393622\\
3.26931556015483	1.17261897059671\\
3.26019588174115	0.982718497951086\\
3.23741867330031	0.779219721340932\\
3.19897605515409	0.561712644588827\\
3.14270097103519	0.330147720524914\\
3.06636311552953	0.0849797754968957\\
2.96781741455216	-0.172678419947656\\
2.84520857548179	-0.440914677424546\\
2.69722299994229	-0.716920591449242\\
2.52336142265399	-0.996979099102004\\
2.32418586746018	-1.27657335860218\\
2.10148095611961	-1.55063605168546\\
1.85827225065407	-1.81392514924874\\
1.59866930506816	-2.06147264959836\\
1.32754437871465	-2.28902302570026\\
1.05010372376851	-2.49337372784765\\
0.771437131170872	-2.67255571357623\\
0.496130795580913	-2.82583698514912\\
0.228001336931145	-2.95357686792502\\
-0.0300302684969267	-3.05698655331028\\
-0.275940431953769	-3.1378566857644\\
-0.50851573553319	-3.19830070194832\\
-0.727216481893093	-3.24054311834333\\
-0.932024563568576	-3.26676350699635\\
-1.12329864221807	-3.27899388469962\\
-1.30164924092964	-3.27906033897986\\
-1.46783839273164	-3.26855755833863\\
-1.62270342414218	-3.24884563813068\\
-1.76710181361321	-3.22106052479423\\
-1.90187309514551	-3.18613172740119\\
-2.02781378660798	-3.14480294762885\\
-2.14566180775921	-3.0976528684895\\
-2.25608750024786	-3.0451144903595\\
-2.35968899977471	-2.98749218015208\\
-2.45699026122276	-2.92497609629639\\
-2.54844047680928	-2.85765395199859\\
-2.6344139584906	-2.78552024918273\\
-2.71520979391882	-2.70848320565843\\
-2.79105074844663	-2.62636964450667\\
-2.86208099170613	-2.53892814329307\\
-2.92836229218099	-2.4458307708287\\
-2.9898683616798	-2.34667378659936\\
-3.04647705843351	-2.24097775733803\\
-3.09796018939814	-2.12818767179549\\
-3.14397071057272	-2.00767382546431\\
-3.18402723796887	-1.87873452025237\\
-3.21749599223031	-1.74060199786171\\
-3.24357066319546	-1.59245351215947\\
-3.2612512708607	-1.43343004080377\\
-3.26932400463725	-1.2626657993691\\
-3.26634533158282	-1.07933234415153\\
-3.25063542710014	-0.882701412409692\\
-3.22028813827935	-0.672230366860738\\
-3.1732069482427	-0.447672601718421\\
-3.1071780824799	-0.209211783624883\\
-3.01999175754808	0.0423873708727519\\
-2.90961887002595	0.305628153185356\\
-2.77444132743723	0.578158055922668\\
-2.61351892168651	0.856709056758445\\
-2.42685603899915	1.13714194062122\\
-2.21561361070748	1.41462476740585\\
-1.98220507942266	1.6839496369487\\
-1.73022872524769	1.93995393006796\\
-1.46422412897447	2.17797530792202\\
-1.18928764889999	2.39425119110813\\
-0.91062143246285	2.58618445637019\\
-0.633105203016797	2.75243455599225\\
-0.360964850674611	2.89284069661211\\
-0.0975764829442119	3.00822147313603\\
0.154593906607722	3.10011179527205\\
0.393942688293211	3.17049342432048\\
0.619612011186499	3.22155838822615\\
0.831341940748543	3.25552473402513\\
1.02931979977233	3.27450805884036\\
1.21404429523693	3.28044238815931\\
};
\addplot [color=mycolor1, forget plot]
  table[row sep=crcr]{%
1.13918986385114	3.39291030692128\\
1.31334254622258	3.38743932459056\\
1.47666231161076	3.37189486946436\\
1.62985336376191	3.34742713098829\\
1.77363116361104	3.31499238731071\\
1.9086936208948	3.27536813870626\\
2.03570151526558	3.22916982839512\\
2.15526566937106	3.17686731060412\\
2.26793883658075	3.11879998754381\\
2.37421068991487	3.05519005756676\\
2.47450466317591	2.98615365655134\\
2.56917568712316	2.91170988787152\\
2.65850808526946	2.83178786715564\\
2.74271305516713	2.74623199020703\\
2.82192527413863	2.65480569155722\\
2.89619824619334	2.55719401700191\\
2.96549806255904	2.45300540251019\\
3.02969529607241	2.34177314884866\\
3.08855480632043	2.2229572204889\\
3.14172331913164	2.09594719327623\\
3.18871478900932	1.96006744155325\\
3.22889379417839	1.81458600194816\\
3.2614576002984	1.65872897810248\\
3.28541812055037	1.49170283761927\\
3.29958586098608	1.31272743994833\\
3.30255912144019	1.12108299720362\\
3.29272322416275	0.916174187090234\\
3.26826625015252	0.697613960216642\\
3.22721935403153	0.465327738155843\\
3.16753056484358	0.219675153705707\\
3.08718005056154	-0.0384191370860458\\
2.98434084579235	-0.307341574723892\\
2.85758089070568	-0.584696951783404\\
2.70608979312119	-0.867277122662939\\
2.52989900831624	-1.15112144140111\\
2.3300519147241	-1.4316889926073\\
2.10867725524729	-1.70414101309216\\
1.86893120683685	-1.96370287284456\\
1.61480018945704	-2.20604842658807\\
1.35079090761119	-2.42763724664457\\
1.08156304935098	-2.62594429031886\\
0.811571843486864	-2.79954935129738\\
0.544778559945462	-2.94808887382429\\
0.284462885417311	-3.07210182987649\\
0.0331430358728775	-3.17281577583671\\
-0.207412600404753	-3.25191821988336\\
-0.436109895771939	-3.3113472100126\\
-0.652418494402153	-3.35312039216399\\
-0.856250380106214	-3.37920883099783\\
-1.04784395731508	-3.39145301224971\\
-1.22766287057359	-3.39151379982221\\
-1.39631320746139	-3.3808496940076\\
-1.55447908508147	-3.36071226228474\\
-1.70287462151121	-3.3321530445752\\
-1.84220946327544	-3.29603688523293\\
-1.97316492744097	-3.25305814882564\\
-2.09637808130158	-3.20375749061753\\
-2.21243149987578	-3.14853775640611\\
-2.32184688148563	-3.08767822001648\\
-2.42508109894911	-3.02134678958923\\
-2.52252359246908	-2.9496100841685\\
-2.61449426618703	-2.87244144934862\\
-2.70124124044346	-2.78972708358762\\
-2.78293794738014	-2.70127051464626\\
-2.85967915133165	-2.60679572088432\\
-2.9314755405021	-2.50594925269726\\
-2.9982465862832	-2.3983017905733\\
-3.05981141648412	-2.28334969254005\\
-3.11587751739229	-2.16051724954773\\
-3.16602719136474	-2.02916059682455\\
-3.20970188427344	-1.88857453477377\\
-3.24618480397113	-1.73800390071199\\
-3.27458273194635	-1.57666159427501\\
-3.29380864897059	-1.40375585733705\\
-3.30256781250384	-1.21852985551498\\
-3.29935127053315	-1.02031683366934\\
-3.28244242619718	-0.808613839943074\\
-3.24994397454638	-0.583175817221317\\
-3.19983384934379	-0.344129229116059\\
-3.13005892010776	-0.0920998288786608\\
-3.03867289103494	0.171657472809961\\
-2.92401889067576	0.445146795822683\\
-2.78494685465088	0.725565289186103\\
-2.62104180619464	1.00931428445458\\
-2.43282490921739	1.29211422453723\\
-2.22188069184517	1.56923392380142\\
-1.99086776908621	1.83581845968917\\
-1.74339015709075	2.08727071340673\\
-1.4837382877407	2.31962080554035\\
-1.21654221879537	2.52981570238502\\
-0.946400969228186	2.71588066109658\\
-0.677553065468151	2.87693743867862\\
-0.413635536840815	3.0130978833525\\
-0.15755103135502	3.12527387942904\\
0.088563304520537	3.21495096129171\\
0.323288483199085	3.28396596140827\\
0.545827235646421	3.33431531819545\\
0.755885505103046	3.36800638923301\\
0.953552195606826	3.38695306611053\\
1.13918986385114	3.39291030692128\\
};
\addplot [color=mycolor1, forget plot]
  table[row sep=crcr]{%
1.07681390015377	3.50387265912038\\
1.25290541920648	3.49833572423572\\
1.4189356767722	3.48252854212395\\
1.57549793994767	3.45751789326495\\
1.72320292935161	3.4241930275202\\
1.86265234412909	3.3832777999272\\
1.99442009920366	3.33534439184401\\
2.11903930354307	3.2808270371971\\
2.23699331814286	3.22003479545064\\
2.34870954353418	3.15316285038042\\
2.45455486234239	3.08030211046827\\
2.55483188912201	3.0014470836132\\
2.64977535661494	2.91650213026931\\
2.73954810062175	2.82528629261796\\
2.82423620396988	2.72753697483497\\
2.90384293447019	2.62291282871868\\
2.9782811746746	2.51099629514403\\
3.04736410730181	2.39129637879328\\
3.11079400754534	2.26325240417023\\
3.16814912553205	2.12623972620519\\
3.21886884942524	1.97957865614578\\
3.26223766078859	1.82254821170636\\
3.29736887515341	1.65440669150492\\
3.32318985127375	1.47442145935703\\
3.33843129116964	1.28191060678544\\
3.34162444246868	1.07629917335834\\
3.33111137687374	0.857192084217782\\
3.30507483070997	0.624464560729761\\
3.26159491273558	0.378368078115702\\
3.19873959952593	0.119645680832525\\
3.11469341897808	-0.15035532707082\\
3.00792316781258	-0.429591854598313\\
2.87737058282481	-0.715278695886092\\
2.7226505650479	-1.00391302290729\\
2.54422273820986	-1.29138984491412\\
2.34349842605664	-1.5732142596992\\
2.12284929698522	-1.8447947354008\\
1.88550002685379	-2.10177872423388\\
1.63531234685439	-2.34037630997864\\
1.37649370738826	-2.55761666979082\\
1.1132808549074	-2.75149747541878\\
0.849650718702709	-2.9210133765116\\
0.589098633806478	-3.06607653770328\\
0.334503372642803	-3.18736095127435\\
0.0880778652927617	-3.28610893254169\\
-0.148610152184288	-3.36393423857315\\
-0.374569942243917	-3.42264617084894\\
-0.589294217619091	-3.46410761740004\\
-0.792658164053592	-3.49013049208802\\
-0.984823260130143	-3.50240573644614\\
-1.16615279750645	-3.50246184449687\\
-1.337141908017	-3.49164495734264\\
-1.49836219357036	-3.4711140416334\\
-1.65041952382722	-3.44184577780922\\
-1.79392288646242	-3.40464505649268\\
-1.92946203233454	-3.36015815167484\\
-2.05759181094738	-3.30888659825559\\
-2.17882137624947	-3.25120053029595\\
-2.29360675984949	-3.18735076066673\\
-2.40234560498806	-3.11747924444821\\
-2.50537310652952	-3.04162781050253\\
-2.6029584036181	-2.95974520593235\\
-2.69530082563154	-2.87169260754258\\
-2.78252550646702	-2.77724783745605\\
-2.86467796709813	-2.67610859633044\\
-2.94171733334199	-2.5678951135311\\
-3.01350791825165	-2.45215272354561\\
-3.07980897269412	-2.3283550251702\\
-3.14026251404752	-2.19590847669295\\
-3.19437930828088	-2.05415953611136\\
-3.24152333948906	-1.90240577421399\\
-3.28089549603646	-1.73991276179689\\
-3.31151778247802	-1.56593892917875\\
-3.33222017760411	-1.37977094642282\\
-3.3416333263787	-1.18077234614903\\
-3.33819154649102	-0.968447894925771\\
-3.32015200737686	-0.742525306038447\\
-3.28563707691093	-0.50305389184007\\
-3.23270715373773	-0.250516304671474\\
-3.15946997428325	0.0140555549435519\\
-3.06422843948946	0.288975714522802\\
-2.94566174443023	0.571831968071082\\
-2.80302423696485	0.85947189989199\\
-2.63633487342317	1.14807102780864\\
-2.44652124165192	1.43329715793668\\
-2.23548085738443	1.71056661957679\\
-2.00603259452234	1.97536480215196\\
-1.76175240525437	2.22358302957913\\
-1.5067141483345	2.45181490629269\\
-1.24517891109999	2.65756275019291\\
-0.981286178864548	2.83932648752665\\
-0.718794577087978	2.9965751045877\\
-0.460902438757844	3.12962440062159\\
-0.210156918784256	3.23945763471365\\
0.0315577520644397	3.32752657542855\\
0.262972928864498	3.39556279593601\\
0.483352111847407	3.44541778923692\\
0.69239179193768	3.47893974872642\\
0.890121699398104	3.49788690805546\\
1.07681390015377	3.50387265912038\\
};
\addplot [color=mycolor1, forget plot]
  table[row sep=crcr]{%
1.02515979667224	3.60981291354574\\
1.20311562436717	3.6042131621784\\
1.37165196277825	3.58816339150138\\
1.53128009861078	3.56265918140955\\
1.68253149195655	3.52853058333861\\
1.8259333787484	3.48645225249679\\
1.96199093354743	3.43695506518422\\
2.09117436581869	3.38043785585983\\
2.21390955397894	3.31717842954101\\
2.33057106041103	3.24734337988316\\
2.44147658709197	3.17099650562936\\
2.54688211315672	3.08810580046505\\
2.64697710091582	2.99854912171739\\
2.74187926984612	2.90211874503844\\
2.83162852659825	2.79852510431886\\
2.91617971318818	2.68740011465374\\
2.99539390766968	2.56830059516605\\
3.06902809714852	2.44071246018251\\
3.13672316160073	2.30405654196441\\
3.19799028367838	2.15769715263519\\
3.25219616601802	2.00095478670175\\
3.29854783101757	1.8331246947889\\
3.33607833874827	1.65350338667567\\
3.36363552033777	1.46142537282874\\
3.37987679574304	1.25631249938791\\
3.38327427887542	1.03773787322616\\
3.37213551898953	0.805505339863061\\
3.34464607516443	0.559743452978278\\
3.29894015671415	0.301009592182757\\
3.23320408185208	0.0303953020543156\\
3.14581355627779	-0.25038151573407\\
3.03549928853415	-0.538917019771991\\
2.90152659503935	-0.8321166106038\\
2.74386511438148	-1.12626478789704\\
2.56331773962297	-1.41718043855402\\
2.36157729213811	-1.70045233319201\\
2.14118811785052	-1.97173013328136\\
1.90540730117711	-2.22702947852382\\
1.65798200631524	-2.46300249869556\\
1.40287823587041	-2.67713114916541\\
1.14400535857114	-2.8678184835532\\
0.88497745374461	-3.03437591453482\\
0.628939318653582	-3.17692474229225\\
0.378467674378691	-3.29624212085679\\
0.13554262520618	-3.39358379222262\\
-0.0984255763087338	-3.47051059170827\\
-0.322533207424746	-3.52873675093961\\
-0.536300907622331	-3.57000886379482\\
-0.739586635061441	-3.59601713677922\\
-0.932503172145304	-3.60833591349153\\
-1.11534548375721	-3.60838819588231\\
-1.2885300856312	-3.59742832529505\\
-1.45254649037625	-3.57653744089128\\
-1.60791959998377	-3.54662725568829\\
-1.75518136156198	-3.50844872441816\\
-1.89484986610153	-3.46260312808364\\
-2.0274141676177	-3.40955388545592\\
-2.15332330756317	-3.34963800690096\\
-2.27297826969436	-3.28307654969192\\
-2.38672582070896	-3.20998374826903\\
-2.49485339177604	-3.13037471143959\\
-2.5975843192724	-3.04417173168765\\
-2.69507289134802	-2.95120936519182\\
-2.78739874658233	-2.85123853588039\\
-2.87456025092164	-2.74393001033717\\
-2.95646655049722	-2.62887769751419\\
-3.03292807461189	-2.50560236111094\\
-3.10364536250223	-2.37355650455739\\
-3.16819623162121	-2.23213140760739\\
-3.22602152236636	-2.08066756335117\\
-3.27640997908704	-1.91847007885575\\
-3.31848329936319	-1.74483093760243\\
-3.35118304149935	-1.5590603234897\\
-3.37326194829048	-1.36052937479863\\
-3.38328330904302	-1.14872660980604\\
-3.37963314927423	-0.923329607521185\\
-3.36055108527246	-0.684292037527928\\
-3.32418620024484	-0.431943504613437\\
-3.26868367304024	-0.167095722866397\\
-3.19230536147705	0.108856576005162\\
-3.09358244533766	0.393857068754825\\
-2.97149044096019	0.685152865976423\\
-2.82562738894695	0.97932494000532\\
-2.65636726985971	1.27240021362405\\
-2.46495643298639	1.56004880299526\\
-2.25352466310871	1.83785155187066\\
-2.02499590669243	2.10160399929956\\
-1.7829041496629	2.34761020364682\\
-1.53114120525856	2.57291909951605\\
-1.27367773558674	2.77546850554358\\
-1.0143017090026	2.95412322218416\\
-0.756409657405247	3.10861616483928\\
-0.502870033834599	3.23941798981028\\
-0.255960938590626	3.34756760350115\\
-0.017371473825098	3.43449392828752\\
0.211750845111671	3.50185165385534\\
0.430725326573695	3.55138427531142\\
0.639252034058069	3.58481934966929\\
0.837326115959362	3.60379494983452\\
1.02515979667224	3.60981291354574\\
};
\addplot [color=mycolor1, forget plot]
  table[row sep=crcr]{%
0.9828277466891	3.70763537359447\\
1.16253075031462	3.70197719960511\\
1.33334009430001	3.68570766277502\\
1.49570695775239	3.65976271051159\\
1.65010344972799	3.62492141184049\\
1.79699993433113	3.58181474962233\\
1.9368480532624	3.53093576568061\\
2.07006805357902	3.47264986453593\\
2.19703921115515	3.40720453038006\\
2.31809233268834	3.33473804069562\\
2.43350349604434	3.25528699630026\\
2.5434883397133	3.16879265862485\\
2.64819633554898	3.07510621461667\\
2.74770457856682	2.97399319805948\\
2.8420107106544	2.86513740040777\\
2.93102467161595	2.74814471903087\\
3.01455905390315	2.62254752861117\\
3.09231794308403	2.487810332777\\
3.16388427545534	2.3433376651173\\
3.22870596337977	2.18848546303003\\
3.286081359338	2.02257742548568\\
3.33514508587561	1.84492815991116\\
3.37485588221517	1.65487516850558\\
3.40398892519859	1.45182182275065\\
3.42113604999295	1.2352932736457\\
3.42471832924838	1.00500652996698\\
3.41301635536507	0.760954447175441\\
3.38422393697969	0.5035008557295\\
3.33653022560203	0.233480401589032\\
3.26823289024617	-0.0477079186871799\\
3.17788032636966	-0.338029307421637\\
3.0644339748185	-0.634783043541202\\
2.92743353400122	-0.93463305884721\\
2.76714030521927	-1.233713376536\\
2.5846302238386	-1.52781119213057\\
2.38181130086192	-1.81261429556168\\
2.16135144700147	-2.08399295313932\\
1.92652001739265	-2.33827507327527\\
1.6809648359463	-2.57247228271275\\
1.4284597389176	-2.78442464350298\\
1.1726614366354	-2.97284981242987\\
0.916908237129788	-3.13730219312071\\
0.664080193666466	-3.2780628929517\\
0.416525582855244	-3.39598864294165\\
0.176046663487098	-3.49234732127893\\
-0.0560693645323572	-3.56866183959142\\
-0.278990371332361	-3.62657612494709\\
-0.492264951839194	-3.66774934602806\\
-0.695745042793501	-3.69377879807574\\
-0.889513035128932	-3.70614832570495\\
-1.0738175758965	-3.7061975177389\\
-1.24901973286528	-3.69510659111797\\
-1.41554952266051	-3.67389233770175\\
-1.57387183832527	-3.64341131302821\\
-1.72446036017558	-3.6043673252688\\
-1.86777791359634	-3.55732108663302\\
-2.00426180745468	-3.50270055590226\\
-2.13431284872163	-3.44081101890906\\
-2.25828692075098	-3.37184433929873\\
-2.37648819938399	-3.29588709096604\\
-2.48916324566855	-3.21292748404959\\
-2.59649535097355	-3.12286114391619\\
-2.69859862106026	-3.02549591924005\\
-2.79551137592754	-2.92055599967027\\
-2.88718852065041	-2.80768573144182\\
-2.97349262037157	-2.68645364415975\\
-3.05418350461285	-2.55635735543462\\
-3.12890635067281	-2.41683021112011\\
-3.19717837652471	-2.26725075241316\\
-3.25837453921522	-2.10695637406438\\
-3.31171301872151	-1.93526283367979\\
-3.35624180374542	-1.75149155094615\\
-3.39082841090712	-1.55500682278472\\
-3.41415566254215	-1.34526505202943\\
-3.4247274669697	-1.12187765939501\\
-3.42088954046595	-0.884688279919619\\
-3.40087069395281	-0.633862869992579\\
-3.36285022003324	-0.369988265853077\\
-3.30505544949928	-0.0941705558364476\\
-3.22589007815862	0.191880185460561\\
-3.12408805098337	0.485796593290265\\
-2.99888000213569	0.78455200019934\\
-2.85015098904599	1.08452724333672\\
-2.67856222643492	1.38165550846322\\
-2.48560896290965	1.67164033388676\\
-2.27359387819706	1.950224895958\\
-2.0455101495427	2.21347602806993\\
-1.80484699864818	2.45803975348327\\
-1.5553471104622	2.68132970671944\\
-1.30075416578548	2.88162457077831\\
-1.04458727672918	3.05807042243704\\
-0.789968913470217	3.21060202528641\\
-0.539518452389949	3.33980861790815\\
-0.295309740119573	3.44677296070807\\
-0.0588814470986692	3.53290880790546\\
0.168715467301757	3.59981463192038\\
0.386848938890716	3.6491533690256\\
0.595228759293788	3.68256119407468\\
0.793830724627843	3.70158370544942\\
0.9828277466891	3.70763537359447\\
};
\addplot [color=mycolor1]
  table[row sep=crcr]{%
0.948683298050514	3.79473319220206\\
1.12996891013208	3.78902241071535\\
1.3027838103604	3.77255915923418\\
1.46753552344581	3.74623053454947\\
1.62465224482371	3.71077288416293\\
1.77456152903207	3.66677970030574\\
1.91767403561802	3.61471073626391\\
2.05437110063153	3.55490128334521\\
2.18499505421831	3.4875709448019\\
2.30984136685525	3.41283153872394\\
2.42915185734506	3.33069397942336\\
2.54310832592638	3.24107414976038\\
2.65182608463458	3.14379790634541\\
2.75534694820212	3.03860547381077\\
2.8536313296431	2.92515559896292\\
2.9465491655584	2.80302996392973\\
3.0338694906698	2.67173851056894\\
3.11524860652315	2.53072651479627\\
3.19021696761039	2.3793844730251\\
3.25816516606007	2.21706211963439\\
3.31832976399147	2.0430881668871\\
3.36978023105031	1.85679760784185\\
3.41140891541445	1.65756857671014\\
3.44192680709927	1.44487070344488\\
3.45986879009691	1.21832646083672\\
3.46361298678702	0.977785965288651\\
3.45141941180798	0.723413822791541\\
3.4214930614047	0.455783738589644\\
3.37207522993717	0.175972765268488\\
3.30156372141992	-0.114357323644244\\
3.20865740947072	-0.412907981729085\\
3.0925136091427	-0.716739496073036\\
2.95289924526556	-1.02233055985344\\
2.79031115637498	-1.3257107709346\\
2.60603988731933	-1.6226624623078\\
2.40215712665582	-1.90897292279738\\
2.18141953804948	-2.18070442957845\\
1.94709823579504	-2.43444260737754\\
1.70275852491129	-2.66748659951055\\
1.45202370644265	-2.877956852001\\
1.19835703486066	-3.0648138241024\\
0.944888096380122	-3.22779787311511\\
0.694297500839398	-3.36731218002033\\
0.448761194715288	-3.48427490957016\\
0.209946190700892	-3.57996468520442\\
-0.0209555786883461	-3.65587741213202\\
-0.243169602253506	-3.71360522906444\\
-0.456269901983927	-3.7547419037936\\
-0.660108423391693	-3.78081428495334\\
-0.854748940791145	-3.79323662458319\\
-1.04040889805107	-3.79328335955738\\
-1.21741050351997	-3.78207578896275\\
-1.38614100991934	-3.76057854694004\\
-1.54702130890241	-3.72960250007616\\
-1.70048159286015	-3.6898114747465\\
-1.84694273429099	-3.64173092446581\\
-1.98680208954437	-3.58575723170233\\
-2.1204225680783	-3.52216679568137\\
-2.24812396922559	-3.45112440165093\\
-2.37017574651009	-3.37269062063008\\
-2.48679050050333	-3.2868281760587\\
-2.59811762050387	-3.19340735751618\\
-2.70423659456836	-3.09221068144532\\
-2.80514959232286	-2.98293711145668\\
-2.90077300437504	-2.86520627071399\\
-2.99092770791458	-2.73856321838213\\
-3.07532793559953	-2.60248453106578\\
-3.1535687736533	-2.45638663503084\\
-3.22511252958097	-2.29963757630341\\
-3.28927451889835	-2.13157368366421\\
-3.34520925490937	-1.95152284702656\\
-3.39189861348128	-1.7588363466938\\
-3.42814429814894	-1.5529312348605\\
-3.45256782598978	-1.33334504716437\\
-3.46362220116165	-1.0998039137315\\
-3.45962024522194	-0.852303713618117\\
-3.43878487541956	-0.591201557879526\\
-3.3993259814164	-0.317311504170883\\
-3.3395463824176	-0.0319942042762583\\
-3.25797518618734	0.262774139257094\\
-3.15352067565559	0.564371272065384\\
-3.02562738470585	0.869554764908886\\
-2.87441511977517	1.17455760801215\\
-2.70077402394131	1.47525676358082\\
-2.50639202484484	1.767403513355\\
-2.29370038309167	2.04688929885519\\
-2.06573813670639	2.31000991273335\\
-1.8259528645897	2.5536887433386\\
-1.57796799985389	2.77562773795799\\
-1.32535178123783	2.9743703475303\\
-1.07141885643386	3.14927865398809\\
-0.819084909373598	3.30044159903781\\
-0.570781691446472	3.42853925898355\\
-0.328428492520302	3.53468897189284\\
-0.093448740680559	3.62029468577783\\
0.13318251344032	3.68691393725669\\
0.350873927242831	3.73614982594868\\
0.55934713845833	3.76956970364108\\
0.758567824726602	3.78864856747597\\
0.948683298050513	3.79473319220206\\
};
\addlegendentry{Аппроксимации}

\addplot [color=mycolor2]
  table[row sep=crcr]{%
2.12132034355964	2.82842712474619\\
2.18582575000323	2.82640660874952\\
2.2459761483502	2.82067996094826\\
2.30316600770785	2.81153894842004\\
2.35833530419819	2.79908319427389\\
2.4121407651283	2.78328504568152\\
2.46505313939319	2.76402339892432\\
2.51741350921551	2.74110157112695\\
2.56946568278775	2.71425682483196\\
2.62137377470791	2.68316571665979\\
2.67322998558975	2.64744786912504\\
2.72505542741334	2.60667010622203\\
2.77679571750944	2.56035271147659\\
2.82831256449113	2.50797963523731\\
2.87937250576692	2.44901464186941\\
2.92963424895255	2.38292549191834\\
2.97863667364875	2.30921809960124\\
3.02579039162557	2.22748193169334\\
3.0703766721051	2.13744642550811\\
3.11155818595542	2.03904567139316\\
3.14840590189045	1.93248505458784\\
3.1799449834646	1.81829951655162\\
3.20521925970046	1.69738980361013\\
3.2233689191928	1.57102231442397\\
3.23371058284854	1.44078163681543\\
3.23580480244767	1.30847305653245\\
3.22949541157146	1.1759834831\\
3.21490900216393	1.04511946165368\\
3.19240995273442	0.917445713323391\\
3.16251366135309	0.794144173493542\\
3.12576385298257	0.675902054710668\\
3.08257536776515	0.562820516214273\\
3.03302813706062	0.454314775018543\\
2.97656426645917	0.348948817128694\\
2.91146921429886	0.244099783305823\\
2.83385680006128	0.135243682019644\\
2.73548220960228	0.0144130439331602\\
2.59871334341581	-0.133168683665202\\
2.38479638365891	-0.336180712113444\\
2.01057175466146	-0.648371525543042\\
1.34210041814469	-1.13791067535239\\
0.375352321166521	-1.75822148716338\\
-0.527448468220416	-2.26262801632552\\
-1.11365847778129	-2.5437869612038\\
-1.4567327767317	-2.68228003389134\\
-1.6691296256022	-2.75259678373195\\
-1.81428800439327	-2.79054335642282\\
-1.92320462744895	-2.81168897715777\\
-2.01132392805712	-2.82302365363059\\
-2.08680248126988	-2.82787464184343\\
-2.15422943553058	-2.82790871121207\\
-2.21634478307291	-2.8239827357917\\
-2.27487422572476	-2.8165264847195\\
-2.33095693947289	-2.80572465930185\\
-2.3853755403757	-2.79160756168836\\
-2.43868469976231	-2.77409782070174\\
-2.49128507004909	-2.75303495626884\\
-2.54346612600745	-2.7281884019911\\
-2.59543032955127	-2.69926454755406\\
-2.64730535029083	-2.66591103001887\\
-2.69914810041563	-2.62772046423944\\
-2.75094277189846	-2.58423541950896\\
-2.80259428647749	-2.53495641420315\\
-2.85391830194335	-2.47935483422118\\
-2.90462903903618	-2.41689283661657\\
-2.95432664759112	-2.34705230341292\\
-3.00248656709011	-2.26937453210355\\
-3.04845424185856	-2.18351130413452\\
-3.0914493816545	-2.08928598065547\\
-3.13058429431929	-1.98676020448289\\
-3.16490008683964	-1.87629788656397\\
-3.19342217916266	-1.75861429449935\\
-3.21523240499265	-1.63479579561046\\
-3.22954956680171	-1.50627700312196\\
-3.23580523135335	-1.37476797309193\\
-3.23369896357047	-1.2421341494958\\
-3.22321880671614	-1.11024302689198\\
-3.20461860896113	-0.980799507713277\\
-3.17835148049538	-0.85519274756288\\
-3.14496439092874	-0.734369602871579\\
-3.10495870881183	-0.618735254776386\\
-3.05861192744495	-0.508062722468019\\
-3.0057323937692	-0.401369642978784\\
-2.94527025004781	-0.296685462718974\\
-2.87460204254734	-0.190562899149911\\
-2.78805571026974	-0.0770301658026245\\
-2.67361421522119	0.0546866392302208\\
-2.50520482475694	0.225105040795738\\
-2.22541867988909	0.474087715816021\\
-1.7209533613293	0.868461990547897\\
-0.880186083660502	1.44500069250654\\
0.110323684998138	2.03883159526995\\
0.859259206598599	2.42737283484743\\
1.30741523004191	2.62511520505649\\
1.57426953542748	2.72304263504847\\
1.74766615441039	2.77435113512036\\
1.87207484559155	2.80265686669549\\
1.96924085114733	2.81832059141065\\
2.0502990590364	2.82612977760851\\
2.12132034355964	2.82842712474619\\
};
\addlegendentry{Сумма Минковского}

\end{axis}

\begin{axis}[%
width=0.798\linewidth,
height=0.597\linewidth,
at={(-0.104\linewidth,-0.066\linewidth)},
scale only axis,
xmin=0,
xmax=1,
ymin=0,
ymax=1,
axis line style={draw=none},
ticks=none,
axis x line*=bottom,
axis y line*=left,
legend style={legend cell align=left, align=left, draw=white!15!black}
]
\end{axis}
\end{tikzpicture}%
        \caption{Эллипсоидальные аппроксимации для 100 направлений.}
\end{figure}

%%%%%%%%%%%%%%%%%%%%%%%%%%%%%%%%%%%%%%%%%%%%%%%%%%%%%%%%%%%%%%%%%%%%%%%%%%%%%%%%
\clearpage
\section{Внутренняя оценка суммы эллипсоидов}

\begin{definition}
        \textit{Сингулярным разложением} матрицы $A \in \setR^{n \times m}$ называется представление матрицы в виде
$$
        A = V \mathit{\Sigma} U^*, 
$$
        где
$$
\begin{aligned}
&V \in \setR^{n \times n}\::\: V^* = V^{-1},\\
&U \in \setR^{m \times m}\::\: U^* = U^{-1},\\
&\mathit{\Sigma} = \mathrm{diag}\left(\sigma_1,\,\ldots,\,\sigma_{\min\{n,\,m\}}\right) \in \setR^{n \times m}\::\:\sigma_1 \geqslant \sigma_2 \geqslant \ldots \geqslant \sigma_{\min\{n,\,m\}}.
\end{aligned}
$$ 
\end{definition}

\begin{theorem}
        Сингулярное разложение
        $A = V \mathit{\Sigma} U^*$
        существует для любой комплексной матрицы $A$.
        Если матрица $A$ вещественная, то матрицы $V$, $\mathit{\Sigma}$ и $U$ также можно выбрать вещественными. 
\end{theorem}

\begin{theorem}
        Старшее сингулярное число $\sigma_1$ матрицы $A = V \mathit{\Sigma} U^*$ является её нормой.
\end{theorem}

\begin{definition}
        Назовём линейное преобразование $\mathcal{A}$ \textit{ортогональным}, если оно сохраняет скалярное произведение, то есть
$$
        \langle \mathcal{A}(x),\,\mathcal{A}(y)\rangle = \langle x,\,y\rangle.
$$
\end{definition}

\begin{theorem}\label{th:unitarnost}
        Необходимым и достаточным условием ортогональности линейного преобразования $\mathcal{A}$ в конечномерном пространстве является унитарность матрицы преобразования $A$, то есть
$$
        A^* = A^{-1}.
$$
\end{theorem}

\begin{assertion}
        Для произвольных векторов $a,\,b\in\setR^{n}$ таких, что $\|a\| = \|b\|$, существует матрица ортогонального преобразования, переводящего $a$ в $b$.
\end{assertion}
\begin{proof}
        
        Построим сингулярное разложение для векторов $a$ и $b$:
$$
        a = V_a \mathit{\Sigma}_a u_a,
        \qquad
        b = V_b \mathit{\Sigma}_b u_b,
$$        
причем $V_a,\,V_b \in \setR^{n \times n}$~--- унитарные матрицы, $u_a,\,u_b\in\{-1,\,1\}\in\setR^1$,
$$
        \mathit{\Sigma}_a = [\sigma_a,\,0,\,\ldots,\,0]\T\in\setR^{n \times 1},\quad
        \mathit{\Sigma}_b = [\sigma_b,\,0,\,\ldots,\,0]\T\in\setR^{n \times 1},\quad
        \sigma_a,\,\sigma_b > 0.
$$
Согласно Теореме~\ref{th:unitarnost} $\sigma_a = \sigma_b$.
Тогда преобразуем выражение для вектора $b$:
\begin{multline*}
        b
        =
        V_b \mathit{\Sigma}_b u_b
        =
        V_b (V_a\T V_a)\mathit{\Sigma}_b u_b
        =
        V_bV_a\T V_a \left( \mathit{\Sigma}_a \frac{\sigma_b}{\sigma_a} \right) \left( u_a\frac{u_b}{u_a} \right)
        =\\=
        V_bV_a\T \frac{\sigma_b\cdot u_b}{\sigma_a\cdot u_a}V_a\mathit{\Sigma}_au_a
        =
        \left(V_b V_a\T \frac{\sigma_b\cdot u_b}{\sigma_a\cdot u_a}\right)a.
\end{multline*}
Так как произведение унитарных матриц есть унитарная матрица, теорема доказана.

\end{proof}

\textbf{Следствие 1.} Далее под ортогональным преобразованием из вектора $a$ в вектор $b$ таких, что $\|a\| = \|b\|$, будем понимать
$$
        \mathrm{Orth}(a,\,b) = u_au_bV_bV_a\T.
$$

\begin{assertion}
        Для суммы Минковского эллипсоидов справедлива следующая оценка
$$
        \sum_{i=1}^n \Varepsilon(q_i,\,Q_i) = 
        \bigcup\limits_{\|l\|=1}\Varepsilon(q_-(l),\,Q_-(l)),
$$
        где
$$
        \begin{aligned}
q_-(l) &= \sum_{i=1}^n q_i,
\\
Q_-(l) &= Q_*\T(l) Q_*(l),
\quad
Q_*(l) = \sum_{i=1}^{n}S_i(l)Q_i^{\nicefrac12},
\\
S_i(l) &= \mathrm{Orth}(Q_i^{\nicefrac12}l,\,\lambda_iQ_1^{\nicefrac12}l),
\quad
\lambda_i = \frac{\langle l,\,Q_il\rangle^{\nicefrac12}}{\langle l,\,Q_1l\rangle^{\nicefrac12}}.
        \end{aligned}
$$
\end{assertion}

\begin{proof}

Будем доказывать для случая $q_i = 0$, $i=\overline{1,\,n}$.
Случай с произвольными центрами~--- аналогично.

Итак рассмотрим эллипсоид $\Varepsilon_- = \Varepsilon(0,\,Q_-)$, $Q_- = Q_*\T Q_*$,
$$
        Q_* = \sum_{i=1}^n S_i Q_i^{\nicefrac12},
$$
где $S_i$~--- некоторые унитарные матрицы. Распишем квадрат опорной функции этого эллипсоида:
\begin{multline*}
        \rho^2(l\,|\,\Varepsilon_-)
        =
        \langle l,\,Q_-l \rangle
        =
        \langle Q_*l,\,Q_*l \rangle
        =
        \sum_{i=1}^n \langle l,\,Q_il \rangle
        +
        \sum_{i \neq j} \left \langle
        S_i Q_i^{\nicefrac12}l,\, S_j Q_j^{\nicefrac12}l
        \right\rangle
        \leqslant\\\leqslant
        \{
        \mbox{Неравенство Коши--Буняковского}
        \}
        \leqslant\\\leqslant
        \sum_{i=1}^n \langle l,\,Q_il \rangle
        +
        \sum_{i \neq j}
        \langle l,\,Q_il \rangle^{\nicefrac12}
        \langle l,\,Q_jl \rangle^{\nicefrac12}
        =
        \left(
        \sum_{i=1}^n \langle l,\,Q_il \rangle^{\nicefrac12}
        \right)^2
        =
        \rho^2\left(
        l\left|
        \sum_{i=1}^n \Varepsilon(q_i,\,Q_i)
        \right.
        \right).
\end{multline*}
Таким образом, получили, что $\Varepsilon_-\subseteq\sum_{i=1}^n\Varepsilon(q_i,\,Q_i)$.

Заметим, что равенство в последней формуле при фиксированном направлении $l \neq 0$ достигается при
$$
        S_iQ_i^{\nicefrac12}l = \lambda_i S_1Q_1^{\nicefrac12}l, 
$$
где $\lambda_i$~--- произвольные неотрицательные константы. Если положить $S_1 = I$, а $\lambda_i$ выбирать, исходя из условий нормировки ($\|Q_i^{\nicefrac12}l\| = \|\lambda_iQ_1^{\nicefrac12}l\|$):
$$
        \lambda_i = \frac{\langle l,\,Q_il\rangle^{\nicefrac12}}{\langle l,\,Q_1l\rangle^{\nicefrac12}},
$$
то получим утверждение теоремы.

\end{proof}
\vfill
\begin{figure}[h]

        \centering
        % This file was created by matlab2tikz.
%
%The latest updates can be retrieved from
%  http://www.mathworks.com/matlabcentral/fileexchange/22022-matlab2tikz-matlab2tikz
%where you can also make suggestions and rate matlab2tikz.
%
\definecolor{mycolor1}{rgb}{0.00000,0.44700,0.74100}%
\definecolor{mycolor2}{rgb}{0.85000,0.32500,0.09800}%
%
\begin{tikzpicture}

\begin{axis}[%
width=0.618\linewidth,
height=0.487\linewidth,
at={(0\linewidth,0\linewidth)},
scale only axis,
xmin=-4,
xmax=4,
xlabel style={font=\color{white!15!black}},
xlabel={$x_1$},
ymin=-3,
ymax=3,
ylabel style={font=\color{white!15!black}},
ylabel={$x_2$},
axis background/.style={fill=white},
axis x line*=bottom,
axis y line*=left,
xmajorgrids,
ymajorgrids,
legend style={at={(0.03,0.97)}, anchor=north west, legend cell align=left, align=left, draw=white!15!black}
]
\addplot [color=mycolor1, forget plot]
  table[row sep=crcr]{%
2.36712178368749	2.74633057020532\\
2.41188305682678	2.74494651398331\\
2.45036041459025	2.74130089565813\\
2.48383536069915	2.73596692388682\\
2.51327948444918	2.72933446739078\\
2.53943919311047	2.72166753585199\\
2.56289492242467	2.71314181506891\\
2.58410302011837	2.70386940753685\\
2.60342571833248	2.69391520541791\\
2.62115280082523	2.68330766515176\\
2.63751738560017	2.67204572812381\\
2.65270745907852	2.66010298488441\\
2.66687427014079	2.64742976011011\\
2.68013832873481	2.63395351039579\\
2.69259349381253	2.61957772004863\\
2.70430943819101	2.6041793136995\\
2.71533261385189	2.58760445132293\\
2.72568568587575	2.56966240776041\\
2.73536523285703	2.55011704085516\\
2.74433729805113	2.52867508999008\\
2.75253007967915	2.50497017880853\\
2.75982261143016	2.47854086090236\\
2.76602760995085	2.44880025137829\\
2.77086559717422	2.41499358451471\\
2.77392567575526	2.37613820122431\\
2.77460549370625	2.33093765236215\\
2.77201821305984	2.27765729690984\\
2.7648464218491	2.21394231571241\\
2.7511098920125	2.13654986655322\\
2.72779317877431	2.04095569483249\\
2.69024836192134	1.92078707468285\\
2.6312539409082	1.7670493095065\\
2.53961413480675	1.56721869756115\\
2.39837999194539	1.30464987086518\\
2.18360687141246	0.959750006540987\\
1.8667826621648	0.516209188824812\\
1.42701784641989	-0.0238781630653391\\
0.87502831245873	-0.620875059217897\\
0.268936641059173	-1.19901922110702\\
-0.309576813214938	-1.68530797756881\\
-0.802100907580029	-2.04896030059078\\
-1.19206161730571	-2.300440537754\\
-1.49041992551985	-2.46708460024485\\
-1.7168759123608	-2.57532489307401\\
-1.88995458136084	-2.64492059461505\\
-2.02413470299114	-2.68919932969544\\
-2.12995725581116	-2.71680265831965\\
-2.21490837909904	-2.73327951527395\\
-2.284286999297	-2.74220891806166\\
-2.34187032729419	-2.74592425079378\\
-2.39038313346388	-2.74596637420131\\
-2.4318186528206	-2.74336542452957\\
-2.46765634360889	-2.73881708355027\\
-2.49901014208278	-2.7327940719517\\
-2.52673028332805	-2.72561759993438\\
-2.551474024202	-2.71750375208859\\
-2.57375536742722	-2.70859393338899\\
-2.5939804436843	-2.69897499500867\\
-2.61247296631128	-2.688692537203\\
-2.6294927100655	-2.67775958603992\\
-2.64524900273341	-2.66616202895009\\
-2.6599105764423	-2.65386167486455\\
-2.67361268902278	-2.64079746083638\\
-2.68646212009346	-2.62688508668123\\
-2.69854042224194	-2.61201517643097\\
-2.70990563030842	-2.59604990865956\\
-2.72059247497691	-2.57881790224895\\
-2.73061098716967	-2.5601069666803\\
-2.73994319179238	-2.53965410007404\\
-2.74853734053432	-2.51713180921804\\
-2.75629877602211	-2.49212938365471\\
-2.76307597883326	-2.46412710489964\\
-2.76863950239953	-2.43246039507853\\
-2.77265014411951	-2.39626942454063\\
-2.7746104875259	-2.35442742359956\\
-2.77379029091263	-2.3054374557388\\
-2.76911009981251	-2.24728211523106\\
-2.75895730729457	-2.17720283029247\\
-2.74089227231279	-2.0913749275097\\
-2.71117633487659	-1.98443346302753\\
-2.66401914162082	-1.84880387542276\\
-2.59041768625865	-1.67383876852635\\
-2.47652724106683	-1.44496938777841\\
-2.30193857203366	-1.14371563581315\\
-2.0396739695219	-0.750852651889909\\
-1.66265564107253	-0.256803878437201\\
-1.16248546087895	0.319450601512933\\
-0.573614265552006	0.917547528939285\\
0.0284025713818862	1.45675943888985\\
0.5685251002902	1.88250662286626\\
1.00960568842464	2.18713123722861\\
1.35150239782658	2.39258093926768\\
1.61139397123953	2.52710219556684\\
1.80908182046774	2.61398894502504\\
1.96116532773652	2.66959557253549\\
2.08006223005889	2.7046815507453\\
2.17466815864962	2.726172144186\\
2.25127885929965	2.73851896753466\\
2.3143624731471	2.74460741055068\\
2.36712178368749	2.74633057020532\\
};
\addplot [color=mycolor1, forget plot]
  table[row sep=crcr]{%
2.10959639276541	2.82833170033868\\
2.13005277719103	2.82769680200034\\
2.14802287471214	2.82599213309054\\
2.16399176571052	2.82344580125796\\
2.17833205481083	2.82021393645668\\
2.19133411253028	2.81640177024618\\
2.20322718958303	2.81207744870569\\
2.21419439157868	2.80728111824648\\
2.22438344129317	2.80203084312009\\
2.23391449132048	2.79632631929519\\
2.24288582434271	2.79015097768998\\
2.25137799857592	2.78347282702145\\
2.25945680590028	2.7762442174586\\
2.26717527446035	2.7684005761966\\
2.27457484262639	2.75985805103207\\
2.28168573976336	2.75050987848557\\
2.28852651598563	2.7402211492275\\
2.29510255122665	2.72882145096567\\
2.30140322103382	2.71609459233055\\
2.30739716790373	2.70176419574736\\
2.31302476509234	2.68547330303474\\
2.31818626415068	2.66675511304507\\
2.32272310676726	2.64499030425581\\
2.32638812037928	2.61934362489206\\
2.32879716902256	2.58866772920285\\
2.32934906157394	2.5513540908222\\
2.32708969033154	2.5050964728495\\
2.32047561983038	2.44650691445407\\
2.3069520645374	2.37047913665329\\
2.28218252151097	2.26911845209114\\
2.23862573128952	2.12995002321369\\
2.16294815306376	1.93306507749366\\
2.03172785596982	1.64737025756866\\
1.80647115907523	1.22912474961823\\
1.43674425455007	0.635722130761753\\
0.895415675386827	-0.122678493566976\\
0.24506740533721	-0.923290250180105\\
-0.373302970705661	-1.59439810765956\\
-0.861342602045722	-2.06153432057587\\
-1.21051470427877	-2.35582021146555\\
-1.45350302607409	-2.53553747281427\\
-1.62427912108232	-2.6457756283534\\
-1.74735161713476	-2.71454779213463\\
-1.83866892421111	-2.75820072395713\\
-1.90838558633582	-2.78623139232265\\
-1.96302614690974	-2.80425772800326\\
-2.00687082453237	-2.81568963402831\\
-2.04279713811814	-2.82265357129101\\
-2.07278812116456	-2.82650998992637\\
-2.09824304490626	-2.82814934742327\\
-2.12017116470066	-2.82816582669431\\
-2.13931522377064	-2.826961933092\\
-2.15623195525253	-2.82481302044249\\
-2.17134569327871	-2.82190798040682\\
-2.18498480181322	-2.81837538766676\\
-2.19740689106534	-2.81430054547294\\
-2.20881656655742	-2.80973669054728\\
-2.21937810527049	-2.8047123437088\\
-2.22922461613086	-2.79923603208632\\
-2.23846471192486	-2.79329914096155\\
-2.24718737595216	-2.78687735484354\\
-2.25546547731453	-2.7799309467999\\
-2.26335822936359	-2.77240402898462\\
-2.27091276797016	-2.76422275734835\\
-2.27816493006503	-2.75529236858525\\
-2.28513922246038	-2.74549279836288\\
-2.2918478707394	-2.73467246525985\\
-2.2982887089987	-2.72263957559321\\
-2.30444148590314	-2.70914996681804\\
-2.31026187654082	-2.69388999154343\\
-2.31567202720427	-2.67645213420482\\
-2.32054568700573	-2.65629974947181\\
-2.32468465040066	-2.63271516862803\\
-2.32778088753628	-2.60472181958332\\
-2.32935449079792	-2.57096482990594\\
-2.32864968427513	-2.52952378807117\\
-2.32445618534273	-2.47761220956851\\
-2.31479431693115	-2.41108425761014\\
-2.29634617602986	-2.32361020748856\\
-2.26340897464661	-2.20528850710705\\
-2.20596668980778	-2.04035887822781\\
-2.10628907021382	-1.80379023894369\\
-1.93390602004408	-1.45787754842354\\
-1.64266626013909	-0.955835256982624\\
-1.18636447694275	-0.272301910216555\\
-0.575627521932152	0.529268583185396\\
0.0765700150810036	1.28296729173816\\
0.635827895370091	1.85302389135356\\
1.05156023208183	2.22653834661497\\
1.34296588926262	2.45673552212428\\
1.54615322639242	2.59724775265813\\
1.69061445079945	2.68411603245696\\
1.79623883340552	2.73880286760147\\
1.87575398878987	2.77375114085251\\
1.93728218443528	2.79624367598011\\
1.98609242944908	2.81064247686136\\
2.02568323586637	2.81963136032678\\
2.05843611040654	2.82490602040426\\
2.08601207953629	2.8275641952934\\
2.10959639276541	2.82833170033868\\
};
\addplot [color=mycolor1, forget plot]
  table[row sep=crcr]{%
-0.751180563789503	1.29301100092275\\
-0.644509272490707	1.2895621961483\\
-0.525853075685341	1.27816235586954\\
-0.394733067383687	1.25710640312682\\
-0.251255417385498	1.22462204940121\\
-0.0963657956622014	1.17906474936042\\
0.0679313764001297	1.11919457949777\\
0.238479475719274	1.04449792159326\\
0.411096645609142	0.95547160953947\\
0.580968788823447	0.85375980073978\\
0.743253360925444	0.742056725339432\\
0.893734219404683	0.623769327860435\\
1.02933068138786	0.502533729359883\\
1.14832654684568	0.381735283717928\\
1.25030593415255	0.264160768651649\\
1.33588484834454	0.151837028466942\\
1.40636299856342	0.046035698606773\\
1.46339657703116	-0.0526160638365817\\
1.50874510492206	-0.143981414805869\\
1.54410401747092	-0.228260871982706\\
1.57101084154411	-0.305863800112672\\
1.59080459914876	-0.37731143443009\\
1.60461898267407	-0.443169442531953\\
1.61339434135726	-0.50400430761343\\
1.6178983941152	-0.560357233657007\\
1.61874953068363	-0.612730111534175\\
1.61643932306818	-0.661579338697342\\
1.6113526202479	-0.707314476704708\\
1.60378462375957	-0.750299690229399\\
1.59395489259106	-0.790856617640623\\
1.58201848840448	-0.829267814089403\\
1.56807456975614	-0.865780234303465\\
1.55217275313418	-0.900608430316382\\
1.53431752429447	-0.933937265273067\\
1.51447093083552	-0.965924014217765\\
1.49255372986368	-0.996699753941291\\
1.4684451102011	-1.02636994778256\\
1.44198106109596	-1.05501411446045\\
1.41295142229282	-1.08268443616728\\
1.38109562819774	-1.10940311195514\\
1.34609715924291	-1.13515819874194\\
1.30757674909159	-1.15989760548665\\
1.26508448786515	-1.18352082038956\\
1.2180911421837	-1.20586786686862\\
1.16597933286764	-1.22670492454442\\
1.10803574356846	-1.24570606215447\\
1.04344637620195	-1.26243069409395\\
0.971298136519321	-1.27629683490271\\
0.890591822483024	-1.2865512138292\\
0.800273888148872	-1.29223914571393\\
0.699296856787975	-1.29218010554772\\
0.586720020892429	-1.28495946080335\\
0.461861094092619	-1.26895243972215\\
0.3245024552969	-1.24240144182672\\
0.175138600776365	-1.2035682010117\\
0.0152226545865632	-1.15097166888395\\
-0.152664711632565	-1.08369478588514\\
-0.324821834412255	-1.00170023256538\\
-0.496685757869856	-0.90605470141019\\
-0.663344832918517	-0.798955642596777\\
-0.820192630077424	-0.683508161867404\\
-0.963532782997699	-0.563297645942172\\
-1.09095770410823	-0.441888842741164\\
-1.20142500062305	-0.322399917701083\\
-1.29507612370551	-0.207246320149147\\
-1.37291296977355	-0.0980684877283813\\
-1.43645044441771	0.0042020468738688\\
-1.48742259418976	0.0992032593483852\\
-1.52757278250983	0.186984897840184\\
-1.55852539637732	0.267866024583987\\
-1.58172126356109	0.342322104544315\\
-1.59839612803496	0.410903378933661\\
-1.60958478267695	0.474180138513564\\
-1.61613840476429	0.532708621189078\\
-1.61874713092658	0.587011554073861\\
-1.61796325588701	0.637568493346236\\
-1.61422266111825	0.684812373734955\\
-1.60786343226419	0.729129764048409\\
-1.59914138437268	0.770863155573305\\
-1.58824260169102	0.810314202630113\\
-1.57529326680601	0.847747236816572\\
-1.56036710004137	0.883392638422727\\
-1.54349071326817	0.917449811620675\\
-1.52464713656047	0.950089605203132\\
-1.50377772010479	0.981456069358695\\
-1.4807825573749	1.0116674553813\\
-1.45551952420259	1.04081635809509\\
-1.42780198561346	1.06896887513935\\
-1.39739519197381	1.09616261560587\\
-1.3640113739163	1.12240333396511\\
-1.32730356145226	1.14765989469051\\
-1.28685821295664	1.17185719094692\\
-1.24218687070083	1.19486655389881\\
-1.19271730234616	1.21649311338865\\
-1.13778500336415	1.23645953866669\\
-1.07662660918946	1.2543856619263\\
-1.00837780643194	1.26976377842146\\
-0.932079852456276	1.28193010776828\\
-0.846700871404774	1.2900342666388\\
-0.751180563789503	1.29301100092275\\
};
\addplot [color=mycolor1, forget plot]
  table[row sep=crcr]{%
1.7832981678353	1.36546589275169\\
2.05592568115162	1.3571744194569\\
2.26714164491445	1.33728673652804\\
2.43111853478521	1.31126156725633\\
2.55943271347477	1.28244136021363\\
2.66093354729971	1.25276022517939\\
2.74218962233567	1.22327936216145\\
2.80802606279813	1.19453958765605\\
2.86198660051068	1.16677896496198\\
2.90668763683666	1.14006289696381\\
2.94407647731309	1.11436054257093\\
2.97561575220362	1.08958916115493\\
3.00241420566575	1.06563949952862\\
3.02531948618789	1.04239000684265\\
3.04498425031883	1.0197144567338\\
3.06191351815595	0.997485660984366\\
3.07649877708056	0.975576844041151\\
3.08904261974966	0.95386158977852\\
3.09977651875239	0.932212879917087\\
3.10887352086999	0.910501505895431\\
3.11645707010401	0.888593987389955\\
3.12260675718842	0.866350032379446\\
3.12736148556441	0.843619502128469\\
3.13072029732992	0.820238784686037\\
3.13264088455171	0.796026421968093\\
3.13303559280109	0.770777769825975\\
3.13176447555128	0.744258389841893\\
3.12862464535523	0.716195767549336\\
3.12333474415734	0.686268814798392\\
3.1155127546208	0.654094433393716\\
3.10464449980951	0.619210182674226\\
3.09003888625681	0.58105180074941\\
3.07076402681151	0.538923992473511\\
3.04555554798029	0.49196258500875\\
3.01268429798848	0.439086064580252\\
2.96976506994043	0.37893515870206\\
2.91348118717393	0.309801736731583\\
2.83919435906196	0.229555616004797\\
2.74041386938704	0.135595692071535\\
2.60813878932407	0.024891213502135\\
2.43022180279671	-0.105743493774361\\
2.19124961386915	-0.258874318262122\\
1.87411209379186	-0.434804268222937\\
1.46518116380506	-0.628916844714506\\
0.964245059921326	-0.829016773640375\\
0.395207777190676	-1.01569772268623\\
-0.193748734909378	-1.16861428361279\\
-0.747622298683287	-1.2757260230298\\
-1.22738617820585	-1.33743352552652\\
-1.61919514519511	-1.36280865849501\\
-1.92824504599151	-1.36321391987884\\
-2.16827266488954	-1.34828035900592\\
-2.35424440438222	-1.32479177202571\\
-2.49912053823983	-1.29705421238368\\
-2.61307727724595	-1.26762654584206\\
-2.7037556868631	-1.23795184232903\\
-2.77678844889193	-1.20879671834807\\
-2.83630900148409	-1.18052984170452\\
-2.88535963560319	-1.15329067060285\\
-2.92619505892335	-1.12708932719676\\
-2.96050096090471	-1.10186489456276\\
-2.98954936666697	-1.07751903751993\\
-3.01430876229406	-1.0539350695173\\
-3.03552236243188	-1.03098844046375\\
-3.05376401994576	-1.00855215247121\\
-3.06947838995699	-0.98649915687812\\
-3.08300991163942	-0.964702930102648\\
-3.09462374740807	-0.943036918843075\\
-3.10452083500761	-0.921373240822967\\
-3.11284852374633	-0.899580840330464\\
-3.11970778215795	-0.877523178040282\\
-3.12515761091966	-0.855055451989272\\
-3.12921702232672	-0.832021282451107\\
-3.13186471906918	-0.808248735438496\\
-3.13303639024817	-0.783545498449621\\
-3.13261931307637	-0.757692949906779\\
-3.13044367314647	-0.730438772495175\\
-3.1262696548114	-0.701487641376271\\
-3.11976885109067	-0.670489360837328\\
-3.1104978196536	-0.63702361653161\\
-3.09786054958415	-0.600580246600419\\
-3.08105502897536	-0.560533616143727\\
-3.05899676892009	-0.516109340945298\\
-3.03020872281921	-0.466341368002129\\
-2.99266220849364	-0.410017606742597\\
-2.94354713401342	-0.345613712643242\\
-2.87894316526232	-0.271219097624817\\
-2.7933613052015	-0.18447081327201\\
-2.67914274745981	-0.0825377845677899\\
-2.5257783825343	0.0377458731415321\\
-2.31943501649626	0.179398198627669\\
-2.04347928473513	0.34412630878845\\
-1.68159158956786	0.530151902243493\\
-1.22537153655025	0.729304947360203\\
-0.685671303026337	0.925396871033157\\
-0.0996443290375696	1.09741544503237\\
0.478086993696042	1.22814867876394\\
0.998202340115934	1.3118123673045\\
1.43427748430264	1.35394217752062\\
1.78329816783529	1.36546589275169\\
};
\addplot [color=mycolor1, forget plot]
  table[row sep=crcr]{%
2.35571328277463	2.22100511820195\\
2.47826782286047	2.21724110275804\\
2.57945563798566	2.2076760066786\\
2.66391338264297	2.19423739652515\\
2.73516709681387	2.17820355908947\\
2.79590243487287	2.16041750763504\\
2.84817524353824	2.1414300272445\\
2.89356994912398	2.12159441970805\\
2.93331677967345	2.10112910266619\\
2.96837799604573	2.08015897733583\\
2.99951128269362	2.05874275651048\\
3.02731643733883	2.03689093924234\\
3.05226984512698	2.01457747210173\\
3.07474996249443	1.99174706294535\\
3.09505610683312	1.96831941038495\\
3.11342217070063	1.94419114633847\\
3.13002638320828	1.91923597242133\\
3.14499787010008	1.89330324619182\\
3.1584204744478	1.86621510225032\\
3.17033405755072	1.83776204934554\\
3.18073327556166	1.8076968479169\\
3.18956359405484	1.77572632588045\\
3.19671403015796	1.74150061739947\\
3.20200576274209	1.70459909187564\\
3.20517527509796	1.66451195769695\\
3.20585002004533	1.62061615374963\\
3.20351362162625	1.57214365762075\\
3.19745620422277	1.51813972897166\\
3.1867033712263	1.45740789312791\\
3.16991440413369	1.38843778139177\\
3.14523622615357	1.30931167074949\\
3.11009471123801	1.2175867111612\\
3.06090031723225	1.11015476668775\\
2.9926453648678	0.983095758531635\\
2.89838838626637	0.831573980390453\\
2.76868761057062	0.649898587637061\\
2.59122233945366	0.432001663954325\\
2.35121103645622	0.172770931598334\\
2.03377775386478	-0.129248435576604\\
1.62959742771365	-0.46779854288743\\
1.14346100946033	-0.825320879963705\\
0.600821886747986	-1.17394011308631\\
0.0442409056935615	-1.4838164622843\\
-0.48163880163311	-1.73456167138362\\
-0.945768934604242	-1.92088020146989\\
-1.33603704925895	-2.04954181981907\\
-1.65480624610948	-2.13266138371325\\
-1.91166875393097	-2.18249310676955\\
-2.11807157549804	-2.2090843405378\\
-2.28459218053074	-2.21985724993794\\
-2.42000314217803	-2.220001836486\\
-2.53121317335989	-2.21304488666874\\
-2.62353282613246	-2.2013486995232\\
-2.70100594205337	-2.18648395128973\\
-2.76670839564354	-2.16948958718696\\
-2.82298812770595	-2.15104809534126\\
-2.87164830808901	-2.13160199457462\\
-2.91408376440492	-2.11143085198241\\
-2.95138155578939	-2.09070216769097\\
-2.98439489737335	-2.06950500443297\\
-3.01379754605468	-2.04787217232614\\
-3.0401239091731	-2.02579474465893\\
-3.06379868525868	-2.00323135043708\\
-3.08515876112665	-1.98011382155867\\
-3.10446929599717	-1.95635020164888\\
-3.12193534452632	-1.93182574111754\\
-3.13770994317351	-1.90640223807004\\
-3.15189925874524	-1.87991589072489\\
-3.16456513585618	-1.85217367237726\\
-3.17572515028752	-1.82294810191056\\
-3.18535005003472	-1.79197014323174\\
-3.19335821697194	-1.75891980928172\\
-3.19960647651815	-1.7234138533874\\
-3.20387617749746	-1.68498968341194\\
-3.20585289974103	-1.64308431012539\\
-3.20509733712814	-1.59700671612584\\
-3.20100372563178	-1.54590148473199\\
-3.19274046786317	-1.48870085916713\\
-3.17916512589608	-1.4240616742884\\
-3.15870247919262	-1.35028304642604\\
-3.12916978909147	-1.26520095598012\\
-3.08752833431522	-1.16605847020069\\
-3.02953722802248	-1.04935888480526\\
-2.9492923192467	-0.910731054901148\\
-2.83866985696159	-0.744886013906675\\
-2.68680635565555	-0.545843632021727\\
-2.48001009197639	-0.307772295691794\\
-2.20297704173767	-0.0269519369913502\\
-1.84266886104903	0.294767101014322\\
-1.39569223068177	0.645751374552216\\
-0.876809173496224	1.0027034275221\\
-0.321315333824451	1.33532954261279\\
0.224920657710854	1.61726471950061\\
0.722602501490466	1.83555657428617\\
1.15023955592394	1.99171058627205\\
1.50385186179304	2.09599160249212\\
1.79026056897879	2.16104741003003\\
2.02047904511769	2.19817110296771\\
2.20572333459932	2.21607809894601\\
2.35571328277463	2.22100511820195\\
};
\addplot [color=mycolor1, forget plot]
  table[row sep=crcr]{%
2.41112293345451	2.67293497202217\\
2.46888392391395	2.67115131797152\\
2.51815038737103	2.6664855191518\\
2.56067745902692	2.65971096292733\\
2.59779343839778	2.65135197568606\\
2.6305157273557	2.64176311344075\\
2.65963253350677	2.63118102002161\\
2.68576094455292	2.61975863731399\\
2.70938859486226	2.60758789630244\\
2.73090382849039	2.59471475314211\\
2.75061769692551	2.58114902195038\\
2.76878007240405	2.56687056330675\\
2.78559143844458	2.55183281196918\\
2.8012114227226	2.53596424504605\\
2.81576478577869	2.51916812567583\\
2.82934531927977	2.50132065759491\\
2.84201790198104	2.48226751893041\\
2.85381878116153	2.46181858333906\\
2.8647539665374	2.43974046017921\\
2.87479541554687	2.41574626808028\\
2.8838744197952	2.3894817673208\\
2.89187122478307	2.36050657363737\\
2.89859935644222	2.32826859854975\\
2.90378227390621	2.29206901942597\\
2.90701863342752	2.2510138451484\\
2.90773033309651	2.20394632395823\\
2.90508412791903	2.14935179353612\\
2.89787220142401	2.08522283788636\\
2.88432854807521	2.00886770995555\\
2.86184503952644	1.916639659751\\
2.82653307728443	1.80356260538734\\
2.77255796402825	1.66284105249546\\
2.69117522918125	1.4853017815288\\
2.56949523498214	1.25900114520614\\
2.38939363438971	0.969699040009248\\
2.12804201406743	0.603783618187781\\
1.76333966100237	0.155977290279563\\
1.28774343536766	-0.358089056184036\\
0.725951136675671	-0.893477459637415\\
0.136840561367889	-1.38815401247061\\
-0.41314368327355	-1.79385084762324\\
-0.881710561305587	-2.09580917159059\\
-1.25843603279283	-2.30612464980497\\
-1.55288592047271	-2.44682454430528\\
-1.7813060775171	-2.53866033056596\\
-1.9593657775584	-2.59741700772015\\
-2.09975332575718	-2.63403834464104\\
-2.21203071942399	-2.65581837430199\\
-2.3032006562605	-2.66755558712192\\
-2.37835240081262	-2.67240733578158\\
-2.44119571607529	-2.67246443748957\\
-2.49445761919681	-2.66912333754563\\
-2.54016488193194	-2.66332432155596\\
-2.57984205540937	-2.6557040973601\\
-2.61464993480629	-2.64669419424165\\
-2.6454827915952	-2.63658501085421\\
-2.67303717734763	-2.62556789562519\\
-2.69786105735113	-2.61376299788296\\
-2.72038922461991	-2.60123775195561\\
-2.74096903869299	-2.58801907026067\\
-2.75987924735955	-2.57410120073673\\
-2.77734377856985	-2.55945048935761\\
-2.793541793656	-2.54400782159856\\
-2.80861487682785	-2.52768920006705\\
-2.82267193569293	-2.51038468758181\\
-2.83579215890869	-2.491955765212\\
-2.84802618740108	-2.47223099410424\\
-2.85939547818274	-2.45099970448246\\
-2.86988964901186	-2.42800324168044\\
-2.87946135839992	-2.40292305034405\\
-2.8880179592727	-2.37536453831611\\
-2.89540870763175	-2.34483518069009\\
-2.90140561910968	-2.31071462828254\\
-2.90567500103259	-2.27221356472263\\
-2.90773500983351	-2.22831655519435\\
-2.90689191136059	-2.17770193064862\\
-2.90214344552749	-2.11862858948262\\
-2.8920308887691	-2.04877526035613\\
-2.87441080959479	-1.96501245936017\\
-2.84610196161767	-1.86308285568593\\
-2.8023432419264	-1.73716857524885\\
-2.7359855348757	-1.5793538010083\\
-2.63637493654125	-1.37909972675918\\
-2.48809271214244	-1.12315205734959\\
-2.27038916682889	-0.796974956580139\\
-1.95964380346866	-0.38978904595946\\
-1.53877736296124	0.0949035646598871\\
-1.01455397500912	0.626922435424875\\
-0.430388009563244	1.14961064156151\\
0.146485075591148	1.60386358275269\\
0.65890842805611	1.95746634107526\\
1.08124245096797	2.21110290073615\\
1.41497076391917	2.38377830939657\\
1.67431219460888	2.49773918811789\\
1.87576040675745	2.57138888298126\\
2.03360307134768	2.61796802131199\\
2.15891606343668	2.64643668053227\\
2.25989814780917	2.66271445002484\\
2.34251989232733	2.67069163243849\\
2.41112293345451	2.67293497202217\\
};
\addplot [color=mycolor1, forget plot]
  table[row sep=crcr]{%
2.2306613868761	2.81803070297518\\
2.25810690410877	2.81718012527407\\
2.28201516003601	2.81491320421253\\
2.30308950693452	2.81155368741527\\
2.32186667064398	2.80732269213538\\
2.33876212010734	2.80236972560499\\
2.35410163785533	2.79679295696782\\
2.36814362598294	2.79065256203978\\
2.38109507726808	2.78397948565953\\
2.39312313224099	2.77678107656757\\
2.40436349690098	2.76904449916777\\
2.41492657297905	2.76073847420239\\
2.42490186835632	2.75181366374805\\
2.43436105653859	2.74220184468911\\
2.44335990651309	2.73181387613749\\
2.45193918373476	2.72053633609572\\
2.46012451013591	2.70822655989054\\
2.4679250472588	2.69470563459613\\
2.47533070951499	2.67974865995893\\
2.48230739246368	2.66307123355111\\
2.48878936392124	2.64431058778368\\
2.49466743013696	2.6229989883018\\
2.4997706103052	2.59852571315108\\
2.50383757236253	2.57008185871995\\
2.50647153235448	2.53657882915073\\
2.50706783155106	2.49652574468635\\
2.50469535887311	2.44784158954154\\
2.49789834998327	2.38756214474631\\
2.48435833264454	2.31137583767289\\
2.46030778970481	2.21288289339427\\
2.41950575059641	2.08242347523106\\
2.35147553534986	1.90531321212694\\
2.23868402405164	1.65958770416935\\
2.05300806937287	1.31467312543502\\
1.75510971736689	0.836517617528178\\
1.30849126874066	0.211189888458031\\
0.720798419557309	-0.511317529079945\\
0.0779140443342507	-1.20791854545575\\
-0.504310404303506	-1.76449354912077\\
-0.963651051034921	-2.15132519346601\\
-1.30133101338113	-2.40097689803586\\
-1.54462840311249	-2.55800318024844\\
-1.72127575581048	-2.65671069052081\\
-1.85211382956641	-2.71925994897991\\
-1.95136885095432	-2.75917169432432\\
-2.02850063951073	-2.78462216400938\\
-2.08981496010035	-2.80061234470425\\
-2.13957654655819	-2.81026070860493\\
-2.18072560715238	-2.81555396271566\\
-2.21533107455486	-2.81778429561671\\
-2.2448783990183	-2.81780785993293\\
-2.27045467227012	-2.81620060789198\\
-2.29286973444643	-2.81335422309845\\
-2.31273685207127	-2.80953638074975\\
-2.33052744438039	-2.80492929976484\\
-2.34660885364804	-2.79965478587334\\
-2.36127083449701	-2.7937906722762\\
-2.37474439994278	-2.78738164589066\\
-2.38721539229624	-2.78044630443655\\
-2.3988343416865	-2.77298159220663\\
-2.40972365384161	-2.76496532407901\\
-2.41998282343293	-2.75635722071773\\
-2.42969213301304	-2.74709867891067\\
-2.43891512794588	-2.73711134941806\\
-2.44770002625738	-2.72629446298113\\
-2.45608010789502	-2.71452071131373\\
-2.46407301201541	-2.70163033252864\\
-2.47167873369492	-2.68742284384416\\
-2.4788759266952	-2.67164557285018\\
-2.48561584719577	-2.65397770782051\\
-2.49181285049689	-2.63400793094603\\
-2.49732966899515	-2.61120267356518\\
-2.50195456246426	-2.58486040061168\\
-2.50536549382862	-2.554044685875\\
-2.50707310865982	-2.51748448173663\\
-2.5063283000101	-2.47342271799767\\
-2.50196930299109	-2.41938216717487\\
-2.49216346758393	-2.35179720417731\\
-2.47396280197052	-2.26542754675948\\
-2.4425289071623	-2.15242386901544\\
-2.38978363183492	-2.00087480010687\\
-2.30214558738244	-1.79274231006888\\
-2.15720197537847	-1.50172612091604\\
-1.92078092466796	-1.09405563207464\\
-1.55158685210626	-0.541139515258476\\
-1.02868761619051	0.144455732587051\\
-0.398536047886839	0.87154781456737\\
0.226482120535896	1.50768862618519\\
0.750336570209395	1.97786021388668\\
1.14618923806126	2.29038532705206\\
1.43290798360277	2.48861007795899\\
1.63980650934666	2.61301346167155\\
1.79139786468341	2.69150173946714\\
1.90500767470046	2.74143986091333\\
1.99224745257609	2.77333611988895\\
2.06083057765806	2.79357153066645\\
2.11593135374411	2.80608480485869\\
2.16108190935215	2.81335835477259\\
2.19874233023691	2.81699046255361\\
2.2306613868761	2.81803070297518\\
};
\addplot [color=mycolor1, forget plot]
  table[row sep=crcr]{%
1.34916822602105	2.61884380257413\\
1.36255348536219	2.61842535800733\\
1.37480683187091	2.61726037653587\\
1.38612752561835	2.61545290337742\\
1.39667669922158	2.61307332288931\\
1.40658638148471	2.61016585503015\\
1.41596613668646	2.60675348673496\\
1.42490800439879	2.60284105382679\\
1.43349020388433	2.598416912692\\
1.44177991815768	2.59345345180439\\
1.44983536863337	2.5879065527139\\
1.45770731417958	2.58171399239607\\
1.46544004576872	2.57479266390242\\
1.47307188948927	2.56703436125339\\
1.48063516614287	2.55829970555158\\
1.48815547206729	2.54840955221251\\
1.49565002401685	2.53713286811873\\
1.50312461938875	2.52416952842272\\
1.5105684469781	2.50912563214587\\
1.51794544380942	2.49147756217942\\
1.52517994674875	2.47051874931118\\
1.53213268434057	2.4452792850012\\
1.53856001621684	2.41440198359606\\
1.5440434126235	2.37594709462455\\
1.54786479332811	2.32707784393597\\
1.54878118856405	2.26354409964061\\
1.54460916042535	2.17882346641676\\
1.53144942360156	2.06269711309166\\
1.50225566229098	1.89898924372039\\
1.44436890588002	1.66251346478269\\
1.33620870262374	1.31721418768455\\
1.14673141186483	0.824026867167908\\
0.849782608285716	0.175946554623806\\
0.461685962801752	-0.548018956645931\\
0.0565516337693074	-1.20228090782282\\
-0.290753796572131	-1.6917502022023\\
-0.554835633179842	-2.01816874854801\\
-0.746358627160081	-2.22639575742822\\
-0.884817909103233	-2.3589451739538\\
-0.986659923035106	-2.44472049770824\\
-1.0634337309332	-2.50144568703854\\
-1.12282433770986	-2.53974002418693\\
-1.16991128279668	-2.5660218412918\\
-1.20809299884232	-2.58425316968421\\
-1.23968783716858	-2.59694161257033\\
-1.26631202944042	-2.60571445312303\\
-1.28911705407786	-2.61165259675128\\
-1.30894048914882	-2.61548905733766\\
-1.32640356331209	-2.617729744945\\
-1.34197546868493	-2.61872870075034\\
-1.35601659862283	-2.61873598932488\\
-1.36880817441296	-2.61792878298912\\
-1.38057291538259	-2.61643186607089\\
-1.39148970600711	-2.61433131778502\\
-1.40170416735516	-2.6116836843858\\
-1.4113363853036	-2.608522081478\\
-1.42048663037264	-2.6048601311055\\
-1.42923963244693	-2.60069429697314\\
-1.43766779293806	-2.5960049543598\\
-1.44583359306231	-2.59075637031308\\
-1.45379136809174	-2.58489564336501\\
-1.46158854894265	-2.57835053799512\\
-1.46926641317011	-2.57102602853142\\
-1.47686032733914	-2.56279922031462\\
-1.48439939058245	-2.55351211746392\\
-1.49190528944329	-2.54296141961962\\
-1.49939002248124	-2.53088409664441\\
-1.50685190825312	-2.51693681519435\\
-1.5142688787915	-2.50066621316221\\
-1.52158734818893	-2.48146525837629\\
-1.5287036798198	-2.4585079950171\\
-1.53543297058567	-2.43064999568616\\
-1.54145557404778	-2.39627321363943\\
-1.5462236004709	-2.35303882637033\\
-1.54879377227572	-2.29748516474355\\
-1.54752207454328	-2.22436239290059\\
-1.53949639900754	-2.12552421645835\\
-1.51947921042305	-1.98811596811773\\
-1.47800070676288	-1.79185510593083\\
-1.39834292507678	-1.50606847102606\\
-1.25373377570828	-1.09096289236972\\
-1.01217443181423	-0.516876345325173\\
-0.663483593147455	0.185309718307808\\
-0.255650441715128	0.892281146234285\\
0.127141962370146	1.46932437295849\\
0.432991150703737	1.87280614464764\\
0.658442536758492	2.13409571476104\\
0.821074581862312	2.30001032816487\\
0.939487523829889	2.40636173951891\\
1.02762555404293	2.47592484154283\\
1.09493604306106	2.52242215305209\\
1.14766234075684	2.55409174160465\\
1.18995036713984	2.57596124909194\\
1.22459961632259	2.59117267921784\\
1.25354049557583	2.60173993143481\\
1.27813357584593	2.60898559194935\\
1.29935845276156	2.61379764430181\\
1.31793477108438	2.61678384887558\\
1.33440123755647	2.61836680110905\\
1.34916822602105	2.61884380257413\\
};
\addplot [color=mycolor1, forget plot]
  table[row sep=crcr]{%
1.33832785075596	1.05856770426014\\
1.71708873794132	1.04706270790821\\
2.00524308812722	1.01997360948505\\
2.22158293985867	0.98568232028984\\
2.38421685443873	0.949191101343268\\
2.5076145272811	0.91313611260959\\
2.6024663989092	0.87874493050746\\
2.67642270413419	0.846477797541033\\
2.7349083019084	0.816402633286111\\
2.78177963434076	0.78840014646188\\
2.81980118326959	0.762271550607141\\
2.85097669866004	0.737793146276745\\
2.87677601172378	0.714742657350381\\
2.89829006124147	0.692910811688235\\
2.91633737206639	0.672105353139954\\
2.93153778559506	0.652151275517667\\
2.9443639890652	0.632889257135411\\
2.95517784931997	0.614173310232574\\
2.96425621180258	0.595868146084142\\
2.97180927822365	0.577846482438437\\
2.97799364973239	0.559986372597205\\
2.982921432524	0.542168554727318\\
2.98666632910257	0.524273773221959\\
2.9892673015886	0.506179992999476\\
2.99073014005198	0.487759402000392\\
2.99102706084517	0.46887507006363\\
2.99009426746616	0.449377098720827\\
2.9878272023742	0.429098051468598\\
2.98407297275139	0.407847392434451\\
2.97861910870765	0.38540457648988\\
2.97117735555587	0.36151031754284\\
2.96136053270739	0.335855404031068\\
2.94864948746336	0.308066221173719\\
2.93234564125112	0.27768587209012\\
2.91150227341512	0.244149475236976\\
2.88482407104405	0.206751914642895\\
2.8505189770917	0.164606224499872\\
2.80607829448728	0.11659141871253\\
2.74795009456766	0.0612912218850788\\
2.67105933973035	-0.00306711709908142\\
2.56812617464324	-0.0786451112657921\\
2.4287765554169	-0.167922293729583\\
2.23862361750258	-0.273355629826909\\
1.97903645009366	-0.396452773264885\\
1.62946597084443	-0.535843685786365\\
1.17549772978648	-0.684372465629206\\
0.623913899953506	-0.827066784603145\\
0.0156603832950979	-0.944197989256905\\
-0.582713904244943	-1.02082682525928\\
-1.1109120882187	-1.05489053430368\\
-1.54004749962896	-1.05543374616649\\
-1.87136800357853	-1.03485403068786\\
-2.12118690488697	-1.00334708083351\\
-2.30861435903366	-0.967504177395819\\
-2.45007070231361	-0.931008400315427\\
-2.55806939607659	-0.895691241500445\\
-2.64167391994001	-0.862335439946601\\
-2.70732795253916	-0.831170891254812\\
-2.75960186319709	-0.802153688786899\\
-2.80175664512797	-0.775115414285042\\
-2.83614241827917	-0.749840185559272\\
-2.86447275609878	-0.726102868844767\\
-2.88801214554155	-0.703686833934361\\
-2.90770435438771	-0.682391113449945\\
-2.92426093572038	-0.662032191603923\\
-2.9382227944058	-0.642443165421628\\
-2.95000341218912	-0.623471699475948\\
-2.9599194431015	-0.604977491641939\\
-2.96821248644699	-0.58682959233689\\
-2.97506458571853	-0.568903718994405\\
-2.98060916195791	-0.55107959889393\\
-2.98493852032551	-0.533238312675461\\
-2.98810867137945	-0.515259573693341\\
-2.99014191896274	-0.497018851100946\\
-2.99102743998762	-0.47838421890837\\
-2.99071988462755	-0.459212783487904\\
-2.98913583058736	-0.439346503305103\\
-2.98614770441886	-0.418607162070534\\
-2.981574503371	-0.396790184171176\\
-2.97516826859042	-0.373656881793238\\
-2.96659470971629	-0.348924587454162\\
-2.95540556317111	-0.322253943280443\\
-2.94099902838946	-0.293232380082817\\
-2.92256272972903	-0.261352524893812\\
-2.898990733363	-0.22598395477037\\
-2.86876167890882	-0.186336478897755\\
-2.82975838133331	-0.141413299954517\\
-2.77899972415068	-0.08995380351499\\
-2.71224382577774	-0.030370353344629\\
-2.62341239539384	0.0393040537726912\\
-2.50379953054959	0.121410226009802\\
-2.34112351143517	0.218486204153281\\
-2.11880645853817	0.332683514259896\\
-1.81669174283804	0.464386512203611\\
-1.41581099600258	0.609717854256119\\
-0.910199242096203	0.757659551907771\\
-0.32300024499543	0.8900063214995\\
0.289093551297695	0.98803779850949\\
0.858233627317148	1.04278813347991\\
1.33832785075596	1.05856770426014\\
};
\addplot [color=mycolor1]
  table[row sep=crcr]{%
2.22721472973942	1.94935886896179\\
2.3886648425121	1.94441487511101\\
2.5195634483927	1.93205457572182\\
2.62670779849127	1.91501766628131\\
2.71531853364	1.89508786650117\\
2.78936864949369	1.87341097537522\\
2.85187814440773	1.85071228241933\\
2.90515032165589	1.82744075521499\\
2.9509520177711	1.80386332654994\\
2.99064783647547	1.78012618777659\\
3.02529909977035	1.7562945351756\\
3.0557365623221	1.73237827445755\\
3.08261387510984	1.7083485310185\\
3.10644696879361	1.68414807630787\\
3.12764310033082	1.65969765855822\\
3.14652223914943	1.63489950162782\\
3.16333268711427	1.60963876367151\\
3.17826225469757	1.5837834336332\\
3.19144589188063	1.55718292750329\\
3.20297034763997	1.52966548843736\\
3.2128761683912	1.50103436869378\\
3.2211571124149	1.47106265802835\\
3.22775682617807	1.43948650786522\\
3.23256237104755	1.40599637010381\\
3.23539387180292	1.3702257107414\\
3.23598913837185	1.33173645766061\\
3.23398152983158	1.29000018434938\\
3.22886850103885	1.24437370419161\\
3.21996707976175	1.19406735000614\\
3.20635080827997	1.13810376656035\\
3.18676025898309	1.07526465092859\\
3.15947593702972	1.0040228169715\\
3.12213828317303	0.922457920675385\\
3.07149545967596	0.828157721349833\\
3.00305879148097	0.718116213117772\\
2.91065699973037	0.588662002756959\\
2.785925244978	0.435496623074383\\
2.61788823849866	0.254008232649376\\
2.39306798253518	0.0401523940333445\\
2.09699730361163	-0.20771900183087\\
1.7183819635995	-0.485932899271695\\
1.25639683089422	-0.782386182967809\\
0.728319451607428	-1.07597695395358\\
0.17048677629427	-1.34156752261761\\
-0.372344302750219	-1.55919115860543\\
-0.863264272549092	-1.72086595365401\\
-1.28299919742863	-1.8302347441912\\
-1.62890152595063	-1.89731652670458\\
-1.90833376158261	-1.93331843320936\\
-2.13240654410545	-1.94782687646357\\
-2.3122537087605	-1.94803363323748\\
-2.45746747168844	-1.93896359662991\\
-2.57574556593285	-1.92399109460735\\
-2.67305603610964	-1.90533078490532\\
-2.75395564612754	-1.88441460443591\\
-2.82190758462804	-1.86215611801615\\
-2.87954766726586	-1.83912801979308\\
-2.92889157233296	-1.81567886095776\\
-2.97149068204595	-1.79200910352774\\
-3.00854741762708	-1.76822049003994\\
-3.04100007190525	-1.74434802267285\\
-3.0695851473103	-1.7203805908067\\
-3.09488323143276	-1.69627413213334\\
-3.11735281839452	-1.67195981527391\\
-3.13735524534064	-1.64734883071173\\
-3.15517299859553	-1.62233479306363\\
-3.17102297575804	-1.59679437416259\\
-3.18506579905015	-1.570586527561\\
-3.19741190629232	-1.54355048208521\\
-3.20812485602917	-1.51550254281508\\
-3.21722203851139	-1.48623162017347\\
-3.22467275544594	-1.45549329519237\\
-3.23039339092638	-1.42300210774249\\
-3.23423911335776	-1.38842161161915\\
-3.23599118590349	-1.35135156242766\\
-3.23533847055249	-1.31131137681683\\
-3.23185101774822	-1.2677187105326\\
-3.22494264006069	-1.21986163848029\\
-3.2138179373885	-1.16686249014231\\
-3.19739719631547	-1.10763095340298\\
-3.17420973909784	-1.0408037811618\\
-3.142242561071	-0.964668760862181\\
-3.09872682936675	-0.877072578179989\\
-3.0398417586358	-0.77531812311728\\
-2.96031892266926	-0.656071422287709\\
-2.85295425309884	-0.515330753928937\\
-2.70811161177175	-0.348574764250833\\
-2.51348980394932	-0.151314840167112\\
-2.25479111797749	0.0795960527623154\\
-1.91840618384291	0.343563906305597\\
-1.49721915275415	0.633014609293368\\
-0.998726638487032	0.931115827109969\\
-0.450328181715587	1.21375185121395\\
0.105465468340197	1.45718721810745\\
0.625945116714116	1.64701594620063\\
1.08251666630864	1.78154116777267\\
1.46486936946524	1.86834353319716\\
1.77626331377619	1.91854897142989\\
2.02656829443923	1.94275375393752\\
2.22721472973941	1.94935886896179\\
};
\addlegendentry{Аппроксимации}

\addplot [color=mycolor2]
  table[row sep=crcr]{%
2.12132034355964	2.82842712474619\\
2.18582575000323	2.82640660874952\\
2.2459761483502	2.82067996094826\\
2.30316600770785	2.81153894842004\\
2.35833530419819	2.79908319427389\\
2.4121407651283	2.78328504568152\\
2.46505313939319	2.76402339892432\\
2.51741350921551	2.74110157112695\\
2.56946568278775	2.71425682483196\\
2.62137377470791	2.68316571665979\\
2.67322998558975	2.64744786912504\\
2.72505542741334	2.60667010622203\\
2.77679571750944	2.56035271147659\\
2.82831256449113	2.50797963523731\\
2.87937250576692	2.44901464186941\\
2.92963424895255	2.38292549191834\\
2.97863667364875	2.30921809960124\\
3.02579039162557	2.22748193169334\\
3.0703766721051	2.13744642550811\\
3.11155818595542	2.03904567139316\\
3.14840590189045	1.93248505458784\\
3.1799449834646	1.81829951655162\\
3.20521925970046	1.69738980361013\\
3.2233689191928	1.57102231442397\\
3.23371058284854	1.44078163681543\\
3.23580480244767	1.30847305653245\\
3.22949541157146	1.1759834831\\
3.21490900216393	1.04511946165368\\
3.19240995273442	0.917445713323391\\
3.16251366135309	0.794144173493542\\
3.12576385298257	0.675902054710668\\
3.08257536776515	0.562820516214273\\
3.03302813706062	0.454314775018543\\
2.97656426645917	0.348948817128694\\
2.91146921429886	0.244099783305823\\
2.83385680006128	0.135243682019644\\
2.73548220960228	0.0144130439331602\\
2.59871334341581	-0.133168683665202\\
2.38479638365891	-0.336180712113444\\
2.01057175466146	-0.648371525543042\\
1.34210041814469	-1.13791067535239\\
0.375352321166521	-1.75822148716338\\
-0.527448468220416	-2.26262801632552\\
-1.11365847778129	-2.5437869612038\\
-1.4567327767317	-2.68228003389134\\
-1.6691296256022	-2.75259678373195\\
-1.81428800439327	-2.79054335642282\\
-1.92320462744895	-2.81168897715777\\
-2.01132392805712	-2.82302365363059\\
-2.08680248126988	-2.82787464184343\\
-2.15422943553058	-2.82790871121207\\
-2.21634478307291	-2.8239827357917\\
-2.27487422572476	-2.8165264847195\\
-2.33095693947289	-2.80572465930185\\
-2.3853755403757	-2.79160756168836\\
-2.43868469976231	-2.77409782070174\\
-2.49128507004909	-2.75303495626884\\
-2.54346612600745	-2.7281884019911\\
-2.59543032955127	-2.69926454755406\\
-2.64730535029083	-2.66591103001887\\
-2.69914810041563	-2.62772046423944\\
-2.75094277189846	-2.58423541950896\\
-2.80259428647749	-2.53495641420315\\
-2.85391830194335	-2.47935483422118\\
-2.90462903903618	-2.41689283661657\\
-2.95432664759112	-2.34705230341292\\
-3.00248656709011	-2.26937453210355\\
-3.04845424185856	-2.18351130413452\\
-3.0914493816545	-2.08928598065547\\
-3.13058429431929	-1.98676020448289\\
-3.16490008683964	-1.87629788656397\\
-3.19342217916266	-1.75861429449935\\
-3.21523240499265	-1.63479579561046\\
-3.22954956680171	-1.50627700312196\\
-3.23580523135335	-1.37476797309193\\
-3.23369896357047	-1.2421341494958\\
-3.22321880671614	-1.11024302689198\\
-3.20461860896113	-0.980799507713277\\
-3.17835148049538	-0.85519274756288\\
-3.14496439092874	-0.734369602871579\\
-3.10495870881183	-0.618735254776386\\
-3.05861192744495	-0.508062722468019\\
-3.0057323937692	-0.401369642978784\\
-2.94527025004781	-0.296685462718974\\
-2.87460204254734	-0.190562899149911\\
-2.78805571026974	-0.0770301658026245\\
-2.67361421522119	0.0546866392302208\\
-2.50520482475694	0.225105040795738\\
-2.22541867988909	0.474087715816021\\
-1.7209533613293	0.868461990547897\\
-0.880186083660502	1.44500069250654\\
0.110323684998138	2.03883159526995\\
0.859259206598599	2.42737283484743\\
1.30741523004191	2.62511520505649\\
1.57426953542748	2.72304263504847\\
1.74766615441039	2.77435113512036\\
1.87207484559155	2.80265686669549\\
1.96924085114733	2.81832059141065\\
2.0502990590364	2.82612977760851\\
2.12132034355964	2.82842712474619\\
};
\addlegendentry{Сумма Минковского}

\end{axis}

\begin{axis}[%
width=0.798\linewidth,
height=0.597\linewidth,
at={(-0.104\linewidth,-0.066\linewidth)},
scale only axis,
xmin=0,
xmax=1,
ymin=0,
ymax=1,
axis line style={draw=none},
ticks=none,
axis x line*=bottom,
axis y line*=left,
legend style={legend cell align=left, align=left, draw=white!15!black}
]
\end{axis}
\end{tikzpicture}%
        \caption{Эллипсоидальные аппроксимации для 10 направлений.}
\end{figure}
\clearpage
\begin{figure}[t]

        \centering
        % This file was created by matlab2tikz.
%
%The latest updates can be retrieved from
%  http://www.mathworks.com/matlabcentral/fileexchange/22022-matlab2tikz-matlab2tikz
%where you can also make suggestions and rate matlab2tikz.
%
\definecolor{mycolor1}{rgb}{0.00000,0.44700,0.74100}%
\definecolor{mycolor2}{rgb}{0.85000,0.32500,0.09800}%
%
\begin{tikzpicture}

\begin{axis}[%
width=0.618\linewidth,
height=0.487\linewidth,
at={(0\linewidth,0\linewidth)},
scale only axis,
xmin=-4,
xmax=4,
xlabel style={font=\color{white!15!black}},
xlabel={$x_1$},
ymin=-3,
ymax=3,
ylabel style={font=\color{white!15!black}},
ylabel={$x_2$},
axis background/.style={fill=white},
axis x line*=bottom,
axis y line*=left,
xmajorgrids,
ymajorgrids,
legend style={at={(0.03,0.97)}, anchor=north west, legend cell align=left, align=left, draw=white!15!black}
]
\addplot [color=mycolor1, forget plot]
  table[row sep=crcr]{%
2.36712178368749	2.74633057020532\\
2.41188305682678	2.74494651398331\\
2.45036041459025	2.74130089565813\\
2.48383536069915	2.73596692388682\\
2.51327948444918	2.72933446739078\\
2.53943919311047	2.72166753585199\\
2.56289492242467	2.71314181506891\\
2.58410302011837	2.70386940753685\\
2.60342571833248	2.69391520541791\\
2.62115280082523	2.68330766515176\\
2.63751738560017	2.67204572812381\\
2.65270745907852	2.66010298488441\\
2.66687427014079	2.64742976011011\\
2.68013832873481	2.63395351039579\\
2.69259349381253	2.61957772004863\\
2.70430943819101	2.6041793136995\\
2.71533261385189	2.58760445132293\\
2.72568568587575	2.56966240776041\\
2.73536523285703	2.55011704085516\\
2.74433729805113	2.52867508999008\\
2.75253007967915	2.50497017880853\\
2.75982261143016	2.47854086090236\\
2.76602760995085	2.44880025137829\\
2.77086559717422	2.41499358451471\\
2.77392567575526	2.37613820122431\\
2.77460549370625	2.33093765236215\\
2.77201821305984	2.27765729690984\\
2.7648464218491	2.21394231571241\\
2.7511098920125	2.13654986655322\\
2.72779317877431	2.04095569483249\\
2.69024836192134	1.92078707468285\\
2.6312539409082	1.7670493095065\\
2.53961413480675	1.56721869756115\\
2.39837999194539	1.30464987086518\\
2.18360687141246	0.959750006540987\\
1.8667826621648	0.516209188824812\\
1.42701784641989	-0.0238781630653391\\
0.87502831245873	-0.620875059217897\\
0.268936641059173	-1.19901922110702\\
-0.309576813214938	-1.68530797756881\\
-0.802100907580029	-2.04896030059078\\
-1.19206161730571	-2.300440537754\\
-1.49041992551985	-2.46708460024485\\
-1.7168759123608	-2.57532489307401\\
-1.88995458136084	-2.64492059461505\\
-2.02413470299114	-2.68919932969544\\
-2.12995725581116	-2.71680265831965\\
-2.21490837909904	-2.73327951527395\\
-2.284286999297	-2.74220891806166\\
-2.34187032729419	-2.74592425079378\\
-2.39038313346388	-2.74596637420131\\
-2.4318186528206	-2.74336542452957\\
-2.46765634360889	-2.73881708355027\\
-2.49901014208278	-2.7327940719517\\
-2.52673028332805	-2.72561759993438\\
-2.551474024202	-2.71750375208859\\
-2.57375536742722	-2.70859393338899\\
-2.5939804436843	-2.69897499500867\\
-2.61247296631128	-2.688692537203\\
-2.6294927100655	-2.67775958603992\\
-2.64524900273341	-2.66616202895009\\
-2.6599105764423	-2.65386167486455\\
-2.67361268902278	-2.64079746083638\\
-2.68646212009346	-2.62688508668123\\
-2.69854042224194	-2.61201517643097\\
-2.70990563030842	-2.59604990865956\\
-2.72059247497691	-2.57881790224895\\
-2.73061098716967	-2.5601069666803\\
-2.73994319179238	-2.53965410007404\\
-2.74853734053432	-2.51713180921804\\
-2.75629877602211	-2.49212938365471\\
-2.76307597883326	-2.46412710489964\\
-2.76863950239953	-2.43246039507853\\
-2.77265014411951	-2.39626942454063\\
-2.7746104875259	-2.35442742359956\\
-2.77379029091263	-2.3054374557388\\
-2.76911009981251	-2.24728211523106\\
-2.75895730729457	-2.17720283029247\\
-2.74089227231279	-2.0913749275097\\
-2.71117633487659	-1.98443346302753\\
-2.66401914162082	-1.84880387542276\\
-2.59041768625865	-1.67383876852635\\
-2.47652724106683	-1.44496938777841\\
-2.30193857203366	-1.14371563581315\\
-2.0396739695219	-0.750852651889909\\
-1.66265564107253	-0.256803878437201\\
-1.16248546087895	0.319450601512933\\
-0.573614265552006	0.917547528939285\\
0.0284025713818862	1.45675943888985\\
0.5685251002902	1.88250662286626\\
1.00960568842464	2.18713123722861\\
1.35150239782658	2.39258093926768\\
1.61139397123953	2.52710219556684\\
1.80908182046774	2.61398894502504\\
1.96116532773652	2.66959557253549\\
2.08006223005889	2.7046815507453\\
2.17466815864962	2.726172144186\\
2.25127885929965	2.73851896753466\\
2.3143624731471	2.74460741055068\\
2.36712178368749	2.74633057020532\\
};
\addplot [color=mycolor1, forget plot]
  table[row sep=crcr]{%
2.35240505464064	2.76075021334929\\
2.39429606969185	2.75945449074689\\
2.43037411286026	2.75603583797069\\
2.46182071964198	2.75102474749105\\
2.48953226372909	2.7447822797979\\
2.51419796801997	2.7375529548978\\
2.53635434440832	2.72949928507746\\
2.55642367540865	2.72072453208959\\
2.57474154346732	2.71128775569393\\
2.59157672930874	2.70121369370841\\
2.6071457034148	2.69049907088173\\
2.62162321073745	2.67911633833517\\
2.63514996235992	2.66701545720637\\
2.6478381125071	2.65412407506413\\
2.65977495873579	2.64034624885663\\
2.67102511914481	2.62555970842345\\
2.68163128510145	2.60961150375056\\
2.69161349794549	2.59231171409547\\
2.70096672992684	2.57342469308632\\
2.70965633462077	2.55265704944946\\
2.71761062979593	2.52964117359486\\
2.72470942365474	2.50391254885291\\
2.73076659186681	2.4748782286687\\
2.73550368771706	2.44177255370809\\
2.73850973115383	2.40359416717204\\
2.73917927620848	2.35901526004097\\
2.73661574356727	2.30624913708954\\
2.72947837780359	2.24285482082059\\
2.7157367198957	2.16544669277417\\
2.69227293033385	2.06926342687234\\
2.65423709364454	1.94753924024194\\
2.59402035163186	1.79063563981677\\
2.49971323572562	1.58501269692596\\
2.35315024241768	1.31256503039365\\
2.12864777932873	0.952067155541005\\
1.79621437753681	0.486680047250032\\
1.33625613088531	-0.078254314338472\\
0.766437217218612	-0.69465416704619\\
0.154503796467904	-1.27852819912494\\
-0.414737908033144	-1.75715359300286\\
-0.888451837771183	-2.10699420135133\\
-1.25739472197187	-2.34495625430441\\
-1.53680289125242	-2.50103042562183\\
-1.74770650766029	-2.60184210371291\\
-1.90850464434176	-2.66650103010414\\
-2.03309498455462	-2.70761509614372\\
-2.13140980780297	-2.73325953784195\\
-2.21042927392142	-2.7485852910304\\
-2.27506472386636	-2.75690363778087\\
-2.32880559720436	-2.76037053426479\\
-2.37416495660386	-2.76040947101594\\
-2.41298006388656	-2.75797261878991\\
-2.44661469143309	-2.75370354023524\\
-2.4760961740651	-2.74803989675378\\
-2.50220924915406	-2.74127920764433\\
-2.52556111884979	-2.73362152710032\\
-2.54662714872804	-2.72519745309993\\
-2.56578336978446	-2.71608663470736\\
-2.583329856457	-2.70632998808298\\
-2.59950769563699	-2.69593763465725\\
-2.61451137202192	-2.68489382800076\\
-2.6284978033132	-2.67315965738214\\
-2.64159285649488	-2.66067399757508\\
-2.65389589416861	-2.64735294979028\\
-2.66548269162024	-2.63308784501451\\
-2.67640689871448	-2.61774172870901\\
-2.68670007084269	-2.60114409074919\\
-2.69637013715653	-2.58308342333688\\
-2.70539798684275	-2.56329695495395\\
-2.71373160169024	-2.54145658319533\\
-2.72127679579021	-2.51714955916172\\
-2.72788306153676	-2.48985177771614\\
-2.73332213400089	-2.45889047089402\\
-2.73725545152916	-2.42339148023845\\
-2.73918432926723	-2.38220377433743\\
-2.73837272033678	-2.33378998386214\\
-2.73372579881979	-2.27606573156975\\
-2.7235964056328	-2.20616156525457\\
-2.70547282555558	-2.12006886105957\\
-2.67547208091285	-2.01211722390011\\
-2.62752302154944	-1.87422773696123\\
-2.55209340449926	-1.69493840971088\\
-2.43439384879093	-1.45844070952189\\
-2.25251213766288	-1.14463119254633\\
-1.97767200659054	-0.732949898706597\\
-1.5822742236057	-0.214800426101576\\
-1.06187245772022	0.384846887446284\\
-0.460163049994717	0.99612986835765\\
0.139878547911511	1.53372324957911\\
0.664968942177641	1.94772499231274\\
1.08537079974014	2.23812186948185\\
1.40696804069012	2.43139655318979\\
1.64956497932395	2.55697472989305\\
1.83339591262838	2.63777387015065\\
1.97462394608275	2.68941180394685\\
2.08504289043958	2.72199559336845\\
2.17298471767437	2.74197177729419\\
2.24429964729586	2.75346450848608\\
2.30312113316005	2.75914104476399\\
2.35240505464064	2.76075021334929\\
};
\addplot [color=mycolor1, forget plot]
  table[row sep=crcr]{%
2.33637221543638	2.77359127925803\\
2.37555924354049	2.77237880820087\\
2.40937123794229	2.76917454596508\\
2.43889707827725	2.76446924085856\\
2.46496343861871	2.7585971151577\\
2.48820653315317	2.75178450780297\\
2.50912212950084	2.744181630855\\
2.52810088298891	2.73588349316341\\
2.54545360964711	2.7269437243832\\
2.56142955275294	2.71738362781807\\
2.57622968413772	2.70719792374284\\
2.5900164143854	2.69635809673865\\
2.60292063833293	2.68481390212879\\
2.61504673328514	2.67249333997666\\
2.62647590475164	2.65930122141101\\
2.63726810274461	2.64511629832438\\
2.64746258445559	2.62978677865868\\
2.65707705402952	2.61312388268925\\
2.66610514396634	2.59489288554136\\
2.67451178632173	2.57480080444892\\
2.68222571341034	2.55247947808503\\
2.68912786135759	2.52746217653571\\
2.69503371742093	2.49915095925159\\
2.69966647054546	2.46677058143266\\
2.70261587912763	2.42930254372515\\
2.70327451243641	2.38538942141003\\
2.70073749781637	2.33319418508968\\
2.69364247410982	2.27019083236438\\
2.67991042087211	2.19285019863564\\
2.65632152704606	2.09616833205675\\
2.61781990553633	1.97297012766626\\
2.556393576908	1.81293522681722\\
2.4593772940689	1.60143082913913\\
2.30730201772547	1.31876620611291\\
2.07262878699913	0.94196376050997\\
1.72391573965135	0.453789171162566\\
1.24344969741539	-0.136391318768668\\
0.656926099516598	-0.770996753093629\\
0.0417886619641087	-1.35809301456615\\
-0.515623703650036	-1.82689899376119\\
-0.969303541781385	-2.16201421670412\\
-1.31723069543787	-2.38645255455632\\
-1.57829933202544	-2.53229442217931\\
-1.77441694676618	-2.62604232656544\\
-1.92364912764283	-2.68605128040974\\
-2.03924856860398	-2.72419816479556\\
-2.13053605171207	-2.74800903746349\\
-2.20400284943405	-2.76225725612886\\
-2.26419305112633	-2.77000296444939\\
-2.31432632834073	-2.77323664854273\\
-2.35671847301735	-2.77327262353084\\
-2.39306188920309	-2.77099058863201\\
-2.42461308321403	-2.76698563780904\\
-2.45231913795599	-2.76166279258598\\
-2.47690408935424	-2.75529748631566\\
-2.49892872559441	-2.74807481098525\\
-2.51883255679587	-2.74011527302354\\
-2.53696365504808	-2.73149180432721\\
-2.55360011571802	-2.72224097327856\\
-2.56896563426422	-2.71237024001503\\
-2.583240872228	-2.7018624133194\\
-2.59657174112777	-2.69067802547869\\
-2.60907536260046	-2.67875604645876\\
-2.62084420261484	-2.66601314846319\\
-2.63194868411551	-2.65234156671407\\
-2.64243842589445	-2.63760545389755\\
-2.65234211207123	-2.62163547080534\\
-2.66166584405338	-2.60422117073813\\
-2.67038963981693	-2.58510049165977\\
-2.6784614893414	-2.5639453284014\\
-2.68578799793619	-2.54034165851501\\
-2.69222006701896	-2.51376194823722\\
-2.69753113435073	-2.48352642453764\\
-2.70138398314494	-2.44874803365179\\
-2.70327961671456	-2.40825314607774\\
-2.70247745867555	-2.3604657339348\\
-2.69786892269196	-2.3032359796149\\
-2.68777407981268	-2.23358396910862\\
-2.66961041986487	-2.14731447222314\\
-2.63934946132841	-2.03844174415473\\
-2.59063069771097	-1.89835713280234\\
-2.51336604572977	-1.71472897173176\\
-2.3917587955001	-1.47040741966589\\
-2.20228433366304	-1.14352791485559\\
-1.91428524384337	-0.712154976758204\\
-1.49989381707048	-0.169092576727588\\
-0.959545669549715	0.453636680936888\\
-0.346998703073263	1.076089367383\\
0.248235503995792	1.60952777200189\\
0.756340489353792	2.01023492636414\\
1.15552289245518	2.28602185753826\\
1.45720069343643	2.46734487511429\\
1.6832308024427	2.58435439754378\\
1.8539597705315	2.65939683811413\\
1.984991847698	2.70730692674371\\
2.087470984034	2.73754728674876\\
2.16917584595401	2.75610613263204\\
2.23553100450468	2.7667989929561\\
2.29035472161473	2.77208920980101\\
2.33637221543638	2.77359127925803\\
};
\addplot [color=mycolor1, forget plot]
  table[row sep=crcr]{%
2.31899511579583	2.78496710127215\\
2.35562981603182	2.78383324360938\\
2.38729775592243	2.78083185798896\\
2.41500170562571	2.77641662221972\\
2.43950357273108	2.77089669699762\\
2.46139029705334	2.76448142446218\\
2.48111974715084	2.75730951010792\\
2.49905313489982	2.74946824664576\\
2.51547819869842	2.74100620271409\\
2.53062596016773	2.731941508788\\
2.54468292505632	2.72226707646172\\
2.55779998553553	2.71195358311898\\
2.57009886950765	2.70095072299508\\
2.58167669807872	2.6891869960311\\
2.59260900645553	2.676568132452\\
2.60295142313444	2.66297410276083\\
2.61274006261332	2.64825451572018\\
2.6219905463938	2.63222203817396\\
2.63069540277576	2.61464325412882\\
2.63881937832462	2.59522608137787\\
2.64629187947257	2.57360243066363\\
2.65299528241172	2.54930414537656\\
2.65874708838325	2.52172927256302\\
2.66327266338513	2.49009418446174\\
2.66616324468682	2.45336466284387\\
2.66681042018401	2.41015524261165\\
2.66430232883478	2.35858004977787\\
2.65725653684722	2.29602884307349\\
2.64354680791206	2.21882754576371\\
2.61985118912713	2.12172286615602\\
2.58090258556044	1.99711165873042\\
2.51826721495296	1.83394833927724\\
2.41847530624473	1.61641992585627\\
2.26065458999511	1.32310807005003\\
2.01528220827559	0.929154871667727\\
1.64952589629397	0.417120007567265\\
1.1482714408185	-0.198666982227323\\
0.54640905154937	-0.850020916681251\\
-0.0691017291823024	-1.43764535137375\\
-0.612199948864583	-1.89453488182135\\
-1.04479878658608	-2.21414158478757\\
-1.3718153396137	-2.42511771476334\\
-1.61516534393746	-2.56107110241224\\
-1.79722379344277	-2.64810143533772\\
-1.93555021217121	-2.70372546121445\\
-2.04270539109275	-2.73908544849924\\
-2.12740163597375	-2.76117652473781\\
-2.19565920649712	-2.77441387849936\\
-2.25167424964972	-2.78162177519906\\
-2.29841322926979	-2.7846360705932\\
-2.33800775259184	-2.78466928587429\\
-2.37201536576894	-2.78253358634189\\
-2.40159285489621	-2.7787788787706\\
-2.42761273157058	-2.77377972370747\\
-2.45074261926824	-2.76779091832854\\
-2.47150014893828	-2.7609835610781\\
-2.49029146552292	-2.75346871855796\\
-2.50743860077189	-2.74531304937084\\
-2.52319916028231	-2.73654908333665\\
-2.53778061266345	-2.72718184373065\\
-2.55135071340469	-2.71719286899586\\
-2.56404509506734	-2.70654228408706\\
-2.57597271479835	-2.69516929825076\\
-2.58721960988191	-2.68299130905499\\
-2.5978512323804	-2.66990163492385\\
-2.60791348668932	-2.65576575356513\\
-2.61743245645768	-2.64041576869778\\
-2.62641265810546	-2.62364263858406\\
-2.63483347157999	-2.60518544727419\\
-2.64264313965648	-2.5847166411094\\
-2.6497493406152	-2.56182162521395\\
-2.65600473681119	-2.53597031735907\\
-2.66118493374687	-2.50647702921167\\
-2.6649546923366	-2.47244312764962\\
-2.66681556845443	-2.43267390052964\\
-2.66602360944957	-2.38555624036334\\
-2.66145791012851	-2.32887614154429\\
-2.65140730054231	-2.25954319945489\\
-2.63321932490039	-2.17317210160491\\
-2.60271796065926	-2.06345017956284\\
-2.5532428781369	-1.92121005516382\\
-2.47411910018783	-1.73318767435553\\
-2.34847036311905	-1.48077624467292\\
-2.1510355617258	-1.1401956316382\\
-1.84919435755573	-0.688107897947388\\
-1.41514372797696	-0.119250957563104\\
-0.855280865430319	0.526083098359011\\
-0.234160930659898	1.1574202658589\\
0.353373767113646	1.68411073867882\\
0.842696187852734	2.07009612486439\\
1.22026952463506	2.33099564995725\\
1.50246043079355	2.50062249797124\\
1.71263134581766	2.60942771688335\\
1.87096320639038	2.67902252218549\\
1.99240428591782	2.72342573228369\\
2.08743319127747	2.75146710747536\\
2.16328836969436	2.76869663302595\\
2.22498824023365	2.77863875701106\\
2.27605403234581	2.78356586654782\\
2.31899511579583	2.78496710127215\\
};
\addplot [color=mycolor1, forget plot]
  table[row sep=crcr]{%
2.30020669994137	2.79497003360658\\
2.33442779348364	2.79391054443853\\
2.36406366192883	2.79110146412444\\
2.39003684377099	2.78696181192526\\
2.41304890569281	2.78177729330581\\
2.43364086342606	2.77574132765827\\
2.45223524061943	2.7689818362945\\
2.46916576788949	2.76157888616564\\
2.48469862514458	2.75357632318545\\
2.4990477973216	2.74498934528772\\
2.51238625496893	2.73580923500262\\
2.52485410803677	2.72600600801785\\
2.53656450354794	2.71552942862473\\
2.54760777625928	2.70430862911225\\
2.55805417112397	2.69225040614971\\
2.56795530659792	2.67923612396515\\
2.57734441530391	2.66511700844601\\
2.58623526248155	2.6497074455038\\
2.5946194802314	2.6327756742544\\
2.60246183676749	2.61403095386041\\
2.60969263975395	2.59310582709883\\
2.61619597887432	2.56953141746383\\
2.62179172332008	2.542702640977\\
2.62620789476024	2.51182856207531\\
2.62903786604208	2.47586050220534\\
2.62967313221243	2.43338630950981\\
2.62719598509371	2.38247244008694\\
2.62020520353049	2.32042471414226\\
2.60652826894451	2.24342194673642\\
2.58274014246277	2.14595321992017\\
2.54335564670146	2.01996541621218\\
2.47949678751822	1.85363764394082\\
2.37683220232506	1.62987621008494\\
2.21297126498285	1.32537261555615\\
1.95626592344349	0.913253136607868\\
1.57259905293611	0.376138452149806\\
1.05034058210994	-0.265538845068828\\
0.43480429548368	-0.931861720771676\\
-0.178039901245535	-1.5171177814555\\
-0.704415039910225	-1.96005554934772\\
-1.11504702584882	-2.26348951010003\\
-1.42134106334064	-2.46111901777462\\
-1.64759369620293	-2.58752837680277\\
-1.81627967256246	-2.66816836568712\\
-1.94431008888774	-2.7196523161846\\
-2.04352060001263	-2.75239019587104\\
-2.12202234173039	-2.77286498390856\\
-2.185382705902	-2.78515204327266\\
-2.237468123753	-2.79185378693279\\
-2.28100707369837	-2.79466128154541\\
-2.31795885555641	-2.79469191916393\\
-2.34975517450592	-2.7926947809758\\
-2.37745987784314	-2.78917754473946\\
-2.40187600344441	-2.78448627762359\\
-2.42361860980879	-2.7788564533485\\
-2.44316509317433	-2.77244605686927\\
-2.46089046981923	-2.76535730926546\\
-2.47709245601405	-2.75765100206573\\
-2.49200950805543	-2.74935591027353\\
-2.50583391707323	-2.74047482637293\\
-2.51872135965986	-2.73098817820737\\
-2.53079784572123	-2.72085581955979\\
-2.54216469215797	-2.71001732897607\\
-2.55290192959034	-2.69839096767582\\
-2.56307038255415	-2.68587129689785\\
-2.5727125250979	-2.67232531332748\\
-2.58185208201898	-2.65758680599976\\
-2.59049219988251	-2.64144844511845\\
-2.59861182582681	-2.62365085150195\\
-2.606159669656	-2.60386752009051\\
-2.61304472916205	-2.5816839133836\\
-2.6191217364694	-2.55656819154151\\
-2.62416887494901	-2.52782972793532\\
-2.6278534437985	-2.49455948149816\\
-2.62967831816298	-2.4555429826136\\
-2.62889718603392	-2.40913136117688\\
-2.62437806539607	-2.35304728947931\\
-2.61437975905681	-2.28408921703587\\
-2.59618015937168	-2.19767714406857\\
-2.56545255417668	-2.08715767216677\\
-2.51522386318292	-1.94277105292442\\
-2.43419553653303	-1.7502485606065\\
-2.30432773508839	-1.48939355868306\\
-2.09848152744881	-1.1343369241126\\
-1.78199733962051	-0.660334255898569\\
-1.32757728226521	-0.0647399507045235\\
-0.748830647813808	0.60249490401494\\
-0.121708119297749	1.24011936632773\\
0.455176324749901	1.75741345798711\\
0.924069208136567	2.12736739288647\\
1.27977406430857	2.37319242181238\\
1.5429475354994	2.53140131594123\\
1.73794202389831	2.63235402061753\\
1.88453375381058	2.69678935047547\\
1.99693909722587	2.73788853009942\\
2.08496394786265	2.76386257147755\\
2.15532142036926	2.77984277321716\\
2.21264266329337	2.78907880461998\\
2.26016874412744	2.79366392340583\\
2.30020669994137	2.79497003360658\\
};
\addplot [color=mycolor1, forget plot]
  table[row sep=crcr]{%
2.27989423025359	2.80367012813669\\
2.31182909506837	2.80268110738935\\
2.33953609037404	2.80005458825546\\
2.36386281781723	2.79617711440432\\
2.38545448717227	2.79131239293695\\
2.40480920655615	2.78563889722791\\
2.42231645590045	2.77927442324971\\
2.43828425414907	2.7722922606117\\
2.45295859430733	2.76473184231069\\
2.46653749575951	2.7566056535963\\
2.4791812362186	2.74790351148229\\
2.49101981016599	2.73859490116034\\
2.50215831474622	2.72862977367262\\
2.51268072396356	2.71793801000599\\
2.52265233631157	2.70642760149479\\
2.53212104114996	2.69398145776753\\
2.54111742329624	2.68045260898249\\
2.54965359336202	2.66565739609871\\
2.55772047072465	2.64936601366224\\
2.56528302613558	2.63128944511472\\
2.57227266530954	2.61106135187156\\
2.57857542753652	2.58821275073957\\
2.58401385557852	2.56213618744709\\
2.58831904026199	2.53203433579653\\
2.5910870560991	2.4968451036069\\
2.59171006355139	2.45513071157973\\
2.58926545338738	2.40491069026903\\
2.58233419211587	2.34340654738009\\
2.56869787676544	2.26664664449753\\
2.54482641466212	2.16885201665013\\
2.50500751445182	2.04149340089486\\
2.4398917399604	1.87191587913018\\
2.33421854445463	1.64162796522165\\
2.16394499060052	1.32524422830162\\
1.8951439591019	0.893735738631335\\
1.49258625080902	0.330161276340648\\
0.949220739602002	-0.337555860020286\\
0.322047854491551	-1.01666579883714\\
-0.284862839313155	-1.59643693298567\\
-0.79218484411334	-2.02345199049009\\
-1.18011059909213	-2.31015726127132\\
-1.4659367423745	-2.49459919940231\\
-1.67570670773632	-2.61180569021119\\
-1.8316675525631	-2.68636393057746\\
-1.94996553587042	-2.73393415501005\\
-2.04168892856542	-2.76420086393755\\
-2.11435811405785	-2.78315379997754\\
-2.1731056775555	-2.79454577443937\\
-2.22148547198689	-2.80077024555496\\
-2.26200198522569	-2.80338244448942\\
-2.29645300988559	-2.80341066898991\\
-2.3261525625886	-2.80154493910368\\
-2.35207766536885	-2.79825337903581\\
-2.37496647227013	-2.79385534504486\\
-2.39538494038225	-2.78856817949742\\
-2.41377286330605	-2.78253755533497\\
-2.43047614459481	-2.77585739089817\\
-2.44576974145904	-2.7685829824573\\
-2.459874171878	-2.76073961003977\\
-2.47296749859146	-2.75232802354111\\
-2.48519406814302	-2.74332768503392\\
-2.49667086234683	-2.73369829912388\\
-2.50749203283999	-2.7233799285648\\
-2.51773198565752	-2.71229181905291\\
-2.52744722803038	-2.70032991313774\\
-2.53667705929853	-2.68736289432676\\
-2.54544306149733	-2.67322644674672\\
-2.55374720206252	-2.65771521859413\\
-2.56156817541707	-2.64057170660506\\
-2.56885534467735	-2.62147088627847\\
-2.57551924031023	-2.59999882451805\\
-2.58141693080415	-2.57562260863434\\
-2.58632953165119	-2.54764751297536\\
-2.58992736438644	-2.51515507764702\\
-2.59171528196931	-2.47691215235485\\
-2.59094547146427	-2.43123506817258\\
-2.58647587216081	-2.37578350774903\\
-2.57653607048487	-2.30724324881995\\
-2.55833386982738	-2.22083344899758\\
-2.52738724967351	-2.10954318377554\\
-2.47639423807661	-1.96298063156187\\
-2.39338883531493	-1.76578796581867\\
-2.2590688978225	-1.49602457183854\\
-2.04425621765017	-1.12553780210027\\
-1.71218911508775	-0.628210747673746\\
-1.23666007004454	-0.00489431482113671\\
-0.63993220037813	0.68322735642396\\
-0.00973267835069244	1.32417897840899\\
0.553493837725301	1.82937408032667\\
1.00045453225166	2.18209994984005\\
1.33414363399572	2.41274133225413\\
1.57879406457006	2.55982595724772\\
1.75926752282458	2.6532640326178\\
1.89473119297408	2.71280842369238\\
1.99861121015519	2.75079006261327\\
2.08003965083966	2.77481706052269\\
2.14522029762639	2.78962086867523\\
2.19841515595272	2.79819150981641\\
2.24260078315941	2.80245391783868\\
2.27989423025359	2.80367012813669\\
};
\addplot [color=mycolor1, forget plot]
  table[row sep=crcr]{%
2.25788870292592	2.81111255677442\\
2.287654916644	2.81019040218352\\
2.3135286657741	2.80773741291854\\
2.33628740238797	2.8041096403732\\
2.35652357987319	2.79955011947793\\
2.37469512569472	2.79422326875967\\
2.39116055804213	2.78823736127986\\
2.40620378693669	2.78165931984663\\
2.42005186230477	2.77452444794562\\
2.43288781059253	2.766842718315\\
2.44485998250258	2.75860262975216\\
2.45608886382766	2.74977325298756\\
2.46667198535894	2.74030482662775\\
2.47668734784351	2.73012807860742\\
2.48619561592421	2.71915230142912\\
2.49524120452002	2.70726207500777\\
2.50385226155616	2.69431238739626\\
2.51203942295326	2.68012172829357\\
2.5197930568574	2.6644624943301\\
2.52707849316102	2.64704770760462\\
2.53382840332611	2.62751254658506\\
2.53993097497982	2.60538841995103\\
2.54521167952737	2.58006611276808\\
2.54940501935253	2.55074262249637\\
2.55211023338693	2.51634321065097\\
2.55272075042301	2.47540513158836\\
2.55030976292687	2.42590114020182\\
2.5434409975938	2.36496711601297\\
2.5298498499946	2.28847602158638\\
2.50589782519448	2.19036749895199\\
2.46563359806467	2.06160428840486\\
2.399202721541	1.88862746060108\\
2.29033437288762	1.65140967199981\\
2.11317626587243	1.32227278785799\\
1.83135769246533	0.869896633398846\\
1.40881012514373	0.278313060321905\\
0.844418841760294	-0.415371904508185\\
0.208111714600051	-1.10458374955839\\
-0.389353359035489	-1.67551535364681\\
-0.875374687306539	-2.08470429598352\\
-1.23998841711074	-2.35422379466866\\
-1.50565468425842	-2.52567217950787\\
-1.69954564436864	-2.63401111711252\\
-1.84339117402125	-2.70277841441165\\
-1.95247849740747	-2.746644511391\\
-2.03713505805024	-2.77457870618071\\
-2.10430312402153	-2.79209626869318\\
-2.15869821427116	-2.80264369279087\\
-2.20357767997811	-2.80841735787632\\
-2.24123489720429	-2.81084482758961\\
-2.27331599547042	-2.81087078825918\\
-2.30102469660419	-2.80912984941941\\
-2.32525672835316	-2.80605301082385\\
-2.34668950945295	-2.80193453872268\\
-2.36584302602081	-2.7969747305353\\
-2.3831218493064	-2.79130767928843\\
-2.39884459497255	-2.78501949729019\\
-2.41326487171072	-2.77816032594382\\
-2.42658635807579	-2.77075218758484\\
-2.43897375075651	-2.76279396009856\\
-2.45056074742211	-2.75426426916161\\
-2.46145584307613	-2.74512277687487\\
-2.47174645662899	-2.73531012829537\\
-2.48150171724134	-2.72474665468707\\
-2.49077409655662	-2.71332979434373\\
-2.49959995033285	-2.70093005551038\\
-2.50799891171573	-2.68738518956101\\
-2.51597193812646	-2.67249204113197\\
-2.52349762850909	-2.65599526143237\\
-2.53052615923393	-2.63757166082288\\
-2.536969773856	-2.61680835723503\\
-2.54268810059327	-2.59317191853194\\
-2.54746548162895	-2.56596418469749\\
-2.55097565952651	-2.53425802893047\\
-2.55272599747514	-2.49680236516834\\
-2.55196784675265	-2.45187920609796\\
-2.54754975556874	-2.39708483508081\\
-2.53767237367441	-2.32898961584025\\
-2.51947199549834	-2.24260356335451\\
-2.48830469404539	-2.13053737931832\\
-2.43651919657954	-1.98171946016395\\
-2.35142882017189	-1.77960281834186\\
-2.21235184848992	-1.50032235638658\\
-1.98788615858815	-1.11322470622252\\
-1.63913213049626	-0.590916485747844\\
-1.14175486096006	0.0611110024336226\\
-0.528318729159554	0.768681689213856\\
0.101619538049815	1.40957805269479\\
0.648126500738803	1.89991988974157\\
1.07179145063932	2.23433020905319\\
1.38341441913172	2.44974622718435\\
1.61005221608102	2.58601032593121\\
1.77663186292935	2.67225729403111\\
1.90153812700941	2.72716117174515\\
1.99736313946378	2.76219713392125\\
2.07256887160442	2.78438736883833\\
2.13286638976315	2.79808153592387\\
2.18216585835773	2.8060240588645\\
2.22319384147655	2.80998144231508\\
2.25788870292592	2.81111255677442\\
};
\addplot [color=mycolor1, forget plot]
  table[row sep=crcr]{%
2.23394855765595	2.81731297801188\\
2.26165542881003	2.81645433828003\\
2.28578522134835	2.81416644391003\\
2.30704958441646	2.81077666426061\\
2.32599148360855	2.80650857474673\\
2.34303114263776	2.8015133554433\\
2.35849801089891	2.79589030904853\\
2.37265334861332	2.78970036790849\\
2.38570639666023	2.7829749647209\\
2.39782607668333	2.7757217405338\\
2.40914951273934	2.76792800626899\\
2.4197882375849	2.7595625174909\\
2.42983265885312	2.75057588292467\\
2.43935515927449	2.74089975437909\\
2.44841205590173	2.7304448060995\\
2.45704452161019	2.71909738093224\\
2.46527845854013	2.70671453784107\\
2.47312318888068	2.69311705736275\\
2.48056867109479	2.67807971880391\\
2.48758072771249	2.66131781199672\\
2.49409343444114	2.64246831961024\\
2.49999728651665	2.62106339378108\\
2.50512088308885	2.59649247127797\\
2.50920239819958	2.56794731733937\\
2.51184457318115	2.53434093478312\\
2.51244251282485	2.49418572108672\\
2.51006559670834	2.44540696967048\\
2.50326034293511	2.38505228302863\\
2.48971464771938	2.30883202088359\\
2.46567636093457	2.21038701377491\\
2.42493925477907	2.08013241442987\\
2.35710180056824	1.90352032711532\\
2.24478420350019	1.65882250405807\\
2.06013938703502	1.31581788713266\\
1.7641832420183	0.840776377405985\\
1.32043053375636	0.219467138432777\\
0.73538690017579	-0.499760516135056\\
0.0930291108947909	-1.1957590007653\\
-0.491215767880153	-1.75424119406673\\
-0.953776314805486	-2.14377244520633\\
-1.29459482026794	-2.39573992748687\\
-1.54045239114532	-2.55441710429746\\
-1.71905444770675	-2.65421650753045\\
-1.85135954830114	-2.71746709182542\\
-1.95172065990269	-2.75782369546833\\
-2.02969800390966	-2.78355326332276\\
-2.09166991701261	-2.79971502556886\\
-2.14195212072707	-2.80946440618582\\
-2.1835205310336	-2.81481166388316\\
-2.21846928352491	-2.81706417099303\\
-2.24830184335604	-2.81708800387865\\
-2.27411836743984	-2.81546568889455\\
-2.29673831343415	-2.81259331688718\\
-2.31678213074571	-2.8087415448277\\
-2.33472667692994	-2.80409461954302\\
-2.35094346387928	-2.79877572531461\\
-2.36572547954666	-2.79286362401391\\
-2.37930626916953	-2.78640361375265\\
-2.39187367454519	-2.77941467487347\\
-2.4035798143547	-2.77189396598476\\
-2.41454836077782	-2.76381938913056\\
-2.42487981794438	-2.7551506532804\\
-2.43465526858326	-2.74582906428706\\
-2.44393888363475	-2.73577611664615\\
-2.45277935678078	-2.72489082997507\\
-2.46121031049194	-2.71304563903389\\
-2.46924960371401	-2.70008048887152\\
-2.47689733375108	-2.68579458068648\\
-2.48413214009023	-2.66993492373347\\
-2.49090514657346	-2.65218042025219\\
-2.49713045655451	-2.63211955817063\\
-2.50267043458998	-2.60921876941312\\
-2.50731287652463	-2.58277689405653\\
-2.51073524368639	-2.55185857118274\\
-2.51244778648669	-2.51519506832468\\
-2.51170143795885	-2.4710338852086\\
-2.50733563119917	-2.41690644816279\\
-2.49752163633762	-2.34926324313888\\
-2.47932146413312	-2.26289383276216\\
-2.44791996027284	-2.15000411186928\\
-2.39529035629046	-1.99878420638165\\
-2.30795966087572	-1.79137743947912\\
-2.1637256527534	-1.50178070923687\\
-1.9287515730817	-1.09660022170732\\
-1.56201415832366	-0.547363379121762\\
-1.04210332896129	0.134299496204066\\
-0.41373736915889	0.85930176547767\\
0.21212340676912	1.49627049229307\\
0.73880059729606	1.96895776582428\\
1.1379412125118	2.28407109237519\\
1.42753200855995	2.48427879822855\\
1.63667724327522	2.6100323100656\\
1.78996263355016	2.68939756091493\\
1.90484457399082	2.73989491206942\\
1.99304948264513	2.77214413487499\\
2.06237662154612	2.79259916061378\\
2.11806121525891	2.80524509917386\\
2.16367804052527	2.81259382941613\\
2.2017171408998	2.81626251283113\\
2.23394855765595	2.81731297801188\\
};
\addplot [color=mycolor1, forget plot]
  table[row sep=crcr]{%
2.20773459083003	2.8222495289209\\
2.23348473718759	2.82145125522293\\
2.2559548318002	2.8193204935936\\
2.27579468679104	2.81615759288709\\
2.29350076287215	2.81216778393027\\
2.30945784242059	2.80748975993564\\
2.32396803739161	2.80221436187556\\
2.33727129498891	2.79639687695855\\
2.34956008794141	2.79006510231986\\
2.36099004991404	2.78322450845005\\
2.37168772374717	2.77586133018681\\
2.38175620219995	2.7679440877519\\
2.3912791795804	2.75942382039327\\
2.40032374934845	2.75023315421115\\
2.40894214567091	2.74028419309749\\
2.41717251349424	2.72946509451973\\
2.42503868370206	2.71763504945624\\
2.43254880918091	2.70461720498769\\
2.43969256182161	2.6901888180198\\
2.44643636739362	2.6740675635357\\
2.45271581340241	2.65589236827855\\
2.4584238176108	2.63519628236681\\
2.4633922399022	2.61136753834277\\
2.4673630846604	2.5835927402505\\
2.46994277399634	2.550772487407\\
2.4705282381474	2.51139364570898\\
2.46818499718316	2.46333216102615\\
2.46144163796091	2.40354278661851\\
2.44793587476921	2.32756282820322\\
2.42379395539544	2.22871109864616\\
2.38253334906901	2.09680478313707\\
2.31315170340179	1.91620172798163\\
2.19703815751015	1.66327247075068\\
2.00413044856463	1.30496254027981\\
1.69266803865093	0.805057180161984\\
1.22639854495036	0.152164411992385\\
0.621530473963147	-0.591629130270673\\
-0.0230689005423035	-1.29031100961773\\
-0.590042352737044	-1.83246425399385\\
-1.02707637532975	-2.20058405277435\\
-1.34373007916348	-2.43471755813798\\
-1.57016562545039	-2.58086921940254\\
-1.73405441270537	-2.67244926556035\\
-1.85536224159972	-2.73044230004995\\
-1.94744880260296	-2.76747089490889\\
-2.0191063409059	-2.79111442363814\\
-2.07616448903187	-2.80599406743089\\
-2.12255587999007	-2.81498850965447\\
-2.16098911599277	-2.81993202932399\\
-2.19337006048283	-2.82201867867359\\
-2.22106776878537	-2.82204050841864\\
-2.24508498412103	-2.82053101044253\\
-2.26616947929479	-2.8178533986692\\
-2.28488817049593	-2.81425608103213\\
-2.30167738166643	-2.80990816867924\\
-2.3168775422682	-2.80492255517029\\
-2.33075753221499	-2.79937107088254\\
-2.34353201168744	-2.79329445550569\\
-2.3553739069556	-2.78670884188176\\
-2.3664234841957	-2.77960980324398\\
-2.37679496514577	-2.77197461198927\\
-2.38658132134539	-2.76376309272819\\
-2.39585766622337	-2.7549172664856\\
-2.40468350734871	-2.74535983930341\\
-2.41310399834668	-2.73499146118265\\
-2.42115022146144	-2.72368654912106\\
-2.42883841979367	-2.71128730962606\\
-2.4361679632172	-2.69759538525143\\
-2.44311764748936	-2.6823602491983\\
-2.44963965175959	-2.6652630245746\\
-2.45565004906105	-2.64589371824984\\
-2.46101406296593	-2.62371878033168\\
-2.46552308833301	-2.59803416998312\\
-2.46885847617421	-2.56789628085639\\
-2.47053353886647	-2.53201838011155\\
-2.46979887629439	-2.48861229487933\\
-2.46548451503161	-2.4351416266843\\
-2.4557308880001	-2.367930037746\\
-2.43752103026658	-2.28153102169332\\
-2.40585540154001	-2.16771136528395\\
-2.35229751005661	-2.01384958783898\\
-2.26250565626076	-1.80063144759285\\
-2.11258557093622	-1.49966066179349\\
-1.86602728037014	-1.07454491114853\\
-1.4797877932246	-0.496095355052655\\
-0.936805752189655	0.215941648042236\\
-0.295978143629587	0.955565328864142\\
0.321438639117415	1.5841689940824\\
0.825136322831452	2.0363613926554\\
1.19865633278655	2.33130040158839\\
1.46632323755697	2.51636869632571\\
1.65850147464714	2.63192520700021\\
1.79906607079213	2.70470478186873\\
1.90442332523672	2.75101495342029\\
1.98541221796483	2.78062512601975\\
2.0491795001313	2.79943900899214\\
2.10050143774229	2.81109359900658\\
2.14263303395119	2.8178803855583\\
2.17784033336939	2.82127556070905\\
2.20773459083003	2.8222495289209\\
};
\addplot [color=mycolor1, forget plot]
  table[row sep=crcr]{%
2.17877089191157	2.8258491631413\\
2.20266195475253	2.82510825619919\\
2.22355303970033	2.82312699966302\\
2.24203576130143	2.82018025751118\\
2.25856281479918	2.8164559480498\\
2.27348562190955	2.81208097374269\\
2.28708055244381	2.80713818397963\\
2.29956747509971	2.80167753770813\\
2.31112305815506	2.79572340861465\\
2.32189040561136	2.78927923667156\\
2.33198608000072	2.78233027086249\\
2.34150521308389	2.77484485181766\\
2.35052516945291	2.76677448124427\\
2.35910806153742	2.75805277512825\\
2.36730228879221	2.74859327147649\\
2.37514316822329	2.73828593921934\\
2.38265262068332	2.72699209253963\\
2.38983775991052	2.7145372308512\\
2.39668807680493	2.70070106681206\\
2.4031706870237	2.68520362462547\\
2.40922276235853	2.66768571092972\\
2.41473970458509	2.6476811525825\\
2.41955668385167	2.62457674239169\\
2.42341956070791	2.59755346032928\\
2.42593840079165	2.56549858692832\\
2.42651175108008	2.52687163564035\\
2.42420061301231	2.47949555154823\\
2.41751383304105	2.42022482642527\\
2.40403424708072	2.34440848059438\\
2.37975463609059	2.24501166999273\\
2.33788680348285	2.11118778930979\\
2.26675737864986	1.92606679741267\\
2.14637058166322	1.66387081861997\\
1.94418590217523	1.28837621186511\\
1.61553544103642	0.760902998569861\\
1.12539644326509	0.0745020623084606\\
0.502231119522599	-0.692028792653104\\
-0.139898794240548	-1.3883092500649\\
-0.685264967956727	-1.90997587733958\\
-1.09481076588507	-2.25501661218449\\
-1.38703683868072	-2.47111344877806\\
-1.59446713793099	-2.60500512550983\\
-1.74420428353216	-2.68867837342763\\
-1.85503000691998	-2.74165972225064\\
-1.93926554994634	-2.77553050794186\\
-2.00493893990024	-2.79719875117125\\
-2.0573470000794	-2.81086507442919\\
-2.10005537972538	-2.81914490964075\\
-2.13551861595477	-2.82370597694059\\
-2.16546446419626	-2.82563535550257\\
-2.19113517288083	-2.82565529704955\\
-2.21344172129508	-2.82425307049057\\
-2.23306440504989	-2.821760889237\\
-2.2505197612451	-2.81840617041766\\
-2.26620596485001	-2.81434373682659\\
-2.28043418264582	-2.80967676338909\\
-2.29345058827861	-2.80447053516409\\
-2.30545204646153	-2.79876149299468\\
-2.31659742199454	-2.79256309496412\\
-2.32701580304022	-2.78586944178028\\
-2.33681249704063	-2.77865724760119\\
-2.3460733713891	-2.770886495416\\
-2.35486791396654	-2.76249994441075\\
-2.36325124550464	-2.75342152166876\\
-2.37126520234114	-2.74355350788806\\
-2.37893850604596	-2.7327722961707\\
-2.38628592868827	-2.72092234306972\\
-2.39330622982574	-2.70780771495221\\
-2.39997845694397	-2.69318032125385\\
-2.40625592354149	-2.67672345836231\\
-2.41205673908975	-2.65802856431732\\
-2.4172490422457	-2.63656193879927\\
-2.42162786717519	-2.61161633019505\\
-2.42487845667244	-2.58223923626837\\
-2.42651708281877	-2.54712463397145\\
-2.42579363098841	-2.50444610188262\\
-2.42152761816296	-2.45159422103287\\
-2.41182571193496	-2.38475522131539\\
-2.39358448774803	-2.29822459719431\\
-2.36160130059077	-2.18328435227048\\
-2.30698424867367	-2.02640706535026\\
-2.21441724776119	-1.80663567049949\\
-2.05810232219544	-1.49287274945234\\
-1.79859143345776	-1.0454639060061\\
-1.39108370550384	-0.435140869323708\\
-0.82480235487054	0.307613699130504\\
-0.174921739515048	1.05796528414176\\
0.429057793926385	1.67312163418713\\
0.906601218695649	2.10195276089735\\
1.25353586858761	2.37594377011342\\
1.4994538511774	2.5459881258258\\
1.67519385396721	2.65166344876194\\
1.80358749812364	2.71814129356404\\
1.89989066764397	2.7604709173079\\
1.97404145527801	2.78758017016175\\
2.03254641694308	2.8048407202027\\
2.0797395803064	2.81555711454915\\
2.11857098081715	2.82181180533255\\
2.15109432677297	2.82494776723096\\
2.17877089191157	2.8258491631413\\
};
\addplot [color=mycolor1, forget plot]
  table[row sep=crcr]{%
2.14638286171033	2.82796423666817\\
2.1685094971813	2.82727778496317\\
2.18790043752099	2.82543857464301\\
2.20509246287125	2.82269741653787\\
2.22049703837008	2.81922588001395\\
2.23443415506741	2.81513972630638\\
2.24715592449306	2.81051424969982\\
2.25886328880231	2.8053943710103\\
2.2697180143095	2.79980122861067\\
2.27985138828809	2.79373634775355\\
2.28937056027584	2.78718405488052\\
2.29836315520485	2.78011253502649\\
2.30690057317799	2.77247374551488\\
2.315040240036	2.76420225953563\\
2.32282695786662	2.75521299287993\\
2.33029340631105	2.74539764554835\\
2.33745974772071	2.73461954720598\\
2.34433217475774	2.72270640758823\\
2.3509000855707	2.70944020643105\\
2.3571313455368	2.6945430608253\\
2.36296474045027	2.6776572978299\\
2.36829814852421	2.6583169976477\\
2.37296998737321	2.63590671969835\\
2.37672981520411	2.60960056517011\\
2.37919099741342	2.57827042825493\\
2.37975296956227	2.54034492229864\\
2.37747065233213	2.4935876616492\\
2.37082972145653	2.43474115356513\\
2.35735041420007	2.35894366511076\\
2.33287430374968	2.25876240221542\\
2.29026634313273	2.12259835813708\\
2.21708805566767	1.93217933656367\\
2.09176081156176	1.65926844659496\\
1.87895213931415	1.26409107160541\\
1.53103973962933	0.705711375456089\\
1.01576355532262	-0.0160168121401194\\
0.376896810506748	-0.802150579667117\\
-0.256940380314764	-1.48973187510391\\
-0.776081907587097	-1.98647819331637\\
-1.15629517575985	-2.30686986686515\\
-1.423932858916	-2.50480321137639\\
-1.61280149684316	-2.6267181121602\\
-1.74893632450621	-2.70279042246697\\
-1.84977142266498	-2.75099469191492\\
-1.9265562643376	-2.78186855061526\\
-1.98656202171616	-2.80166594608952\\
-2.03456899024968	-2.81418391236825\\
-2.07379132074096	-2.82178736325071\\
-2.10644193588614	-2.82598625428656\\
-2.13407993566315	-2.82776659093309\\
-2.1578277979326	-2.82778475114576\\
-2.17850995701177	-2.82648439457411\\
-2.19674311789018	-2.8241684818417\\
-2.21299635522654	-2.82104461336558\\
-2.22763190433368	-2.81725411473201\\
-2.24093335951189	-2.81289097509428\\
-2.25312549401949	-2.80801429077649\\
-2.26438839561164	-2.80265643842267\\
-2.27486766896483	-2.79682835032\\
-2.28468185966512	-2.79052274220007\\
-2.29392786811972	-2.78371581231931\\
-2.3026848646701	-2.7763677098011\\
-2.31101703952361	-2.76842191179093\\
-2.31897539098827	-2.75980352179609\\
-2.32659865096067	-2.75041638306813\\
-2.33391335056484	-2.74013877136725\\
-2.3409329250638	-2.72881726957966\\
-2.34765562652752	-2.71625820452968\\
-2.35406082818988	-2.70221570273356\\
-2.36010302314961	-2.68637493163734\\
-2.36570236978041	-2.66832832882923\\
-2.37072988989644	-2.64754140238456\\
-2.37498415356	-2.62330269720879\\
-2.37815406021903	-2.59464921277142\\
-2.37975834147586	-2.56025294094827\\
-2.37904510441422	-2.51824449556296\\
-2.37482104878933	-2.46593286163663\\
-2.36515394587688	-2.39935067236393\\
-2.34684223351532	-2.31250396918446\\
-2.31445313585167	-2.19612735400624\\
-2.2585767975166	-2.03566252518764\\
-2.16278387532862	-1.80827193345053\\
-1.99910792181745	-1.47978249385533\\
-1.72488195578284	-1.00704113098572\\
-1.29408696393288	-0.361795492332068\\
-0.704865784088944	0.41126327946243\\
-0.0506195584320581	1.16697280720031\\
0.534227949968228	1.76287629141842\\
0.982439525342405	2.16547356894941\\
1.3019570069277	2.41784782802175\\
1.52636237507873	2.57302658281129\\
1.68619549819523	2.66913835746718\\
1.80294774058144	2.72958802234225\\
1.890643171054	2.76813310488603\\
1.95831241293265	2.79287176878262\\
2.01183572375014	2.80866181552588\\
2.05512133356105	2.81849028502511\\
2.09082835129564	2.82424123560066\\
2.12080904146535	2.82713164121354\\
2.14638286171033	2.82796423666817\\
};
\addplot [color=mycolor1, forget plot]
  table[row sep=crcr]{%
2.10959639276541	2.82833170033868\\
2.13005277719103	2.82769680200034\\
2.14802287471214	2.82599213309054\\
2.16399176571052	2.82344580125796\\
2.17833205481083	2.82021393645668\\
2.19133411253028	2.81640177024618\\
2.20322718958303	2.81207744870569\\
2.21419439157868	2.80728111824648\\
2.22438344129317	2.80203084312009\\
2.23391449132048	2.79632631929519\\
2.24288582434271	2.79015097768998\\
2.25137799857592	2.78347282702145\\
2.25945680590028	2.7762442174586\\
2.26717527446035	2.7684005761966\\
2.27457484262639	2.75985805103207\\
2.28168573976336	2.75050987848557\\
2.28852651598563	2.7402211492275\\
2.29510255122665	2.72882145096567\\
2.30140322103382	2.71609459233055\\
2.30739716790373	2.70176419574736\\
2.31302476509234	2.68547330303474\\
2.31818626415068	2.66675511304507\\
2.32272310676726	2.64499030425581\\
2.32638812037928	2.61934362489206\\
2.32879716902256	2.58866772920285\\
2.32934906157394	2.5513540908222\\
2.32708969033154	2.5050964728495\\
2.32047561983038	2.44650691445407\\
2.3069520645374	2.37047913665329\\
2.28218252151097	2.26911845209114\\
2.23862573128952	2.12995002321369\\
2.16294815306376	1.93306507749366\\
2.03172785596982	1.64737025756866\\
1.80647115907523	1.22912474961823\\
1.43674425455007	0.635722130761753\\
0.895415675386827	-0.122678493566976\\
0.24506740533721	-0.923290250180105\\
-0.373302970705661	-1.59439810765956\\
-0.861342602045722	-2.06153432057587\\
-1.21051470427877	-2.35582021146555\\
-1.45350302607409	-2.53553747281427\\
-1.62427912108232	-2.6457756283534\\
-1.74735161713476	-2.71454779213463\\
-1.83866892421111	-2.75820072395713\\
-1.90838558633582	-2.78623139232265\\
-1.96302614690974	-2.80425772800326\\
-2.00687082453237	-2.81568963402831\\
-2.04279713811814	-2.82265357129101\\
-2.07278812116456	-2.82650998992637\\
-2.09824304490626	-2.82814934742327\\
-2.12017116470066	-2.82816582669431\\
-2.13931522377064	-2.826961933092\\
-2.15623195525253	-2.82481302044249\\
-2.17134569327871	-2.82190798040682\\
-2.18498480181322	-2.81837538766676\\
-2.19740689106534	-2.81430054547294\\
-2.20881656655742	-2.80973669054728\\
-2.21937810527049	-2.8047123437088\\
-2.22922461613086	-2.79923603208632\\
-2.23846471192486	-2.79329914096155\\
-2.24718737595216	-2.78687735484354\\
-2.25546547731453	-2.7799309467999\\
-2.26335822936359	-2.77240402898462\\
-2.27091276797016	-2.76422275734835\\
-2.27816493006503	-2.75529236858525\\
-2.28513922246038	-2.74549279836288\\
-2.2918478707394	-2.73467246525985\\
-2.2982887089987	-2.72263957559321\\
-2.30444148590314	-2.70914996681804\\
-2.31026187654082	-2.69388999154343\\
-2.31567202720427	-2.67645213420482\\
-2.32054568700573	-2.65629974947181\\
-2.32468465040066	-2.63271516862803\\
-2.32778088753628	-2.60472181958332\\
-2.32935449079792	-2.57096482990594\\
-2.32864968427513	-2.52952378807117\\
-2.32445618534273	-2.47761220956851\\
-2.31479431693115	-2.41108425761014\\
-2.29634617602986	-2.32361020748856\\
-2.26340897464661	-2.20528850710705\\
-2.20596668980778	-2.04035887822781\\
-2.10628907021382	-1.80379023894369\\
-1.93390602004408	-1.45787754842354\\
-1.64266626013909	-0.955835256982624\\
-1.18636447694275	-0.272301910216555\\
-0.575627521932152	0.529268583185396\\
0.0765700150810036	1.28296729173816\\
0.635827895370091	1.85302389135356\\
1.05156023208183	2.22653834661497\\
1.34296588926262	2.45673552212428\\
1.54615322639242	2.59724775265813\\
1.69061445079945	2.68411603245696\\
1.79623883340552	2.73880286760147\\
1.87575398878987	2.77375114085251\\
1.93728218443528	2.79624367598011\\
1.98609242944908	2.81064247686136\\
2.02568323586637	2.81963136032678\\
2.05843611040654	2.82490602040426\\
2.08601207953629	2.8275641952934\\
2.10959639276541	2.82833170033868\\
};
\addplot [color=mycolor1, forget plot]
  table[row sep=crcr]{%
2.0669689142298	2.8265001166603\\
2.08585218487773	2.82591377173339\\
2.10248435713509	2.824335791098\\
2.11730176510645	2.82197287100359\\
2.13064039591691	2.81896657267359\\
2.14276272243324	2.81541218008658\\
2.15387647302746	2.8113710713481\\
2.16414796811114	2.80687885374025\\
2.17371172139776	2.80195064490338\\
2.18267741955627	2.79658435510239\\
2.19113501896617	2.79076249383181\\
2.19915845114387	2.78445280557154\\
2.2068082596289	2.77760788518744\\
2.21413336951179	2.77016380211266\\
2.22117209536673	2.76203765196474\\
2.227952408255	2.75312383596871\\
2.23449139325266	2.74328872273134\\
2.24079371916766	2.73236314815236\\
2.24684878895986	2.72013192028773\\
2.25262600707546	2.70631905782578\\
2.25806722804191	2.69056680598815\\
2.26307483266911	2.67240537597125\\
2.26749281998738	2.65120855342423\\
2.27107644143286	2.62612730046412\\
2.27344254173279	2.59598829347994\\
2.27398653889816	2.55913525875812\\
2.27174012871236	2.51317476307575\\
2.26512077120442	2.45455886598506\\
2.25147861067759	2.37788450758131\\
2.22625743325768	2.27469939362629\\
2.18142130782272	2.13147407201847\\
2.10255582239114	1.92633668518003\\
1.96404509251501	1.62482097010485\\
1.72381856839937	1.17882240111403\\
1.329183818134	0.545395040165709\\
0.761790068907563	-0.249787526505068\\
0.106628164730735	-1.05674278201958\\
-0.487505353795216	-1.70185391858681\\
-0.939357911409621	-2.13448330265905\\
-1.25594103658005	-2.40134149815461\\
-1.47431937685956	-2.56286478580134\\
-1.62749848260503	-2.66174359928501\\
-1.73804377195806	-2.72351381077397\\
-1.82030363294048	-2.76283520906365\\
-1.8833232358948	-2.78817183992138\\
-1.93289296053334	-2.80452423009049\\
-1.97280938807871	-2.81493111993595\\
-2.00562765622305	-2.82129200404624\\
-2.03311197406603	-2.82482564197129\\
-2.05651006259846	-2.82633216974524\\
-2.07672408698264	-2.82634706367953\\
-2.09441965051151	-2.82523401181922\\
-2.11009691749222	-2.82324233404957\\
-2.12413805178504	-2.82054327423604\\
-2.13683950901854	-2.81725336930394\\
-2.1484344317427	-2.81344970908028\\
-2.15910844240422	-2.80917997274218\\
-2.16901094268674	-2.80446900297834\\
-2.178263291945	-2.79932300498311\\
-2.18696477081792	-2.79373204110065\\
-2.19519693282776	-2.78767122462506\\
-2.20302674361282	-2.78110083463718\\
-2.21050876536302	-2.77396543908935\\
-2.2176865377223	-2.76619199986685\\
-2.22459321797863	-2.75768682147306\\
-2.23125145805402	-2.74833107563165\\
-2.23767239863978	-2.73797446554853\\
-2.24385353281524	-2.72642635543562\\
-2.24977500441801	-2.71344333660651\\
-2.25539361411527	-2.69871165550837\\
-2.26063332867985	-2.68182206458618\\
-2.26537028309853	-2.66223325496721\\
-2.26940886619963	-2.63921770416699\\
-2.27244298687703	-2.61177982613277\\
-2.27399205537368	-2.57852946857343\\
-2.27329264767083	-2.53748169639891\\
-2.26911034445943	-2.4857320441213\\
-2.25940393141311	-2.41891713013389\\
-2.24071038667139	-2.33030101858926\\
-2.20699631199343	-2.20921528422324\\
-2.1475106522554	-2.03845428181591\\
-2.04296071306462	-1.79036922549699\\
-1.85994625905196	-1.42317525718662\\
-1.5486658362723	-0.886598994775004\\
-1.06463898401359	-0.161388990802826\\
-0.435691160374269	0.6644529412044\\
0.205745366158514	1.40610510429633\\
0.732167458520347	1.9429031736525\\
1.11235206856804	2.28455348115237\\
1.37509630217415	2.492128076491\\
1.55741791816583	2.61821320681005\\
1.68704870813498	2.69616214335981\\
1.78204832273847	2.74534615469033\\
1.85379818103074	2.77687987495302\\
1.90951601160325	2.79724714593898\\
1.95387541865583	2.81033207185613\\
1.98998084771081	2.81852894203929\\
2.01994885409201	2.82335459591496\\
2.04525882008697	2.8257939325355\\
2.0669689142298	2.8265001166603\\
};
\addplot [color=mycolor1, forget plot]
  table[row sep=crcr]{%
2.01629649870323	2.8216947796992\\
2.03371192787167	2.82115372757937\\
2.04909758148215	2.8196937697169\\
2.06284386947368	2.81750145142994\\
2.07525232111241	2.81470461391784\\
2.08655918486678	2.81138915448681\\
2.09695197926938	2.80761003738521\\
2.10658128152195	2.80339853083658\\
2.11556923308719	2.79876688561593\\
2.12401573403612	2.79371120614847\\
2.13200297131505	2.78821297113196\\
2.13959870991857	2.78223946498994\\
2.14685862733202	2.77574324077276\\
2.15382786330377	2.76866062152778\\
2.16054187042183	2.76090914028469\\
2.16702657163593	2.75238370062266\\
2.17329774507153	2.74295109035814\\
2.17935944791232	2.73244227361185\\
2.1852011364565	2.72064158145531\\
2.1907929012884	2.7072714544067\\
2.19607785082874	2.69197065410794\\
2.20096002912548	2.67426267219099\\
2.20528513440281	2.65350909646456\\
2.20880931586443	2.62883936170034\\
2.21114769507717	2.59904253518429\\
2.2116874471933	2.56239654634731\\
2.20943715928125	2.51639175507511\\
2.20275831617271	2.45727185066445\\
2.188872968427	2.37925324006683\\
2.1629384810545	2.27317642871619\\
2.11628700004405	2.1241865211064\\
2.03314705487453	1.90797639336758\\
1.88522789170033	1.58603353217718\\
1.62647712510769	1.10567006058556\\
1.20337195083471	0.426430541886378\\
0.611915909614419	-0.402858236553646\\
-0.0377639343599761	-1.20354789310859\\
-0.597100482331314	-1.81117113185058\\
-1.00757427958943	-2.20428707001593\\
-1.29021696967227	-2.44256084179465\\
-1.48412953405143	-2.58599032846721\\
-1.62023730138053	-2.67384690614635\\
-1.71879227503168	-2.72891443587442\\
-1.79245059734471	-2.76412199699261\\
-1.84914108871065	-2.78691245609935\\
-1.89393410940838	-2.80168792222779\\
-1.93015885351913	-2.81113148167195\\
-1.96006170027521	-2.81692666023866\\
-1.98519848830143	-2.82015798991299\\
-2.00667318546355	-2.82154029086012\\
-2.02528665005883	-2.82155369216577\\
-2.0416316546784	-2.82052532925747\\
-2.05615503547555	-2.81868002036959\\
-2.06919925101121	-2.81617239719858\\
-2.08103074658255	-2.81310765133649\\
-2.09185967675809	-2.80955510860833\\
-2.10185384863354	-2.80555716250382\\
-2.11114872107	-2.80113511420794\\
-2.11985465684564	-2.79629287490345\\
-2.12806221881559	-2.79101911843369\\
-2.13584603650612	-2.78528823438026\\
-2.14326759134354	-2.77906026763665\\
-2.15037714299369	-2.77227990629132\\
-2.15721492382378	-2.76487447169454\\
-2.16381164722527	-2.75675075432964\\
-2.17018829476462	-2.74779040808994\\
-2.17635505229918	-2.73784344117483\\
-2.18230913715818	-2.72671909182738\\
-2.1880310680229	-2.7141730012679\\
-2.19347862762006	-2.69988901191023\\
-2.19857727062877	-2.68345298618764\\
-2.20320488089757	-2.66431451591879\\
-2.20716729498335	-2.64172983860735\\
-2.21015833063649	-2.61467490171091\\
-2.2116931024764	-2.58170984489889\\
-2.21099398493157	-2.54076243038014\\
-2.20679020984049	-2.48877291954632\\
-2.19695552159172	-2.42109700701258\\
-2.17783500661387	-2.33048097653156\\
-2.142969037832	-2.20528818446957\\
-2.08067540438981	-2.02650626200969\\
-1.96972237526817	-1.76327837650438\\
-1.77324535376412	-1.36912188000483\\
-1.43794190280115	-0.791111827159751\\
-0.924545711399306	-0.0216540259654149\\
-0.283953095218948	0.819957185386306\\
0.335043173914783	1.53607343206824\\
0.820645184263823	2.03143236592395\\
1.16236468113286	2.33857029213583\\
1.39605637439492	2.52320245266567\\
1.55792497526828	2.63514208823294\\
1.67327854150371	2.70450306114709\\
1.75815357520141	2.74844279268488\\
1.82254887470734	2.77674235019102\\
1.87278527967306	2.79510456930404\\
1.91295729365938	2.80695333270049\\
1.94579044293073	2.81440658009161\\
1.97314833573224	2.81881138097168\\
1.99633775367937	2.82104590983873\\
2.01629649870323	2.8216947796992\\
};
\addplot [color=mycolor1, forget plot]
  table[row sep=crcr]{%
1.95408713286161	2.81255877444144\\
1.97015665108589	2.81205922536446\\
1.98440383011239	2.81070703591457\\
1.99717590274892	2.80866986239339\\
2.00874206890337	2.80606267207798\\
2.01931402447436	2.80296252225127\\
2.02906040923129	2.79941828627224\\
2.03811713116402	2.79545704113902\\
2.04659483732625	2.79108817309935\\
2.05458436756647	2.78630585336903\\
2.06216074721035	2.78109027757561\\
2.06938608809292	2.77540788802964\\
2.07631163782244	2.76921066958993\\
2.08297912134429	2.76243450284066\\
2.08942144040516	2.75499645363844\\
2.09566272220139	2.74679075818894\\
2.10171762517856	2.73768310713825\\
2.10758970122228	2.72750261231144\\
2.11326845507579	2.71603051211328\\
2.1187244940024	2.70298416525528\\
2.12390175446257	2.68799407744965\\
2.12870510331387	2.67057039352651\\
2.13298040690076	2.65005309663722\\
2.13648199862587	2.62553640960846\\
2.13881848174384	2.5957513300483\\
2.13936021939874	2.55887846075589\\
2.13707705910718	2.51224175080742\\
2.13024522821122	2.4517937944577\\
2.11590219113997	2.37122954964349\\
2.08880702511242	2.26043600118698\\
2.03943638612533	2.10280117567733\\
1.95023384582017	1.87087470725969\\
1.78957421552096	1.52125205780096\\
1.50724163123535	0.997097348433753\\
1.05218124740212	0.266322314398444\\
0.442891559848105	-0.58852021357033\\
-0.185833290433792	-1.36392590129485\\
-0.698009436945026	-1.92057888359518\\
-1.06198920354467	-2.26924086719763\\
-1.30958905743568	-2.47798367576543\\
-1.4792957117902	-2.60350567562873\\
-1.59889578491176	-2.68070188805705\\
-1.68600931275967	-2.72937275373668\\
-1.75152908476803	-2.76068754710666\\
-1.80226698636391	-2.78108311856666\\
-1.84258863188146	-2.79438234601673\\
-1.87537167226418	-2.80292773170737\\
-1.90256666928812	-2.80819742761536\\
-1.92553095322346	-2.81114894635193\\
-1.94523210358837	-2.81241666419805\\
-1.96237516540708	-2.81242866498418\\
-1.97748420843269	-2.81147777919341\\
-1.99095587711757	-2.80976585505906\\
-2.00309533870785	-2.80743194832296\\
-2.01414091135235	-2.80457058999968\\
-2.02428125088093	-2.80124376919405\\
-2.03366754272059	-2.79748882437222\\
-2.0424222723539	-2.79332358755083\\
-2.05064560332622	-2.78874961214502\\
-2.05842004429573	-2.78375399369517\\
-2.06581385882923	-2.77831008210545\\
-2.07288351712786	-2.77237723631783\\
-2.07967537858105	-2.76589965726303\\
-2.08622670867571	-2.75880423126099\\
-2.09256605897104	-2.75099720604875\\
-2.09871296179467	-2.7423593867516\\
-2.10467679770348	-2.73273935563581\\
-2.1104545639566	-2.72194395253888\\
-2.11602707550614	-2.70972484716044\\
-2.12135281425098	-2.69575939801092\\
-2.12635811517624	-2.67962296813396\\
-2.1309214701633	-2.66074817727884\\
-2.13484812202904	-2.63836371385647\\
-2.13782819258103	-2.61140038594983\\
-2.13936610387029	-2.57834332604201\\
-2.13865849025632	-2.53699337538637\\
-2.13437691750635	-2.48407135459853\\
-2.12426956517293	-2.41454448693609\\
-2.10441034382447	-2.32045540155919\\
-2.06775511912624	-2.18887241749459\\
-2.00137768373468	-1.99841461647744\\
-1.8815491498074	-1.71418308494054\\
-1.6673253143189	-1.28445643542233\\
-1.30292284263472	-0.656185427944006\\
-0.760561802985589	0.157087992584297\\
-0.120400018076629	0.99873582485722\\
0.46087903608505	1.67162366241341\\
0.897137918336744	2.11679559453497\\
1.19773552170667	2.38700539604759\\
1.40216485547805	2.54851917659725\\
1.54406119305533	2.64664238340825\\
1.64571156583332	2.70775941662276\\
1.72097033471938	2.74671763256643\\
1.77842930535205	2.7719666864416\\
1.82352311167654	2.78844764991827\\
1.85978356502206	2.79914158907219\\
1.88957179553683	2.80590281834557\\
1.91450981240302	2.80991739269252\\
1.93574032729193	2.81196268811317\\
1.95408713286161	2.81255877444144\\
};
\addplot [color=mycolor1, forget plot]
  table[row sep=crcr]{%
1.87457785195785	2.79663504639062\\
1.88945545218441	2.79617219509606\\
1.90270374359799	2.79491450953976\\
1.91462952242024	2.79301206176028\\
1.92547182633201	2.79056780972743\\
1.93541953820537	2.78765050384051\\
1.94462383638467	2.78430319829366\\
1.95320713164883	2.78054883237584\\
1.96126955988095	2.77639378663779\\
1.96889373707553	2.7718299711776\\
1.97614824742985	2.76683577821289\\
1.98309017679418	2.76137607604001\\
1.98976689233669	2.7554013038897\\
1.99621718508488	2.74884562490728\\
2.0024718207702	2.74162398977357\\
2.00855347402061	2.73362783867479\\
2.01447593856784	2.7247190026417\\
2.02024239483717	2.71472112543319\\
2.02584235012124	2.70340756507238\\
2.03124660207401	2.69048416881407\\
2.03639913691442	2.67556440825667\\
2.04120411953491	2.65813286935502\\
2.04550480009076	2.63749057763998\\
2.04904874009366	2.61267129583654\\
2.05142923430248	2.58231024186075\\
2.05198409645007	2.54443273360228\\
2.04961575014777	2.49610446347231\\
2.04246165430275	2.4328367742712\\
2.02727238009869	2.3475506408997\\
1.99820950966423	2.22874757739577\\
1.9445086891535	2.05733205488968\\
1.84614113374679	1.80162784832694\\
1.66730383173018	1.41247595138888\\
1.35437802897999	0.831416506907343\\
0.866004494455484	0.0466867728527042\\
0.253507304421055	-0.813411613844761\\
-0.332010221993016	-1.53608422884368\\
-0.783291433215933	-2.02675635656377\\
-1.0960575116577	-2.32639806983683\\
-1.30784508177712	-2.50494509822122\\
-1.45374421032671	-2.612849065112\\
-1.55745382918991	-2.67978157783237\\
-1.63370370471678	-2.7223779641566\\
-1.6915732725475	-2.75003312998556\\
-1.73676242378907	-2.76819602222803\\
-1.77294782937584	-2.78012947651889\\
-1.80257071204522	-2.78785003090417\\
-1.82729792907952	-2.79264072957431\\
-1.84829752840052	-2.79533912096034\\
-1.8664076462336	-2.79650397780766\\
-1.8822427616688	-2.79651467256384\\
-1.89626224752837	-2.79563203275335\\
-1.90881565562071	-2.79403651912157\\
-1.92017329436437	-2.79185267951468\\
-1.93054729272853	-2.78916507242008\\
-1.94010637508435	-2.78602874578762\\
-1.94898639167708	-2.78247614163633\\
-1.9572979261832	-2.7785215770886\\
-1.9651318482346	-2.7741640131358\\
-1.9725633874208	-2.76938854442878\\
-1.97965511282255	-2.76416685834055\\
-1.98645907017771	-2.75845677846827\\
-1.99301823301222	-2.75220090018766\\
-1.99936734793394	-2.74532422449441\\
-2.00553318496923	-2.73773058396824\\
-2.0115341292134	-2.72929751211378\\
-2.01737895576195	-2.71986900884221\\
-2.02306449523449	-2.70924536160379\\
-2.02857168889946	-2.69716873091932\\
-2.03385919308024	-2.6833024952447\\
-2.03885311904793	-2.66720119032324\\
-2.04343049559734	-2.64826594721389\\
-2.04739225140634	-2.62567703875512\\
-2.05041821524351	-2.5982893830741\\
-2.05199037599498	-2.56446652853883\\
-2.05125843748935	-2.52180971963489\\
-2.04679724268251	-2.46670334973505\\
-2.03615563241697	-2.39353204771595\\
-2.01499378005436	-2.29330477661769\\
-1.97540492246692	-2.15123132676112\\
-1.90269410937579	-1.94265037912345\\
-1.76980293844462	-1.62748139356961\\
-1.53124519946957	-1.14892440398809\\
-1.13209359195833	-0.460484046162954\\
-0.566808533969557	0.387816394586287\\
0.0520366019978877	1.20215165598319\\
0.576535381829118	1.80967253888786\\
0.954864286568924	2.19582800001589\\
1.21211636398751	2.42708312606015\\
1.38729554854803	2.56547899463559\\
1.50978522747304	2.65017391625239\\
1.59834333757868	2.70341329806673\\
1.66452000788624	2.73766611350783\\
1.71548697704509	2.7600597349861\\
1.7558056309989	2.77479364434049\\
1.7884611523011	2.78442313328132\\
1.81546395096188	2.7905512025093\\
1.83820504074484	2.794211411217\\
1.8576710696949	2.7960861786337\\
1.87457785195785	2.79663504639062\\
};
\addplot [color=mycolor1, forget plot]
  table[row sep=crcr]{%
1.76784387513612	2.76928910145383\\
1.78174633759923	2.76885615522374\\
1.79419693541599	2.76767383044685\\
1.80546484911209	2.76587600985613\\
1.81576126131739	2.76355453796707\\
1.82525415931308	2.76077035293493\\
1.83407887909368	2.75756084426591\\
1.84234572307335	2.75394466141325\\
1.85014552584246	2.74992473191324\\
1.85755374913669	2.7454899539701\\
1.8646334942403	2.74061583435693\\
1.87143768860931	2.73526420470525\\
1.87801060916971	2.72938204000661\\
1.88438883113162	2.72289930259853\\
1.89060162587557	2.7157256259312\\
1.89667076319854	2.70774551650771\\
1.90260958893445	2.69881156524049\\
1.90842113042871	2.68873488479607\\
1.91409480045653	2.67727156911894\\
1.91960097508577	2.66410330775859\\
1.92488222337915	2.6488092124617\\
1.92983910128206	2.63082412809549\\
1.9343068731227	2.60937566180876\\
1.93801667340984	2.58338686500575\\
1.94052922309179	2.55132202917289\\
1.94111869405914	2.51093572559874\\
1.93856325046462	2.45885292681666\\
1.93075569678326	2.38984749944438\\
1.91395891517441	2.29557553517638\\
1.88135412086077	2.16233871312354\\
1.82023064403477	1.96727962024497\\
1.70698513939229	1.67293893640523\\
1.5009626367681	1.22458307939976\\
1.14920561952737	0.571075287988628\\
0.634671529376524	-0.256529150340702\\
0.0487615290360517	-1.08023744885568\\
-0.464160373241134	-1.71379205799029\\
-0.840794541274635	-2.12341792386813\\
-1.09857110524551	-2.37037717746715\\
-1.27425441166773	-2.51846834010297\\
-1.39693091863683	-2.60918252109282\\
-1.48545234652581	-2.66630303200044\\
-1.55147603147699	-2.70318039482592\\
-1.60224137064384	-2.72743650225245\\
-1.64234632461511	-2.74355321834049\\
-1.67479425152706	-2.75425225602386\\
-1.70160378481403	-2.76123825526265\\
-1.72416901355081	-2.76560911350238\\
-1.74347710347502	-2.76808940820817\\
-1.76024319539013	-2.76916723159805\\
-1.77499620104009	-2.7691767210339\\
-1.78813472615172	-2.76834914861335\\
-1.79996436786742	-2.76684528428748\\
-1.8107231218077	-2.76477629836321\\
-1.82059902686968	-2.76221746239001\\
-1.82974263443093	-2.75921720197294\\
-1.83827595617522	-2.75580306113597\\
-1.84629896779417	-2.75198554221772\\
-1.85389438056917	-2.7477604169714\\
-1.86113115584889	-2.7431098675017\\
-1.8680670790099	-2.73800265396649\\
-1.87475059904002	-2.73239338527917\\
-1.88122205763355	-2.72622086635664\\
-1.88751436365687	-2.71940539282717\\
-1.89365310351775	-2.71184474423268\\
-1.89965600355715	-2.70340846884428\\
-1.90553156197023	-2.69392982790673\\
-1.91127652217843	-2.68319442880026\\
-1.91687162919518	-2.67092405027451\\
-1.92227472893751	-2.6567533208856\\
-1.92740961653825	-2.64019553048539\\
-1.93214788588446	-2.62059153274098\\
-1.93627893869243	-2.59703169657767\\
-1.93945940251928	-2.56823380041851\\
-1.9411256975553	-2.53234698407273\\
-1.94033865257595	-2.48662825826946\\
-1.93549900187693	-2.42689383328213\\
-1.92381069939923	-2.34656509677893\\
-1.90024268269791	-2.23498378595543\\
-1.85550126153486	-2.07446712487143\\
-1.77220791794184	-1.83557709268416\\
-1.61884814998775	-1.47187517105885\\
-1.34621301713298	-0.924796279998125\\
-0.909481515083475	-0.170949326882521\\
-0.340914514642232	0.683260693916561\\
0.223422225044149	1.42661131604494\\
0.669486332528397	1.9435442286894\\
0.982190677905218	2.26275433782589\\
1.19459131176185	2.45367579279806\\
1.34083225948148	2.56919246942168\\
1.44460153384385	2.6409310904603\\
1.52074563315668	2.686699595453\\
1.57843171408775	2.71655270800445\\
1.6234100295506	2.73631178759889\\
1.65938328561586	2.74945551357152\\
1.68880511280075	2.75812989491714\\
1.71334758481246	2.76369848551413\\
1.73418031619097	2.76705069509871\\
1.75214124503063	2.76877985050119\\
1.76784387513612	2.76928910145383\\
};
\addplot [color=mycolor1, forget plot]
  table[row sep=crcr]{%
1.61613855159024	2.72141247921817\\
1.62941391966627	2.72099848689358\\
1.64139747806358	2.71986002293162\\
1.65232379011111	2.71811627703685\\
1.66237878254442	2.71584885191853\\
1.67171183356178	2.71311119457135\\
1.68044447792192	2.70993483944171\\
1.68867675842975	2.70633345329975\\
1.69649190694504	2.70230529579381\\
1.70395981265679	2.69783446704066\\
1.71113958433924	2.69289114772267\\
1.71808140792551	2.68743091323758\\
1.72482782235928	2.68139309808504\\
1.73141447211586	2.67469808160963\\
1.73787033314867	2.66724324384313\\
1.74421733919648	2.65889717938977\\
1.75046924315976	2.64949152834833\\
1.75662941219957	2.63880944005132\\
1.76268703975575	2.62656915235071\\
1.76861090034744	2.61240031796586\\
1.77433915991991	2.59580931610619\\
1.77976267171791	2.57612745194278\\
1.78469722220312	2.55243193700056\\
1.78883652472459	2.5234224961629\\
1.79167072701979	2.48722377925767\\
1.79234135213091	2.44106055482288\\
1.789375703746	2.38070977151951\\
1.78018693291654	2.29955531830014\\
1.76011199813813	2.1869387238229\\
1.72055198175909	2.02533260345223\\
1.64553280840373	1.7859655735198\\
1.50647561611139	1.42449133190946\\
1.25979031965253	0.887322769791866\\
0.867422530010792	0.157466147395609\\
0.358701421246773	-0.662129182426281\\
-0.149473375810737	-1.37749528953538\\
-0.557458530087428	-1.88173099747749\\
-0.848618942779954	-2.19841186792069\\
-1.04949403742542	-2.39082010930412\\
-1.18952358701084	-2.50882522014053\\
-1.28985205145128	-2.58299321675379\\
-1.36403888976222	-2.63085119744051\\
-1.42059441277052	-2.66243215945022\\
-1.46492211832912	-2.68360718214946\\
-1.50053416058839	-2.69791494306816\\
-1.52977527810634	-2.70755423580167\\
-1.55425307769521	-2.71393093611635\\
-1.57509782069847	-2.71796727543747\\
-1.5931228018649	-2.72028177271289\\
-1.60892578158929	-2.72129691153713\\
-1.62295469491897	-2.72130530601542\\
-1.63555120422068	-2.72051134533719\\
-1.64698019083996	-2.71905795828844\\
-1.65745012410178	-2.71704410986192\\
-1.66712738619165	-2.71453637411781\\
-1.67614651188316	-2.71157661673854\\
-1.68461761334463	-2.70818704280787\\
-1.69263182725761	-2.70437339026219\\
-1.70026534302248	-2.70012674922752\\
-1.70758238708614	-2.69542428859632\\
-1.71463741304862	-2.69022902975082\\
-1.72147665723247	-2.68448869518187\\
-1.72813914936835	-2.67813355668702\\
-1.73465720626276	-2.67107309642847\\
-1.74105637209399	-2.66319115557991\\
-1.74735469006647	-2.65433905492944\\
-1.75356107936389	-2.64432589293457\\
-1.7596724213162	-2.6329048006232\\
-1.76566868254957	-2.61975326128974\\
-1.7715049366081	-2.60444451716606\\
-1.77709833358423	-2.58640528663938\\
-1.78230661343683	-2.56485196448675\\
-1.78689208331813	-2.5386921773767\\
-1.79045991879808	-2.50636914342008\\
-1.79234981279428	-2.46560917445441\\
-1.79144039174069	-2.41300112137448\\
-1.78578601903851	-2.34327839389174\\
-1.77192472130788	-2.24807066239538\\
-1.74353879464788	-2.11373299263696\\
-1.68889795825026	-1.91774897557392\\
-1.58643016183899	-1.6238749767559\\
-1.39964365988511	-1.18075199110278\\
-1.08242507481589	-0.543631661955198\\
-0.621555837314145	0.25307854518944\\
-0.09641245414283	1.04328968947428\\
0.36874050269807	1.65658250176025\\
0.716347118554585	2.05952788811988\\
0.958339109629879	2.30652938594418\\
1.12558565639423	2.45682608727719\\
1.2436561905534	2.55006469283883\\
1.32958975333439	2.60945667416062\\
1.39412727092238	2.64823864447461\\
1.44403321655362	2.67405908965072\\
1.48364979593202	2.69145857862251\\
1.51583699422256	2.70321614386247\\
1.54253000575099	2.71108399841515\\
1.56507272792322	2.71619739538668\\
1.58442127374257	2.71930968084827\\
1.60127112895018	2.72093100837461\\
1.61613855159024	2.72141247921817\\
};
\addplot [color=mycolor1, forget plot]
  table[row sep=crcr]{%
1.38732790552402	2.63465819500296\\
1.40062943147086	2.6342425157038\\
1.41278139958255	2.63308730039004\\
1.42398694382011	2.63129832534375\\
1.43440982960286	2.62894733458447\\
1.44418383730999	2.62607976906692\\
1.45341964378235	2.62271986095831\\
1.4622099269933	2.61887384173129\\
1.47063318375156	2.61453172508766\\
1.47875659248889	2.60966793076692\\
1.48663814362397	2.60424087250832\\
1.49432817971773	2.59819151597249\\
1.5018704234068	2.5914408000625\\
1.50930251238285	2.58388568931988\\
1.51665599722341	2.57539346488786\\
1.52395567717654	2.56579363848176\\
1.53121803284551	2.55486654492852\\
1.53844833348756	2.54232716514553\\
1.54563569904242	2.52780193788258\\
1.55274489072743	2.51079503816241\\
1.55970271801618	2.49063848845685\\
1.56637535903263	2.46641691101483\\
1.57252996400346	2.43685161615735\\
1.57776839447886	2.40011803141076\\
1.5814103256968	2.35355156732736\\
1.58228214227961	2.29316361847308\\
1.57832723800457	2.21283228039354\\
1.5658755767825	2.10294548415539\\
1.53827929241975	1.94819438992214\\
1.48349357885858	1.72440073080168\\
1.38055319409598	1.39581707532157\\
1.19781585501386	0.920306801172927\\
0.904628098562938	0.280688803757821\\
0.509489430092227	-0.456114835477672\\
0.0855409875837257	-1.14058062935781\\
-0.283257072849337	-1.66029752573636\\
-0.564207735286064	-2.00757771839611\\
-0.766873947036603	-2.22793675491897\\
-0.912235725107974	-2.36710609782894\\
-1.01829550715567	-2.45644119828739\\
-1.09766249484591	-2.51508683724071\\
-1.15866437634191	-2.55442282650588\\
-1.20675941635745	-2.58126903813666\\
-1.24557031288401	-2.59980192357757\\
-1.27755060032258	-2.61264593418094\\
-1.30439982093182	-2.62149347003746\\
-1.32732197226194	-2.62746251368772\\
-1.34718845519265	-2.63130761041243\\
-1.36464270165228	-2.63354740438033\\
-1.38016871998496	-2.63454360949568\\
-1.3941369208368	-2.63455102054594\\
-1.40683536048265	-2.63374982835509\\
-1.41849144100645	-2.63226685697424\\
-1.42928724575547	-2.63018969575471\\
-1.43937055152318	-2.62757615751248\\
-1.44886285155347	-2.62446057350835\\
-1.45786527494397	-2.62085787180965\\
-1.46646299787239	-2.61676602896612\\
-1.47472855005451	-2.61216724990523\\
-1.48272428903746	-2.60702806608776\\
-1.4905042219424	-2.60129841465141\\
-1.49811528344374	-2.59490964867501\\
-1.50559811854367	-2.58777131199924\\
-1.51298735896656	-2.57976637202622\\
-1.52031131173482	-2.57074441656157\\
-1.52759088287588	-2.56051205144964\\
-1.53483741525638	-2.54881933036228\\
-1.54204888849166	-2.53534041788314\\
-1.54920354220679	-2.51964568188709\\
-1.55624931637771	-2.50116077122663\\
-1.56308631896982	-2.47910550075426\\
-1.56953737967292	-2.45240071311581\\
-1.57529774141912	-2.41952321802537\\
-1.57984730402533	-2.37827470462384\\
-1.58229399048494	-2.32540535526911\\
-1.58108764032128	-2.25598889593011\\
-1.5734870898689	-2.16237360962026\\
-1.55455945900639	-2.03243925699405\\
-1.51534183875637	-1.84688071019744\\
-1.43979552430412	-1.57587327333968\\
-1.30141382543751	-1.17872959520849\\
-1.06600085542706	-0.619443701596676\\
-0.716526411785143	0.0840243813567158\\
-0.295063910234367	0.814367939190121\\
0.109186222039112	1.42365553711546\\
0.434684938257419	1.85305282279251\\
0.67402696988451	2.1304579326942\\
0.845463387434966	2.30536939321677\\
0.969271852192695	2.41657625809892\\
1.06071279964974	2.48875193121301\\
1.13006486318988	2.53666300667769\\
1.1840651285775	2.56909994731102\\
1.22715045655147	2.59138317239276\\
1.26229400270659	2.60681253448976\\
1.29153199480616	2.61748891908155\\
1.31629096703128	2.62478392175053\\
1.33759255451442	2.62961371435005\\
1.35618374488957	2.63260257864501\\
1.3726213487659	2.63418297028529\\
1.38732790552402	2.63465819500296\\
};
\addplot [color=mycolor1, forget plot]
  table[row sep=crcr]{%
1.02739733741893	2.47348557476368\\
1.04219991052851	2.47302138058375\\
1.0559950337039	2.47170853917827\\
1.06895728691558	2.46963781625163\\
1.08123207453146	2.46686791514125\\
1.09294215182295	2.46343119833349\\
1.10419258201004	2.45933728611963\\
1.11507453640509	2.45457500278217\\
1.12566822414637	2.44911293646631\\
1.13604514731848	2.44289872233356\\
1.14626980813599	2.43585702324449\\
1.15640093689847	2.42788604669408\\
1.16649225345643	2.41885228009767\\
1.17659271108451	2.40858292391278\\
1.18674608692328	2.39685521926411\\
1.19698965773281	2.38338145237328\\
1.20735150090413	2.36778779134156\\
1.21784563190738	2.34958414090792\\
1.2284636298166	2.32812067278729\\
1.2391604290419	2.30252424743977\\
1.24983023012145	2.2716040012872\\
1.26026537740859	2.23370898637197\\
1.2700854025811	2.18651048837497\\
1.27861313515567	2.12666571765968\\
1.28465631246388	2.04929712833414\\
1.286121874765	1.94719876205682\\
1.2793458872528	1.80969238589744\\
1.25799632015502	1.62123209077316\\
1.21156045628604	1.36058221253479\\
1.12428293704218	1.00343260412228\\
0.977774300260783	0.534507895522493\\
0.762495229105826	-0.0276962808078741\\
0.494580699093191	-0.614415920687728\\
0.21512444262883	-1.13704908249558\\
-0.0363748816684043	-1.54366353098258\\
-0.242170781785461	-1.83370229251349\\
-0.403006881609096	-2.03239387455115\\
-0.5271200897453	-2.16723258118143\\
-0.623437344275255	-2.25937028699492\\
-0.699259962306375	-2.32318702120188\\
-0.760009693129276	-2.36804447342719\\
-0.809580221423752	-2.39998858735189\\
-0.850750354948673	-2.42295559567968\\
-0.885515681979411	-2.43954711168527\\
-0.915326660655649	-2.45151305628743\\
-0.941252924563969	-2.4600514134627\\
-0.96409544920585	-2.46599588098061\\
-0.984463303547178	-2.46993500067531\\
-1.0028266286991	-2.47228901446572\\
-1.0195536418362	-2.47336025691699\\
-1.03493682563864	-2.473366694953\\
-1.0492117187949	-2.47246453206161\\
-1.06257058284439	-2.47076357138599\\
-1.07517247546841	-2.46833767408633\\
-1.08715077022958	-2.46523180445155\\
-1.09861883702821	-2.46146661564033\\
-1.10967437803078	-2.4570411775543\\
-1.1204027631107	-2.4519342071522\\
-1.13087960249329	-2.44610398495604\\
-1.1411727158489	-2.43948699841739\\
-1.15134359471152	-2.43199521991977\\
-1.16144839927243	-2.42351178387252\\
-1.17153847205639	-2.41388465128115\\
-1.18166027864401	-2.4029176125877\\
-1.19185458362132	-2.3903576388808\\
-1.20215451307336	-2.37587708363649\\
-1.21258190036276	-2.35904845940414\\
-1.22314088434211	-2.33930829887762\\
-1.23380699293633	-2.31590468120806\\
-1.24450865180094	-2.28781990495046\\
-1.25509574716695	-2.25365477084764\\
-1.2652856878568	-2.2114528242258\\
-1.27456978008644	-2.1584300336945\\
-1.28204888502655	-2.09055610856803\\
-1.28614302408898	-2.00190929011678\\
-1.28408101466374	-1.88371283442261\\
-1.27103285053692	-1.72302784374413\\
-1.238776102119	-1.50145517093768\\
-1.17419465210873	-1.19547538391306\\
-1.05944501671691	-0.782915478248143\\
-0.878472672005636	-0.26187964957388\\
-0.632896379370723	0.323789932064092\\
-0.353427167312457	0.888326759231871\\
-0.0842104180387018	1.35587313105521\\
0.145234985297924	1.70192712671374\\
0.327761757720484	1.94263941038087\\
0.469069566250933	2.10629884367719\\
0.578250597443543	2.21759988385927\\
0.663530707415728	2.29413829132118\\
0.731244886631844	2.34754665688213\\
0.785996928136564	2.38534607007067\\
0.831075464460832	2.41240712373885\\
0.868832417770454	2.43192315659928\\
0.900966209411553	2.44602308723871\\
0.928720063416563	2.45615169171875\\
0.953017804165661	2.46330641382896\\
0.974556567004055	2.46818659097927\\
0.99387047974921	2.47128894435639\\
1.01137486888268	2.47296968927403\\
1.02739733741893	2.47348557476368\\
};
\addplot [color=mycolor1, forget plot]
  table[row sep=crcr]{%
0.47620713588277	2.18437004943608\\
0.496625391790089	2.18372579956675\\
0.516344844398201	2.18184549881844\\
0.535514439991744	2.17877969559682\\
0.554268779286526	2.17454430135902\\
0.5727315313037	2.16912250662038\\
0.591018288661362	2.16246485952589\\
0.60923897433689	2.15448754932154\\
0.627499870189142	2.14506878020796\\
0.645905296015645	2.13404295309354\\
0.664558918125901	2.12119217378726\\
0.683564598818918	2.10623435253214\\
0.703026598024824	2.0888068213686\\
0.723048782054685	2.06844393344197\\
0.743732243240506	2.0445464723026\\
0.76517032472752	2.01633983223327\\
0.787439373166844	1.98281678410573\\
0.810582443206192	1.94265922636535\\
0.834581400690629	1.89413184551802\\
0.859310076920423	1.83493981906352\\
0.884456983623625	1.76204465156248\\
0.909400721240225	1.6714420062521\\
0.933016540050802	1.55793417073609\\
0.953395887840482	1.41499903073798\\
0.967492159003808	1.23499780370961\\
0.970805297122381	1.01018388457675\\
0.957426852106025	0.735142385460073\\
0.921010892175009	0.410912963375699\\
0.857092982242873	0.0493880011383427\\
0.766035925310678	-0.325839751088052\\
0.654323155910683	-0.6855301599157\\
0.53246034017998	-1.00515751171179\\
0.410885578518104	-1.27203135999729\\
0.296971860217316	-1.48520273938611\\
0.194415318420365	-1.65092506949056\\
0.104138541697818	-1.77801892895423\\
0.0254860864408128	-1.87506077908946\\
-0.0428805888734549	-1.94924200381424\\
-0.102465461955605	-2.00617381507678\\
-0.154689883582295	-2.05008085854172\\
-0.200798345906226	-2.08409273935246\\
-0.241840318683732	-2.11051574146561\\
-0.278684796607555	-2.13105091112399\\
-0.312045929833457	-2.14695799135387\\
-0.342509935822128	-2.15917489270236\\
-0.370559393274054	-2.16840355335845\\
-0.396593767743847	-2.17517133861407\\
-0.420946197254415	-2.17987493192\\
-0.443897002523521	-2.18281175264095\\
-0.465684490818418	-2.18420246095624\\
-0.486513590890022	-2.18420703600205\\
-0.50656277870641	-2.18293614879374\\
-0.525989668611247	-2.18045901049187\\
-0.544935567027176	-2.17680849067284\\
-0.56352922005612	-2.17198401849787\\
-0.581889931748121	-2.16595256483222\\
-0.600130183878707	-2.15864782773647\\
-0.618357847292737	-2.14996758545978\\
-0.636678034922541	-2.13976902101053\\
-0.655194601897781	-2.1278616415685\\
-0.674011240920889	-2.11399719299014\\
-0.693232039511397	-2.09785567790774\\
-0.71296124121597	-2.07902619108133\\
-0.733301755241592	-2.05698074329029\\
-0.754351638642492	-2.0310385007615\\
-0.776197250911278	-2.00031686454427\\
-0.79890092114673	-1.96366452461323\\
-0.822479567427569	-1.91957013046458\\
-0.846868468608206	-1.86603893716692\\
-0.871860950673296	-1.80042997719165\\
-0.89700991179925	-1.71925125011644\\
-0.921471626576344	-1.61792774868017\\
-0.943770128972071	-1.49060270372679\\
-0.961474196303445	-1.33013337698189\\
-0.970838148433354	-1.12862641091917\\
-0.966609375019177	-0.879085111628614\\
-0.942457436254818	-0.578726314182426\\
-0.892610136125284	-0.233577186077621\\
-0.814687491133952	0.138325729779689\\
-0.712153021676426	0.509387866614885\\
-0.593959795072005	0.851454979769253\\
-0.471086238914411	1.14547925137416\\
-0.352660605060765	1.38504616933722\\
-0.244166059846886	1.57346115459866\\
-0.147758427943964	1.7187390553571\\
-0.0634346279898705	1.82980920482635\\
0.009888947335459	1.91462255823226\\
0.0736783576399879	1.97957090125034\\
0.129415354679625	2.02953758502394\\
0.178438527373515	2.06816309229685\\
0.221894380239587	2.09813451538813\\
0.260739303706182	2.12143181788064\\
0.295761446410512	2.1395179456237\\
0.327607743123823	2.15347913236043\\
0.356809782611324	2.16412628696682\\
0.38380626313929	2.1720676188883\\
0.408961606107032	2.17776054426846\\
0.432581034590971	2.18154881346394\\
0.454922659440202	2.18368910282942\\
0.47620713588277	2.18437004943608\\
};
\addplot [color=mycolor1, forget plot]
  table[row sep=crcr]{%
-0.218432368692388	1.75006885244458\\
-0.178478874550534	1.74879469991791\\
-0.137402821732203	1.74486457968917\\
-0.0950083444367945	1.73807094169262\\
-0.0510875079760354	1.72813839612093\\
-0.00542085812119394	1.71471388417246\\
0.0422207735712061	1.69735461930125\\
0.0920724954094568	1.67551384024463\\
0.144368606585523	1.64852472000514\\
0.199331491102878	1.61558333583957\\
0.257154326987754	1.57573256717545\\
0.317975337776623	1.52785033449011\\
0.38184103876284	1.47064791060494\\
0.448656175742909	1.40268720968177\\
0.518119426896689	1.32242971859341\\
0.589647322399636	1.22833301036837\\
0.662295171885714	1.11901109920361\\
0.734693343049372	0.993468019785329\\
0.805028151840547	0.851394813056192\\
0.871103239544788	0.69348622035243\\
0.93051018123024	0.521692030405634\\
0.98090791377875	0.33929151936184\\
1.02036266320661	0.150699125173134\\
1.04765596401809	-0.0390090651407171\\
1.06246077478911	-0.224743756910975\\
1.06532985788274	-0.402041008726533\\
1.05751428953611	-0.567514609618132\\
1.04068841502765	-0.719035767469351\\
1.01667106016879	-0.85566276912545\\
0.987206502897894	-0.97741184673942\\
0.953828795059489	-1.08497118276009\\
0.91780206834694	-1.17943256047259\\
0.880115114284985	-1.26207804852348\\
0.841507293520577	-1.33423012225647\\
0.802507999458383	-1.397157525876\\
0.763478381307864	-1.45202337733812\\
0.724649355479105	-1.49986213387365\\
0.686153487547168	-1.54157460811198\\
0.648050358305878	-1.57793325628698\\
0.610346013277956	-1.60959257302992\\
0.573007458842296	-1.63710138046797\\
0.535973205639494	-1.66091513805353\\
0.499160751098688	-1.68140726081827\\
0.462471736243642	-1.69887895800313\\
0.425795355670333	-1.71356740441454\\
0.389010464018177	-1.7256522148187\\
0.351986713401945	-1.7352602619678\\
0.314584973745266	-1.74246889659897\\
0.276657229172496	-1.7473076147454\\
0.238046105928604	-1.74975818716593\\
0.198584169077412	-1.74975322574803\\
0.158093126445626	-1.74717311794589\\
0.116383100910635	-1.74184121782058\\
0.0732521805527515	-1.73351714785357\\
0.0284865375923655	-1.72188805042904\\
-0.0181384683047321	-1.70655765052496\\
-0.0668556056456335	-1.68703308344085\\
-0.117900656747315	-1.66270965522525\\
-0.171503689684931	-1.63285412029728\\
-0.227875117489127	-1.596587801435\\
-0.287184486792264	-1.55287210664826\\
-0.349529547676552	-1.50050090893681\\
-0.414893083737573	-1.43810700479428\\
-0.483085696016494	-1.36419340466288\\
-0.553674989290907	-1.27720393252153\\
-0.625906368285954	-1.1756498278953\\
-0.69862865302886	-1.05830639631927\\
-0.770248394068362	-0.92448131101201\\
-0.838746521318586	-0.774329508164403\\
-0.901792002438366	-0.609150208681351\\
-0.956969634936358	-0.431563950437059\\
-1.00209899767848	-0.245460409394229\\
-1.03557164603545	-0.0556587079783228\\
-1.0566040611219	0.132673297390674\\
-1.06532245902061	0.314687447316354\\
-1.06265948272779	0.48641850018189\\
-1.05011439250815	0.645103748192245\\
-1.0294659048525	0.789229178833693\\
-1.00251785216747	0.918366425926476\\
-0.970921189464369	1.03290364240365\\
-0.936078507461518	1.13376138894612\\
-0.899114494127462	1.22214923861014\\
-0.860888801271196	1.29938442668872\\
-0.822030537946831	1.36677135618774\\
-0.78297990227181	1.42553035213702\\
-0.74402853215203	1.47676173626029\\
-0.705354597016382	1.52143297586337\\
-0.667051389562434	1.56037963029418\\
-0.629149625407273	1.59431370034004\\
-0.591634288114514	1.62383527509477\\
-0.554457028715936	1.64944500234484\\
-0.51754507661473	1.67155598997075\\
-0.48080747746677	1.69050442247485\\
-0.444139313285142	1.70655857697429\\
-0.407424412692465	1.71992614383045\\
-0.370536936812692	1.7307598657518\\
-0.333342130857583	1.73916154979285\\
-0.295696461168918	1.74518450695214\\
-0.257447309526412	1.74883445104043\\
-0.218432368692388	1.75006885244458\\
};
\addplot [color=mycolor1, forget plot]
  table[row sep=crcr]{%
-0.751180563789503	1.29301100092275\\
-0.644509272490707	1.2895621961483\\
-0.525853075685341	1.27816235586954\\
-0.394733067383687	1.25710640312682\\
-0.251255417385498	1.22462204940121\\
-0.0963657956622014	1.17906474936042\\
0.0679313764001297	1.11919457949777\\
0.238479475719274	1.04449792159326\\
0.411096645609142	0.95547160953947\\
0.580968788823447	0.85375980073978\\
0.743253360925444	0.742056725339432\\
0.893734219404683	0.623769327860435\\
1.02933068138786	0.502533729359883\\
1.14832654684568	0.381735283717928\\
1.25030593415255	0.264160768651649\\
1.33588484834454	0.151837028466942\\
1.40636299856342	0.046035698606773\\
1.46339657703116	-0.0526160638365817\\
1.50874510492206	-0.143981414805869\\
1.54410401747092	-0.228260871982706\\
1.57101084154411	-0.305863800112672\\
1.59080459914876	-0.37731143443009\\
1.60461898267407	-0.443169442531953\\
1.61339434135726	-0.50400430761343\\
1.6178983941152	-0.560357233657007\\
1.61874953068363	-0.612730111534175\\
1.61643932306818	-0.661579338697342\\
1.6113526202479	-0.707314476704708\\
1.60378462375957	-0.750299690229399\\
1.59395489259106	-0.790856617640623\\
1.58201848840448	-0.829267814089403\\
1.56807456975614	-0.865780234303465\\
1.55217275313418	-0.900608430316382\\
1.53431752429447	-0.933937265273067\\
1.51447093083552	-0.965924014217765\\
1.49255372986368	-0.996699753941291\\
1.4684451102011	-1.02636994778256\\
1.44198106109596	-1.05501411446045\\
1.41295142229282	-1.08268443616728\\
1.38109562819774	-1.10940311195514\\
1.34609715924291	-1.13515819874194\\
1.30757674909159	-1.15989760548665\\
1.26508448786515	-1.18352082038956\\
1.2180911421837	-1.20586786686862\\
1.16597933286764	-1.22670492454442\\
1.10803574356846	-1.24570606215447\\
1.04344637620195	-1.26243069409395\\
0.971298136519321	-1.27629683490271\\
0.890591822483024	-1.2865512138292\\
0.800273888148872	-1.29223914571393\\
0.699296856787975	-1.29218010554772\\
0.586720020892429	-1.28495946080335\\
0.461861094092619	-1.26895243972215\\
0.3245024552969	-1.24240144182672\\
0.175138600776365	-1.2035682010117\\
0.0152226545865632	-1.15097166888395\\
-0.152664711632565	-1.08369478588514\\
-0.324821834412255	-1.00170023256538\\
-0.496685757869856	-0.90605470141019\\
-0.663344832918517	-0.798955642596777\\
-0.820192630077424	-0.683508161867404\\
-0.963532782997699	-0.563297645942172\\
-1.09095770410823	-0.441888842741164\\
-1.20142500062305	-0.322399917701083\\
-1.29507612370551	-0.207246320149147\\
-1.37291296977355	-0.0980684877283813\\
-1.43645044441771	0.0042020468738688\\
-1.48742259418976	0.0992032593483852\\
-1.52757278250983	0.186984897840184\\
-1.55852539637732	0.267866024583987\\
-1.58172126356109	0.342322104544315\\
-1.59839612803496	0.410903378933661\\
-1.60958478267695	0.474180138513564\\
-1.61613840476429	0.532708621189078\\
-1.61874713092658	0.587011554073861\\
-1.61796325588701	0.637568493346236\\
-1.61422266111825	0.684812373734955\\
-1.60786343226419	0.729129764048409\\
-1.59914138437268	0.770863155573305\\
-1.58824260169102	0.810314202630113\\
-1.57529326680601	0.847747236816572\\
-1.56036710004137	0.883392638422727\\
-1.54349071326817	0.917449811620675\\
-1.52464713656047	0.950089605203132\\
-1.50377772010479	0.981456069358695\\
-1.4807825573749	1.0116674553813\\
-1.45551952420259	1.04081635809509\\
-1.42780198561346	1.06896887513935\\
-1.39739519197381	1.09616261560587\\
-1.3640113739163	1.12240333396511\\
-1.32730356145226	1.14765989469051\\
-1.28685821295664	1.17185719094692\\
-1.24218687070083	1.19486655389881\\
-1.19271730234616	1.21649311338865\\
-1.13778500336415	1.23645953866669\\
-1.07662660918946	1.2543856619263\\
-1.00837780643194	1.26976377842146\\
-0.932079852456276	1.28193010776828\\
-0.846700871404774	1.2900342666388\\
-0.751180563789503	1.29301100092275\\
};
\addplot [color=mycolor1, forget plot]
  table[row sep=crcr]{%
-0.792711621111321	0.98157953768228\\
-0.524039126534901	0.972855450713703\\
-0.223155287672921	0.943970299352618\\
0.0970845287357334	0.89265178624017\\
0.418361554453311	0.820112091604991\\
0.721568299281959	0.731202980446408\\
0.992069737039553	0.632937470787459\\
1.22249537797474	0.532313651163314\\
1.41221608098798	0.434728026924781\\
1.56495976330159	0.343484831498922\\
1.68635092136143	0.260092913835551\\
1.78224001120135	0.18483850219387\\
1.8578635205525	0.11730988759003\\
1.9175611884238	0.0567670347987813\\
1.96478181381882	0.00236466154353296\\
2.00220485764216	-0.0467286451778812\\
2.03188865709344	-0.0912751891546783\\
2.05540712823268	-0.131948588281725\\
2.07396278328999	-0.169332489372267\\
2.08847528569881	-0.203927707183566\\
2.09964889554858	-0.236162518549415\\
2.10802305513758	-0.266403470902291\\
2.11401000820589	-0.294965496142634\\
2.11792262303556	-0.322120867955179\\
2.11999485572392	-0.348106911291836\\
2.12039666474396	-0.3731325425674\\
2.11924469208138	-0.39738378512693\\
2.11660964448466	-0.421028418889408\\
2.11252101571752	-0.444219913015945\\
2.10696956246702	-0.467100769619532\\
2.09990776123253	-0.489805381061177\\
2.09124831286927	-0.512462475132387\\
2.08086060958534	-0.53519719055088\\
2.06856492130052	-0.558132786625655\\
2.05412387949755	-0.581391940158127\\
2.03723062115011	-0.605097510735529\\
2.01749268582802	-0.629372548500999\\
1.99441041849142	-0.654339154204018\\
1.96734820656221	-0.680115545407622\\
1.93549637849465	-0.706810282115394\\
1.89782106738893	-0.734511980507324\\
1.85299897045652	-0.763271883400435\\
1.79933414963148	-0.793075222898754\\
1.73465580805775	-0.823795289805376\\
1.6562014095423	-0.855121592693686\\
1.56050256136341	-0.886451175769476\\
1.44331848846043	-0.916732514784459\\
1.29971241791409	-0.944260595272382\\
1.12444293603457	-0.966452087976906\\
0.912917778625771	-0.979697607096246\\
0.662923339355622	-0.979494985082252\\
0.37698478839068	-0.961143526917589\\
0.0643945057451018	-0.92113151248099\\
-0.258836596504117	-0.858806199962149\\
-0.573293343729396	-0.77729038059577\\
-0.86154610949324	-0.682778190636289\\
-1.11246542293541	-0.58253462290651\\
-1.32228123826288	-0.482886497173722\\
-1.4928788430727	-0.388183675967567\\
-1.62920072357579	-0.300767744684041\\
-1.73713626895234	-0.221463029054589\\
-1.82229189975114	-0.15015026080482\\
-1.88946780204592	-0.0862173983858334\\
-1.9425473720943	-0.0288514537115337\\
-1.98457653540813	0.0227960100964114\\
-2.01790588850602	0.0695260087674356\\
-2.04433584795107	0.112057504232829\\
-2.06524208259898	0.151018597474137\\
-2.08167602871728	0.186950344449971\\
-2.09444236973088	0.220315931258629\\
-2.10415752737457	0.251511445216503\\
-2.11129331982863	0.280876428860417\\
-2.11620933433931	0.308703443840538\\
-2.11917680800605	0.335246401829789\\
-2.12039612423771	0.360727673755501\\
-2.1200094724428	0.385344097860135\\
-2.11810978260752	0.409272042657603\\
-2.11474671244081	0.432671680630897\\
-2.10993020723372	0.455690611939787\\
-2.10363194826658	0.478466953807131\\
-2.09578483471324	0.50113198435242\\
-2.08628048961741	0.523812399829979\\
-2.07496462746163	0.546632209530777\\
-2.06162995442018	0.569714248797749\\
-2.04600607729657	0.593181230636013\\
-2.02774565683554	0.617156169080839\\
-2.0064057380819	0.641761875216961\\
-1.98142280905275	0.667119021938585\\
-1.95207967357141	0.693341953353095\\
-1.91746169957402	0.720530913984127\\
-1.87639952256073	0.748758596599108\\
-1.82739512479565	0.778047728646369\\
-1.76852903775977	0.808334698837547\\
-1.69734967106284	0.839411914748699\\
-1.61075435885193	0.870838996005181\\
-1.50489091967973	0.901811446283295\\
-1.37514641412866	0.930979107423511\\
-1.21635403674298	0.95622448292148\\
-1.02343225058082	0.974458275051307\\
-0.792711621111322	0.98157953768228\\
};
\addplot [color=mycolor1, forget plot]
  table[row sep=crcr]{%
-0.392413722930891	0.84554181943936\\
0.0574991168375939	0.831154097362173\\
0.505147659949656	0.788519822044919\\
0.911684194589035	0.723753194528262\\
1.25373099685688	0.64685994138717\\
1.52646987318868	0.567132827487189\\
1.73723815412086	0.490727610699746\\
1.89787424528086	0.420675286662048\\
2.02003209129655	0.3578942088356\\
2.11336301864759	0.302168930108651\\
2.18525659578056	0.252792732245178\\
2.24116986978698	0.208915948011435\\
2.28507546096545	0.169710004235972\\
2.31985801824868	0.134432497075667\\
2.34762016050889	0.102444204275014\\
2.36990526938917	0.0732049691631955\\
2.38785567468071	0.0462616886765516\\
2.40232392619075	0.0212344929143888\\
2.41395099003992	-0.00219632298715742\\
2.42322138148395	-0.0243017186686428\\
2.43050222936105	-0.0453135288483911\\
2.43607108364639	-0.0654324372620509\\
2.44013575490199	-0.0848345583017478\\
2.44284842789974	-0.103676892510789\\
2.44431557320258	-0.122101899568361\\
2.44460468239501	-0.140241400069661\\
2.44374849975782	-0.158219985998165\\
2.44174716327257	-0.17615809341235\\
2.43856846387418	-0.194174870617511\\
2.43414625602581	-0.212390960687017\\
2.42837688181337	-0.230931307666362\\
2.42111328258965	-0.249928089576639\\
2.41215624176518	-0.269523875938011\\
2.40124189793098	-0.289875098650625\\
2.38802424562686	-0.311155904824948\\
2.37205073996559	-0.333562413755809\\
2.35272825286137	-0.357317299644011\\
2.32927537032245	-0.382674413614582\\
2.30065521410733	-0.409922742037863\\
2.26548044975565	-0.439388182657007\\
2.22187884264946	-0.471430051093477\\
2.16730403669588	-0.506426266136148\\
2.0982739817986	-0.544735706230507\\
2.01002436055964	-0.586616641754953\\
1.89609176532488	-0.632064808809404\\
1.74792684764033	-0.680515169409561\\
1.55484842567927	-0.730343280848899\\
1.30505799123314	-0.778160890317086\\
0.988919447812327	-0.818148046207319\\
0.605416543083506	-0.842192477576616\\
0.169844949337834	-0.841990436587597\\
-0.284265742050115	-0.81313340510902\\
-0.715481032707434	-0.758309777154651\\
-1.09145467912348	-0.686181762958299\\
-1.39848995418638	-0.60688389900676\\
-1.63889402833189	-0.528262755877788\\
-1.82305794230454	-0.454813385393407\\
-1.96310514827734	-0.388371966541922\\
-2.06978862969178	-0.329187492207117\\
-2.15160841140705	-0.27673952309024\\
-2.21493276229014	-0.230220283082745\\
-2.26442186776805	-0.188777889956994\\
-2.30346035018527	-0.151622578146054\\
-2.33450919511601	-0.118063158297597\\
-2.35936804367039	-0.0875111519170718\\
-2.37936328361864	-0.0594716327281494\\
-2.3954807891641	-0.033529829150519\\
-2.40845918011913	-0.00933749359531244\\
-2.41885544449055	0.0133993926007298\\
-2.42709131700411	0.0349311323748096\\
-2.43348622202874	0.0554730815850008\\
-2.43828076060055	0.0752128930362168\\
-2.44165345782466	0.0943165027594429\\
-2.44373262028527	0.112933115758461\\
-2.44460455599637	0.131199419032673\\
-2.44431899168683	0.149243217004679\\
-2.44289222122969	0.167186655319756\\
-2.44030829102082	0.185149175644964\\
-2.43651834129811	0.203250326852542\\
-2.43143805178636	0.22161254615892\\
-2.42494296339643	0.240364016156838\\
-2.41686124184706	0.259641698268294\\
-2.40696318622118	0.279594636712781\\
-2.39494642873272	0.300387613833229\\
-2.38041526965797	0.322205206779803\\
-2.36285186960453	0.345256226766583\\
-2.34157597558486	0.369778376738065\\
-2.31568834666596	0.396042667919569\\
-2.28399090003332	0.424356551513887\\
-2.24487367983333	0.45506358830937\\
-2.19615515333315	0.488535316920607\\
-2.13485898358148	0.525146943201869\\
-2.05691063622175	0.565221183974557\\
-1.95675047473234	0.608912266997585\\
-1.82691017093183	0.655983984208223\\
-1.65773700274838	0.705418249335526\\
-1.43775505952901	0.75480505417595\\
-1.15564363113835	0.799599025224289\\
-0.805088588744831	0.832719416165173\\
-0.392413722930893	0.84554181943936\\
};
\addplot [color=mycolor1, forget plot]
  table[row sep=crcr]{%
0.127583927346025	0.822354782701324\\
0.648213232803182	0.80605800795337\\
1.10318107339043	0.76304053956409\\
1.46974443062957	0.704857395593081\\
1.7506277306205	0.641835239258246\\
1.96083296670893	0.580446894605021\\
2.1172920353496	0.523753890913169\\
2.23435929047659	0.472709568819359\\
2.32289332314376	0.427209264636958\\
2.39071777840021	0.386710389660878\\
2.44337011656711	0.350545072678569\\
2.48476144153763	0.318059921922197\\
2.51767062198877	0.288669271144002\\
2.54409160763447	0.261868466494899\\
2.56547156713278	0.237230258626693\\
2.58287306167405	0.214395113582127\\
2.5970847403339	0.193060239851003\\
2.60869741614965	0.172969283164263\\
2.61815680955908	0.153903340296842\\
2.62580044199271	0.135673377279496\\
2.63188363663354	0.118113912642419\\
2.63659792076735	0.101077754940444\\
2.64008402751842	0.0844315774446671\\
2.6424409643375	0.0680521305771022\\
2.64373212040189	0.0518229159299322\\
2.64398903950231	0.0356311665529758\\
2.64321323224248	0.0193649933701248\\
2.64137620229844	0.00291056587197173\\
2.63841768691518	-0.0138508038767062\\
2.63424193709751	-0.0310458117512815\\
2.62871166336885	-0.0488126995318836\\
2.62163901919879	-0.0673053570165812\\
2.61277264673791	-0.0866981974870052\\
2.60177931095254	-0.107192055562649\\
2.58821791195275	-0.129021398355355\\
2.57150255840735	-0.152463170319034\\
2.55084970446004	-0.177847574298192\\
2.52520179584149	-0.205570946037465\\
2.49311601422567	-0.236110429802094\\
2.45260104892505	-0.270039025142066\\
2.40087706065533	-0.308036934055735\\
2.3340250863367	-0.350889310674258\\
2.24648755503755	-0.39944821162938\\
2.13040049228801	-0.454512077012021\\
1.97483523001082	-0.516532619024773\\
1.76533290111822	-0.585000852211627\\
1.48484120470863	-0.657355061917321\\
1.11829318779626	-0.727527135003885\\
0.663073259146776	-0.785196445728787\\
0.142231167785234	-0.818072895775999\\
-0.393472173993291	-0.818155783814035\\
-0.886049713082783	-0.787199461778638\\
-1.29779320358844	-0.73511883652189\\
-1.62009373036179	-0.673452659624767\\
-1.86347479476686	-0.610680751481175\\
-2.04480161083383	-0.551419036273427\\
-2.17999142771539	-0.497516541074019\\
-2.28164162385726	-0.449294850545918\\
-2.35900309711399	-0.406375673156854\\
-2.41866449572068	-0.368128112625037\\
-2.46527788530321	-0.33388094484314\\
-2.5021365149178	-0.303011228645238\\
-2.53159107832171	-0.274973555532582\\
-2.55533779171109	-0.249302966880185\\
-2.57461494025345	-0.225607431139841\\
-2.59033677587351	-0.203557141000713\\
-2.6031851909469	-0.182873735592355\\
-2.61367299057704	-0.163320619972174\\
-2.62218795812845	-0.144694689635349\\
-2.62902380288774	-0.126819405365888\\
-2.63440202887595	-0.109539031254651\\
-2.63848741459588	-0.092713817117327\\
-2.641398900009	-0.0762159155355208\\
-2.64321707729127	-0.059925845691677\\
-2.6439890701039	-0.043729338835614\\
-2.64373129273313	-0.0275144184669946\\
-2.64243035866194	-0.0111685801382348\\
-2.64004222456362	0.00542405958235494\\
-2.63648948404545	0.0223857783896487\\
-2.63165654158267	0.0398485720872036\\
-2.62538217455736	0.0579579507065615\\
-2.61744869653946	0.0768773987940208\\
-2.60756652062552	0.0967937282660691\\
-2.59535231747372	0.117923594147524\\
-2.58029806141494	0.140521481901149\\
-2.56172689521631	0.164889487239771\\
-2.53872967067157	0.191389142111881\\
-2.51007287587656	0.220455274617765\\
-2.47406396369724	0.252611164735925\\
-2.42835338315418	0.288482511170148\\
-2.3696439929403	0.328803780544085\\
-2.29327057097972	0.374401992519135\\
-2.19261540300272	0.426125476488849\\
-2.05837353081808	0.484651867778157\\
-1.87786003310808	0.550056379298748\\
-1.635040280013	0.620972440860603\\
-1.31294328821675	0.693264653039546\\
-0.901053209393393	0.758704786362719\\
-0.408199139990258	0.805422528040251\\
0.127583927346023	0.822354782701324\\
};
\addplot [color=mycolor1, forget plot]
  table[row sep=crcr]{%
0.5647960623138	0.853812859999501\\
1.07128673553608	0.838198408636122\\
1.48039777394432	0.799662583060077\\
1.79266817538587	0.750166491000075\\
2.02479185719599	0.698110293912658\\
2.19632768966487	0.648020529098301\\
2.32386018402829	0.601806757171454\\
2.4198235022426	0.559959493407524\\
2.49307769479211	0.522306821137164\\
2.54982610613474	0.488416853654953\\
2.59440759520677	0.457790987159596\\
2.62987906959654	0.429948280836494\\
2.65841847283103	0.404457029515137\\
2.68159732786201	0.380942142385574\\
2.70056417724772	0.359082240673385\\
2.71616855270952	0.338602997986055\\
2.72904546107385	0.319269651181613\\
2.73967355164051	0.300879897931048\\
2.74841557811197	0.283257601678915\\
2.75554679891187	0.266247368715491\\
2.76127503362634	0.249709913689685\\
2.76575483893485	0.233518077021313\\
2.76909744165955	0.21755334638614\\
2.77137751332221	0.201702738965491\\
2.77263748964209	0.185855909231112\\
2.77288986443052	0.169902352837051\\
2.77211767622501	0.153728577765967\\
2.77027322578096	0.1372151074127\\
2.76727488704598	0.120233165192337\\
2.76300167813254	0.102640864674644\\
2.7572850126839	0.0842786907811072\\
2.74989671666541	0.0649640032548598\\
2.7405319138553	0.044484220158523\\
2.72878466789091	0.0225882441383666\\
2.71411318339692	-0.00102442090286279\\
2.69578969943864	-0.0267175372222971\\
2.67282762221409	-0.0549357456712831\\
2.64387443893777	-0.086226620103167\\
2.60705283594023	-0.121267363051761\\
2.55972344891701	-0.160894741313181\\
2.49813067786754	-0.206132626129549\\
2.41688109230449	-0.258201271274281\\
2.30820646225063	-0.31846933337672\\
2.16102794156049	-0.388261527787755\\
1.9600942649918	-0.468349327874165\\
1.68618532970365	-0.557854874809137\\
1.31978937039226	-0.652388661475404\\
0.851701177658143	-0.742091957403252\\
0.2996878226251	-0.812222786327461\\
-0.283403565055562	-0.849311024806142\\
-0.829131909311639	-0.849671124327323\\
-1.28856516732028	-0.820991188416604\\
-1.64777878514603	-0.775658460547183\\
-1.91749270169338	-0.724097527473434\\
-2.11699928820803	-0.67265504472608\\
-2.26472299505105	-0.624376210638953\\
-2.37516131876585	-0.58033858751573\\
-2.45884957200914	-0.540632718202448\\
-2.52320806872325	-0.504922443915455\\
-2.57342209009238	-0.472726917545838\\
-2.61312907901512	-0.44355003443715\\
-2.64490534747896	-0.416933488381197\\
-2.67059795700049	-0.392473816110554\\
-2.69154829428359	-0.36982347867641\\
-2.70874274362947	-0.34868555022563\\
-2.72291490439807	-0.328806414191347\\
-2.73461560117952	-0.309968379463334\\
-2.74426133726505	-0.291982958094855\\
-2.75216816003143	-0.274685010667489\\
-2.75857551538068	-0.257927732765276\\
-2.7636631154935	-0.241578364791952\\
-2.76756282819312	-0.225514479799037\\
-2.7703669221514	-0.209620702756656\\
-2.77213354502348	-0.193785722015356\\
-2.77288999071339	-0.177899461140576\\
-2.77263407401168	-0.161850282759114\\
-2.77133373868317	-0.145522093310267\\
-2.7689248503333	-0.128791207021445\\
-2.76530694288232	-0.111522807349164\\
-2.76033647049569	-0.0935668124380141\\
-2.75381683187417	-0.0747529051583457\\
-2.7454840339232	-0.054884424790499\\
-2.73498627666103	-0.033730733354312\\
-2.7218548619301	-0.01101756388009\\
-2.70546248424923	0.0135852635734658\\
-2.68496288535053	0.040479485056165\\
-2.65920263327697	0.0701581547916423\\
-2.62659082253035	0.103230057451741\\
-2.58490501885585	0.140448374709387\\
-2.53100118982908	0.18274056801885\\
-2.4603826165084	0.23122979984345\\
-2.36657498260569	0.287222701791902\\
-2.24027879008065	0.35210449981766\\
-2.06840652840338	0.42701652250624\\
-1.8335570524236	0.512090553182067\\
-1.51554768948008	0.604966091936343\\
-1.09817415321068	0.698692925584068\\
-0.583632372478345	0.780585967110578\\
-0.00768759183282086	0.835429000391897\\
0.564796062313798	0.853812859999501\\
};
\addplot [color=mycolor1, forget plot]
  table[row sep=crcr]{%
0.890712145070811	0.909450418757661\\
1.35615885261011	0.895223268347426\\
1.7178770133649	0.861204771906555\\
1.98883562135947	0.81827185341504\\
2.18934297244442	0.773305016262608\\
2.33813014010674	0.729851990461745\\
2.44972390595194	0.68940712468328\\
2.5346150158692	0.652382072672868\\
2.60017637536743	0.618678388031736\\
2.65156020319445	0.58798790340672\\
2.69238512568549	0.559939164936878\\
2.72521891863578	0.534163927537106\\
2.75190661716258	0.510324101099737\\
2.77379135916652	0.488119891841291\\
2.79186327696568	0.467289454487744\\
2.80686061899949	0.447605063345819\\
2.81933911290857	0.428868163364543\\
2.82972005140612	0.410904358831404\\
2.83832395807279	0.393558763793993\\
2.84539433942666	0.376691836434651\\
2.85111450317749	0.360175680602173\\
2.85561942298616	0.343890737238368\\
2.85900396567129	0.327722764353783\\
2.86132834493843	0.311559995282265\\
2.86262134708813	0.29529036032451\\
2.86288163617489	0.278798650749535\\
2.86207725184185	0.261963492909228\\
2.8601432339406	0.244653981187701\\
2.85697711819763	0.226725788861691\\
2.85243181853318	0.208016532305813\\
2.84630510839095	0.18834010198256\\
2.83832448478395	0.167479587806611\\
2.82812556800403	0.145178310381336\\
2.81522123766127	0.121128317454752\\
2.79895724360707	0.094955515907978\\
2.7784477619024	0.0662003995574959\\
2.75248083374686	0.0342931637967893\\
2.71937815937494	-0.00147795233693322\\
2.67678546413278	-0.0420055273043307\\
2.62135794787402	-0.0884069185173281\\
2.54829130368562	-0.142064245531875\\
2.45064075021913	-0.204634033811873\\
2.31839972834169	-0.277960726236148\\
2.13747019310304	-0.363747848256886\\
1.88917402634202	-0.46271092344986\\
1.55217347467057	-0.572854125554091\\
1.11031757645939	-0.686931045624751\\
0.5686004428524	-0.790900393956026\\
-0.0317095263207011	-0.867395661633784\\
-0.621049791188676	-0.905112993988739\\
-1.13650530825543	-0.905617152832956\\
-1.5495086819587	-0.87991946706539\\
-1.86345932543428	-0.840329516553253\\
-2.0966362049343	-0.795758438133868\\
-2.26919022981483	-0.751261298010872\\
-2.3978358398636	-0.70921082060183\\
-2.49498482746347	-0.67046583142385\\
-2.56944700736389	-0.635131647095147\\
-2.62738491506972	-0.602979271852111\\
-2.67311188480114	-0.573656849805782\\
-2.70967167159994	-0.546789350398643\\
-2.73923734713559	-0.522021743950802\\
-2.7633805729318	-0.49903476354143\\
-2.78325281465181	-0.477547928176154\\
-2.79970790674062	-0.457317036615965\\
-2.81338569230257	-0.438129586166547\\
-2.82476970026708	-0.419799710568105\\
-2.83422733473163	-0.402163320645339\\
-2.84203813121379	-0.385073692666452\\
-2.84841374148752	-0.368397543383253\\
-2.85351207522891	-0.352011538208387\\
-2.85744721363654	-0.335799140336732\\
-2.86029616421148	-0.319647693949695\\
-2.86210314856951	-0.303445628825438\\
-2.86288184224531	-0.28707966883864\\
-2.8626157732811	-0.270431918524477\\
-2.86125690299825	-0.253376687088789\\
-2.85872223112257	-0.235776885282436\\
-2.8548880620305	-0.217479794390573\\
-2.84958130814093	-0.198311954384588\\
-2.84256684818636	-0.178072845120389\\
-2.83352944016367	-0.156526934314425\\
-2.82204791572865	-0.133393532481749\\
-2.80755820480412	-0.108333723894724\\
-2.78929991967601	-0.0809334392355528\\
-2.76623839708287	-0.050681533377657\\
-2.73694969564132	-0.016941637589268\\
-2.69944929699592	0.0210831417040737\\
-2.65093531208999	0.0643921817503108\\
-2.58740373957987	0.114231187293853\\
-2.5030803834195	0.172122051690198\\
-2.38961812563455	0.239836476890025\\
-2.23508483033654	0.319213739722195\\
-2.0230669203918	0.411616201279232\\
-1.73304317172197	0.516683621167623\\
-1.34476477329517	0.630124557518481\\
-0.850287967294496	0.741278414960865\\
-0.271751240440297	0.833554342235435\\
0.332303788006903	0.891315090999931\\
0.890712145070809	0.909450418757661\\
};
\addplot [color=mycolor1, forget plot]
  table[row sep=crcr]{%
1.13228895415094	0.976232367083907\\
1.55453529055446	0.963380785631541\\
1.87735344896523	0.933035113576346\\
2.1184845775288	0.894825371350644\\
2.29787683748837	0.854585882214751\\
2.43227481251057	0.815326896230432\\
2.53422473485194	0.778370124569441\\
2.61269575478856	0.744139423495457\\
2.67399866107858	0.712620330446479\\
2.72257214445541	0.683604727923391\\
2.76156142008329	0.656814173994275\\
2.79322006604216	0.631958960343927\\
2.81918314674446	0.608764303135775\\
2.84065202094255	0.586980174694858\\
2.85851989743146	0.56638325395648\\
2.87345784029931	0.546775281493775\\
2.88597427105367	0.527979946664411\\
2.8964565493718	0.509839330570172\\
2.90520028440166	0.492210366795147\\
2.91243011729296	0.474961497954436\\
2.91831446402342	0.457969563902499\\
2.92297587905606	0.441116886757648\\
2.92649814196841	0.424288482122503\\
2.92893078207311	0.407369305689946\\
2.93029147526219	0.390241429259922\\
2.93056652851896	0.372781023828083\\
2.92970947811905	0.354855005560475\\
2.92763764022047	0.336317169435942\\
2.92422624024647	0.317003591131751\\
2.91929947813593	0.296727015455653\\
2.91261751677722	0.275269863030624\\
2.90385784701549	0.252375368188296\\
2.89258868678762	0.227736201213942\\
2.87823086126403	0.200979719574153\\
2.86000274508799	0.171648736327014\\
2.83683995383567	0.139176417712259\\
2.8072769916719	0.102853728926342\\
2.76927123580572	0.0617880301729651\\
2.71993968118953	0.0148527389768783\\
2.65516586661613	-0.0393677412042734\\
2.56902248807935	-0.102622103283209\\
2.45296226998788	-0.176981508802543\\
2.29481334781964	-0.264668724270688\\
2.07792820730373	-0.367504549350102\\
1.78167534396632	-0.485598501371426\\
1.38600553881976	-0.614970273259653\\
0.883559945743928	-0.744805485886824\\
0.297063929592292	-0.857551335033007\\
-0.314995550482536	-0.935748231473546\\
-0.881964052990284	-0.972192941185301\\
-1.35656539196237	-0.972743377863418\\
-1.72742597329563	-0.949699043438517\\
-2.00684087248312	-0.914467679869741\\
-2.21475183366217	-0.874719671788787\\
-2.36982151425778	-0.834722796055881\\
-2.48667231671519	-0.796519916364948\\
-2.57594861399704	-0.760908291075883\\
-2.64517968856829	-0.728051258689338\\
-2.69965514074735	-0.697816255793652\\
-2.74310677989146	-0.669949578355601\\
-2.77819289831537	-0.644162341846309\\
-2.80683000624538	-0.620170316931991\\
-2.83041747520835	-0.597710497828228\\
-2.8499897378227	-0.57654625934367\\
-2.86632008078914	-0.55646714197932\\
-2.87999208997297	-0.537286293160247\\
-2.89144933133931	-0.518837048204695\\
-2.90103022890587	-0.500969346659074\\
-2.90899273449339	-0.483546279852304\\
-2.91553183885285	-0.466440864345504\\
-2.9207919572066	-0.449533035674002\\
-2.9248755439451	-0.432706806771718\\
-2.92784882832088	-0.415847509214427\\
-2.9297452356072	-0.398839018717003\\
-2.9305668130403	-0.381560851171685\\
-2.93028377937382	-0.36388499688355\\
-2.92883213178227	-0.345672334683861\\
-2.9261090476084	-0.326768430524965\\
-2.92196558218108	-0.306998472609269\\
-2.91619585082817	-0.286161021550744\\
-2.90852144106672	-0.264020152464737\\
-2.89856915113163	-0.24029542772493\\
-2.88583917141691	-0.214648955648693\\
-2.86965932428903	-0.186668556753896\\
-2.8491186548788	-0.155845786349874\\
-2.8229700613926	-0.121547306806751\\
-2.78948610531244	-0.0829780416000335\\
-2.74624382975354	-0.0391351503135862\\
-2.6898027911544	0.0112456755571244\\
-2.61522702752864	0.0697429611071754\\
-2.5153954382363	0.13827444486632\\
-2.38007896894432	0.21902535174251\\
-2.19493334032315	0.314123658023469\\
-1.9410880248057	0.424762276878524\\
-1.59722619747505	0.549365716412347\\
-1.14760097929895	0.6808116647911\\
-0.597875475194193	0.804535971041456\\
0.0103610704461334	0.901750895964801\\
0.607930437625238	0.959080304254248\\
1.13228895415094	0.976232367083907\\
};
\addplot [color=mycolor1, forget plot]
  table[row sep=crcr]{%
1.31724795584908	1.04879793416756\\
1.70068896800859	1.03714926463591\\
1.99243412563981	1.00972335390571\\
2.21128528948877	0.975035371074232\\
2.37559068430626	0.938170427444379\\
2.50007029924285	0.901800397049627\\
2.59560899674705	0.867161038389809\\
2.66999240803846	0.834708218963402\\
2.72873547749773	0.804501169978652\\
2.77575356376531	0.776411417061493\\
2.81384956025627	0.750231993046598\\
2.84505236265008	0.725732440318642\\
2.87084839353232	0.702685120391255\\
2.89233964894694	0.680876613176355\\
2.91035210944293	0.660111518528291\\
2.92551071394318	0.640212499121538\\
2.93829169714277	0.621018557932336\\
2.9490594557421	0.602382565596242\\
2.95809270468511	0.58416853462776\\
2.96560310047765	0.56624886235795\\
2.97174845934883	0.548501617650293\\
2.97664199416488	0.530807866768432\\
2.98035851181535	0.513048988155423\\
2.98293817100018	0.495103896110834\\
2.98438814435474	0.476846068882062\\
2.9846823200263	0.458140250826432\\
2.98375898607868	0.438838666034884\\
2.98151623928633	0.41877653748519\\
2.9778046186795	0.397766646238988\\
2.9724161465914	0.375592583047378\\
2.96506851384193	0.352000231958747\\
2.95538249317645	0.326686872401422\\
2.94284968615133	0.299287082441922\\
2.92678621646407	0.269354364850047\\
2.90626568799413	0.236337108267457\\
2.88002119272049	0.199547193952261\\
2.84630076925584	0.158119443323986\\
2.80265275945175	0.110960658291323\\
2.74560663983224	0.0566894522876568\\
2.67020291053394	-0.00642470003833229\\
2.56932201190236	-0.0804962544596073\\
2.43279967300743	-0.167962268872052\\
2.24648657562714	-0.271266028968002\\
1.9919207905751	-0.391979782927913\\
1.64840702094397	-0.528949767320616\\
1.20066961434231	-0.675428087235624\\
0.65373091120456	-0.816901827471298\\
0.0466851295588091	-0.933778186550022\\
-0.554257251248188	-1.01071739348441\\
-1.08724660964121	-1.04507959019524\\
-1.52142481475842	-1.04562549335195\\
-1.85690865491003	-1.02478708833223\\
-2.10975536194308	-0.992899466037865\\
-2.29924077089661	-0.956664390850311\\
-2.44204554602821	-0.919821926019723\\
-2.55090826368052	-0.884223183205733\\
-2.63505602361024	-0.850651395512304\\
-2.70104332736356	-0.819329232452809\\
-2.75351347363108	-0.790203558051256\\
-2.79577501936664	-0.76309717269976\\
-2.83020913831105	-0.737786709899157\\
-2.85854980616021	-0.714040992926969\\
-2.88207503608867	-0.691638662656213\\
-2.90173766121912	-0.670375131436804\\
-2.9182553876661	-0.650064167795247\\
-2.93217336191435	-0.630536882903907\\
-2.94390805059626	-0.61163954606285\\
-2.95377827091342	-0.593230944566961\\
-2.9620272587906	-0.575179625835013\\
-2.96883837402682	-0.557361158923949\\
-2.97434618406172	-0.539655444784479\\
-2.97864408732369	-0.521944044947472\\
-2.98178923332578	-0.50410746231682\\
-2.98380520340743	-0.48602228163461\\
-2.98468268775975	-0.467558052713413\\
-2.98437819755078	-0.448573771047592\\
-2.98281065760069	-0.428913773217576\\
-2.9798555075506	-0.408402813770303\\
-2.97533566543103	-0.386840020298865\\
-2.96900833349175	-0.363991327101009\\
-2.96054608874203	-0.339579856150916\\
-2.9495099034091	-0.313273536888723\\
-2.93531053375228	-0.284669024188519\\
-2.91715286645675	-0.253270685689284\\
-2.89395496323989	-0.218463112035864\\
-2.86422917280108	-0.179475359715508\\
-2.82590609659725	-0.135335264769788\\
-2.77607276040422	-0.0848134410459427\\
-2.7105844092644	-0.0263608877946367\\
-2.62349944761116	0.041944146798071\\
-2.5062973391022	0.122395838482514\\
-2.34692584809889	0.217499779025785\\
-2.12902993939484	0.329424887102003\\
-1.83250714465007	0.458686424420379\\
-1.4379374353601	0.601721274169193\\
-0.938024113384351	0.74798013060162\\
-0.353923457943785	0.879607203169263\\
0.258945755246423	0.977741613950653\\
0.832029281899767	1.0328566526933\\
1.31724795584908	1.04879793416756\\
};
\addplot [color=mycolor1, forget plot]
  table[row sep=crcr]{%
1.46475086798591	1.12486267541903\\
1.81455064496186	1.11424173504341\\
2.08101811692771	1.0891848432148\\
2.28239477396513	1.05725633865747\\
2.43516101366466	1.02297096369503\\
2.55222786165662	0.988759043180935\\
2.6431073278131	0.955802910425065\\
2.7146373091078	0.924590286752199\\
2.77170413818777	0.895241476523484\\
2.81781236316433	0.867692314960797\\
2.85549699948198	0.841793134484645\\
2.88661151674617	0.817360852200531\\
2.91252656590965	0.794205432401846\\
2.93426699299933	0.772142515743081\\
2.95260677318198	0.750998663464851\\
2.96813529734919	0.73061271717463\\
2.98130405122357	0.710835161507174\\
2.9924597435533	0.691526494724127\\
3.00186794581836	0.672555128082024\\
3.00972997636107	0.653795067109678\\
3.01619486979958	0.63512347785137\\
3.02136766535384	0.616418154327055\\
3.02531482399341	0.597554849653668\\
3.02806727630516	0.578404395087712\\
3.02962136384206	0.558829498131974\\
3.02993773187217	0.538681075652315\\
3.02893803222613	0.517793934940117\\
3.02649907337657	0.495981559240114\\
3.0224437786017	0.473029678094332\\
3.0165279391859	0.4486881991649\\
3.00842121624705	0.422660937455598\\
2.9976800587588	0.394592389157719\\
2.98370902355612	0.364050550863591\\
2.96570518575104	0.330504479387123\\
2.94257758377414	0.293294950382676\\
2.91282947836327	0.251596311720892\\
2.8743850248728	0.204367746024086\\
2.82433329036807	0.150293442953563\\
2.75855200597052	0.0877155387590932\\
2.67116583336082	0.0145754690441363\\
2.55380702427253	-0.0715921184010218\\
2.39473271379309	-0.173506451243602\\
2.17814159007815	-0.293605490771841\\
1.88475266322618	-0.432752885013195\\
1.49594165548874	-0.587840504337622\\
1.00421008637066	-0.748816520881164\\
0.428204497449267	-0.897960863886098\\
-0.181069342161435	-1.01542826939592\\
-0.758071027725421	-1.08942430520555\\
-1.25383701882624	-1.12144949613573\\
-1.65114711302072	-1.12196791913176\\
-1.95706424890432	-1.10296348206271\\
-2.18873552451611	-1.0737368824625\\
-2.36396063075085	-1.04021870875464\\
-2.49749047958932	-1.00576048059793\\
-2.60046006633146	-0.972081963203389\\
-2.68094737638745	-0.93996520250237\\
-2.74473272823448	-0.909684049981042\\
-2.79595080402652	-0.881250064378457\\
-2.83757850259748	-0.854547535635447\\
-2.87178039260318	-0.829405541040784\\
-2.90014809676444	-0.805635276710233\\
-2.92386524475683	-0.783048536119421\\
-2.94382142623585	-0.761466065933136\\
-2.96069143501722	-0.74072055377238\\
-2.97499083610984	-0.720656820926039\\
-2.98711525579079	-0.701130600286\\
-2.99736835333593	-0.682006626587287\\
-3.00598180523958	-0.66315640643376\\
-3.01312954511842	-0.644455835646819\\
-3.01893776823853	-0.625782717974056\\
-3.02349170434944	-0.607014171680475\\
-3.02683980356726	-0.588023866192023\\
-3.0289957113926	-0.568678996357998\\
-3.02993819078351	-0.548836868510815\\
-3.02960895059597	-0.528340934081635\\
-3.02790813295673	-0.507016057545366\\
-3.02468696775183	-0.484662740045331\\
-3.01973678363577	-0.461049931197972\\
-3.01277312067384	-0.435905940663681\\
-3.00341304428323	-0.408906797741841\\
-2.99114279818484	-0.379661190623549\\
-2.97527147807427	-0.347690840003647\\
-2.95486418674673	-0.312404833834234\\
-2.92864474542825	-0.273066128691932\\
-2.89485294071803	-0.228748292490477\\
-2.85103389105989	-0.178281118439137\\
-2.79372728790955	-0.120186213158118\\
-2.71801409109878	-0.0526109543084696\\
-2.6168773368519	0.0267122368710863\\
-2.48037329926872	0.120411627235978\\
-2.29477385383028	0.231169688316104\\
-2.04231301726284	0.360863536376334\\
-1.7031762285242	0.50873932397908\\
-1.26255393483827	0.668547632236554\\
-0.724234652787264	0.82617364302703\\
-0.12359505347142	0.961691386230613\\
0.477364392125094	1.05806587603998\\
1.01792006988566	1.11014470800911\\
1.46475086798591	1.12486267541903\\
};
\addplot [color=mycolor1, forget plot]
  table[row sep=crcr]{%
1.58697213121126	1.20336136317783\\
1.9076340390308	1.19362316957183\\
2.1529623596323	1.17054474461314\\
2.33998746848524	1.14088153826959\\
2.48338418761784	1.10869017780001\\
2.59449987839305	1.07621039709076\\
2.68170112752524	1.04458260799585\\
2.75104372054698	1.01432015572903\\
2.80689578120153	0.985592670342716\\
2.8524221421894	0.958388406892165\\
2.88993526501501	0.932604821861844\\
2.92114215149815	0.908098078821548\\
2.94731633975157	0.884709440025805\\
2.96941768474312	0.86227875839831\\
2.98817617409695	0.840650810103552\\
3.00415099310331	0.819677676168766\\
3.01777246146683	0.799218958659433\\
3.02937199965124	0.77914081640432\\
3.03920361601012	0.759314352276649\\
3.04745928077584	0.739613624834528\\
3.05427978771138	0.719913405723052\\
3.05976217492766	0.700086712141951\\
3.06396439936458	0.680002083025932\\
3.0669076774521	0.659520521792054\\
3.06857667662451	0.638491987024067\\
3.06891753693403	0.616751267716092\\
3.06783349085648	0.594113025247864\\
3.06517760298044	0.570365713580987\\
3.06074183339464	0.545263994757396\\
3.05424118955854	0.518519139618314\\
3.04529109862083	0.489786733138532\\
3.03337519605724	0.458650778994819\\
3.01779932079387	0.424603012444904\\
2.99762538192359	0.387015895099739\\
2.97157555727582	0.345107440606509\\
2.93789252757304	0.297895896533356\\
2.894134677273	0.244142888401816\\
2.83687646773538	0.182286152050274\\
2.76127581055905	0.110370308801734\\
2.66047170131774	0.0260025409344433\\
2.52481587585663	-0.0735981350095048\\
2.34109787068459	-0.191304729055157\\
2.09235121362609	-0.329248668687806\\
1.75971183833163	-0.48704866481498\\
1.32881027591122	-0.658999800722057\\
0.802036457999092	-0.831568463994606\\
0.210734221357192	-0.984824584999507\\
-0.38777641280512	-1.10035461917923\\
-0.934360738786419	-1.17053852025654\\
-1.39336140266842	-1.20022669008197\\
-1.75779374171375	-1.20070866807381\\
-2.03862293124132	-1.18325589981369\\
-2.25276893646319	-1.15622999995266\\
-2.41635313761951	-1.12492897948904\\
-2.54239200337834	-1.09239593812663\\
-2.64066508308718	-1.06024728385831\\
-2.71829858239608	-1.02926440743518\\
-2.78043476129419	-0.999762345382136\\
-2.83078849227627	-0.971805143099149\\
-2.87206179875188	-0.945327442609912\\
-2.90623884510072	-0.920201620544917\\
-2.93479211592366	-0.896274071926745\\
-2.95882589717626	-0.873384213184259\\
-2.97917639446744	-0.851373868808164\\
-2.99648202677968	-0.830091335297682\\
-3.01123315083976	-0.80939251750509\\
-3.02380748740073	-0.789140465686734\\
-3.03449549213506	-0.769204040001337\\
-3.04351854501094	-0.749456087159715\\
-3.05104190548527	-0.72977131651809\\
-3.05718374616062	-0.710023945644536\\
-3.062021132934	-0.690085111628488\\
-3.06559349611069	-0.669819992783676\\
-3.06790388643737	-0.649084542835442\\
-3.06891809699603	-0.627721697409026\\
-3.06856152709846	-0.605556863823372\\
-3.06671343951736	-0.582392443487743\\
-3.06319798543667	-0.558001054676279\\
-3.05777100043022	-0.532117013905994\\
-3.0501010498453	-0.504425486699422\\
-3.03974243419978	-0.474548522123226\\
-3.02609671962065	-0.442026930521274\\
-3.0083576305549	-0.406296649987609\\
-2.98543152971664	-0.366657904102167\\
-2.95582179344142	-0.322235192168796\\
-2.91745967145321	-0.271926286852041\\
-2.86745638659513	-0.214339737284079\\
-2.80174211919316	-0.147724783229074\\
-2.71455205963207	-0.0699093564136836\\
-2.59773493723347	0.0217100643328273\\
-2.43994355498867	0.130022399139329\\
-2.22603357434645	0.257683214119972\\
-1.93763912498983	0.405860982827894\\
-1.55696130812789	0.571904053643628\\
-1.07620703943236	0.74636557032752\\
-0.511274730013405	0.911924373514858\\
0.0916394559526066	1.04810571255123\\
0.670514117540115	1.14105418007271\\
1.17582961916804	1.18979941964901\\
1.58697213121126	1.20336136317783\\
};
\addplot [color=mycolor1, forget plot]
  table[row sep=crcr]{%
1.69154852785871	1.28370401175513\\
1.98672928216181	1.27473415003994\\
2.21390108424491	1.25335406547444\\
2.3886721671133	1.22562492412593\\
2.5240760054015	1.19521991116207\\
2.63011715172874	1.1642170823112\\
2.71419350409143	1.13371768638542\\
2.78169954664035	1.10425279247967\\
2.83656251862296	1.07603088498327\\
2.88165563630413	1.04908292777186\\
2.91909840184877	1.02334554935167\\
2.9504695155905	0.998707999571044\\
2.97695656811281	0.975038180312408\\
2.99946129654955	0.952196653784043\\
3.01867392699092	0.930043754760351\\
3.03512601603836	0.908442745974919\\
3.04922825187172	0.887260696633115\\
3.06129762746649	0.866368037516756\\
3.07157699757694	0.845637323770712\\
3.08024907326989	0.824941487340937\\
3.08744624698973	0.804151710236768\\
3.0932571761229	0.783134953971837\\
3.0977307141138	0.761751114076917\\
3.10087751639272	0.739849715679729\\
3.10266942737847	0.717266016282757\\
3.10303654479367	0.693816327060933\\
3.10186163011529	0.66929229710414\\
3.0989712573794	0.643453818580479\\
3.09412272589533	0.616020096217404\\
3.0869852483128	0.586658271738282\\
3.07711317949214	0.554968791982241\\
3.06390794727981	0.520466449283701\\
3.04656369490717	0.482555705357621\\
3.0239891727972	0.440498567965223\\
2.99469475660702	0.393373038710646\\
2.95662820936091	0.340020310101148\\
2.90693577599555	0.278980258156785\\
2.84161740355516	0.208419226490437\\
2.75504125426462	0.126065791553188\\
2.63929931271235	0.0291976991417976\\
2.4834665440008	-0.0852183864878517\\
2.27307022427829	-0.220025901007928\\
1.99064539634526	-0.376669674047728\\
1.61917471026681	-0.552941606294356\\
1.15055721309653	-0.740034199829811\\
0.598048773833731	-0.921163600349624\\
0.00314733408814727	-1.07549448712044\\
-0.575841737161633	-1.18736911049668\\
-1.08929438614472	-1.25336201237515\\
-1.51347869615247	-1.28082049737753\\
-1.84866477490039	-1.2812637280361\\
-2.10779694582459	-1.26515094305336\\
-2.3069495686061	-1.24000737687748\\
-2.46060174261895	-1.21059800012478\\
-2.58025054920589	-1.17970719790656\\
-2.67452367686858	-1.1488614252525\\
-2.74974342309568	-1.11883741880872\\
-2.8105111479618	-1.08998157033276\\
-2.86018303277302	-1.06240009352769\\
-2.90122398017439	-1.03606910801023\\
-2.93546093651639	-1.01089725903163\\
-2.964261486689	-0.986760774587864\\
-2.98865933226719	-0.963522655409174\\
-3.00944268677859	-0.941042759048854\\
-3.02721690008306	-0.919182661092823\\
-3.04244911808689	-0.897807516718047\\
-3.05550031833885	-0.876786189958432\\
-3.06664836696297	-0.855990364817512\\
-3.07610458458484	-0.835293028743695\\
-3.08402551471939	-0.814566525998151\\
-3.09052103557451	-0.79368025920963\\
-3.09565956145848	-0.772498038722634\\
-3.09947078479444	-0.750875021230973\\
-3.10194617235881	-0.728654128915337\\
-3.10303721737287	-0.705661788879955\\
-3.10265123399875	-0.681702772720765\\
-3.10064423317521	-0.656553840372643\\
-3.09681010303249	-0.629955793032172\\
-3.09086488530946	-0.601603407626702\\
-3.08242432191236	-0.571132549326719\\
-3.07097193975157	-0.538103528363932\\
-3.05581359250599	-0.501979477312295\\
-3.03601235654665	-0.462098187829636\\
-3.01029466309495	-0.417635526260147\\
-2.97691413610257	-0.367558441195495\\
-2.93345344412972	-0.310566192624902\\
-2.87653678974171	-0.245021002147175\\
-2.80141888517497	-0.168876660325787\\
-2.70141956169065	-0.0796317756926746\\
-2.56721359818832	0.0256249862756282\\
-2.38613071547999	0.149929294633429\\
-2.14200616450009	0.295636668224313\\
-1.81689468557862	0.46271509430863\\
-1.3968259799178	0.646009298609677\\
-0.882832291688758	0.832645321473129\\
-0.302368412278713	1.00289687492031\\
0.29193820929031	1.13726589176048\\
0.842910913972665	1.22582272036213\\
1.31296805875141	1.27120728874838\\
1.69154852785871	1.28370401175513\\
};
\addplot [color=mycolor1, forget plot]
  table[row sep=crcr]{%
1.7832981678353	1.36546589275169\\
2.05592568115162	1.3571744194569\\
2.26714164491445	1.33728673652804\\
2.43111853478521	1.31126156725633\\
2.55943271347477	1.28244136021363\\
2.66093354729971	1.25276022517939\\
2.74218962233567	1.22327936216145\\
2.80802606279813	1.19453958765605\\
2.86198660051068	1.16677896496198\\
2.90668763683666	1.14006289696381\\
2.94407647731309	1.11436054257094\\
2.97561575220362	1.08958916115493\\
3.00241420566575	1.06563949952862\\
3.02531948618789	1.04239000684265\\
3.04498425031883	1.0197144567338\\
3.06191351815595	0.997485660984367\\
3.07649877708056	0.975576844041152\\
3.08904261974966	0.95386158977852\\
3.09977651875239	0.932212879917087\\
3.10887352086999	0.910501505895431\\
3.11645707010401	0.888593987389956\\
3.12260675718842	0.866350032379446\\
3.12736148556441	0.84361950212847\\
3.13072029732992	0.820238784686038\\
3.13264088455171	0.796026421968094\\
3.13303559280109	0.770777769825975\\
3.13176447555128	0.744258389841893\\
3.12862464535523	0.716195767549337\\
3.12333474415734	0.686268814798393\\
3.1155127546208	0.654094433393716\\
3.10464449980951	0.619210182674226\\
3.09003888625681	0.581051800749411\\
3.07076402681151	0.538923992473511\\
3.04555554798029	0.491962585008751\\
3.01268429798848	0.439086064580252\\
2.96976506994043	0.37893515870206\\
2.91348118717393	0.309801736731583\\
2.83919435906196	0.229555616004797\\
2.74041386938704	0.135595692071535\\
2.60813878932407	0.0248912135021348\\
2.43022180279671	-0.105743493774361\\
2.19124961386915	-0.258874318262122\\
1.87411209379186	-0.434804268222938\\
1.46518116380506	-0.628916844714507\\
0.964245059921325	-0.829016773640376\\
0.395207777190675	-1.01569772268623\\
-0.193748734909379	-1.1686142836128\\
-0.747622298683288	-1.2757260230298\\
-1.22738617820585	-1.33743352552652\\
-1.61919514519511	-1.36280865849501\\
-1.92824504599151	-1.36321391987884\\
-2.16827266488954	-1.34828035900592\\
-2.35424440438222	-1.32479177202571\\
-2.49912053823983	-1.29705421238368\\
-2.61307727724595	-1.26762654584206\\
-2.7037556868631	-1.23795184232903\\
-2.77678844889193	-1.20879671834807\\
-2.83630900148409	-1.18052984170452\\
-2.88535963560319	-1.15329067060285\\
-2.92619505892335	-1.12708932719676\\
-2.96050096090471	-1.10186489456276\\
-2.98954936666697	-1.07751903751993\\
-3.01430876229406	-1.0539350695173\\
-3.03552236243188	-1.03098844046375\\
-3.05376401994576	-1.00855215247121\\
-3.06947838995699	-0.98649915687812\\
-3.08300991163942	-0.964702930102649\\
-3.09462374740807	-0.943036918843076\\
-3.10452083500761	-0.921373240822968\\
-3.11284852374633	-0.899580840330465\\
-3.11970778215795	-0.877523178040283\\
-3.12515761091966	-0.855055451989273\\
-3.12921702232672	-0.832021282451107\\
-3.13186471906918	-0.808248735438497\\
-3.13303639024817	-0.783545498449622\\
-3.13261931307637	-0.757692949906779\\
-3.13044367314647	-0.730438772495175\\
-3.1262696548114	-0.701487641376272\\
-3.11976885109067	-0.670489360837329\\
-3.11049781965361	-0.63702361653161\\
-3.09786054958415	-0.600580246600419\\
-3.08105502897536	-0.560533616143728\\
-3.05899676892009	-0.516109340945298\\
-3.03020872281921	-0.466341368002129\\
-2.99266220849364	-0.410017606742598\\
-2.94354713401342	-0.345613712643242\\
-2.87894316526232	-0.271219097624817\\
-2.7933613052015	-0.18447081327201\\
-2.67914274745981	-0.0825377845677898\\
-2.5257783825343	0.0377458731415323\\
-2.31943501649625	0.179398198627669\\
-2.04347928473513	0.344126308788451\\
-1.68159158956786	0.530151902243493\\
-1.22537153655025	0.729304947360204\\
-0.685671303026336	0.925396871033158\\
-0.099644329037568	1.09741544503237\\
0.478086993696043	1.22814867876394\\
0.998202340115935	1.3118123673045\\
1.43427748430264	1.35394217752062\\
1.78329816783529	1.36546589275169\\
};
\addplot [color=mycolor1, forget plot]
  table[row sep=crcr]{%
1.86527941841213	1.44825240370401\\
2.11771325155239	1.44056775786807\\
2.3146438691887	1.42201660055715\\
2.46888019254694	1.39752963749764\\
2.59071684335125	1.37015790542439\\
2.68800519087505	1.34170343400817\\
2.76659677043846	1.31318515783504\\
2.83082010235443	1.2851462271798\\
2.88388025783308	1.25784608835794\\
2.92816364116387	1.23137735726302\\
2.96546102639441	1.20573592840368\\
2.99712789566332	1.1808626349622\\
3.02419904593052	1.15666774659049\\
3.04747055975572	1.13304512390001\\
3.06755865595758	1.10988011611193\\
3.08494214443492	1.08705364341692\\
3.09999317660819	1.06444391896843\\
3.11299954624969	1.0419266711386\\
3.12418079062586	1.01937436463494\\
3.13369963737515	0.996654693924047\\
3.1416698415423	0.973628477145177\\
3.14816109038466	0.950146978784336\\
3.15320137109218	0.926048613223259\\
3.1567769610436	0.901154914561983\\
3.15882998113876	0.875265590354205\\
3.15925322167056	0.848152399268906\\
3.15788167619843	0.819551496607721\\
3.15447986339269	0.789153767778748\\
3.14872352711161	0.756592507909527\\
3.14017360532388	0.721427595068109\\
3.12823933821891	0.683125037112243\\
3.1121258845059	0.641030451681497\\
3.09075960873231	0.594334702650082\\
3.06268100836197	0.542029687526621\\
3.02589079683469	0.482852475934354\\
2.97762897175065	0.415217431606972\\
2.91406095861527	0.33714041809399\\
2.82984383017217	0.246170597184527\\
2.71756319068948	0.139371466952623\\
2.56710335408355	0.0134469599782818\\
2.36521648184592	-0.134794739196671\\
2.09600634147434	-0.307322133291899\\
1.74374612949788	-0.502780130865966\\
1.29973652733544	-0.713622372095883\\
0.772685574497584	-0.924268036294543\\
0.195840532604639	-1.1136404366881\\
-0.379672184980449	-1.26317811794792\\
-0.905127232671636	-1.36486458989047\\
-1.35185024182953	-1.42235333482624\\
-1.71378205463731	-1.44579970039503\\
-1.99931648626235	-1.44616912203778\\
-2.2222613254083	-1.43229003289289\\
-2.39639418110116	-1.41028844269828\\
-2.53330505257749	-1.38406879379923\\
-2.64202280717088	-1.35598825095052\\
-2.72933700426184	-1.32740987741486\\
-2.80028183599667	-1.29908456453855\\
-2.85858020554726	-1.27139510629733\\
-2.90699530226382	-1.24450638015938\\
-2.947592193925	-1.21845598805502\\
-2.98192717794727	-1.19320836075959\\
-3.01118336923205	-1.16868675508286\\
-3.03626759247698	-1.14479193034465\\
-3.05787979721232	-1.1214127859477\\
-3.07656301310694	-1.09843211952335\\
-3.09273946787627	-1.0757293913073\\
-3.10673677727398	-1.05318161540029\\
-3.11880691461837	-1.03066303560194\\
-3.12913982678066	-1.00804395896556\\
-3.13787297099832	-0.985188940451968\\
-3.14509762024053	-0.961954392812934\\
-3.15086246552017	-0.938185609944088\\
-3.15517478867469	-0.913713121998979\\
-3.15799925541525	-0.888348234484449\\
-3.1592541569643	-0.86187753192624\\
-3.15880468027373	-0.83405604094017\\
-3.1564524775992	-0.804598638814661\\
-3.1519203913931	-0.773169152187795\\
-3.14483060713148	-0.739366405369039\\
-3.13467366330692	-0.70270623935128\\
-3.12076451078705	-0.662598226770153\\
-3.1021799921693	-0.618315471439615\\
-3.07766944933406	-0.568955574380272\\
-3.04552637025706	-0.513390775284264\\
-3.00340387231053	-0.450205952538559\\
-2.94805082150653	-0.377625791225276\\
-2.87494099838897	-0.293439685493208\\
-2.77777324861214	-0.194950415666673\\
-2.64785882128024	-0.0790107997470756\\
-2.47353686337519	0.0577132524403142\\
-2.24007102783224	0.217997577923155\\
-1.93107839760373	0.402477340836931\\
-1.53320039607419	0.607063582008598\\
-1.04509831762917	0.820231299118718\\
-0.487511116244679	1.02294809819989\\
0.0954524566832367	1.19419249706947\\
0.651027569613964	1.32000441649981\\
1.13916833508337	1.39857409095482\\
1.54310927986971	1.43761596787514\\
1.86527941841213	1.44825240370401\\
};
\addplot [color=mycolor1, forget plot]
  table[row sep=crcr]{%
1.93943026851774	1.53163957203499\\
2.17360049581859	1.52450375594013\\
2.35755267889386	1.507167450353\\
2.50283805042945	1.48409467494661\\
2.61862701362222	1.45807589297092\\
2.711904720556	1.43078980113568\\
2.78789794972007	1.40321063191535\\
2.85049812106648	1.37587728963447\\
2.90260827294845	1.34906340471532\\
2.94640665985586	1.32288242012228\\
2.98354018983974	1.29735179089412\\
3.01526428573785	1.27243191584892\\
3.04254355050506	1.24804954958427\\
3.06612428546851	1.22411167389182\\
3.08658691232949	1.20051347002365\\
3.10438401945045	1.17714260332082\\
3.1198680486297	1.15388115911178\\
3.1333114246421	1.13060603329294\\
3.14492107221718	1.10718824793783\\
3.15484865623774	1.08349144998392\\
3.16319744069907	1.05936970978761\\
3.17002633210458	1.03466463528547\\
3.17535141046174	1.00920173680874\\
3.17914502232256	0.982785903580248\\
3.18133228639511	0.955195775051067\\
3.1817846142805	0.926176699198573\\
3.18030954356728	0.895431856345077\\
3.17663577468156	0.862610980826365\\
3.17039173647708	0.827295923087385\\
3.16107519214855	0.788982051656446\\
3.14801021191297	0.747054194608803\\
3.13028610628902	0.700755480999907\\
3.10667040258241	0.649147136556571\\
3.0754844031259	0.591057219627259\\
3.03442515971809	0.525016963546477\\
2.98031231722511	0.449186006034351\\
2.9087346617035	0.361274943673765\\
2.81357717269425	0.258490731375111\\
2.68644585098409	0.137567426068143\\
2.51612277515245	-0.00498383250365952\\
2.28846610535659	-0.172159159422162\\
1.98770377648844	-0.364935344947061\\
1.60066040195149	-0.579748947424175\\
1.1249587311468	-0.805731300108648\\
0.578529135762317	-1.02424182226127\\
0.0018151964207621	-1.21369394943156\\
-0.554595484228764	-1.35836326169271\\
-1.04995644620431	-1.4542802652993\\
-1.46493689603643	-1.50770529730677\\
-1.79936987895032	-1.52937259081181\\
-2.06360046551447	-1.52970881922359\\
-2.27107724781439	-1.51678493296196\\
-2.43440000580924	-1.4961417936104\\
-2.56393648952145	-1.47132809713904\\
-2.66771714564749	-1.44451759463789\\
-2.75179187549888	-1.41699534929173\\
-2.82067103513665	-1.38949139261048\\
-2.87771388935971	-1.36239547667343\\
-2.92543291381231	-1.33589101005423\\
-2.96572023314904	-1.3100372813353\\
-3.00001227802128	-1.28481949674121\\
-3.02940845809071	-1.26017900999828\\
-3.05475657477219	-1.23603138924932\\
-3.07671444666484	-1.21227698949568\\
-3.09579454834497	-1.18880686910994\\
-3.11239646055606	-1.16550577220273\\
-3.12683048862806	-1.14225321584285\\
-3.13933478547637	-1.11892329975785\\
-3.15008759391245	-1.09538359128941\\
-3.15921570692029	-1.07149326640162\\
-3.1667998656831	-1.04710056941051\\
-3.1728775236196	-1.02203956524038\\
-3.17744316287597	-0.996126082116127\\
-3.18044612691394	-0.969152667859541\\
-3.18178570066164	-0.940882299647012\\
-3.18130289768291	-0.911040485996277\\
-3.17876806450486	-0.879305271180267\\
-3.17386293502143	-0.845294485638946\\
-3.16615509087293	-0.808549370543012\\
-3.15506180294167	-0.768513432790178\\
-3.13979879576737	-0.724505062774383\\
-3.11930738664225	-0.675682109879071\\
-3.09215045321041	-0.620996384380717\\
-3.05636355479973	-0.559136266347788\\
-3.00924235488592	-0.488457035834185\\
-2.94704246831324	-0.406902965112206\\
-2.86456748527221	-0.311936416048813\\
-2.75463822885288	-0.200514624736791\\
-2.6075037101399	-0.0692071419161662\\
-2.41043964816916	0.0853601783129438\\
-2.14818441113122	0.265428168733131\\
-1.80548913225713	0.470070070247356\\
-1.37333584735421	0.692353225606642\\
-0.858513701023624	0.917297761136096\\
-0.29079596436584	1.12382307127971\\
0.281914664267367	1.29216766341901\\
0.811647335494608	1.41220223974685\\
1.26785920299175	1.48566882945716\\
1.64171802014129	1.52181328657332\\
1.93943026851773	1.53163957203499\\
};
\addplot [color=mycolor1, forget plot]
  table[row sep=crcr]{%
2.00696136401616	1.61515115761543\\
2.2244763751215	1.60851615796231\\
2.39650471451627	1.59229668680998\\
2.53345495378374	1.57054151686175\\
2.64351148367024	1.54580578256746\\
2.73290434414652	1.51965195360396\\
2.80631279325486	1.49300744753006\\
2.86724111811561	1.46640128633536\\
2.91832130806057	1.44011502928095\\
2.96154209634674	1.41427730154465\\
2.99841739547142	1.38892246172996\\
3.03010867183627	1.36402680232611\\
3.057513552894	1.3395307305017\\
3.08133005773032	1.31535217735229\\
3.10210330264744	1.29139447609956\\
3.12025956983251	1.26755070454325\\
3.1361311872701	1.24370571539386\\
3.14997463568489	1.219736596381\\
3.16198356202128	1.19551199641695\\
3.17229784946792	1.17089055449665\\
3.1810095042562	1.14571853095339\\
3.18816581973541	1.11982663858188\\
3.19377003078697	1.09302598566912\\
3.19777944566942	1.06510296096169\\
3.20010080960119	1.03581280137891\\
3.20058238477912	1.00487147697869\\
3.19900188865654	0.971945394114084\\
3.1950489674993	0.936638245959665\\
3.18830022702405	0.898474118681664\\
3.17818389819098	0.856875683725423\\
3.16392984492455	0.811135976447896\\
3.14449863349101	0.760381917347674\\
3.11848055397372	0.703527498527821\\
3.08395162400545	0.639214758512009\\
3.03826882809506	0.565742073837507\\
2.97778233785116	0.480983635635591\\
2.89744247597732	0.382314922117848\\
2.79029578776044	0.266583707551525\\
2.64692727178493	0.130216750300535\\
2.45507544953773	-0.0303578786756252\\
2.20001268640501	-0.217674749787857\\
1.86685064606414	-0.431254689935425\\
1.44620976602339	-0.664782761045597\\
0.943127699983683	-0.903873517789795\\
0.38427994838395	-1.12747132535379\\
-0.185353231066425	-1.31470988115095\\
-0.718599426479986	-1.45343541687645\\
-1.18333170544918	-1.54346315654748\\
-1.5681980712356	-1.59302554230365\\
-1.87733032808084	-1.61305360077976\\
-2.12214127410076	-1.61335939866085\\
-2.31547278170568	-1.60130961030503\\
-2.46880280663759	-1.58192293699056\\
-2.59141439882153	-1.55843016806524\\
-2.69046679613017	-1.53283655796367\\
-2.77136399372475	-1.50635073996917\\
-2.83815504437088	-1.47967750060326\\
-2.89387509055317	-1.45320736984484\\
-2.94081000509277	-1.42713625155903\\
-2.98069325725668	-1.40153995435671\\
-3.01484960365108	-1.37642030837201\\
-3.04429922644163	-1.35173353392588\\
-3.06983315014823	-1.32740752815835\\
-3.09206798947786	-1.3033521953867\\
-3.11148582640992	-1.27946536477639\\
-3.12846332614156	-1.25563585908362\\
-3.14329298033504	-1.2317446693827\\
-3.15619849444608	-1.20766480801879\\
-3.16734571270759	-1.18326016620529\\
-3.17685002186825	-1.15838353860248\\
-3.18478083537304	-1.13287386037785\\
-3.19116349026798	-1.10655261035256\\
-3.19597865600009	-1.07921925153787\\
-3.19915912844725	-1.05064549600183\\
-3.20058363534984	-1.0205680844799\\
-3.20006697823932	-0.988679652526426\\
-3.19734543923824	-0.954617103743766\\
-3.19205583125112	-0.917946715583595\\
-3.18370578514848	-0.878144954382671\\
-3.17163173039152	-0.8345736703227\\
-3.15493937296599	-0.786447997426901\\
-3.13241909538237	-0.732794970696447\\
-3.10242537324393	-0.672400795585811\\
-3.06270493050221	-0.603745367988815\\
-3.01015344216256	-0.524925208081585\\
-2.9404774856664	-0.433572977477883\\
-2.84774445136671	-0.326798395754649\\
-2.7238374474047	-0.201211234613841\\
-2.55793839316163	-0.0531560686324172\\
-2.33641919046957	0.120602410235937\\
-2.04400323109662	0.321404499629473\\
-1.66760176663596	0.546225264813016\\
-1.20384676724017	0.784847999392526\\
-0.66815026467459	1.01902764897822\\
-0.0976764318487073	1.22667643555794\\
0.459022871868458	1.3904123014413\\
0.960698937316442	1.50414893360095\\
1.38575196981399	1.57262394183115\\
1.73160305051423	1.60606634792779\\
2.00696136401616	1.61515115761543\\
};
\addplot [color=mycolor1, forget plot]
  table[row sep=crcr]{%
2.06860660762213	1.6982564075881\\
2.27083390287344	1.69208158484801\\
2.43181623095735	1.67689744800409\\
2.56093236612886	1.65638138520399\\
2.66550083508625	1.63287461855828\\
2.75109090401043	1.60782966645149\\
2.82189989396105	1.58212559838171\\
2.88108778101375	1.55627688899532\\
2.93104242025631	1.53056767567032\\
2.9735794665914	1.50513681595202\\
3.01008960653438	1.48003139638351\\
3.04164595097124	1.45524024098408\\
3.06908218678636	1.43071476405002\\
3.09304952831468	1.4063817792773\\
3.11405833439906	1.3821511459237\\
3.13250858656795	1.35792004598616\\
3.14871219749462	1.3335750035248\\
3.16290923474801	1.30899232412895\\
3.17527950851923	1.28403735238804\\
3.18595050823006	1.25856275768579\\
3.19500232471316	1.23240592553061\\
3.20246991898054	1.20538542817751\\
3.20834286164587	1.17729645745394\\
3.21256243981641	1.14790501164691\\
3.21501578237673	1.11694052610887\\
3.21552635813266	1.08408651333649\\
3.21383981384117	1.04896862160267\\
3.20960358553135	1.01113932017236\\
3.20233795866104	0.970058163217434\\
3.19139515971534	0.925066269822954\\
3.17590148070908	0.875353300549936\\
3.15467517274135	0.819914882186446\\
3.12610969095244	0.757498330656529\\
3.08800775922697	0.686535139968228\\
3.03734714023266	0.605061175825374\\
2.96995616008087	0.510632303323317\\
2.8800827372595	0.400259326796594\\
2.75987252483657	0.270420895487558\\
2.59886936571566	0.117280023737274\\
2.38388581270329	-0.0626631847495668\\
2.10003245751582	-0.271146178427355\\
1.73420453571025	-0.505713869055155\\
1.28204877937458	-0.756817313998955\\
0.756667781044695	-1.00661495204683\\
0.1922733449297	-1.23255116919391\\
-0.364437061939789	-1.41564239051156\\
-0.871800672737549	-1.54769918540689\\
-1.30616008159199	-1.63187423596462\\
-1.66268725551021	-1.67779647802568\\
-1.94852783530275	-1.69631403914339\\
-2.17554581779588	-1.69659209740831\\
-2.35584395382825	-1.6853483502887\\
-2.49985475124989	-1.66713418663842\\
-2.61590047339757	-1.64489452404878\\
-2.71037792764455	-1.62047891864067\\
-2.78812467574338	-1.59502118880521\\
-2.85278177135636	-1.56919735303032\\
-2.90709431663539	-1.54339350921435\\
-2.95314268875547	-1.51781282664167\\
-2.99251442364579	-1.49254305174153\\
-3.02643004080702	-1.46759887639381\\
-3.0558346556689	-1.44294839981031\\
-3.08146467381201	-1.41852950986356\\
-3.10389645791012	-1.39425983034538\\
-3.12358193679098	-1.37004250838304\\
-3.14087468969792	-1.34576925536863\\
-3.15604899636746	-1.32132151185452\\
-3.16931359396485	-1.2965702596509\\
-3.18082134004983	-1.2713747764242\\
-3.19067558088456	-1.24558047154152\\
-3.19893371714453	-1.21901582613657\\
-3.20560820661074	-1.19148836502251\\
-3.21066501481064	-1.16277949856979\\
-3.21401929166534	-1.13263797735477\\
-3.21552778542046	-1.10077159067195\\
-3.21497716899955	-1.06683660114395\\
-3.21206700150163	-1.03042423025091\\
-3.20638541338284	-0.991043282215236\\
-3.1973746947755	-0.948097708047462\\
-3.18428265137816	-0.900857571416836\\
-3.16609369567306	-0.848421520785672\\
-3.14143095431247	-0.789668619176481\\
-3.10841702795366	-0.723197550674446\\
-3.06447656286716	-0.647252563797837\\
-3.00605961778449	-0.559639709938102\\
-2.9282649168792	-0.457647549503845\\
-2.82435759990059	-0.338010483311221\\
-2.68523395877069	-0.197001944847137\\
-2.49904103887176	-0.0308321456585343\\
-2.25149282157391	0.163357481220727\\
-1.92794704359109	0.385569683479203\\
-1.5185982643987	0.630131842732555\\
-1.02683517234253	0.883261618608591\\
-0.476530090381141	1.12394512252728\\
0.0899721270408058	1.33026131171234\\
0.626255381738577	1.48807579992018\\
1.09878526349904	1.59525133632222\\
1.49389074129826	1.65892050951114\\
1.81374487298738	1.68985191952004\\
2.06860660762213	1.6982564075881\\
};
\addplot [color=mycolor1, forget plot]
  table[row sep=crcr]{%
2.12479054205496	1.78038134069434\\
2.31291656161758	1.77463153108908\\
2.46360610366426	1.76041283002961\\
2.58531390643501	1.7410691972296\\
2.68459681276724	1.71874663610339\\
2.76644398208781	1.69479360268786\\
2.83462713558066	1.67003996425573\\
2.8919991948091	1.64498193509097\\
2.94072770888523	1.61990174728722\\
2.98247012200416	1.59494421232009\\
3.01850295537561	1.57016544374821\\
3.04981632875234	1.54556375037565\\
3.0771830307112	1.52109910765688\\
3.10120907214989	1.49670526622932\\
3.12237077455906	1.47229705683084\\
3.1410420104021	1.44777449975262\\
3.15751415986059	1.42302472183765\\
3.17201058532855	1.39792229432523\\
3.18469687103318	1.37232834788918\\
3.19568766556411	1.34608864468236\\
3.20505065072055	1.31903065742965\\
3.21280790299115	1.29095960015476\\
3.21893468318756	1.2616532579701\\
3.22335545718919	1.23085536183297\\
3.22593668726858	1.1982671369072\\
3.22647560444775	1.16353650877242\\
3.22468373243725	1.12624426815375\\
3.22016332006521	1.08588625994957\\
3.2123739627508	1.04185036683857\\
3.20058542991375	0.993386703785246\\
3.18381089780502	0.939569061922303\\
3.16071221931497	0.87924535425997\\
3.12946538640294	0.810974926595604\\
3.08757007304783	0.732951832674429\\
3.03158313835534	0.642917163875137\\
2.95675589737613	0.538073721361076\\
2.85656905904271	0.415039483078827\\
2.72221216084583	0.26992395087485\\
2.54219661269735	0.0986971519466855\\
2.30259833799656	-0.101860489881717\\
1.988914542344	-0.332282706047911\\
1.590833038462	-0.587588465309065\\
1.11013273152794	-0.854634144926016\\
0.568083887970402	-1.11247084944338\\
0.00463629186577767	-1.33814162830174\\
-0.53445543917634	-1.51552597345371\\
-1.01432355133255	-1.6404783228451\\
-1.41910545379632	-1.71894465365314\\
-1.74911013215975	-1.76145651305498\\
-2.01349136861769	-1.7785815895257\\
-2.22414161555226	-1.778834444578\\
-2.39236488665411	-1.76833812291586\\
-2.52763190541574	-1.75122476343541\\
-2.637414179359	-1.73018111639456\\
-2.72743941803059	-1.70691245766884\\
-2.80204676583525	-1.68247968465132\\
-2.86451532605169	-1.65752740762065\\
-2.91733005804378	-1.63243301732164\\
-2.96238471945608	-1.6074024926176\\
-3.00113255564112	-1.58253150520725\\
-3.03469685032081	-1.55784423610873\\
-3.06395171936462	-1.53331792796646\\
-3.08958118440682	-1.50889827723035\\
-3.11212246190421	-1.48450889048134\\
-3.13199774897241	-1.46005683471198\\
-3.14953755486476	-1.43543555342924\\
-3.16499773010165	-1.41052593585689\\
-3.17857169594814	-1.38519601128781\\
-3.19039890215735	-1.35929952905157\\
-3.20057018425654	-1.33267353477869\\
-3.20913041012869	-1.30513493834513\\
-3.21607856529649	-1.27647596930937\\
-3.22136519811773	-1.24645831777657\\
-3.22488690195018	-1.21480565062894\\
-3.22647721994893	-1.18119406369038\\
-3.22589298028387	-1.14523986795464\\
-3.22279455206183	-1.10648390022672\\
-3.21671778036694	-1.06437128407666\\
-3.20703430696839	-1.01822524069456\\
-3.19289546606382	-0.967213175566549\\
-3.17315277669444	-0.910302912228278\\
-3.14624504277518	-0.846206799830856\\
-3.11003815033464	-0.773311962456872\\
-3.06159927613152	-0.689597282062443\\
-2.99688444241334	-0.592544204307182\\
-2.91032342357126	-0.479064016328777\\
-2.79431509370997	-0.345497907382085\\
-2.63873561239191	-0.187811386672545\\
-2.43077687368962	-0.0022104490946153\\
-2.15584030237021	0.213483429925804\\
-1.80070129893057	0.457436150053204\\
-1.35998856707431	0.720810101212633\\
-0.844619331415884	0.986195276047295\\
-0.286098770515454	1.23058945444942\\
0.270545939309583	1.43341860165406\\
0.783237520117522	1.58436045908561\\
1.22637739195548	1.68490620595368\\
1.5930049503019	1.74399940268459\\
1.88877271927099	1.77260242882456\\
2.12479054205496	1.78038134069434\\
};
\addplot [color=mycolor1, forget plot]
  table[row sep=crcr]{%
2.1757455399383	1.86092947588109\\
2.35081968343383	1.85557361347775\\
2.4918814069607	1.84225863208274\\
2.60655824336578	1.8240283222734\\
2.70073573799555	1.80285014400824\\
2.77889236357058	1.77997422491894\\
2.84442307195399	1.75618106444592\\
2.89990649664374	1.73194580475101\\
2.94731124659916	1.70754511313093\\
2.98815020451916	1.68312612891192\\
3.02359428704186	1.6587507692563\\
3.0545558807579	1.63442411379075\\
3.08175001423489	1.61011247806171\\
3.10573928683399	1.58575475408463\\
3.1269669296582	1.56126929111975\\
3.14578113135792	1.5365577545954\\
3.1624528498072	1.51150686354687\\
3.17718866730384	1.48598855560677\\
3.19013976072305	1.45985889229939\\
3.20140769291341	1.43295585051682\\
3.21104744439648	1.40509601867434\\
3.21906786098303	1.37607010761512\\
3.22542946417315	1.3456370815782\\
3.23003932927788	1.31351660071788\\
3.23274245037814	1.27937933174952\\
3.23330864339078	1.2428345148618\\
3.23141353732528	1.20341395994266\\
3.22661149742104	1.16055137087074\\
3.21829731089289	1.11355555601054\\
3.20565200704221	1.06157568726971\\
3.18756610020821	1.00355637933419\\
3.16253064865351	0.938180154314155\\
3.12848273803986	0.86379528743524\\
3.08258772403524	0.778329148142527\\
3.02093765895846	0.679193262511462\\
2.93814948064986	0.563201169264176\\
2.82687253540112	0.426552585933946\\
2.6772961224536	0.265001045394384\\
2.4769450478802	0.0744276433387469\\
2.21143140255826	-0.147838875464467\\
1.86731172809882	-0.400655940343124\\
1.43813288752117	-0.675973742953314\\
0.932668968336354	-0.956875982837009\\
0.379868962039353	-1.21994253881364\\
-0.176755743005223	-1.44298613217454\\
-0.69465073899282	-1.61347070951245\\
-1.1462983201646	-1.73111613176759\\
-1.52265813963826	-1.80408987649808\\
-1.82793983029911	-1.84342025116485\\
-2.07253666415207	-1.85926110979957\\
-2.26808650642126	-1.85949111268943\\
-2.42508096947947	-1.84969048308533\\
-2.55211304014538	-1.83361451173693\\
-2.65590054705101	-1.81371615279375\\
-2.74158252091892	-1.79156687627748\\
-2.81305835960616	-1.7681569245645\\
-2.87328539301801	-1.7440977085159\\
-2.92451497364393	-1.71975452013379\\
-2.96847135624044	-1.69533243440478\\
-3.0064843306294	-1.67093159810833\\
-3.03958667374776	-1.64658270937511\\
-3.06858558161	-1.62226969316863\\
-3.09411507998217	-1.59794405612958\\
-3.11667455875612	-1.57353377393765\\
-3.13665713747485	-1.54894852044796\\
-3.15437050281795	-1.52408237866027\\
-3.17005208114538	-1.49881474013717\\
-3.18387984234486	-1.47300981269993\\
-3.19597961166421	-1.44651495919136\\
-3.20642944476207	-1.41915794600328\\
-3.21526135973226	-1.39074306432466\\
-3.22246048722934	-1.36104598217436\\
-3.2279614680494	-1.32980707750475\\
-3.23164166782781	-1.29672287980269\\
-3.23331045712067	-1.26143509754435\\
-3.23269337812526	-1.22351651876016\\
-3.22940942610948	-1.18245282867667\\
-3.22293882882447	-1.1376190811391\\
-3.21257749123535	-1.08824918889927\\
-3.19737252597229	-1.03339639195026\\
-3.17603082171242	-0.971882330094916\\
-3.14678925483359	-0.902232374478002\\
-3.10723101369695	-0.8225959581938\\
-3.054028521735	-0.730654356005426\\
-2.98259295602703	-0.62352803360588\\
-2.88662269817132	-0.497717904034039\\
-2.75759083133511	-0.349160982008038\\
-2.58434167608116	-0.173565222105404\\
-2.35324979693792	0.0326908767820533\\
-2.04985818817793	0.270734018763883\\
-1.66325731093179	0.536350555225738\\
-1.19359855779719	0.817107480987676\\
-0.659653834827181	1.09216856819398\\
-0.0991467107972801	1.3375460775094\\
0.442715382529131	1.53508043637685\\
0.929717232907285	1.67851643798467\\
1.34384825322769	1.77250800435862\\
1.68360721994237	1.82727950637311\\
1.95708707666518	1.85372676041588\\
2.1757455399383	1.86092947588109\\
};
\addplot [color=mycolor1, forget plot]
  table[row sep=crcr]{%
2.22159676622086	1.93930856982817\\
2.38456348915545	1.93431869239652\\
2.51659941474423	1.92185152071188\\
2.62459243618175	1.90468011902611\\
2.71383537803312	1.88460851340267\\
2.7883560991336	1.86279418799137\\
2.85121539348207	1.83996875348906\\
2.90474663628562	1.81658429638293\\
2.95073880373225	1.79290902967154\\
2.99057285119828	1.76908942219381\\
3.02532224916302	1.7451904488079\\
3.05582684850611	1.72122159542384\\
3.08274717489659	1.69715354787979\\
3.10660441249865	1.67292872537141\\
3.127809887305	1.64846767764792\\
3.14668677481328	1.62367262888872\\
3.16348596110748	1.59842897211503\\
3.17839740498815	1.57260520022359\\
3.19155791838624	1.54605154151156\\
3.20305595341838	1.51859740878168\\
3.21293371855797	1.49004764492252\\
3.22118671215024	1.4601774351994\\
3.227760530906	1.42872564251366\\
3.23254455564754	1.39538619351755\\
3.23536180344011	1.35979698777029\\
3.23595382175374	1.32152560535341\\
3.23395892799329	1.28005083611285\\
3.22888128420597	1.23473873318586\\
3.22004712651428	1.18481150089176\\
3.2065427863991	1.12930708628236\\
3.18712675833704	1.06702695026784\\
3.1601048199004	0.996469414928267\\
3.12315314447153	0.915746872100953\\
3.07307027181208	0.822488485651201\\
3.0054376929282	0.713739031638763\\
2.91417909926155	0.585885607914834\\
2.7910507029328	0.434688447808658\\
2.62521279462519	0.255575061560006\\
2.4032948862381	0.044481007647173\\
2.11081059009323	-0.200384751809641\\
1.73616153777326	-0.475677649798832\\
1.27780703965718	-0.769790461836722\\
0.752028056882593	-1.06208929013006\\
0.194399578661223	-1.32756797160026\\
-0.350316856720065	-1.54593554218786\\
-0.844475622829684	-1.70866935710458\\
-1.26787394401055	-1.81898888345277\\
-1.61719807411477	-1.88673288114333\\
-1.89950611666071	-1.92310517359197\\
-2.12585585919881	-1.93776143897764\\
-2.30744776846662	-1.937970749563\\
-2.453976108978	-1.92881915090876\\
-2.57323699998103	-1.91372272803877\\
-2.67128003518321	-1.89492236072458\\
-2.75272515508203	-1.87386549694145\\
-2.82108293983526	-1.85147437242084\\
-2.87902424937034	-1.82832618816331\\
-2.92859056026966	-1.80477156292672\\
-2.97135227022156	-1.78101165755016\\
-3.00852588244833	-1.75714819470257\\
-3.04106019066816	-1.73321582747918\\
-3.06969959513531	-1.70920299863323\\
-3.09503068404653	-1.68506523966372\\
-3.11751656635825	-1.66073343720329\\
-3.13752218229105	-1.63611867783153\\
-3.1553328871236	-1.61111468952351\\
-3.17116792414091	-1.58559850903179\\
-3.1851899036903	-1.55942974263533\\
-3.19751103107226	-1.53244860321826\\
-3.20819653289102	-1.5044727669215\\
-3.21726548504536	-1.47529297523436\\
-3.22468901655486	-1.44466719677413\\
-3.2303856241741	-1.41231304342215\\
-3.23421305309102	-1.37789799510356\\
-3.235955841838	-1.34102681304299\\
-3.23530714524298	-1.30122529861235\\
-3.2318427684834	-1.25791926980496\\
-3.22498437026108	-1.21040727021301\\
-3.2139473887235	-1.15782510271635\\
-3.1976672355856	-1.09909984365504\\
-3.17469450283523	-1.03289070644772\\
-3.14304623510161	-0.957514408594192\\
-3.09999606227808	-0.870854547522102\\
-3.04178279133616	-0.770260111763278\\
-2.96322004042956	-0.652452227928976\\
-2.85721209015153	-0.513489278488834\\
-2.71425395122918	-0.348902459295057\\
-2.52217422497574	-0.15421940103064\\
-2.26673597004074	0.0737793821462952\\
-1.93418981696836	0.334730225925443\\
-1.51691609518709	0.621483026274706\\
-1.02151355087482	0.917723764851995\\
-0.474421557665009	1.19967161875598\\
0.0822682578220813	1.4434858734613\\
0.60542847324188	1.63428244591121\\
1.06555720769938	1.76985019514753\\
1.45151187150373	1.85746785386418\\
1.76607348285571	1.908183370787\\
2.0189513713455	1.93263715122637\\
2.22159676622086	1.93930856982817\\
};
\addplot [color=mycolor1, forget plot]
  table[row sep=crcr]{%
2.26242433708805	2.01495975701713\\
2.41414617789258	2.01031029942162\\
2.53771272321917	1.99863920907641\\
2.6393508791053	1.9824750944135\\
2.72382923025413	1.96347238228063\\
2.79477724470694	1.94270157923801\\
2.8549589261095	1.92084645526483\\
2.90648822956053	1.89833479687986\\
2.95099201579987	1.8754241778549\\
2.98973104374063	1.85225799090357\\
3.02368912523749	1.82890199066744\\
3.05363870720377	1.80536806266584\\
3.0801891695655	1.78162956416864\\
3.10382246092602	1.75763103475301\\
3.12491940707309	1.73329407085491\\
3.14377907181281	1.70852050670912\\
3.16063284991186	1.68319361541998\\
3.1756544582572	1.65717775520509\\
3.18896660690415	1.63031668309961\\
3.20064483218905	1.60243060602038\\
3.21071872453098	1.57331191267627\\
3.2191705545088	1.54271941140635\\
3.22593106467386	1.51037077388933\\
3.23087192170737	1.47593273907774\\
3.23379397786403	1.43900845148347\\
3.23441002415314	1.39912107769105\\
3.23232006341119	1.35569254873925\\
3.22697619457315	1.30801590090295\\
3.21763284831761	1.25521923424936\\
3.20327617608307	1.19621881822066\\
3.18252367432787	1.12965848973131\\
3.15348149758124	1.05383259621917\\
3.11354260866627	0.966591258648163\\
3.05910531395786	0.865231748365437\\
2.98519334228269	0.746392684964905\\
2.88497752433065	0.605997106085439\\
2.74926341418225	0.43935036706806\\
2.56617333589524	0.241604392998829\\
2.32158625195492	0.00893748363705662\\
2.0013801507348	-0.259164636303382\\
1.59667626378294	-0.556598110954975\\
1.11180299427944	-0.867816262874491\\
0.570628563809619	-1.16878568287357\\
0.0138354865941234	-1.43397669740479\\
-0.51477784008836	-1.64597382440369\\
-0.983585324057899	-1.80041115102751\\
-1.37923667624514	-1.90352622692107\\
-1.70304879787909	-1.96633146823376\\
-1.96406395146819	-1.9999609332631\\
-2.17358366620682	-2.01352459856801\\
-2.3422597299364	-2.01371519107483\\
-2.47902161586027	-2.00516977239792\\
-2.59094452144229	-1.99099881414488\\
-2.68348498162809	-1.97325066716072\\
-2.76080429913143	-1.95325798961672\\
-2.82606866053358	-1.93187800249144\\
-2.88169362637296	-1.90965337745034\\
-2.92953211223144	-1.88691822322425\\
-2.97101501167211	-1.86386742935304\\
-3.00725510852754	-1.84060192483462\\
-3.03912354064474	-1.81715816586222\\
-3.06730606704634	-1.79352725982213\\
-3.09234454601841	-1.76966721413769\\
-3.11466755692612	-1.7455105517505\\
-3.13461298630067	-1.72096872693834\\
-3.15244458048663	-1.6959342476818\\
-3.16836386841219	-1.67028106039175\\
-3.18251841524694	-1.64386351247528\\
-3.19500703061484	-1.61651403425936\\
-3.20588228444575	-1.58803954494798\\
-3.21515044773085	-1.55821646686304\\
-3.22276874634791	-1.52678411218415\\
-3.22863956562872	-1.49343607275455\\
-3.23260093919311	-1.45780908245487\\
-3.23441225649563	-1.41946861840475\\
-3.23373357302114	-1.37789024595265\\
-3.23009612544867	-1.33243537787069\\
-3.22286052918501	-1.28231970232306\\
-3.21115751492079	-1.22657205308672\\
-3.19380375532933	-1.1639810301653\\
-3.16918216277058	-1.09302646845147\\
-3.13507199916185	-1.01179349857178\\
-3.08840988556377	-0.917869845171345\\
-3.02496091257697	-0.808235214543548\\
-2.93888704748241	-0.679171346043859\\
-2.82223666826812	-0.526263839427488\\
-2.66448461949251	-0.344648301373574\\
-2.45249402739633	-0.129781031181872\\
-2.17169849900763	0.120868484719446\\
-1.80972935627272	0.404948643356206\\
-1.3632561340928	0.711840294406582\\
-0.84598563937002	1.02126035664147\\
-0.291314470632887	1.30722853904193\\
0.256386012127547	1.54720538202523\\
0.757911370961256	1.73018016716924\\
1.19073558268672	1.85774057613982\\
1.54966147374486	1.93923803701874\\
1.84070705774828	1.98616599240323\\
2.07456070878453	2.00877850877765\\
2.26242433708805	2.01495975701713\\
};
\addplot [color=mycolor1, forget plot]
  table[row sep=crcr]{%
2.29830749724921	2.08738533347698\\
2.4395811740186	2.08305263854005\\
2.55520061617609	2.07212901265497\\
2.65080211567494	2.05692218109676\\
2.73068984585204	2.03894971670617\\
2.79814044812796	2.01920078004988\\
2.85565401536398	1.99831282384068\\
2.90514813770187	1.97668873039448\\
2.94810347329143	1.9545738782152\\
2.98567147129506	1.9321067243545\\
3.01875369636157	1.90935197166206\\
3.04806022379772	1.88632225308333\\
3.07415270212239	1.86299217817905\\
3.09747616469615	1.83930722377463\\
3.11838252328554	1.81518906403112\\
3.13714782939007	1.79053835531842\\
3.15398476936351	1.7652356059088\\
3.16905140194706	1.73914049674307\\
3.1824568002718	1.71208982970108\\
3.1942639845052	1.68389413204003\\
3.20449029378011	1.65433281748077\\
3.21310511863907	1.62314767854759\\
3.22002467010205	1.59003434623226\\
3.22510316709981	1.55463118714243\\
3.22811944042731	1.51650489916663\\
3.22875742400764	1.47513179685374\\
3.22657825605299	1.42987342926179\\
3.22098063445581	1.37994473301228\\
3.2111445119643	1.32437239859616\\
3.19595098563718	1.2619405822732\\
3.17386813346458	1.19112073535739\\
3.14278851341989	1.10998268755293\\
3.09979955470441	1.01608649672324\\
3.04086530125618	0.906361827493813\\
2.96040347532639	0.776999705219256\\
2.85077245154363	0.623421613075426\\
2.70177576333412	0.440470155704622\\
2.50051308750921	0.223096331657488\\
2.2323179458462	-0.0320410681959813\\
1.88399246213748	-0.323721965763938\\
1.45030338557161	-0.642524096561426\\
0.942220642271537	-0.968737361939567\\
0.390812497285217	-1.2755120554387\\
-0.159963629894488	-1.53794078881014\\
-0.669198713934636	-1.74224053078579\\
-1.11183040302902	-1.88809740401201\\
-1.48062795069045	-1.98423320154642\\
-1.78052030750788	-2.04240516957082\\
-2.02184432018373	-2.0734973663325\\
-2.21584486452425	-2.08605362771345\\
-2.37256441865051	-2.08622730090106\\
-2.50021033879442	-2.0782481673088\\
-2.60520716935758	-2.0649511812546\\
-2.69248456073879	-2.04820986400352\\
-2.76579786844252	-2.02925083248949\\
-2.82800777561613	-2.00886956147172\\
-2.88130214750432	-1.98757446254011\\
-2.92736452028104	-1.96568194613657\\
-2.96749945605449	-1.94337886547424\\
-3.00272500583715	-1.9207634826045\\
-3.03384076001914	-1.8978723035551\\
-3.06147798141842	-1.87469756194609\\
-3.08613661192687	-1.85119844159636\\
-3.10821261761388	-1.82730802935857\\
-3.128018148681	-1.80293727330643\\
-3.14579626615936	-1.77797674994651\\
-3.16173145589966	-1.75229672683094\\
-3.17595675407387	-1.72574578494571\\
-3.18855800163099	-1.69814809973293\\
-3.19957549213643	-1.6692993440335\\
-3.2090030483687	-1.63896105116648\\
-3.21678433039166	-1.60685314597231\\
-3.22280591241952	-1.57264420130932\\
-3.2268863325603	-1.53593879205867\\
-3.22875987206479	-1.4962610814842\\
-3.22805319432644	-1.45303346778393\\
-3.22425207670579	-1.40554872534058\\
-3.21665417134161	-1.35293358966853\\
-3.20430186210206	-1.29410118586178\\
-3.18588664028215	-1.22768920689045\\
-3.15961284078826	-1.15198064948813\\
-3.12300419233677	-1.06480505265109\\
-3.07263255395004	-0.963422490360464\\
-3.00374830420568	-0.844404217321268\\
-2.90980681553143	-0.703551162898204\\
-2.78194086289829	-0.535948802319476\\
-2.60857622479041	-0.336361891275331\\
-2.37570095198536	-0.100320780841988\\
-2.06878304888541	0.173669299639037\\
-1.67760012151999	0.480726745207704\\
-1.20406911546428	0.80630008205186\\
-0.669318227607618	1.12628547167269\\
-0.112520488089901	1.41346196908147\\
0.421789369974737	1.64766180912662\\
0.899662855971952	1.82206522437116\\
1.3053487755271	1.94165717647392\\
1.63860176170692	2.01733702594874\\
1.90778748090552	2.06074259319959\\
2.12409187880228	2.08165636290458\\
2.29830749724921	2.08738533347698\\
};
\addplot [color=mycolor1, forget plot]
  table[row sep=crcr]{%
2.32935350426214	2.15617173173143\\
2.46092075586192	2.15213372489249\\
2.56908832940956	2.14191139927175\\
2.65896471538462	2.12761282504564\\
2.73444205483902	2.1106304941265\\
2.79848402521174	2.09187777804395\\
2.85335580032668	2.07194771005019\\
2.90079919596766	2.05121820007574\\
2.94216311250622	2.02992142129865\\
2.97849976370664	2.00818952500258\\
3.01063548002523	1.98608474349786\\
3.03922285172473	1.96361914362191\\
3.06477921334039	1.94076744429417\\
3.08771509049744	1.91747510467296\\
3.10835519914207	1.89366310208727\\
3.12695383187255	1.86923030050607\\
3.14370591329698	1.84405396201361\\
3.15875459586971	1.81798871118315\\
3.17219595232114	1.79086408294747\\
3.18408106369956	1.76248063963401\\
3.19441557281967	1.7326045112283\\
3.20315654396047	1.70096007769185\\
3.21020621229256	1.66722035772634\\
3.21540188668415	1.63099447862448\\
3.21850084250327	1.59181135878443\\
3.21915844446517	1.54909841790525\\
3.21689688452632	1.50215371941553\\
3.21106068009092	1.45010943213044\\
3.20075328108649	1.39188388794337\\
3.18474656782996	1.32611890190494\\
3.16135148736598	1.25109869757437\\
3.12823359553641	1.16464748759466\\
3.08215268206203	1.06400625588533\\
3.01860416618456	0.945699465708896\\
2.93135083717691	0.805427264160013\\
2.81187975223788	0.638072770612336\\
2.64894909567976	0.438017195115797\\
2.42867976698624	0.200113082811465\\
2.13613065485561	-0.0782098645709827\\
1.75967758187403	-0.393487132995316\\
1.29866158436661	-0.732453427227594\\
0.771202537901096	-1.07121256327827\\
0.214728980558727	-1.38091874404249\\
-0.325494961838313	-1.6384149650196\\
-0.812965102505627	-1.83404646195004\\
-1.22924624190243	-1.97125460517688\\
-1.57235707714977	-2.06070931296474\\
-1.84993959636118	-2.11455784134355\\
-2.07308909793711	-2.14330769833361\\
-2.25278582519434	-2.1549355994704\\
-2.39843774449618	-2.15509398815662\\
-2.51757797964365	-2.14764365112562\\
-2.61604479854629	-2.13517105857273\\
-2.6982992584146	-2.11939098848893\\
-2.76773681442649	-2.1014323062276\\
-2.82694677743785	-2.08203220493066\\
-2.87791380142478	-2.06166558681209\\
-2.92216930744609	-2.04063052932208\\
-2.96090359475513	-2.01910460671347\\
-2.99504838736969	-1.9971819926349\\
-3.02533758060229	-1.97489786432459\\
-3.05235202489591	-1.95224434892787\\
-3.07655261064923	-1.92918075751268\\
-3.09830472024621	-1.90563987805827\\
-3.11789623002166	-1.88153146092839\\
-3.13555059924298	-1.85674360684358\\
-3.15143610824056	-1.83114247836945\\
-3.16567195008156	-1.80457054930332\\
-3.1783315980804	-1.77684344724499\\
-3.1894436318143	-1.74774530867145\\
-3.19898997864918	-1.71702243439424\\
-3.20690128843532	-1.68437489038377\\
-3.21304887521986	-1.64944552896162\\
-3.21723229289842	-1.61180569135058\\
-3.21916110914594	-1.57093657538571\\
-3.21842872934844	-1.52620489105142\\
-3.21447509338645	-1.47683096313165\\
-3.20653357442546	-1.4218468714199\\
-3.19355525520003	-1.36004158880333\\
-3.17410072549746	-1.28988955477375\\
-3.14618550490828	-1.20945917541377\\
-3.10706046072684	-1.11629952448398\\
-3.05290489374579	-1.00730967183723\\
-2.978412885197	-0.878611223092022\\
-2.87627789570386	-0.72548180564268\\
-2.73666064701904	-0.542483349892518\\
-2.54692508885849	-0.324051637474437\\
-2.29232017876867	-0.0659769457238991\\
-1.9587960087572	0.231793697508506\\
-1.53911030580676	0.561283420912842\\
-1.04127224491097	0.903661418785349\\
-0.493744333777135	1.23140511216232\\
0.0600689388895753	1.51714943946215\\
0.577425408082559	1.74399826308549\\
1.03044074659414	1.90937750121675\\
1.40961161396351	2.02117636793413\\
1.71867229167716	2.09137080250741\\
1.96760462743302	2.13151173759376\\
2.16773707616045	2.15085998477536\\
2.32935350426214	2.15617173173143\\
};
\addplot [color=mycolor1, forget plot]
  table[row sep=crcr]{%
2.35571328277463	2.22100511820195\\
2.47826782286047	2.21724110275804\\
2.57945563798566	2.2076760066786\\
2.66391338264297	2.19423739652515\\
2.73516709681387	2.17820355908947\\
2.79590243487287	2.16041750763504\\
2.84817524353824	2.1414300272445\\
2.89356994912398	2.12159441970805\\
2.93331677967345	2.10112910266619\\
2.96837799604573	2.08015897733583\\
2.99951128269362	2.05874275651048\\
3.02731643733883	2.03689093924234\\
3.05226984512698	2.01457747210173\\
3.07474996249443	1.99174706294535\\
3.09505610683312	1.96831941038495\\
3.11342217070063	1.94419114633847\\
3.13002638320828	1.91923597242133\\
3.14499787010008	1.89330324619182\\
3.1584204744478	1.86621510225032\\
3.17033405755072	1.83776204934554\\
3.18073327556166	1.8076968479169\\
3.18956359405484	1.77572632588045\\
3.19671403015796	1.74150061739947\\
3.20200576274209	1.70459909187564\\
3.20517527509796	1.66451195769695\\
3.20585002004533	1.62061615374963\\
3.20351362162625	1.57214365762075\\
3.19745620422277	1.51813972897166\\
3.1867033712263	1.45740789312791\\
3.16991440413369	1.38843778139177\\
3.14523622615357	1.30931167074949\\
3.11009471123801	1.2175867111612\\
3.06090031723225	1.11015476668775\\
2.9926453648678	0.983095758531635\\
2.89838838626637	0.831573980390453\\
2.76868761057062	0.649898587637061\\
2.59122233945366	0.432001663954325\\
2.35121103645622	0.172770931598334\\
2.03377775386478	-0.129248435576604\\
1.62959742771365	-0.46779854288743\\
1.14346100946033	-0.825320879963705\\
0.600821886747986	-1.17394011308631\\
0.0442409056935615	-1.4838164622843\\
-0.48163880163311	-1.73456167138362\\
-0.945768934604242	-1.92088020146989\\
-1.33603704925895	-2.04954181981907\\
-1.65480624610948	-2.13266138371325\\
-1.91166875393097	-2.18249310676955\\
-2.11807157549804	-2.2090843405378\\
-2.28459218053074	-2.21985724993794\\
-2.42000314217803	-2.220001836486\\
-2.53121317335989	-2.21304488666874\\
-2.62353282613246	-2.2013486995232\\
-2.70100594205337	-2.18648395128973\\
-2.76670839564354	-2.16948958718696\\
-2.82298812770595	-2.15104809534126\\
-2.87164830808901	-2.13160199457462\\
-2.91408376440492	-2.11143085198241\\
-2.95138155578939	-2.09070216769097\\
-2.98439489737335	-2.06950500443297\\
-3.01379754605468	-2.04787217232614\\
-3.0401239091731	-2.02579474465893\\
-3.06379868525868	-2.00323135043708\\
-3.08515876112665	-1.98011382155867\\
-3.10446929599717	-1.95635020164888\\
-3.12193534452632	-1.93182574111754\\
-3.13770994317351	-1.90640223807004\\
-3.15189925874524	-1.87991589072489\\
-3.16456513585618	-1.85217367237726\\
-3.17572515028752	-1.82294810191056\\
-3.18535005003472	-1.79197014323174\\
-3.19335821697194	-1.75891980928172\\
-3.19960647651815	-1.7234138533874\\
-3.20387617749746	-1.68498968341194\\
-3.20585289974103	-1.64308431012539\\
-3.20509733712814	-1.59700671612584\\
-3.20100372563178	-1.54590148473199\\
-3.19274046786317	-1.48870085916713\\
-3.17916512589608	-1.4240616742884\\
-3.15870247919262	-1.35028304642604\\
-3.12916978909147	-1.26520095598012\\
-3.08752833431522	-1.16605847020069\\
-3.02953722802248	-1.04935888480526\\
-2.9492923192467	-0.910731054901148\\
-2.83866985696159	-0.744886013906675\\
-2.68680635565555	-0.545843632021727\\
-2.48001009197639	-0.307772295691794\\
-2.20297704173767	-0.0269519369913502\\
-1.84266886104903	0.294767101014322\\
-1.39569223068177	0.645751374552216\\
-0.876809173496224	1.0027034275221\\
-0.321315333824451	1.33532954261279\\
0.224920657710854	1.61726471950061\\
0.722602501490466	1.83555657428617\\
1.15023955592394	1.99171058627205\\
1.50385186179304	2.09599160249212\\
1.79026056897879	2.16104741003003\\
2.02047904511769	2.19817110296771\\
2.20572333459932	2.21607809894601\\
2.35571328277463	2.22100511820195\\
};
\addplot [color=mycolor1, forget plot]
  table[row sep=crcr]{%
2.37758606528632	2.28167835318404\\
2.49177779236238	2.27816889033819\\
2.58643654129522	2.26921889562557\\
2.66577691988729	2.25659271656946\\
2.73299918861372	2.24146446362388\\
2.79054164262947	2.22461204330309\\
2.84027340016404	2.20654634251511\\
2.88363789642987	2.18759675296512\\
2.92175850758457	2.16796776122089\\
2.95551605130682	2.14777643322451\\
2.98560570694247	2.12707723384355\\
3.01257893511641	2.10587836965161\\
3.03687443210908	2.08415237025142\\
3.05884100197166	2.06184266324354\\
3.07875438884017	2.03886726768229\\
3.09682950155217	2.01512031012179\\
3.11322901475573	1.99047177773649\\
3.12806899273829	1.96476571310981\\
3.14142191386205	1.93781689048177\\
3.1533172425677	1.90940586879712\\
3.16373947432695	1.87927217334283\\
3.17262333881433	1.8471051976523\\
3.17984555601306	1.81253222273927\\
3.18521215787343	1.77510270119484\\
3.18843985746846	1.73426762499693\\
3.18912918620102	1.68935235999834\\
3.18672600829574	1.63952075883262\\
3.18046639082213	1.58372764217373\\
3.16929742868451	1.52065589925095\\
3.15176323195933	1.4486336729205\\
3.12584069311508	1.36552687954528\\
3.08870415640836	1.26860400463144\\
3.03639360410928	1.1543768731755\\
2.96336381683274	1.01843987857661\\
2.8619195262561	0.855374897665057\\
2.72163445749676	0.658883198778094\\
2.52908689796336	0.422472560497084\\
2.26870515950629	0.141233696761046\\
1.92608821230028	-0.184774505377792\\
1.49499092845071	-0.545931274374462\\
0.986418384452357	-0.920048933956963\\
0.432982829304749	-1.27571827247748\\
-0.119136128434278	-1.58321574433414\\
-0.627655027613959	-1.82575970727758\\
-1.06757582968432	-2.00240478044851\\
-1.43255385347586	-2.12275069775533\\
-1.72842728644552	-2.19990845867801\\
-1.9661116709102	-2.24602123664282\\
-2.15710471968667	-2.27062600113633\\
-2.31149491765359	-2.28061196117877\\
-2.43743478465075	-2.28074408647642\\
-2.5412582089816	-2.27424702197866\\
-2.62780100444458	-2.26328076567257\\
-2.70073509484606	-2.24928521530659\\
-2.76285212574483	-2.23321675902357\\
-2.81628506334122	-2.21570678829857\\
-2.86267486836922	-2.19716683987995\\
-2.9032937035981	-2.17785815738066\\
-2.93913541648268	-2.15793774301906\\
-2.97098193864479	-2.13748886740034\\
-2.99945211478397	-2.11654123389789\\
-3.02503771560578	-2.09508417081845\\
-3.04813004952554	-2.07307503562399\\
-3.06903960224466	-2.05044423866413\\
-3.08801041732097	-2.02709778038385\\
-3.10523040918521	-2.00291784861494\\
-3.12083841282735	-1.97776177798815\\
-3.13492847566302	-1.95145948973794\\
-3.14755165113287	-1.92380937821806\\
-3.15871533073761	-1.89457246868706\\
-3.16837992399641	-1.86346452125125\\
-3.17645243491076	-1.83014558084612\\
-3.1827761531404	-1.79420625383446\\
-3.18711523016356	-1.755149705987\\
-3.18913227693457	-1.71236799799602\\
-3.18835620094771	-1.66511087430589\\
-3.18413615430636	-1.61244447464527\\
-3.17557549147682	-1.55319664750823\\
-3.16143678281318	-1.48588469655782\\
-3.14000494688445	-1.40862079042619\\
-3.10889042539018	-1.31899075571488\\
-3.06474890541543	-1.21390562553849\\
-3.00289199147913	-1.0894368677975\\
-2.91677516560724	-0.940675663175271\\
-2.79740245163396	-0.761722570024749\\
-2.63283920452629	-0.546042725412211\\
-2.40836566454116	-0.287620868514534\\
-2.10836455303723	0.0164973432651121\\
-1.72141396712113	0.362048522724962\\
-1.24883324577633	0.733216621124839\\
-0.712552708207109	1.10224434366042\\
-0.153813549914169	1.43692681283272\\
0.38088733213369	1.71300064169735\\
0.856965914982096	1.92187713206091\\
1.25926017399376	2.0688092073437\\
1.58849820036728	2.16591618255055\\
1.85380508806444	2.22618318092815\\
2.06677018214786	2.26052458407289\\
2.23831992589926	2.27710592776164\\
2.37758606528632	2.28167835318404\\
};
\addplot [color=mycolor1, forget plot]
  table[row sep=crcr]{%
2.39521560661132	2.33808947020594\\
2.50165301737442	2.33481631441034\\
2.59021226763556	2.32644121742552\\
2.66473014948301	2.31458089640988\\
2.72811656402173	2.30031450520744\\
2.78258937535318	2.28435986778607\\
2.8298509249221	2.26719041706212\\
2.87121826114054	2.24911257660912\\
2.90771855167311	2.23031702064168\\
2.94015893115056	2.21091268772948\\
2.96917775336511	2.19094933533025\\
2.9952823265977	2.17043239030596\\
3.01887677215071	2.14933252757017\\
3.04028259129508	2.12759154668238\\
3.05975376304627	2.10512554827704\\
3.07748764274719	2.08182603017976\\
3.0936325248088	2.05755925643596\\
3.1082924230745	2.03216405481763\\
3.12152937147492	2.0054480379043\\
3.13336332535661	1.97718209618812\\
3.14376952253335	1.94709285958934\\
3.15267291408491	1.91485264802381\\
3.15993896395509	1.88006621244632\\
3.16535969787814	1.84225328166619\\
3.16863329139493	1.8008255486039\\
3.16933462925266	1.75505621888867\\
3.16687300610201	1.7040395707407\\
3.16043127369357	1.646637119965\\
3.14887800857505	1.5814059914449\\
3.13064037732106	1.50650418960118\\
3.1035201403668	1.41956730718509\\
3.06442914594481	1.31755356243158\\
3.00901637102516	1.19656307143056\\
2.93116469893273	1.05166211781725\\
2.82237535363197	0.876802042322309\\
2.67118404936331	0.665044383789976\\
2.46305877837921	0.409511599850905\\
2.18178892819626	0.105702096678543\\
1.81392666502375	-0.244362060607632\\
1.35711711704518	-0.62712930014674\\
0.829177392341424	-1.01559675794281\\
0.269344016277399	-1.37549260102969\\
-0.274259788466607	-1.67834639642003\\
-0.763154680849728	-1.91159759274296\\
-1.17858291271912	-2.07844526629714\\
-1.51926750260519	-2.19079797487089\\
-1.7937312306146	-2.26237864675983\\
-2.013711434124	-2.3050575482378\\
-2.19053980187294	-2.32783626535857\\
-2.33376756044192	-2.33709825588292\\
-2.4509528573853	-2.33721912976606\\
-2.54790269211076	-2.3311502847474\\
-2.62902585113351	-2.32086907669468\\
-2.69766227901308	-2.30769673096066\\
-2.7563504132456	-2.2925139611881\\
-2.80703147496583	-2.2759046309315\\
-2.85120130171226	-2.25825088157889\\
-2.89002186142181	-2.2397960981004\\
-2.92440287539562	-2.22068665899547\\
-2.95506162620203	-2.20099964421522\\
-2.98256691656624	-2.18076116422536\\
-3.00737148513045	-2.1599583329753\\
-3.02983594952401	-2.13854684018862\\
-3.05024644954002	-2.11645537990111\\
-3.06882751475763	-2.09358772798619\\
-3.08575120843401	-2.06982294411827\\
-3.10114324655063	-2.0450139461929\\
-3.11508651472267	-2.01898452948588\\
-3.12762217267833	-1.99152475185492\\
-3.13874831762105	-1.96238445921168\\
-3.14841594642258	-1.9312645639131\\
-3.15652168143442	-1.89780549389601\\
-3.16289636663906	-1.86157198093439\\
-3.16728814587444	-1.82203302644447\\
-3.16933792492491	-1.77853544149122\\
-3.16854407998	-1.73026876914547\\
-3.16421174030815	-1.67621863279578\\
-3.15537971184138	-1.6151046176509\\
-3.14071483019245	-1.54529779542672\\
-3.11835896528835	-1.46471234062786\\
-3.0857080945164	-1.37066645943963\\
-3.03909709476125	-1.25971278195924\\
-2.97336311152426	-1.12745369940347\\
-2.88127896804591	-0.968395991395105\\
-2.7529218658074	-0.775985408972052\\
-2.57524500381626	-0.543126418552202\\
-2.33255204446239	-0.263728412551782\\
-2.00920815844371	0.0640732522834683\\
-1.59607758629773	0.433054732063467\\
-1.10000687731179	0.822759794085273\\
-0.550220947233858	1.20119336628135\\
0.00730350945160745	1.53525815237335\\
0.527202673577963	1.80377330441763\\
0.980457052624716	2.00268705038353\\
1.35787365899489	2.14055887331388\\
1.66406241824938	2.23087830678547\\
1.90978990059699	2.28670063179813\\
2.10687494963863	2.31848086352427\\
2.26583641719627	2.33384372724978\\
2.39521560661132	2.33808947020594\\
};
\addplot [color=mycolor1, forget plot]
  table[row sep=crcr]{%
2.40888077172283	2.3902330200096\\
2.50813233444798	2.38717908606175\\
2.59100005011747	2.37934066202057\\
2.66098213322652	2.3682008737186\\
2.72072924962287	2.35475236287878\\
2.77226248661235	2.33965762874679\\
2.81713501474985	2.32335511691346\\
2.85655047962942	2.30612938999659\\
2.8914493925071	2.28815765007762\\
2.92257224517012	2.26954064736369\\
2.95050578023133	2.25032319062776\\
2.97571704516746	2.23050763640493\\
2.99857852155886	2.21006254104203\\
3.01938665427382	2.18892788205039\\
3.03837541110589	2.16701774112359\\
3.05572600135215	2.14422099246346\\
3.07157351078336	2.12040029297962\\
3.0860109244689	2.09538948320704\\
3.09909077261541	2.06898934996484\\
3.110824418657	2.04096155145672\\
3.1211787858935	2.01102034331287\\
3.13007005924853	1.97882155055495\\
3.13735356516316	1.94394798350517\\
3.1428085744848	1.90589016811499\\
3.14611611703187	1.86402081910777\\
3.14682693374144	1.81756088702465\\
3.14431526271719	1.76553421586626\\
3.1377120291608	1.70670683383351\\
3.12580787897313	1.63950571972651\\
3.10691202114415	1.56191082000341\\
3.07864686425438	1.47131399451154\\
3.03765166922578	1.36434169898715\\
2.97916448488124	1.236649989726\\
2.89646197940779	1.08273284380471\\
2.78019192232968	0.895861341069844\\
2.61780086671964	0.668427872973931\\
2.39365140476653	0.393224344394331\\
2.09108675215881	0.066400864885246\\
1.69815533443762	-0.30756101600652\\
1.21720249138408	-0.710637308388417\\
0.673242697314047	-1.11100197657112\\
0.111271203340768	-1.47238612026827\\
-0.420354927699846	-1.76865799316422\\
-0.888053846144788	-1.99185475738594\\
-1.2791714796078	-2.14896926102479\\
-1.59673873350113	-2.253711661735\\
-1.85127279553356	-2.3200990957414\\
-2.05494113784792	-2.35961363973481\\
-2.21875836531109	-2.3807149952148\\
-2.35171732991723	-2.38931115041\\
-2.4608135960206	-2.38942186162935\\
-2.55137267379465	-2.38375139117903\\
-2.62741912650439	-2.37411210124081\\
-2.69199612519804	-2.36171752052997\\
-2.7474161315611	-2.34737908193473\\
-2.79544906409573	-2.33163659063348\\
-2.83746076631175	-2.31484448727144\\
-2.87451414789845	-2.29722896426605\\
-2.90744300093537	-2.27892588091402\\
-2.93690601309933	-2.260005954406\\
-2.96342644474103	-2.24049142405698\\
-2.98742137697517	-2.22036690515763\\
-3.00922329903649	-2.19958618681875\\
-3.02909598395469	-2.17807609740442\\
-3.04724601225948	-2.15573813949756\\
-3.06383087348644	-2.13244830482443\\
-3.0789642517126	-2.10805526648643\\
-3.09271884414577	-2.08237697620772\\
-3.10512683912712	-2.05519554276558\\
-3.11617796394417	-2.02625011386991\\
-3.12581477556484	-1.99522730843816\\
-3.13392457607716	-1.96174852900089\\
-3.1403269462786	-1.92535320050831\\
-3.14475534472522	-1.88547660169896\\
-3.14683042644459	-1.84142044089567\\
-3.14602156332116	-1.79231363703443\\
-3.14159130388004	-1.73705986251555\\
-3.13251492454502	-1.67426729172221\\
-3.11736346176195	-1.60215481634377\\
-3.09413337798508	-1.51842823891945\\
-3.05999944396679	-1.42012104981175\\
-3.01096128934895	-1.30340086206869\\
-2.94135499957542	-1.16336257543353\\
-2.8432282197083	-0.993879982512098\\
-2.70567713390113	-0.787699302400684\\
-2.51450836283949	-0.537165999597471\\
-2.2531264418093	-0.236249327948188\\
-1.90623194822527	0.115453118936052\\
-1.46769549096299	0.507185376642161\\
-0.950611710694782	0.913494425757623\\
-0.391314880114672	1.29859008381683\\
0.16092568827981	1.62959380211794\\
0.663450670397169	1.88921030518518\\
1.09326257240482	2.07787932923935\\
1.44658202591583	2.2069691762133\\
1.73111770281516	2.2909095372093\\
1.95873315187099	2.34261929249117\\
2.14121952233708	2.37204479129414\\
2.28861434227596	2.38628816417568\\
2.40888077172283	2.3902330200096\\
};
\addplot [color=mycolor1, forget plot]
  table[row sep=crcr]{%
2.41888318380138	2.43818635730555\\
2.51147824027556	2.43533570150822\\
2.58904001753741	2.42799778514815\\
2.654762915214	2.41753476294698\\
2.71106571560561	2.40486046826886\\
2.75979350888519	2.39058653892867\\
2.80236582063492	2.37511886450643\\
2.83988441701825	2.35872135292565\\
2.87321169884619	2.34155823262177\\
2.9030278718053	2.32372217986472\\
2.92987282283087	2.3052529845246\\
2.95417692552194	2.28614979942586\\
2.97628375721661	2.26637893920662\\
2.99646682350968	2.24587849060945\\
3.01494175254589	2.22456052875055\\
3.0318749635532	2.20231141395228\\
3.04738947390946	2.17899041324983\\
3.06156824349506	2.15442671091284\\
3.07445523077159	2.12841471565484\\
3.08605412360892	2.10070741684961\\
3.09632448182186	2.07100736819664\\
3.10517475647273	2.03895466388302\\
3.11245129298451	2.00411099441553\\
3.11792192439974	1.96593849533151\\
3.12125203444865	1.92377159147383\\
3.12196989283308	1.87677934295999\\
3.11941645171259	1.82391486427375\\
3.11267237480603	1.76384718481904\\
3.100451490343	1.69486951063386\\
3.08094472458089	1.61477657455123\\
3.05159173440928	1.52070370295161\\
3.008749919873	1.40892425665714\\
2.94722700922035	1.27461720696809\\
2.85965903539448	1.11165831545422\\
2.73578992327268	0.912586932136786\\
2.56192833078652	0.66909961063489\\
2.32135179263591	0.373729959642227\\
1.99719390871752	0.0235653536649638\\
1.57960139546763	-0.373916177171587\\
1.07639734702074	-0.79572889688355\\
0.519932676754131	-1.20541169193261\\
-0.0401808144590513	-1.56571184119036\\
-0.556982479305206	-1.85380500958858\\
-1.00251770771059	-2.06647459713139\\
-1.36985885057439	-2.21406294236225\\
-1.66558777908723	-2.31161292294382\\
-1.90163229697547	-2.37318111115254\\
-2.0902909759994	-2.40978353887551\\
-2.24216045712678	-2.4293446369676\\
-2.36567306671119	-2.43732844588558\\
-2.46729667711223	-2.4374299717228\\
-2.55191766208865	-2.43212985980174\\
-2.62321463845191	-2.42309129638358\\
-2.68396502687506	-2.41143003860427\\
-2.7362792241262	-2.39789414199754\\
-2.78177383166558	-2.3829826772989\\
-2.82169808008363	-2.36702412299338\\
-2.85702574275742	-2.35022828069243\\
-2.88852204894739	-2.33272076557404\\
-2.9167925847934	-2.3145659349788\\
-2.94231919346127	-2.29578204363779\\
-2.9654864259931	-2.27635107291352\\
-2.98660104464018	-2.25622481115633\\
-3.00590633144478	-2.23532819032211\\
-3.02359241755068	-2.213560499177\\
-3.03980345588516	-2.19079482401433\\
-3.05464216166413	-2.16687586668619\\
-3.06817200400236	-2.14161612451635\\
-3.08041711743815	-2.11479026317304\\
-3.09135978700224	-2.0861273514492\\
-3.10093511580077	-2.05530043606938\\
-3.10902217497754	-2.02191269248365\\
-3.11543051501338	-1.98547906615454\\
-3.11988031622032	-1.94540188226001\\
-3.1219735731995	-1.90093830482429\\
-3.12115239057052	-1.85115671672689\\
-3.11663849111376	-1.79487802364494\\
-3.10734508823608	-1.73059656028233\\
-3.09174796201318	-1.65637386358516\\
-3.06769657194691	-1.56969770035753\\
-3.03213858413043	-1.4673001803593\\
-2.98072466317665	-1.3449370790949\\
-2.90726347981168	-1.19715618329567\\
-2.80303457019989	-1.01714771007273\\
-2.65609915783797	-0.796913211476596\\
-2.45108995970766	-0.528249316603974\\
-2.17061792601485	-0.205349676514746\\
-1.80012956956224	0.170304936508782\\
-1.33725525923846	0.583846334044822\\
-0.801923235957124	1.0045985704694\\
-0.237080966282234	1.39362898194891\\
0.306283202348292	1.71941108888218\\
0.789517259102474	1.96912730676105\\
1.19575951981373	2.14748633998509\\
1.52597909755489	2.26815294435095\\
1.79027540467186	2.3461286629649\\
2.00117230186604	2.39404152151433\\
2.17024730915772	2.42130366179175\\
2.3070153558866	2.43451861795335\\
2.41888318380138	2.43818635730555\\
};
\addplot [color=mycolor1, forget plot]
  table[row sep=crcr]{%
2.42553419358814	2.48209316550803\\
2.51196378081259	2.47943096870429\\
2.58458213605233	2.47255952928049\\
2.64631060549358	2.46273135324056\\
2.69936012467214	2.45078846621064\\
2.74541804078164	2.43729579639707\\
2.78578391756268	2.42262902919096\\
2.82146785269512	2.4070326697568\\
2.85326176109995	2.39065857672067\\
2.88179127793504	2.37359159422191\\
2.9075537462819	2.35586655222134\\
2.93094614675948	2.33747938774235\\
2.95228567701671	2.31839416027534\\
2.97182487470824	2.29854709497804\\
2.98976259824536	2.27784836062498\\
3.00625176062514	2.25618199419038\\
3.02140439839821	2.23340416756501\\
3.03529440988456	2.20933981837323\\
3.04795808237633	2.18377751003194\\
3.0593923196453	2.15646222458848\\
3.06955025068456	2.1270856048844\\
3.07833361533312	2.09527292729243\\
3.08558093838671	2.06056577433141\\
3.09104995792431	2.02239895096344\\
3.09439197162826	1.98006960102621\\
3.09511456410917	1.93269567074746\\
3.09252736404776	1.87915976927787\\
3.08566273996497	1.81803305009449\\
3.07315925787811	1.7474720507549\\
3.05308983755072	1.66507989221115\\
3.02270871379862	1.56772318318652\\
2.97808286959396	1.45130101909892\\
2.91357071143426	1.31048152477025\\
2.82113288193912	1.13847444146357\\
2.68955830356001	0.927033947156868\\
2.50397138595715	0.667137086569504\\
2.24660171292782	0.351150802508159\\
1.9006557586107	-0.022570977245387\\
1.45903192426012	-0.442983784762838\\
0.935743711773808	-0.881729012214833\\
0.370352399760586	-1.29810141004348\\
-0.184260873698683	-1.65496980145545\\
-0.683992062919049	-1.93362144158048\\
-1.10690190382878	-2.13553351675148\\
-1.45125304056741	-2.27390535791362\\
-1.72646580436798	-2.36469590442473\\
-1.94539735940974	-2.42180269160389\\
-2.12025416477838	-2.45572707555127\\
-2.26115164240814	-2.47387374299914\\
-2.37597227077536	-2.48129425404923\\
-2.47069213137682	-2.48138746964015\\
-2.54979752218387	-2.47643153711324\\
-2.61665524824535	-2.46795461928887\\
-2.67380430738313	-2.45698365254657\\
-2.7231740280917	-2.44420873454906\\
-2.76624351461178	-2.43009134652392\\
-2.804157207019	-2.41493573690072\\
-2.83780856311567	-2.39893619990212\\
-2.86790084645478	-2.38220850145124\\
-2.89499150134421	-2.36481077198873\\
-2.91952471037678	-2.34675729424175\\
-2.94185536793424	-2.32802739607041\\
-2.96226673544778	-2.30857086859801\\
-2.98098335819641	-2.28831080921709\\
-2.99818033221118	-2.26714443625605\\
-3.01398964954104	-2.24494217170217\\
-3.02850407393302	-2.22154509715157\\
-3.04177877142498	-2.19676072585496\\
-3.0538307123217	-2.17035687706612\\
-3.06463564530312	-2.14205326725037\\
-3.0741221912185	-2.11151022469609\\
-3.08216227591155	-2.07831366445504\\
-3.08855666595789	-2.04195509708647\\
-3.09301371161753	-2.00180494507621\\
-3.09511842166653	-1.95707675054247\\
-3.09428751965502	-1.90677891236173\\
-3.08970390002161	-1.8496493307119\\
-3.08022054787565	-1.78406675893942\\
-3.06421905138112	-1.70793096236617\\
-3.03940094597951	-1.61850272426761\\
-3.00248164581652	-1.51219654758801\\
-2.9487497377427	-1.38432932031328\\
-2.87146019074656	-1.22886076983121\\
-2.76108079899122	-1.03824471047934\\
-2.60458386264906	-0.803692396134978\\
-2.38540845463613	-0.516471258475225\\
-2.08550773718212	-0.171195817604975\\
-1.69154178596567	0.228301500494895\\
-1.20566764318991	0.662469355804623\\
-0.655061178092835	1.09533772275853\\
-0.0884979704197673	1.48566918100562\\
0.442917165571432	1.80437867094387\\
0.905533049896758	2.04349773065613\\
1.28846169219662	2.21165095136388\\
1.59671409466504	2.32430362517328\\
1.84216252941488	2.39672320743641\\
2.0376485110943	2.44113560505875\\
2.19440565187908	2.46641069964569\\
2.3214083155111	2.47868070833737\\
2.42553419358814	2.48209316550803\\
};
\addplot [color=mycolor1, forget plot]
  table[row sep=crcr]{%
2.42914281428483	2.52214630241406\\
2.5098606425793	2.51965884751003\\
2.57787455974418	2.51322203854237\\
2.63586005304591	2.50398887156166\\
2.68584133577255	2.4927359032191\\
2.7293640482531	2.47998518591892\\
2.76761984560266	2.46608444165716\\
2.80153619318871	2.45126002906961\\
2.83184130663175	2.4356521034578\\
2.8591113714304	2.41933800609414\\
2.88380507716818	2.40234775957729\\
2.90628899315419	2.38467416302029\\
2.92685624829824	2.36627908944838\\
2.94574022954792	2.3470970047594\\
2.96312448231146	2.32703633653899\\
2.97914961171824	2.30597904757141\\
2.99391769395348	2.2837785644528\\
3.00749447417415	2.2602560429447\\
3.01990942140661	2.23519479358935\\
3.03115350419836	2.20833252217004\\
3.04117431513899	2.17935083817664\\
3.04986787277575	2.14786122530522\\
3.05706601807852	2.11338631890824\\
3.06251772928118	2.07533485311084\\
3.06586179724353	2.03296796742609\\
3.06658697040766	1.9853536256505\\
3.06397364630225	1.93130461639889\\
3.05700809149553	1.86929391704424\\
3.04425552116582	1.79733918115981\\
3.02367162803375	1.71284623573629\\
2.99232317303891	1.61240136761138\\
2.94597874879659	1.49150834188721\\
2.87852863649959	1.34428980571425\\
2.78122268186052	1.16323950414677\\
2.64184244700154	0.939270709619785\\
2.44428403194152	0.662620627599521\\
2.16978442306006	0.325602725991836\\
1.80195354986152	-0.071787254720533\\
1.33713714180375	-0.514344623845644\\
0.796155506874123	-0.968029684742976\\
0.225385408830048	-1.38848241834113\\
-0.320495137767467	-1.7398323501747\\
-0.801466793838679	-2.00808984704342\\
-1.20169669685893	-2.19920949262367\\
-1.52401280356067	-2.3287433088123\\
-1.7800298904326	-2.41320755995004\\
-1.9831462530216	-2.46619055938935\\
-2.14531363799268	-2.49765257942019\\
-2.27613067877928	-2.51449980177372\\
-2.38294898519408	-2.52140184213171\\
-2.47128834382054	-2.52148752949664\\
-2.54527068598905	-2.51685142827531\\
-2.60798137799079	-2.50889931954025\\
-2.66174506026186	-2.49857737059843\\
-2.70832843285511	-2.4865226709097\\
-2.74908701514419	-2.47316205670539\\
-2.78507089292095	-2.45877723410144\\
-2.81710103606706	-2.44354791224032\\
-2.845824633339	-2.427580480807\\
-2.87175544628144	-2.41092706846434\\
-2.89530339855213	-2.39359809108922\\
-2.91679634871061	-2.37557028981545\\
-2.936496102415	-2.35679153970488\\
-2.95461009070748	-2.33718323428111\\
-2.97129969082941	-2.31664072628661\\
-2.9866858343625	-2.29503207100744\\
-3.00085229047382	-2.27219513542462\\
-3.01384679610838	-2.24793297581879\\
-3.02568000197646	-2.22200722554269\\
-3.03632198587308	-2.19412905234739\\
-3.04569582223605	-2.1639470178792\\
-3.05366734848216	-2.13103087246186\\
-3.06002977695509	-2.09484990883986\\
-3.0644810800858	-2.0547439292298\\
-3.06659099391686	-2.00988408513625\\
-3.06575284006289	-1.95921974918372\\
-3.06111285637371	-1.90140609570132\\
-3.05146592213324	-1.83470518767195\\
-3.03510092726093	-1.75685131364158\\
-3.00957113235041	-1.66487000420575\\
-2.97135518288351	-1.5548423264199\\
-2.91536692282082	-1.42161891556544\\
-2.83428115319192	-1.25852896894489\\
-2.7177092002887	-1.05723449872739\\
-2.55148008890396	-0.808110195610234\\
-2.31782762169535	-0.501924537373715\\
-1.99821561543265	-0.133944263918715\\
-1.58104134546577	0.289132020717469\\
-1.07374775719885	0.742526958041193\\
-0.510972336618491	1.18507853541027\\
0.0537157539135096	1.57423152002543\\
0.570635840131045	1.88433173513163\\
1.01181445364394	2.11241994787035\\
1.37197100770443	2.27059809947887\\
1.65945988956961	2.37567301381497\\
1.88740169647059	2.44293063520625\\
2.06869210106127	2.48411823489821\\
2.21413317989098	2.5075678533355\\
2.3321570981413	2.51896913700286\\
2.42914281428483	2.52214630241406\\
};
\addplot [color=mycolor1, forget plot]
  table[row sep=crcr]{%
2.43000560883911	2.55857157411362\\
2.50542930976853	2.5562462145367\\
2.5691541465995	2.55021438887263\\
2.62363377252617	2.54153864695314\\
2.67072424682261	2.53093579785419\\
2.71184359297447	2.51888853629303\\
2.74808617215659	2.50571873005342\\
2.78030480632662	2.49163582113813\\
2.80917004144808	2.47676895742148\\
2.83521318831917	2.46118836197196\\
2.85885777907958	2.44491946682238\\
2.88044266719758	2.42795207482972\\
2.90023901419008	2.41024599991185\\
2.91846271780422	2.39173410286394\\
2.93528334893509	2.37232328020922\\
2.9508303105995	2.35189370892216\\
2.9651966635197	2.33029645555811\\
2.97844084339309	2.30734939306867\\
2.99058629571449	2.28283120819943\\
3.00161884823388	2.25647310514837\\
3.01148139949471	2.22794759411449\\
3.02006518730663	2.19685346859266\\
3.02719646128919	2.16269568598247\\
3.032616740877	2.12485832201185\\
3.03595387466936	2.08256800212379\\
3.03667964313811	2.03484413297623\\
3.03404737943109	1.98043076123442\\
3.02699959897154	1.91770289468106\\
3.01403034841165	1.84453769696113\\
2.99297927404867	1.75813866559983\\
2.9607241064281	1.65480066598533\\
2.91272749734146	1.52961106448614\\
2.84239275664357	1.37611132381162\\
2.74022248164007	1.18602667489593\\
2.59293678693402	0.949371056469283\\
2.38316159844013	0.655625245963671\\
2.09121669429894	0.297186618404667\\
1.70149648802604	-0.123883606133013\\
1.21452152414961	-0.58761333599355\\
0.6584099902644	-1.05409912183383\\
0.0857008954183244	-1.47609947001611\\
-0.448648923464263	-1.82012192354314\\
-0.909667247310918	-2.07730900092981\\
-1.28747756439644	-2.25775275935131\\
-1.58881445245579	-2.3788685303394\\
-1.82692255391993	-2.45742904132522\\
-2.01543400573807	-2.50660329190737\\
-2.16593074767542	-2.53580052536536\\
-2.287479025774	-2.5514529918328\\
-2.38692357646327	-2.55787740509025\\
-2.46936206415589	-2.55795626182771\\
-2.53858447918773	-2.55361744838998\\
-2.59742172857421	-2.54615564023323\\
-2.64800528589124	-2.53644346222473\\
-2.69195542105061	-2.52506949611656\\
-2.73051630339294	-2.51242866609342\\
-2.76465287164565	-2.49878175250296\\
-2.79512054459071	-2.48429480801279\\
-2.82251568425663	-2.46906536428564\\
-2.84731236472344	-2.45313983471818\\
-2.8698893167037	-2.43652493831251\\
-2.89054973919312	-2.41919495783558\\
-2.909535846854	-2.40109598879854\\
-2.92703944376479	-2.38214789987525\\
-2.94320940052873	-2.36224442497792\\
-2.95815660549424	-2.34125158732077\\
-2.97195672065852	-2.31900447936387\\
-2.98465086686821	-2.29530226248473\\
-2.99624416382948	-2.26990108413472\\
-3.00670183073618	-2.24250441619777\\
-3.01594228058818	-2.21275007129988\\
-3.02382627194575	-2.18019282211358\\
-3.03014065252787	-2.14428108934219\\
-3.03457444334637	-2.10432551822\\
-3.03668382089796	-2.05945635209584\\
-3.03584072775242	-2.00856523662128\\
-3.03115702909942	-1.95022534945272\\
-3.02137183160458	-1.88258151604598\\
-3.00468315655043	-1.80319948472831\\
-2.97849609747813	-1.70886187224915\\
-2.93904850659789	-1.59530079115891\\
-2.88086707514858	-1.45687279619227\\
-2.79601948321354	-1.28623227530518\\
-2.6732142885432	-1.07419103369826\\
-2.49708205345216	-0.810240129508162\\
-2.24864845706541	-0.484691325075486\\
-1.90909185335305	-0.0937332482755964\\
-1.46912479590529	0.352510784013221\\
-0.942205516224551	0.823542318810933\\
-0.370427105395691	1.27329407034341\\
0.18907995510647	1.65898618259533\\
0.689464296947835	1.9592427279986\\
1.10880856622092	2.17608537037877\\
1.44693610665946	2.32460882647534\\
1.71488697109807	2.42255091117002\\
1.92659462130799	2.48502084216652\\
2.09481030058443	2.52323798408973\\
2.22984903760658	2.54500951727224\\
2.33961026878607	2.55561145439992\\
2.43000560883911	2.55857157411362\\
};
\addplot [color=mycolor1, forget plot]
  table[row sep=crcr]{%
2.42839893897157	2.59161347785827\\
2.49891160776086	2.58943859617106\\
2.5586393815682	2.58378428441539\\
2.60983529655238	2.57563073807164\\
2.65420358518009	2.56564016929899\\
2.69304703887209	2.55425912691462\\
2.72737207287741	2.54178559052053\\
2.75796392474883	2.52841326757055\\
2.78544082091949	2.51426099890381\\
2.8102932758938	2.4993923045474\\
2.83291279628798	2.48382828371773\\
2.85361294631788	2.46755592724537\\
2.87264482072889	2.45053315679859\\
2.89020833727843	2.43269141648883\\
2.90646031256835	2.41393631034023\\
2.92151995843505	2.39414654095953\\
2.93547218584112	2.37317121902336\\
2.94836889534608	2.35082545040057\\
2.96022823981116	2.32688394425288\\
2.97103163931986	2.30107219914192\\
2.98071808016543	2.27305459043335\\
2.98917489968384	2.24241837019282\\
2.99622378987899	2.20865215841507\\
3.00160005897145	2.17111689337597\\
3.00492213694687	2.12900633796492\\
3.00564668835006	2.08129299979105\\
3.00300217589035	2.02665358622318\\
2.99588981025141	1.96336577112433\\
2.98273484323523	1.88916514773764\\
2.9612623557401	1.80104840378502\\
2.92815985732262	1.69500827984193\\
2.8785767525192	1.56569451323293\\
2.80541021102086	1.40603032094314\\
2.69837761964472	1.20691688874364\\
2.54308121316076	0.957407214341849\\
2.32083702813122	0.646213341099079\\
2.01114531096811	0.265981362292742\\
1.59961914501329	-0.178687219208589\\
1.09170162030625	-0.662444078680426\\
0.523148560624259	-1.13948493470032\\
-0.0482280155187187	-1.56062127744243\\
-0.568683111535514	-1.89578522694911\\
-1.00897944928141	-2.1414632250918\\
-1.36486403376478	-2.31146022447935\\
-1.64632583661454	-2.42459818521446\\
-1.86775609182295	-2.49765963341006\\
-2.04278178152133	-2.54331658749513\\
-2.18253654625458	-2.57042917760987\\
-2.29555273001758	-2.58498190658312\\
-2.38819486713318	-2.59096580666851\\
-2.46517079139088	-2.59103845424658\\
-2.52996784564671	-2.58697614108782\\
-2.58518641381692	-2.5799724839866\\
-2.6327834639797	-2.57083303956825\\
-2.67424707129207	-2.56010195883131\\
-2.71072081715863	-2.54814477250849\\
-2.74309261413974	-2.53520286296155\\
-2.77205847163025	-2.52142953749779\\
-2.79816859225998	-2.50691400797144\\
-2.82186093299831	-2.49169729619974\\
-2.84348578504061	-2.47578263684834\\
-2.86332383214565	-2.45914202411449\\
-2.88159938833896	-2.4417199475055\\
-2.89848998425722	-2.42343496121357\\
-2.91413309060001	-2.40417945271752\\
-2.92863048354321	-2.3838177685781\\
-2.94205053165504	-2.36218268433478\\
-2.95442848652006	-2.33907004504167\\
-2.96576466316621	-2.31423123091869\\
-2.97602017368924	-2.28736289594039\\
-2.98510959416056	-2.25809315870114\\
-2.99288955404653	-2.2259630587995\\
-2.99914166929046	-2.19040157886496\\
-3.00354738752565	-2.15069180315162\\
-3.00565100845932	-2.10592474384398\\
-3.00480512126117	-2.05493589499166\\
-3.00008956030611	-1.99621754249474\\
-2.99019013318851	-1.92779720802456\\
-2.97321606849211	-1.8470695906231\\
-2.9464247454543	-1.75056724231749\\
-2.90580954052106	-1.63365798584864\\
-2.84549761729985	-1.49017579735687\\
-2.75692173228313	-1.31205379608571\\
-2.62783923073872	-1.08919163341809\\
-2.44162569282153	-0.810148681924428\\
-2.17810548285373	-0.464835980349436\\
-1.81841420228852	-0.0506761365086647\\
-1.35621023012268	0.418182881435659\\
-0.811643913041554	0.905094544320476\\
-0.234026790647956	1.35956208676017\\
0.31732084940483	1.73973430997547\\
0.799593610230153	2.02919135077307\\
1.19704506377141	2.23474970256805\\
1.51401904710699	2.37399798823262\\
1.76364336609869	2.46524767908681\\
1.96030951561495	2.52328093884892\\
2.11647783080989	2.55876082706519\\
2.24194455351926	2.57898823216638\\
2.3440931018766	2.58885379549762\\
2.42839893897157	2.59161347785827\\
};
\addplot [color=mycolor1, forget plot]
  table[row sep=crcr]{%
2.42457355287415	2.621523428073\\
2.49052555348901	2.61948838573613\\
2.54652557549044	2.61418624100554\\
2.5946447751745	2.60652205238196\\
2.63644992345651	2.59710806248537\\
2.67313947238414	2.58635759423051\\
2.70564012646988	2.57454655749056\\
2.73467577519072	2.56185404543694\\
2.76081706891956	2.54838927883922\\
2.78451735616315	2.53420950163965\\
2.80613891728896	2.51933176099744\\
2.82597220488806	2.50374044594455\\
2.84424995897383	2.48739177711343\\
2.86115748122019	2.47021599066643\\
2.8768399395112	2.45211765197708\\
2.89140727205672	2.43297431092441\\
2.90493702657521	2.41263353221843\\
2.91747527242207	2.39090817293977\\
2.92903553501016	2.36756961221875\\
2.93959549574904	2.34233844187454\\
2.94909094576697	2.31487187547121\\
2.95740613582958	2.28474679228606\\
2.96435916638781	2.2514368545102\\
2.96968031544688	2.21428145292397\\
2.97298005798564	2.17244325254973\\
2.97370175105131	2.12484969488447\\
2.97105116935237	2.0701118059777\\
2.96389070753553	2.00641091054438\\
2.95057931564856	1.93134038439232\\
2.92872919451826	1.84168607285294\\
2.8948366770195	1.73312814808155\\
2.8437306653131	1.59985712136739\\
2.76778238029053	1.43413919446595\\
2.65588398862367	1.22599238599821\\
2.49246037204388	0.963443405009668\\
2.25748017035919	0.634428293595332\\
1.92974695414895	0.232038141126992\\
1.49658300369732	-0.236056243872965\\
0.969109087334938	-0.738532999794152\\
0.390884592398836	-1.22381266260643\\
-0.17609913651268	-1.64182639793885\\
-0.680710189455144	-1.96686693340394\\
-1.0998697153845	-2.20079527770734\\
-1.43448717000876	-2.36065433874548\\
-1.69718709536818	-2.46625907080926\\
-1.90310149597363	-2.53420367432117\\
-2.06566949584141	-2.57661116539494\\
-2.19552571443911	-2.60180275093068\\
-2.30067674173315	-2.61534176757871\\
-2.3870346266155	-2.62091880392253\\
-2.45894748078343	-2.62098579549941\\
-2.51962636315109	-2.61718088248714\\
-2.57146235558115	-2.61060556819087\\
-2.61625436119793	-2.60200413307862\\
-2.65537077903324	-2.59187997978379\\
-2.68986407397265	-2.5805715534327\\
-2.72055229588059	-2.56830226729342\\
-2.74807748000692	-2.5552135604492\\
-2.77294781574204	-2.54138686457017\\
-2.79556832752743	-2.52685815230927\\
-2.81626333304886	-2.51162741180158\\
-2.83529292886278	-2.49566454406377\\
-2.85286505359682	-2.47891262867918\\
-2.86914418934422	-2.46128913358444\\
-2.88425741063562	-2.44268538492361\\
-2.89829822701907	-2.42296441587734\\
-2.91132845317397	-2.4019571466252\\
-2.92337815065671	-2.37945668613773\\
-2.93444349154553	-2.35521036779604\\
-2.94448216812327	-2.32890891078919\\
-2.95340567856407	-2.30017180805798\\
-2.96106740583477	-2.26852763901446\\
-2.96724479935986	-2.23338743466187\\
-2.97161304719748	-2.19400840336833\\
-2.97370620162521	-2.1494441447997\\
-2.97285949413804	-2.09847578960964\\
-2.96812308252839	-2.03951613824585\\
-2.95813202714109	-1.97047573735495\\
-2.9409090029799	-1.88857618308768\\
-2.91356436109427	-1.79009320308102\\
-2.87184350497758	-1.67001501497011\\
-2.8094614865568	-1.52162358128125\\
-2.71718707040197	-1.33608164949733\\
-2.58177514845009	-1.10231058322602\\
-2.38528794053165	-0.807888873787678\\
-2.10636620236644	-0.442398846979578\\
-1.72638849802968	-0.00485656194973809\\
-1.24263964739572	0.48592730484768\\
-0.682563345208584	0.986820029965961\\
-0.102218521560992	1.44355819563319\\
0.438335201205228	1.8163867024402\\
0.901333779174422	2.09433643676666\\
1.27709655064445	2.28870855265296\\
1.57386992374557	2.41909611378573\\
1.80634015035226	2.50408009849308\\
1.98907227577506	2.55800280219291\\
2.1341303634353	2.59095821724414\\
2.25077741915636	2.60976288276926\\
2.34590198860912	2.6189491003489\\
2.42457355287415	2.621523428073\\
};
\addplot [color=mycolor1, forget plot]
  table[row sep=crcr]{%
2.41875121389676	2.64855056580978\\
2.4804621851032	2.64664564603231\\
2.53298198384958	2.64167236070519\\
2.57821643935736	2.63446706530605\\
2.61760751089612	2.62559618406988\\
2.65225890000336	2.61544246022642\\
2.68302485996611	2.60426140570847\\
2.71057344301215	2.59221854971585\\
2.73543192786561	2.57941415423297\\
2.75801972560871	2.56589961230934\\
2.77867238846366	2.55168820942244\\
2.79765920720138	2.53676195645932\\
2.81519610462298	2.52107557701103\\
2.83145499419961	2.50455831766021\\
2.84657039211788	2.48711396420712\\
2.86064379112477	2.46861923583169\\
2.87374608562724	2.44892055691462\\
2.88591814898051	2.42782904582382\\
2.89716947953687	2.40511338836097\\
2.90747462505364	2.38049005708748\\
2.91676683306505	2.35361006794347\\
2.92492801306316	2.32404109433784\\
2.93177356786976	2.29124323227207\\
2.93702985183273	2.25453595001245\\
2.940300775593	2.21305264996538\\
2.94101813331877	2.16567766229958\\
2.93836715219469	2.11095817986936\\
2.93117389733477	2.04698043291975\\
2.91773356609498	1.971195270361\\
2.89554727910843	1.88017399131407\\
2.86091934145589	1.76927385505018\\
2.80835074464214	1.63220393324588\\
2.72966595866595	1.4605325021307\\
2.61288926072011	1.24333100600682\\
2.44120489286289	0.967530144238057\\
2.19319903262895	0.620288794588344\\
1.84713034824417	0.195375852040556\\
1.39258091998237	-0.295881881399157\\
0.847097432884703	-0.815618147791448\\
0.262016003075326	-1.30678087031817\\
-0.297752906266213	-1.71958679315429\\
-0.784953348055309	-2.0334850455994\\
-1.18284747663061	-2.25558368389165\\
-1.49696516448511	-2.40566648821301\\
-1.74199732638048	-2.50417547685707\\
-1.93348132882357	-2.56736047948983\\
-2.08453129304809	-2.60676337977591\\
-2.20525285892648	-2.63018218870162\\
-2.303141400249	-2.64278522805763\\
-2.38368413694762	-2.64798585573839\\
-2.45089726047376	-2.64804768345522\\
-2.50773920845619	-2.64448267291651\\
-2.55641057010319	-2.6383081754167\\
-2.59856667747303	-2.63021237239539\\
-2.63546727246695	-2.62066123673404\\
-2.66808204557137	-2.60996823327106\\
-2.69716550575917	-2.59834013178333\\
-2.7233105241102	-2.58590733562874\\
-2.74698695784711	-2.57274402416935\\
-2.76856973595611	-2.55888146843808\\
-2.78835940702573	-2.54431666008423\\
-2.80659720941267	-2.52901761584986\\
-2.82347607789838	-2.51292621290439\\
-2.83914854982753	-2.49595906868554\\
-2.85373220922532	-2.47800673592301\\
-2.8673130622666	-2.45893129569143\\
-2.87994703717352	-2.43856226799976\\
-2.89165961838025	-2.41669059623414\\
-2.902443432702	-2.39306027588406\\
-2.91225337540026	-2.36735696387606\\
-2.92099855885655	-2.33919258990395\\
-2.92852993172743	-2.30808455024994\\
-2.9346217685964	-2.27342743290474\\
-2.93894423737863	-2.23445430645854\\
-2.94102270266134	-2.19018327078028\\
-2.94017697916224	-2.13934303530154\\
-2.93542987776555	-2.08026855077139\\
-2.92536828840323	-2.01075402551339\\
-2.90793065937658	-1.92784624848952\\
-2.88008112556126	-1.82755761831534\\
-2.83731364442982	-1.70448125563007\\
-2.77291809674371	-1.55131641033625\\
-2.67696845008103	-1.35840314840475\\
-2.53516236288159	-1.11361346343406\\
-2.32818792810975	-0.803494524702402\\
-2.0335326165553	-0.417390918102354\\
-1.63315184890276	0.0436749996658289\\
-1.12868460204515	0.55555786694248\\
-0.555370186023146	1.06841079251652\\
0.0246852414975189	1.52504568575042\\
0.552154009179338	1.88894204508582\\
0.995072749419049	2.15489125427073\\
1.34954746142368	2.33827777055983\\
1.62710857431764	2.46023536325952\\
1.84354179456123	2.53936050375954\\
2.01336098445757	2.58947345133322\\
2.1481605412785	2.62009781566314\\
2.25666811573677	2.63758949473537\\
2.34530095960704	2.64614792059426\\
2.41875121389676	2.64855056580978\\
};
\addplot [color=mycolor1, forget plot]
  table[row sep=crcr]{%
2.41112293345451	2.67293497202217\\
2.46888392391395	2.67115131797152\\
2.51815038737103	2.6664855191518\\
2.56067745902692	2.65971096292733\\
2.59779343839778	2.65135197568606\\
2.6305157273557	2.64176311344075\\
2.65963253350677	2.63118102002161\\
2.68576094455292	2.61975863731399\\
2.70938859486226	2.60758789630244\\
2.73090382849039	2.59471475314211\\
2.75061769692551	2.58114902195038\\
2.76878007240405	2.56687056330675\\
2.78559143844458	2.55183281196918\\
2.8012114227226	2.53596424504605\\
2.81576478577869	2.51916812567583\\
2.82934531927976	2.50132065759491\\
2.84201790198104	2.48226751893041\\
2.85381878116153	2.46181858333906\\
2.8647539665374	2.43974046017921\\
2.87479541554687	2.41574626808028\\
2.8838744197952	2.3894817673208\\
2.89187122478307	2.36050657363737\\
2.89859935644222	2.32826859854975\\
2.90378227390621	2.29206901942597\\
2.90701863342752	2.2510138451484\\
2.90773033309651	2.20394632395823\\
2.90508412791903	2.14935179353612\\
2.89787220142401	2.08522283788636\\
2.88432854807521	2.00886770995555\\
2.86184503952644	1.916639659751\\
2.82653307728443	1.80356260538734\\
2.77255796402825	1.66284105249546\\
2.69117522918125	1.4853017815288\\
2.56949523498214	1.25900114520614\\
2.38939363438971	0.969699040009248\\
2.12804201406743	0.60378361818778\\
1.76333966100236	0.155977290279563\\
1.28774343536765	-0.358089056184037\\
0.72595113667567	-0.893477459637415\\
0.136840561367889	-1.38815401247061\\
-0.413143683273551	-1.79385084762324\\
-0.881710561305588	-2.09580917159059\\
-1.25843603279283	-2.30612464980497\\
-1.55288592047271	-2.44682454430528\\
-1.7813060775171	-2.53866033056596\\
-1.9593657775584	-2.59741700772015\\
-2.09975332575718	-2.63403834464104\\
-2.21203071942399	-2.65581837430199\\
-2.3032006562605	-2.66755558712192\\
-2.37835240081262	-2.67240733578158\\
-2.44119571607529	-2.67246443748957\\
-2.49445761919681	-2.66912333754563\\
-2.54016488193194	-2.66332432155596\\
-2.57984205540937	-2.6557040973601\\
-2.61464993480629	-2.64669419424165\\
-2.6454827915952	-2.63658501085421\\
-2.67303717734763	-2.62556789562519\\
-2.69786105735113	-2.61376299788296\\
-2.72038922461991	-2.60123775195561\\
-2.74096903869299	-2.58801907026067\\
-2.75987924735955	-2.57410120073672\\
-2.77734377856985	-2.55945048935761\\
-2.793541793656	-2.54400782159856\\
-2.80861487682785	-2.52768920006705\\
-2.82267193569293	-2.51038468758181\\
-2.83579215890868	-2.491955765212\\
-2.84802618740108	-2.47223099410424\\
-2.85939547818274	-2.45099970448246\\
-2.86988964901186	-2.42800324168044\\
-2.87946135839992	-2.40292305034405\\
-2.8880179592727	-2.37536453831611\\
-2.89540870763175	-2.34483518069009\\
-2.90140561910968	-2.31071462828254\\
-2.90567500103259	-2.27221356472263\\
-2.90773500983351	-2.22831655519435\\
-2.90689191136059	-2.17770193064862\\
-2.90214344552749	-2.11862858948262\\
-2.8920308887691	-2.04877526035613\\
-2.87441080959479	-1.96501245936017\\
-2.84610196161767	-1.86308285568593\\
-2.8023432419264	-1.73716857524884\\
-2.7359855348757	-1.5793538010083\\
-2.63637493654125	-1.37909972675918\\
-2.48809271214243	-1.12315205734959\\
-2.27038916682889	-0.796974956580139\\
-1.95964380346866	-0.389789045959459\\
-1.53877736296124	0.0949035646598881\\
-1.01455397500912	0.626922435424875\\
-0.430388009563244	1.14961064156151\\
0.146485075591149	1.60386358275269\\
0.65890842805611	1.95746634107526\\
1.08124245096797	2.21110290073615\\
1.41497076391917	2.38377830939657\\
1.67431219460888	2.49773918811789\\
1.87576040675745	2.57138888298126\\
2.03360307134768	2.61796802131199\\
2.15891606343668	2.64643668053227\\
2.25989814780917	2.66271445002484\\
2.34251989232733	2.67069163243849\\
2.41112293345451	2.67293497202217\\
};
\addplot [color=mycolor1, forget plot]
  table[row sep=crcr]{%
2.40184832977892	2.69490294654388\\
2.45592399283202	2.69323249711691\\
2.50214465959359	2.68885462745153\\
2.54212772027489	2.68248487007762\\
2.57709765951613	2.67460878682083\\
2.60799303796255	2.66555490609426\\
2.63554167666137	2.65554239876409\\
2.6603140114805	2.64471252194086\\
2.68276133876676	2.6331494654426\\
2.7032434880069	2.62089414511776\\
2.72204899549531	2.60795318726871\\
2.73940987242113	2.59430452761714\\
2.75551239638478	2.57990051837748\\
2.77050489726047	2.56466908339177\\
2.78450318336705	2.54851321266359\\
2.79759401240609	2.53130889797802\\
2.80983681818068	2.51290144869255\\
2.82126373107969	2.49309996625585\\
2.83187775268152	2.47166957438063\\
2.84164873539826	2.44832077323747\\
2.85050654188269	2.42269497733214\\
2.85833036546549	2.39434486122415\\
2.86493260415113	2.36270750643164\\
2.87003477162677	2.32706741444136\\
2.87323149515886	2.28650507241896\\
2.87393635669052	2.23982470919166\\
2.87129963211028	2.18545186014563\\
2.86408200330457	2.12128703196035\\
2.85045876424552	2.04449595538954\\
2.82771432238764	1.95121039336567\\
2.79176614553139	1.8361102048244\\
2.73643546713606	1.69187094277892\\
2.65238485586856	1.5085310432453\\
2.52576058108595	1.27305716438514\\
2.33705616610302	0.969957779644263\\
2.06200028345344	0.584866453040137\\
1.67835833154431	0.113785735228833\\
1.18214610812648	-0.42263604568819\\
0.605896103323672	-0.971925405430728\\
0.0155723827383446	-1.46775406562183\\
-0.522312519362957	-1.86462709186122\\
-0.971324398289734	-2.15404224631969\\
-1.32715043845893	-2.35271822870257\\
-1.60279530089054	-2.48444397602867\\
-1.81560852479424	-2.57000911025981\\
-1.98117106386415	-2.62464285417094\\
-2.1116732147888	-2.65868520750944\\
-2.21612975662973	-2.67894743218132\\
-2.30107153790832	-2.68988207556836\\
-2.37121551086357	-2.69440981171719\\
-2.42998826643122	-2.69446257712194\\
-2.4799043774965	-2.69133079768861\\
-2.52283159053019	-2.68588400343966\\
-2.56017497653528	-2.67871156117301\\
-2.59300495343547	-2.67021324110272\\
-2.62214686783574	-2.66065811175114\\
-2.64824425085508	-2.65022322168541\\
-2.6718039350366	-2.63901919428186\\
-2.69322855158082	-2.6271071999645\\
-2.7128401395691	-2.61451012441759\\
-2.73089740259439	-2.60121972067263\\
-2.74760834240366	-2.58720087567082\\
-2.76313944898201	-2.57239369124244\\
-2.77762224240739	-2.55671378533672\\
-2.79115768371606	-2.54005100487191\\
-2.80381875835273	-2.52226656886728\\
-2.81565135572206	-2.50318850185678\\
-2.82667339617656	-2.48260504946375\\
-2.83687196698051	-2.46025556649031\\
-2.84619799109553	-2.43581810407097\\
-2.85455762548908	-2.40889255730016\\
-2.86179910668912	-2.37897771193176\\
-2.86769303203279	-2.3454397646135\\
-2.87190292593754	-2.30746876063673\\
-2.8739411298695	-2.26401771203514\\
-2.87310214305253	-2.2137166667641\\
-2.86836083396254	-2.15475036650014\\
-2.85821536473773	-2.08468305677749\\
-2.84044272991812	-2.00020763943901\\
-2.81171705992515	-1.8967905929009\\
-2.76701826400222	-1.7681864806928\\
-2.69874331942785	-1.60582994541959\\
-2.59547450038752	-1.39824243423878\\
-2.44061218668422	-1.13095957820658\\
-2.21190189403025	-0.78830978415099\\
-1.88467872020738	-0.359531309080791\\
-1.44327960281451	0.148851239629981\\
-0.900402958899709	0.699900932966638\\
-0.307870210739911	1.23020995198984\\
0.263072675579177	1.67991415655052\\
0.758799419972088	2.02207454581275\\
1.16029175073245	2.26323582897871\\
1.47391126142261	2.42552507639461\\
1.71600758027358	2.53191511984887\\
1.90345289959078	2.60044748272965\\
2.050174427334	2.64374495208587\\
2.16669927035574	2.67021656755708\\
2.26070957776861	2.6853697803666\\
2.33775391788163	2.6928077177244\\
2.40184832977892	2.69490294654388\\
};
\addplot [color=mycolor1, forget plot]
  table[row sep=crcr]{%
2.39105565046959	2.71466394891098\\
2.44168643913923	2.7130993739671\\
2.48505087718409	2.70899156370742\\
2.52264008371797	2.70300275790928\\
2.5555834337588	2.69558274045283\\
2.58474724027934	2.68703595981662\\
2.61080394512603	2.67756538313932\\
2.63428115432691	2.66730141297161\\
2.65559676206356	2.6563210478661\\
2.67508435447998	2.64466053875074\\
2.69301172311061	2.63232359642322\\
2.70959440943117	2.61928645031054\\
2.72500558850878	2.60550057022014\\
2.73938317702514	2.59089353517344\\
2.75283475043425	2.57536830065925\\
2.76544062920564	2.55880093494172\\
2.7772553117192	2.54103673622695\\
2.78830726502361	2.5218844812342\\
2.79859690981943	2.50110836859014\\
2.80809242516848	2.47841698035865\\
2.8167227152342	2.4534482560457\\
2.82436647131996	2.42574900413702\\
2.83083564371333	2.39474678989435\\
2.83585067333367	2.35971101846401\\
2.8390032983272	2.31969850350355\\
2.83970027217576	2.27347651354509\\
2.83707728986047	2.21941285894894\\
2.82986581942565	2.15531758873091\\
2.81618486650372	2.07821402943217\\
2.79321303813971	1.98400893839282\\
2.75667255046015	1.86702687505075\\
2.70003133071598	1.71938838577005\\
2.61333263100642	1.53029269807504\\
2.48170337858554	1.28553493342318\\
2.28417482600422	0.968284901730114\\
1.99500961002305	0.563450340431757\\
1.592112701967	0.0687015669963673\\
1.07581729000886	-0.489513382033007\\
0.487110705977186	-1.05080876082787\\
-0.101641479202603	-1.54545169258212\\
-0.625362480657865	-1.93196938522036\\
-1.05415748912718	-2.20840572355328\\
-1.38948135606153	-2.39565815264789\\
-1.64718968517319	-2.51882184377782\\
-1.84534326727985	-2.59849595956324\\
-1.99925943132172	-2.64928709852718\\
-2.12058058446563	-2.68093414564767\\
-2.21777859831219	-2.69978770858757\\
-2.29693437442552	-2.70997680820576\\
-2.36241671214485	-2.71420298673928\\
-2.4173901737337	-2.71425176296527\\
-2.46417386728752	-2.71131600828767\\
-2.50448964864668	-2.70620012054386\\
-2.53963310878231	-2.69944981929434\\
-2.57059186364337	-2.69143552820743\\
-2.59812807799416	-2.68240655774621\\
-2.62283663401238	-2.67252668224414\\
-2.64518657839968	-2.66189767317864\\
-2.6655509604152	-2.65057488898442\\
-2.68422850198075	-2.6385775051426\\
-2.70145943030626	-2.62589502007647\\
-2.71743705844548	-2.61249106726922\\
-2.73231619149175	-2.59830516623178\\
-2.74621908146117	-2.58325277124674\\
-2.7592393960627	-2.5672237744592\\
-2.77144446655925	-2.55007945339229\\
-2.7828759086264	-2.5316476957986\\
-2.79354854264693	-2.51171616361689\\
-2.80344735085607	-2.49002284799106\\
-2.81252196710058	-2.46624318803999\\
-2.82067785705647	-2.43997253467787\\
-2.82776284595137	-2.4107021746446\\
-2.83354688066196	-2.37778629462227\\
-2.83769169898016	-2.34039601769445\\
-2.83970513134109	-2.29745476968859\\
-2.83887159838726	-2.2475464208599\\
-2.83414520261156	-2.18878349072564\\
-2.82398340099586	-2.11861679335381\\
-2.80608582235899	-2.03356028041366\\
-2.77698255603545	-1.92879753547014\\
-2.73139010013438	-1.79763801733551\\
-2.6612351704066	-1.63082968596641\\
-2.55429669422483	-1.41588772712857\\
-2.39272325480453	-1.13704585751673\\
-2.15268490071564	-0.77744335145647\\
-1.8085585557974	-0.326512109778409\\
-1.34662019034505	0.205578402133016\\
-0.786342626004456	0.774402472667871\\
-0.188013121231037	1.31003964582746\\
0.374409853793793	1.75315073077823\\
0.852071731470989	2.08291484474334\\
1.23266559621533	2.3115591569358\\
1.52687413548001	2.46381909057387\\
1.75266676341099	2.56305005688437\\
1.92702017200067	2.62679740425486\\
2.06339979492001	2.66704294905067\\
2.17176767297531	2.69166092518201\\
2.2593053602677	2.70577013195945\\
2.33116355135832	2.71270670876987\\
2.39105565046959	2.71466394891098\\
};
\addplot [color=mycolor1, forget plot]
  table[row sep=crcr]{%
2.37884203763194	2.73240879223995\\
2.4262463534695	2.73094342879247\\
2.46692757621165	2.72708936391518\\
2.50226073696543	2.72145961913605\\
2.53328781311229	2.71447087784343\\
2.56080871004439	2.70640526457967\\
2.5854449290535	2.69745069568468\\
2.60768463890829	2.68772748032498\\
2.62791494206404	2.67730593553107\\
2.6464452044794	2.6662179994832\\
2.66352405277528	2.65446472577958\\
2.6793518021746	2.64202084628609\\
2.69408951242514	2.62883713957879\\
2.70786547887196	2.61484103799892\\
2.72077968780334	2.59993568796996\\
2.73290655569953	2.58399750554662\\
2.74429609976839	2.5668721135433\\
2.75497352690937	2.5483683842569\\
2.76493705589793	2.52825011894172\\
2.77415357525633	2.50622464329364\\
2.78255144923767	2.48192724851407\\
2.79000936000601	2.4548999034889\\
2.79633942553089	2.42456191960397\\
2.80126181244299	2.39016913416001\\
2.80436642492361	2.35075648925176\\
2.80505458049549	2.30505631604419\\
2.80244918709069	2.25138075418868\\
2.79525467387766	2.18745099670203\\
2.78153604010664	2.11014802606856\\
2.758367560554	2.01514983613783\\
2.7212744512416	1.89641365784961\\
2.66336095320443	1.74547682738926\\
2.57402172341646	1.55064360100373\\
2.43730295388958	1.29644711496952\\
2.23068579536726	0.964623854548406\\
1.92695107239325	0.53940119782852\\
1.50447501405737	0.0205785582624191\\
0.968746023487389	-0.558742255112305\\
0.369736970208452	-1.13000185512183\\
-0.214705012903247	-1.62115746921311\\
-0.722436941866099	-1.9959639182891\\
-1.13057305149817	-2.25912794505457\\
-1.44588359012524	-2.43522464577601\\
-1.68651158363303	-2.55023300337369\\
-1.87089179250685	-2.62437145274093\\
-2.01394001635376	-2.67157654255079\\
-2.1267181216797	-2.70099465656543\\
-2.21716486116874	-2.71853801731161\\
-2.29093332892937	-2.72803299245666\\
-2.35206672882154	-2.7319778944991\\
-2.40348677360069	-2.73202299109313\\
-2.44733230308416	-2.72927115191948\\
-2.48519096075237	-2.72446665978726\\
-2.51825772237007	-2.71811489139163\\
-2.54744411116073	-2.71055909230139\\
-2.57345419832581	-2.70203023728123\\
-2.59683809457397	-2.69267976219929\\
-2.61803003408079	-2.68260120792063\\
-2.63737578015784	-2.67184454230672\\
-2.65515252366643	-2.66042552932652\\
-2.67158341559993	-2.64833164167971\\
-2.68684818699216	-2.63552545590418\\
-2.70109084093572	-2.62194610110999\\
-2.71442507394136	-2.60750907732493\\
-2.7269378446073	-2.59210456801815\\
-2.73869132010812	-2.57559421036609\\
-2.74972326774459	-2.55780613071517\\
-2.76004579562609	-2.53852787825735\\
-2.76964215815257	-2.51749667161289\\
-2.7784610964171	-2.49438607796518\\
-2.78640783518803	-2.46878782609997\\
-2.79333033579016	-2.44018684377489\\
-2.79899859316409	-2.40792670023395\\
-2.80307347538253	-2.37116126318891\\
-2.80505951574642	-2.32878629733392\\
-2.80423264437882	-2.27934157092956\\
-2.79952819480964	-2.22086929786549\\
-2.78936520895674	-2.15070788412002\\
-2.7713680059692	-2.06519087440544\\
-2.74192293759335	-1.95921180708588\\
-2.69547797082562	-1.82561616358676\\
-2.62347135048132	-1.65442475362061\\
-2.51283474043797	-1.43207320182174\\
-2.34438634992853	-1.1413920454077\\
-2.09264625946526	-0.76427828495814\\
-1.73114810967095	-0.2905765432957\\
-1.24871320700027	0.265184724864449\\
-0.672449713736175	0.850361898452582\\
-0.0709690600813887	1.38896482709428\\
0.480509122770547	1.82356632238975\\
0.938992152198613	2.14015562258572\\
1.29878931037204	2.35633720929923\\
1.57431739364831	2.4989422377873\\
1.78470493929849	2.59140743991512\\
1.94680749926314	2.65067669043398\\
2.0735538240495	2.68807927237218\\
2.1743349153968	2.71097316510059\\
2.2558500138841	2.72411098961202\\
2.3228751080575	2.73058038763443\\
2.37884203763194	2.73240879223995\\
};
\addplot [color=mycolor1, forget plot]
  table[row sep=crcr]{%
2.36527365438272	2.74830870908072\\
2.40964991746339	2.7469364995168\\
2.44780579915534	2.74332128955086\\
2.48100929698089	2.73803052651152\\
2.51022183713887	2.73145019589874\\
2.53618210125969	2.72384168192079\\
2.55946459271266	2.71537889659003\\
2.58052106015224	2.70617275202405\\
2.59971013934627	2.69628735428491\\
2.61731878089818	2.68575065701308\\
2.63357785800937	2.6745612993535\\
2.64867357201913	2.66269271269721\\
2.6627557512408	2.65009516481937\\
2.67594377880519	2.63669612760843\\
2.68833062790302	2.62239914930745\\
2.69998528743098	2.60708124673624\\
2.71095369814491	2.59058868007056\\
2.7212581648169	2.57273080906794\\
2.73089503983224	2.55327153086106\\
2.73983025987883	2.5319175355474\\
2.74799202070095	2.50830224501375\\
2.75525943556605	2.48196376072738\\
2.76144534500368	2.45231434242239\\
2.7662703691099	2.4185977235074\\
2.76932355026417	2.37982870970839\\
2.77000206564684	2.33470665069354\\
2.76741771764363	2.28149000033681\\
2.76024993931903	2.2178126088649\\
2.74651183528815	2.14041300741655\\
2.72317454125283	2.04473625024061\\
2.68556393474381	1.92435911002144\\
2.62640871451058	1.77020478075194\\
2.53442204404956	1.56962081682207\\
2.39250057485056	1.30577768731714\\
2.17647867610017	0.958875721384523\\
1.8576501387814	0.512529816055839\\
1.41526549058398	-0.030780494230754\\
0.860889959227299	-0.630372554180624\\
0.253891702488302	-1.20940152139326\\
-0.323558108606942	-1.69481349550972\\
-0.813700808611469	-2.05671810266489\\
-1.20091960518074	-2.30643522281392\\
-1.4967680434039	-2.47167953487244\\
-1.72114720592667	-2.57892790820529\\
-1.89257879775753	-2.64786151485634\\
-2.02546999690482	-2.69171490991658\\
-2.13028266636831	-2.71905474556955\\
-2.21443591196754	-2.73537676422754\\
-2.28317683298629	-2.74422401082905\\
-2.3402439663865	-2.74790596622875\\
-2.38833355093365	-2.7479476596335\\
-2.42941776898398	-2.74536870731669\\
-2.46496045199247	-2.74085775992939\\
-2.49606383485234	-2.73488281086318\\
-2.52356930303391	-2.72776187791145\\
-2.54812735187924	-2.71970888677516\\
-2.57024676673503	-2.7108637874166\\
-2.59032961545124	-2.70131246095496\\
-2.60869642198736	-2.69109987555867\\
-2.6256044407356	-2.68023866329648\\
-2.6412609982819	-2.66871448687934\\
-2.65583323412694	-2.65648905152365\\
-2.66945514000038	-2.64350127674358\\
-2.68223249488475	-2.6296669046439\\
-2.69424607072992	-2.61487663978795\\
-2.70555330792793	-2.59899275958483\\
-2.71618850375176	-2.58184397872029\\
-2.72616139775262	-2.56321817320308\\
-2.73545385022647	-2.54285234259458\\
-2.74401406053032	-2.52041887789477\\
-2.75174741329837	-2.49550675676421\\
-2.75850249706639	-2.46759563065094\\
-2.76404998802014	-2.43601978130414\\
-2.76805072479322	-2.39991742195724\\
-2.77000706768743	-2.35815851420924\\
-2.76918793999981	-2.30924073259587\\
-2.76451178202781	-2.25113782679321\\
-2.75436136354981	-2.18107670228624\\
-2.73628754151374	-2.09520878021411\\
-2.70653284298508	-1.98812972299983\\
-2.65927065768527	-1.85220040887384\\
-2.58543021080492	-1.67666991121794\\
-2.47104662203931	-1.44681282868073\\
-2.29552004164257	-1.14394425989429\\
-2.03164242041711	-0.748667517327314\\
-1.65225577728419	-0.251513538736312\\
-1.14943023701283	0.327810235700239\\
-0.558776498007111	0.927735895787799\\
0.0431409482925311	1.46687835816249\\
0.581416088700065	1.89118337284762\\
1.01983159465553	2.19397491574896\\
1.35905688462949	2.39782266432595\\
1.61664699951616	2.53115395667853\\
1.8124797275961	2.61722576532144\\
1.96310543470836	2.67229944803121\\
2.08086225396267	2.70704894646139\\
2.17457171154062	2.72833581772816\\
2.2504702101494	2.74056777650213\\
2.31298100828986	2.74660085820289\\
2.36527365438272	2.74830870908072\\
};
\addplot [color=mycolor1, forget plot]
  table[row sep=crcr]{%
2.3503853119908	2.76251494643398\\
2.39191393217561	2.76123037943232\\
2.42768859660251	2.75784042750851\\
2.45887833723077	2.75287022845848\\
2.48637012136388	2.74667722880223\\
2.510846019112	2.73950350164073\\
2.53283704608621	2.73150990440398\\
2.55276121997183	2.72279858834094\\
2.57095078490094	2.71342788164366\\
2.58767188916159	2.70342205775786\\
2.60313891570007	2.69277756784621\\
2.61752494851065	2.68146672721784\\
2.63096937682453	2.66943946143598\\
2.64358330724213	2.656623455247\\
2.65545321580629	2.64292285417333\\
2.66664308967188	2.62821550969925\\
2.67719515377777	2.61234860840996\\
2.68712912855599	2.59513236017421\\
2.69643979677686	2.57633121563916\\
2.70509244243733	2.55565180726134\\
2.71301542150625	2.53272641584916\\
2.72008867039546	2.50709018833066\\
2.7261262507295	2.47814946676669\\
2.73084989653595	2.44513726724934\\
2.73384867932385	2.40704990747715\\
2.73451683301892	2.36255561318433\\
2.73195661543024	2.30986101940349\\
2.72482435363388	2.2465139818327\\
2.711083155337	2.16911018188178\\
2.68760184537471	2.07285693063183\\
2.64950384721276	1.95093593613899\\
2.58912858477636	1.79362188109626\\
2.49447036384301	1.58723660965141\\
2.34719855894177	1.31347506200585\\
2.12139403076715	0.950889813008775\\
1.78687361851418	0.482581582351276\\
1.32425335231122	-0.0856291170117338\\
0.752183398073065	-0.704480587244766\\
0.139677587891934	-1.28892186282411\\
-0.42816231067355	-1.76638553844743\\
-0.899324293288053	-2.11435121474952\\
-1.26551887600693	-2.35054509822281\\
-1.54249592627137	-2.50526302268155\\
-1.7514250245256	-2.60513147346486\\
-1.91067266168712	-2.66916706266552\\
-2.03405548013089	-2.70988262961109\\
-2.13142588241301	-2.73528065477131\\
-2.20969916046123	-2.75046160356716\\
-2.27373753915316	-2.75870303328772\\
-2.3269942216892	-2.76213862713827\\
-2.37195570736334	-2.7621771646753\\
-2.41043972602469	-2.75976104871229\\
-2.44379557720665	-2.75552731042341\\
-2.47303976527881	-2.74990921479259\\
-2.49894883600904	-2.74320130790478\\
-2.52212372944933	-2.73560163068733\\
-2.54303497524366	-2.72723942343221\\
-2.5620548349764	-2.71819343096336\\
-2.57948042108719	-2.70850398289026\\
-2.59555047760972	-2.69818083952949\\
-2.61045762762849	-2.68720805526125\\
-2.62435730680135	-2.67554663761915\\
-2.63737420432198	-2.66313546515162\\
-2.6496067533685	-2.64989070435837\\
-2.66113000678477	-2.63570379354718\\
-2.67199706851311	-2.62043790966508\\
-2.68223910229461	-2.60392267912295\\
-2.69186378362787	-2.5859467119911\\
-2.70085187355923	-2.56624730312205\\
-2.70915133990608	-2.54449631638781\\
-2.71666808282094	-2.52028079436336\\
-2.72325175723032	-2.49307613112037\\
-2.72867429229312	-2.46220857806186\\
-2.73259726361307	-2.42680221280929\\
-2.73452189320905	-2.38570295979474\\
-2.73371147248218	-2.33736830171394\\
-2.72906928957888	-2.27970522948764\\
-2.71894380765585	-2.20982984310806\\
-2.70081399747552	-2.12370930394393\\
-2.67077795434588	-2.01563260904588\\
-2.62272724530581	-1.87745314027105\\
-2.54705860167537	-1.69759855679207\\
-2.42885477702257	-1.46009112163033\\
-2.24599939558024	-1.14460545563963\\
-1.96947512171323	-0.730403961985268\\
-1.57163178709569	-0.209047206830658\\
-1.04860508517012	0.3936365298367\\
-0.445360833644853	1.00649873650033\\
0.154214806613381	1.54369452870495\\
0.67719354757618	1.95604462684509\\
1.09485028267557	2.24455198983996\\
1.41382211716279	2.43625168009219\\
1.65421359806668	2.5606892612365\\
1.8362909434323	2.6407179570803\\
1.97615061287271	2.69185560424471\\
2.08550273385151	2.7241245251592\\
2.17260633020613	2.74391022237171\\
2.24325488576203	2.7552954844211\\
2.30153959553947	2.76092014780299\\
2.3503853119908	2.76251494643398\\
};
\addplot [color=mycolor1, forget plot]
  table[row sep=crcr]{%
2.33417922251919	2.77515857935272\\
2.37302446345421	2.77395663381085\\
2.40654963067401	2.77077951041994\\
2.43583199686454	2.76611296825101\\
2.46168950562335	2.76028785819516\\
2.48475173000904	2.7535282337604\\
2.50550933109755	2.74598276018145\\
2.5243489983959	2.73774540873181\\
2.54157844074649	2.72886912755511\\
2.55744444898051	2.71937479102455\\
2.57214604857724	2.70925687233551\\
2.58584410069049	2.6984867418507\\
2.59866826704092	2.68701413897711\\
2.61072194848004	2.67476712102847\\
2.62208558667726	2.66165061021449\\
2.63281854814552	2.64754350692396\\
2.6429596636201	2.6322941888767\\
2.65252635132696	2.61571404867556\\
2.6615120867452	2.59756851130044\\
2.66988176493976	2.57756468474824\\
2.6775641922802	2.55533438297514\\
2.68444047616295	2.53041064659606\\
2.69032634517252	2.50219495706871\\
2.69494524328088	2.46991090852706\\
2.69788708180155	2.4325378702118\\
2.69854424714761	2.38871466854988\\
2.69601088276543	2.33659781368121\\
2.68892192270199	2.27365026140483\\
2.67519210905055	2.19632400672814\\
2.65158830987743	2.09958293128851\\
2.61302737146264	1.9761971289107\\
2.55144333649516	1.81575407886478\\
2.45406877280891	1.60347200034296\\
2.30125726814113	1.31944292483288\\
2.06521822926748	0.94045104662107\\
1.71432392256784	0.44922297385281\\
1.23115674064735	-0.144283789810344\\
0.64254576172766	-0.781166412816022\\
0.0271944808310615	-1.36848884553553\\
-0.528487228620339	-1.83585571054603\\
-0.979468638449927	-2.16898652663144\\
-1.32465584313549	-2.39166122610503\\
-1.58337414476832	-2.5361915531421\\
-1.77761447908935	-2.62904253610715\\
-1.92538542429413	-2.68846398527979\\
-2.039851616804	-2.72623686500676\\
-2.13025406291663	-2.74981681312022\\
-2.20302147092732	-2.76392931196305\\
-2.26265139697322	-2.77160284798201\\
-2.31232951967749	-2.77480711141059\\
-2.35434685090081	-2.77484271481674\\
-2.39037762959059	-2.77258026463459\\
-2.42166492224548	-2.76860877200978\\
-2.44914575953804	-2.76332915972775\\
-2.47353657346962	-2.75701408575001\\
-2.495392336303	-2.74984676076529\\
-2.51514806328149	-2.74194642214541\\
-2.5331483197224	-2.73338515779818\\
-2.54966844276935	-2.72419899145912\\
-2.56492994429187	-2.71439505253689\\
-2.57911174959065	-2.7039559742482\\
-2.59235838752929	-2.69284222744608\\
-2.60478588133654	-2.68099280546934\\
-2.61648583152902	-2.66832446685929\\
-2.62752799079809	-2.65472957859868\\
-2.63796147539942	-2.64007245461603\\
-2.64781461497922	-2.62418392938969\\
-2.65709329070221	-2.60685372101366\\
-2.66577742457177	-2.58781989333687\\
-2.67381502638283	-2.56675438293266\\
-2.68111282640781	-2.54324305418456\\
-2.68752193708849	-2.51675799216027\\
-2.69281605417463	-2.48661859130609\\
-2.69665818485541	-2.45193621304128\\
-2.69854935756574	-2.41153439090227\\
-2.6977484920616	-2.36383216765948\\
-2.69314531457329	-2.30667127533311\\
-2.68305573365365	-2.23705737709181\\
-2.66488806280295	-2.15077062614394\\
-2.63459468046404	-2.04178326316977\\
-2.58577654841186	-1.90141537153563\\
-2.50827077513128	-1.7172172108897\\
-2.38614393387481	-1.47185548805787\\
-2.19565192404784	-1.1432245218687\\
-1.90588545328172	-0.709206749592182\\
-1.48896435522745	-0.162825728991575\\
-0.946038384383019	0.462888162769359\\
-0.332236635993454	1.0866376214684\\
0.262157127043303	1.6193428477606\\
0.767906914478278	2.01820505971951\\
1.16428518567672	2.2920606042203\\
1.46339148439475	2.47184040961561\\
1.68730986127867	2.58775757832882\\
1.85638013965914	2.66207117916182\\
1.98612588689922	2.70951094253041\\
2.08760480900204	2.73945608556077\\
2.16852420162778	2.75783643569324\\
2.23425447519047	2.76842852203312\\
2.28857408052101	2.77367002737662\\
2.33417922251919	2.77515857935272\\
};
\addplot [color=mycolor1, forget plot]
  table[row sep=crcr]{%
2.31662243599076	2.78635024525841\\
2.35293417322862	2.78522633722784\\
2.38433045880942	2.78225065824871\\
2.41180325991813	2.7778722261458\\
2.43610635920612	2.77239704930667\\
2.45782051375344	2.76603233073719\\
2.47739883494592	2.75891532736314\\
2.49519883713148	2.75113236092503\\
2.51150535913619	2.74273136444641\\
2.52654713200155	2.73373007205922\\
2.54050884222562	2.72412117266028\\
2.55353993303382	2.71387524921832\\
2.56576097919128	2.70294199797262\\
2.57726818947959	2.69124999428929\\
2.58813638716678	2.67870509974901\\
2.59842065983668	2.66518745748042\\
2.60815673126147	2.65054687583518\\
2.61735996812363	2.63459623152095\\
2.62602277037307	2.61710230605039\\
2.63410987622555	2.59777316859469\\
2.64155079762947	2.57624078212906\\
2.64822811992764	2.55203685768825\\
2.65395963466464	2.52455898553249\\
2.6584710289845	2.4930225250869\\
2.66135378373019	2.45639130137023\\
2.66199942680694	2.41327629119163\\
2.65949527222369	2.36178533355917\\
2.6524563644985	2.29929721566371\\
2.63875037745638	2.22211878641674\\
2.61504196243896	2.12496360453309\\
2.57603596220367	2.00017105922659\\
2.51324192059867	1.83659727800389\\
2.41308093908502	1.61826797727075\\
2.25448927532998	1.32352756237873\\
2.00767471046191	0.927262586440358\\
1.6396299310979	0.412023553211461\\
1.13564164668971	-0.207132141700543\\
0.531891017170401	-0.860550597649819\\
-0.0834486858662258	-1.44803462631231\\
-0.624497145372981	-1.90321556429516\\
-1.05427299723325	-2.22074456387053\\
-1.37856985036945	-2.4299693354227\\
-1.61965084392854	-2.56465624946155\\
-1.79992400599472	-2.65083349015314\\
-1.93687163112228	-2.70590311341063\\
-2.04296139720988	-2.74091147029444\\
-2.12682657384936	-2.76278570212073\\
-2.19442722084214	-2.77589557972898\\
-2.24991539535961	-2.78303561167106\\
-2.2962256307893	-2.78602219784378\\
-2.33546637180161	-2.78605506640721\\
-2.36917823068472	-2.78393789758166\\
-2.39850548094883	-2.78021491926245\\
-2.42431127936886	-2.77525686174071\\
-2.44725617194405	-2.7693159257264\\
-2.46785237352234	-2.76256144815528\\
-2.48650184003541	-2.75510330734979\\
-2.50352333224779	-2.74700737306404\\
-2.51917188068512	-2.7383056689549\\
-2.53365291367801	-2.72900291552327\\
-2.54713256339255	-2.7190804973541\\
-2.55974516930548	-2.7084984965352\\
-2.57159866170076	-2.69719616348986\\
-2.58277827002371	-2.68509100112774\\
-2.59334882299732	-2.67207648161534\\
-2.60335576135157	-2.65801827063435\\
-2.6128248474056	-2.64274867898989\\
-2.62176040693981	-2.62605887200621\\
-2.6301417522632	-2.60768811336371\\
-2.6379171756221	-2.58730895947424\\
-2.64499451442092	-2.56450678882722\\
-2.65122668480172	-2.53875124671864\\
-2.65638960695603	-2.50935594678835\\
-2.66014834317394	-2.47542083303262\\
-2.66200458033747	-2.43574854134014\\
-2.66121400291597	-2.38872122347622\\
-2.65665419266781	-2.33211656121882\\
-2.64660999519241	-2.26282968154675\\
-2.62841985146878	-2.17645012552654\\
-2.59788825718166	-2.06662154701748\\
-2.54831481543425	-1.92410119775107\\
-2.46894529719174	-1.73549809709094\\
-2.34275646897583	-1.48200569342412\\
-2.14425031032493	-1.1395812012755\\
-1.8405441962249	-0.68470252093548\\
-1.40387341614752	-0.112406976598289\\
-0.841502519951627	0.535834141894993\\
-0.219445282773875	1.16814747973172\\
0.366866749718658	1.69376187769669\\
0.853610391505406	2.07772463285486\\
1.22833872432155	2.33666348088705\\
1.50801765790865	2.50478235141778\\
1.71616753137343	2.61254193101756\\
1.87292919968403	2.68144670794859\\
1.99315921357654	2.72540708647408\\
2.08724855622801	2.75317114021806\\
2.16236616371458	2.770233056107\\
2.22347880009503	2.78008048176229\\
2.27407015977558	2.78496175200151\\
2.31662243599076	2.78635024525841\\
};
\addplot [color=mycolor1, forget plot]
  table[row sep=crcr]{%
2.29764238025503	2.79617948043294\\
2.33155776382023	2.7951294127764\\
2.36093593396223	2.79234472148207\\
2.38668934938194	2.78824006338051\\
2.40951199610384	2.78309819003935\\
2.4299391191885	2.77711051459377\\
2.44838879937177	2.77040359931626\\
2.46519131130015	2.76305660106635\\
2.48061012244609	2.75511277219028\\
2.49485707323489	2.74658694414486\\
2.50810342978124	2.73747019904336\\
2.5204879439354	2.72773247612557\\
2.53212268191843	2.71732355777748\\
2.54309712414534	2.70617266786296\\
2.55348085049531	2.69418675228871\\
2.56332497685799	2.68124736911181\\
2.57266237715152	2.66720596998822\\
2.58150658947687	2.65187718369467\\
2.58984914288974	2.63502948882822\\
2.5976548223296	2.61637234942624\\
2.60485406837627	2.59553842849492\\
2.61133121281677	2.57205880296976\\
2.61690645784271	2.54532803886171\\
2.62130820407685	2.51455431738734\\
2.62413014725095	2.47868715322246\\
2.62476382931063	2.43631099338295\\
2.62229085265392	2.38548613158552\\
2.61530762106444	2.32350741071506\\
2.60163561168301	2.24653420776192\\
2.57783619728613	2.14902122911547\\
2.53839509382159	2.02285473924832\\
2.47437437686926	1.85610842486001\\
2.37132537767582	1.63151301647124\\
2.20664964662485	1.32549983636793\\
1.94841036224136	0.910921515403966\\
1.5623335377777	0.370431734902438\\
1.03732013187301	-0.274642951463033\\
0.420138859990144	-0.942770060131745\\
-0.192120709557591	-1.52749140993308\\
-0.716137112867406	-1.96845936081391\\
-1.12384148689449	-2.26973704776009\\
-1.42744562508792	-2.46563371112109\\
-1.65151005015505	-2.59082142391287\\
-1.81849752043933	-2.67064968374397\\
-1.94522530502567	-2.72160981845203\\
-2.0434324665042	-2.75401654034908\\
-2.12115231049674	-2.77428731292085\\
-2.18389439956994	-2.78645439868368\\
-2.23548337572101	-2.79309220063738\\
-2.27861768269272	-2.7958735475304\\
-2.31523492728519	-2.79590386087272\\
-2.3467509926555	-2.7939242853969\\
-2.37421804826164	-2.79043718477528\\
-2.39843040416711	-2.78578503878942\\
-2.41999651456956	-2.78020088696098\\
-2.43938871911307	-2.77384106170873\\
-2.45697812031413	-2.76680666955355\\
-2.47305937586617	-2.75915776364104\\
-2.48786853217569	-2.75092264808629\\
-2.50159596948731	-2.74210383819441\\
-2.5143958431194	-2.73268162772698\\
-2.52639295089314	-2.72261584436207\\
-2.53768764755075	-2.71184612372187\\
-2.54835920793859	-2.70029084918318\\
-2.55846787556392	-2.68784475507024\\
-2.5680556957367	-2.67437504951772\\
-2.57714610153282	-2.65971575798996\\
-2.58574207511867	-2.64365979494904\\
-2.5938225208964	-2.62594800823379\\
-2.60133622402374	-2.60625406320411\\
-2.60819237113583	-2.58416347219998\\
-2.61424598537452	-2.55914421883699\\
-2.61927561430565	-2.53050509658149\\
-2.62294892622753	-2.49733578203864\\
-2.62476901977294	-2.45841930989898\\
-2.62398934180504	-2.41210221761324\\
-2.61947654268191	-2.35609894231874\\
-2.6094855718454	-2.28719332139928\\
-2.59128520730172	-2.2007795031393\\
-2.56052874427497	-2.09015840588625\\
-2.51020112791778	-1.94549010324041\\
-2.42891927198263	-1.75236866575036\\
-2.29848437511039	-1.49037886305939\\
-2.09150070389166	-1.13336474617869\\
-1.77303719770791	-0.656399625920843\\
-1.3159015701882	-0.0572398538969062\\
-0.734747581320188	0.612789276987113\\
-0.107048664920237	1.25102495865522\\
0.468223340686164	1.76689291123208\\
0.934333044396128	2.13466154427952\\
1.28716764037607	2.37850745298307\\
1.54789253602088	2.53524599458945\\
1.74095319874678	2.6351980127523\\
1.88605718476763	2.69897948316981\\
1.99732059302863	2.73966108812865\\
2.08446122808108	2.76537413976779\\
2.15412473300267	2.7811966420266\\
2.21089301465229	2.79034350550865\\
2.25797171794367	2.79488540293541\\
2.29764238025503	2.79617948043294\\
};
\addplot [color=mycolor1, forget plot]
  table[row sep=crcr]{%
2.27711966681336	2.80471326690848\\
2.30876470732822	2.80373318111351\\
2.33622690217155	2.80112983301123\\
2.36034442597795	2.79728567370915\\
2.38175539889466	2.7924616367204\\
2.40095253020421	2.786834310215\\
2.41832114071146	2.78052021289298\\
2.43416601174845	2.77359177997818\\
2.44873059324809	2.76608788938589\\
2.46221089244057	2.7580206871475\\
2.47476558676405	2.74937981003638\\
2.48652339508863	2.74013468284281\\
2.49758839947065	2.73023528881658\\
2.50804377223615	2.71961161443807\\
2.5179541893136	2.70817181549812\\
2.52736707217291	2.69579901338413\\
2.53631267572461	2.68234648611736\\
2.54480290812179	2.66763084538734\\
2.5528286079959	2.65142256070542\\
2.56035478463002	2.63343286564032\\
2.56731300020612	2.61329559929467\\
2.57358956422711	2.59054180412344\\
2.57900738873048	2.5645637648759\\
2.5832979924475	2.53456337815224\\
2.58605783945938	2.49947686287451\\
2.58667922539985	2.45786315029811\\
2.58423895852489	2.40773566618065\\
2.57731573035855	2.34630482938462\\
2.56368513748234	2.26957901976898\\
2.53980314901611	2.17174333944803\\
2.49992699240001	2.04420410512122\\
2.43464331397276	1.87419253977717\\
2.32856450729637	1.64302484427267\\
2.1574207074857	1.32502897461142\\
1.88697506485576	0.890884177558295\\
1.48187061299944	0.32374191689811\\
0.935748434048338	-0.347378077649403\\
0.307228859364719	-1.02797244346139\\
-0.298652652289612	-1.60678448308606\\
-0.803317499659814	-2.03157712530496\\
-1.18822916270127	-2.3160609748746\\
-1.47140286407627	-2.49879359146632\\
-1.67906419043248	-2.61482261881218\\
-1.83340828547876	-2.68860811215353\\
-1.95047408647239	-2.73568281866713\\
-2.04125105712786	-2.76563715013135\\
-2.11318329204814	-2.78439779818418\\
-2.17134788716336	-2.79567665281743\\
-2.21925906595674	-2.80184076968151\\
-2.25939302529013	-2.80442825172745\\
-2.29352720274335	-2.80445617227769\\
-2.3229607998811	-2.80260711193358\\
-2.34865991490779	-2.79934421143854\\
-2.37135453980611	-2.7949834601364\\
-2.39160445426495	-2.78973991383537\\
-2.40984471980418	-2.78375769234307\\
-2.42641757489392	-2.77712966684287\\
-2.44159511028663	-2.76991044141953\\
-2.45559558365701	-2.76212485746951\\
-2.46859526417699	-2.75377341099572\\
-2.48073706973861	-2.74483544778696\\
-2.49213684364279	-2.73527066118357\\
-2.50288783414984	-2.72501918483355\\
-2.51306373878632	-2.71400040098098\\
-2.52272052208447	-2.70211044165896\\
-2.53189708615449	-2.68921822164761\\
-2.54061474783449	-2.67515968626879\\
-2.54887533342882	-2.65972975940298\\
-2.55665751642393	-2.64267120488666\\
-2.56391075765847	-2.62365921967494\\
-2.57054580187642	-2.60227998503542\\
-2.57642004026832	-2.57800049216393\\
-2.58131499451859	-2.55012553322605\\
-2.58490141245528	-2.51773547914119\\
-2.58668444774294	-2.4795948022608\\
-2.58591614512812	-2.43401533480081\\
-2.58145318549996	-2.3786485182647\\
-2.57152137197258	-2.31016523795941\\
-2.55331927040942	-2.22375785222391\\
-2.52234413115481	-2.11236728680922\\
-2.4712496105744	-1.96551581309859\\
-2.38797880348207	-1.76769630314829\\
-2.25305643044805	-1.49672758793427\\
-2.03702474111524	-1.1241431238314\\
-1.70284397425902	-0.62365208058744\\
-1.22450190265727	0.00335988167970699\\
-0.625509477875658	0.694114905155601\\
0.00485503797733931	1.3352611742484\\
0.566072085352282	1.83867314563659\\
1.01006365019401	2.18906550931705\\
1.34087070187219	2.41771869486442\\
1.58313849312425	2.56337217786309\\
1.76176167440832	2.65585265256686\\
1.89581443079859	2.7147768970127\\
1.99861613727433	2.75236420962431\\
2.07921101972012	2.77614516660634\\
2.14373753399703	2.79080032912494\\
2.19641075219014	2.79928685625849\\
2.24017376828498	2.80350843912919\\
2.27711966681336	2.80471326690848\\
};
\addplot [color=mycolor1, forget plot]
  table[row sep=crcr]{%
2.25487688810311	2.81199323983976\\
2.28436799553838	2.81107956910574\\
2.3100089430402	2.80864861750409\\
2.33256834724573	2.80505258966712\\
2.35263201677768	2.8005319114714\\
2.37065281728478	2.79524922692905\\
2.38698536527165	2.7893116067201\\
2.40191053147013	2.78278517030559\\
2.41565297676594	2.77570470126259\\
2.42839383599721	2.76807985783397\\
2.4402799545843	2.7598989765926\\
2.45143061802544	2.75113108183596\\
2.4619424020688	2.74172645626618\\
2.47189255390494	2.73161594466065\\
2.48134115440656	2.72070901609231\\
2.49033218214529	2.7088904763179\\
2.49889348116304	2.69601557851705\\
2.50703550697458	2.68190310486602\\
2.51474856656858	2.66632575464071\\
2.52199804710781	2.64899683529037\\
2.52871679629322	2.62955174742179\\
2.53479329508715	2.6075219805043\\
2.54005341352389	2.5822981249259\\
2.54423212086449	2.55307647608831\\
2.54692909744006	2.51878068145456\\
2.54753797263876	2.47794475558929\\
2.54513142675877	2.4285353127635\\
2.53827094999473	2.36767688352728\\
2.52468583058409	2.29122165295407\\
2.5007230709378	2.19307131555196\\
2.46039904329072	2.06411951158295\\
2.3937904251023	1.89068329845897\\
2.28448754298865	1.65252313899048\\
2.10638869534037	1.32164367346809\\
1.82279138568968	0.86641600175977\\
1.39754469963532	0.271049611753693\\
0.830426111489532	-0.426006181532845\\
0.193139376052034	-1.11630813728006\\
-0.402818609233224	-1.68582385182827\\
-0.885895466495825	-2.09254689146715\\
-1.24742522357496	-2.35979211481426\\
-1.51048266855685	-2.52955874150764\\
-1.7023428922059	-2.63676350644936\\
-1.84464870306425	-2.7047947972659\\
-1.95256916176499	-2.74819161705105\\
-2.03633170215196	-2.77583072299165\\
-2.10280408108491	-2.79316676413249\\
-2.15664861365666	-2.80360735984377\\
-2.20108504615934	-2.80932396858313\\
-2.23838006958827	-2.81172803992354\\
-2.27016067988777	-2.81175371512871\\
-2.2976166988921	-2.81002861637584\\
-2.32163361419796	-2.80697906088408\\
-2.34288119505579	-2.8028961489738\\
-2.36187363699789	-2.79797802636934\\
-2.37901107343624	-2.79235732394082\\
-2.39460868067106	-2.78611916859778\\
-2.40891737554244	-2.77931305185168\\
-2.42213871244104	-2.77196058679941\\
-2.43443570141001	-2.76406041841396\\
-2.44594069599043	-2.75559107179853\\
-2.45676111987736	-2.74651221050273\\
-2.46698354232244	-2.73676456192228\\
-2.47667642714893	-2.72626860544654\\
-2.48589173823608	-2.7149219818025\\
-2.49466546269359	-2.70259544602571\\
-2.50301699232055	-2.68912703001946\\
-2.51094716403585	-2.67431387862586\\
-2.51843457479595	-2.65790094139657\\
-2.52542951768997	-2.63956528977856\\
-2.53184447167411	-2.61889420571681\\
-2.53753941350928	-2.59535422173673\\
-2.54229912529256	-2.56824676694474\\
-2.54579782094551	-2.53664362217958\\
-2.54754322323746	-2.49929139094935\\
-2.54678661119326	-2.45446760501918\\
-2.5423753367342	-2.39976018367811\\
-2.53250625450951	-2.3317241259899\\
-2.51430614558058	-2.24534136350177\\
-2.48310931004538	-2.13317135168002\\
-2.43121704898488	-1.98404962205009\\
-2.3458439814805	-1.78126542388781\\
-2.20611831389706	-1.50068724615164\\
-1.98033238609531	-1.11131775015247\\
-1.62930687482662	-0.585608586595307\\
-1.12902246122067	0.0702408160438668\\
-0.513522037910772	0.780218103330241\\
0.116111253157241	1.42083254405841\\
0.660204998816192	1.90902779295882\\
1.08073270905983	2.24097041048473\\
1.38947351393834	2.45439728713025\\
1.61379618081722	2.58927045308193\\
1.77860543104869	2.67460103391161\\
1.90217247786813	2.72891623217254\\
1.99697791304627	2.7635793316525\\
2.07139663550914	2.78553726263968\\
2.13107658113404	2.79909109356255\\
2.17988317892706	2.80695414441097\\
2.22051134717918	2.81087290798355\\
2.25487688810311	2.81199323983976\\
};
\addplot [color=mycolor1, forget plot]
  table[row sep=crcr]{%
2.2306613868761	2.81803070297518\\
2.25810690410877	2.81718012527407\\
2.28201516003601	2.81491320421253\\
2.30308950693452	2.81155368741527\\
2.32186667064398	2.80732269213538\\
2.33876212010734	2.80236972560499\\
2.35410163785533	2.79679295696782\\
2.36814362598294	2.79065256203978\\
2.38109507726808	2.78397948565953\\
2.39312313224099	2.77678107656757\\
2.40436349690098	2.76904449916777\\
2.41492657297905	2.76073847420239\\
2.42490186835632	2.75181366374805\\
2.43436105653859	2.74220184468911\\
2.44335990651309	2.73181387613749\\
2.45193918373476	2.72053633609572\\
2.46012451013591	2.70822655989054\\
2.4679250472588	2.69470563459613\\
2.47533070951499	2.67974865995893\\
2.48230739246368	2.66307123355111\\
2.48878936392124	2.64431058778368\\
2.49466743013696	2.6229989883018\\
2.4997706103052	2.59852571315108\\
2.50383757236253	2.57008185871996\\
2.50647153235448	2.53657882915073\\
2.50706783155106	2.49652574468635\\
2.50469535887311	2.44784158954154\\
2.49789834998327	2.38756214474631\\
2.48435833264454	2.31137583767289\\
2.46030778970481	2.21288289339427\\
2.41950575059641	2.08242347523106\\
2.35147553534986	1.90531321212694\\
2.23868402405164	1.65958770416935\\
2.05300806937287	1.31467312543502\\
1.75510971736689	0.836517617528177\\
1.30849126874066	0.21118988845803\\
0.720798419557306	-0.511317529079947\\
0.0779140443342484	-1.20791854545575\\
-0.504310404303508	-1.76449354912078\\
-0.963651051034922	-2.15132519346601\\
-1.30133101338113	-2.40097689803586\\
-1.54462840311249	-2.55800318024844\\
-1.72127575581048	-2.65671069052081\\
-1.85211382956641	-2.71925994897991\\
-1.95136885095432	-2.75917169432432\\
-2.02850063951073	-2.78462216400938\\
-2.08981496010035	-2.80061234470425\\
-2.13957654655818	-2.81026070860493\\
-2.18072560715238	-2.81555396271566\\
-2.21533107455486	-2.81778429561671\\
-2.2448783990183	-2.81780785993293\\
-2.27045467227012	-2.81620060789198\\
-2.29286973444643	-2.81335422309845\\
-2.31273685207127	-2.80953638074976\\
-2.33052744438039	-2.80492929976484\\
-2.34660885364804	-2.79965478587334\\
-2.36127083449701	-2.7937906722762\\
-2.37474439994278	-2.78738164589066\\
-2.38721539229624	-2.78044630443655\\
-2.3988343416865	-2.77298159220663\\
-2.40972365384161	-2.76496532407901\\
-2.41998282343293	-2.75635722071773\\
-2.42969213301304	-2.74709867891067\\
-2.43891512794588	-2.73711134941806\\
-2.44770002625738	-2.72629446298113\\
-2.45608010789501	-2.71452071131373\\
-2.46407301201541	-2.70163033252864\\
-2.47167873369492	-2.68742284384416\\
-2.4788759266952	-2.67164557285018\\
-2.48561584719577	-2.65397770782051\\
-2.49181285049689	-2.63400793094603\\
-2.49732966899515	-2.61120267356519\\
-2.50195456246426	-2.58486040061168\\
-2.50536549382862	-2.554044685875\\
-2.50707310865982	-2.51748448173663\\
-2.5063283000101	-2.47342271799767\\
-2.50196930299109	-2.41938216717487\\
-2.49216346758393	-2.35179720417731\\
-2.47396280197052	-2.26542754675948\\
-2.4425289071623	-2.15242386901544\\
-2.38978363183492	-2.00087480010687\\
-2.30214558738244	-1.79274231006888\\
-2.15720197537847	-1.50172612091604\\
-1.92078092466796	-1.09405563207464\\
-1.55158685210626	-0.541139515258474\\
-1.02868761619051	0.144455732587054\\
-0.398536047886836	0.871547814567372\\
0.226482120535897	1.50768862618519\\
0.750336570209397	1.97786021388669\\
1.14618923806126	2.29038532705206\\
1.43290798360277	2.48861007795899\\
1.63980650934666	2.61301346167155\\
1.79139786468341	2.69150173946715\\
1.90500767470046	2.74143986091333\\
1.99224745257609	2.77333611988895\\
2.06083057765806	2.79357153066645\\
2.11593135374411	2.80608480485869\\
2.16108190935215	2.81335835477259\\
2.1987423302369	2.81699046255361\\
2.2306613868761	2.81803070297518\\
};
\addplot [color=mycolor1, forget plot]
  table[row sep=crcr]{%
2.20411869675044	2.82279803928925\\
2.22962048977649	2.82200742847636\\
2.25187975884548	2.81989662779306\\
2.27153851214511	2.81676257166763\\
2.28908734150646	2.8128081720023\\
2.30490655602639	2.80817054307244\\
2.31929481642068	2.80293945579783\\
2.33248937844525	2.79716948382655\\
2.34468059668352	2.790887965464\\
2.35602242581111	2.78410009843371\\
2.36664007142398	2.77679198415477\\
2.37663555958338	2.76893211680475\\
2.38609173629602	2.76047159494436\\
2.39507502710029	2.75134317398984\\
2.40363715134788	2.74145914604152\\
2.41181587340279	2.73070790680094\\
2.41963476568831	2.71894892691552\\
2.42710183821288	2.70600566388891\\
2.43420673359399	2.69165569968379\\
2.44091596335171	2.67561702216299\\
2.44716531868489	2.65752881254623\\
2.45284803969333	2.63692423640233\\
2.45779641799448	2.61319136133464\\
2.46175296364436	2.58551609671097\\
2.464324582366	2.55279737375112\\
2.46490843664902	2.51351861804726\\
2.46256950944333	2.46554910456749\\
2.45583393945943	2.40583098262153\\
2.44233259023548	2.32987794822825\\
2.41817504775361	2.23096462491157\\
2.37684071941484	2.09882560974564\\
2.30724399164613	1.91766663513266\\
2.19060259752446	1.66359274901303\\
1.99654643199345	1.30315549704448\\
1.68294141792844	0.799814089190649\\
1.21362999681624	0.142652742032227\\
0.606266086420765	-0.604237935836287\\
-0.0382997540981736	-1.30291832432706\\
-0.602702520520728	-1.84263705063436\\
-1.03625418886858	-2.2078342624618\\
-1.34972907059829	-2.43962109553829\\
-1.573657269817	-2.58415548714345\\
-1.73566550234319	-2.67468489667198\\
-1.8555751156271	-2.73200944254198\\
-1.94661241757109	-2.76861601619204\\
-2.01746950767089	-2.79199533022614\\
-2.07390549224461	-2.80671263664546\\
-2.11980415561342	-2.81561147538022\\
-2.15784017121593	-2.82050384345695\\
-2.18989557116555	-2.82256946670836\\
-2.21732242434057	-2.82259104341717\\
-2.24111117383757	-2.82109587128214\\
-2.26200054487199	-2.81844301039579\\
-2.28055069787866	-2.81487805671626\\
-2.29719284121756	-2.81056820777978\\
-2.31226348081426	-2.80562505579916\\
-2.32602845262016	-2.80011955472088\\
-2.33870003161801	-2.79409186837045\\
-2.35044925968698	-2.78755777079688\\
-2.3614149052239	-2.78051263692365\\
-2.37170999556761	-2.77293366269971\\
-2.38142655029902	-2.76478069164113\\
-2.39063892875089	-2.75599584069135\\
-2.3994060499628	-2.74650197585847\\
-2.40777262176581	-2.73619996140739\\
-2.41576940802391	-2.72496447442836\\
-2.42341245167745	-2.71263801805474\\
-2.43070103652721	-2.69902255510953\\
-2.43761398623936	-2.68386788206692\\
-2.44410362433956	-2.66685541318944\\
-2.45008628715942	-2.64757535338349\\
-2.45542757757212	-2.62549415098482\\
-2.45991936623893	-2.5999073759985\\
-2.46324351677259	-2.56987031334723\\
-2.46491374110498	-2.53409380753901\\
-2.4641805794229	-2.49078487179822\\
-2.45987276887681	-2.4373979208442\\
-2.45012653546256	-2.37023935955886\\
-2.431914115951	-2.28383038390616\\
-2.40021064168641	-2.16987733211045\\
-2.34652301112763	-2.01564625766097\\
-2.25638881346475	-1.8016197458856\\
-2.10567806579636	-1.4990673696351\\
-1.85751247124311	-1.07118544502344\\
-1.46860003919241	-0.488731048675673\\
-0.922577686600991	0.22731151805632\\
-0.280350139184698	0.968585112721327\\
0.335608902264612	1.59573520344585\\
0.83607009051639	2.0450385502171\\
1.20616863030855	2.33728236795813\\
1.47098269446232	2.5203802444838\\
1.66098315442301	2.63462778072726\\
1.7999268326179	2.70656813576981\\
1.90407499279771	2.7523467538357\\
1.98414951579792	2.78162251588829\\
2.0472127716962	2.80022858519715\\
2.09798230247861	2.81175764958274\\
2.13967244255459	2.81847325874164\\
2.1745208493223	2.821833771635\\
2.20411869675044	2.82279803928925\\
};
\addplot [color=mycolor1, forget plot]
  table[row sep=crcr]{%
2.17475110014827	2.82621406094749\\
2.19840629905796	2.82548043297719\\
2.21909692391879	2.82351815726881\\
2.23740723939928	2.82059887586622\\
2.25378441387765	2.81690831762872\\
2.26857565585033	2.81257189340399\\
2.2820540835625	2.80767144137197\\
2.2944370290331	2.802256246321\\
2.30589916343327	2.79635024908732\\
2.31658200720803	2.78995663347652\\
2.3266008614282	2.7830605250051\\
2.33604985169515	2.77563024340975\\
2.34500554288033	2.7676173513012\\
2.35352941858832	2.75895559282036\\
2.36166939494934	2.74955869074923\\
2.36946043370963	2.7393168467454\\
2.37692421752611	2.7280916469946\\
2.38406773332394	2.71570889107058\\
2.39088045524881	2.70194860287711\\
2.39732959413596	2.68653110035243\\
2.40335253200047	2.66909741690414\\
2.40884499632096	2.64918145274385\\
2.41364258794858	2.62616976861731\\
2.41749166427431	2.59924253789691\\
2.42000274952655	2.56728517874866\\
2.42057456170659	2.5287534147332\\
2.41826742025695	2.48146286987346\\
2.41158738197971	2.42225418535761\\
2.39810962268442	2.34645031439725\\
2.37380833079006	2.24696727030339\\
2.3318526433345	2.11286577582151\\
2.26047519304961	1.92710354457398\\
2.13948575923478	1.66360065413531\\
1.93599803799582	1.28569213576058\\
1.6049581204052	0.754385732317796\\
1.1116028431317	0.0634672083856049\\
0.486210425470154	-0.705837383852517\\
-0.155193166771875	-1.40136778907375\\
-0.697401134220052	-1.9200360919257\\
-1.10321661268172	-2.26194328464883\\
-1.39223713720369	-2.47567262411068\\
-1.59721657004419	-2.60798306054165\\
-1.74514557972819	-2.69064598612555\\
-1.85463854625831	-2.74299009034185\\
-1.93787834147517	-2.77646035973368\\
-2.00279362154754	-2.79787835832038\\
-2.05461279398557	-2.81139102620275\\
-2.096854717804	-2.8195803565799\\
-2.13194167177978	-2.82409297020052\\
-2.16157885351556	-2.82600241442489\\
-2.18699247474654	-2.82602211732219\\
-2.20908191952115	-2.82463350550535\\
-2.22851896072046	-2.82216487363621\\
-2.24581377331859	-2.8188409846643\\
-2.26135971457817	-2.81481485408586\\
-2.27546425534414	-2.81018842718826\\
-2.28837070089014	-2.80502616081054\\
-2.30027366745368	-2.79936395201235\\
-2.31133024281672	-2.79321492096289\\
-2.32166810233478	-2.78657298305792\\
-2.33139142673926	-2.77941478337059\\
-2.34058518567037	-2.77170032704136\\
-2.34931815648383	-2.76337246931098\\
-2.35764490647126	-2.75435529488997\\
-2.36560685442574	-2.74455129426235\\
-2.37323242620906	-2.73383711406644\\
-2.38053621180108	-2.72205749859815\\
-2.38751689890873	-2.70901682261767\\
-2.39415357385661	-2.69446730265018\\
-2.40039970249558	-2.67809250334192\\
-2.40617366261997	-2.6594840268271\\
-2.41134397360421	-2.63810811816795\\
-2.41570614112739	-2.61325705037652\\
-2.41894590520874	-2.5839770657284\\
-2.42057989803758	-2.54895945903643\\
-2.41985785096006	-2.5063725201115\\
-2.41559776917939	-2.45359674618959\\
-2.40590158809854	-2.38679936516417\\
-2.38765383558727	-2.30024009281017\\
-2.3556235539057	-2.18513341673795\\
-2.30085516358809	-2.02782512370027\\
-2.2078958747088	-1.80712729616441\\
-2.05068074657191	-1.49156354189462\\
-1.78935724797637	-1.0410346164601\\
-1.37892684098203	-0.426330064342468\\
-0.809611013312882	0.320429150179243\\
-0.158861852805466	1.07182267954521\\
0.442957258278228	1.68480962051844\\
0.916848606964636	2.11037621371517\\
1.26024549942691	2.38157864841056\\
1.50333812085555	2.54967090031518\\
1.67696961654767	2.65407864288348\\
1.80381260079955	2.71975355658681\\
1.89896626186733	2.76157779184113\\
1.97225037057874	2.78837006608586\\
2.03008883770538	2.80543387882379\\
2.0767591327988	2.81603146028465\\
2.11517249949107	2.82221875161342\\
2.14735572388684	2.82532186585778\\
2.17475110014827	2.82621406094749\\
};
\addplot [color=mycolor1, forget plot]
  table[row sep=crcr]{%
2.14185167975353	2.82811886201345\\
2.16375472099596	2.82743931153048\\
2.1829554808675	2.82561810950134\\
2.19998381115138	2.82290302535754\\
2.21524596830833	2.81946356060583\\
2.22905797280445	2.81541406673926\\
2.24166887153974	2.81082888159783\\
2.25327721545887	2.80575228780801\\
2.26404288803612	2.80020501357924\\
2.27409568351791	2.79418834106226\\
2.28354156224498	2.78768647908714\\
2.29246720103243	2.78066759198863\\
2.30094324708977	2.77308369344406\\
2.30902653528962	2.76486947593528\\
2.31676141495498	2.75594002685406\\
2.32418023496131	2.74618726101363\\
2.33130293876362	2.73547475649129\\
2.33813560686435	2.72363049229966\\
2.3446676308791	2.71043671862902\\
2.35086697694649	2.69561579141908\\
2.35667264118912	2.67881018875666\\
2.36198282021328	2.65955395646265\\
2.36663634360813	2.63723126333636\\
2.37038322814755	2.61101516183607\\
2.37283722330557	2.57977529934503\\
2.37339778982183	2.54193586381645\\
2.37111887335268	2.49525205412532\\
2.36448275111581	2.43645056417114\\
2.3510006908997	2.36063993576219\\
2.32649277166447	2.26033118819243\\
2.2837747325805	2.12381882607868\\
2.21029948587487	1.93263172647691\\
2.084263967828	1.65818455267862\\
1.86994484410035	1.26020846482804\\
1.5193384815145	0.697504952025946\\
1.0006983097713	-0.0289546968303891\\
0.360049021024791	-0.817323633554986\\
-0.272206581709307	-1.50322890734363\\
-0.787566691977589	-1.99637801408196\\
-1.16381823972836	-2.31343899538887\\
-1.42823676075027	-2.50899376903536\\
-1.61471437745805	-2.62936567702651\\
-1.74911207095017	-2.70446720841903\\
-1.84867712973555	-2.75206417414387\\
-1.92451708705135	-2.78255795851524\\
-1.98380493175901	-2.80211837265008\\
-2.03125468735656	-2.81449094713054\\
-2.07003569180899	-2.82200876941633\\
-2.10233018110642	-2.82616180320071\\
-2.12967588442384	-2.82792326372033\\
-2.1531801245671	-2.82794119873254\\
-2.17365639415808	-2.82665375450862\\
-2.1917133623488	-2.82436019316507\\
-2.20781410744761	-2.82126560905936\\
-2.22231632455808	-2.81750962028811\\
-2.23550012474565	-2.81318505342203\\
-2.24758757759959	-2.808350221096\\
-2.25875665118115	-2.80303698494289\\
-2.26915127475398	-2.7972559568432\\
-2.27888866191015	-2.79099967643615\\
-2.2880646510665	-2.7842442757679\\
-2.29675756692326	-2.77694992385298\\
-2.30503093126348	-2.76906018715897\\
-2.31293522298438	-2.76050031582731\\
-2.32050878383483	-2.75117434743358\\
-2.32777787103906	-2.74096079067564\\
-2.33475575469844	-2.72970648920066\\
-2.34144062746594	-2.7172180427969\\
-2.34781190937511	-2.70324983788496\\
-2.3538242488797	-2.68748724591913\\
-2.35939806941366	-2.66952277861396\\
-2.3644047611806	-2.64882175930822\\
-2.36864333865978	-2.62467206252469\\
-2.37180314504003	-2.59610912865902\\
-2.37340316803587	-2.56180177567943\\
-2.37269115478618	-2.51987449893684\\
-2.36847187271704	-2.46762474132698\\
-2.35880747173788	-2.40106346667514\\
-2.34048239923269	-2.31415587911096\\
-2.30803121233605	-2.19755920777985\\
-2.25197055349214	-2.03656905299043\\
-2.15571304768967	-1.80808100704446\\
-1.99098591135525	-1.47748835353313\\
-1.71468104540268	-1.00116744224884\\
-1.28068252930592	-0.351115377356678\\
-0.68853915531236	0.425809568332435\\
-0.0341533716274803	1.18172463156856\\
0.547734367825091	1.77464307462038\\
0.991880077971176	2.17360147631448\\
1.30776112775099	2.42310769102853\\
1.52937635180833	2.57635806517277\\
1.6871706168914	2.67124391386155\\
1.80244009974611	2.73092572845202\\
1.88904281906363	2.76899036570571\\
1.95589065974379	2.79342860115457\\
2.00878314321231	2.8090324361412\\
2.05157406424349	2.81874849587472\\
2.08688552050022	2.824435670925\\
2.11654422500564	2.8272949813574\\
2.14185167975353	2.82811886201345\\
};
\addplot [color=mycolor1, forget plot]
  table[row sep=crcr]{%
2.10439665617207	2.8282298624729\\
2.12464188065137	2.82760148176032\\
2.14243234288805	2.82591382277774\\
2.15824659004586	2.82339212332122\\
2.17245231744745	2.82019056110771\\
2.18533615434218	2.8164130354986\\
2.19712446440675	2.81212678727802\\
2.2079981010471	2.80737135710589\\
2.21810301410286	2.80216441784895\\
2.22755795084046	2.79650543077886\\
2.23646007529932	2.79037770929812\\
2.24488905472067	2.78374923438941\\
2.25290997458345	2.77657239889961\\
2.26057530994044	2.76878272888608\\
2.26792607712629	2.76029651585636\\
2.27499219932393	2.75100717441069\\
2.2817920267666	2.74077999580701\\
2.28833084080954	2.7294447746609\\
2.29459801820751	2.7167855079095\\
2.30056230294151	2.70252594691874\\
2.306164269904	2.6863091344892\\
2.31130446643525	2.66766802527548\\
2.31582470136927	2.64598260556978\\
2.31947817809422	2.62041612789413\\
2.32188099444431	2.58981831435422\\
2.32243170733342	2.55257512298983\\
2.32017470859998	2.50637009777338\\
2.31356213260709	2.44779635654283\\
2.30002812104162	2.37171232033471\\
2.2752102358232	2.27015675826242\\
2.23150840732329	2.13052870772527\\
2.15545993513703	1.93268391936682\\
2.02337848732076	1.6451206627233\\
1.79632734061758	1.22354909389806\\
1.42353359793657	0.625220579339345\\
0.878771555207121	-0.138018972691501\\
0.227351529764195	-0.940000491971334\\
-0.388387094450216	-1.60829303205258\\
-0.871991154686601	-2.07120251933972\\
-1.21698954120204	-2.36197680636123\\
-1.45675809677698	-2.53931410504228\\
-1.62520636758421	-2.64804972105209\\
-1.74661180451078	-2.71589014876617\\
-1.83671977780629	-2.75896476121326\\
-1.90554053800263	-2.78663504150652\\
-1.95950195902525	-2.80443718838813\\
-2.00282013971824	-2.8157317148965\\
-2.03832968303853	-2.82261478727292\\
-2.06798441766638	-2.8264279081047\\
-2.09316335521594	-2.82804944289212\\
-2.1148614193132	-2.8280657096879\\
-2.13381102198757	-2.8268740115869\\
-2.15056131628629	-2.82474621291356\\
-2.16553098346993	-2.82186884012359\\
-2.17904410894577	-2.81836885520921\\
-2.19135502121799	-2.81433046201439\\
-2.20266577905565	-2.80980615459494\\
-2.21313866319811	-2.80482396386185\\
-2.22290520522737	-2.79939210980444\\
-2.23207276452311	-2.79350180566567\\
-2.24072932584376	-2.78712866617418\\
-2.24894696417187	-2.7802329739112\\
-2.25678426640306	-2.77275891335017\\
-2.26428788314733	-2.76463276304384\\
-2.27149328874582	-2.75575992192013\\
-2.27842473786471	-2.7460205166516\\
-2.28509430723748	-2.73526317200658\\
-2.2914997822963	-2.7232962958047\\
-2.29762096294645	-2.70987589060062\\
-2.3034136760499	-2.69468838494862\\
-2.30880031805775	-2.67732616054952\\
-2.3136549741375	-2.6572521367095\\
-2.31777982205	-2.633747608672\\
-2.32086716431094	-2.60583389424976\\
-2.32243714590556	-2.57215208683478\\
-2.32173325091376	-2.53077426119804\\
-2.31754253154096	-2.47890003515275\\
-2.30787822319011	-2.41235776020951\\
-2.28940538814651	-2.32476930466821\\
-2.25638205356715	-2.20614149639993\\
-2.19870458744134	-2.04054099553217\\
-2.09845541803663	-1.80262170813905\\
-1.92480803984779	-1.45417862427278\\
-1.63113994870503	-0.947953115606942\\
-1.1713375961867	-0.259161487476996\\
-0.557980698113119	0.545889993495298\\
0.0933622157606264	1.29865044866206\\
0.648758293724474	1.86480006513732\\
1.06001811194664	2.23430726860776\\
1.34770707165212	2.46157149408546\\
1.5481469960902	2.60018432009282\\
1.69063965301853	2.68586874206893\\
1.79484746233692	2.73982193018783\\
1.87332468029321	2.77431381426718\\
1.93407508714821	2.79652186191819\\
1.98228898802755	2.81074462611771\\
2.02141253032432	2.81962733025033\\
2.05379189979266	2.82484177055891\\
2.08106385814126	2.82747058619653\\
2.10439665617207	2.8282298624729\\
};
\addplot [color=mycolor1, forget plot]
  table[row sep=crcr]{%
2.06086420360006	2.82606288482026\\
2.07954946709803	2.82548265069804\\
2.09601333227688	2.82392060654805\\
2.11068597508046	2.82158074459897\\
2.12389875922843	2.81860278558362\\
2.13591063685797	2.81508075559912\\
2.14692662145578	2.81107517523556\\
2.15711091717045	2.8066210743889\\
2.16659637331059	2.80173319287336\\
2.17549135875597	2.79640920788889\\
2.18388478249776	2.79063150275514\\
2.19184974351591	2.78436777594098\\
2.19944612711559	2.77757063694663\\
2.20672234502208	2.77017621530973\\
2.21371632230606	2.76210169905723\\
2.22045574990295	2.75324160072957\\
2.22695753276466	2.74346240291337\\
2.23322625410922	2.73259503550656\\
2.23925132297113	2.72042434612167\\
2.24500223933518	2.70667428353889\\
2.25042103772944	2.69098682330781\\
2.25541034879725	2.67289155617102\\
2.25981445309361	2.65176103973526\\
2.26338882491472	2.62674395617998\\
2.26575027047244	2.59666286660416\\
2.26629346484606	2.55985413589397\\
2.26404769597393	2.51391112540796\\
2.2574242625088	2.45526195276396\\
2.24375881457129	2.37845948752776\\
2.21846221176634	2.274969050451\\
2.17342512124737	2.13110596293197\\
2.09407399513783	1.92471095921482\\
1.95447412087448	1.62083113027563\\
1.71205614108905	1.17076849580878\\
1.31391113289961	0.531696757122038\\
0.743197812322143	-0.268179794547498\\
0.0880735063622423	-1.07514858084357\\
-0.502153092938349	-1.71605812303739\\
-0.948894897869495	-2.14381008954486\\
-1.26111566886754	-2.40699538801196\\
-1.47628697745874	-2.56614725303258\\
-1.62720510986181	-2.66356637669721\\
-1.73615321181851	-2.72444379483013\\
-1.81726335384155	-2.76321533960986\\
-1.87943510189821	-2.78821089182298\\
-1.92836393046319	-2.8043517092476\\
-1.96778443891464	-2.81462919399292\\
-2.00021066507747	-2.82091400872068\\
-2.02737901295415	-2.82440695845339\\
-2.05051798869225	-2.8258967520207\\
-2.07051619978823	-2.82591144571268\\
-2.08802949379759	-2.82480982424699\\
-2.1035508863799	-2.82283791974444\\
-2.11745721549868	-2.82016474710658\\
-2.13004090895075	-2.81690532193666\\
-2.14153202214416	-2.81313569461374\\
-2.15211378353517	-2.80890283901094\\
-2.16193371986965	-2.80423112841719\\
-2.17111171051065	-2.79912646820911\\
-2.17974586158463	-2.79357874503894\\
-2.18791679254925	-2.78756298892149\\
-2.19569072795659	-2.78103946535897\\
-2.20312164729712	-2.77395278139286\\
-2.21025264096103	-2.76622997677119\\
-2.21711653288359	-2.75777745964334\\
-2.22373574576197	-2.7484765167114\\
-2.23012128791669	-2.73817695858281\\
-2.23627061292907	-2.72668822158121\\
-2.24216391578484	-2.71376689048698\\
-2.24775813592976	-2.6990990562707\\
-2.25297745790512	-2.68227505042866\\
-2.2576982894797	-2.66275268114385\\
-2.26172528803559	-2.63980274396403\\
-2.26475249081184	-2.61242658332343\\
-2.26629899592143	-2.57922854142314\\
-2.26559997429051	-2.53821383269632\\
-2.26141709345376	-2.48646024070003\\
-2.25169962860753	-2.41957196711612\\
-2.23296265066169	-2.33075292350808\\
-2.19912348829124	-2.20922159971954\\
-2.13932243627577	-2.03756007410029\\
-2.03403793857676	-1.78773830321613\\
-1.8494498252271	-1.41739349238135\\
-1.53527988504902	-0.875835879740883\\
-1.04748582114543	-0.144956876075536\\
-0.416552666778595	0.683547362854895\\
0.222686672033858	1.42271253383222\\
0.744243343940997	1.95457610884938\\
1.11956379822977	2.29186389120879\\
1.37853086268745	2.49645631370144\\
1.55815545418005	2.62067636550857\\
1.68588914043314	2.69748424140915\\
1.77953750501375	2.74596836911959\\
1.85030312395161	2.77706928414767\\
1.90528603804559	2.7971677366425\\
1.94908325868717	2.81008670307786\\
1.9847488763622	2.81818362897938\\
2.0143657302615	2.82295266501718\\
2.03939014961669	2.82536442418953\\
2.06086420360006	2.82606288482026\\
};
\addplot [color=mycolor1, forget plot]
  table[row sep=crcr]{%
2.00891888614438	2.82078560705859\\
2.02615105351259	2.82025020819442\\
2.04138134146949	2.81880495921137\\
2.05499436364695	2.81663386540661\\
2.06728730449655	2.81386303773614\\
2.07849310905539	2.81057718774165\\
2.0887967526755	2.8068304663362\\
2.09834683370745	2.80265358719083\\
2.1072639405311	2.79805842886958\\
2.11564674666219	2.79304085349802\\
2.12357646721724	2.78758219053757\\
2.13112009777789	2.78164964140228\\
2.13833271062673	2.77519572165534\\
2.14525897670313	2.76815674485985\\
2.15193399614463	2.7604502456637\\
2.15838344164903	2.75197112141917\\
2.1646229334839	2.74258612157438\\
2.1706564564729	2.73212610543142\\
2.17647347428268	2.72037518130714\\
2.18204415717123	2.70705536878776\\
2.18731175146273	2.69180468204638\\
2.19218046700744	2.67414532896778\\
2.19649613010137	2.65343672764823\\
2.2000148408192	2.62880466190764\\
2.20235120160236	2.59903203066476\\
2.2028907851417	2.56238623040231\\
2.20063819958206	2.51633934780944\\
2.19394581432378	2.45710273878048\\
2.18001545400075	2.37883434803851\\
2.15396010897569	2.27226688933048\\
2.10701532869381	2.12234499819615\\
2.02320496199439	1.90439750994316\\
1.87384438654784	1.57932410863569\\
1.61233813108486	1.0938464330092\\
1.18524109310871	0.408176807620506\\
0.590973948330326	-0.425113895273302\\
-0.0569771957156509	-1.22374196019026\\
-0.610890663283743	-1.82551182531642\\
-1.01557587894421	-2.21309612205472\\
-1.29369663941072	-2.44756019768786\\
-1.48442922186893	-2.58863746352841\\
-1.61834726432864	-2.67508007316861\\
-1.71537580525566	-2.72929425484456\\
-1.78794419320833	-2.76398050957151\\
-1.84383522588946	-2.78644933724247\\
-1.8880263018093	-2.80102607714932\\
-1.92378673181909	-2.81034847094563\\
-1.95332352668201	-2.81607261785346\\
-1.97816604215308	-2.81926604942691\\
-1.99940001727583	-2.8206328007354\\
-2.01781348573809	-2.82064601381138\\
-2.03399001271016	-2.81962821404895\\
-2.04836970007472	-2.81780113109942\\
-2.06128999748746	-2.81531730242975\\
-2.07301356805585	-2.81228048793155\\
-2.08374767205679	-2.8087590311147\\
-2.09365787595991	-2.80479465302263\\
-2.10287788718483	-2.80040819927139\\
-2.11151668915681	-2.79560327921136\\
-2.11966375314433	-2.79036837479183\\
-2.1273928436062	-2.78467776236583\\
-2.1347647587155	-2.77849142890533\\
-2.14182922407171	-2.77175404110533\\
-2.1486260635108	-2.76439291855656\\
-2.15518569053768	-2.75631485205166\\
-2.16152888367223	-2.7474014767175\\
-2.16766571440516	-2.73750273431448\\
-2.17359336831123	-2.72642770719064\\
-2.17929240878103	-2.71393172718295\\
-2.18472072980177	-2.69969807263371\\
-2.18980394329585	-2.68331162355013\\
-2.19442009162995	-2.66422030083433\\
-2.19837507527803	-2.64167752756128\\
-2.20136247904574	-2.61465451014093\\
-2.20289646410985	-2.58170334133977\\
-2.20219684149331	-2.54073794130088\\
-2.19798678378918	-2.48867432855378\\
-2.18812642630346	-2.42082483760648\\
-2.16893087296831	-2.32985653195647\\
-2.1338749289035	-2.20398573926515\\
-2.07113493439352	-2.02392816627601\\
-1.95918946251008	-1.75835244731557\\
-1.76067650217052	-1.36011692833705\\
-1.4218821918633	-0.776081944412222\\
-0.904606984461733	-0.000769264299073928\\
-0.26323362215036	0.841939097541932\\
0.351784892089782	1.55350881341773\\
0.831431807622443	2.04281570297915\\
1.1679223194889	2.34525963145284\\
1.39779849369872	2.52687794682429\\
1.55702877496454	2.63699263458927\\
1.67055879911057	2.70525662425323\\
1.7541482407559	2.7485304122958\\
1.8176132090598	2.77642083912154\\
1.86715809602388	2.79453009809739\\
1.90680295862112	2.80622323505773\\
1.93922490440843	2.81358303291265\\
1.96625534626517	2.81793503296059\\
1.98917915602654	2.82014390649056\\
2.00891888614438	2.82078560705859\\
};
\addplot [color=mycolor1, forget plot]
  table[row sep=crcr]{%
1.94484137383948	2.81093474107766\\
1.96074547841375	2.81044028934514\\
1.97485329975312	2.80910128846411\\
1.9875066311284	2.80708302175642\\
1.99897060680077	2.80449883772538\\
2.00945384083686	2.80142467844377\\
2.01912260826149	2.79790864308162\\
2.02811098393352	2.79397726781706\\
2.03652818247299	2.78963955849741\\
2.04446391838361	2.78488941551565\\
2.0519923310893	2.77970683640786\\
2.05917483669373	2.7740581098061\\
2.06606214117594	2.76789508755222\\
2.07269555549188	2.76115351547\\
2.07910767558369	2.7537502987825\\
2.08532241656818	2.74557945784403\\
2.0913543072874	2.73650637307818\\
2.09720684250933	2.72635969607218\\
2.10286953093216	2.71491997250002\\
2.10831302764683	2.7019035098721\\
2.11348133002652	2.68693920708129\\
2.11827931989926	2.66953473105818\\
2.12255271702205	2.64902619986201\\
2.12605531829587	2.6245017198388\\
2.12839434568635	2.59468243768769\\
2.12893701888778	2.55773275922355\\
2.12664639890571	2.51094936877903\\
2.11978435331607	2.45023777811195\\
2.10535805591883	2.36920952937128\\
2.07806245337108	2.25760066386937\\
2.0282387279616	2.09852454951016\\
1.93805368929678	1.8640500671643\\
1.77538712578065	1.51006526183966\\
1.48949970851246	0.979304317245783\\
1.03007431292167	0.241476455505507\\
0.419288702040341	-0.615550762137571\\
-0.205214267148519	-1.3858245453107\\
-0.710233576241256	-1.934731191373\\
-1.06777672003211	-2.27723504612438\\
-1.31073289439563	-2.48206333033099\\
-1.47730305556258	-2.6052645536573\\
-1.59479054014759	-2.68109634887966\\
-1.68045100632792	-2.72895477760171\\
-1.74494270113762	-2.75977779842846\\
-1.79493183413157	-2.77987210230925\\
-1.83469310631565	-2.79298630915439\\
-1.86704631039807	-2.80141951316405\\
-1.8939042740545	-2.80662379841013\\
-1.91659908828351	-2.80954060534081\\
-1.93608103539775	-2.81079415696333\\
-1.95304303474956	-2.81080598165623\\
-1.96800047775809	-2.80986459578593\\
-1.98134368007364	-2.80816896161207\\
-1.9933731150669	-2.80585617766109\\
-2.00432356697461	-2.80301943279653\\
-2.01438099583165	-2.79971978771933\\
-2.02369450740841	-2.79599393431396\\
-2.03238496811591	-2.79185925149242\\
-2.04055127247485	-2.78731697238645\\
-2.04827493068313	-2.78235396204598\\
-2.05562342072536	-2.77694339766353\\
-2.06265259796019	-2.77104449764707\\
-2.06940834678536	-2.76460133187308\\
-2.07592757486766	-2.75754064221577\\
-2.08223857638946	-2.74976849232457\\
-2.08836071411339	-2.74116543002282\\
-2.0943032764969	-2.73157966067934\\
-2.10006323587494	-2.7208174601858\\
-2.10562143584944	-2.70862964564853\\
-2.11093641784446	-2.69469227718988\\
-2.11593456480623	-2.67857872496947\\
-2.12049432251631	-2.65971851899145\\
-2.12442063129081	-2.63733549520008\\
-2.12740273215173	-2.61035271796791\\
-2.12894294327842	-2.57724272324503\\
-2.12823326147808	-2.53578540274212\\
-2.12393536894113	-2.4826658651701\\
-2.11377659503182	-2.41278885724156\\
-2.09378679643449	-2.31808502786372\\
-2.05682850107485	-2.18541886954297\\
-1.98978033523441	-1.99304251906782\\
-1.86853263904077	-1.70545123782526\\
-1.65156467308278	-1.27022149623518\\
-1.28293800086662	-0.634646104230499\\
-0.737052344015008	0.18398021550805\\
-0.098257033791466	1.02392006128289\\
0.476762529669879	1.68961487921372\\
0.905933475822523	2.12757086932723\\
1.20098228351461	2.39279553822506\\
1.40158390766722	2.55128457826433\\
1.54091048947349	2.64762983803274\\
1.64081458187349	2.70769620331022\\
1.71485558795971	2.74602353869501\\
1.77144028197587	2.77088807459421\\
1.81588842973417	2.7871328303387\\
1.85165952558558	2.79768228501973\\
1.88106812247197	2.80435722786817\\
1.90570546513969	2.80832331021653\\
1.92669343243673	2.81034517052612\\
1.94484137383948	2.81093474107766\\
};
\addplot [color=mycolor1, forget plot]
  table[row sep=crcr]{%
1.86245943044403	2.79384728481862\\
1.87719543502277	2.79338878598613\\
1.89032623322453	2.79214220986903\\
1.90215354364192	2.79025543168445\\
1.91291262912871	2.78782990592033\\
1.92278952809222	2.78493333572121\\
1.93193325090473	2.78160803032954\\
1.94046454086296	2.77787638382461\\
1.94848224298122	2.77374436072882\\
1.9560679706476	2.7692035335771\\
1.96328952997607	2.76423199665512\\
1.97020340681034	2.75879432743904\\
1.97685651220648	2.75284065093106\\
1.9832872994753	2.74630476024209\\
1.98952629549268	2.73910114180536\\
1.9955960190075	2.73112062779437\\
2.00151017625338	2.72222422964287\\
2.00727191226183	2.71223446324973\\
2.01287072863135	2.7009231085034\\
2.01827741105682	2.68799377045739\\
2.02343586479672	2.67305668527072\\
2.0282499908962	2.65559169200297\\
2.03256238150448	2.63489272300859\\
2.03611914740736	2.60998272367851\\
2.03851057489655	2.57948003739637\\
2.03906841393883	2.54138299418465\\
2.03668297320485	2.49271294721524\\
2.02946742013352	2.42890633522266\\
2.01412311941155	2.34275423411632\\
1.98471102608089	2.22252873580691\\
1.93026212035532	2.04873132203026\\
1.83035376990038	1.78902783774467\\
1.64857280295573	1.39347046401649\\
1.33103518000575	0.803821593145323\\
0.838526356380404	0.0123651422389971\\
0.227394302186296	-0.845931946906854\\
-0.350442449398308	-1.55919759768021\\
-0.792727001687744	-2.04011003837801\\
-1.09849427935741	-2.33304905062391\\
-1.30556878808597	-2.50762108621555\\
-1.44839577807619	-2.61325125302515\\
-1.5500787806155	-2.67887462553143\\
-1.62495459176856	-2.72070262573729\\
-1.68186320611161	-2.74789805962356\\
-1.72635986223458	-2.76578228194085\\
-1.76203238599039	-2.77754636466723\\
-1.79126603796787	-2.78516531107247\\
-1.81569144812738	-2.7898974168338\\
-1.83645257630568	-2.79256507392194\\
-1.85437113561667	-2.79371753788091\\
-1.87005014771785	-2.7937280692581\\
-1.88394082426028	-2.79285349083442\\
-1.89638679433492	-2.79127159116071\\
-1.9076539962759	-2.78910510451997\\
-1.9179512873667	-2.78643733752859\\
-1.92744491188462	-2.78332245749679\\
-1.93626882061528	-2.77979227183243\\
-1.9445321310142	-2.77586062439388\\
-1.95232457527549	-2.77152610482186\\
-1.9597204993907	-2.76677349436795\\
-1.96678178833739	-2.76157418987653\\
-1.97355996345272	-2.75588571626226\\
-1.98009760410597	-2.74965033109007\\
-1.98642917078006	-2.74279262357416\\
-1.99258123797982	-2.73521589730112\\
-1.99857207095989	-2.72679698194947\\
-2.00441038565776	-2.7173789181426\\
-2.01009299557332	-2.70676066179558\\
-2.01560083891203	-2.69468249590454\\
-2.02089253581957	-2.68080511086198\\
-2.02589404403578	-2.66467913195244\\
-2.03048196669752	-2.64569990157713\\
-2.03445624511432	-2.62303895654064\\
-2.03749460805865	-2.59553774446397\\
-2.03907476422507	-2.56153854272313\\
-2.0383378448221	-2.51860812301941\\
-2.03384155505112	-2.46307344849564\\
-2.02310021273392	-2.38922080245886\\
-2.00170367524217	-2.28788674924078\\
-1.96160130236399	-2.14397601173785\\
-1.88781061634304	-1.93230359683834\\
-1.75275925925105	-1.61201577623333\\
-1.51037323548226	-1.12576982920136\\
-1.10632840669905	-0.428838530197344\\
-0.539072435537281	0.422527341385257\\
0.0748267257916535	1.23045553389087\\
0.590332220910206	1.82760354405671\\
0.960489345185461	2.20542733579409\\
1.21194722364818	2.43147322229943\\
1.38332012285093	2.56686017357665\\
1.50332175285635	2.64983329006234\\
1.59021747160967	2.70207230964653\\
1.65524948912868	2.73573203272077\\
1.70540363675527	2.75776811117796\\
1.74512825602826	2.77228466203469\\
1.77733825522655	2.78178258186608\\
1.80399917194136	2.7878329240668\\
1.82647257532337	2.79144994386784\\
1.84572528341125	2.79330408577433\\
1.86245943044403	2.79384728481862\\
};
\addplot [color=mycolor1, forget plot]
  table[row sep=crcr]{%
1.75108552716988	2.76447222447479\\
1.76488221422702	2.76404250582548\\
1.77724897716815	2.7628680858765\\
1.78845029864056	2.76108084132467\\
1.79869390550384	2.75877123151757\\
1.80814521289382	2.7559992047306\\
1.81693762491815	2.75280140880389\\
1.8251799843703	2.7491959003046\\
1.83296202077276	2.74518509199559\\
1.84035836173541	2.74075739148653\\
1.84743148519643	2.73588779376549\\
1.85423386213361	2.73053755456242\\
1.86080944715795	2.72465296307848\\
1.86719460207298	2.7181631319376\\
1.8734184728415	2.71097661217749\\
1.87950277219999	2.70297650298354\\
1.88546083530285	2.69401353487635\\
1.89129569572855	2.68389632399633\\
1.89699674425462	2.67237756415484\\
1.90253423198378	2.65913424202541\\
1.90785037105443	2.64373885549779\\
1.912844900282	2.62561677736276\\
1.91735139494529	2.60398177517081\\
1.92109767190124	2.57773622969796\\
1.92363808680396	2.54531280800104\\
1.92423467860873	2.50441642898222\\
1.92164237946125	2.45159196649833\\
1.91370900678464	2.38148059663562\\
1.89660921227259	2.28551375431683\\
1.863350321132	2.14961015011902\\
1.80088724098503	1.95028198099332\\
1.68504321311595	1.64918888648009\\
1.47455060025472	1.19108672751144\\
1.1172436115952	0.527194676546508\\
0.600838439980181	-0.303565110777849\\
0.0217110314597016	-1.11787439832514\\
-0.479295792600991	-1.73676848784417\\
-0.845266816030316	-2.13480615233511\\
-1.09566377390525	-2.37469239957717\\
-1.26664301717949	-2.51881458141324\\
-1.38634200994895	-2.60732445286478\\
-1.47293770426302	-2.66320074056089\\
-1.53767767061608	-2.69936009220179\\
-1.58756039597431	-2.72319384635182\\
-1.62704090309671	-2.73905920237903\\
-1.65903569364839	-2.74960854146887\\
-1.68550910797049	-2.7565067516797\\
-1.70782031556884	-2.76082825634916\\
-1.72693340842951	-2.76328338763886\\
-1.74354789771438	-2.76435137514603\\
-1.75818186757464	-2.76436071503665\\
-1.77122627072982	-2.76353901027093\\
-1.78298119838695	-2.76204459183504\\
-1.79368062347662	-2.75998696923435\\
-1.80350960739666	-2.75744024887193\\
-1.81261647449423	-2.75445200543617\\
-1.82112155809316	-2.75104912596252\\
-1.82912356382167	-2.74724156652161\\
-1.83670424214872	-2.74302460222861\\
-1.84393183205323	-2.73837991938817\\
-1.85086358368856	-2.73327573973745\\
-1.85754756019065	-2.72766604744722\\
-1.86402383820955	-2.72148888718672\\
-1.87032515965923	-2.71466359834352\\
-1.87647702221151	-2.70708672901466\\
-1.8824971214059	-2.69862621252646\\
-1.88839395763576	-2.68911315896781\\
-1.89416427351294	-2.67833026766958\\
-1.89978875254881	-2.66599532654949\\
-1.90522502065326	-2.651737399139\\
-1.9103963233033	-2.63506187913752\\
-1.91517306923613	-2.61529820075033\\
-1.91934228271213	-2.59151986780491\\
-1.92255598803098	-2.56241917110889\\
-1.92424181720092	-2.52610575635664\\
-1.92344382818246	-2.47977378495112\\
-1.91853047983872	-2.4191366990536\\
-1.90664281269112	-2.33744360035013\\
-1.88262584056747	-2.2237426721095\\
-1.8369431681246	-2.05985534486197\\
-1.75176729647091	-1.81557078204867\\
-1.5949202394313	-1.44359327983578\\
-1.31697355185953	-0.885815607926106\\
-0.875685625693936	-0.123991058980787\\
-0.309394011606504	0.726956426715789\\
0.24468052449709	1.4568868622189\\
0.678903972736431	1.96012466793852\\
0.982576432505579	2.27011531894379\\
1.18906142341872	2.45571542385598\\
1.33156579852306	2.56827739628308\\
1.43294892172296	2.63836437217429\\
1.50752689674116	2.68319025734382\\
1.5641525322609	2.7124937892268\\
1.60839096210157	2.73192732354029\\
1.64383384795358	2.74487692142254\\
1.67286639677993	2.75343629233736\\
1.6971172922832	2.75893855142175\\
1.71772785572109	2.76225488140204\\
1.73551709087349	2.76396740628574\\
1.75108552716988	2.76447222447479\\
};
\addplot [color=mycolor1, forget plot]
  table[row sep=crcr]{%
1.59153714869271	2.71283556446399\\
1.60476885427873	2.71242284118941\\
1.61672829479412	2.71128658950149\\
1.62764574399131	2.70954418885353\\
1.63770404439344	2.70727595560098\\
1.64705034381964	2.70453435482357\\
1.65580456300521	2.7013500981366\\
1.66406558417514	2.69773608650507\\
1.67191581871882	2.69368979286982\\
1.67942459554526	2.68919444322856\\
1.68665066610256	2.68421919232426\\
1.69364402000437	2.67871836758366\\
1.70044712881062	2.67262974952996\\
1.7070956719489	2.66587175041363\\
1.71361873731923	2.65833922744017\\
1.72003841862168	2.64989750103547\\
1.7263686373693	2.64037391124481\\
1.73261287780803	2.6295458885489\\
1.73876030068066	2.61712396012184\\
1.74477933195565	2.60272722453063\\
1.75060718655548	2.58584737245532\\
1.75613266133152	2.56579488854982\\
1.76116748404913	2.54161687298595\\
1.7653976787231	2.51196854115029\\
1.76829905973407	2.4749071956872\\
1.76898648679696	2.42755320362948\\
1.76593736086363	2.36551782061183\\
1.75647060848263	2.28191688309793\\
1.73574466156984	2.16565555246177\\
1.69482869501793	1.99851591129642\\
1.61718261347504	1.75076654277894\\
1.47352994683164	1.37732437680515\\
1.22051805952349	0.826298262825352\\
0.824104530899889	0.088737425767855\\
0.320869760337405	-0.722231139262628\\
-0.172427576851556	-1.41677043474581\\
-0.564628150823845	-1.90152232511448\\
-0.844164857945204	-2.20555524352634\\
-1.03761732801986	-2.39084560899199\\
-1.17307874967628	-2.50499544609819\\
-1.27057497062747	-2.57706625586182\\
-1.34296497124043	-2.62376301758625\\
-1.39835026011063	-2.6546892307529\\
-1.44189706837267	-2.67549040360413\\
-1.47697734489243	-2.68958397191685\\
-1.50585079022601	-2.69910168381581\\
-1.53007197621501	-2.70541126204121\\
-1.5507371833947	-2.70941263399981\\
-1.56863738283019	-2.71171095290876\\
-1.58435533435496	-2.71272050602903\\
-1.59832868964385	-2.71272876601862\\
-1.61089193830001	-2.71193681655228\\
-1.6223048854811	-2.7104853955052\\
-1.6327723668004	-2.70847195331515\\
-1.64245814281193	-2.705961951801\\
-1.65149485039195	-2.70299636918225\\
-1.65999123121105	-2.6995966271188\\
-1.66803744320138	-2.69576769607146\\
-1.67570899375596	-2.69149984374145\\
-1.6830696565729	-2.68676929722642\\
-1.69017361294467	-2.68153795039362\\
-1.69706697091202	-2.67575213645695\\
-1.70378874713416	-2.6693403819491\\
-1.71037133513253	-2.66220994472263\\
-1.71684041912074	-2.65424179609826\\
-1.72321421250686	-2.64528351033067\\
-1.72950178671067	-2.63513923500997\\
-1.7357000808777	-2.62355547256329\\
-1.74178889769397	-2.61020070304512\\
-1.74772270731284	-2.59463574492772\\
-1.75341723808441	-2.57626987072965\\
-1.75872731967868	-2.55429450216108\\
-1.76340965469847	-2.52758076017022\\
-1.76705891020817	-2.4945172764498\\
-1.7689952370932	-2.45274676514304\\
-1.76806082381912	-2.39872691925673\\
-1.76224153191576	-2.32698079683655\\
-1.74794563915439	-2.22879585130807\\
-1.71861081452158	-2.08997419296628\\
-1.66206523014974	-1.88716184619098\\
-1.55606064235132	-1.58313617346189\\
-1.36362833150289	-1.12657390203955\\
-1.04038019630243	-0.477214984779318\\
-0.579488947169715	0.319746071660618\\
-0.0653357731346624	1.09359382563594\\
0.383438388475592	1.68535391512193\\
0.717113301025199	2.07215344123786\\
0.949730301037335	2.3095776614735\\
1.11114866251162	2.45462986861789\\
1.2256305822835	2.54503016285746\\
1.30931556895403	2.60286539536645\\
1.37240768645625	2.64077716802134\\
1.42136037563239	2.66610339414253\\
1.46033399259921	2.6832198294233\\
1.49207970831041	2.69481567438568\\
1.51846579304674	2.70259274207218\\
1.54079383062352	2.70765720831968\\
1.55999247373734	2.71074520454552\\
1.57673895742309	2.71235644715024\\
1.59153714869271	2.71283556446399\\
};
\addplot [color=mycolor1, forget plot]
  table[row sep=crcr]{%
1.34916822602105	2.61884380257413\\
1.36255348536219	2.61842535800733\\
1.37480683187091	2.61726037653587\\
1.38612752561835	2.61545290337742\\
1.39667669922158	2.61307332288931\\
1.40658638148471	2.61016585503015\\
1.41596613668646	2.60675348673496\\
1.42490800439879	2.60284105382679\\
1.43349020388433	2.598416912692\\
1.44177991815768	2.59345345180439\\
1.44983536863337	2.5879065527139\\
1.45770731417958	2.58171399239607\\
1.46544004576872	2.57479266390242\\
1.47307188948927	2.56703436125339\\
1.48063516614287	2.55829970555158\\
1.48815547206729	2.54840955221251\\
1.49565002401685	2.53713286811873\\
1.50312461938875	2.52416952842272\\
1.5105684469781	2.50912563214587\\
1.51794544380942	2.49147756217942\\
1.52517994674875	2.47051874931118\\
1.53213268434057	2.4452792850012\\
1.53856001621684	2.41440198359606\\
1.5440434126235	2.37594709462455\\
1.54786479332811	2.32707784393597\\
1.54878118856405	2.26354409964061\\
1.54460916042535	2.17882346641676\\
1.53144942360156	2.06269711309166\\
1.50225566229098	1.89898924372039\\
1.44436890588002	1.66251346478269\\
1.33620870262374	1.31721418768455\\
1.14673141186483	0.824026867167908\\
0.849782608285716	0.175946554623806\\
0.461685962801752	-0.548018956645931\\
0.0565516337693074	-1.20228090782282\\
-0.290753796572131	-1.6917502022023\\
-0.554835633179842	-2.01816874854801\\
-0.746358627160081	-2.22639575742822\\
-0.884817909103233	-2.3589451739538\\
-0.986659923035106	-2.44472049770824\\
-1.0634337309332	-2.50144568703854\\
-1.12282433770986	-2.53974002418693\\
-1.16991128279668	-2.5660218412918\\
-1.20809299884232	-2.58425316968421\\
-1.23968783716858	-2.59694161257033\\
-1.26631202944042	-2.60571445312303\\
-1.28911705407786	-2.61165259675128\\
-1.30894048914882	-2.61548905733766\\
-1.32640356331209	-2.617729744945\\
-1.34197546868493	-2.61872870075034\\
-1.35601659862283	-2.61873598932488\\
-1.36880817441296	-2.61792878298912\\
-1.38057291538259	-2.61643186607089\\
-1.39148970600711	-2.61433131778502\\
-1.40170416735516	-2.6116836843858\\
-1.4113363853036	-2.608522081478\\
-1.42048663037264	-2.6048601311055\\
-1.42923963244693	-2.60069429697314\\
-1.43766779293806	-2.5960049543598\\
-1.44583359306231	-2.59075637031308\\
-1.45379136809174	-2.58489564336501\\
-1.46158854894265	-2.57835053799512\\
-1.46926641317011	-2.57102602853142\\
-1.47686032733914	-2.56279922031462\\
-1.48439939058245	-2.55351211746392\\
-1.49190528944329	-2.54296141961962\\
-1.49939002248124	-2.53088409664441\\
-1.50685190825312	-2.51693681519435\\
-1.5142688787915	-2.50066621316221\\
-1.52158734818893	-2.48146525837629\\
-1.5287036798198	-2.4585079950171\\
-1.53543297058567	-2.43064999568616\\
-1.54145557404778	-2.39627321363943\\
-1.5462236004709	-2.35303882637033\\
-1.54879377227572	-2.29748516474355\\
-1.54752207454328	-2.22436239290059\\
-1.53949639900754	-2.12552421645835\\
-1.51947921042305	-1.98811596811773\\
-1.47800070676288	-1.79185510593083\\
-1.39834292507678	-1.50606847102606\\
-1.25373377570828	-1.09096289236972\\
-1.01217443181423	-0.516876345325173\\
-0.663483593147455	0.185309718307808\\
-0.255650441715128	0.892281146234285\\
0.127141962370146	1.46932437295849\\
0.432991150703737	1.87280614464764\\
0.658442536758492	2.13409571476104\\
0.821074581862312	2.30001032816487\\
0.939487523829889	2.40636173951891\\
1.02762555404293	2.47592484154283\\
1.09493604306106	2.52242215305209\\
1.14766234075684	2.55409174160465\\
1.18995036713984	2.57596124909194\\
1.22459961632259	2.59117267921784\\
1.25354049557583	2.60173993143481\\
1.27813357584593	2.60898559194935\\
1.29935845276156	2.61379764430181\\
1.31793477108438	2.61678384887558\\
1.33440123755647	2.61836680110905\\
1.34916822602105	2.61884380257413\\
};
\addplot [color=mycolor1, forget plot]
  table[row sep=crcr]{%
0.967127731322892	2.44416335727229\\
0.982332194010056	2.44368626762445\\
0.996552044814818	2.44233274530933\\
1.00995827727741	2.44019085862995\\
1.02269424379618	2.43731666682272\\
1.03488178584442	2.43373960992044\\
1.04662592338407	2.42946583601168\\
1.05801847480111	2.42447989317166\\
1.06914086623855	2.41874501885892\\
1.08006630690967	2.41220210613007\\
1.09086144272083	2.4047672886263\\
1.10158754476246	2.39632794378619\\
1.1123012330301	2.38673674442344\\
1.12305466889795	2.3758031658969\\
1.13389505829192	2.36328154206065\\
1.1448631692255	2.34885430132771\\
1.15599034564836	2.33210831537199\\
1.16729313058477	2.31250121396592\\
1.17876398162417	2.28931282755925\\
1.19035546509962	2.26157423670833\\
1.20195337503873	2.22796263054996\\
1.21333074848429	2.18664338354513\\
1.22406849048966	2.13503020764076\\
1.23341712189669	2.06941885226787\\
1.24005475817105	1.98443112174681\\
1.24166587486424	1.87219700793212\\
1.23423019400315	1.72125667828972\\
1.21092403527738	1.51545106534366\\
1.1608076636804	1.23401181971476\\
1.06853441869398	0.856195783028522\\
0.918509277596076	0.37568184812346\\
0.706921090474597	-0.177265218634865\\
0.454153574876131	-0.731103204332548\\
0.198053204246737	-1.21018280203609\\
-0.0297170443403976	-1.57844849886536\\
-0.216534441156716	-1.84171786003203\\
-0.364040069802484	-2.02391708406028\\
-0.479373086051614	-2.14919888770883\\
-0.570079113544991	-2.23595700009679\\
-0.642374348028329	-2.29679725082808\\
-0.700942767541163	-2.34003907604897\\
-0.74920207237023	-2.37113492729172\\
-0.789628565488254	-2.39368484478053\\
-0.824025247785614	-2.41009884374503\\
-0.853718838415478	-2.42201651725366\\
-0.879698514691617	-2.43057160046117\\
-0.902712302398931	-2.43655996902277\\
-0.923334050260447	-2.44054766045806\\
-0.942010285885441	-2.44294135282445\\
-0.959093334892685	-2.44403503307794\\
-0.974865011083859	-2.44404132101184\\
-0.989553777211925	-2.44311272620087\\
-1.00334733594278	-2.44135616819522\\
-1.01640198546763	-2.43884288475897\\
-1.02884965678783	-2.43561509341293\\
-1.04080326848977	-2.43169028198276\\
-1.05236084290732	-2.42706367824267\\
-1.06360869408717	-2.42170922170841\\
-1.07462390235819	-2.41557919023216\\
-1.08547621833959	-2.40860249140423\\
-1.0962294803223	-2.40068149139744\\
-1.10694257411915	-2.39168710090024\\
-1.11766990440936	-2.38145164553381\\
-1.12846126953048	-2.36975878532062\\
-1.13936092020188	-2.35632936855015\\
-1.15040540856971	-2.34080153894434\\
-1.161619549011	-2.32270254857145\\
-1.1730093313512	-2.30140838005199\\
-1.1845497978101	-2.27608515356535\\
-1.19616443885553	-2.24560290782785\\
-1.20769006987176	-2.20840694698507\\
-1.21881648820972	-2.16232344046854\\
-1.22898182152812	-2.10426306489148\\
-1.23718963783854	-2.02976886695529\\
-1.24168907404894	-1.93233760331378\\
-1.23942408473097	-1.80245496695398\\
-1.22513423035626	-1.62642196393417\\
-1.19008816520814	-1.38559159763217\\
-1.12101043541084	-1.05813668864252\\
-1.0014442958404	-0.627979956134983\\
-0.81975968717137	-0.104520799938123\\
-0.583468119003452	0.459358633320534\\
-0.324135709273898	0.983425913460733\\
-0.0793924327840767	1.4085297153344\\
0.12836183073586	1.72185853875973\\
0.294802905494589	1.94133196694191\\
0.425234391644833	2.09237330396133\\
0.527378472679993	2.19648606699859\\
0.608201715833219	2.26901494489367\\
0.673135098941547	2.32022387243635\\
0.726187730882803	2.35684603539622\\
0.770268586484437	2.38330546508546\\
0.807488419671494	2.40254198929109\\
0.839391495499743	2.41653933886846\\
0.86712141444169	2.42665821415481\\
0.891536825910838	2.43384682854805\\
0.913291732109646	2.43877538523974\\
0.932891444261152	2.44192316766833\\
0.950731920128378	2.44363578722103\\
0.967127731322892	2.44416335727229\\
};
\addplot [color=mycolor1, forget plot]
  table[row sep=crcr]{%
0.389833257663135	2.13494205399992\\
0.411637012190412	2.13425332784037\\
0.432830583088378	2.13223175394764\\
0.453560693998461	2.12891569696232\\
0.473962470616696	2.1243075922011\\
0.494162472193208	2.11837498054376\\
0.514281289056747	2.11104967794315\\
0.534435780021275	2.10222504132626\\
0.554740988262769	2.09175113467283\\
0.575311732094167	2.07942742190971\\
0.596263810567844	2.06499239801034\\
0.61771468137522	2.04810929394353\\
0.639783341588963	2.02834662784339\\
0.662588940613527	2.00515189276218\\
0.686247331235421	1.9778160401717\\
0.71086424216981	1.94542562462237\\
0.736522914514755	1.90679856576309\\
0.763262708752509	1.86039867194929\\
0.79104312719537	1.80422397231211\\
0.819684692661886	1.73566605619677\\
0.84877425633593	1.65134548513089\\
0.8775187845775	1.54694903865653\\
0.904532894477926	1.41714080153142\\
0.92756298448584	1.25570782784642\\
0.943208032311654	1.0562361419601\\
0.946821913125114	0.813724878526878\\
0.932953901423398	0.527382672453897\\
0.896710338603812	0.203985376296863\\
0.835916667242952	-0.140357545376625\\
0.752908499578301	-0.482727521836322\\
0.654348119758859	-0.800227870027934\\
0.548859926898402	-1.07695991065846\\
0.44418277966519	-1.30673881764637\\
0.345565275630988	-1.49126309862293\\
0.255665541234909	-1.63650765373541\\
0.175238506635367	-1.74971298877467\\
0.103924297554882	-1.83768399565498\\
0.0408371288553757	-1.90612416305119\\
-0.0150784938297504	-1.95954090229694\\
-0.0648626253155773	-2.00138952748404\\
-0.109458271066525	-2.03428043107457\\
-0.149685353120126	-2.06017497680803\\
-0.186241564546573	-2.08054652619508\\
-0.219714564140881	-2.09650461534767\\
-0.250597692592461	-2.10888771948177\\
-0.279305584592668	-2.11833147055275\\
-0.306188200451515	-2.12531846439242\\
-0.331542844513849	-2.13021452029508\\
-0.355624209521529	-2.13329503609864\\
-0.378652673165659	-2.13476408783637\\
-0.400821124569248	-2.13476816612125\\
-0.422300589621874	-2.13340588180217\\
-0.443244891729413	-2.13073456289361\\
-0.463794545707637	-2.12677436071603\\
-0.484080044336877	-2.12151025149346\\
-0.50422466156071	-2.11489213373237\\
-0.524346862829057	-2.10683306033622\\
-0.544562379318148	-2.09720548953563\\
-0.564985964787353	-2.0858352733634\\
-0.585732805647301	-2.07249290875139\\
-0.606919487137589	-2.0568813339019\\
-0.628664316710347	-2.03861923676051\\
-0.651086646328595	-2.01721842383059\\
-0.674304580568565	-1.9920532442971\\
-0.698430046329205	-1.96231935134776\\
-0.723559536426158	-1.92697821841399\\
-0.74975777688409	-1.88468292350284\\
-0.777029900022185	-1.8336801421848\\
-0.805275191803477	-1.7716840361193\\
-0.834211995258691	-1.69572212075661\\
-0.863259374614334	-1.60196627480951\\
-0.891359111601299	-1.48559354762292\\
-0.91672889412078	-1.34078686680571\\
-0.936571286668781	-1.16109958864773\\
-0.946851495651384	-0.940548679227383\\
-0.942415107275109	-0.675822351313804\\
-0.917855840029638	-0.369524113914628\\
-0.869352468419604	-0.0330937167554029\\
-0.79685460095464	0.313315627814919\\
-0.705042783184192	0.64580264069854\\
-0.601938366123092	0.944299305729285\\
-0.496026410028535	1.19775590143363\\
-0.393896012203038	1.4043430387089\\
-0.299443774043	1.56834629821078\\
-0.214272432014153	1.69666805894454\\
-0.138490527366839	1.79646760641875\\
-0.0714171469981265	1.87403677915985\\
-0.0120495608528579	1.93447222249387\\
0.0406752945867412	1.9817307191636\\
0.0877548063802227	2.01881897258306\\
0.130071524556737	2.04800035130604\\
0.168383188990257	2.07097447639075\\
0.203330610054358	2.08901939862155\\
0.235452276844371	2.10309921970234\\
0.265200305703159	2.11394374610395\\
0.292955363290979	2.12210681128536\\
0.319039707043503	2.12800878190824\\
0.34372819552645	2.13196747775331\\
0.3672574253493	2.13422062045018\\
0.389833257663135	2.13494205399992\\
};
\addplot [color=mycolor1, forget plot]
  table[row sep=crcr]{%
-0.306227207130532	1.68738862435859\\
-0.261379969916529	1.68595590112493\\
-0.214795732375831	1.68149627030184\\
-0.166246224137784	1.67371381817293\\
-0.115490049238434	1.66223302401073\\
-0.0622759237369259	1.64658738880165\\
-0.00634848347055719	1.62620670504928\\
0.0525421478533969	1.60040374645278\\
0.114625475832831	1.56836188210587\\
0.180088519800896	1.52912624559407\\
0.249044311164358	1.48160272123015\\
0.321488996788948	1.42457117256621\\
0.397246489756056	1.35672182432064\\
0.475901979302771	1.27672582252266\\
0.556730063408933	1.18335123520745\\
0.638630085725047	1.07563162763586\\
0.720089496514892	0.953082849635422\\
0.799202472199641	0.815943148058473\\
0.873769773125081	0.665385314239871\\
0.941490615215035	0.503628545179529\\
1.00022702608551	0.333879795582411\\
1.04828569470517	0.160073036160582\\
1.08464168652753	-0.0135568843967586\\
1.1090398574796	-0.182963393066682\\
1.12195182993611	-0.344713618929643\\
1.12441644185153	-0.496254799550155\\
1.11782296278774	-0.635992012894199\\
1.10369788255835	-0.763209101950142\\
1.08353585909217	-0.877896160245083\\
1.05868966724269	-0.980546613175455\\
1.03031467339685	-1.07196831526202\\
0.999354059786426	-1.1531310335292\\
0.966549613685775	-1.22505576466716\\
0.93246562408045	-1.28874165082839\\
0.897517326266407	-1.34512240587057\\
0.861998816568914	-1.39504383065693\\
0.826107890702841	-1.43925528372961\\
0.789966863130557	-1.47840971076188\\
0.753639324493781	-1.51306844756036\\
0.717143229840711	-1.54370829105481\\
0.680460870916968	-1.57072926138341\\
0.643546299255664	-1.59446211120164\\
0.606330711723059	-1.61517504828705\\
0.568726230975791	-1.6330793898466\\
0.530628433127489	-1.64833401267997\\
0.491917904745029	-1.66104853925081\\
0.45246105559447	-1.67128523096314\\
0.412110373885495	-1.67905956335708\\
0.370704287784431	-1.68433944479124\\
0.328066791675956	-1.68704301850036\\
0.28400701016003	-1.6870349644935\\
0.238318910849118	-1.68412120003892\\
0.1907814445447	-1.6780418757101\\
0.141159496489401	-1.66846259387974\\
0.0892061853211523	-1.65496386297681\\
0.0346672579765807	-1.63702898237799\\
-0.0227113930501748	-1.61403088740444\\
-0.0831717223145854	-1.58521905359259\\
-0.14692529189719	-1.54970847037251\\
-0.214126735869181	-1.50647406242591\\
-0.284836778083663	-1.45435584111197\\
-0.358973134768415	-1.39208244206276\\
-0.436249208267781	-1.31832313485091\\
-0.516103814772592	-1.23177981597386\\
-0.597630840345963	-1.13132891789912\\
-0.679525486295592	-1.01621571017297\\
-0.760071672073524	-0.886287460348263\\
-0.837198593872501	-0.742227607523675\\
-0.908626835330616	-0.585727435943091\\
-0.972101222376384	-0.419519954114775\\
-1.02567269773647	-0.247220301936271\\
-1.06796069544957	-0.0729730573829008\\
-1.09832162886203	0.0990210510858061\\
-1.11687756207078	0.264977523870333\\
-1.12440879632921	0.421882170576312\\
-1.12215751534027	0.567662524495283\\
-1.11160612860006	0.701180382220813\\
-1.09428272680449	0.822097483980406\\
-1.07162111358625	0.930680848521104\\
-1.04487938471457	1.02760274479467\\
-1.01510672826523	1.11376841378628\\
-0.983143214392898	1.19018455560286\\
-0.949638454263744	1.25786832554039\\
-0.915078603157057	1.31779009367543\\
-0.879814992991123	1.37084140862849\\
-0.844090707442856	1.41782027514089\\
-0.808063458887253	1.45942747136213\\
-0.771824346577996	1.49626934968383\\
-0.73541271568474	1.52886402113299\\
-0.698827615228074	1.55764892337876\\
-0.662036427415693	1.58298854439512\\
-0.624981212737102	1.60518158636642\\
-0.587583244066306	1.62446717822998\\
-0.549746121380202	1.64102993885522\\
-0.511357782671937	1.65500379988569\\
-0.472291663383129	1.66647454828795\\
-0.43240720889068	1.67548106428663\\
-0.391549913177206	1.68201522434357\\
-0.34955104250065	1.68602042054413\\
-0.306227207130532	1.68738862435859\\
};
\addplot [color=mycolor1, forget plot]
  table[row sep=crcr]{%
-0.788141800536172	1.24177643190908\\
-0.666143911461141	1.23782621644052\\
-0.529385546548301	1.22468264945793\\
-0.377598032925018	1.200305816704\\
-0.211503328462869	1.16270309148098\\
-0.033151035064384	1.11025323342796\\
0.153892741506608	1.04211045907784\\
0.344656280172416	0.958584774469102\\
0.533260689535979	0.861346057866915\\
0.713737270154117	0.753323480156497\\
0.880924531852972	0.63828695584622\\
1.03114656753452	0.52024313111183\\
1.16248175759686	0.402853793230899\\
1.27462492668523	0.28904266342515\\
1.36849308754741	0.180844573010999\\
1.44575813645	0.0794519643076454\\
1.50843747841042	-0.0146278478275331\\
1.55859879314411	-0.101382776585321\\
1.59817999649187	-0.181122138434487\\
1.62889940432551	-0.254339608923593\\
1.65222602201337	-0.321615605657848\\
1.66938482224438	-0.383553199496267\\
1.68137941856557	-0.440739234980044\\
1.68902122202907	-0.493722931482854\\
1.69295899129496	-0.543005874624971\\
1.69370577571634	-0.589039019145862\\
1.69166205939839	-0.632223739114816\\
1.68713488522892	-0.672915006659931\\
1.68035320404118	-0.711425502482463\\
1.67147987539793	-0.748029937218246\\
1.66062077750401	-0.782969164058048\\
1.64783144199166	-0.816453845999399\\
1.63312155709335	-0.848667545586881\\
1.61645760174872	-0.879769157777472\\
1.59776379377272	-0.909894624692575\\
1.57692146172094	-0.939157864666103\\
1.55376688433758	-0.967650822563707\\
1.52808758489613	-0.995442505897755\\
1.49961702314798	-1.02257681156117\\
1.46802760103714	-1.0490688693812\\
1.43292190168589	-1.07489952909151\\
1.39382213589897	-1.10000749601614\\
1.3501579134582	-1.12427848162136\\
1.30125274841025	-1.14753059274397\\
1.24631024376023	-1.16949507441256\\
1.18440182338321	-1.18979152402478\\
1.11445938149836	-1.20789696049985\\
1.03527852900531	-1.22310892923822\\
0.945541403358974	-1.23450458424801\\
0.843872161198516	-1.24090101572704\\
0.728942407956638	-1.24082761900118\\
0.599645483779305	-1.23252919451498\\
0.45535282880379	-1.21402729099758\\
0.296245415148313	-1.18327224528407\\
0.12367254535335	-1.13841001001318\\
-0.0595661179947407	-1.07815481856255\\
-0.249164163302196	-1.00219849022497\\
-0.439610195384136	-0.911522452056087\\
-0.624866148744066	-0.808460173525445\\
-0.799261981459283	-0.696429661152239\\
-0.958317522331549	-0.579398221318837\\
-1.099228135799	-0.461263700985717\\
-1.22091815479127	-0.345352723626748\\
-1.32375128971609	-0.23414923729855\\
-1.40907894887731	-0.129252037800879\\
-1.47879034409057	-0.0314877875681934\\
-1.53495731057438	0.0589073177118155\\
-1.57959879465035	0.142101542524296\\
-1.61454971079868	0.218510972211179\\
-1.64140474522503	0.288682572195118\\
-1.6615088143786	0.353214325527291\\
-1.67597278214956	0.412704793477366\\
-1.68570039259146	0.467723849275245\\
-1.69141815567655	0.518797624047021\\
-1.69370383155179	0.5664024541869\\
-1.69301154729384	0.61096420467578\\
-1.68969292068307	0.652860571574267\\
-1.68401425006494	0.692424841635785\\
-1.67617013167615	0.729950176179679\\
-1.66629395933414	0.765693866861409\\
-1.65446574894632	0.799881247186684\\
-1.64071766956537	0.83270908325435\\
-1.62503758428124	0.864348343181529\\
-1.60737082322744	0.894946278472889\\
-1.58762033419848	0.924627755549562\\
-1.56564528659377	0.953495759321603\\
-1.54125814301605	0.981630956651841\\
-1.51422016176231	1.00909015662244\\
-1.4842352568206	1.03590343561629\\
-1.45094212840502	1.06206960633996\\
-1.41390460262001	1.08754959936643\\
-1.37260021186315	1.11225719459006\\
-1.32640725558765	1.13604639638229\\
-1.27459098054248	1.15869461401389\\
-1.21629022788945	1.17988074243297\\
-1.15050707983079	1.19915734777188\\
-1.07610391598764	1.21591665067463\\
-0.991815076452057	1.22935121391861\\
-0.89628409556123	1.23841270101852\\
-0.788141800536172	1.24177643190908\\
};
\addplot [color=mycolor1, forget plot]
  table[row sep=crcr]{%
-0.759481808475352	0.954811303080049\\
-0.464414977518427	0.945239358918495\\
-0.137178419317819	0.913848151332419\\
0.205107567194767	0.859035732518902\\
0.540479022822526	0.783362229568471\\
0.848583269917223	0.693065650564576\\
1.11622371763869	0.595880372562254\\
1.33889721598359	0.498671849537412\\
1.51877538882173	0.406168785154704\\
1.66154340595806	0.320896882719042\\
1.7738798578266	0.243732277033466\\
1.86204322726675	0.174544792552995\\
1.93131125706806	0.1126930881872\\
1.98589505992412	0.0573370457405482\\
2.02905758504342	0.0076098029816707\\
2.06329054280205	-0.0372991895483263\\
2.09048543210269	-0.0781114817153033\\
2.11207672999923	-0.115453032641855\\
2.12915410498925	-0.149859865136441\\
2.14254744934964	-0.181788580002967\\
2.15289020636786	-0.211628032439913\\
2.16066612363249	-0.23971046643039\\
2.16624361766087	-0.266321434193131\\
2.16990095182832	-0.291708335604611\\
2.17184459428808	-0.316087645044589\\
2.17222246794996	-0.339650985413606\\
2.17113330922803	-0.362570232747666\\
2.16863298116275	-0.385001826950307\\
2.16473830582924	-0.407090443167509\\
2.15942876212331	-0.428972152863411\\
2.15264621487007	-0.450777177012557\\
2.14429268050241	-0.472632306258149\\
2.13422597635368	-0.494663032559375\\
2.12225292850496	-0.516995399962178\\
2.10811960970095	-0.539757532349487\\
2.09149782429426	-0.56308072320629\\
2.07196672858698	-0.587099860284975\\
2.04898804725626	-0.611952780355284\\
2.02187279783734	-0.63777786381841\\
1.98973676223325	-0.664708718811725\\
1.95144120150381	-0.692864065538602\\
1.90551470116897	-0.722329760700233\\
1.85005211103009	-0.753128096722297\\
1.78258862099129	-0.785166872057058\\
1.69995396028697	-0.818157275872694\\
1.59812923262143	-0.851486233524169\\
1.47216693255121	-0.884028798398856\\
1.31630617031597	-0.913898206754499\\
1.12452333785629	-0.938173284195572\\
0.891854337270322	-0.952738768752893\\
0.616726685582375	-0.952519250789799\\
0.303922965480029	-0.932459702043087\\
-0.0334113403580325	-0.889312146670426\\
-0.375047850181613	-0.823482720324207\\
-0.698972336787196	-0.739562366176011\\
-0.987934845355978	-0.644862943327322\\
-1.23316467192344	-0.546927933872179\\
-1.43387775149854	-0.451627708476155\\
-1.59438233199738	-0.362543496434898\\
-1.72110483307549	-0.281292466100066\\
-1.82062781970597	-0.208173978793897\\
-1.8987518250651	-0.142752382824162\\
-1.96021453734701	-0.0842579579464069\\
-2.00873229014713	-0.0318223597645517\\
-2.0471598069178	0.0153996296826504\\
-2.07766884025279	0.0581760899462323\\
-2.10190646029094	0.0971806018546626\\
-2.12112253576385	0.132993026489251\\
-2.13626770401609	0.166108231550759\\
-2.14806686295431	0.196947494882579\\
-2.15707362420401	0.225870042184848\\
-2.16371041277271	0.253183617661105\\
-2.16829789455687	0.279153715091212\\
-2.17107649345363	0.304011444539451\\
-2.17222201559771	0.327960160283018\\
-2.17185682732312	0.351181027234208\\
-2.17005760469696	0.373837707814064\\
-2.16686035050174	0.396080335234385\\
-2.16226312799609	0.418048915242808\\
-2.15622676391288	0.439876272143686\\
-2.14867360495218	0.461690627913561\\
-2.13948425462151	0.48361787463078\\
-2.12849205420061	0.505783567443355\\
-2.11547488614215	0.528314622955141\\
-2.10014365189416	0.551340648194742\\
-2.08212648761002	0.574994735392162\\
-2.06094740621443	0.599413416964235\\
-2.0359975682226	0.624735250132764\\
-2.00649677097877	0.651097138197798\\
-1.97144202503041	0.678626911798155\\
-1.92953936980179	0.70742976115612\\
-1.87911471112878	0.737564649699697\\
-1.81800030310009	0.769004637087864\\
-1.7433975081286	0.801571960947746\\
-1.65172775176408	0.834835094188598\\
-1.53850984188323	0.867952605280023\\
-1.39835502910763	0.899453036089182\\
-1.22526238417232	0.926964181094606\\
-1.01351140865888	0.94697152112484\\
-0.759481808475353	0.954811303080049\\
};
\addplot [color=mycolor1, forget plot]
  table[row sep=crcr]{%
-0.324618520603023	0.837628893195232\\
0.141649275301933	0.822760735661914\\
0.596881567596064	0.779453357448525\\
1.0021395912943	0.714932864138764\\
1.3371668551542	0.639646692660512\\
1.60075351461313	0.562611696507694\\
1.80261577585573	0.489443428944657\\
1.95563109944708	0.422718211680479\\
2.07167039486615	0.363082993210658\\
2.16024047217468	0.310200454343011\\
2.2284812634194	0.263332621855819\\
2.28160509550794	0.22164422755014\\
2.32338034966147	0.184339950814669\\
2.35653235040297	0.150715553516891\\
2.38304317205515	0.120168430000446\\
2.404366145207	0.0921909486791049\\
2.42157654482902	0.0663577986355005\\
2.43547688557959	0.0423123706696491\\
2.44667060615226	0.0197542087083929\\
2.45561387912914	-0.0015718189538681\\
2.46265225208374	-0.0218845843751558\\
2.46804668999558	-0.0413741934350738\\
2.47199212297803	-0.060208170850693\\
2.47463060535432	-0.078536598710198\\
2.47606051073955	-0.0964964436946383\\
2.47634271644072	-0.114215275566578\\
2.47550439601706	-0.131814549621168\\
2.47354079088737	-0.149412601717515\\
2.47041513469672	-0.167127486921179\\
2.46605673063414	-0.185079781328189\\
2.46035700892522	-0.203395460421398\\
2.45316319636694	-0.222208964955927\\
2.44426898542543	-0.241666564632794\\
2.4334012623827	-0.26193012664474\\
2.42020149376975	-0.283181382844243\\
2.4041997078324	-0.305626750788811\\
2.38477804171665	-0.329502671013201\\
2.36111941247666	-0.35508121863716\\
2.33213482280654	-0.382675321490496\\
2.29635992462281	-0.412642056840548\\
2.25180763874941	-0.445380800680903\\
2.1957592938514	-0.48131971990023\\
2.1244740268668	-0.520877920134854\\
2.03280193886641	-0.564379475232355\\
1.91371936967003	-0.611877464082452\\
1.75790851186681	-0.662822678866233\\
1.55376102383069	-0.715501866957153\\
1.28867822650361	-0.766243460037781\\
0.953077794663284	-0.808695686544094\\
0.547943309889831	-0.83411341498746\\
0.0929521451383333	-0.833937257179846\\
-0.373411518321853	-0.804349215721416\\
-0.807532662055344	-0.749203359525841\\
-1.17887417675215	-0.678000007762318\\
-1.47743180119758	-0.600914131355315\\
-1.70860296234771	-0.525324568057738\\
-1.88443811822814	-0.455202648837094\\
-2.01761943598438	-0.392020878970487\\
-2.11889173447411	-0.335838989529977\\
-2.19653701794795	-0.286066804736578\\
-2.25666864061581	-0.241892517007049\\
-2.3037204981097	-0.202490471586393\\
-2.34089583370865	-0.167107781766933\\
-2.3705167105481	-0.135091149775208\\
-2.39427849140452	-0.105886833462269\\
-2.41342985701326	-0.0790300757239985\\
-2.4288987866803	-0.0541315781974386\\
-2.44138060147967	-0.030864274957714\\
-2.4513997071776	-0.00895161256239827\\
-2.45935313333664	0.0118423818282623\\
-2.4655414085426	0.0317213486275997\\
-2.4701905387983	0.0508634012982991\\
-2.47346764623651	0.0694267567626212\\
-2.47549200150645	0.0875544409215963\\
-2.4763426177843	0.105378288172085\\
-2.4760631789495	0.123022421945924\\
-2.47466478847366	0.140606376124166\\
-2.47212680681249	0.158247996346675\\
-2.46839586278682	0.176066245797393\\
-2.46338295398581	0.194184031342863\\
-2.45695837006585	0.212731161827081\\
-2.4489439566827	0.231847549143729\\
-2.4391019566509	0.251686761427661\\
-2.42711927798657	0.272420030759712\\
-2.41258548757802	0.29424079441131\\
-2.39496202999984	0.317369787919756\\
-2.37353900245125	0.342060569360835\\
-2.34737411283272	0.368605057766904\\
-2.31520600437833	0.397338061428061\\
-2.27533076853194	0.428638560214717\\
-2.2254262700896	0.462923138093415\\
-2.16230492395649	0.500622438918312\\
-2.08157571644412	0.542123177117839\\
-1.97721206215354	0.587643828606011\\
-1.84108271073518	0.636990557641272\\
-1.66267094055474	0.689118820395049\\
-1.4295781838639	0.74144414246256\\
-1.12998729811571	0.789012731023362\\
-0.758476839361802	0.824122371675962\\
-0.324618520603024	0.837628893195232\\
};
\addplot [color=mycolor1, forget plot]
  table[row sep=crcr]{%
0.191200421204062	0.824265275129285\\
0.71325592275401	0.807963511485502\\
1.16345759404792	0.765425392551587\\
1.52252052148191	0.708448771096694\\
1.79587642946785	0.647122748221874\\
1.9997464806358	0.587587061317099\\
2.15129352137004	0.532674364167705\\
2.26469168011602	0.483229473891889\\
2.35052395916227	0.439117036612124\\
2.41636375999719	0.399802546765864\\
2.46755317204297	0.364641438372434\\
2.50785972415183	0.333007084390018\\
2.5399593089937	0.304338968099849\\
2.56577253858966	0.278154207307022\\
2.58669425502056	0.254043669529567\\
2.60374943665117	0.231662595783897\\
2.61769948976084	0.210720119939289\\
2.62911526822383	0.190969460988905\\
2.63842769866772	0.172199374681048\\
2.64596319880481	0.154226933192088\\
2.6519686414066	0.1368914962194\\
2.65662901872332	0.120049672038544\\
2.66007991001322	0.103571061667474\\
2.66241615420358	0.0873345957210633\\
2.66369765410891	0.071225294808439\\
2.66395290566747	0.0551313030145495\\
2.66318060062165	0.0389410570502846\\
2.66134945605483	0.0225404598091692\\
2.65839625066598	0.00580992571133416\\
2.65422187103344	-0.0113788442080452\\
2.64868496619965	-0.0291665176937357\\
2.64159254520199	-0.0477104051036007\\
2.63268648800917	-0.0671895773652435\\
2.62162441457903	-0.0878111005514818\\
2.60795257653423	-0.109817710668383\\
2.59106725814224	-0.133497293444217\\
2.57015937826012	-0.159194529467751\\
2.54413424393695	-0.187324931307483\\
2.51149425802899	-0.218391047844017\\
2.47016627992243	-0.252999427725799\\
2.41724696483688	-0.291874127123704\\
2.34862987349136	-0.335856254401098\\
2.25847372891593	-0.385865586756277\\
2.13849379012915	-0.442773260758363\\
1.97717055072386	-0.507086259797404\\
1.75931890171518	-0.578280346889523\\
1.46727702516943	-0.653613616569171\\
1.08617723612172	-0.726578100557776\\
0.615460439420353	-0.786230272795211\\
0.0822835338711193	-0.819919980671972\\
-0.458755077233071	-0.82004513526283\\
-0.94927335353285	-0.789253224250595\\
-1.35445595364017	-0.738024537394433\\
-1.66900199975997	-0.677853055182354\\
-1.90537723635566	-0.616892504600364\\
-2.08108780044098	-0.55946756251732\\
-2.21202005649414	-0.507262547905664\\
-2.31051905310019	-0.46053519056928\\
-2.38556530379483	-0.418899791716337\\
-2.44352386353713	-0.38174322250028\\
-2.48887853447455	-0.34842024462932\\
-2.52480076725475	-0.318334243633819\\
-2.55355433630708	-0.290963369918397\\
-2.57677354714022	-0.265862593178887\\
-2.59565238416989	-0.242656271453801\\
-2.61107314174097	-0.221027883359438\\
-2.62369442650236	-0.200709758129085\\
-2.63401188594597	-0.181473866908048\\
-2.64240050775657	-0.163123947349628\\
-2.6491443324095	-0.145488902178262\\
-2.65445744888948	-0.128417291597656\\
-2.65849884811449	-0.111772710978315\\
-2.66138285164333	-0.0954298538359701\\
-2.66318625772025	-0.0792710803233651\\
-2.66395295068996	-0.0631833320099094\\
-2.66369643639807	-0.0470552498199148\\
-2.66240055017646	-0.0307743616965002\\
-2.66001840294916	-0.0142242090219379\\
-2.65646945892056	0.00271872448881402\\
-2.65163445089512	0.0201884314122112\\
-2.64534760931634	0.0383333196120541\\
-2.63738537336875	0.0573208771826185\\
-2.62745031680829	0.0773433114590995\\
-2.61514838243709	0.0986244601731538\\
-2.5999565620817	0.121428321698415\\
-2.58117670599581	0.146069576303246\\
-2.55786892677725	0.172926416392208\\
-2.52875468402557	0.202455744887065\\
-2.49207457729266	0.235210057075738\\
-2.44537863236689	0.271853482986669\\
-2.38521758771343	0.313170239913061\\
-2.30669532153084	0.360049484777172\\
-2.20284719577246	0.413411291637233\\
-2.06386455985811	0.474001591350687\\
-1.87639089359498	0.541924817937139\\
-1.62366965142184	0.615730639568154\\
-1.2884049236082	0.690980647379509\\
-0.861085217365527	0.758884641393307\\
-0.353732957529172	0.807004156430304\\
0.19120042120406	0.824265275129285\\
};
\addplot [color=mycolor1, forget plot]
  table[row sep=crcr]{%
0.613287781213732	0.86010797340398\\
1.11511674731886	0.844659582097954\\
1.51768457820977	0.806751425025845\\
1.82376588317216	0.758240526604769\\
2.0509352373577	0.707296184889737\\
2.21880388620532	0.658276776173363\\
2.34371427611363	0.613012334620282\\
2.43782537863024	0.571971938560831\\
2.50977234567591	0.534990445442697\\
2.56559421684612	0.501653186781069\\
2.6095150415272	0.471480657201323\\
2.64451273544842	0.444009402733239\\
2.67271096584353	0.418822514218511\\
2.69564373909931	0.395556952080649\\
2.71443341082278	0.373900962812272\\
2.72991101586408	0.353587826889258\\
2.74269832317876	0.334388753816769\\
2.75326437201219	0.316106104164191\\
2.7619648292203	0.298567354270926\\
2.7690696320498	0.281619874727851\\
2.77478251664214	0.265126448280041\\
2.7792548183247	0.248961399474088\\
2.78259512976479	0.233007195475165\\
2.78487586620252	0.217151380124197\\
2.78613741627532	0.201283709565552\\
2.786390289109	0.185293361868146\\
2.78561546044436	0.169066091981578\\
2.78376294122543	0.152481195182978\\
2.78074841529247	0.135408125132969\\
2.77644759298665	0.117702584711849\\
2.77068767443536	0.099201866301543\\
2.76323496938589	0.0797191598005111\\
2.75377722007983	0.059036467714995\\
2.74189842845168	0.0368956642359729\\
2.72704285592108	0.0129871103311403\\
2.70846311833792	-0.0130648973866359\\
2.68514459016501	-0.0417206655341929\\
2.65569413444983	-0.0735483955114565\\
2.61817476350412	-0.109252478769324\\
2.56985843389196	-0.149705358213237\\
2.50685675263342	-0.195977005304862\\
2.4235774834724	-0.249345084109551\\
2.31195942416062	-0.311243951870691\\
2.16051209060582	-0.383058764199705\\
1.95347159308953	-0.465579411248454\\
1.67116729490645	-0.557829172076192\\
1.29418386758641	-0.655100904556978\\
0.814779939119185	-0.746989151120519\\
0.253872314130891	-0.818278428838929\\
-0.332556849944329	-0.855612892402552\\
-0.87571841253119	-0.8559992687482\\
-1.32918499612098	-0.827708316243736\\
-1.68184436630816	-0.783209949142495\\
-1.94594384520362	-0.732724453171794\\
-2.14116067344381	-0.682388056643545\\
-2.28577545587929	-0.635124543600925\\
-2.39400860433696	-0.591965379762945\\
-2.47614130760329	-0.552996748802563\\
-2.53940003140643	-0.517896024319721\\
-2.58883226427853	-0.486201189920134\\
-2.62798013867384	-0.457434674716157\\
-2.65935451449369	-0.431154363145532\\
-2.68475737519204	-0.406970189330942\\
-2.70549880674259	-0.384545403180371\\
-2.72254321948034	-0.363591643989305\\
-2.73660859151485	-0.343862040311453\\
-2.74823448933603	-0.325144182371768\\
-2.75782918081735	-0.307253691506714\\
-2.76570258746931	-0.290028595005323\\
-2.77208950790246	-0.273324488302417\\
-2.77716604157746	-0.257010376088385\\
-2.78106115814751	-0.240965055130788\\
-2.78386470419543	-0.225073898559042\\
-2.78563269502162	-0.209225906815504\\
-2.78639042625926	-0.1933108961482\\
-2.78613370669727	-0.177216697306981\\
-2.7848283235279	-0.160826232686187\\
-2.78240767628803	-0.144014327774542\\
-2.77876833081706	-0.126644090570417\\
-2.77376302238152	-0.108562658267982\\
-2.76719034317613	-0.089596061029038\\
-2.75877993484113	-0.069542884450331\\
-2.74817139791829	-0.0481663218772056\\
-2.73488421319197	-0.0251840934520167\\
-2.71827456526713	-0.000255576276756123\\
-2.69747278524308	0.0270346355114722\\
-2.67129175488589	0.0571975662114773\\
-2.63809141127379	0.0908656856475478\\
-2.59557666716171	0.128823353730755\\
-2.54049503654738	0.172038704817913\\
-2.46818719690168	0.221686691139371\\
-2.37193666515798	0.279136317997162\\
-2.24209351508817	0.345838650164078\\
-2.06510027125647	0.422981151964812\\
-1.82303820318996	0.510667469332616\\
-1.49546056309268	0.606340464715187\\
-1.06683713755812	0.702604396053219\\
-0.54172868935217	0.78620186581936\\
0.0405998642924217	0.84168534999174\\
0.61328778121373	0.86010797340398\\
};
\addplot [color=mycolor1, forget plot]
  table[row sep=crcr]{%
0.926393358475887	0.917706271260934\\
1.38607263076893	0.903665969310752\\
1.7422016593669	0.870176702341646\\
2.00870321996272	0.827950270283077\\
2.20597730285032	0.783707671619406\\
2.3525210968257	0.740908734638542\\
2.46258707079647	0.701016603024936\\
2.54644387593162	0.664441849350182\\
2.61130548498574	0.63109723987738\\
2.66221582625019	0.600689035826832\\
2.70272100981106	0.572859548493884\\
2.73534034840694	0.547252309736575\\
2.76188629984635	0.523538805080332\\
2.78367988503753	0.50142682268343\\
2.80169603717409	0.480660429960683\\
2.81666241613587	0.461016464022597\\
2.82912726059294	0.442299855659283\\
2.83950647434844	0.424338836503721\\
2.84811662193418	0.406980457837514\\
2.85519822357702	0.390086551090085\\
2.86093225475325	0.373530120957746\\
2.86545178198856	0.357192100266858\\
2.86885001841987	0.34095836999222\\
2.8711856407957	0.324716937328195\\
2.87248589696127	0.308355158613597\\
2.87274779812196	0.291756886363926\\
2.87193749709468	0.274799407022081\\
2.86998777400157	0.257350015428979\\
2.86679335848258	0.239262040489446\\
2.86220358397008	0.220370090483013\\
2.85601155816496	0.20048422133587\\
2.84793859215063	0.179382641104984\\
2.83761197920958	0.15680244216955\\
2.82453322963001	0.132427693010208\\
2.8080323544497	0.105874023092661\\
2.78720144224183	0.0766686140974801\\
2.76079711622966	0.0442243342015672\\
2.72709580713587	0.00780680891789201\\
2.68367726847776	-0.0335059966993132\\
2.62709978763967	-0.0808694015873919\\
2.55241651917453	-0.135713101802578\\
2.45247556986243	-0.199749541256519\\
2.31698107075966	-0.274879353275059\\
2.13146950088213	-0.362838550985554\\
1.87691333707886	-0.46429770138626\\
1.53194166745117	-0.577051012339622\\
1.08127310409524	-0.693415470461945\\
0.532134511957492	-0.798831354735789\\
-0.0714694655728092	-0.8757747316268\\
-0.659035959560127	-0.913403657671121\\
-1.16938961652807	-0.913918239494097\\
-1.57651842842924	-0.888592637089835\\
-1.88539534489321	-0.849644055527742\\
-2.11474579460562	-0.805803912152524\\
-2.2845970351354	-0.762002705065793\\
-2.41138888813924	-0.720557114614023\\
-2.50728045874624	-0.682312710598663\\
-2.58089194300237	-0.647381474947317\\
-2.63825421303898	-0.615547964013723\\
-2.68359190121149	-0.586474698142311\\
-2.71988950170122	-0.559799491160688\\
-2.74928039355392	-0.535177979701714\\
-2.77330943370495	-0.512299438072088\\
-2.79310978287705	-0.490890090764409\\
-2.80952261211717	-0.470710936834829\\
-2.82317887741413	-0.451553467220147\\
-2.83455576976416	-0.433234848244482\\
-2.84401608759779	-0.415593254807294\\
-2.85183594086373	-0.39848360497849\\
-2.85822435538924	-0.381773742998982\\
-2.86333714558525	-0.365341024272321\\
-2.86728663107626	-0.349069215733003\\
-2.87014823964302	-0.332845608679193\\
-2.87196466935385	-0.316558233809178\\
-2.87274801434064	-0.300093061929035\\
-2.87248004864847	-0.283331064086473\\
-2.87111067944473	-0.266144988608059\\
-2.86855439809411	-0.248395686846582\\
-2.86468434858169	-0.229927781173933\\
-2.85932336565911	-0.210564413826294\\
-2.85223096645201	-0.190100738411087\\
-2.84308474492088	-0.16829571084183\\
-2.83145381965049	-0.144861596391372\\
-2.81676076664323	-0.119450430002067\\
-2.79822658503956	-0.0916364535841698\\
-2.77479031253233	-0.0608933422327805\\
-2.74499035284064	-0.0265649348789936\\
-2.70678760765723	0.0121714882088764\\
-2.65730028311564	0.0563488156370922\\
-2.59240679033264	0.107255467266795\\
-2.50616063962901	0.166465520453224\\
-2.38996807466756	0.235808451596563\\
-2.23156557364936	0.317172385697494\\
-2.01416104841412	0.411922540994109\\
-1.71698590740431	0.519583279163353\\
-1.32012399554165	0.635540125842404\\
-0.817175390612008	0.748615574455404\\
-0.232995168271264	0.841817858572328\\
0.371751247386078	0.899672684197672\\
0.926393358475885	0.917706271260934\\
};
\addplot [color=mycolor1, forget plot]
  table[row sep=crcr]{%
1.15917423619427	0.985452996961435\\
1.57605581690044	0.97276915929865\\
1.89441343642087	0.942843264494399\\
2.13227683472436	0.905150342403449\\
2.30942407846641	0.865413163486818\\
2.44232774805049	0.826589535101777\\
2.54329864806614	0.789986729526301\\
2.62113458251352	0.756032335676635\\
2.6820299188861	0.724722222994033\\
2.73034635185163	0.69585971565669\\
2.76917854378417	0.669176729546374\\
2.8007467931326	0.644392180882967\\
2.82666408408834	0.621238171226077\\
2.84811697000639	0.599470039069359\\
2.86598860922422	0.578868577273424\\
2.88094314711286	0.559238633514245\\
2.8934841634484	0.540406200432067\\
2.90399555917694	0.522215017548692\\
2.91277040371961	0.504523151960812\\
2.92003140075226	0.487199742292122\\
2.92594540855611	0.470121947586638\\
2.93063364090557	0.453172070861024\\
2.93417862736339	0.43623478986836\\
2.93662863152781	0.419194406100511\\
2.93799994881924	0.401932006621231\\
2.93827728833417	0.384322415691087\\
2.93741225391446	0.366230789939307\\
2.93531975068679	0.34750867822912\\
2.93187192760633	0.327989321208702\\
2.92688899133663	0.307481900752102\\
2.92012584763972	0.285764359574226\\
2.91125297782018	0.262574288144739\\
2.89982913907324	0.237597210458558\\
2.88526223146898	0.210451384549692\\
2.86675275384526	0.180667969274259\\
2.84321129383027	0.147665126580317\\
2.81313689528631	0.110714438865427\\
2.77443615632001	0.0688982370962552\\
2.72415278113033	0.0210578615885913\\
2.6580642866756	-0.0342625301334779\\
2.57009133700856	-0.0988597098543036\\
2.45147526364566	-0.174856072119426\\
2.28977247488069	-0.26451353004594\\
2.068051001765	-0.369643323146168\\
1.76555664303862	-0.49022874188461\\
1.3626712298896	-0.62196851668595\\
0.853498448339034	-0.753558732612605\\
0.263067043959383	-0.867084769427221\\
-0.34858168173437	-0.945253034235647\\
-0.911523896811392	-0.981455505235464\\
-1.38070599042454	-0.982007452859509\\
-1.7465684190745	-0.959275748618546\\
-2.02212325732471	-0.924530624568435\\
-2.22731078007852	-0.885302064978562\\
-2.3805439581278	-0.845777644851068\\
-2.49618350570722	-0.807969731568821\\
-2.58467037086583	-0.77267217971083\\
-2.65339185681341	-0.740056351422403\\
-2.70754271270146	-0.71000099655222\\
-2.75079235057624	-0.682263459422644\\
-2.78575807593243	-0.656564373723719\\
-2.8143293235244	-0.632627245573377\\
-2.83788741038465	-0.610195162061931\\
-2.8574545906673	-0.589036205167183\\
-2.87379582783211	-0.568943497701945\\
-2.88748893302291	-0.549732869739\\
-2.89897339195555	-0.531239620165527\\
-2.90858467669404	-0.513315071683582\\
-2.91657853317752	-0.495823221250298\\
-2.92314822799873	-0.478637586915617\\
-2.92843674480804	-0.461638250687693\\
-2.93254525685895	-0.44470904568158\\
-2.9355387481116	-0.427734808183106\\
-2.93744933310586	-0.410598597259897\\
-2.93827758337044	-0.393178768144887\\
-2.93799196850087	-0.375345765687661\\
-2.93652633419291	-0.35695847674322\\
-2.93377514082864	-0.337859941550254\\
-2.92958594539539	-0.317872169395212\\
-2.92374828895184	-0.296789727417868\\
-2.91597769794034	-0.274371665981091\\
-2.90589283948323	-0.250331200815598\\
-2.89298286274102	-0.224322382197802\\
-2.87656041322932	-0.195922740098191\\
-2.85569341655181	-0.164610613804974\\
-2.82910502268571	-0.129735616258361\\
-2.79502541026458	-0.0904806364713247\\
-2.75097066118413	-0.0458144539665412\\
-2.69341216109772	0.00556331091612401\\
-2.61728671989778	0.0652755618461577\\
-2.51529289624149	0.135290740074671\\
-2.37695842808772	0.217842197730919\\
-2.18764946166002	0.315079127957206\\
-1.92825794555101	0.42813693392327\\
-1.57755809935925	0.555223929896933\\
-1.12070581839738	0.688795150854699\\
-0.565356701070268	0.813805801254907\\
0.04470746832057	0.911337695620967\\
0.639826601490393	0.968452871849045\\
1.15917423619427	0.985452996961435\\
};
\addplot [color=mycolor1, forget plot]
  table[row sep=crcr]{%
1.33832785075596	1.05856770426014\\
1.71708873794132	1.04706270790821\\
2.00524308812722	1.01997360948505\\
2.22158293985867	0.98568232028984\\
2.38421685443873	0.949191101343268\\
2.5076145272811	0.91313611260959\\
2.6024663989092	0.87874493050746\\
2.67642270413419	0.846477797541033\\
2.7349083019084	0.816402633286111\\
2.78177963434076	0.78840014646188\\
2.81980118326959	0.762271550607141\\
2.85097669866004	0.737793146276745\\
2.87677601172378	0.714742657350381\\
2.89829006124147	0.692910811688235\\
2.91633737206639	0.672105353139954\\
2.93153778559506	0.652151275517667\\
2.9443639890652	0.632889257135411\\
2.95517784931997	0.614173310232574\\
2.96425621180258	0.595868146084142\\
2.97180927822365	0.577846482438437\\
2.97799364973239	0.559986372597205\\
2.982921432524	0.542168554727318\\
2.98666632910257	0.524273773221959\\
2.9892673015886	0.506179992999476\\
2.99073014005198	0.487759402000392\\
2.99102706084517	0.46887507006363\\
2.99009426746616	0.449377098720827\\
2.9878272023742	0.429098051468598\\
2.98407297275139	0.407847392434451\\
2.97861910870765	0.38540457648988\\
2.97117735555587	0.36151031754284\\
2.96136053270739	0.335855404031068\\
2.94864948746336	0.308066221173719\\
2.93234564125112	0.27768587209012\\
2.91150227341512	0.244149475236976\\
2.88482407104405	0.206751914642895\\
2.8505189770917	0.164606224499872\\
2.80607829448728	0.11659141871253\\
2.74795009456766	0.0612912218850788\\
2.67105933973035	-0.00306711709908142\\
2.56812617464324	-0.0786451112657921\\
2.4287765554169	-0.167922293729583\\
2.23862361750258	-0.273355629826909\\
1.97903645009366	-0.396452773264885\\
1.62946597084443	-0.535843685786365\\
1.17549772978648	-0.684372465629206\\
0.623913899953506	-0.827066784603145\\
0.0156603832950979	-0.944197989256905\\
-0.582713904244943	-1.02082682525928\\
-1.1109120882187	-1.05489053430368\\
-1.54004749962896	-1.05543374616649\\
-1.87136800357853	-1.03485403068786\\
-2.12118690488697	-1.00334708083351\\
-2.30861435903366	-0.967504177395819\\
-2.45007070231361	-0.931008400315427\\
-2.55806939607659	-0.895691241500445\\
-2.64167391994001	-0.862335439946601\\
-2.70732795253916	-0.831170891254812\\
-2.75960186319709	-0.802153688786899\\
-2.80175664512797	-0.775115414285042\\
-2.83614241827917	-0.749840185559272\\
-2.86447275609878	-0.726102868844767\\
-2.88801214554155	-0.703686833934361\\
-2.90770435438771	-0.682391113449945\\
-2.92426093572038	-0.662032191603923\\
-2.9382227944058	-0.642443165421628\\
-2.95000341218912	-0.623471699475948\\
-2.9599194431015	-0.604977491641939\\
-2.96821248644699	-0.58682959233689\\
-2.97506458571853	-0.568903718994405\\
-2.98060916195791	-0.55107959889393\\
-2.98493852032551	-0.533238312675461\\
-2.98810867137945	-0.515259573693341\\
-2.99014191896274	-0.497018851100946\\
-2.99102743998762	-0.47838421890837\\
-2.99071988462755	-0.459212783487904\\
-2.98913583058736	-0.439346503305103\\
-2.98614770441886	-0.418607162070534\\
-2.981574503371	-0.396790184171176\\
-2.97516826859042	-0.373656881793238\\
-2.96659470971629	-0.348924587454162\\
-2.95540556317111	-0.322253943280443\\
-2.94099902838946	-0.293232380082817\\
-2.92256272972903	-0.261352524893812\\
-2.898990733363	-0.22598395477037\\
-2.86876167890882	-0.186336478897755\\
-2.82975838133331	-0.141413299954517\\
-2.77899972415068	-0.08995380351499\\
-2.71224382577774	-0.030370353344629\\
-2.62341239539384	0.0393040537726912\\
-2.50379953054959	0.121410226009802\\
-2.34112351143517	0.218486204153281\\
-2.11880645853817	0.332683514259896\\
-1.81669174283804	0.464386512203611\\
-1.41581099600258	0.609717854256119\\
-0.910199242096203	0.757659551907771\\
-0.32300024499543	0.8900063214995\\
0.289093551297695	0.98803779850949\\
0.858233627317148	1.04278813347991\\
1.33832785075596	1.05856770426014\\
};
\addplot [color=mycolor1, forget plot]
  table[row sep=crcr]{%
1.48195524741105	1.13499600551441\\
1.82771084245232	1.12449792923133\\
2.09121070631347	1.09971889927061\\
2.29056202585284	1.06811011905157\\
2.44200222497211	1.03412112200209\\
2.5582243973705	1.00015506933652\\
2.64857921337322	0.967388428797628\\
2.71979414609436	0.936312684715988\\
2.77668243987642	0.907055225989295\\
2.82270084726161	0.879559355166109\\
2.86035316545087	0.85368207890044\\
2.89147238204279	0.829245848782393\\
2.91741560862578	0.806065029860011\\
2.93919865663493	0.783958663217387\\
2.95758940438435	0.762755871451644\\
2.97317306982624	0.74229736750022\\
2.98639822665132	0.722434940873958\\
2.9976094924628	0.703029924367371\\
3.00707087054564	0.683951164254744\\
3.01498242644814	0.665072750318575\\
3.02149210661462	0.646271611685118\\
3.02670391018261	0.627424996845516\\
3.03068320802233	0.60840780146909\\
3.03345969887125	0.589089668398643\\
3.03502825503851	0.569331749997934\\
3.03534770547678	0.548982986595672\\
3.03433740348466	0.52787571031182\\
3.03187120178483	0.505820325345909\\
3.02776817638863	0.482598737354212\\
3.0217790587727	0.457956097902255\\
3.01356679025977	0.431590285446009\\
3.00267880784298	0.403138350895174\\
2.98850746087563	0.372158903955268\\
2.9702331191046	0.33810910603951\\
2.94674172850081	0.300314597460034\\
2.91650432599898	0.257930436006693\\
2.87739975294452	0.209891288074655\\
2.82645308816859	0.154850528636367\\
2.75945194470782	0.0911125846386886\\
2.67039604942962	0.0165753827145177\\
2.55075154984237	-0.0712701882940495\\
2.38857573601612	-0.175171812629412\\
2.16788371645304	-0.297546184921392\\
1.86937402705675	-0.439126103017382\\
1.47485565200519	-0.596498607285113\\
0.978024766685259	-0.759158540164599\\
0.399280619236233	-0.909031700309866\\
-0.209222691531337	-1.02637010365967\\
-0.782516564403628	-1.09990426092498\\
-1.27340608465234	-1.13162079951875\\
-1.66619789036641	-1.13213477534486\\
-1.96860925157471	-1.11334737001622\\
-2.19781258107627	-1.08443074321809\\
-2.37139232321267	-1.05122597349927\\
-2.50386060594156	-1.01704058497466\\
-2.60616221235153	-0.983579677611965\\
-2.68624082346912	-0.951625326146515\\
-2.74978669526022	-0.921457337484466\\
-2.80087536645179	-0.893094774491843\\
-2.84244511941849	-0.866429077725958\\
-2.87663525618585	-0.841295442698824\\
-2.90502076456061	-0.817510021061471\\
-2.92877422841182	-0.79488848704882\\
-2.94877782711592	-0.77325454830531\\
-2.9657013206889	-0.752443091626493\\
-2.98005679825384	-0.732300512698212\\
-2.99223742913519	-0.712683603755642\\
-3.00254507357401	-0.693457726910913\\
-3.0112100193343	-0.674494643567592\\
-3.01840504582332	-0.655670170619315\\
-3.0242552972786	-0.636861720027347\\
-3.02884494997068	-0.617945709956243\\
-3.0322213045204	-0.598794790290315\\
-3.03439666846907	-0.579274789687264\\
-3.03534817700269	-0.559241256781215\\
-3.03501550055304	-0.538535428402326\\
-3.03329617915488	-0.51697940709317\\
-3.03003807513199	-0.494370262762469\\
-3.02502811051957	-0.47047268187135\\
-3.01797600141845	-0.445009663280249\\
-3.00849104089377	-0.417650592306105\\
-2.99604899719174	-0.387995802807642\\
-2.97994470381437	-0.355556454808895\\
-2.95922364687653	-0.31972822362765\\
-2.93258239896873	-0.279756978818976\\
-2.89822256475082	-0.234694525213657\\
-2.85363542562227	-0.183343111872696\\
-2.79528466319329	-0.124190093700613\\
-2.71814479248552	-0.0553419146260905\\
-2.61505367687108	0.0255137852239566\\
-2.47588198710784	0.121044266512722\\
-2.28669913561442	0.233941461232089\\
-2.02961011797109	0.366015313191922\\
-1.68496239277464	0.516299897716017\\
-1.23873660878186	0.67815187200221\\
-0.696280598734906	0.83700673279516\\
-0.0946098487340973	0.972777609559879\\
0.503930688914962	1.06878127437136\\
1.03995889446949	1.12043375648529\\
1.48195524741105	1.13499600551441\\
};
\addplot [color=mycolor1, forget plot]
  table[row sep=crcr]{%
1.60151165444191	1.21376353859017\\
1.91865392825986	1.20413163687514\\
2.16146055251207	1.18128912214335\\
2.34678220315943	1.15189476983143\\
2.48906901537889	1.11995145575896\\
2.5994810729338	1.08767646335323\\
2.68624928650806	1.05620504317403\\
2.75533692442773	1.02605331630931\\
2.81105064312716	0.997396557024017\\
2.85651480026093	0.970229115480782\\
2.89401520611343	0.944453983022338\\
2.9252412308416	0.919931966326893\\
2.95145466158424	0.896508048638974\\
2.97360744874277	0.874024968197798\\
2.99242420165329	0.852329669320216\\
3.00846039089377	0.831275798408647\\
3.0221437148727	0.810724016880521\\
3.03380368338312	0.790541112544326\\
3.04369284201066	0.770598441972192\\
3.05200195991931	0.750769978441166\\
3.05887075305835	0.730930088589762\\
3.06439519462896	0.710951068268945\\
3.06863209306283	0.690700406598752\\
3.07160133872594	0.670037700497916\\
3.07328599384936	0.648811099381227\\
3.07363019438329	0.626853113692275\\
3.07253461944072	0.603975564926566\\
3.06984903396344	0.579963382149292\\
3.06536108641892	0.554566853084517\\
3.05878009508824	0.527491807468921\\
3.04971390971152	0.498387035823426\\
3.0376359782482	0.466828017325126\\
3.0218383119463	0.432295740259596\\
3.00136387252382	0.394149060626916\\
2.97490864014783	0.351588725528002\\
2.94067879239094	0.303611091883521\\
2.89618159570869	0.248950233714589\\
2.83791994276145	0.186009859425842\\
2.76095254672642	0.112794272297332\\
2.65828483846744	0.0268670495822667\\
2.52010059775551	-0.0745900511703682\\
2.33301050835554	-0.194457881804245\\
2.07994111740238	-0.334801432192292\\
1.74218505871029	-0.495034299008016\\
1.30608372407202	-0.669071004772725\\
0.775416924238896	-0.842931253667855\\
0.183026458477091	-0.996488532947787\\
-0.413353316298409	-1.11162359925376\\
-0.955704250961396	-1.18127373485712\\
-1.41002817879594	-1.21066319585894\\
-1.77044404461756	-1.21114015108714\\
-2.04826567042507	-1.19387320821478\\
-2.26032756649037	-1.16710896417554\\
-2.422531658596	-1.13607078891378\\
-2.54768386419904	-1.10376561082803\\
-2.64540258288297	-1.07179752334559\\
-2.72270144076245	-1.04094758760794\\
-2.78464714192121	-1.01153548207928\\
-2.83490465866034	-0.983631313694125\\
-2.8761431824996	-0.957175611324294\\
-2.91032521757151	-0.932045857202917\\
-2.93890886945483	-0.908092622076673\\
-2.96298879862279	-0.88515861147164\\
-2.98339470092799	-0.86308816155869\\
-3.00076053537135	-0.841731422246322\\
-3.01557354829834	-0.82094559995982\\
-3.02820923357816	-0.800594580985712\\
-3.03895638740859	-0.780547661154982\\
-3.04803507758448	-0.76067776795101\\
-3.05560943928836	-0.740859364190165\\
-3.06179658640004	-0.720966104807109\\
-3.06667248976738	-0.70086824380515\\
-3.07027535439509	-0.68042973592618\\
-3.07260677896861	-0.659504934019606\\
-3.07363076847567	-0.637934739640746\\
-3.0732704647743	-0.61554201418963\\
-3.07140223244709	-0.592125994445668\\
-3.06784645563073	-0.567455372607121\\
-3.06235402300107	-0.541259588542124\\
-3.05458694179213	-0.51321773089914\\
-3.04409073683727	-0.48294424294041\\
-3.03025511957711	-0.449970369159299\\
-3.01225764702602	-0.413719960902326\\
-2.98898242622094	-0.373477916714031\\
-2.95890193537095	-0.328349285909294\\
-2.91990424359213	-0.277207246542429\\
-2.86904005845689	-0.218629624450325\\
-2.80215512061681	-0.150828347646298\\
-2.71336889273852	-0.0715887173907743\\
-2.59437845303026	0.0217350657353069\\
-2.43365766506504	0.132058542911812\\
-2.21590322689587	0.262015160923235\\
-1.92274195279636	0.412645902646693\\
-1.53677203889217	0.581005147098066\\
-1.05126636470144	0.75720443482184\\
-0.483696118163764	0.923554599581091\\
0.118643869731508	1.05962467768482\\
0.694129470594868	1.1520422511544\\
1.19480147923485	1.20034609016361\\
1.60151165444191	1.21376353859017\\
};
\addplot [color=mycolor1, forget plot]
  table[row sep=crcr]{%
1.7041846009319	1.2943141820301\\
1.99626617055785	1.28543761178804\\
2.22124175614415	1.26426293819397\\
2.39453017416156	1.23676782393869\\
2.52896408637316	1.20657960123409\\
2.63438716042362	1.175756675815\\
2.71808169542672	1.14539515887809\\
2.78536305329583	1.11602783814595\\
2.84010543395788	1.08786756517456\\
2.88514673507583	1.06095024777421\\
2.9225829191502	1.03521712036419\\
2.95397691776185	1.01056136374741\\
2.98050567802373	0.986854067154508\\
3.00306368390844	0.963958280238492\\
3.02233615866452	0.941736206473757\\
3.03885115178787	0.920052443127552\\
3.05301683469795	0.898774933595854\\
3.06514832973588	0.87777458064277\\
3.07548702698937	0.85692405068559\\
3.08421440519897	0.836096051577946\\
3.09146172447689	0.81516121575212\\
3.0973165011175	0.793985624376277\\
3.1018263403014	0.772427941024521\\
3.10500044296881	0.750336069553689\\
3.10680888273521	0.727543199906796\\
3.10717953780926	0.703863049386844\\
3.10599233300521	0.679084038331082\\
3.10307016596914	0.652962050479293\\
3.09816551812357	0.625211310957289\\
3.09094122611171	0.595492758529565\\
3.08094312764365	0.563399082666323\\
3.0675611678925	0.528435331338291\\
3.04997386749585	0.489993674758581\\
3.0270685348402	0.447320569827702\\
2.99732588623989	0.399474334660572\\
2.95865242299797	0.34527135316296\\
2.90813688499269	0.283220630985283\\
2.84169952147278	0.211451189241431\\
2.75360015287075	0.127649168935877\\
2.63579015752125	0.0290504103860355\\
2.47718044409962	-0.0874048975346086\\
2.26315440049889	-0.224539491410399\\
1.97624101171958	-0.383676370857818\\
1.59977690800976	-0.562325166838529\\
1.12661068121587	-0.751246369018012\\
0.571426793541186	-0.933269951884109\\
-0.0231945330767768	-1.08754580197928\\
-0.599163055661934	-1.19885002630157\\
-1.10821263274332	-1.26428398020831\\
-1.5280243443977	-1.29146157859367\\
-1.85963020768602	-1.29189976322708\\
-2.11613328811536	-1.27594927750475\\
-2.31347293789909	-1.25103332489436\\
-2.46592186792284	-1.22185313681699\\
-2.58479381981758	-1.19116200190927\\
-2.67857891676389	-1.16047520182953\\
-2.75350344627299	-1.13056847341001\\
-2.81410388947056	-1.10179161552201\\
-2.86369313535435	-1.07425566637257\\
-2.90470726510581	-1.04794158899572\\
-2.93895397442576	-1.02276231742486\\
-2.96778790699597	-0.99859763792049\\
-2.99223398977876	-0.975313379666422\\
-3.01307442280304	-0.952771567434943\\
-3.03091037717764	-0.930835368339748\\
-3.0462060383433	-0.909371033620749\\
-3.05932022560958	-0.888248094174011\\
-3.07052916260407	-0.867338521470358\\
-3.08004284009438	-0.84651524436022\\
-3.08801663398478	-0.82565021998284\\
-3.09455929844392	-0.804612137619662\\
-3.09973806520067	-0.783263755080808\\
-3.10358128812587	-0.761458808292563\\
-3.10607883621961	-0.73903838347614\\
-3.10718022598436	-0.715826588655281\\
-3.10679026744153	-0.691625299739336\\
-3.10476174708494	-0.666207678808018\\
-3.10088434981452	-0.639310060399069\\
-3.09486858146107	-0.610621666132709\\
-3.08632282322089	-0.579771428161253\\
-3.07472072399117	-0.546310967265258\\
-3.0593547589155	-0.509692477113805\\
-3.03926972115841	-0.469239927412545\\
-3.01316684589102	-0.42411168591123\\
-2.97926479270935	-0.373252582288746\\
-2.9350975127546	-0.315334137109362\\
-2.87722141885989	-0.248684455917916\\
-2.80079795942275	-0.171217097031371\\
-2.69902250417858	-0.0803873652608693\\
-2.56241494987765	0.026752980820651\\
-2.37814204960349	0.153247837121803\\
-2.1299429759065	0.30138943179547\\
-1.80001296258586	0.470949407623367\\
-1.37501405570806	0.656404734221779\\
-0.857224232733782	0.844434285581143\\
-0.275494477662446	1.01507514468077\\
0.317042707769568	1.14906003328664\\
0.864103828984793	1.23699824488219\\
1.32963119236916	1.28194972632408\\
1.7041846009319	1.2943141820301\\
};
\addplot [color=mycolor1, forget plot]
  table[row sep=crcr]{%
1.79451333687795	1.37623258172961\\
2.06437894430077	1.36802412314627\\
2.27364278731825	1.34831904966155\\
2.43629400799905	1.32250318949409\\
2.56373188637826	1.29387890390197\\
2.66466706633369	1.26436245928459\\
2.74556856172955	1.2350096720366\\
2.81119281358986	1.20636206852063\\
2.86503696965654	1.17866094921984\\
2.90968598827017	1.15197566340107\\
2.94706591443675	1.12627917729905\\
2.97862492315894	1.10149207155819\\
3.00546185166876	1.07750782499653\\
3.02841748975463	1.05420703582886\\
3.04813968400119	1.03146509239063\\
3.06513002218468	1.00915594522162\\
3.07977748081739	0.987153535081539\\
3.09238274743192	0.965331782097081\\
3.10317577097379	0.943563653214854\\
3.11232829045301	0.921719589106468\\
3.1199625284934	0.89966542345103\\
3.12615683127849	0.877259828969968\\
3.13094873227639	0.854351252443963\\
3.13433567233171	0.830774240267158\\
3.13627339068008	0.806344996398444\\
3.13667178150951	0.780855947410237\\
3.13538775939515	0.754069006761072\\
3.13221435902192	0.725707123922968\\
3.12686486295436	0.695443563909494\\
3.11895013854585	0.662888178353199\\
3.10794647251382	0.627569690518339\\
3.09314987331339	0.588912719714877\\
3.07361085634227	0.546207932699933\\
3.04804084787039	0.498573404835621\\
3.01467720302991	0.444905216476886\\
2.97108820312788	0.38381604428972\\
2.91389271499949	0.313563319504904\\
2.83836419754006	0.231976297684306\\
2.73789479389656	0.136410169507705\\
2.60333854470715	0.0237965139468894\\
2.42239579536337	-0.109060546546738\\
2.17956106428595	-0.264668607987185\\
1.85784309023286	-0.443144421608707\\
1.4441719282026	-0.639516259375913\\
0.939457202380576	-0.841139668930312\\
0.368918685510627	-1.0283302451733\\
-0.218665528781946	-1.18090618141992\\
-0.768973098121028	-1.28733879088908\\
-1.24436510686666	-1.34848886147076\\
-1.63213016362057	-1.3736032385062\\
-1.93796849459237	-1.3740036794835\\
-2.1756594985174	-1.35921433796541\\
-2.36001623058481	-1.33592856434747\\
-2.50381134739832	-1.30839697445668\\
-2.61706185013918	-1.27915087612558\\
-2.70729043857883	-1.24962273688704\\
-2.78004683255238	-1.2205774311595\\
-2.83940783860141	-1.19238591443792\\
-2.88837747174335	-1.16519138917053\\
-2.92918459732128	-1.13900792106641\\
-2.96349738916597	-1.11377818139752\\
-2.99257591894407	-1.08940686552014\\
-3.01738044126433	-1.06577972293003\\
-3.03864843761554	-1.04277407823606\\
-3.05694970805371	-1.02026430306792\\
-3.07272598295598	-0.998124270223542\\
-3.08631952632432	-0.976227977914113\\
-3.09799380993182	-0.954449030816098\\
-3.1079483738202	-0.932659362888273\\
-3.11632931733683	-0.910727400847761\\
-3.12323638902248	-0.888515747455641\\
-3.12872729505156	-0.865878380687456\\
-3.13281957597797	-0.84265729997381\\
-3.13549017384011	-0.818678491650597\\
-3.1366725962201	-0.793747023326093\\
-3.13625135215725	-0.767641003014309\\
-3.13405305519659	-0.740104045476918\\
-3.12983322099334	-0.71083576621519\\
-3.1232572748037	-0.679479662680571\\
-3.11387354646997	-0.645607531759416\\
-3.10107494690942	-0.608699304418124\\
-3.08404441443837	-0.568116856511122\\
-3.06167684265302	-0.523070018181276\\
-3.03246673353144	-0.472572782098417\\
-2.9943459418341	-0.415387948339402\\
-2.94444958092933	-0.349959983556741\\
-2.87878171397394	-0.274340653051243\\
-2.79175109683607	-0.18612422426896\\
-2.67556694180055	-0.0824371959007606\\
-2.51956678073377	0.0399140019352076\\
-2.30978231456736	0.183929908341486\\
-2.02956844260352	0.351203194158337\\
-1.66291465278432	0.539685603279386\\
-1.20226587375373	0.740783582827908\\
-0.659782514063547	0.937902771892843\\
-0.0736961714596904	1.10995578588396\\
0.501412124460094	1.24010882029349\\
1.01738293052278	1.32311336938\\
1.44914772331576	1.36482948316528\\
1.79451333687795	1.37623258172961\\
};
\addplot [color=mycolor1, forget plot]
  table[row sep=crcr]{%
1.875380198175	1.45912264622704\\
2.12532607372983	1.45151275523711\\
2.32049313714941	1.43312663690383\\
2.47351961229483	1.40883077969227\\
2.59454470957599	1.38164056702845\\
2.69129962768452	1.35334146430337\\
2.76954939823841	1.32494670095129\\
2.8335622883477	1.29699922662878\\
2.88650196898471	1.26976072903191\\
2.93072649099663	1.2433268912786\\
2.96800716044344	1.21769670697208\\
2.99968601153698	1.19281378547366\\
3.02678847325747	1.16859071790335\\
3.05010403242527	1.14492320634226\\
3.07024420039637	1.12169798193793\\
3.08768436550023	1.09879692266456\\
3.10279412762091	1.07609881054382\\
3.11585930640243	1.05347958179882\\
3.12709783060887	1.0308115652349\\
3.13667102518602	1.00796198065811\\
3.14469132014911	0.984790824374708\\
3.15122704385755	0.961148168735424\\
3.15630468366305	0.936870825940929\\
3.15990876251869	0.91177825868291\\
3.16197926064529	0.885667551171414\\
3.1624062783217	0.858307174811905\\
3.16102135810059	0.829429184553865\\
3.15758452313218	0.798719355335712\\
3.15176558918009	0.765804602668832\\
3.14311759426987	0.73023681658467\\
3.13103914923143	0.691471966427712\\
3.11472098212371	0.648843010440361\\
3.09306970442142	0.601524809885759\\
3.06459858593356	0.548489034635153\\
3.02727063241863	0.488447303179922\\
2.97827358720255	0.419782361339985\\
2.91370094648205	0.340471870835639\\
2.82811273852315	0.248021431188159\\
2.71396950645002	0.139450861872087\\
2.56100997894323	0.011434072810402\\
2.35585577163837	-0.139207939611686\\
2.08258910796728	-0.314338081602671\\
1.72574780809793	-0.512344302816052\\
1.27738430552161	-0.725264891854567\\
0.747410964471658	-0.937093700635873\\
0.170151684734942	-1.12661859938555\\
-0.403149960273714	-1.27559530863182\\
-0.924739677714599	-1.37654191730408\\
-1.36723218025528	-1.43348961008313\\
-1.72543970187286	-1.45669521773722\\
-2.00807292192314	-1.45706014702998\\
-2.22891235056166	-1.44331104409805\\
-2.40157978026183	-1.42149356437748\\
-2.53749727933407	-1.39546326605403\\
-2.64555530926629	-1.36755240066736\\
-2.7324408719078	-1.33911374280202\\
-2.80311577466288	-1.31089573584426\\
-2.86125293005364	-1.28328246873827\\
-2.90958126713142	-1.25644161240548\\
-2.95014237282275	-1.2304139170949\\
-2.98447637900907	-1.20516677800092\\
-3.01375518981585	-1.18062600865979\\
-3.03887778105092	-1.15669444918053\\
-3.0605385408469	-1.13326260668455\\
-3.07927649620449	-1.11021444400938\\
-3.09551093137147	-1.08743018066274\\
-3.10956722998844	-1.06478721623195\\
-3.12169559672649	-1.04215982907142\\
-3.13208449045607	-1.0194180211406\\
-3.14087001922426	-0.9964257010892\\
-3.14814212743523	-0.97303827856067\\
-3.15394809027255	-0.949099656396012\\
-3.15829357749059	-0.924438536709023\\
-3.16114132532165	-0.898863889641476\\
-3.16240723260219	-0.872159360523098\\
-3.16195344608523	-0.844076303511913\\
-3.15957768583335	-0.814325018612185\\
-3.15499763915335	-0.782563624334002\\
-3.14782865669049	-0.748383809417997\\
-3.13755212396971	-0.711292464246528\\
-3.12347062022884	-0.670687892829954\\
-3.10464412095523	-0.625828968409226\\
-3.07979879451023	-0.575795296373511\\
-3.04719609928311	-0.519436404776792\\
-3.00444475117117	-0.455308738835184\\
-2.9482322176708	-0.381602060475576\\
-2.87394842006821	-0.296064553876908\\
-2.77518113622301	-0.195954326761797\\
-2.64310416802847	-0.0780848431611926\\
-2.46591133519486	0.0608914956915911\\
-2.22877348479082	0.223698787331001\\
-1.91540232194173	0.410797198251262\\
-1.5129282868656	0.617755172649118\\
-1.02102501419702	0.832596173905048\\
-0.461678914241442	1.03596879881983\\
0.120322339013655	1.20694603259918\\
0.672695012185963	1.33204369199023\\
1.15664250869975	1.40994398642384\\
1.5565348565067	1.44859625100981\\
1.875380198175	1.45912264622704\\
};
\addplot [color=mycolor1, forget plot]
  table[row sep=crcr]{%
1.94860967361305	1.54255432833086\\
2.1805183787397	1.53548649136023\\
2.36285721209787	1.5183012557299\\
2.50702155221514	1.49540564898627\\
2.62204553358807	1.46955805053239\\
2.71480973562695	1.44242159207405\\
2.79046529403399	1.41496449889465\\
2.85285038384193	1.387724683507\\
2.90483097030853	1.36097714777595\\
2.94855954345884	1.33483762364216\\
2.98566504672178	1.30932602862144\\
3.01739029515218	1.28440503990512\\
3.04469096502283	1.26000335267682\\
3.06830696683483	1.23602950040386\\
3.08881408202697	1.21237982371928\\
3.10666146617262	1.18894277034559\\
3.12219895498825	1.16560084981681\\
3.1356969214283	1.14223103857391\\
3.14736059175103	1.1187041016261\\
3.15734013078588	1.09488308635246\\
3.16573737353531	1.07062110330677\\
3.17260975468686	1.04575840770177\\
3.17797172733908	1.02011871395454\\
3.18179373411327	0.993504600659352\\
3.18399856897966	0.965691784078038\\
3.18445471760671	0.936421945322298\\
3.18296595418294	0.905393680381121\\
3.17925605963709	0.872250992770669\\
3.17294694888811	0.836568554960588\\
3.16352766531549	0.79783271720215\\
3.15031049243682	0.755416938424182\\
3.13236866825522	0.70854997319778\\
3.10844763558118	0.656274849324365\\
3.07683817380715	0.59739663051999\\
3.03519503220057	0.530417717636257\\
2.98027938081541	0.453462247419244\\
2.90760015624451	0.364198731229946\\
2.81093650936482	0.259788036163911\\
2.6817630125737	0.136922409161699\\
2.50872093805967	-0.00790498392072498\\
2.2775720519001	-0.177646504168353\\
1.97261341510108	-0.373116462736028\\
1.58109419230744	-0.590422057659901\\
1.10155200657516	-0.818241223317062\\
0.553103473504285	-1.03757477755222\\
-0.023057140602249	-1.22686053176458\\
-0.576640773712478	-1.37080621485596\\
-1.06801045513199	-1.46595664204834\\
-1.4789594019965	-1.51886494276565\\
-1.80996604361589	-1.54031035988743\\
-2.07155855448015	-1.54064245584021\\
-2.27711730768827	-1.52783705568473\\
-2.43909171547293	-1.50736341137314\\
-2.56770027554192	-1.48272667724163\\
-2.67085281140652	-1.45607779784481\\
-2.75450984985187	-1.42869176395499\\
-2.82311814696074	-1.4012955399815\\
-2.87999250812671	-1.37427930959536\\
-2.92761450788321	-1.34782843907536\\
-2.96785477081937	-1.32200465604213\\
-3.00213469242636	-1.29679556598416\\
-3.03154308672126	-1.27214464320716\\
-3.05692021372257	-1.24796920448344\\
-3.0789184572603	-1.22417095912209\\
-3.098046311607	-1.20064193399821\\
-3.11470037786735	-1.17726747430447\\
-3.1291886612983	-1.15392734812871\\
-3.14174746141081	-1.13049556681816\\
-3.15255343902769	-1.10683927077373\\
-3.16173193739188	-1.0828168593773\\
-3.16936226115321	-1.05827542579354\\
-3.17548032873782	-1.03304746813961\\
-3.18007887317197	-1.00694677181647\\
-3.18310514345613	-0.979763281858598\\
-3.18445582470331	-0.951256699217872\\
-3.1839686197656	-0.921148431652043\\
-3.18140957999802	-0.889111398528125\\
-3.17645478670739	-0.854757018710966\\
-3.16866429456784	-0.817618491103143\\
-3.15744524822052	-0.777129201102349\\
-3.1419996226369	-0.732594759038347\\
-3.12124991115256	-0.683156840821503\\
-3.09373304327001	-0.627746789871623\\
-3.05744864793884	-0.56502719964194\\
-3.00964261337217	-0.493321254018828\\
-2.94650206075707	-0.410534317308221\\
-2.86273816213399	-0.314084086158319\\
-2.75105243695848	-0.200882268889368\\
-2.60155406129097	-0.0674650913852828\\
-2.40138990312832	0.0895347979608175\\
-2.13526444165352	0.272262905902826\\
-1.78815012598701	0.479549466436689\\
-1.35169578093293	0.704055024761356\\
-0.833802555613541	0.930355778532272\\
-0.26531739322843	1.13717633843827\\
0.305598348497514	1.30500708020502\\
0.831763869877206	1.42424215832089\\
1.28384927376364	1.49704828505071\\
1.65393178900214	1.5328286466653\\
1.94860967361305	1.54255432833086\\
};
\addplot [color=mycolor1, forget plot]
  table[row sep=crcr]{%
2.01533812164163	1.62604405276322\\
2.23078249240264	1.61947135062162\\
2.40132090076523	1.60339148684674\\
2.53722108374555	1.58180236172912\\
2.64654784552852	1.55723001253831\\
2.73543985727512	1.53122219873246\\
2.80850990487305	1.50470009413296\\
2.86921499079895	1.47819106428946\\
2.92015375315593	1.45197729022359\\
2.96329141677968	1.42618900054377\\
3.00012525770858	1.40086244377317\\
3.03180488741777	1.37597573400241\\
3.05921940824373	1.35147086268574\\
3.08306063861848	1.32726703647262\\
3.10386911975224	1.30326853228884\\
3.12206769401675	1.27936903780104\\
3.1379860358974	1.25545368599787\\
3.15187850511821	1.2313995175315\\
3.16393696944925	1.20707480229813\\
3.17429972452134	1.18233745375814\\
3.18305725410281	1.15703263297448\\
3.19025527804017	1.13098953708787\\
3.1958952892501	1.1040172808161\\
3.19993255524039	1.07589969643826\\
3.20227132538617	1.04638878691185\\
3.20275671237797	1.01519645831191\\
3.20116236797684	0.981984021383658\\
3.19717260000278	0.946348776642342\\
3.19035690968367	0.907806772208036\\
3.18013396639077	0.865770541065649\\
3.1657206394711	0.819520290456693\\
3.14605968485133	0.768166672827977\\
3.11971681260298	0.710603047605297\\
3.08473396396719	0.645445388655745\\
3.03842085756421	0.570959515297282\\
2.9770625239435	0.484979939102609\\
2.89552117857411	0.384836153263169\\
2.78672909121182	0.267328088123135\\
2.64113583717848	0.128845047388666\\
2.44635087067706	-0.0341853588117058\\
2.18759873351322	-0.224214143276535\\
1.85017027932879	-0.440534454589518\\
1.42527425881919	-0.676434248589246\\
0.918975998779657	-0.917067119843662\\
0.359018686232294	-1.14112455367898\\
-0.209233416725197	-1.32792310389627\\
-0.739226389230772	-1.46581173965135\\
-1.19996215139496	-1.55507001437202\\
-1.58102344345021	-1.60414390620066\\
-1.88700244748479	-1.62396754081746\\
-2.12940113609882	-1.62426956441044\\
-2.32097043321478	-1.61232871930589\\
-2.4730471142931	-1.59309969429799\\
-2.59478196509427	-1.56977421460537\\
-2.69322879116622	-1.54433650080771\\
-2.77371333840301	-1.51798531534492\\
-2.84022849832474	-1.49142186722937\\
-2.89576968323859	-1.46503638282487\\
-2.94259477266895	-1.43902599538774\\
-2.98241747368191	-1.41346832118405\\
-3.01654851341223	-1.38836707616566\\
-3.04599803371542	-1.36368019711536\\
-3.07154980001532	-1.33933701616698\\
-3.09381511158766	-1.31524854837838\\
-3.11327209412083	-1.29131339930806\\
-3.13029440294031	-1.26742083623812\\
-3.14517216961788	-1.24345196690316\\
-3.15812717037828	-1.2192795917351\\
-3.16932358304173	-1.19476705222343\\
-3.17887525422785	-1.16976623492159\\
-3.18685006373035	-1.14411477395435\\
-3.19327170606398	-1.11763240256517\\
-3.1981189768976	-1.090116321054\\
-3.20132242555929	-1.06133536274487\\
-3.2027579853432	-1.03102264123454\\
-3.20223688804579	-0.998866241078235\\
-3.19949076565381	-0.964497359602382\\
-3.19415028193235	-0.927475108504999\\
-3.1857148364916	-0.887266930464672\\
-3.1735097247936	-0.843223275531408\\
-3.15662545454323	-0.794544834323392\\
-3.1338315005953	-0.740240317604256\\
-3.10345340779718	-0.679072720821543\\
-3.06319775596855	-0.6094927493767\\
-3.00990465381885	-0.529560827287165\\
-2.93920465871892	-0.436866525052448\\
-2.84506407514115	-0.328471731263217\\
-2.71923958708128	-0.200941316919633\\
-2.55077516534008	-0.0505963821734142\\
-2.32594077898868	0.125764016628224\\
-2.02950284685265	0.329331705284922\\
-1.64873641905498	0.556766844342745\\
-1.18109928201364	0.797398721387658\\
-0.643132582441513	1.03258585633054\\
-0.0728119622014619	1.24019417476993\\
0.481442381812748	1.40322281470281\\
0.979351688292782	1.51611243646592\\
1.40041618766564	1.5839478109451\\
1.74275843194902	1.61705145338868\\
2.01533812164163	1.62604405276322\\
};
\addplot [color=mycolor1, forget plot]
  table[row sep=crcr]{%
2.07625084362876	1.70905454066351\\
2.27657171102993	1.7029371442451\\
2.43616868402318	1.68788290929235\\
2.56429398464666	1.66752361618015\\
2.66816102380818	1.64417396855024\\
2.7532589566817	1.61927255932821\\
2.82372647505532	1.59369206321064\\
2.88268142697289	1.56794476004416\\
2.93248158859062	1.54231477478672\\
2.97492112869881	1.51694197169365\\
3.01137531548528	1.49187481670289\\
3.04290611533935	1.46710353913919\\
3.07033908435744	1.44258080620787\\
3.09431943631997	1.41823444553482\\
3.1153530375915	1.39397505235657\\
3.13383644291012	1.36970025017026\\
3.15007888440732	1.34529670157174\\
3.1643182593009	1.3206405388398\\
3.17673253706303	1.29559660685734\\
3.18744755077955	1.27001672492065\\
3.19654179403765	1.24373704136143\\
3.20404857173142	1.21657445123097\\
3.20995561726957	1.18832195568965\\
3.2142020609549	1.15874274941089\\
3.21667238637311	1.12756271826917\\
3.21718671125588	1.09446090317859\\
3.21548633545506	1.05905732599157\\
3.21121295525575	1.02089736817394\\
3.2038791708604	0.979431632163099\\
3.19282679991494	0.933989895601004\\
3.17716789949916	0.88374740859986\\
3.15570109554437	0.827681459205336\\
3.12679262240682	0.764516053245468\\
3.08820733464323	0.692653240751548\\
3.03687042097796	0.610092261916117\\
2.96853800615594	0.51434486165235\\
2.8773614535736	0.402372075235921\\
2.75536443681216	0.27060407493264\\
2.59195470341043	0.115173991350096\\
2.37383613389199	-0.0673944967456265\\
2.08613598372201	-0.278706153942123\\
1.71605520807854	-0.516007374804534\\
1.25998293614149	-0.769296722337668\\
0.732095567027905	-1.02030053062889\\
0.167467140251177	-1.24634563118402\\
-0.387183572051126	-1.42877179658333\\
-0.891025359473241	-1.55991959660899\\
-1.32146597833854	-1.64333876514649\\
-1.6744246737234	-1.68880234054156\\
-1.95736105292541	-1.70713148781447\\
-2.18216337866193	-1.70740611297996\\
-2.36083222890854	-1.69626317837733\\
-2.50366964198596	-1.67819669876535\\
-2.618880768512	-1.6561163679259\\
-2.71276999592862	-1.63185226658598\\
-2.79010612252365	-1.60652857080875\\
-2.8544804346866	-1.58081732677704\\
-2.90860242718634	-1.55510371807489\\
-2.9545270643359	-1.52959151788961\\
-2.99382369479805	-1.50436972387311\\
-3.02769974333328	-1.47945445076631\\
-3.05709081854048	-1.45481514211133\\
-3.08272635217036	-1.43039082628619\\
-3.10517752533419	-1.40610000507192\\
-3.12489235374168	-1.38184641564834\\
-3.14222139692114	-1.35752206036053\\
-3.15743653465407	-1.33300836373815\\
-3.17074451852888	-1.30817597341254\\
-3.18229647416781	-1.28288349586974\\
-3.19219413596961	-1.25697530235906\\
-3.20049329266957	-1.23027842460988\\
-3.20720467138463	-1.20259846415877\\
-3.21229225948476	-1.17371434856197\\
-3.21566882949251	-1.14337167090636\\
-3.21718816250861	-1.11127423516403\\
-3.21663312446418	-1.07707328820045\\
-3.213698289137	-1.0403537381232\\
-3.2079651557215	-1.00061642671769\\
-3.19886708211181	-0.957255233004433\\
-3.18563971506946	-0.909527440378471\\
-3.16725076793578	-0.856515441995987\\
-3.1423002681693	-0.797077617113549\\
-3.1088787136591	-0.729786419488706\\
-3.06436609974594	-0.652853169202736\\
-3.00515074354965	-0.564043495916574\\
-2.92624747803168	-0.460598558864161\\
-2.82081182873313	-0.339202275322706\\
-2.67960829495215	-0.196085812311738\\
-2.49065346318377	-0.027450510797915\\
-2.23959860846959	0.169492024219199\\
-1.91193949263403	0.394534069462105\\
-1.49838501044073	0.641617587095279\\
-1.00327221339828	0.896484499037834\\
-0.451531360927953	1.13781127815106\\
0.113997515571919	1.34378714168818\\
0.647354154723447	1.50075046197496\\
1.11604264788142	1.60706016135398\\
1.50733999281103	1.67011773368875\\
1.82394103035216	1.70073476993218\\
2.07625084362876	1.70905454066351\\
};
\addplot [color=mycolor1, forget plot]
  table[row sep=crcr]{%
2.1317414423095	1.79100753091255\\
2.31810513092557	1.78531088071093\\
2.46750057327107	1.77121361329816\\
2.58826913120071	1.75201867108555\\
2.6868750793552	1.7298478205699\\
2.76823712944396	1.70603634702307\\
2.83607514814713	1.68140766269239\\
2.89320444625466	1.65645536988618\\
2.94176541488176	1.63146116440283\\
2.9833958590435	1.60657035406556\\
3.01935802552462	1.58183998349703\\
3.05063158093467	1.55726939291785\\
3.07798158660956	1.5328195071364\\
3.102008275447	1.50842484691612\\
3.12318358668422	1.48400078265371\\
3.14187800771922	1.45944761539587\\
3.15838023912751	1.43465247446497\\
3.17291145028976	1.40948963655129\\
3.1856353486624	1.3838196170224\\
3.19666488245242	1.35748720900995\\
3.20606608599172	1.33031851643538\\
3.21385932206391	1.30211692135325\\
3.22001794513899	1.27265782792174\\
3.22446417591456	1.24168192229003\\
3.22706171143083	1.20888656827981\\
3.22760426130214	1.17391481141768\\
3.22579875322447	1.1363412764689\\
3.22124132591228	1.09565400392063\\
3.21338333449859	1.0512309698093\\
3.20148330571805	1.00230967401717\\
3.18453892980651	0.947947802132128\\
3.16119056676129	0.886972688037045\\
3.12958422734702	0.817917450365342\\
3.08717771115147	0.738943010388461\\
3.03046968605458	0.647749429602825\\
2.9546318816494	0.541490732974257\\
2.85303999781711	0.416731621646288\\
2.71675510959344	0.269534018621001\\
2.53415623186526	0.0958496398074653\\
2.29124129015289	-0.107486009875524\\
1.97360510470139	-0.340815789433208\\
1.57137861878927	-0.598787897134276\\
1.08721218216131	-0.867771357129065\\
0.543419735910371	-1.1264524054491\\
-0.0194546428704111	-1.35190795116236\\
-0.555956970215128	-1.52845094451582\\
-1.03216247322138	-1.65245572918629\\
-1.4331592886672	-1.73019094646337\\
-1.75983247900571	-1.77227422021084\\
-2.02153685237199	-1.78922555794723\\
-2.23014518586385	-1.7894752978946\\
-2.39685537138576	-1.77907268401652\\
-2.53101872335505	-1.76209832106834\\
-2.6400030291327	-1.74120709171542\\
-2.72945476977381	-1.71808621165136\\
-2.80365248373414	-1.6937872099948\\
-2.86583106514623	-1.66895044366125\\
-2.91844350981957	-1.6439518970657\\
-2.96336049115918	-1.61899762638342\\
-3.00201852468984	-1.59418407199124\\
-3.03552869182068	-1.56953642646061\\
-3.06475613262073	-1.54503293912038\\
-3.0903781973561	-1.52062017474582\\
-3.11292708052982	-1.4962223993832\\
-3.13282113780473	-1.4717470929754\\
-3.150387877346	-1.44708784287421\\
-3.16588073720748	-1.42212539481393\\
-3.17949112278348	-1.39672732661129\\
-3.19135671140971	-1.3707466002794\\
-3.20156667958285	-1.34401909927058\\
-3.21016422967143	-1.31636014226515\\
-3.21714655389592	-1.28755986472617\\
-3.22246214491573	-1.25737726036396\\
-3.22600511642653	-1.22553256485005\\
-3.2276059021702	-1.1916975322311\\
-3.22701731790171	-1.15548298868904\\
-3.22389444409388	-1.11642283618932\\
-3.21776604144455	-1.07395340886645\\
-3.20799413922056	-1.02738675304841\\
-3.19371689091799	-0.975876023855393\\
-3.17376758645825	-0.918370838279727\\
-3.14655965614383	-0.853560298705166\\
-3.10992354808926	-0.7798020024311\\
-3.06087701812257	-0.69503783253758\\
-2.9953078345687	-0.596704182786383\\
-2.9075537768713	-0.481660592789777\\
-2.78989601647332	-0.346195893384541\\
-2.63207603324366	-0.186238557768118\\
-2.42116974669818	0.00199399013596253\\
-2.14256889254755	0.220565435545492\\
-1.78329827876146	0.467362244745593\\
-1.33864880940448	0.733098980637337\\
-0.820555955097994	0.999900587110749\\
-0.261428776333567	1.24457532845698\\
0.293542875111646	1.44680762682395\\
0.802975958022219	1.59679851197821\\
1.24229289017552	1.69648104919221\\
1.60531664101736	1.75499481314277\\
1.89807274077814	1.78330660078127\\
2.1317414423095	1.79100753091255\\
};
\addplot [color=mycolor1, forget plot]
  table[row sep=crcr]{%
2.18202401463123	1.87130548893601\\
2.35546460910354	1.8659989740022\\
2.49531402264379	1.85279783305457\\
2.60909837557366	1.83470888706013\\
2.70262154246927	1.81367741452399\\
2.78029989221005	1.79094111660265\\
2.84548241910839	1.76727406506573\\
2.90071413752772	1.74314848567836\\
2.94793913106009	1.71884008947843\\
2.98865234844797	1.69449608538423\\
3.02401151948419	1.67017893583495\\
3.05491925982564	1.64589442281208\\
3.08208328967226	1.62160953907504\\
3.10606067976851	1.59726372493208\\
3.12729042043629	1.57277568874334\\
3.14611738903238	1.54804722797296\\
3.16280989598737	1.52296493819135\\
3.17757233784711	1.49740035074913\\
3.19055400743643	1.47120880606978\\
3.20185475129721	1.4442272037834\\
3.21152788042525	1.41627064377201\\
3.21958049832703	1.38712786329406\\
3.2259711816471	1.35655526949354\\
3.23060470516282	1.32426925100999\\
3.23332321384119	1.28993631496692\\
3.23389287124117	1.25316042374075\\
3.23198450364748	1.21346668630825\\
3.22714603978494	1.17028027905468\\
3.21876351368187	1.12289912394304\\
3.20600591176761	1.07045845048474\\
3.18774702488831	1.01188497623521\\
3.16245452589533	0.945838247822802\\
3.12803267039657	0.870637166155992\\
3.08160075712392	0.784171977175059\\
3.01918676590626	0.683808451374954\\
2.93532042725459	0.566306538405956\\
2.82253783362597	0.427809633335044\\
2.67089509680695	0.26402657173068\\
2.46779379775515	0.0708365130902191\\
2.19881035353723	-0.1543370372527\\
1.85069554742905	-0.410094596109592\\
1.41757943731687	-0.687947573759316\\
0.909196433566452	-0.970485391769853\\
0.355432238463405	-1.23402580139253\\
-0.199907051267903	-1.45656752040886\\
-0.714824033161	-1.62608083233949\\
-1.16276950598761	-1.74276677810291\\
-1.53551456365519	-1.81504155347651\\
-1.83769667471329	-1.85397291447147\\
-2.07982491225251	-1.86965351824121\\
-2.27348859687057	-1.86988069903074\\
-2.42907384371724	-1.86016742457127\\
-2.55506502775769	-1.84422262288643\\
-2.6580880518937	-1.82447035710515\\
-2.74321038548067	-1.80246535066994\\
-2.81427788442654	-1.77918880243121\\
-2.87420854073542	-1.75524769527952\\
-2.92522499346317	-1.73100553369938\\
-2.96903053613899	-1.70666703771278\\
-3.00693959320492	-1.68233271304135\\
-3.03997360262699	-1.65803391122278\\
-3.06893132085974	-1.63375526384628\\
-3.09444042702253	-1.60944889993664\\
-3.11699547832575	-1.58504325462593\\
-3.13698585647355	-1.56044825124497\\
-3.15471629743739	-1.53555798040615\\
-3.17042183318646	-1.51025157238872\\
-3.18427841674602	-1.48439267580391\\
-3.19641008883668	-1.4578277602352\\
-3.20689322696851	-1.4303833171076\\
-3.21575815862114	-1.40186191715796\\
-3.22298818821358	-1.37203697719307\\
-3.22851585503685	-1.34064597961906\\
-3.23221597727604	-1.30738176323769\\
-3.2338947115149	-1.27188135077356\\
-3.23327342309961	-1.23371158453574\\
-3.2299655588147	-1.19235059325209\\
-3.22344385254215	-1.14716379976709\\
-3.21299395546891	-1.09737280136007\\
-3.1976488032231	-1.04201504444782\\
-3.17609552255039	-0.979891886962476\\
-3.14654328864766	-0.909502698571209\\
-3.10653638634683	-0.828963820567699\\
-3.05269282312502	-0.735915134149284\\
-2.98034873213667	-0.627427148135196\\
-2.88310229766976	-0.499944753247218\\
-2.75230159650874	-0.349351808342394\\
-2.57665665987365	-0.171327690835457\\
-2.34244890996455	0.0377108016772661\\
-2.03527926982411	0.278721927644738\\
-1.64461155087895	0.547139911134911\\
-1.17139168307723	0.830037161564239\\
-0.635415342870461	1.10615942595841\\
-0.0750890768055588	1.35147228458665\\
0.464530598780093	1.54820045262041\\
0.948070919357095	1.69062377979606\\
1.35846538811358	1.78377048989356\\
1.69483670196028	1.83799684704665\\
1.96553148603393	1.86417465145828\\
2.18202401463123	1.87130548893601\\
};
\addplot [color=mycolor1]
  table[row sep=crcr]{%
2.22721472973942	1.94935886896179\\
2.3886648425121	1.94441487511101\\
2.5195634483927	1.93205457572182\\
2.62670779849127	1.91501766628131\\
2.71531853364	1.89508786650117\\
2.78936864949369	1.87341097537522\\
2.85187814440773	1.85071228241933\\
2.90515032165589	1.82744075521499\\
2.9509520177711	1.80386332654994\\
2.99064783647547	1.78012618777659\\
3.02529909977035	1.7562945351756\\
3.0557365623221	1.73237827445755\\
3.08261387510984	1.7083485310185\\
3.10644696879361	1.68414807630787\\
3.12764310033082	1.65969765855822\\
3.14652223914943	1.63489950162782\\
3.16333268711427	1.60963876367151\\
3.17826225469757	1.5837834336332\\
3.19144589188063	1.55718292750329\\
3.20297034763997	1.52966548843736\\
3.2128761683912	1.50103436869378\\
3.2211571124149	1.47106265802835\\
3.22775682617807	1.43948650786522\\
3.23256237104755	1.40599637010381\\
3.23539387180292	1.3702257107414\\
3.23598913837185	1.33173645766061\\
3.23398152983158	1.29000018434938\\
3.22886850103885	1.24437370419161\\
3.21996707976175	1.19406735000614\\
3.20635080827997	1.13810376656035\\
3.18676025898309	1.07526465092859\\
3.15947593702972	1.0040228169715\\
3.12213828317303	0.922457920675385\\
3.07149545967596	0.828157721349833\\
3.00305879148097	0.718116213117772\\
2.91065699973037	0.588662002756959\\
2.785925244978	0.435496623074383\\
2.61788823849866	0.254008232649376\\
2.39306798253518	0.0401523940333445\\
2.09699730361163	-0.20771900183087\\
1.7183819635995	-0.485932899271695\\
1.25639683089422	-0.782386182967809\\
0.728319451607428	-1.07597695395358\\
0.17048677629427	-1.34156752261761\\
-0.372344302750219	-1.55919115860543\\
-0.863264272549092	-1.72086595365401\\
-1.28299919742863	-1.8302347441912\\
-1.62890152595063	-1.89731652670458\\
-1.90833376158261	-1.93331843320936\\
-2.13240654410545	-1.94782687646357\\
-2.3122537087605	-1.94803363323748\\
-2.45746747168844	-1.93896359662991\\
-2.57574556593285	-1.92399109460735\\
-2.67305603610964	-1.90533078490532\\
-2.75395564612754	-1.88441460443591\\
-2.82190758462804	-1.86215611801615\\
-2.87954766726586	-1.83912801979308\\
-2.92889157233296	-1.81567886095776\\
-2.97149068204595	-1.79200910352774\\
-3.00854741762708	-1.76822049003994\\
-3.04100007190525	-1.74434802267285\\
-3.0695851473103	-1.7203805908067\\
-3.09488323143276	-1.69627413213334\\
-3.11735281839452	-1.67195981527391\\
-3.13735524534064	-1.64734883071173\\
-3.15517299859553	-1.62233479306363\\
-3.17102297575804	-1.59679437416259\\
-3.18506579905015	-1.570586527561\\
-3.19741190629232	-1.54355048208521\\
-3.20812485602917	-1.51550254281508\\
-3.21722203851139	-1.48623162017347\\
-3.22467275544594	-1.45549329519237\\
-3.23039339092638	-1.42300210774249\\
-3.23423911335776	-1.38842161161915\\
-3.23599118590349	-1.35135156242766\\
-3.23533847055249	-1.31131137681683\\
-3.23185101774822	-1.2677187105326\\
-3.22494264006069	-1.21986163848029\\
-3.2138179373885	-1.16686249014231\\
-3.19739719631547	-1.10763095340298\\
-3.17420973909784	-1.0408037811618\\
-3.142242561071	-0.964668760862181\\
-3.09872682936675	-0.877072578179989\\
-3.0398417586358	-0.77531812311728\\
-2.96031892266926	-0.656071422287709\\
-2.85295425309884	-0.515330753928937\\
-2.70811161177175	-0.348574764250833\\
-2.51348980394932	-0.151314840167112\\
-2.25479111797749	0.0795960527623154\\
-1.91840618384291	0.343563906305597\\
-1.49721915275415	0.633014609293368\\
-0.998726638487032	0.931115827109969\\
-0.450328181715587	1.21375185121395\\
0.105465468340197	1.45718721810745\\
0.625945116714116	1.64701594620063\\
1.08251666630864	1.78154116777267\\
1.46486936946524	1.86834353319716\\
1.77626331377619	1.91854897142989\\
2.02656829443923	1.94275375393752\\
2.22721472973941	1.94935886896179\\
};
\addlegendentry{Аппроксимации}

\addplot [color=mycolor2]
  table[row sep=crcr]{%
2.12132034355964	2.82842712474619\\
2.18582575000323	2.82640660874952\\
2.2459761483502	2.82067996094826\\
2.30316600770785	2.81153894842004\\
2.35833530419819	2.79908319427389\\
2.4121407651283	2.78328504568152\\
2.46505313939319	2.76402339892432\\
2.51741350921551	2.74110157112695\\
2.56946568278775	2.71425682483196\\
2.62137377470791	2.68316571665979\\
2.67322998558975	2.64744786912504\\
2.72505542741334	2.60667010622203\\
2.77679571750944	2.56035271147659\\
2.82831256449113	2.50797963523731\\
2.87937250576692	2.44901464186941\\
2.92963424895255	2.38292549191834\\
2.97863667364875	2.30921809960124\\
3.02579039162557	2.22748193169334\\
3.0703766721051	2.13744642550811\\
3.11155818595542	2.03904567139316\\
3.14840590189045	1.93248505458784\\
3.1799449834646	1.81829951655162\\
3.20521925970046	1.69738980361013\\
3.2233689191928	1.57102231442397\\
3.23371058284854	1.44078163681543\\
3.23580480244767	1.30847305653245\\
3.22949541157146	1.1759834831\\
3.21490900216393	1.04511946165368\\
3.19240995273442	0.917445713323391\\
3.16251366135309	0.794144173493542\\
3.12576385298257	0.675902054710668\\
3.08257536776515	0.562820516214273\\
3.03302813706062	0.454314775018543\\
2.97656426645917	0.348948817128694\\
2.91146921429886	0.244099783305823\\
2.83385680006128	0.135243682019644\\
2.73548220960228	0.0144130439331602\\
2.59871334341581	-0.133168683665202\\
2.38479638365891	-0.336180712113444\\
2.01057175466146	-0.648371525543042\\
1.34210041814469	-1.13791067535239\\
0.375352321166521	-1.75822148716338\\
-0.527448468220416	-2.26262801632552\\
-1.11365847778129	-2.5437869612038\\
-1.4567327767317	-2.68228003389134\\
-1.6691296256022	-2.75259678373195\\
-1.81428800439327	-2.79054335642282\\
-1.92320462744895	-2.81168897715777\\
-2.01132392805712	-2.82302365363059\\
-2.08680248126988	-2.82787464184343\\
-2.15422943553058	-2.82790871121207\\
-2.21634478307291	-2.8239827357917\\
-2.27487422572476	-2.8165264847195\\
-2.33095693947289	-2.80572465930185\\
-2.3853755403757	-2.79160756168836\\
-2.43868469976231	-2.77409782070174\\
-2.49128507004909	-2.75303495626884\\
-2.54346612600745	-2.7281884019911\\
-2.59543032955127	-2.69926454755406\\
-2.64730535029083	-2.66591103001887\\
-2.69914810041563	-2.62772046423944\\
-2.75094277189846	-2.58423541950896\\
-2.80259428647749	-2.53495641420315\\
-2.85391830194335	-2.47935483422118\\
-2.90462903903618	-2.41689283661657\\
-2.95432664759112	-2.34705230341292\\
-3.00248656709011	-2.26937453210355\\
-3.04845424185856	-2.18351130413452\\
-3.0914493816545	-2.08928598065547\\
-3.13058429431929	-1.98676020448289\\
-3.16490008683964	-1.87629788656397\\
-3.19342217916266	-1.75861429449935\\
-3.21523240499265	-1.63479579561046\\
-3.22954956680171	-1.50627700312196\\
-3.23580523135335	-1.37476797309193\\
-3.23369896357047	-1.2421341494958\\
-3.22321880671614	-1.11024302689198\\
-3.20461860896113	-0.980799507713277\\
-3.17835148049538	-0.85519274756288\\
-3.14496439092874	-0.734369602871579\\
-3.10495870881183	-0.618735254776386\\
-3.05861192744495	-0.508062722468019\\
-3.0057323937692	-0.401369642978784\\
-2.94527025004781	-0.296685462718974\\
-2.87460204254734	-0.190562899149911\\
-2.78805571026974	-0.0770301658026245\\
-2.67361421522119	0.0546866392302208\\
-2.50520482475694	0.225105040795738\\
-2.22541867988909	0.474087715816021\\
-1.7209533613293	0.868461990547897\\
-0.880186083660502	1.44500069250654\\
0.110323684998138	2.03883159526995\\
0.859259206598599	2.42737283484743\\
1.30741523004191	2.62511520505649\\
1.57426953542748	2.72304263504847\\
1.74766615441039	2.77435113512036\\
1.87207484559155	2.80265686669549\\
1.96924085114733	2.81832059141065\\
2.0502990590364	2.82612977760851\\
2.12132034355964	2.82842712474619\\
};
\addlegendentry{Сумма Минковского}

\end{axis}

\begin{axis}[%
width=0.798\linewidth,
height=0.597\linewidth,
at={(-0.104\linewidth,-0.066\linewidth)},
scale only axis,
xmin=0,
xmax=1,
ymin=0,
ymax=1,
axis line style={draw=none},
ticks=none,
axis x line*=bottom,
axis y line*=left,
legend style={legend cell align=left, align=left, draw=white!15!black}
]
\end{axis}
\end{tikzpicture}%
        \caption{Эллипсоидальные аппроксимации для 100 направлений.}
\end{figure}
\begin{figure}[b]

        \centering
        % This file was created by matlab2tikz.
%
%The latest updates can be retrieved from
%  http://www.mathworks.com/matlabcentral/fileexchange/22022-matlab2tikz-matlab2tikz
%where you can also make suggestions and rate matlab2tikz.
%
\definecolor{mycolor1}{rgb}{0.00000,0.44700,0.74100}%
\definecolor{mycolor2}{rgb}{0.85000,0.32500,0.09800}%
%
\begin{tikzpicture}

\begin{axis}[%
width=0.618\linewidth,
height=0.471\linewidth,
at={(0\linewidth,0\linewidth)},
scale only axis,
xmin=-8,
xmax=8,
xlabel style={font=\color{white!15!black}},
xlabel={$x_1$},
ymin=-6,
ymax=6,
ylabel style={font=\color{white!15!black}},
ylabel={$x_2$},
axis background/.style={fill=white},
axis x line*=bottom,
axis y line*=left,
xmajorgrids,
ymajorgrids,
legend style={at={(0.03,0.97)}, anchor=north west, legend cell align=left, align=left, draw=white!15!black}
]
\addplot [color=mycolor1, forget plot]
  table[row sep=crcr]{%
1.1711787112841	3.34230777773576\\
1.34444389618673	3.33686722689733\\
1.50648318534277	3.32144701348819\\
1.65806242150429	3.29723890999844\\
1.79995419924852	3.26523166869947\\
1.93290799118999	3.22622794766654\\
2.05763032437368	3.18086276334402\\
2.17477223041259	3.12962150018857\\
2.28492171716205	3.07285634836238\\
2.3885995032216	3.01080060567632\\
2.48625667387058	2.94358064140543\\
2.57827324651315	2.87122553999274\\
2.66495688065443	2.79367457178694\\
2.74654114431786	2.71078271324095\\
2.82318287015103	2.62232448808267\\
2.8949582146523	2.52799644473262\\
2.96185708693385	2.42741864049226\\
3.02377565347494	2.3201355847161\\
3.0805066685131	2.20561721602438\\
3.13172744617751	2.08326066796132\\
3.17698540702193	1.95239382914097\\
3.21568133490354	1.81228204209558\\
3.24705081945703	1.66213971753183\\
3.27014489780722	1.50114915910447\\
3.28381172002077	1.32848945666784\\
3.28668221628109	1.14337881243048\\
3.2771642749897	0.945133918403077\\
3.25345178932588	0.733249674250213\\
3.21355684151227	0.507501133889645\\
3.15537469570401	0.268066487797808\\
3.07679115926015	0.015664562009573\\
2.97583883725379	-0.248307401701393\\
2.85090136682411	-0.521659161442773\\
2.70095216778915	-0.801347823435408\\
2.52579793500058	-1.08350783899399\\
2.32628156602984	-1.3635980584876\\
2.10439191407115	-1.63667268179468\\
1.86323617293213	-1.89775167737354\\
1.60685736899554	-2.14223401489281\\
1.33991789027605	-2.3662777508081\\
1.06730585933085	-2.56707514274059\\
0.793739425298697	-2.74297851870391\\
0.523437853608312	-2.89347215049445\\
0.259902596754298	-3.0190207516634\\
0.00581903236219554	-3.1208447467499\\
-0.2369372555121	-3.20067372983419\\
-0.467220354507	-3.2605178574641\\
-0.684493265340969	-3.30248032509504\\
-0.888694234930048	-3.32861897022119\\
-1.0801092196493	-3.34085455680139\\
-1.25926106809304	-3.34091776153004\\
-1.42681950330378	-3.33032515794113\\
-1.58353177449541	-3.31037507736533\\
-1.73017158467123	-3.2821558685242\\
-1.86750300244094	-3.24656096722397\\
-1.99625599202063	-3.20430689704777\\
-2.11711054093165	-3.15595168681459\\
-2.23068687011842	-3.10191219445658\\
-2.33753972875556	-3.04247952074268\\
-2.43815523449964	-2.97783215102365\\
-2.5329490934295	-2.90804674702345\\
-2.6222653208492	-2.83310668006404\\
-2.70637479428206	-2.75290849531896\\
-2.78547311726505	-2.66726655590999\\
-2.85967737154948	-2.5759161597878\\
-2.92902140016315	-2.47851546990414\\
-2.99344930844609	-2.37464666482809\\
-3.05280690958282	-2.26381681735669\\
-3.10683089311493	-2.14545915779927\\
-3.15513558252945	-2.01893559215682\\
-3.19719730262603	-1.88354163878489\\
-3.23233664163192	-1.73851533219976\\
-3.25969932318791	-1.58305212075492\\
-3.27823706849968	-1.41632833369867\\
-3.28669080350863	-1.23753634540448\\
-3.28357990805066	-1.04593497683649\\
-3.26720291066679	-0.840918687724783\\
-3.23565696131938	-0.622108318461704\\
-3.18688516768166	-0.389463978954261\\
-3.11876167995196	-0.143416533926881\\
-3.02922302993614	0.11499241610004\\
-2.91644915653916	0.383980836121862\\
-2.77908754849337	0.660934801326638\\
-2.61649915361415	0.942389492277687\\
-2.42898799999709	1.22411526429499\\
-2.21796402213223	1.50132757866017\\
-1.9859883878281	1.76901289011117\\
-1.7366683581316	2.02232922526914\\
-1.47440268220478	2.25701269019761\\
-1.20401775520347	2.46971278707818\\
-0.930363383198716	2.65819606075749\\
-0.657943143023997	2.82139307813531\\
-0.390637102663735	2.95930311865273\\
-0.131543811712393	3.07279931782965\\
0.117062449314586	3.1633871238911\\
0.353684227850852	3.23296276733506\\
0.577496523037587	3.28360333410188\\
0.788216624026908	3.3174035782658\\
0.985971690059736	3.33636162404406\\
1.1711787112841	3.34230777773576\\
};
\addplot [color=mycolor2, forget plot]
  table[row sep=crcr]{%
2.36712178368749	2.74633057020532\\
2.41188305682678	2.74494651398331\\
2.45036041459025	2.74130089565813\\
2.48383536069915	2.73596692388682\\
2.51327948444918	2.72933446739078\\
2.53943919311047	2.72166753585199\\
2.56289492242467	2.71314181506891\\
2.58410302011837	2.70386940753685\\
2.60342571833248	2.69391520541791\\
2.62115280082523	2.68330766515176\\
2.63751738560017	2.67204572812381\\
2.65270745907852	2.66010298488441\\
2.66687427014079	2.64742976011011\\
2.68013832873481	2.63395351039579\\
2.69259349381253	2.61957772004863\\
2.70430943819101	2.6041793136995\\
2.71533261385189	2.58760445132293\\
2.72568568587575	2.56966240776041\\
2.73536523285703	2.55011704085516\\
2.74433729805113	2.52867508999008\\
2.75253007967915	2.50497017880853\\
2.75982261143016	2.47854086090236\\
2.76602760995085	2.44880025137829\\
2.77086559717422	2.41499358451471\\
2.77392567575526	2.37613820122431\\
2.77460549370625	2.33093765236215\\
2.77201821305984	2.27765729690984\\
2.7648464218491	2.21394231571241\\
2.7511098920125	2.13654986655322\\
2.72779317877431	2.04095569483249\\
2.69024836192134	1.92078707468285\\
2.6312539409082	1.7670493095065\\
2.53961413480675	1.56721869756115\\
2.39837999194539	1.30464987086518\\
2.18360687141246	0.959750006540987\\
1.8667826621648	0.516209188824812\\
1.42701784641989	-0.0238781630653391\\
0.87502831245873	-0.620875059217897\\
0.268936641059173	-1.19901922110702\\
-0.309576813214938	-1.68530797756881\\
-0.802100907580029	-2.04896030059078\\
-1.19206161730571	-2.300440537754\\
-1.49041992551985	-2.46708460024485\\
-1.7168759123608	-2.57532489307401\\
-1.88995458136084	-2.64492059461505\\
-2.02413470299114	-2.68919932969544\\
-2.12995725581116	-2.71680265831965\\
-2.21490837909904	-2.73327951527395\\
-2.284286999297	-2.74220891806166\\
-2.34187032729419	-2.74592425079378\\
-2.39038313346388	-2.74596637420131\\
-2.4318186528206	-2.74336542452957\\
-2.46765634360889	-2.73881708355027\\
-2.49901014208278	-2.7327940719517\\
-2.52673028332805	-2.72561759993438\\
-2.551474024202	-2.71750375208859\\
-2.57375536742722	-2.70859393338899\\
-2.5939804436843	-2.69897499500867\\
-2.61247296631128	-2.688692537203\\
-2.6294927100655	-2.67775958603992\\
-2.64524900273341	-2.66616202895009\\
-2.6599105764423	-2.65386167486455\\
-2.67361268902278	-2.64079746083638\\
-2.68646212009346	-2.62688508668123\\
-2.69854042224194	-2.61201517643097\\
-2.70990563030842	-2.59604990865956\\
-2.72059247497691	-2.57881790224895\\
-2.73061098716967	-2.5601069666803\\
-2.73994319179238	-2.53965410007404\\
-2.74853734053432	-2.51713180921804\\
-2.75629877602211	-2.49212938365471\\
-2.76307597883326	-2.46412710489964\\
-2.76863950239953	-2.43246039507853\\
-2.77265014411951	-2.39626942454063\\
-2.7746104875259	-2.35442742359956\\
-2.77379029091263	-2.3054374557388\\
-2.76911009981251	-2.24728211523106\\
-2.75895730729457	-2.17720283029247\\
-2.74089227231279	-2.0913749275097\\
-2.71117633487659	-1.98443346302753\\
-2.66401914162082	-1.84880387542276\\
-2.59041768625865	-1.67383876852635\\
-2.47652724106683	-1.44496938777841\\
-2.30193857203366	-1.14371563581315\\
-2.0396739695219	-0.750852651889909\\
-1.66265564107253	-0.256803878437201\\
-1.16248546087895	0.319450601512933\\
-0.573614265552006	0.917547528939285\\
0.0284025713818862	1.45675943888985\\
0.5685251002902	1.88250662286626\\
1.00960568842464	2.18713123722861\\
1.35150239782658	2.39258093926768\\
1.61139397123953	2.52710219556684\\
1.80908182046774	2.61398894502504\\
1.96116532773652	2.66959557253549\\
2.08006223005889	2.7046815507453\\
2.17466815864962	2.726172144186\\
2.25127885929965	2.73851896753466\\
2.3143624731471	2.74460741055068\\
2.36712178368749	2.74633057020532\\
};
\addplot [color=mycolor1, forget plot]
  table[row sep=crcr]{%
1.27349527731192	3.20471819063596\\
1.44428602357601	3.19936336562619\\
1.60261405164943	3.18430366081914\\
1.7494617470495	3.16085789974953\\
1.88579739700425	3.13011008910517\\
2.0125428514028	3.09293326009606\\
2.13055396048019	3.05001429713906\\
2.24060988431472	3.00187741815919\\
2.34340825178317	2.94890508541948\\
2.43956391894906	2.89135583426738\\
2.5296096946948	2.82937892878458\\
2.61399786648816	2.76302598507574\\
2.69310169254936	2.69225981668208\\
2.76721625454656	2.61696080274643\\
2.83655821210403	2.53693109253768\\
2.90126408806333	2.45189696354759\\
2.96138675922161	2.36150966207012\\
3.01688984642981	2.26534508952985\\
3.06763970522629	2.16290276913649\\
3.11339473035919	2.05360465173142\\
3.15379172653358	1.93679451605467\\
3.18832919463209	1.81173900963237\\
3.21634758276645	1.67763178656044\\
3.23700691962578	1.53360274915386\\
3.24926287231705	1.3787350990062\\
3.25184326398312	1.21209372070441\\
3.24322856924441	1.03276925597542\\
3.22164196864756	0.839942844419659\\
3.18505716005814	0.632976472406207\\
3.13123499683714	0.411532478777436\\
3.05780235675828	0.175722049785833\\
2.96238692637022	-0.0737245209502354\\
2.84281753478955	-0.335283926070913\\
2.69738875896765	-0.606494094576824\\
2.52516943358411	-0.883881067768981\\
2.32630972784737	-1.16300759537571\\
2.10227863372139	-1.43868074143903\\
1.85595607299018	-1.70532290329812\\
1.59152340317857	-1.95746223706772\\
1.31414419128342	-2.19025259681242\\
1.02948815699354	-2.39991296505422\\
0.743198214823109	-2.58399557142257\\
0.460411488715003	-2.74144341261505\\
0.185417358609968	-2.87245752802465\\
-0.0785138856106611	-2.97823672044315\\
-0.329145589188702	-3.06066541941308\\
-0.565175433800149	-3.12201330013307\\
-0.786073059350422	-3.16468601912692\\
-0.991897989417819	-3.1910422753473\\
-1.18312736420307	-3.20327509398365\\
-1.36050921912967	-3.20334606094207\\
-1.52494663407333	-3.19295843529209\\
-1.67741164040707	-3.17355605030546\\
-1.81888459259207	-3.14633753472363\\
-1.95031370082384	-3.11227829447245\\
-2.07258962512226	-3.07215522866845\\
-2.18653078029436	-3.02657109539301\\
-2.29287589627036	-2.97597680757907\\
-2.39228121567003	-2.92069083506565\\
-2.48532040738634	-2.86091544019825\\
-2.57248581455441	-2.79674979051335\\
-2.65419005132678	-2.72820015727833\\
-2.73076724021758	-2.65518748343299\\
-2.80247336639283	-2.57755263051448\\
-2.86948534028096	-2.49505961995269\\
-2.93189842458056	-2.40739718984169\\
-2.98972171238905	-2.31417900956967\\
-3.04287135449774	-2.21494294566854\\
-3.09116124130123	-2.10914986787104\\
-3.13429086713906	-1.99618264208025\\
-3.1718301681864	-1.87534619761917\\
-3.20320126676684	-1.74586990360306\\
-3.2276573295482	-1.60691396750067\\
-3.24425923086295	-1.45758219425911\\
-3.25185150619326	-1.29694420959691\\
-3.24904030415707	-1.12407109656362\\
-3.2341778115443	-0.938089160971005\\
-3.2053599817185	-0.738256903482426\\
-3.16044720340501	-0.524069668561859\\
-3.09712028967762	-0.295394018093037\\
-3.01298571486023	-0.0526286044658788\\
-2.90574246407405	0.203120663078268\\
-2.77341566658946	0.469875931666726\\
-2.61464718527509	0.744671830950842\\
-2.42901069043125	1.02353733575874\\
-2.21729346307338	1.3016207686597\\
-1.98167017289334	1.57348138033746\\
-1.72569877677608	1.83352877918035\\
-1.45410337080575	2.0765414570174\\
-1.17236633997143	2.2981599756885\\
-0.886209684202	2.49524939806107\\
-0.601076144049998	2.66606314379628\\
-0.321711220923973	2.81019956313654\\
-0.0519055926422069	2.92839608031109\\
0.205593759258074	3.02223382531029\\
0.449034858659099	3.09382481911079\\
0.677523760761812	3.14553385255023\\
0.890847711899254	3.1797617627868\\
1.08929520758861	3.1987956202668\\
1.27349527731192	3.20471819063596\\
};
\addplot [color=mycolor2, forget plot]
  table[row sep=crcr]{%
2.33603117668886	2.77383773533797\\
2.37516464725374	2.77262691357743\\
2.40893170213665	2.76942690337039\\
2.43841940062857	2.76472767070704\\
2.46445304757259	2.75886290926017\\
2.48766781562485	2.75205859965667\\
2.50855867157367	2.74446471145197\\
2.52751564883675	2.7361760908235\\
2.54484907620749	2.7272462605596\\
2.56080781241566	2.7176964566711\\
2.57559252411427	2.7075213606192\\
2.58936537895104	2.69669243896537\\
2.60225707842181	2.68515944438462\\
2.61437184672533	2.67285038571794\\
2.62579076958642	2.65967009126052\\
2.63657370548617	2.64549733588026\\
2.64675984467711	2.63018035376686\\
2.65636684635875	2.61353039177146\\
2.66538831824105	2.59531274799315\\
2.67378918633813	2.57523445329204\\
2.68149819424238	2.55292734178153\\
2.68839630448344	2.52792464682231\\
2.69429904156482	2.49962833648105\\
2.69892963366187	2.46726298324861\\
2.70187786222841	2.42980975324494\\
2.70253626684757	2.38591063440619\\
2.69999981971571	2.33372758652888\\
2.69290573546456	2.27073288158737\\
2.67917401784381	2.19339441362953\\
2.65558277526102	2.09670321392352\\
2.61707186745425	1.97347563859052\\
2.55562090465979	1.81337698205752\\
2.45854873336742	1.60175127155448\\
2.30635872094887	1.31887400601422\\
2.07147274626518	0.941730524350173\\
1.72241984531358	0.453080187872498\\
1.24153219533735	-0.137619467154939\\
0.654680997818204	-0.772582252750343\\
0.0395069632936128	-1.35971688279848\\
-0.517637861558398	-1.82830042522886\\
-0.97089736627434	-2.16310655256113\\
-1.31839640890298	-2.38726933266721\\
-1.57909721866367	-2.53290594747214\\
-1.7749207828929	-2.62651336246446\\
-1.92392397638943	-2.6864302500371\\
-2.03934578994682	-2.72451851239282\\
-2.13049454066881	-2.74829318691323\\
-2.20385170818753	-2.76252013130226\\
-2.26395409160786	-2.77025452721126\\
-2.31401603417494	-2.7734836002745\\
-2.35634944249354	-2.7735195169949\\
-2.39264387200647	-2.77124055076024\\
-2.42415372127008	-2.76724084171689\\
-2.45182449666335	-2.76192476872171\\
-2.47637904157075	-2.75556732998468\\
-2.49837723207992	-2.74835332254689\\
-2.51825787341866	-2.74040305387863\\
-2.53636848776742	-2.731789323547\\
-2.5529867379886	-2.72254861451314\\
-2.56833597715801	-2.7126883350026\\
-2.58259659458752	-2.70219126617895\\
-2.59591428500911	-2.69101793056547\\
-2.60840599783681	-2.67910730161753\\
-2.62016406333094	-2.66637606485201\\
-2.63125879931482	-2.65271647587191\\
-2.64173974572939	-2.63799271230598\\
-2.65163553102222	-2.62203546177158\\
-2.66095222192739	-2.60463430292545\\
-2.66966982116945	-2.58552719290596\\
-2.67773632155005	-2.56438603244357\\
-2.68505834758345	-2.54079678062751\\
-2.69148683325652	-2.51423184316855\\
-2.69679525610873	-2.48401131577275\\
-2.70064643363102	-2.44924789564855\\
-2.70254137209289	-2.40876750845558\\
-2.70173941511501	-2.36099335456907\\
-2.69713171116671	-2.30377429501523\\
-2.68703767939672	-2.23412816295588\\
-2.6688733753315	-2.1478558869707\\
-2.63860733006011	-2.03896517257946\\
-2.58987303019969	-1.89883626876177\\
-2.51257074237411	-1.71511913654336\\
-2.39088248288237	-1.47063540976362\\
-2.20124949426826	-1.14348303786958\\
-1.91297510885859	-0.711698128877647\\
-1.49818929428948	-0.168118349712293\\
-0.95743799505463	0.455077656114234\\
-0.344692469317055	1.07773548420951\\
0.250413715583845	1.61106228502936\\
0.758152826604524	2.01148285385328\\
1.1568976469304	2.28696840373081\\
1.45817328343662	2.46805007856471\\
1.68387274034641	2.58488856364122\\
1.85434182773653	2.65981681155943\\
1.98517230036076	2.70765318707294\\
2.0874946534908	2.73784727198925\\
2.1690763903508	2.75637813802873\\
2.23533360006066	2.76705520229934\\
2.29007830562403	2.77233778434643\\
2.33603117668886	2.77383773533797\\
};
\addplot [color=mycolor1, forget plot]
  table[row sep=crcr]{%
1.39922803212736	3.07685806631181\\
1.5674959045967	3.07159193962536\\
1.72184120893004	3.05691962276783\\
1.86354618020044	3.03430258576604\\
1.99383941594411	3.00492434527477\\
2.11386334227955	2.96972519188087\\
2.22465770222629	2.92943638880441\\
2.32715347530744	2.88461124035927\\
2.42217316267472	2.83565189476223\\
2.51043459705609	2.78283160405981\\
2.59255635115887	2.726312621507\\
2.66906346531654	2.66616012709227\\
2.74039265271167	2.60235263950445\\
2.80689642206507	2.53478936498328\\
2.86884572677734	2.46329489204857\\
2.92643083902505	2.38762159185311\\
2.9797601812272	2.30745004343857\\
3.02885684319924	2.22238778377882\\
3.0736524851218	2.13196669573504\\
3.1139782873357	2.03563940659629\\
3.14955257373531	1.9327751939472\\
3.1799647289397	1.82265610911853\\
3.20465508608693	1.70447436470255\\
3.22289063840819	1.57733253330384\\
3.23373681095027	1.44024881653547\\
3.23602624800484	1.29217060644661\\
3.22832680506483	1.13200078307626\\
3.20891290060994	0.958642592487318\\
3.17574729649874	0.771070271643144\\
3.12648431983601	0.568433254718831\\
3.05851021117547	0.350200754759043\\
2.96904057175168	0.116349140091352\\
2.85529634653472	-0.132415162797099\\
2.71477438513296	-0.394420939308942\\
2.54561149927938	-0.666831451081862\\
2.34700920116711	-0.945545149042266\\
2.11964414530881	-1.22527369541402\\
1.86595263790311	-1.4998524733049\\
1.59017201622944	-1.76278083859751\\
1.29806813358121	-2.00790742861739\\
0.996371540126414	-2.23010725848011\\
0.692045715840486	-2.42578400848801\\
0.391565820616375	-2.59308707986694\\
0.100367539191781	-2.73183034725394\\
-0.177450479556643	-2.84318720481823\\
-0.439174975715654	-2.92927763544535\\
-0.683341813504384	-2.99275420405671\\
-0.909499160727534	-3.03645626209875\\
-1.11794272622867	-3.06316020920615\\
-1.30947089185102	-3.07542341942466\\
-1.48518319381155	-3.075503866566\\
-1.64632800142023	-3.06533330557286\\
-1.79419521627932	-3.04652411049664\\
-1.93004558742833	-3.02039455884332\\
-2.05506754143312	-2.98800216298559\\
-2.17035346301742	-2.95017858038136\\
-2.27688899630141	-2.90756246369496\\
-2.37555057125685	-2.86062847315891\\
-2.46710774227685	-2.80971180667064\\
-2.55222799210507	-2.7550282383983\\
-2.63148242817394	-2.69668997476434\\
-2.70535133355913	-2.63471776521598\\
-2.77422888817933	-2.56904972849322\\
-2.83842659644711	-2.49954732641521\\
-2.89817508338303	-2.42599886934325\\
-2.95362398038452	-2.34812089091296\\
-3.00483963484408	-2.26555769827939\\
-3.05180036004648	-2.17787939952491\\
-3.09438890639883	-2.08457874433518\\
-3.13238179607603	-1.98506720320871\\
-3.16543513925834	-1.87867087564694\\
-3.19306656963785	-1.76462708792787\\
-3.21463304463943	-1.64208295384835\\
-3.22930452229022	-1.5100977722518\\
-3.23603405874339	-1.36765196884681\\
-3.23352582137385	-1.21366638503477\\
-3.22020408341029	-1.04703704589849\\
-3.19418868326632	-0.866691949135152\\
-3.15328586629968	-0.671677510030317\\
-3.0950078113619	-0.461282278775401\\
-3.01663883454107	-0.23520307506533\\
-2.91536954396166	0.00624810083846552\\
-2.788518847893	0.261910999084898\\
-2.63385304009768	0.529552178029452\\
-2.44998681340866	0.805705536654112\\
-2.23681291575937	1.0856526413263\\
-1.99586497695243	1.36361286468693\\
-1.7304935342104	1.63317401023785\\
-1.44575426300667	1.88792082565777\\
-1.14798021434768	2.12213786617229\\
-0.844113546482297	2.33141799237157\\
-0.540956133719094	2.51303014802294\\
-0.244517307879121	2.66598261140652\\
0.0404151513087646	2.79081670105124\\
0.310442717751683	2.88923292693133\\
0.563501955589499	2.96366615687849\\
0.798667665060156	3.0168997946765\\
1.01589444674839	3.05176677289582\\
1.21575754923351	3.07094824312334\\
1.39922803212736	3.07685806631181\\
};
\addplot [color=mycolor2, forget plot]
  table[row sep=crcr]{%
2.2994083961655	2.79535024862062\\
2.33353375244637	2.79429370988155\\
2.36308892670794	2.79149226656542\\
2.38899329883575	2.78736357111642\\
2.41194606003088	2.78219240353602\\
2.43248641775227	2.77617155467401\\
2.45103550347488	2.76942852007211\\
2.46792596493415	2.76204308154774\\
2.48342313216485	2.75405889901689\\
2.49774031819916	2.74549105552355\\
2.5110499595793	2.73633077068861\\
2.52349174076519	2.72654803607802\\
2.53517847016565	2.71609262142336\\
2.54620021486251	2.70489368741015\\
2.55662701141105	2.69285807709674\\
2.56651032075193	2.67986721498817\\
2.57588326302111	2.6657723971624\\
2.58475953213234	2.65038808497362\\
2.59313072766097	2.63348259180601\\
2.60096162271741	2.61476524012883\\
2.60818256613162	2.59386860939319\\
2.61467772279225	2.57032380743149\\
2.62026706536589	2.54352563953533\\
2.62467873315634	2.51268289255035\\
2.62750619885164	2.47674632157668\\
2.6281409707708	2.43430271030891\\
2.62566512363811	2.38341658835518\\
2.61867669255461	2.32139034728934\\
2.60500128248795	2.24439673584546\\
2.58120961318637	2.14691411622279\\
2.54180743093121	2.02087048189424\\
2.47789806938556	1.85441211532424\\
2.37511364375075	1.63039068861648\\
2.21099885010276	1.32541676630578\\
1.95381570727021	0.912531393401144\\
1.56939776571841	0.374364860947135\\
1.046278826301	-0.268373525565837\\
0.430224874707573	-0.935264083236797\\
-0.182442693596096	-1.52035883315628\\
-0.708085217542624	-1.96268482913067\\
-1.11780394439882	-2.26544616804894\\
-1.42325717122235	-2.46253403104913\\
-1.64882507222525	-2.58856111738334\\
-1.81697924574442	-2.66894695501801\\
-1.94460168654153	-2.72026685041513\\
-2.04349791307372	-2.7529009999111\\
-2.12175472716191	-2.7733118805893\\
-2.18492142143171	-2.78556135910919\\
-2.23685134067867	-2.79224307449155\\
-2.28026355484299	-2.79504237948199\\
-2.31711056707747	-2.79507291549566\\
-2.34881912484968	-2.79308127701272\\
-2.37644941612019	-2.78957347677697\\
-2.40080174380478	-2.78489445799613\\
-2.42248909688625	-2.7792789317652\\
-2.44198728711015	-2.77288436531976\\
-2.45967010440917	-2.7658126305574\\
-2.47583430670791	-2.75812428765506\\
-2.49071759523835	-2.74984796403495\\
-2.50451166243361	-2.74098636524612\\
-2.51737170827005	-2.73151987698026\\
-2.52942336291049	-2.72140834464276\\
-2.54076764180827	-2.71059136440109\\
-2.55148433878095	-2.69898723541734\\
-2.56163409628376	-2.68649057275093\\
-2.57125925396019	-2.67296843883943\\
-2.58038344507941	-2.65825469619767\\
-2.58900976450084	-2.64214209085254\\
-2.59711714568452	-2.62437131392721\\
-2.60465432163687	-2.60461591281531\\
-2.61153034875796	-2.58246136459062\\
-2.61760004960281	-2.55737577301318\\
-2.6226417206743	-2.52866832876134\\
-2.62632277559825	-2.49542958836365\\
-2.62814615814032	-2.45644430058206\\
-2.6273654794016	-2.41006215886969\\
-2.62284832830332	-2.3540032622465\\
-2.61285230238781	-2.28506149819036\\
-2.59465244750208	-2.19864880205721\\
-2.56391581427681	-2.08809753480014\\
-2.51365623921814	-1.94362297330129\\
-2.43254884392664	-1.75091368452257\\
-2.30250431096509	-1.48970502660703\\
-2.09630378072396	-1.13403862962795\\
-1.77920298094463	-0.659113081286025\\
-1.32393593034075	-0.0624067233417369\\
-0.744435555399311	0.605702954626298\\
-0.117127565699744	1.24352382354984\\
0.459258666404913	1.76037734431491\\
0.927284812249035	2.12965076349379\\
1.28209324409541	2.37485769803888\\
1.54450086032372	2.53260670741068\\
1.73889001480468	2.63324617543375\\
1.88501585025053	2.69747674676842\\
1.99706351776802	2.73844513430852\\
2.0848113928928	2.76433742219006\\
2.15495148237977	2.78026822507319\\
2.21209951830831	2.7894763258861\\
2.25948547345075	2.79404790700794\\
2.2994083961655	2.79535024862062\\
};
\addplot [color=mycolor1, forget plot]
  table[row sep=crcr]{%
1.55639256905055	2.96512930676979\\
1.72194558400121	2.95995959644899\\
1.87187651283173	2.94571684865297\\
2.0078786804364	2.92401863646909\\
2.13152133501261	2.89614746612155\\
2.24422281966148	2.86310233390609\\
2.34724285915032	2.82564636890681\\
2.44168604901987	2.78434809112897\\
2.52851125160194	2.73961565129869\\
2.60854350893253	2.69172436094\\
2.68248638804546	2.64083822956509\\
2.7509335267552	2.58702633781514\\
2.81437867968246	2.53027483711349\\
2.87322387711878	2.4704952622475\\
2.92778547640811	2.40752971955752\\
2.97829795635211	2.34115339425146\\
3.02491531197481	2.27107471816354\\
3.06770987022764	2.19693346115496\\
3.10666827982388	2.1182969614416\\
3.14168433945825	2.03465470072291\\
3.17254822689084	1.94541147237283\\
3.19893158908426	1.8498795065186\\
3.22036787230806	1.74727013771703\\
3.23622724996468	1.63668597826971\\
3.24568561424095	1.51711516213764\\
3.24768745381299	1.38743014141285\\
3.24090323560218	1.24639485411533\\
3.22368343638461	1.09268593009691\\
3.19401403685822	0.924935969093117\\
3.14948258066916	0.741809589413677\\
3.08727020568363	0.542125192126739\\
3.00419323787546	0.325035616898807\\
2.89682647873737	0.0902762099333626\\
2.76174504315679	-0.161524827654143\\
2.59591463370637	-0.428505535065347\\
2.39723087888561	-0.707269802183286\\
2.16514959094201	-0.992739868771152\\
1.90127011825929	-1.27829123417025\\
1.60966980423026	-1.55625824365859\\
1.29679620703209	-1.81878348201875\\
0.970842704091363	-2.05883160002194\\
0.640728210317661	-2.2710859078631\\
0.31496544686145	-2.45247227019177\\
0.000731420951075753	-2.60220433110341\\
-0.296661411711697	-2.72142462868948\\
-0.573868667402983	-2.81262680605443\\
-0.829299485755192	-2.87905047295545\\
-1.06273726530523	-2.92417697995347\\
-1.27492670448002	-2.95137672496134\\
-1.4672079009809	-2.96370222005834\\
-1.64123122171026	-2.96379418017215\\
-1.79875555766596	-2.95386278896524\\
-1.94151727446253	-2.93571227958493\\
-2.07115260235749	-2.91078617505879\\
-2.18915727467812	-2.88021893913876\\
-2.29687043821154	-2.84488604726812\\
-2.39547334929136	-2.8054486049037\\
-2.48599634346008	-2.7623911007221\\
-2.56932981975949	-2.71605221826704\\
-2.64623656817436	-2.66664926811465\\
-2.71736382975639	-2.61429703952844\\
-2.78325415636491	-2.55902189368348\\
-2.84435454835788	-2.50077184161734\\
-2.90102358102023	-2.43942323201905\\
-2.95353634450248	-2.37478455042308\\
-3.00208705769334	-2.30659771959634\\
-3.04678919949292	-2.23453720000039\\
-3.08767294758247	-2.15820712535994\\
-3.12467963559481	-2.07713667842455\\
-3.157652843028	-1.9907739264093\\
-3.18632562825139	-1.89847841126238\\
-3.21030331962046	-1.79951295359335\\
-3.22904122207912	-1.6930354213926\\
-3.24181662900644	-1.57809169429292\\
-3.2476947425595	-1.45361180105303\\
-3.24548865203891	-1.31841232158201\\
-3.23371463787672	-1.17120972867802\\
-3.21054610855636	-1.01065146104482\\
-3.17377289507163	-0.835374085829839\\
-3.12077789444459	-0.644100498574099\\
-3.04855037558751	-0.435789619224154\\
-2.95376394575681	-0.209850277976997\\
-2.83295454904904	0.0335776083465671\\
-2.68283396756642	0.293291226760688\\
-2.50075751710368	0.566692571995787\\
-2.28532087855846	0.849548411045377\\
-2.03698913670611	1.13596864265365\\
-1.75858277962264	1.41872109049825\\
-1.45541039978119	1.68992187208685\\
-1.13490047418897	1.94199817831076\\
-0.805751468807456	2.1686805329637\\
-0.476814596794987	2.36573772500202\\
-0.156028488172868	2.53126267406597\\
0.150323962960791	2.66549690559156\\
0.437924755038436	2.77033638872705\\
0.70434694889485	2.84871913201416\\
0.948738849466358	2.9040595499326\\
1.17141446008507	2.93981790811621\\
1.3734572074647	2.95922355824186\\
1.55639256905055	2.96512930676979\\
};
\addplot [color=mycolor2, forget plot]
  table[row sep=crcr]{%
2.25648148144183	2.81152788377295\\
2.28611854761594	2.81060971185361\\
2.31188301568343	2.80816706740784\\
2.33454818820421	2.80455419539165\\
2.3547033978906	2.80001290555633\\
2.37280419498227	2.79470678315618\\
2.38920726532027	2.78874353664228\\
2.40419509161646	2.78218971157443\\
2.41799360188431	2.77508036725945\\
2.43078493427593	2.76742532867936\\
2.44271673263075	2.75921301833719\\
2.45390891910292	2.75041248499556\\
2.46445857600876	2.74097398775105\\
2.47444335022407	2.73082830913038\\
2.48392363222904	2.71988482420959\\
2.49294363197708	2.70802821854096\\
2.50153135458867	2.69511360418144\\
2.50969735110584	2.68095960758999\\
2.51743196071794	2.66533876684148\\
2.52470053987407	2.64796423727343\\
2.53143584233227	2.62847130085012\\
2.53752619284462	2.60639140337675\\
2.5427972491997	2.58111523783408\\
2.54698373201901	2.5518394717744\\
2.54968508637137	2.51748860956842\\
2.55029483420252	2.47659838720192\\
2.54788592718438	2.42713868356417\\
2.54102104395717	2.36624006572208\\
2.52743272100318	2.28976577528399\\
2.50347567868718	2.19163770921451\\
2.46318348874838	2.06278631195425\\
2.39666951885793	1.88959472276959\\
2.28759809150629	1.6519365163752\\
2.11000044012681	1.3219854402897\\
1.82735074161516	0.868276902077198\\
1.4035409516527	0.274924351724607\\
0.837870821427439	-0.420341133225879\\
0.201098027006392	-1.1100709252071\\
-0.395668872192818	-1.68034674015666\\
-0.880314982894813	-2.08838402879004\\
-1.24348455333275	-2.35683855332146\\
-1.507927557234	-2.52749837896194\\
-1.70086564924503	-2.63530514001686\\
-1.8439884403667	-2.7037269596881\\
-1.95252793963369	-2.74737273901532\\
-2.03676473790973	-2.77516838314625\\
-2.10360615644679	-2.79260071553722\\
-2.15774274589135	-2.80309798095731\\
-2.2024142036615	-2.80884485754463\\
-2.23990137594796	-2.81126134200048\\
-2.27184140715831	-2.81128716861249\\
-2.29943149129235	-2.80955366557274\\
-2.32356254689679	-2.8064896336933\\
-2.3449084001642	-2.80238785252393\\
-2.36398631770982	-2.79744760908678\\
-2.38119878662054	-2.79180230981551\\
-2.39686280734885	-2.78553760429208\\
-2.41123072445566	-2.7787033287183\\
-2.42450521898287	-2.77132131320246\\
-2.43685019558303	-2.76339032616339\\
-2.44839872074248	-2.75488894591073\\
-2.45925878639586	-2.74577683503627\\
-2.46951741246095	-2.73599467701678\\
-2.4792434156444	-2.72546287236615\\
-2.48848902910439	-2.71407895406076\\
-2.49729043540429	-2.70171354578068\\
-2.50566715422489	-2.6882045300937\\
-2.5136200862035	-2.67334889198013\\
-2.52112782906662	-2.65689142202918\\
-2.52814061359786	-2.63850905241165\\
-2.53457079335508	-2.61778897725139\\
-2.54027815951578	-2.59419774705328\\
-2.54504725993472	-2.56703700835965\\
-2.54855205743463	-2.53537912159193\\
-2.55030008292073	-2.49797191743159\\
-2.54954265294916	-2.45309530925004\\
-2.54512775459313	-2.39834166414063\\
-2.53525426020177	-2.33027415548639\\
-2.51705401219814	-2.24388966249411\\
-2.48587290072106	-2.13177492923707\\
-2.43403748503547	-1.98281496038808\\
-2.34881497238625	-1.7803862861264\\
-2.20943487700202	-1.50049942326001\\
-1.98435235053856	-1.11234032997585\\
-1.63453666967953	-0.588442579984248\\
-1.13579855446416	0.0653739113129525\\
-0.521391255039876	0.774076688632134\\
0.108412266451223	1.41484922999526\\
0.65379499857227	1.90419112353122\\
1.07599241105833	2.23744713376225\\
1.38626469515738	2.45193098293142\\
1.61181650075467	2.58754262213015\\
1.77756525843863	2.67335950991766\\
1.90184282954443	2.72798703390261\\
1.99718914272427	2.76284793443485\\
2.07202535032604	2.78492909576245\\
2.13203289943638	2.79855737073537\\
2.18110095632948	2.80646258005474\\
2.2219412271938	2.8104018315038\\
2.25648148144183	2.81152788377295\\
};
\addplot [color=mycolor1, forget plot]
  table[row sep=crcr]{%
1.75789046365152	2.87851795059347\\
1.92027634890559	2.8734607469727\\
2.06510472919776	2.85971412359751\\
2.19462378613701	2.83905983815503\\
2.31083799198744	2.81287128241266\\
2.41549985134058	2.78219036762057\\
2.51012168931292	2.74779378118756\\
2.59599693355803	2.71024730376353\\
2.67422461521597	2.66994888995881\\
2.74573358597395	2.62716200532674\\
2.81130464942246	2.58204086930475\\
2.87158979656993	2.53464911970734\\
2.92712827874576	2.48497317127276\\
2.9783595203119	2.43293127841629\\
3.02563297884943	2.37837907014263\\
3.06921506861588	2.32111211682189\\
3.10929321275395	2.2608659162411\\
3.14597700189229	2.19731354772933\\
3.17929632141098	2.13006113583178\\
3.20919617033611	2.05864118915142\\
3.23552773207612	1.98250384064836\\
3.25803507156029	1.9010060256971\\
3.27633662949837	1.81339871826247\\
3.2899004779828	1.71881254651789\\
3.29801213040874	1.61624249714943\\
3.29973364297103	1.50453310189055\\
3.29385296085882	1.38236664313758\\
3.27882323389149	1.2482587430566\\
3.25269363956878	1.10056848829626\\
3.2130368754626	0.937534243902808\\
3.15688500245051	0.757351543854944\\
3.08069596130842	0.558315211712442\\
2.98038849216385	0.339051842429407\\
2.85150160724088	0.0988657329701925\\
2.68954871829273	-0.161797793059755\\
2.49062888672968	-0.440814094262145\\
2.25230211025796	-0.733886406655619\\
1.97461137213715	-1.03430825388099\\
1.66096254637496	-1.33322930069096\\
1.31845337095567	-1.62057461227626\\
0.95732198834882	-1.88650319769001\\
0.589521285820144	-2.12298216439976\\
0.226850550200004	-2.32492867802353\\
-0.120713759157173	-2.49056304573379\\
-0.446003476157265	-2.62099290234318\\
-0.74485004764125	-2.71934212688711\\
-1.01568637964931	-2.78979839031806\\
-1.25887356310461	-2.83683344653703\\
-1.47600882825676	-2.86468792887994\\
-1.66935421231228	-2.87709929910705\\
-1.8414272424568	-2.87720508913988\\
-1.99473930098236	-2.86755172407349\\
-2.13164670382921	-2.85015597603359\\
-2.25427862994713	-2.82658529012176\\
-2.36451303746788	-2.79803830517911\\
-2.46398008560633	-2.76541678541352\\
-2.5540796809791	-2.72938590743431\\
-2.63600495018158	-2.69042280163042\\
-2.71076690996416	-2.64885455478097\\
-2.77921779159952	-2.60488729808164\\
-2.84207178352668	-2.55862798542887\\
-2.89992269720416	-2.51010026245484\\
-2.9532584501465	-2.45925556676253\\
-3.00247243669148	-2.40598034463359\\
-3.04787190743458	-2.35010004350221\\
-3.08968345364505	-2.29138034936013\\
-3.12805562209542	-2.2295259831704\\
-3.16305858336976	-2.1641772475904\\
-3.19468064919358	-2.0949044234456\\
-3.22282128331157	-2.02120005676339\\
-3.24728007602649	-1.94246916029047\\
-3.26774095669969	-1.8580173964057\\
-3.28375071035319	-1.76703744404503\\
-3.29469066885278	-1.66859403579765\\
-3.29974031987945	-1.56160867076005\\
-3.29783163093767	-1.44484589838781\\
-3.28759333321043	-1.3169045219713\\
-3.26728562584676	-1.17621934205837\\
-3.23472836344095	-1.02108243033109\\
-3.18723070850975	-0.849697569083211\\
-3.12153866748348	-0.660287153423712\\
-3.03382996236572	-0.451276164627668\\
-2.91980308567388	-0.221579177725888\\
-2.77492503123962	0.0289933634714278\\
-2.59490818701817	0.299224837097703\\
-2.37645882199743	0.585956016402782\\
-2.11825076772733	0.883689041791611\\
-1.82192293749608	1.18457256302336\\
-1.49273614167424	1.47898872208505\\
-1.1394873165461	1.75677715837975\\
-0.773492177610049	2.00881906414316\\
-0.406862835189353	2.22845893954485\\
-0.0506604067013423	2.41227436255297\\
0.286471226804994	2.5600189838237\\
0.598883778632005	2.67393043307189\\
0.883783632138136	2.75777669318722\\
1.14065779203266	2.8159688844512\\
1.3705662997484	2.85291106863863\\
1.57550018828592	2.87261358525458\\
1.75789046365152	2.87851795059347\\
};
\addplot [color=mycolor2, forget plot]
  table[row sep=crcr]{%
2.20548089943823	2.82259492056614\\
2.2310756781062	2.82180144067627\\
2.25341388640125	2.81968316600984\\
2.27314045325173	2.81653830900669\\
2.29074816700108	2.81257064955839\\
2.30661901094925	2.80791789296031\\
2.32105293751804	2.80267021067225\\
2.33428820981845	2.79688244343441\\
2.34651597522875	2.7905821012801\\
2.35789081685965	2.78377448399526\\
2.36853844083513	2.77644574287847\\
2.37856127247992	2.76856438176995\\
2.38804247531326	2.76008147691584\\
2.39704872485898	2.75092973518643\\
2.40563193312944	2.74102137806329\\
2.41383000688234	2.7302447118763\\
2.42166661519277	2.71845910236796\\
2.42914982141766	2.70548789061109\\
2.43626927894883	2.69110853666488\\
2.44299146696314	2.67503891108904\\
2.4492521000945	2.65691809977387\\
2.45494429738749	2.63627922524096\\
2.45990018838628	2.6125104174842\\
2.46386209338476	2.58479784748374\\
2.46643673675452	2.55204107421729\\
2.46702119429798	2.51272481779732\\
2.4646806518639	2.46472086352629\\
2.45794217040651	2.40497610469501\\
2.44443919338808	2.32901312638822\\
2.42028758225714	2.23012319567484\\
2.37898108438301	2.0980720291474\\
2.30946546712203	1.91712270122127\\
2.19302308476548	1.663480368929\\
1.99939986372173	1.30384507534387\\
1.68660213849871	0.801798313227344\\
1.21843474900127	0.146242380030686\\
0.612004139575645	-0.599490011561192\\
-0.0325838159621133	-1.29818124981711\\
-0.597959798471228	-1.83882174568265\\
-1.03282192882622	-2.20511881340256\\
-1.34748998865958	-2.43778649253415\\
-1.57235808532948	-2.58292713607252\\
-1.73507076486851	-2.67385011156894\\
-1.85550395067858	-2.73142497897582\\
-1.94693408294812	-2.76818955272608\\
-2.01809082677471	-2.79166777757401\\
-2.07475971740468	-2.80644585476573\\
-2.1208428346447	-2.8153804828952\\
-2.15902755535032	-2.8202920002674\\
-2.19120483273771	-2.82236549524384\\
-2.21873309374681	-2.82238716666129\\
-2.24260738515186	-2.82088663063703\\
-2.26356981838853	-2.81822450201677\\
-2.28218308232773	-2.81464742928703\\
-2.29888029991503	-2.81032332627052\\
-2.31399944553803	-2.80536427229711\\
-2.32780749443154	-2.79984154937507\\
-2.34051761431891	-2.7937955368736\\
-2.35230155238913	-2.78724214311511\\
-2.36329863765224	-2.78017681720398\\
-2.37362234450946	-2.77257678363149\\
-2.38336504885646	-2.76440187873777\\
-2.39260139227276	-2.75559418343563\\
-2.40139051405918	-2.74607650370016\\
-2.40977728886199	-2.73574962341048\\
-2.41779259964182	-2.72448812209074\\
-2.42545256414998	-2.71213439161339\\
-2.43275649824911	-2.69849027468424\\
-2.43968321494595	-2.6833054465566\\
-2.44618498344281	-2.66626121236649\\
-2.45217804115527	-2.64694770291678\\
-2.45752784854098	-2.62483136757586\\
-2.46202609767439	-2.59920792305946\\
-2.46535445985736	-2.56913307163661\\
-2.46702649734364	-2.5333185694399\\
-2.46629277352527	-2.48997324111257\\
-2.46198251161568	-2.43655495878129\\
-2.45223352200701	-2.3693766236903\\
-2.43402210812818	-2.28297159406618\\
-2.40233294227727	-2.169068980355\\
-2.34869425491878	-2.01497722303928\\
-2.25868913883791	-1.80125551068714\\
-2.10827647682834	-1.49929940663201\\
-1.86071669856813	-1.07246009907866\\
-1.47281070198137	-0.491513662073968\\
-0.927929490211162	0.223025382031232\\
-0.286220268576624	0.963687847270177\\
0.330295777802333	1.59139355984857\\
0.83197763016014	2.04178659571054\\
1.20336178905379	2.3350431603245\\
1.4692458554266	2.51888008351382\\
1.66006234090684	2.63361810391247\\
1.79961307762266	2.7058727610355\\
1.90421394291801	2.75185040212565\\
1.98463077787894	2.78125135738429\\
2.04795758667964	2.79993522196825\\
2.09893391063224	2.81151127678533\\
2.14078931321682	2.81825353172206\\
2.17577207927344	2.82162702115639\\
2.20548089943823	2.82259492056614\\
};
\addplot [color=mycolor1, forget plot]
  table[row sep=crcr]{%
2.02528855722988	2.8314627757905\\
2.18364256490326	2.82654712415527\\
2.32232089875638	2.81339711001726\\
2.44429966663803	2.79395569352328\\
2.55211787844742	2.7696676812432\\
2.64791266060073	2.74159314789719\\
2.73346857162937	2.71049808923055\\
2.81026898835822	2.67692434239347\\
2.87954376279983	2.64124227745091\\
2.94231081353296	2.60368983175353\\
2.99941111331336	2.56440098104945\\
3.05153738352734	2.52342612389607\\
3.09925713864025	2.48074627096032\\
3.14303078842218	2.43628243603582\\
3.18322543891858	2.38990122822675\\
3.22012490614952	2.34141733232042\\
3.25393630364273	2.29059331914839\\
3.28479339967968	2.23713703273042\\
3.31275676441611	2.18069664229565\\
3.33781053586055	2.12085331497088\\
3.3598554178422	2.05711135386713\\
3.37869727070575	1.98888555754715\\
3.3940303537022	1.91548550066498\\
3.40541391624517	1.83609643696888\\
3.41224041172014	1.74975663464195\\
3.41369314422302	1.65533126155474\\
3.40869073030275	1.55148360479601\\
3.39581554937713	1.43664570536127\\
3.37322376767736	1.30899285086508\\
3.33853634462256	1.16643045198713\\
3.28871512464255	1.0066085044921\\
3.21993811678531	0.826988980881285\\
3.12750686843875	0.625005188811786\\
3.0058500389319	0.398366670510614\\
2.84873066938348	0.145569184539406\\
2.64980675565619	-0.133354564135314\\
2.40369384305287	-0.435895060253338\\
2.10755788715585	-0.756164807355284\\
1.76295726042551	-1.08448738339765\\
1.37722220051641	-1.40802207807992\\
0.96344170103583	-1.71267650518036\\
0.538553849418496	-1.98584885182753\\
0.120068015783935	-2.21889121823761\\
-0.277199285299691	-2.40824585880885\\
-0.643099822950995	-2.55500109057541\\
-0.972465706932159	-2.66343648969624\\
-1.26423038869403	-2.73937577017955\\
-1.52009738124336	-2.78889636149337\\
-1.74329963282843	-2.81755650507628\\
-1.93768265558867	-2.83005667671636\\
-2.10712845574068	-2.83017869236823\\
-2.25524470362919	-2.82086681791378\\
-2.38522909864705	-2.80436229258904\\
-2.4998363994733	-2.78234341125037\\
-2.60139901110545	-2.75604987509515\\
-2.69187128999656	-2.72638475821845\\
-2.77288092865411	-2.69399440086462\\
-2.84577893581548	-2.65932925322217\\
-2.91168441125529	-2.62268931527768\\
-2.97152283345203	-2.58425754092257\\
-3.02605783251237	-2.54412399449862\\
-3.07591697091745	-2.50230293447381\\
-3.12161223055129	-2.45874445601906\\
-3.16355589123708	-2.41334187899278\\
-3.20207238300434	-2.36593571503202\\
-3.2374065516678	-2.31631477087814\\
-3.26972861743574	-2.26421472691255\\
-3.29913593606771	-2.20931435452912\\
-3.32565148958407	-2.15122939143132\\
-3.34921883136506	-2.089503972563\\
-3.36969297794654	-2.02359941386525\\
-3.38682646430956	-1.95288007154884\\
-3.40024944867019	-1.87659596779902\\
-3.40944235934206	-1.7938619205939\\
-3.41369912774013	-1.70363310855882\\
-3.41207859133372	-1.60467746319032\\
-3.40334129865304	-1.49554621970708\\
-3.3858689805257	-1.37454571988291\\
-3.35756493606156	-1.23971668570371\\
-3.31573660462241	-1.08883244372372\\
-3.25696856650511	-0.919435900458483\\
-3.17700812873283	-0.728947083919059\\
-3.07071029907343	-0.514887822395679\\
-2.93212679730042	-0.275282186335697\\
-2.75486957540281	-0.00928591918375054\\
-2.53290747287493	0.28195292691212\\
-2.26190501443769	0.594330814017856\\
-1.94100156034264	0.92006421124295\\
-1.57453530079942	1.24773500771413\\
-1.17283460989327	1.56356523148982\\
-0.751263100438952	1.85385259421138\\
-0.327490030079096	2.10772970392302\\
0.0819728852720324	2.31905616496465\\
0.464503781835654	2.48673557094366\\
0.81249731627353	2.61366335439495\\
1.12299079725902	2.70508271574215\\
1.39647235888047	2.76707313014238\\
1.63554733779064	2.80551841565643\\
1.8438470703118	2.82556921699742\\
2.02528855722988	2.8314627757905\\
};
\addplot [color=mycolor2, forget plot]
  table[row sep=crcr]{%
2.14285046027138	2.82808756846245\\
2.16480232914021	2.82740651091117\\
2.18404462096453	2.82558137619315\\
2.2011087002191	2.82286059781474\\
2.21640196057306	2.81941412880184\\
2.23024128938241	2.81535662847848\\
2.24287640264083	2.81076264354564\\
2.25450637336062	2.80567659600452\\
2.26529149609173	2.80011930365334\\
2.27536189104698	2.79409210175755\\
2.28482377819513	2.78757922486641\\
2.2937640413391	2.78054884188152\\
2.30225349204603	2.7729529542352\\
2.3103490941841	2.76472622844156\\
2.31809529588709	2.75578371453312\\
2.32552451819258	2.74601728057389\\
2.33265675227913	2.73529045062384\\
2.33949810305412	2.72343114522586\\
2.34603796346328	2.7102215560015\\
2.3522442775425	2.69538398746693\\
2.35805599539808	2.67856088584384\\
2.36337124406967	2.65928630618278\\
2.36802876326672	2.63694450529085\\
2.37177846998737	2.61070876898528\\
2.37423403079971	2.57944924242777\\
2.37479490319824	2.54159109198231\\
2.37251525076091	2.49489137507254\\
2.36587810379294	2.43608027600758\\
2.35239671276938	2.36027289623144\\
2.32789590352291	2.25999265404638\\
2.28520228049655	2.12355755666052\\
2.2117927329946	1.93254039078384\\
2.08591373136419	1.65843271354778\\
1.87192808224271	1.26107452856134\\
1.52191544631949	0.699323977495348\\
1.00401328873093	-0.0260975653771542\\
0.363748071527552	-0.813984659430715\\
-0.2688644928636	-1.50026846347405\\
-0.78505984725054	-1.99421220096836\\
-1.1621814512543	-2.3120046844514\\
-1.42730501323027	-2.50808040891141\\
-1.61430555121665	-2.62878977651985\\
-1.74908287219487	-2.70410346009035\\
-1.84892536871792	-2.75183309069718\\
-1.92497168994207	-2.78240988395908\\
-1.98441631057396	-2.80202204953594\\
-2.03198774293826	-2.81442637276709\\
-2.07086511691551	-2.82196289321932\\
-2.1032373715361	-2.82612594060473\\
-2.1306469031497	-2.82789152306617\\
-2.15420434386284	-2.82790950724923\\
-2.17472557537473	-2.82661924323466\\
-2.19282102135335	-2.8243208006842\\
-2.20895506929689	-2.82121982114766\\
-2.22348640556655	-2.81745629556059\\
-2.23669590171479	-2.81312330436156\\
-2.24880621761687	-2.80827933132791\\
-2.2599957841219	-2.80295635057308\\
-2.27040889628117	-2.79716504396977\\
-2.28016305849344	-2.79089798964273\\
-2.2893543410282	-2.78413133413755\\
-2.29806125319669	-2.77682524217822\\
-2.30634746269321	-2.76892326079756\\
-2.31426356178228	-2.7603506082126\\
-2.32184797734626	-2.75101127969837\\
-2.32912702634906	-2.74078373327658\\
-2.33611401487627	-2.72951475592708\\
-2.34280714846345	-2.717010888234\\
-2.34918583682532	-2.70302646046144\\
-2.35520469439475	-2.68724680041715\\
-2.36078408665524	-2.66926440501115\\
-2.36579532337417	-2.64854464007714\\
-2.37003732217863	-2.62437552995137\\
-2.37319932996421	-2.59579286104035\\
-2.37480028000697	-2.56146615364026\\
-2.3740880014168	-2.51952125498264\\
-2.36986769099099	-2.46725815618681\\
-2.3602027423829	-2.40069260335542\\
-2.34188069376843	-2.31379881220432\\
-2.30944331514121	-2.19725108711924\\
-2.25342347579248	-2.03637727552634\\
-2.15726868574138	-1.8081318719905\\
-1.99277374169129	-1.47800379565238\\
-1.7169275365176	-1.00247264364514\\
-1.2836338369358	-0.353478106408908\\
-0.692128251530688	0.422602884473332\\
-0.0377635087113686	1.17848387603558\\
0.544781901732044	1.77206572627127\\
0.989822588600999	2.17182516324887\\
1.30650097833272	2.42196025609287\\
1.52872677843643	2.57563263041337\\
1.68696675115438	2.67078647683365\\
1.80256019550217	2.73063605644361\\
1.88940158578403	2.76880563330631\\
1.95642880997747	2.79330947859752\\
2.00945904775913	2.80895397684342\\
2.05235799123028	2.81869458288088\\
2.08775582377096	2.82439568419062\\
2.11748483909939	2.82726178475665\\
2.14285046027137	2.82808756846245\\
};
\addplot [color=mycolor1, forget plot]
  table[row sep=crcr]{%
2.39662868959425	2.84951082055756\\
2.54951631809144	2.84478316146543\\
2.68057577202693	2.83236964475803\\
2.79368931059282	2.81435210655368\\
2.89200956708609	2.79221227117425\\
2.97808052474476	2.76699436724974\\
3.05395082474626	2.73942510677382\\
3.12127101874159	2.71000031675759\\
3.18137334210608	2.67904668602785\\
3.23533549826592	2.6467653770816\\
3.28403088381477	2.61326255303785\\
3.32816773512192	2.5785704660063\\
3.36831937711644	2.54266168096823\\
3.40494735390697	2.50545821962719\\
3.43841882493674	2.46683683518139\\
3.46901925630134	2.42663121128447\\
3.49696112960295	2.3846315684237\\
3.52238912121525	2.34058192200471\\
3.54538195800655	2.29417504129906\\
3.56595091331318	2.24504498662911\\
3.58403464861197	2.19275693830407\\
3.599489807854	2.13679386361574\\
3.61207640388388	2.0765393910213\\
3.62143656457363	2.01125607413664\\
3.6270645884653	1.9400580467324\\
3.62826545094727	1.86187693524784\\
3.62409786740233	1.77541990303787\\
3.61329677027294	1.67911905078622\\
3.59416873522277	1.57107248739018\\
3.56445296290204	1.44897996970295\\
3.52114107269472	1.31008145360664\\
3.46025386082096	1.15111750821774\\
3.3765877477028	0.96834975663678\\
3.26347767903156	0.757711152712086\\
3.11269107120892	0.515200371534801\\
2.91468049121388	0.237677533331435\\
2.65956107621816	-0.075792005356255\\
2.33922687728764	-0.42207479588027\\
1.95070870148828	-0.79208503593864\\
1.49992499673157	-1.17005005898236\\
1.00369620923486	-1.53532883829985\\
0.487714276647776	-1.86704741987091\\
-0.0196037104394338	-2.14958815963688\\
-0.494266489177459	-2.37589438155797\\
-0.921094490007363	-2.54715856090294\\
-1.29407185464677	-2.67002059913229\\
-1.61405066260037	-2.7533617882459\\
-1.88582793930159	-2.80600842284946\\
-2.11580931472076	-2.8355752154477\\
-2.31055960343822	-2.8481265843826\\
-2.4760753310711	-2.84826673357076\\
-2.61751342695858	-2.83939071981517\\
-2.73916306057416	-2.82395684149332\\
-2.84452673110119	-2.8037235494515\\
-2.93643778478825	-2.77993637218657\\
-3.01717928551152	-2.7534680827133\\
-3.08858977970994	-2.72492085634415\\
-3.15215174070506	-2.69469950321393\\
-3.20906307246607	-2.66306341554223\\
-3.26029379479034	-2.63016310153615\\
-3.30663043348983	-2.59606561165381\\
-3.3487104699092	-2.56077192816399\\
-3.38704883368716	-2.52422846610237\\
-3.42205801644831	-2.48633416034531\\
-3.45406300707221	-2.44694412473863\\
-3.48331191905296	-2.40587051060485\\
-3.5099828937072	-2.36288092139323\\
-3.53418760693175	-2.31769452613295\\
-3.55597146508519	-2.26997583310268\\
-3.57531032803307	-2.21932591896988\\
-3.59210332225446	-2.1652707442698\\
-3.60616097787423	-2.10724601437021\\
-3.61718750864978	-2.04457786249847\\
-3.62475551469728	-1.97645844436705\\
-3.62827067991085	-1.90191536813603\\
-3.6269231170911	-1.8197738030447\\
-3.61962086504921	-1.72861025449313\\
-3.60489972458324	-1.62669764617954\\
-3.58080240451189	-1.51194306149107\\
-3.5447195921549	-1.38182328893359\\
-3.49318786598444	-1.23333098397222\\
-3.42164818690977	-1.06295867761356\\
-3.32419146548788	-0.866772876631447\\
-3.19336668810003	-0.64066903948441\\
-3.0202170318666	-0.380945389129189\\
-2.79484166778538	-0.0853597202615685\\
-2.50789777204493	0.24524215579012\\
-2.15336789331032	0.60494911172521\\
-1.7323041212614	0.981292060555261\\
-1.25603671291713	1.35564225777066\\
-0.74637262927147	1.70653835040823\\
-0.23129746125784	2.01512023532681\\
0.262194573726511	2.26986342003333\\
0.714231227266017	2.46807926651014\\
1.11438689035123	2.61410444947348\\
1.46044698874666	2.71605977891121\\
1.75558068285544	2.78301068879355\\
2.00562516112094	2.82326121103597\\
2.21719544458287	2.84365845322455\\
2.39662868959425	2.84951082055756\\
};
\addplot [color=mycolor2, forget plot]
  table[row sep=crcr]{%
2.06125160838935	2.82609185066612\\
2.07994925124581	2.82551123450406\\
2.09642363792935	2.82394819411031\\
2.11110532962897	2.82160689086134\\
2.12432597927662	2.81862716062496\\
2.13634475923024	2.81510310819691\\
2.14736685278038	2.81109530783169\\
2.15755659642858	2.80663882557853\\
2.16704694356782	2.80174842493572\\
2.17594634547779	2.79642179774859\\
2.18434377663302	2.79064133531615\\
2.19231238805512	2.78437473909726\\
2.19991210618597	2.77757461782079\\
2.20719137482391	2.77017709745021\\
2.2141881433638	2.76209936047244\\
2.22093012022366	2.7532359127787\\
2.22743422158551	2.74345323023988\\
2.23370503599168	2.73258223742678\\
2.23973197208311	2.72040778022596\\
2.24548452388585	2.70665381282011\\
2.25090471474374	2.69096232906942\\
2.25589515985366	2.67286296058742\\
2.26030012253552	2.65172834480686\\
2.26387506387652	2.62670731102676\\
2.26623679389436	2.59662268469411\\
2.26678003680328	2.55981130214178\\
2.26453423629103	2.51386736871076\\
2.25791108650606	2.45522053002431\\
2.24424716451873	2.37842646554096\\
2.21895543141708	2.27495574519942\\
2.17393123321108	2.13113357503807\\
2.09461115491916	1.92481897622353\\
1.95508074839113	1.62108993957651\\
1.71280218315468	1.17128586036876\\
1.31487939823906	0.532571479500302\\
0.744373465911252	-0.267011737689928\\
0.0892415081571163	-1.0739861548735\\
-0.501235778002658	-1.71516536094982\\
-0.94830109085819	-2.14322609187765\\
-1.26079643502598	-2.40664257753268\\
-1.47616902870883	-2.56594328497232\\
-1.62722885702031	-2.66345388897609\\
-1.73627697297688	-2.72438721319827\\
-1.81745907450743	-2.76319317280894\\
-1.87968387289177	-2.78821006640888\\
-1.92865279428784	-2.80436411932649\\
-1.96810432160467	-2.8146496981553\\
-2.00055506641086	-2.82093927034229\\
-2.02774317329863	-2.82443476449945\\
-2.05089835097381	-2.82592560443978\\
-2.0709100552968	-2.8259403106581\\
-2.08843474420915	-2.82483797460678\\
-2.103965880597	-2.8228648340703\\
-2.11788063559145	-2.82019004338289\\
-2.13047168893757	-2.81692871332125\\
-2.1419692891369	-2.81315695930412\\
-2.15255681437367	-2.80892179935339\\
-2.16238190864915	-2.80424763616303\\
-2.17156454387382	-2.79914039392197\\
-2.18020289961054	-2.79358997034857\\
-2.18837765371054	-2.78757140076561\\
-2.19615507704615	-2.78104495165163\\
-2.2035891855187	-2.77395522776637\\
-2.21072309758051	-2.76622926419841\\
-2.21758965797856	-2.75777346288815\\
-2.22421130370965	-2.74846910370544\\
-2.23059905134031	-2.73816599101705\\
-2.23675035689961	-2.72667355727781\\
-2.2426454121733	-2.7137483885527\\
-2.24824114796816	-2.69907658675329\\
-2.25346173532385	-2.68224851142628\\
-2.25818356520954	-2.66272202846742\\
-2.26221127873464	-2.63976804230512\\
-2.26523890612181	-2.61238809494481\\
-2.26678556693555	-2.57918688165498\\
-2.26608652368133	-2.53817024701707\\
-2.26190369584654	-2.48641710820626\\
-2.25218696779675	-2.41953373467694\\
-2.23345280986463	-2.3307278718439\\
-2.19962169333636	-2.20922521273513\\
-2.13984083508523	-2.03762134861856\\
-2.0346032412399	-1.78791047607263\\
-1.85011538205858	-1.41776640448773\\
-1.53612890561482	-0.87652508965204\\
-1.04857221250265	-0.146003337612458\\
-0.417760327509739	0.682338124572095\\
0.221622946878704	1.42166635464838\\
0.74348915317282	1.95384390101691\\
1.11911647718661	2.2914069222994\\
1.37832089111808	2.49618674388445\\
1.55811461255222	2.62052375633131\\
1.68596713518099	2.69740311002973\\
1.77970009060015	2.74593105205812\\
1.85052730166635	2.77705905146031\\
1.90555619448917	2.79717432215525\\
1.94938858132061	2.81010366980761\\
1.98508170672757	2.8182068466805\\
2.0147205208524	2.8229794234728\\
2.03976279585362	2.82539290711205\\
2.06125160838935	2.82609185066612\\
};
\addplot [color=mycolor1, forget plot]
  table[row sep=crcr]{%
2.94343378702665	2.98150695436743\\
3.08885591882588	2.97703004882834\\
3.21054796459534	2.96551824604259\\
3.31341894940247	2.94914292861528\\
3.40124887065957	2.92937341442907\\
3.47695420897439	2.90719878294888\\
3.54279594379573	2.88327856753722\\
3.60053768793696	2.85804436184619\\
3.65156420497661	2.83176816902748\\
3.69696972781973	2.8046083443135\\
3.73762369200408	2.77664040394839\\
3.77421969304537	2.74787752274462\\
3.80731196955673	2.7182838987095\\
3.83734254235399	2.68778306693339\\
3.86466125736386	2.65626250988181\\
3.88954032096338	2.62357541000337\\
3.91218441887612	2.58954003765049\\
3.93273712261889	2.55393700506867\\
3.95128396880596	2.51650440568945\\
3.96785231077364	2.47693066782098\\
3.98240775591448	2.43484475941707\\
3.99484668103311	2.38980316445684\\
4.0049839205663	2.34127278980445\\
4.0125341945373	2.28860863059909\\
4.01708510995117	2.23102459677709\\
4.01805852587469	2.16755535943817\\
4.01465557245049	2.09700640285185\\
4.00577846405327	2.01788869870269\\
3.98991922134651	1.92833370145539\\
3.96500134333717	1.8259841226427\\
3.92815550330806	1.7078572334151\\
3.87540572228343	1.57018268928028\\
3.80124242044465	1.40823126569638\\
3.69807474295245	1.21618474376124\\
3.55561366713805	0.987168345099138\\
3.36039355718265	0.713697986754594\\
3.0959705528776	0.388984026151238\\
2.74485482730161	0.00965943621581311\\
2.29357605845332	-0.419860896077798\\
1.74123809312363	-0.88273264558884\\
1.10809223296134	-1.3486339523367\\
0.436257105221601	-1.78050448022638\\
-0.222411249521406	-2.14740128269557\\
-0.824456983247826	-2.43456471604976\\
-1.34594972456221	-2.64394799960773\\
-1.78189217938341	-2.78766825269008\\
-2.13923520644798	-2.8808316329203\\
-2.42990512098094	-2.93720386055972\\
-2.66642312116667	-2.96765779361241\\
-2.85990342575475	-2.98016035853\\
-3.01947047104845	-2.980318962509\\
-3.15233141227306	-2.97199814102729\\
-3.26407678003267	-2.95783324946641\\
-3.35901422996584	-2.93961142538769\\
-3.44046250987799	-2.91853917939846\\
-3.51098770154975	-2.89542548863729\\
-3.57258504950268	-2.87080559978002\\
-3.62681601558868	-2.84502437405132\\
-3.67491062596942	-2.81829233073741\\
-3.71784366349641	-2.79072328945494\\
-3.75639138919313	-2.76235953884773\\
-3.7911738055063	-2.73318844920671\\
-3.82268613730342	-2.70315310412262\\
-3.85132218756576	-2.67215862982249\\
-3.87739146216411	-2.64007529563112\\
-3.90113138683763	-2.60673904059147\\
-3.9227155029489	-2.57194977960228\\
-3.94225818044096	-2.53546760974441\\
-3.95981608835485	-2.49700683985336\\
-3.97538638144289	-2.45622757786813\\
-3.98890126247118	-2.41272440886056\\
-4.00021822633899	-2.36601146055877\\
-4.00910483726414	-2.31550286003529\\
-4.01521627070026	-2.26048721032088\\
-4.01806297880681	-2.20009423360216\\
-4.01696458777847	-2.1332511189094\\
-4.01098433732702	-2.05862538227507\\
-3.99883581308979	-1.97455027346157\\
-3.97875018437911	-1.87892820292819\\
-3.94828757872079	-1.76910799411692\\
-3.90407115412907	-1.64173460153032\\
-3.84141925200013	-1.49257885102639\\
-3.75385682688332	-1.31637716254411\\
-3.63252009480728	-1.10676098963577\\
-3.46556723356335	-0.856454205776585\\
-3.23794182258222	-0.55808102507739\\
-2.93227133418999	-0.206114291382088\\
-2.53220302067924	0.199545148698177\\
-2.02933798624782	0.648742309963167\\
-1.43256471133232	1.11760073735716\\
-0.773780891425569	1.57105760630653\\
-0.102173643872332	1.97343260536243\\
0.532508754172318	2.30116044932552\\
1.09595877732084	2.54836533236184\\
1.57436796768815	2.72307496482937\\
1.96969470722219	2.83964895043039\\
2.29210292472253	2.9128645884254\\
2.55419399350715	2.95510989188397\\
2.76792508339582	2.97575476480715\\
2.94343378702665	2.98150695436743\\
};
\addplot [color=mycolor2, forget plot]
  table[row sep=crcr]{%
1.94396230060006	2.81077700198304\\
1.95985108879152	2.81028302218281\\
1.97394601517608	2.80894524160945\\
1.98658836727855	2.80692872305329\\
1.99804290145849	2.80434666455556\\
2.00851794541265	2.80127490447696\\
2.01817955450266	2.79776146990237\\
2.02716163337153	2.7938328465329\\
2.0355732636903	2.7894980045526\\
2.04350405557244	2.78475081876226\\
2.05102806631827	2.77957126772946\\
2.0582066475911	2.77392562507895\\
2.0650904552572	2.76776572935734\\
2.07172076202316	2.761027312666\\
2.07813013562979	2.75362726375721\\
2.08434247168474	2.74545958092923\\
2.09037228714733	2.73638961317069\\
2.09622307156667	2.726245965862\\
2.10188433396748	2.71480911571078\\
2.10732673361643	2.70179526631605\\
2.11249427288296	2.68683315867038\\
2.11729183363615	2.66943021738284\\
2.12156511946277	2.64892218404249\\
2.12506787274033	2.62439657120363\\
2.12740717858999	2.59457357186253\\
2.12794994908125	2.55761602734519\\
2.12565858578613	2.51081799353142\\
2.11879356735678	2.4500804507019\\
2.10435914717095	2.36900692454446\\
2.077044104215	2.25731897225445\\
2.02717662267011	2.09810365785094\\
1.93689704487503	1.86338398381203\\
1.77403815419916	1.50898111967002\\
1.48781236794104	0.977591282821741\\
1.02797794570424	0.239102328779913\\
0.417065751156591	-0.618110712345637\\
-0.207022888965703	-1.38788049639511\\
-0.71136150653178	-1.93605025310214\\
-1.0682992731387	-2.27797487868556\\
-1.31082101086512	-2.48243692785583\\
-1.47709853453944	-2.60542161615692\\
-1.59438921386909	-2.68112630281884\\
-1.67991444787889	-2.72890911966843\\
-1.74431053192608	-2.75968640544852\\
-1.794230072076	-2.77975270835501\\
-1.83393929055117	-2.79284972826202\\
-1.8662525883997	-2.80127251714169\\
-1.89307927889216	-2.80647073286878\\
-1.91574910331315	-2.80938432065387\\
-1.93521073419012	-2.81063655925458\\
-1.95215596271729	-2.81064836758024\\
-1.96709937311284	-2.80970786100419\\
-1.98043069252134	-2.80801373357572\\
-1.99244995795617	-2.80570290192155\\
-2.00339162522548	-2.8028684301293\\
-2.01344140390623	-2.79957129251754\\
-2.02274820662335	-2.79584812070656\\
-2.03143274948098	-2.79171625123768\\
-2.0395938092478	-2.7871768871126\\
-2.04731280353105	-2.78221687147284\\
-2.05465713752028	-2.77680936482239\\
-2.06168260965477	-2.77091357168549\\
-2.06843506041373	-2.76447354870151\\
-2.0749513644328	-2.75741602296175\\
-2.08125979218648	-2.74964703925448\\
-2.08737969086333	-2.74104711921343\\
-2.09332034049265	-2.73146443019397\\
-2.09907871112645	-2.72070519171487\\
-2.10463564888983	-2.70852013631634\\
-2.10994970029888	-2.69458519617113\\
-2.11494725179667	-2.6784735460981\\
-2.11950674301536	-2.65961441457825\\
-2.12343308270449	-2.63723116573116\\
-2.12641542374375	-2.61024611324654\\
-2.12795587734824	-2.57713057502082\\
-2.12724598828813	-2.53566242125648\\
-2.12294648080199	-2.48252332081042\\
-2.112782671042	-2.41261201185111\\
-2.09278017382524	-2.31784838961673\\
-2.05579255081375	-2.18507740519626\\
-1.98867972953625	-1.99251610396963\\
-1.86729573432262	-1.70460212758402\\
-1.65006553490393	-1.26884645769957\\
-1.2810391545868	-0.632579777231207\\
-0.734829643605394	0.186538733238462\\
-0.0961807704141871	1.02629442439069\\
0.478237086983504	1.69129758873056\\
0.906738389298453	2.12857176341927\\
1.20126694458971	2.39332900349002\\
1.4015110305165	2.55153550099923\\
1.54059805830277	2.64771501683872\\
1.64033955717226	2.70768355489814\\
1.7142672731182	2.74595219806201\\
1.77077068523864	2.77078098484228\\
1.81515884049049	2.78700379272891\\
1.85088449176798	2.79753982960281\\
1.88025785289774	2.80420676365557\\
1.90486730397671	2.8081683476314\\
1.92583278470842	2.81018803520323\\
1.94396230060006	2.81077700198304\\
};
\addplot [color=mycolor1, forget plot]
  table[row sep=crcr]{%
3.8069742121015	3.32601021063417\\
3.94310767731051	3.32183896413045\\
4.05420275841036	3.31134312526451\\
4.14616621219568	3.29671363519818\\
4.22331079729383	3.27935618385384\\
4.28882087296	3.26017297175802\\
4.3450761920556	3.23973943178931\\
4.39387674716691	3.2184157825262\\
4.43659966989989	3.19641802187611\\
4.47430959606783	3.17386344125139\\
4.50783701640624	3.15079991969402\\
4.53783440608565	3.12722471731482\\
4.56481674587108	3.10309632370882\\
4.58919091662597	3.07834157938611\\
4.61127701392177	3.05285944670632\\
4.63132365137258	3.02652226330007\\
4.64951864016164	2.99917494528458\\
4.66599594085826	2.9706323448972\\
4.68083940812854	2.94067475863317\\
4.69408353516085	2.90904139284285\\
4.70571110826766	2.87542139564259\\
4.71564736063074	2.83944182921608\\
4.7237498186394	2.80065165304653\\
4.7297924996768	2.75850037441778\\
4.73344235137475	2.71230943916012\\
4.73422467258974	2.66123359920901\\
4.73147249347479	2.60420828388831\\
4.72425214485595	2.53987725159715\\
4.71125292455703	2.46649229210029\\
4.69062195021153	2.38177326153695\\
4.65971462450232	2.28271219372795\\
4.61471493250551	2.16530026432807\\
4.55005689082875	2.0241538357331\\
4.45755195938648	1.85202485053174\\
4.32511660075767	1.63922707388558\\
4.13506999383621	1.3731567878721\\
3.86233660637841	1.03846759065608\\
3.47398655764083	0.619251405989274\\
2.93385906115449	0.105608985940811\\
2.21819414526292	-0.493647269701428\\
1.34285086028235	-1.13739433911321\\
0.381733763959551	-1.75511273586638\\
-0.555945978397205	-2.27758418022185\\
-1.3801338200065	-2.67098549388462\\
-2.05273030094319	-2.94130697961128\\
-2.57903978312717	-3.11502187892367\\
-2.98402766689781	-3.22074360659178\\
-3.29550436360794	-3.28123933156228\\
-3.53714176607316	-3.31240885437513\\
-3.72707871394694	-3.32471889480433\\
-3.87861866314147	-3.32489355374631\\
-4.00137183341736	-3.31722202297675\\
-4.10227530238411	-3.30444276587309\\
-4.18636976282832	-3.28831018307363\\
-4.2573555868939	-3.26995072743412\\
-4.31798139885556	-3.25008592041586\\
-4.37031381264431	-3.22917262122432\\
-4.41592509522117	-3.2074919276244\\
-4.4560246114975	-3.18520597582556\\
-4.4915517058208	-3.16239444919179\\
-4.52324194637074	-3.13907806679447\\
-4.55167477689722	-3.11523355816124\\
-4.57730801706905	-3.09080293236409\\
-4.60050290583896	-3.06569879140361\\
-4.62154219991242	-3.03980676424577\\
-4.64064302591274	-3.01298569489415\\
-4.65796560952048	-2.9850659110852\\
-4.67361857917333	-2.95584566959024\\
-4.68766120322395	-2.92508567906909\\
-4.7001026196294	-2.89250141118904\\
-4.71089781383341	-2.85775269790244\\
-4.71993974885331	-2.82042984810432\\
-4.72704659614644	-2.7800351641179\\
-4.73194237861246	-2.73595824833843\\
-4.73422840048079	-2.68744279296103\\
-4.73334141867878	-2.63354154044063\\
-4.72849231313563	-2.57305464653371\\
-4.71857557025886	-2.50444457872921\\
-4.70203446371177	-2.42571771042901\\
-4.67665827273233	-2.33425874303738\\
-4.63927465991455	-2.22659917351108\\
-4.58528080037595	-2.09809665954561\\
-4.50793119077666	-1.94250354931282\\
-4.39727709648894	-1.75142557911221\\
-4.23867072522465	-1.51375682493256\\
-4.01092918938735	-1.21541927095236\\
-3.68491262374884	-0.840304133414801\\
-3.22495230318177	-0.374306646763107\\
-2.59825348631594	0.185029069108261\\
-1.79721195477659	0.813917342883822\\
-0.866410835864464	1.45435571377579\\
0.0967248201703723	2.03142822392588\\
0.98588338027817	2.4907984854268\\
1.73578374400764	2.82009224229463\\
2.33281856586493	3.03836073769983\\
2.79490690793268	3.17478938223685\\
3.14984561154891	3.25550255453373\\
3.42379321322665	3.29972949659059\\
3.6376413178944	3.32043092197198\\
3.8069742121015	3.32601021063417\\
};
\addplot [color=mycolor2, forget plot]
  table[row sep=crcr]{%
1.74677956590797	2.76321383348015\\
1.76055079162803	2.76278489084878\\
1.77289752304441	2.76161235880782\\
1.78408307816614	2.75982761735725\\
1.79431432821077	2.75752078244666\\
1.80375605241051	2.75475155614031\\
1.81254117813544	2.75155640064587\\
1.82077818881318	2.74795322267997\\
1.82855654270017	2.74394430324449\\
1.83595066345758	2.73951792268229\\
1.84302287744233	2.73464894166046\\
1.84982554550421	2.72929846348412\\
1.85640254543905	2.72341259492248\\
1.86279018923641	2.71692022200251\\
1.86901759477247	2.70972960689701\\
1.87510646340486	2.70172347333093\\
1.881070129905	2.69275205592092\\
1.88691163070849	2.68262330610839\\
1.89262035073271	2.67108901358273\\
1.89816650666999	2.6578249161294\\
1.90349221348034	2.64240175754852\\
1.90849698956012	2.62424240164907\\
1.91301395760529	2.60255695453783\\
1.91677004982509	2.57624233466723\\
1.91931793079299	2.54372286171832\\
1.91991642535571	2.50269036399665\\
1.91731432610301	2.44966862491135\\
1.90934758813476	2.37926393771651\\
1.8921676553373	2.28284881141417\\
1.858736269888	2.14624189106937\\
1.79592121051184	1.94579193806572\\
1.67939844049359	1.64293488187096\\
1.46775176340378	1.18231590857994\\
1.10905206980362	0.515816473456423\\
0.592268408768907	-0.315590085670382\\
0.0149776246806529	-1.12735122067658\\
-0.482966401320024	-1.74247563274645\\
-0.846247138018683	-2.13758925789272\\
-1.09479727169634	-2.37570531565814\\
-1.26460364302433	-2.51883787815067\\
-1.38356177823104	-2.60679926878341\\
-1.46967907361087	-2.66236645582977\\
-1.53410063802864	-2.69834771482707\\
-1.58376484002151	-2.7220768972298\\
-1.62309106660028	-2.73788014820263\\
-1.65497417750504	-2.74839259095045\\
-1.68136499101193	-2.75526922566731\\
-1.70361398362453	-2.75957864139342\\
-1.72267950622259	-2.76202763288094\\
-1.73925718341232	-2.76309323105423\\
-1.75386240879692	-2.76310253391312\\
-1.76688423497511	-2.76228223573638\\
-1.77862138744392	-2.76079006369596\\
-1.78930684247465	-2.75873511590543\\
-1.79912492476823	-2.75619120953072\\
-1.80822340856862	-2.75320570703901\\
-1.8167222134742	-2.749805330294\\
-1.82471973285041	-2.74599989645953\\
-1.83229748174183	-2.74178455266795\\
-1.83952352294432	-2.73714085580965\\
-1.84645497692253	-2.73203688565233\\
-1.85313981422446	-2.72642646053219\\
-1.85961804889195	-2.72024742257549\\
-1.86592238451269	-2.71341885602795\\
-1.87207829967102	-2.70583698036699\\
-1.8781034848286	-2.6973692982315\\
-1.88400644278413	-2.68784634661996\\
-1.88978391652406	-2.67705005108697\\
-1.89541657260374	-2.6646971390544\\
-1.90086197669035	-2.65041519717875\\
-1.90604322535744	-2.63370752631203\\
-1.91083040880429	-2.61390053769103\\
-1.91500991591535	-2.5900632758779\\
-1.91823254593586	-2.5608813002931\\
-1.91992359964181	-2.52445383804068\\
-1.91912269779814	-2.47795849089767\\
-1.91418977930217	-2.41708166155436\\
-1.9022493147976	-2.33502720544078\\
-1.8781137711729	-2.22076643402019\\
-1.83218329233983	-2.05599163654765\\
-1.74651403576271	-1.81029310733755\\
-1.58876039137854	-1.43616386293894\\
-1.30945691793092	-0.875652336430001\\
-0.867067023985944	-0.111895638767759\\
-0.301474533825516	0.73804118811955\\
0.249914151277534	1.46445740076682\\
0.681128085091329	1.96421367923717\\
0.982532897783201	2.27188940194\\
1.18754066861232	2.45616064097104\\
1.32911446155926	2.56798673754701\\
1.42990435001473	2.63766308133445\\
1.50409356324664	2.68225497396183\\
1.56045640237134	2.71142230728431\\
1.60451183081133	2.73077531966039\\
1.63982383437755	2.7436770095946\\
1.66876057492792	2.75220807282934\\
1.69293993199161	2.75769405610699\\
1.71349618514047	2.76100161394126\\
1.73124362836514	2.7627100897579\\
1.74677956590797	2.76321383348015\\
};
\addplot [color=mycolor1, forget plot]
  table[row sep=crcr]{%
5.23359968598068	4.06228569073568\\
5.36158749680671	4.05838065589803\\
5.46374098524715	4.04874039012966\\
5.54679183289556	4.03553600666112\\
5.61543665918267	4.02009609472798\\
5.67301780996828	4.00323838663666\\
5.7219588007418	3.98546437097936\\
5.76404748331935	3.96707561940243\\
5.80062350346106	3.94824452309343\\
5.83270461781342	3.92905791881896\\
5.86107325043917	3.90954423800289\\
5.88633671013261	3.8896904063681\\
5.90896962530503	3.86945219768965\\
5.92934413252931	3.84876026701216\\
5.94775144446764	3.82752320063858\\
5.96441718888765	3.80562836909689\\
5.9795120919587	3.78294101085854\\
5.99315901563786	3.75930172366362\\
6.00543694852082	3.73452234366319\\
6.0163822227875	3.70838001337468\\
6.0259869344175	3.6806090475804\\
6.0341942325797	3.65088997141281\\
6.0408897648076	3.61883478967854\\
6.04588804778265	3.58396709869379\\
6.04891177486158	3.54569499249963\\
6.04956090262756	3.50327371982051\\
6.0472664971646	3.45575351572038\\
6.04122127904133	3.40190563377832\\
6.03027372644046	3.34011579792157\\
6.0127639465138	3.26822818248925\\
5.98626454625468	3.18331316289279\\
5.94716346988753	3.08131622318111\\
5.8899795581516	2.95652060026942\\
5.80622154889659	2.8007204513202\\
5.68247087909964	2.60196159196587\\
5.49719491920497	2.34271072987884\\
5.21573614177751	1.99755522791973\\
4.7837443466677	1.53164430053319\\
4.12360982185974	0.904557796144839\\
3.15088009759361	0.0909888040006891\\
1.84002957784789	-0.872152820396208\\
0.319106348884118	-1.84937470406752\\
-1.15275137205982	-2.66989860688977\\
-2.36798134970148	-3.25059532097248\\
-3.27589286565043	-3.61600344575553\\
-3.92595046499675	-3.83088160761399\\
-4.38901224547138	-3.95194490186727\\
-4.72349231149159	-4.01701000298456\\
-4.97038901434058	-4.04891587652087\\
-5.15700419762603	-4.0610447774668\\
-5.3013489603052	-4.06123206442097\\
-5.41541488601347	-4.0541167851449\\
-5.507322473188	-4.04248567917992\\
-5.58268038608153	-4.02803518021852\\
-5.6454410834196	-4.0118073560535\\
-5.69844424824095	-3.99444337905209\\
-5.74376716383899	-3.97633364667063\\
-5.7829545913717	-3.95770831182488\\
-5.81717231615215	-3.93869275066441\\
-5.84731150137254	-3.91934195400816\\
-5.87406075604608	-3.89966196286193\\
-5.89795661305006	-3.87962314242784\\
-5.91941928808485	-3.85916816308982\\
-5.93877819398976	-3.83821641512527\\
-5.95629015390419	-3.81666588762123\\
-5.97215225555894	-3.79439310081417\\
-5.98651061370172	-3.77125138458311\\
-5.9994658310114	-3.74706757739965\\
-6.01107558656411	-3.72163703610851\\
-6.02135447578981	-3.69471666499943\\
-6.03027092845338	-3.6660154633198\\
-6.03774069318905	-3.63518182071692\\
-6.04361593943317	-3.60178641648343\\
-6.04766840603931	-3.56529903705841\\
-6.04956408873287	-3.5250568181969\\
-6.0488254886897	-3.48022018583099\\
-6.04477507184966	-3.42971085631841\\
-6.03644966793568	-3.37212324025763\\
-6.02246892500576	-3.30559577658956\\
-6.00082957465836	-3.22762096104647\\
-5.96857745114017	-3.13476030577434\\
-5.9212743543892	-3.02221052364306\\
-5.85211579503254	-2.88313695112032\\
-5.75045222363542	-2.70765027157872\\
-5.5993092530868	-2.48127354568202\\
-5.37134478901386	-2.18282749426356\\
-5.02288978461698	-1.78221201710447\\
-4.48777477941243	-1.24061089878179\\
-3.68049222212999	-0.520924958516225\\
-2.53409425373193	0.378121776297124\\
-1.0909212452937	1.37045141300698\\
0.439385935937216	2.287447501653\\
1.79910038993385	2.99051600874801\\
2.85847914257785	3.4563076483258\\
3.62862072756977	3.73827308049239\\
4.17677865185423	3.90035497170982\\
4.56938522686428	3.98976950678617\\
4.85591727395001	4.03610457934639\\
5.06993291680334	4.05686649742183\\
5.23359968598067	4.06228569073568\\
};
\addplot [color=mycolor2, forget plot]
  table[row sep=crcr]{%
1.33301670838737	2.61204888329464\\
1.34644320755614	2.61162908613279\\
1.35874493797112	2.61045944960314\\
1.37011960182753	2.60864331060449\\
1.38072728200983	2.60625048816175\\
1.39069931887731	2.60332468400919\\
1.40014485071414	2.59988834624712\\
1.40915568633019	2.59594569807516\\
1.41780996332873	2.59148436178709\\
1.42617490018531	2.58647582139012\\
1.43430884830736	2.58087482761002\\
1.44226277441712	2.57461773110508\\
1.45008124159532	2.56761961353\\
1.45780289896554	2.55976995270321\\
1.46546042493265	2.55092638520112\\
1.47307978398209	2.54090588631409\\
1.48067853267487	2.52947232623802\\
1.48826271427597	2.5163188064435\\
1.49582155700891	2.50104230408664\\
1.50331863627902	2.4831067372466\\
1.51067718701888	2.46178822934725\\
1.51775549857982	2.43609242260905\\
1.52430509226474	2.40462695806892\\
1.52989828989807	2.36540053415675\\
1.53380007949194	2.31549947077581\\
1.53473644177133	2.25055725852743\\
1.53046737484438	2.16387448001793\\
1.51699215125942	2.0449675132812\\
1.48709098642182	1.87729236425733\\
1.42784761001922	1.63526643156743\\
1.31745996182202	1.28282952524043\\
1.12526661095823	0.78250845831663\\
0.827178623124766	0.131829563845013\\
0.442626569180643	-0.585643728223974\\
0.045598959997802	-1.22687943553216\\
-0.292887299848114	-1.7039329392721\\
-0.550174643240503	-2.02194789246879\\
-0.737231321118399	-2.22531219891865\\
-0.872923550490124	-2.35520786936517\\
-0.973073002559847	-2.43955468949996\\
-1.04880622763155	-2.49550923857833\\
-1.10755169111278	-2.53338648794925\\
-1.15423717537672	-2.55944352147708\\
-1.19217088269001	-2.57755596109554\\
-1.22361657877387	-2.59018418731236\\
-1.25015675706226	-2.59892911557715\\
-1.27292158124762	-2.60485662231768\\
-1.29273488333345	-2.60869099274247\\
-1.31020886281663	-2.61093297797532\\
-1.32580668819217	-2.61193351417705\\
-1.33988468529741	-2.61194075318417\\
-1.35272130697319	-2.61113064541139\\
-1.36453738377374	-2.60962714494715\\
-1.37551051859768	-2.6075157086508\\
-1.38578547830666	-2.60485235082933\\
-1.39548180163462	-2.60166966586222\\
-1.4046994376114	-2.59798070619169\\
-1.41352296466125	-2.59378126805965\\
-1.42202476441441	-2.58905091388597\\
-1.43026740319638	-2.58375290074115\\
-1.43830538703167	-2.57783305831397\\
-1.44618638843291	-2.5712175450445\\
-1.45395198429384	-2.56380928878409\\
-1.46163788389146	-2.55548276837135\\
-1.46927355295561	-2.54607658901317\\
-1.47688103818478	-2.53538300946849\\
-1.48447264150035	-2.52313313308341\\
-1.49204684223526	-2.5089757795822\\
-1.49958144278572	-2.49244694411555\\
-1.50702218051136	-2.47292493719459\\
-1.51426374550244	-2.44956327808078\\
-1.5211177698076	-2.42118828226447\\
-1.52725792808844	-2.38613941931682\\
-1.53212386393177	-2.34201503276749\\
-1.53474935346189	-2.28525898061997\\
-1.5334484297272	-2.21047785433787\\
-1.52523299326183	-2.10930790940293\\
-1.50473158650961	-1.96857741655307\\
-1.4622569408778	-1.7675997104122\\
-1.38081867461681	-1.47540980909169\\
-1.23360993864157	-1.0527993701081\\
-0.989734196601563	-0.473118481173226\\
-0.641946753959142	0.227373334762711\\
-0.240284393996546	0.923745133831825\\
0.133534335238773	1.48729623857675\\
0.431414965234564	1.88026635099027\\
0.651290951575217	2.13508720401846\\
0.810394867871117	2.29739657590908\\
0.926643646894778	2.40180047104697\\
1.01345598548921	2.47031496538093\\
1.07994787151582	2.51624539095708\\
1.13216521285489	2.54760839904196\\
1.17413710720869	2.56931384510598\\
1.2085930466781	2.58444002420209\\
1.23742064298458	2.59496564299668\\
1.26195372251315	2.60219342990986\\
1.28315480571647	2.60699994060218\\
1.30173242034655	2.60998623944193\\
1.31821791114351	2.61157092933178\\
1.33301670838737	2.61204888329464\\
};
\addplot [color=mycolor1, forget plot]
  table[row sep=crcr]{%
7.0392772787345	5.11903163413247\\
7.16626142891188	5.11516704726827\\
7.26629530947662	5.10573296770008\\
7.34678121583159	5.0929404252796\\
7.41275009540204	5.07810515705988\\
7.46770818903169	5.06201734166849\\
7.51415523785521	5.04515049114053\\
7.55390997795022	5.0277825360959\\
7.58831969447633	5.01006760274933\\
7.6183982542733	4.99207933311409\\
7.64491893503354	4.97383734027816\\
7.66847801198178	4.95532339772637\\
7.68953899601874	4.93649119519741\\
7.70846377852075	4.91727191806879\\
7.72553470301668	4.89757698320867\\
7.7409701764394	4.87729870260255\\
7.75493552033362	4.85630928720692\\
7.76755014829301	4.83445835564298\\
7.77889171864666	4.81156891940326\\
7.78899757154672	4.78743163993666\\
7.79786345915379	4.76179696211644\\
7.80543926619556	4.73436449125311\\
7.81162104112765	4.70476865681686\\
7.81623814294102	4.67255923743224\\
7.81903354520929	4.63717461823094\\
7.81963414633169	4.59790456633448\\
7.81750600161046	4.5538375966568\\
7.8118861644611	4.50378523524493\\
7.80167728653956	4.44617092905127\\
7.78528137648615	4.37886367333006\\
7.76033149490167	4.29892322355866\\
7.72324748141428	4.2022006198842\\
7.6684797357689	4.08269666016919\\
7.58718501700106	3.93150794554762\\
7.46484515066429	3.73506430634936\\
7.27690174669529	3.47216907133276\\
6.98076794077052	3.10918441338855\\
6.50205235605686	2.59321050777568\\
5.71605754331805	1.84720902799201\\
4.44720912225841	0.787085083320875\\
2.56880033484852	-0.591757799732749\\
0.257878165560372	-2.07608138743818\\
-1.95873600611616	-3.31246837059941\\
-3.67246583887669	-4.13229019081978\\
-4.84890959120493	-4.60637352971057\\
-5.62924820005475	-4.86462617628409\\
-6.152693041282	-5.00162775078038\\
-6.51414732826121	-5.07201566016901\\
-6.77218075665923	-5.10540017020252\\
-6.96238590458728	-5.11778413520988\\
-7.10672674008617	-5.11798405531009\\
-7.21911589964707	-5.1109811032576\\
-7.3086237166577	-5.09965865272981\\
-7.38133267142339	-5.08571943340421\\
-7.4414301898953	-5.07018253002399\\
-7.4918688504861	-5.05366035915892\\
-7.53477577997466	-5.03651722848336\\
-7.57171304439667	-5.01896232518367\\
-7.60384725922387	-5.00110537118762\\
-7.63206250849633	-4.98299044337303\\
-7.65703701014054	-4.96461668067336\\
-7.67929606109214	-4.9459508960254\\
-7.69924911114715	-4.92693503066006\\
-7.71721597128642	-4.90749018638949\\
-7.73344539439188	-4.88751825480663\\
-7.74812813760957	-4.86690171678298\\
-7.76140587128517	-4.84550189100654\\
-7.77337678567625	-4.82315569550259\\
-7.78409836629924	-4.79967080559342\\
-7.79358749509538	-4.77481891194556\\
-7.80181773494674	-4.74832657256971\\
-7.80871331842344	-4.7198628778369\\
-7.81413892636756	-4.68902276011232\\
-7.81788371978822	-4.65530420766401\\
-7.8196371394281	-4.61807677139272\\
-7.81895247498845	-4.57653739295516\\
-7.81519171634902	-4.52964741199717\\
-7.80744098485871	-4.47604106957693\\
-7.79437851885377	-4.41388992474236\\
-7.77406412241355	-4.34069755793578\\
-7.74359506366882	-4.25298149230635\\
-7.69852849172415	-4.14576846289678\\
-7.63188425699374	-4.01177432194864\\
-7.53237456803583	-3.84004333817358\\
-7.38118488394934	-3.61366203084588\\
-7.14605406356131	-3.3059544239265\\
-6.77063751144112	-2.87457534687001\\
-6.15761509575811	-2.25458614919317\\
-5.15347938851971	-1.36027702764371\\
-3.58354295228443	-0.130374145500047\\
-1.43877767222122	1.34333463692235\\
0.89679834516737	2.74303223250638\\
2.88793902590139	3.77351412635487\\
4.3202076583399	4.40403635626887\\
5.27891129380902	4.75547535999178\\
5.91612850328448	4.94410764971809\\
6.34932856187132	5.04287378380005\\
6.65346014382276	5.09210914897521\\
6.87414979638642	5.11354760941571\\
7.0392772787345	5.11903163413247\\
};
\addplot [color=mycolor2, forget plot]
  table[row sep=crcr]{%
0.340159402357135	2.10600086215548\\
0.3628465262389	2.10528376142592\\
0.384983538536663	2.10317175270087\\
0.406716283993941	2.09969488583627\\
0.428180678011108	2.09484635528048\\
0.449505533367152	2.08858296968783\\
0.470815018388807	2.08082372707867\\
0.492230799808478	2.07144640903074\\
0.513873889130809	2.06028194820058\\
0.535866166731864	2.04710614126222\\
0.558331494769348	2.03162805630885\\
0.581396234946618	2.01347419948199\\
0.605188838922992	1.99216713731932\\
0.629837943593957	1.96709679723808\\
0.655468026444841	1.93748207838122\\
0.682191073509881	1.90231972724602\\
0.710091758583656	1.86031679879876\\
0.739202155707758	1.80980281214685\\
0.769459818573213	1.74861884556833\\
0.80064008913943	1.67398536129912\\
0.832250213425196	1.58236272978444\\
0.863371344598456	1.46934596208096\\
0.892440839855638	1.32968951028285\\
0.916994864544829	1.1576478771209\\
0.933461665095238	0.947922142732608\\
0.93722024091161	0.697507134450695\\
0.923254154813351	0.408373538265392\\
0.887613191091914	0.0899584839014675\\
0.829292619497089	-0.240624003826465\\
0.751350217007357	-0.562241095216022\\
0.660192239960307	-0.855958820248454\\
0.563363666195249	-1.10998962709869\\
0.46734720222843	-1.32075071805171\\
0.376485188537052	-1.49074974756309\\
0.293014529295588	-1.62559329921078\\
0.21763453804227	-1.73168296113876\\
0.150124373920323	-1.8149523813449\\
0.0898064836170812	-1.88038152662915\\
0.0358346873675908	-1.93193627925224\\
-0.0126491668834016	-1.97268813284077\\
-0.0564408523233183	-2.00498324783389\\
-0.0962456055018899	-2.03060375647545\\
-0.132673687998756	-2.05090219860937\\
-0.166246925494325	-2.06690671537743\\
-0.197409467037838	-2.07940074765622\\
-0.226539405413789	-2.08898241554729\\
-0.253959729761681	-2.09610838452086\\
-0.279948024299138	-2.10112612642857\\
-0.304744788313741	-2.10429756290364\\
-0.328560461293696	-2.10581629915087\\
-0.351581317507659	-2.10582004535157\\
-0.373974412350106	-2.10439936011743\\
-0.395891752286059	-2.10160350283493\\
-0.417473838109259	-2.09744391778838\\
-0.438852705211635	-2.09189566598317\\
-0.460154557460578	-2.08489694885098\\
-0.481502063043633	-2.07634671334331\\
-0.503016349197679	-2.06610017415172\\
-0.524818694049379	-2.05396192018484\\
-0.547031861225919	-2.03967607218156\\
-0.569780945882671	-2.02291270740463\\
-0.593193482477847	-2.00324944399919\\
-0.617398378026468	-1.98014665881395\\
-0.64252293665151	-1.95291428015224\\
-0.668686764263143	-1.92066745506512\\
-0.695990582565688	-1.88226770787954\\
-0.724496790253406	-1.83624571219609\\
-0.754196798300473	-1.78070209222836\\
-0.78495757674282	-1.71318514683432\\
-0.816436598886198	-1.63055203632806\\
-0.84795154121966	-1.5288386027832\\
-0.878292379011432	-1.40320237913186\\
-0.905478215418113	-1.24807521708215\\
-0.926507279688136	-1.0577656117023\\
-0.937246943050733	-0.827828665751056\\
-0.932746846728689	-0.557384782911339\\
-0.90829524435065	-0.251913355457592\\
-0.861193639531173	0.0751070859219576\\
-0.792430440350056	0.40386223059073\\
-0.7069485204366	0.713524546104237\\
-0.612037533875673	0.988336489411677\\
-0.514928609605447	1.22072882229166\\
-0.421088281198913	1.41053544394599\\
-0.333753957128362	1.56216496577172\\
-0.254311790829908	1.68184199557282\\
-0.182931519533185	1.77583455664028\\
-0.119117675453817	1.84962629757004\\
-0.062081846382775	1.90768220695569\\
-0.0109584532815586	1.95350093424583\\
0.035085202877312	1.98976990269858\\
0.0768012723721879	2.0185345950784\\
0.114847160032888	2.04134741988353\\
0.149787719570286	2.05938727992309\\
0.182104487988261	2.07355139534386\\
0.212207153711672	2.08452419672195\\
0.240444960310469	2.09282839920086\\
0.267117068662005	2.09886263746276\\
0.292481564777994	2.1029290972206\\
0.316763114824385	2.10525371981033\\
0.340159402357135	2.10600086215548\\
};
\addplot [color=mycolor1, forget plot]
  table[row sep=crcr]{%
6.61613611895751	4.8641942745862\\
6.74267149711494	4.86034166598725\\
6.84257530198247	4.85091880632069\\
6.92309873222533	4.83811961587769\\
6.98919202422371	4.82325590380537\\
7.0443173727696	4.80711879880973\\
7.09095015006891	4.79018426152558\\
7.13089554947986	4.77273282959086\\
7.16549338683827	4.75492090558833\\
7.19575351193484	4.73682393903221\\
7.22244715811728	4.71846288110962\\
7.24616967660651	4.69982042039387\\
7.26738427455025	4.68085079423991\\
7.28645286113455	4.66148541809789\\
7.30365793713208	4.64163566081575\\
7.31921809172996	4.62119353646822\\
7.33329877818253	4.60003072596646\\
7.34601943704234	4.5779960943986\\
7.357457604939	4.55491167731867\\
7.36765031073442	4.53056693316483\\
7.37659276261626	4.5047108687695\\
7.38423401969428	4.47704140898084\\
7.39046896592001	4.44719106021156\\
7.39512539095489	4.41470745448598\\
7.39794422323946	4.37902666722518\\
7.39854977688489	4.33943613597095\\
7.39640496071006	4.29502233073107\\
7.39074321390929	4.24459563420822\\
7.38046349486532	4.18658047707388\\
7.36396512538055	4.11885139407387\\
7.33888219380859	4.03848309344072\\
7.30164578431777	3.94136087081416\\
7.24674328402024	3.82155971649234\\
7.16543172741847	3.67033470293414\\
7.04344941432565	3.47445678193261\\
6.85688671052396	3.21347769087934\\
6.56480894289311	2.85543554581792\\
6.09704081110383	2.35120251270526\\
5.33922949371652	1.631832926147\\
4.13741497712578	0.627506738731737\\
2.39201362139747	-0.653956054596282\\
0.271129584082802	-2.0163128519437\\
-1.76706418105212	-3.15304630900311\\
-3.36477841515265	-3.9171910106998\\
-4.48057736347614	-4.36672322110522\\
-5.23178650862857	-4.61527876569066\\
-5.74140679225126	-4.74863483186194\\
-6.09621456353023	-4.81771508798844\\
-6.35101845621712	-4.85067488062163\\
-6.53967088879576	-4.86295399720598\\
-6.68330764764067	-4.86315077538039\\
-6.79543275686343	-4.85616295793073\\
-6.88490770779658	-4.8448438231133\\
-6.95770489220665	-4.83088712753518\\
-7.01795224457751	-4.81531109806863\\
-7.06856961994611	-4.79873010511294\\
-7.11166597664871	-4.78151108154078\\
-7.14879330431607	-4.76386568823666\\
-7.18111271547939	-4.74590569465811\\
-7.20950540074214	-4.7276767463185\\
-7.23464817377506	-4.7091791018496\\
-7.25706576499147	-4.69038029875894\\
-7.27716750980255	-4.67122266148723\\
-7.29527332439532	-4.65162737865237\\
-7.31163214276176	-4.63149616603733\\
-7.32643488682237	-4.6107110889196\\
-7.33982331218368	-4.58913282362824\\
-7.35189556689479	-4.56659742372425\\
-7.36270892512731	-4.54291147591349\\
-7.37227984687155	-4.51784535141028\\
-7.38058121681288	-4.49112404964878\\
-7.38753628018219	-4.46241485842959\\
-7.39300835961935	-4.43131067100693\\
-7.39678481775374	-4.39730723605238\\
-7.39855278713633	-4.35977175928978\\
-7.39786269021448	-4.31789894229912\\
-7.39407311222847	-4.27064842485589\\
-7.38626644212047	-4.21665415887213\\
-7.37311752223128	-4.15409054931505\\
-7.35268482515135	-4.08047058639661\\
-7.32207055776348	-3.99233467925752\\
-7.27685311900696	-3.88476016134848\\
-7.21011400666836	-3.75057132588405\\
-7.1107263430797	-3.57904450869508\\
-6.96028340185397	-3.35377011677173\\
-6.72755817023565	-3.04918973401285\\
-6.3588483963266	-2.62547601246506\\
-5.76350289791787	-2.02328116456653\\
-4.80347182629616	-1.16809173024485\\
-3.33099395850477	-0.0142908269286488\\
-1.3533002898372	1.34482324438388\\
0.788163082342835	2.6281575200801\\
2.62994410822462	3.58116089787993\\
3.97684522927382	4.1739541276016\\
4.89332317159997	4.50983315037412\\
5.51049081152359	4.69249098100198\\
5.93412383523154	4.78905695763134\\
6.23362438927599	4.83753309979793\\
6.45206700619369	4.8587482316194\\
6.61613611895751	4.8641942745862\\
};
\addplot [color=mycolor2, forget plot]
  table[row sep=crcr]{%
-0.808897193889222	1.2068070685132\\
-0.67471894432654	1.20245833535103\\
-0.523610279594059	1.1879328887286\\
-0.355602286690072	1.16095101337708\\
-0.172113986428172	1.11941457990547\\
0.0236781959832208	1.06184570982191\\
0.226763932915542	0.987874892639806\\
0.430695552019423	0.898606246212317\\
0.628471241032463	0.796666634737447\\
0.813666088851403	0.685849809625889\\
0.981411473905152	0.570458525142341\\
1.12889670178906	0.454591664773143\\
1.25532005519737	0.341614771537379\\
1.3614533580257	0.233920459000961\\
1.44906751721652	0.132944344606522\\
1.52041068524428	0.0393323749214226\\
1.57782741047141	-0.0468418764510821\\
1.62352634881597	-0.125874872234994\\
1.65946499918193	-0.198273711617701\\
1.68731177841586	-0.264643643790927\\
1.70845233617472	-0.325615284853456\\
1.7240172970881	-0.381801300816121\\
1.7349175934188	-0.433773178884676\\
1.74187988803718	-0.482050834467961\\
1.74547854069128	-0.527099946580128\\
1.74616281641526	-0.569333639502137\\
1.74427919808968	-0.609116372085116\\
1.74008919611336	-0.64676873530157\\
1.73378323461243	-0.682572399495952\\
1.72549120574174	-0.716774788177067\\
1.71529021613791	-0.749593254844176\\
1.70320995215191	-0.78121865142323\\
1.68923598730834	-0.811818232943133\\
1.67331125692419	-0.841537862784264\\
1.65533583439673	-0.870503477728018\\
1.63516506164871	-0.898821748127393\\
1.61260601211971	-0.926579827945359\\
1.58741219891264	-0.953844031504873\\
1.55927638618584	-0.980657195786712\\
1.52782132637932	-1.00703438474726\\
1.49258824550724	-1.03295646066862\\
1.45302296379723	-1.05836088380746\\
1.40845972292123	-1.08312890825763\\
1.35810318442154	-1.10706813633384\\
1.30100981290468	-1.12988922487335\\
1.23607118507752	-1.1511755157412\\
1.16200398222345	-1.17034471327827\\
1.0773548933259	-1.18660286237797\\
0.980533637945768	-1.19889346213648\\
0.869893534347268	-1.20584951255666\\
0.743884766397251	-1.2057645440841\\
0.601306128617595	-1.19661014135249\\
0.441667761094257	-1.17613898136997\\
0.265638700923344	-1.14211484190948\\
0.075483669219649	-1.09268866130737\\
-0.124686947267792	-1.02687859490506\\
-0.329061590529294	-0.945022081749565\\
-0.530781194663546	-0.849003795889475\\
-0.722990736443141	-0.742102114593396\\
-0.899945764438998	-0.62845748146599\\
-1.05777997546494	-0.512352747117631\\
-1.1947231245095	-0.397568796071803\\
-1.31082900078583	-0.286996637511487\\
-1.40744077312449	-0.182536253222041\\
-1.48662530858524	-0.0852023743823495\\
-1.55071680461233	0.0046725042194259\\
-1.60201264240033	0.0872227383389218\\
-1.64260458317222	0.162867011670197\\
-1.67430690470771	0.232172845558747\\
-1.69864409443724	0.295765188241698\\
-1.71687011040416	0.354269643004955\\
-1.73000118943305	0.408280244530071\\
-1.73885186708886	0.458343413176045\\
-1.74406894338996	0.504951945121495\\
-1.74616114088827	0.548544855252462\\
-1.74552383846293	0.589510369296379\\
-1.74245906604218	0.628190390074527\\
-1.73719127666227	0.664885439876693\\
-1.72987949587503	0.699859508876439\\
-1.72062641237825	0.733344499759449\\
-1.70948488701277	0.765544109744199\\
-1.6964622550066	0.796637072143682\\
-1.68152269475052	0.82677971561631\\
-1.66458784175479	0.856107805505731\\
-1.64553574026817	0.884737616685108\\
-1.62419814702131	0.912766155007379\\
-1.60035613145138	0.940270395453863\\
-1.57373385601163	0.967305337574049\\
-1.54399037377056	0.993900589216507\\
-1.51070925984532	1.02005507320129\\
-1.47338592058589	1.04572930415321\\
-1.43141254031662	1.0708345031942\\
-1.38406089980276	1.09521761509475\\
-1.33046384917194	1.11864109614215\\
-1.26959722289676	1.14075622649914\\
-1.20026571257888	1.16106882787121\\
-1.12109901138918	1.1788969420808\\
-1.0305687562377	1.19332178033102\\
-0.927042489768694	1.20313689765039\\
-0.808897193889222	1.2068070685132\\
};
\addplot [color=mycolor1, forget plot]
  table[row sep=crcr]{%
4.56917235808472	3.70268895532509\\
4.69993758331261	3.69869279347102\\
4.80517945949634	3.68875692811782\\
4.891315800367	3.67505917995538\\
4.96290040132152	3.65895608509231\\
5.02321828383969	3.64129574673471\\
5.07467792544773	3.62260597381011\\
5.11907237396694	3.60320901432834\\
5.15775539338562	3.58329249739931\\
5.19176199696943	3.56295381135076\\
5.22189209387517	3.54222806388184\\
5.24876929621458	3.52110568085189\\
5.27288272085366	3.49954330350572\\
5.29461693707984	3.4774702130657\\
5.31427347814993	3.45479163779903\\
5.33208619457745	3.43138974763909\\
5.34823195812602	3.40712277935365\\
5.36283768770315	3.38182247942628\\
5.3759842700924	3.3552898503993\\
5.3877076260163	3.32728900317498\\
5.39799687590262	3.29753872307469\\
5.40678924438351	3.26570112173645\\
5.41396095682356	3.23136643455092\\
5.41931285689038	3.19403258541623\\
5.42254870979102	3.15307750500636\\
5.42324298996123	3.10772124497457\\
5.42079311891639	3.05697350605851\\
5.41434817106706	2.9995600221317\\
5.40270123876636	2.93381789265783\\
5.38412463252741	2.85754479201215\\
5.35611365638706	2.767779099135\\
5.31498215350318	2.66047633057464\\
5.25521576712938	2.53003141868109\\
5.16843041292733	2.36857973242306\\
5.04170453134492	2.1650099448242\\
4.85499737117693	1.90370247719085\\
4.57753953354758	1.56336054093121\\
4.16419338875672	1.11739931029442\\
3.55651701769009	0.539884205672537\\
2.70077846049034	-0.176194625640385\\
1.59598179398859	-0.988275705995668\\
0.34581999886164	-1.79163729311536\\
-0.868505547238941	-2.46843827368793\\
-1.89989521026677	-2.96104252617115\\
-2.70104992150966	-3.28328830450124\\
-3.2968512620211	-3.4801094636453\\
-3.73503601365084	-3.59459902752057\\
-4.0596436972177	-3.65770455314172\\
-4.30398704887353	-3.68925815552074\\
-4.49148882414584	-3.7014315428108\\
-4.63824137087147	-3.70161391312862\\
-4.75529535780737	-3.69430712149698\\
-4.85031570845045	-3.68227869809872\\
-4.92869739665469	-3.66724603675176\\
-4.99430017400813	-3.65028169006803\\
-5.04993128854323	-3.63205557469415\\
-5.09766502239726	-3.61298162986896\\
-5.13905677865899	-3.59330788855458\\
-5.17528853037799	-3.57317251755698\\
-5.2072690591085	-3.55263902527036\\
-5.23570398384655	-3.53171846300699\\
-5.26114528157326	-3.51038332169017\\
-5.28402664366843	-3.4885759784651\\
-5.30468886032385	-3.46621343317026\\
-5.32339802304281	-3.44318938443409\\
-5.34035840243682	-3.41937425196426\\
-5.35572121908679	-3.39461345056524\\
-5.3695900665462	-3.36872399804908\\
-5.38202339202143	-3.34148935081348\\
-5.39303413670034	-3.31265217544548\\
-5.40258633778172	-3.28190455342899\\
-5.41058815087062	-3.24887484707502\\
-5.41688030653274	-3.21311008692461\\
-5.42121838611503	-3.17405221510248\\
-5.42324636218192	-3.13100574642952\\
-5.4224573917147	-3.08309325166024\\
-5.4181355313962	-3.02919330887705\\
-5.40926827985627	-2.9678528711087\\
-5.39441364137628	-2.89716183780545\\
-5.37149503593939	-2.81457122256595\\
-5.33747996512858	-2.71662666136708\\
-5.28786926267767	-2.59857519079825\\
-5.2158766204963	-2.45378606231255\\
-5.11110805423888	-2.27291421151429\\
-4.9574722721946	-2.0427622478448\\
-4.730065934337	-1.7449762831817\\
-4.39127956248136	-1.35535458235569\\
-3.88852005730529	-0.846294424062207\\
-3.16171272147315	-0.198036167364282\\
-2.17576608285899	0.575550783655259\\
-0.978405967993526	1.39910336774656\\
0.277135513527748	2.15141677350008\\
1.41235144448438	2.73817910064784\\
2.32873253767492	3.14086783121862\\
3.0217321914352	3.3944321931087\\
3.53266232012548	3.54541338136133\\
3.90919611930924	3.63111514804455\\
4.19018187717937	3.67652388323898\\
4.40369542251996	3.6972200520764\\
4.56917235808472	3.70268895532509\\
};
\addplot [color=mycolor2, forget plot]
  table[row sep=crcr]{%
-0.264450999198444	0.83214045157284\\
0.214248321186214	0.816914604110463\\
0.67404969566841	0.773214467044598\\
1.07668900558516	0.709144913196186\\
1.40495128677245	0.635400964505414\\
1.66060100929812	0.560697628911048\\
1.85510232803173	0.490203198683265\\
2.00198977903837	0.426152500911911\\
2.11319212545656	0.36900377396721\\
2.19803948641445	0.31834386558543\\
2.26344449923751	0.27342323238087\\
2.31441326848521	0.23342546846939\\
2.35454816309746	0.197585432330489\\
2.38644761730663	0.165230879414624\\
2.41199890705233	0.135788848554691\\
2.43258512757489	0.108777538152803\\
2.44922954376228	0.0837934455534409\\
2.46269600712489	0.0604980669807242\\
2.47355903123215	0.0386058415277215\\
2.48225297757613	0.017873829332876\\
2.48910679927453	-0.00190689675315377\\
2.49436870924686	-0.0209183594280127\\
2.49822372585025	-0.0393215490951768\\
2.50080609268576	-0.0572613155803396\\
2.50220791857965	-0.0748705063894331\\
2.50248493397078	-0.092273546530673\\
2.5016599400423	-0.109589626731456\\
2.49972428823439	-0.126935644927175\\
2.49663753563351	-0.144429030651753\\
2.49232524956892	-0.162190573051497\\
2.48667475943331	-0.180347369905173\\
2.47952845167494	-0.199036016081142\\
2.47067394697809	-0.218406153292496\\
2.45983014943958	-0.238624505032164\\
2.44662766294206	-0.259879513653218\\
2.43058135236097	-0.28238666560122\\
2.41105177329823	-0.306394505602213\\
2.38719064187274	-0.33219114112874\\
2.35786324977618	-0.360110606123845\\
2.32153750736643	-0.390537555348389\\
2.27612499249618	-0.423906948327799\\
2.2187544573104	-0.460691807730163\\
2.14545511915088	-0.501365289826523\\
2.05073370658385	-0.546310803193739\\
1.92706761617213	-0.59563317347751\\
1.76445875015479	-0.648796620887027\\
1.55049610602075	-0.704004144224202\\
1.27194805388467	-0.75732112120375\\
0.919473693297603	-0.801912560176133\\
0.49613899942984	-0.828489649186642\\
0.025692080922134	-0.828340341005733\\
-0.449320029613414	-0.798245514008142\\
-0.88412407220191	-0.743051905881648\\
-1.25034969255114	-0.672857552252268\\
-1.54125404579476	-0.597764286432564\\
-1.76463721550565	-0.524729734527018\\
-1.93369422062092	-0.457314658533467\\
-2.06140671142647	-0.396728677678376\\
-2.15842705775399	-0.34290581788556\\
-2.23282103499755	-0.295217495165058\\
-2.29048061161413	-0.252858746887685\\
-2.33565292845559	-0.215030107629663\\
-2.37139548208082	-0.181010551314087\\
-2.39992046853641	-0.150177905945483\\
-2.42284161642236	-0.122006247725035\\
-2.44134718298104	-0.0960546201814722\\
-2.45632033484567	-0.0719536257249904\\
-2.46842298187979	-0.0493926339595112\\
-2.47815443889097	-0.0281085739030853\\
-2.48589273303814	-0.00787648831769808\\
-2.49192386517619	0.0114982926954471\\
-2.49646261608196	0.0301866086436648\\
-2.49966732599551	0.0483408202054516\\
-2.5016502881176	0.0660992978795976\\
-2.50248485714598	0.0835902585406819\\
-2.50220999635652	0.100935130134395\\
-2.50083271248245	0.11825159952906\\
-2.49832861592892	0.135656479949484\\
-2.49464066471791	0.153268522416851\\
-2.48967597965003	0.171211289612278\\
-2.48330043261038	0.189616209620683\\
-2.47533048441306	0.208625929581672\\
-2.46552145099452	0.228398092645316\\
-2.45355096278771	0.249109660476085\\
-2.4389957876638	0.270961887204528\\
-2.42129931899874	0.294185997820034\\
-2.39972575103622	0.319049491914846\\
-2.37329508448751	0.345862697986741\\
-2.34069038956722	0.374984577974526\\
-2.30012498648155	0.40682550239901\\
-2.24915245698217	0.441842163594816\\
-2.18439784997777	0.480514825696315\\
-2.101188611267	0.523287783563198\\
-1.99308212590181	0.570437524813583\\
-1.85135785953935	0.621808199153835\\
-1.66474020557732	0.67632943271341\\
-1.42005374316342	0.731253673206172\\
-1.10516915155054	0.781251166446899\\
-0.715719010491333	0.818066616494719\\
-0.264450999198446	0.83214045157284\\
};
\addplot [color=mycolor1, forget plot]
  table[row sep=crcr]{%
3.17677417039348	3.06220363740278\\
3.31935043399715	3.05782096511845\\
3.43769887712985	3.046630132178\\
3.53706168261554	3.03081663195866\\
3.62140584972235	3.01183421791775\\
3.6937481576491	2.9906465340053\\
3.75639911613872	2.96788699145282\\
3.81114261030455	2.94396424135472\\
3.85936723600453	2.91913183226332\\
3.90216223782819	2.89353430231897\\
3.94038776148057	2.86723767038152\\
3.97472648412622	2.84024947963139\\
4.0057216751557	2.81253172303048\\
4.03380526928449	2.78400879558825\\
4.05931847405358	2.75457184164777\\
4.08252666933673	2.72408034563199\\
4.10362979670892	2.69236145438735\\
4.12276901302913	2.6592072545807\\
4.14003004327764	2.62437001620671\\
4.15544337087205	2.58755522164173\\
4.16898111242984	2.54841200325468\\
4.1805501005934	2.50652038741957\\
4.189980298719	2.46137446315747\\
4.19700713642881	2.41236022919505\\
4.20124560249939	2.35872638667083\\
4.20215284043759	2.29954569137659\\
4.19897438473111	2.23366361187286\\
4.19066679605515	2.15962992618775\\
4.17578595253724	2.07560756621235\\
4.15232521745815	1.97925172399264\\
4.11748082877196	1.86755173387132\\
4.0673135628575	1.73663058035474\\
3.99626893137781	1.58150700167523\\
3.89652289800053	1.39585416370095\\
3.75716409757238	1.17185931374062\\
3.56337212632166	0.900438705505084\\
3.29612665441547	0.572327965997373\\
2.93370576079621	0.180882711667256\\
2.45705472102703	-0.272674545189229\\
1.86050778278718	-0.772482316597883\\
1.16451685236404	-1.2845483777012\\
0.419258436400468	-1.76359479187261\\
-0.310451065573799	-2.17009581352747\\
-0.970512410976555	-2.48499188487467\\
-1.53304219695017	-2.71091402645631\\
-1.99465023348299	-2.86314635273768\\
-2.36617534434433	-2.96004404359145\\
-2.6633874611004	-3.01771046191725\\
-2.90173787310418	-3.04841749658235\\
-3.09431332561097	-3.06087319520195\\
-3.25147897947426	-3.06103735904053\\
-3.38119118486251	-3.05291928331043\\
-3.48947990122615	-3.03919651729106\\
-3.58090349168907	-3.02165201979975\\
-3.65891865332526	-3.00147015489306\\
-3.72616295980993	-2.97943339228997\\
-3.78466451464927	-2.95605216745467\\
-3.83599561684905	-2.9316505741551\\
-3.88138499309718	-2.90642301715518\\
-3.92179985990072	-2.88047171211381\\
-3.95800612186526	-2.85383144118423\\
-3.99061268873467	-2.82648570719649\\
-4.02010416898246	-2.7983769604186\\
-4.04686494837156	-2.76941261446765\\
-4.07119676264246	-2.73946793465286\\
-4.09333122120634	-2.70838645170393\\
-4.11343825448174	-2.67597824779936\\
-4.13163108182871	-2.64201622778269\\
-4.14796798375716	-2.60623028989809\\
-4.16245087258137	-2.56829911936056\\
-4.17502035336242	-2.52783912031028\\
-4.18554661162926	-2.48438975275164\\
-4.19381500561436	-2.43739422323724\\
-4.19950460942144	-2.38617405770586\\
-4.20215704988444	-2.32989552089704\\
-4.20113165723341	-2.26752509195969\\
-4.1955409938594	-2.19777021687624\\
-4.18415793433676	-2.11900032717899\\
-4.16528125285283	-2.02914175339791\\
-4.13654073378355	-1.92553911355845\\
-4.0946150919557	-1.8047762910793\\
-4.03482779602556	-1.66245544209737\\
-3.9505828429046	-1.4929500758959\\
-3.83262165276153	-1.28919420220208\\
-3.66816704733271	-1.04267411268889\\
-3.44026358955362	-0.743995026519903\\
-3.12816604364904	-0.384708680621449\\
-2.71048582054154	0.0387050776214842\\
-2.17322418619597	0.518512363739549\\
-1.52242222011607	1.02971849467443\\
-0.794040529230507	1.53102853372387\\
-0.0485561944193281	1.97767233534558\\
0.651596897764944	2.33925712839912\\
1.26464629340802	2.60828678671132\\
1.77598975688712	2.79508091712544\\
2.19072196670672	2.91742086361568\\
2.52306177366165	2.99292260089135\\
2.78903776233008	3.03581499533663\\
3.00303791269362	3.05649995546078\\
3.17677417039348	3.06220363740278\\
};
\addplot [color=mycolor2, forget plot]
  table[row sep=crcr]{%
0.661273782960425	0.866962186619865\\
1.15796205284672	0.851693038651411\\
1.55385438750588	0.814423660462159\\
1.85381732281985	0.766885959235898\\
2.07617283338148	0.717021594888785\\
2.24051638617859	0.669030895999442\\
2.36292503570404	0.624672111410093\\
2.45527929738384	0.584396976084643\\
2.52599281815815	0.548048739106013\\
2.5809449574233	0.515230268539075\\
2.62424905343567	0.485480899947359\\
2.65880723615788	0.458354204400774\\
2.68669128762984	0.433447573305808\\
2.70939944994863	0.410409558737287\\
2.72802917122473	0.388937630991543\\
2.74339392644928	0.368772335789921\\
2.75610293693549	0.349690567260384\\
2.76661614353836	0.331499105872901\\
2.77528250747898	0.314028834342263\\
2.78236692940358	0.297129709844776\\
2.78806927369543	0.280666427697226\\
2.79253780974802	0.264514657597637\\
2.79587860673463	0.248557718915537\\
2.79816189726337	0.232683562264001\\
2.79942606438194	0.216781929025155\\
2.79967964434784	0.200741562886482\\
2.79890153260681	0.184447344700957\\
2.79703940171752	0.167777212056612\\
2.79400616164061	0.150598705922155\\
2.78967408895369	0.132764956360156\\
2.78386599120974	0.114109874654288\\
2.77634241366005	0.0944422566234181\\
2.76678337573911	0.0735384172667591\\
2.75476234844049	0.0511328667796173\\
2.73970900106703	0.0269064030239178\\
2.72085542073849	0.000470848901667607\\
2.69715767306884	-0.028650448966868\\
2.66718017648142	-0.0610471935020732\\
2.62892365232438	-0.0974520813058362\\
2.57956760595732	-0.138774628197668\\
2.51508546665734	-0.186132564282447\\
2.42967877912178	-0.240862688463524\\
2.31498413577605	-0.304466200882717\\
2.15909101178252	-0.378387607580392\\
1.94572497836975	-0.463428493655018\\
1.65480195610755	-0.558496052466978\\
1.26711459734576	-0.658537049264833\\
0.77657514989749	-0.752577661258321\\
0.207369162339691	-0.824951408390095\\
-0.381574349653424	-0.862479910369312\\
-0.921511686533697	-0.862890828216087\\
-1.36871821476444	-0.835005611462908\\
-1.71481278672137	-0.791341871441737\\
-1.97341590826614	-0.741908687426995\\
-2.16448928144107	-0.692640408697396\\
-2.30612758268735	-0.646348835619765\\
-2.41226322272021	-0.604025166399864\\
-2.49292438342471	-0.565753910039911\\
-2.55514806487159	-0.531226821686693\\
-2.60384842392396	-0.500000681614672\\
-2.64247594030227	-0.471616062174229\\
-2.67347880035625	-0.445646546514242\\
-2.69861595862575	-0.421714983761723\\
-2.71916769755915	-0.399494981073737\\
-2.73607754777726	-0.378706371822785\\
-2.75004866970255	-0.359108715867074\\
-2.76160996555653	-0.340494620357544\\
-2.77116190922859	-0.322683592178605\\
-2.77900862574165	-0.305516634102342\\
-2.78538051207993	-0.288851575556715\\
-2.7904502368164	-0.272559038920184\\
-2.7943440032954	-0.25651891211297\\
-2.79714932739429	-0.240617193347299\\
-2.79892014917985	-0.224743077476029\\
-2.79967979249433	-0.20878615729131\\
-2.79942205736934	-0.192633613239204\\
-2.79811054174722	-0.176167258913038\\
-2.79567611360316	-0.15926029546696\\
-2.79201226690305	-0.141773603712912\\
-2.78696786713369	-0.123551365565637\\
-2.7803364887445	-0.104415753390757\\
-2.7718411169053	-0.0841603527255041\\
-2.7611123526863	-0.0625418867677141\\
-2.74765730453616	-0.0392696879289773\\
-2.73081488156519	-0.0139922179360294\\
-2.70969093033265	0.0137202006216811\\
-2.68306312346527	0.0443973335648985\\
-2.64924005970246	0.0786963249046509\\
-2.60585085623834	0.117433966448259\\
-2.54953005584578	0.161620602371638\\
-2.47544936887848	0.212484709782595\\
-2.37664170939938	0.271459228033626\\
-2.24309728766334	0.34006138581561\\
-2.06078357220008	0.419521432230619\\
-1.81127955202615	0.509903493366639\\
-1.47394379994429	0.608430286702134\\
-1.03407711740967	0.707231481400013\\
-0.498790464781053	0.792473568594955\\
0.0891670364609763	0.848526699392398\\
0.661273782960423	0.866962186619865\\
};
\addplot [color=mycolor1, forget plot]
  table[row sep=crcr]{%
2.35723974553073	2.84416812417343\\
2.51069738272667	2.83942105928315\\
2.64251651332025	2.82693423750504\\
2.75649096143632	2.80877852846073\\
2.85571591958509	2.78643416026401\\
2.94269893019672	2.76094838995348\\
3.0194661373763	2.73305269663057\\
3.08765471098761	2.70324792118058\\
3.14858937420538	2.6718652646589\\
3.20334402719018	2.63910956301037\\
3.25279054010171	2.60508969437066\\
3.29763695043819	2.56983965516583\\
3.33845707939824	2.53333281809185\\
3.37571323580098	2.49549112362842\\
3.40977331694871	2.45619039921715\\
3.44092328605599	2.4152625915511\\
3.46937571450358	2.37249539254628\\
3.49527481797884	2.32762950385245\\
3.51869817541443	2.28035359214749\\
3.53965508182247	2.23029681840946\\
3.55808123108122	2.17701866432022\\
3.57382912947303	2.11999561772275\\
3.58665327772499	2.05860411151641\\
3.5961886953016	1.99209893948848\\
3.60192075740274	1.91958621630789\\
3.60314353362642	1.83998985438197\\
3.59890283311901	1.75201060116323\\
3.5879190029878	1.65407712794404\\
3.56848336662896	1.54428989508587\\
3.53832153606045	1.42036130172444\\
3.49441792939479	1.27956130976565\\
3.43280137970586	1.11868852519635\\
3.34830708785551	0.934105777173234\\
3.23436466491018	0.72190970105947\\
3.08292848391465	0.478344752388961\\
2.88477324192349	0.200607294019178\\
2.63049975041719	-0.111837186200972\\
2.31261787611998	-0.455485174088717\\
1.92874622554254	-0.82108653917307\\
1.48506617658905	-1.1931092075257\\
0.998035770061693	-1.55162561181268\\
0.492319908450856	-1.87674651350774\\
-0.00500095663271922	-2.15371625030846\\
-0.47105533379721	-2.37591157248377\\
-0.891240774070467	-2.54450272708476\\
-1.25958652926944	-2.66583184732752\\
-1.57666258567759	-2.74841093472104\\
-1.84686814876165	-2.80074834040904\\
-2.07622761179615	-2.83023154463703\\
-2.27099612042322	-2.84278135592224\\
-2.43694060189584	-2.84291981785256\\
-2.57905701509396	-2.83399968754863\\
-2.70152578587721	-2.81846070138456\\
-2.80777793317631	-2.79805587627732\\
-2.90060091198471	-2.7740319654112\\
-2.98224904339135	-2.74726589081205\\
-3.05454342556257	-2.71836484841287\\
-3.11895641284093	-2.68773847340791\\
-3.1766804691153	-2.65565027605371\\
-3.22868309348335	-2.62225396515838\\
-3.27575004481665	-2.58761881845915\\
-3.31851901725403	-2.55174708859276\\
-3.35750561464144	-2.51458554782193\\
-3.39312311075937	-2.47603262256475\\
-3.42569713509985	-2.43594209198227\\
-3.45547611349754	-2.39412397304877\\
-3.48263801896802	-2.35034294797793\\
-3.50729374033659	-2.30431447852061\\
-3.52948713952374	-2.25569857292167\\
-3.54919162456181	-2.20409100832365\\
-3.56630279339919	-2.14901165176407\\
-3.58062637817612	-2.08988935876831\\
-3.59186031064004	-2.02604275824673\\
-3.599569200735	-1.95665606520587\\
-3.60314883250387	-1.88074892881337\\
-3.60177739968447	-1.79713929342847\\
-3.59434912274173	-1.70439847331892\\
-3.57938469372077	-1.60079841368833\\
-3.55491199194827	-1.48425299354551\\
-3.51831051117761	-1.35225924191538\\
-3.46611579371748	-1.20185223707173\\
-3.39378973718194	-1.02960194588312\\
-3.295486126987	-0.831704729697735\\
-3.16388956206597	-0.604258631528481\\
-2.99029254310449	-0.343853312370394\\
-2.76519701600118	-0.0486215255346563\\
-2.4798214882969	0.280188756232953\\
-2.12877973569313	0.636373170420278\\
-1.71359192911296	1.00747964391791\\
-1.24556624327728	1.37536326684738\\
-0.745794375128828	1.71945418960585\\
-0.241015860211289	2.0218665981489\\
0.243063883633684	2.27174582229565\\
0.687439762821932	2.46659503015395\\
1.08197874485966	2.6105629683591\\
1.42431832352954	2.71141548526532\\
1.7172655368299	2.77786498936206\\
1.96625697011267	2.81794182510089\\
2.17755897671675	2.83831004944716\\
2.35723974553073	2.84416812417343\\
};
\addplot [color=mycolor2, forget plot]
  table[row sep=crcr]{%
1.18991530942429	0.996510309110826\\
1.60054088847078	0.984021430456015\\
1.91377244156376	0.954577655347213\\
2.14791203027807	0.917473554270012\\
2.32251535465373	0.878305494358859\\
2.45373375878422	0.839972813807476\\
2.55360600181979	0.803767192939426\\
2.63073362704449	0.77012093230025\\
2.69117823317898	0.739041899974995\\
2.73921392535886	0.710346566448502\\
2.77787795324581	0.683778702006049\\
2.80935286990539	0.659067073284671\\
2.83522664770944	0.635951634127522\\
2.85666909055672	0.614193833395493\\
2.874552002102	0.593579139214968\\
2.8895317218807	0.573915920283808\\
2.90210637368735	0.555032767841512\\
2.91265596359902	0.536775278792383\\
2.92147069986961	0.51900277283484\\
2.92877109843297	0.501585135454873\\
2.93472224979762	0.484399835099438\\
2.93944383389818	0.467329089405828\\
2.94301693519272	0.450257116591205\\
2.94548833765155	0.43306738503193\\
2.94687270661257	0.415639756143704\\
2.94715284928651	0.397847396552401\\
2.94627805622044	0.379553310728468\\
2.94416033513793	0.360606310769947\\
2.94066812865059	0.340836191532789\\
2.93561682499492	0.320047811504501\\
2.92875498041601	0.298013685916316\\
2.91974460543797	0.274464570164272\\
2.90813302103366	0.24907733921039\\
2.89331250200592	0.221459244560428\\
2.87446193899167	0.191127356986687\\
2.85046167457741	0.157481714562739\\
2.81976792364993	0.119770512339495\\
2.78022600914904	0.0770459308095315\\
2.72879132806117	0.0281107725380253\\
2.66111397018444	-0.0285389374000364\\
2.57093264307373	-0.0947569876766271\\
2.44923757809347	-0.17272541723286\\
2.28326719813619	-0.264748920737332\\
2.05576903546383	-0.372618964547878\\
1.74588642840482	-0.496154129648031\\
1.33459016541791	-0.630655150650412\\
0.817792878440208	-0.764236171010543\\
0.223210775629237	-0.878588321361184\\
-0.387488105528008	-0.956662577001607\\
-0.94544396218029	-0.992562807753562\\
-1.40823102692931	-0.993115571787897\\
-1.7683138275155	-0.970744942497015\\
-2.03945352735151	-0.936555785635127\\
-2.24154691063925	-0.89791728601457\\
-2.39270406004862	-0.858926887362741\\
-2.50698083273759	-0.821563309124965\\
-2.59458434079788	-0.786617165495715\\
-2.66273967323147	-0.754269284531161\\
-2.71653355994551	-0.724411464036381\\
-2.75956449421103	-0.696813713045393\\
-2.79440326481326	-0.671207546223913\\
-2.8229086170185	-0.64732529779844\\
-2.84644142929244	-0.62491699822928\\
-2.86601018748154	-0.603756084633368\\
-2.88237045822303	-0.58363974573412\\
-2.8960935416649	-0.564386845692692\\
-2.90761432763437	-0.545834891706278\\
-2.91726496392581	-0.527836746921811\\
-2.92529870926235	-0.510257397710516\\
-2.93190687920515	-0.49297088342211\\
-2.93723082678732	-0.475857394184777\\
-2.94137025218947	-0.4588004893601\\
-2.94438869164446	-0.441684360051673\\
-2.94631671937614	-0.424391039555058\\
-2.94715315692405	-0.406797447755458\\
-2.94686438542758	-0.388772134015282\\
-2.94538166956615	-0.370171553943397\\
-2.94259620012493	-0.350835674540475\\
-2.93835131578995	-0.330582644817653\\
-2.93243103516107	-0.309202189085831\\
-2.92454356182793	-0.286447270097373\\
-2.91429773507599	-0.262023419987743\\
-2.90116935647461	-0.235574939359954\\
-2.88445272456195	-0.206666914582644\\
-2.86319023870879	-0.174761714560406\\
-2.83606910795067	-0.13918836809934\\
-2.80126834127463	-0.099103196869256\\
-2.75623050704679	-0.0534408333077383\\
-2.69732087016683	-0.000857646209580513\\
-2.61932360436197	0.0603221566113784\\
-2.51472211999705	0.132126713227889\\
-2.37275701371166	0.21684430587183\\
-2.17846019986387	0.316643567600139\\
-1.9124699695339	0.43258002030732\\
-1.55372938073449	0.562588128013722\\
-1.08855103789696	0.69860911923824\\
-0.526985364418182	0.825043987363956\\
0.0847239315437372	0.922867884974586\\
0.676587249516098	0.979694143100387\\
1.18991530942428	0.996510309110826\\
};
\addplot [color=mycolor1, forget plot]
  table[row sep=crcr]{%
1.84790736958456	2.85539624508498\\
2.00892487226021	2.85038734938052\\
2.15161460587838	2.83684839195735\\
2.27847365068873	2.81662215054955\\
2.39169364626266	2.79111153615171\\
2.49316493440788	2.76136860581943\\
2.58449971813475	2.72816920137418\\
2.66706285803228	2.69207280870998\\
2.7420039377821	2.65346917087495\\
2.81028731684173	2.61261379962829\\
2.87271868648525	2.56965450771895\\
2.92996763362065	2.52465080306163\\
2.98258621771325	2.47758763347676\\
3.03102378366391	2.42838463079329\\
3.07563829063698	2.37690170719065\\
3.1167044047097	2.32294161098691\\
3.15441852079681	2.2662498495781\\
3.18890076607437	2.20651222680153\\
3.22019390087017	2.14335011429239\\
3.24825887456952	2.07631347802289\\
3.27296661059502	2.00487161356855\\
3.29408538192824	1.9284015154322\\
3.311262895117	1.84617383684868\\
3.32400193215177	1.7573365242676\\
3.33162812979233	1.66089649997351\\
3.33324826531379	1.55570032408616\\
3.32769739679072	1.44041576290418\\
3.31347363266625	1.31351787885098\\
3.28866065417024	1.1732859918368\\
3.25084120056975	1.01782206668806\\
3.19701083076644	0.84510710050348\\
3.12351212828304	0.653119751975032\\
3.02602680639661	0.440049138668592\\
2.89968682572986	0.204636619901509\\
2.73938992651037	-0.0533306138224429\\
2.54041231199207	-0.332393904670119\\
2.29936787692003	-0.628772510767444\\
2.01543038031951	-0.935917839041958\\
1.6915119616486	-1.24459598152934\\
1.33488281903928	-1.54376444301386\\
0.956733747481965	-1.8222114167435\\
0.570572069272705	-2.07049238790772\\
0.189937153897144	-2.28244672825619\\
-0.173716559496603	-2.45575876703295\\
-0.512282604500782	-2.5915247044428\\
-0.821236103988375	-2.6932132753532\\
-1.09911022893855	-2.76551271300529\\
-1.34663993521902	-2.81339851505886\\
-1.56590874874089	-2.84153589706066\\
-1.7596689634603	-2.85398154078695\\
-1.93087453770099	-2.85409308307999\\
-2.08239763586714	-2.84455751625081\\
-2.21687874717862	-2.82747429843868\\
-2.33666353451119	-2.80445436217213\\
-2.44379094050302	-2.77671491761417\\
-2.54000866850522	-2.74516153003493\\
-2.62680121605391	-2.71045525623359\\
-2.70542187198806	-2.67306562981485\\
-2.77692404965429	-2.63331144754917\\
-2.84218970045887	-2.59139154037586\\
-2.9019538958854	-2.54740752817672\\
-2.95682537871724	-2.50138022522718\\
-3.00730322378136	-2.45326101069311\\
-3.05378987429397	-2.40293915875533\\
-3.09660082588792	-2.35024585224454\\
-3.13597116970703	-2.2949553819004\\
-3.17205910656379	-2.23678385430369\\
-3.20494641896013	-2.17538558811419\\
-3.23463574062408	-2.11034726537688\\
-3.26104429275771	-2.04117982108518\\
-3.28399355861275	-1.96730800458932\\
-3.30319413986797	-1.88805754448315\\
-3.31822478069885	-1.80263992193353\\
-3.32850427156098	-1.71013495560625\\
-3.33325469357533	-1.6094718081165\\
-3.33145432608481	-1.4994097747024\\
-3.32177869960428	-1.37852151887596\\
-3.30252908668592	-1.24518358106745\\
-3.27154981225758	-1.09758240218248\\
-3.22614016820033	-0.933749189986639\\
-3.16297496042255	-0.751643881709563\\
-3.07806162632474	-0.549316485789908\\
-2.96678257255502	-0.325180258718806\\
-2.82409665235852	-0.0784282751791394\\
-2.64499270725013	0.190400163799164\\
-2.42527520518298	0.478760764606481\\
-2.16267778572755	0.781519542224893\\
-1.85811745768085	1.09072924063741\\
-1.51666503575119	1.39608862323249\\
-1.14768296517624	1.68623139338685\\
-0.763781266444752	1.95059658468864\\
-0.378778991262895	2.18124436867327\\
-0.00539508952684124	2.37393399176962\\
0.346498809368338	2.52815978543668\\
0.670614869027265	2.64635173309969\\
0.964053154865023	2.73272374538172\\
1.22656050661286	2.79220371918969\\
1.45964412621222	2.82966611705173\\
1.66579343430517	2.84949388873804\\
1.84790736958455	2.85539624508498\\
};
\addplot [color=mycolor2, forget plot]
  table[row sep=crcr]{%
1.50442737389697	1.14866712477609\\
1.84486918014159	1.13833029946581\\
2.10448822423007	1.11391457163613\\
2.3011971502059	1.08272290206621\\
2.45090974726946	1.04912002215542\\
2.56603302766523	1.01547382385565\\
2.65570683287098	0.982953137479371\\
2.72651451942016	0.952054315709736\\
2.78317393134374	0.922913951356243\\
2.82907939203983	0.895485069268227\\
2.86669388594656	0.869633380781392\\
2.89782359894184	0.845188565340106\\
2.9238079099883	0.821970740313312\\
2.94565078800753	0.799803392836786\\
2.96411211305387	0.778518992555047\\
2.97977162374811	0.757960692973232\\
2.99307406546933	0.737981977689385\\
3.00436130164132	0.718445251619882\\
3.01389526417741	0.699219902810375\\
3.02187435680076	0.680180095175765\\
3.02844507448732	0.661202401810001\\
3.03371002059801	0.642163299896373\\
3.03773309491289	0.6229364922162\\
3.04054232644606	0.603389979670131\\
3.04213058976255	0.583382773532295\\
3.04245423894509	0.562761098074081\\
3.0414294909429	0.541353887770826\\
3.03892616141411	0.518967322661358\\
3.03475806787604	0.495378063865181\\
3.02866902212632	0.470324740603633\\
3.02031277145188	0.443497090399019\\
3.00922441801632	0.414521954395087\\
2.99477959777149	0.382945070774552\\
2.97613580484955	0.348207292672608\\
2.95214736276928	0.309613518652939\\
2.92124119161162	0.266292391618661\\
2.88123413210468	0.217145053754499\\
2.8290638188195	0.160782847536577\\
2.760394989693	0.0954589923200107\\
2.66905771559782	0.0190128403057332\\
2.54629428349928	-0.0711224631742372\\
2.37989684982899	-0.177729197414635\\
2.15364768360016	-0.303187106141744\\
1.84825198609768	-0.448038536550649\\
1.44615947755651	-0.60844406975839\\
0.942723826257913	-0.773286400684488\\
0.360664582674148	-0.924044527718796\\
-0.246475983542887	-1.04114576216124\\
-0.814639713355748	-1.11403929809116\\
-1.29900475074848	-1.1453421728842\\
-1.6858359904969	-1.14584998396187\\
-1.98365383877806	-1.12734683822528\\
-2.2096341597025	-1.09883498124102\\
-2.38106886874338	-1.06603877517639\\
-2.51215503156031	-1.03220860298567\\
-2.61358846378269	-0.999030512193673\\
-2.69313749782962	-0.967286591441651\\
-2.75637485266642	-0.937264373264917\\
-2.80729884540031	-0.908992680864494\\
-2.8487972968282	-0.882372273200659\\
-2.88297646159004	-0.857246331048154\\
-2.91138955008335	-0.83343748109554\\
-2.93519467329699	-0.810766471372551\\
-2.95526429330422	-0.789060880396075\\
-2.9722615635818	-0.768158466556946\\
-2.98669400668361	-0.747907675110618\\
-2.99895155926237	-0.728166667978184\\
-3.00933370906833	-0.708801604667799\\
-3.01806890596801	-0.68968454834918\\
-3.02532839426926	-0.670691171741342\\
-3.03123591202487	-0.651698322615945\\
-3.0358742176301	-0.632581439134721\\
-3.03928905678174	-0.61321175856261\\
-3.04149092036105	-0.593453225817017\\
-3.04245472768803	-0.573158972216394\\
-3.04211736953351	-0.552167193253061\\
-3.04037283349348	-0.530296201450184\\
-3.0370643805986	-0.507338360176015\\
-3.03197290808334	-0.483052509307378\\
-3.0248001654215	-0.45715436481122\\
-3.01514480943405	-0.429304200989976\\
-3.00246826818674	-0.399090895578792\\
-2.98604584657951	-0.366011128581075\\
-2.96489616756454	-0.329442190017435\\
-2.9376784926671	-0.288606540810655\\
-2.90254216532864	-0.242526202439522\\
-2.85690484271088	-0.189965799366191\\
-2.79712643370924	-0.12936605430645\\
-2.71803656827352	-0.0587780226198686\\
-2.6122765961754	0.0241704844411688\\
-2.4694686152808	0.122197004054777\\
-2.27541753962575	0.238000463103092\\
-2.01207771082531	0.373289058075866\\
-1.66006143102537	0.526794996331922\\
-1.20647397080343	0.69133291546151\\
-0.658778300565647	0.851746047432427\\
-0.0560913168613796	0.987773404311966\\
0.53895165990967	1.08323818102995\\
1.06884545129749	1.13431195086951\\
1.50442737389697	1.14866712477609\\
};
\addplot [color=mycolor1, forget plot]
  table[row sep=crcr]{%
1.51239328471355	2.99187321897874\\
1.67867446084939	2.98667759938608\\
1.82979360645117	2.9723192337742\\
1.96732600132283	2.95037450977582\\
2.09274479981551	2.92210090208251\\
2.20739186151525	2.88848350304977\\
2.31246699395216	2.85027875729118\\
2.40902834647901	2.80805283095023\\
2.49799897887384	2.76221379370481\\
2.58017633617323	2.71303772572031\\
2.65624255967014	2.66068929612474\\
2.72677436488139	2.60523750766455\\
2.79225172868971	2.54666729889171\\
2.85306493714009	2.48488762098636\\
2.909519716944	2.41973650692446\\
2.96184025249874	2.35098355132742\\
3.01016990623873	2.27833013390277\\
3.05456943315731	2.20140765669543\\
3.09501242365782	2.11977403322802\\
3.13137763273854	2.03290867596972\\
3.16343776850579	1.940206292353\\
3.19084423431321	1.84096994158089\\
3.21310727192629	1.73440405875188\\
3.22957098227844	1.61960856793017\\
3.23938288106664	1.49557584701932\\
3.24145810240536	1.36119325479827\\
3.23443928798276	1.21525526659365\\
3.21665487213498	1.05649103857775\\
3.1860812605918	0.883615367048722\\
3.14031868860227	0.69541319788971\\
3.07659653554911	0.490869227322065\\
2.99183113068201	0.269353035253762\\
2.88276573014075	0.0308638592568937\\
2.74622404177374	-0.223675933520155\\
2.57949812677245	-0.492115900945798\\
2.38086022397749	-0.77083354274097\\
2.15013257224817	-1.05465556042521\\
1.88918173053705	-1.33705294216358\\
1.60215899641311	-1.61066869021488\\
1.29533267398154	-1.86812846675396\\
0.976472892525725	-2.10295720056563\\
0.653915809529938	-2.31035347869866\\
0.335560089471391	-2.48761382957053\\
0.0280584572646769	-2.63413417386167\\
-0.263636859153305	-2.75106554317231\\
-0.536360822565975	-2.84078738522603\\
-0.78854748650978	-2.9063621604053\\
-1.01989715152375	-2.95108008829427\\
-1.2310105092283	-2.97813745863803\\
-1.42306013077515	-2.99044419746322\\
-1.59753024291552	-2.9905329869835\\
-1.75602894663267	-2.98053721568007\\
-1.90016322894113	-2.96220964925042\\
-2.03146246118451	-2.93696139900913\\
-2.15133644489144	-2.90590801895207\\
-2.26105652077574	-2.8699151112142\\
-2.36175113824257	-2.82963956649311\\
-2.45440983880741	-2.78556486918132\\
-2.53989160079024	-2.73803019190504\\
-2.61893493296653	-2.68725365585423\\
-2.69216809009164	-2.63335040332824\\
-2.76011842697453	-2.57634618838206\\
-2.82322030867127	-2.51618714518441\\
-2.88182122806917	-2.45274630271093\\
-2.93618590272331	-2.38582731274327\\
-2.9864981671721	-2.31516576443959\\
-3.03286046949205	-2.24042838372721\\
-3.07529073763118	-2.16121036730898\\
-3.11371631342569	-2.07703108769674\\
-3.14796457012237	-1.98732843946967\\
-3.17774974499562	-1.89145219638467\\
-3.20265545172952	-1.7886569420333\\
-3.22211232123201	-1.67809546434025\\
-3.23537031247882	-1.55881402314529\\
-3.24146553402507	-1.42975168383362\\
-3.23918207764659	-1.2897470423625\\
-3.22701062328347	-1.13755722026301\\
-3.20310775495492	-0.971895988183196\\
-3.16526342101049	-0.791500110601493\\
-3.11088911190784	-0.595234941923725\\
-3.03704607498221	-0.382250722140419\\
-2.94054020210392	-0.152197712258686\\
-2.81811512770277	0.09450194458729\\
-2.66677171975208	0.356348286764419\\
-2.48422218721787	0.630477714379279\\
-2.26944337938398	0.912487333112226\\
-2.02322943415994	1.19648116206932\\
-1.74858204526264	1.475429817124\\
-1.45076091491646	1.74185422835084\\
-1.13688708903981	1.98871806337898\\
-0.815141623939223	2.21030456709899\\
-0.49376054385753	2.40283490282641\\
-0.180100480124219	2.56467997680183\\
0.120003514970815	2.69617191608695\\
0.40250210416808	2.79914631394991\\
0.665065380523587	2.87638840670827\\
0.906806359627473	2.93112342909832\\
1.12792007247506	2.96662628190493\\
1.32933028778223	2.98596700459762\\
1.51239328471354	2.99187321897874\\
};
\addplot [color=mycolor2, forget plot]
  table[row sep=crcr]{%
1.72274085757999	1.31026474427909\\
2.01026614301254	1.30152528012605\\
2.23201584329209	1.28065232276943\\
2.40312518212022	1.25350117177955\\
2.5361316963277	1.22363200205133\\
2.6406436810082	1.1930742780253\\
2.72377440734015	1.16291636519221\\
2.79072342258328	1.13369337333612\\
2.84528715314021	1.10562441056408\\
2.89025159926759	1.07875253892653\\
2.9276780567195	1.05302569164154\\
2.95910618091147	1.02834278534591\\
2.98569717105014	1.00457957025386\\
3.00833475323026	0.98160273795666\\
3.02769671139579	0.959277228056095\\
3.04430586543995	0.937469588196718\\
3.05856662245834	0.916049031872386\\
3.07079129744337	0.89488713481783\\
3.081219074456	0.87385669836927\\
3.09002956953075	0.85283006267768\\
3.09735232592094	0.831677002423555\\
3.10327312584447	0.810262240935\\
3.10783767485561	0.788442550430583\\
3.11105295868559	0.766063350952403\\
3.1128863527598	0.742954667906426\\
3.11326235223077	0.718926249874456\\
3.11205655667378	0.693761577131293\\
3.10908625587767	0.667210399324376\\
3.10409657886324	0.638979319139999\\
3.09674062726076	0.608719777182073\\
3.08655122798497	0.576012580850285\\
3.07290077666084	0.540347848747005\\
3.05494390693558	0.501098916829594\\
3.03153513242212	0.457488415542408\\
3.00110980286464	0.408544518113625\\
2.96151132350235	0.353045653896105\\
2.90974057504121	0.289453696806686\\
2.84159626443886	0.215840907962605\\
2.75117357532672	0.129829395360097\\
2.63021169159603	0.0285929215176733\\
2.46737755444351	-0.0909646386826638\\
2.24784969233456	-0.231626738984467\\
1.95417622824219	-0.394518792626553\\
1.57027438833949	-0.576708641133046\\
1.0904661840107	-0.768301080556169\\
0.531564905798336	-0.951569183923089\\
-0.0623418971493884	-1.10568563341469\\
-0.633617989988535	-1.2161021722935\\
-1.13605816033267	-1.28069638907403\\
-1.54939419097584	-1.30745770207334\\
-1.87572926586082	-1.30788835936333\\
-2.12836987817241	-1.29217629577013\\
-2.32304612299803	-1.26759473529956\\
-2.47372557267457	-1.23875160323609\\
-2.591453345156	-1.20835455614213\\
-2.68451842074422	-1.17790230524224\\
-2.75900664524931	-1.14816890818611\\
-2.81935945503074	-1.11950898553585\\
-2.8688261931858	-1.09204052968102\\
-2.90980054357077	-1.06575153233741\\
-2.94406181904487	-1.04056117627549\\
-2.97294553842738	-1.01635444607972\\
-2.99746363021922	-0.993001310143848\\
-3.01838936394926	-0.970366966950246\\
-3.03631768842475	-0.948316912478931\\
-3.05170836803424	-0.926718996332547\\
-3.06491698868893	-0.905443712048879\\
-3.07621730464331	-0.884363429157137\\
-3.08581730005135	-0.863350957130833\\
-3.09387058304999	-0.842277640219219\\
-3.10048420139264	-0.821011062591899\\
-3.10572358823947	-0.799412363166568\\
-3.10961505950969	-0.777333099373584\\
-3.11214605015043	-0.754611546325656\\
-3.11326306418744	-0.731068263346829\\
-3.11286709410248	-0.706500695997612\\
-3.11080600893108	-0.680676501158869\\
-3.10686308067396	-0.653325177135875\\
-3.10074036495277	-0.624127440491527\\
-3.09203500161607	-0.592701605607224\\
-3.08020554612993	-0.558585981669299\\
-3.06452402195011	-0.521216001401317\\
-3.04400726239405	-0.479894455357776\\
-3.01731796332176	-0.433752905048441\\
-2.9826213076251	-0.381702319818754\\
-2.93737677453569	-0.322371818288306\\
-2.87803728329398	-0.254037504262141\\
-2.79962226426849	-0.174551946279693\\
-2.69513849711839	-0.0813055615369774\\
-2.55487350140267	0.0287033971132761\\
-2.36575521837367	0.158525653958557\\
-2.11139719754735	0.310347034826338\\
-1.77424476755579	0.483627088348244\\
-1.34196332706271	0.672275560929426\\
-0.818726179635004	0.862306244954394\\
-0.23541171032341	1.03343880109232\\
0.354231974656627	1.1667926388925\\
0.895347405057608	1.25378950593271\\
1.35412989498003	1.298095646019\\
1.72274085757999	1.31026474427909\\
};
\addplot [color=mycolor1, forget plot]
  table[row sep=crcr]{%
1.28286346141528	3.19376085633426\\
1.45344992176763	3.18841316743345\\
1.61146233751843	3.17338414571857\\
1.75790426515861	3.15000377012599\\
1.89376274945308	3.11936411942584\\
2.01997584878284	3.08234393604552\\
2.13741318854943	3.03963409856881\\
2.24686552217705	2.99176163900756\\
2.34904020189213	2.93911108178861\\
2.44456026355682	2.88194260248964\\
2.53396546817392	2.82040693037988\\
2.61771412103994	2.75455715049177\\
2.69618483097944	2.68435767238165\\
2.76967760447452	2.60969067558258\\
2.83841381879587	2.53036035107147\\
2.90253470641246	2.44609525787348\\
2.96209802805208	2.35654912167942\\
3.01707262919853	2.26130043251898\\
3.06733057921585	2.15985126527585\\
3.11263660015073	2.05162586609647\\
3.15263452503901	1.93596973871599\\
3.18683061376612	1.8121502503604\\
3.21457374287693	1.67936018272564\\
3.23503283868411	1.53672620317594\\
3.24717252973688	1.38332493572243\\
3.24972896934425	1.21821014777914\\
3.24118924870029	1.04045544358652\\
3.21977988880917	0.849217546494955\\
3.1834725580871	0.643825333362289\\
3.13001814199517	0.423898551439897\\
3.05702282840415	0.189496614979834\\
2.96208046408279	-0.0587090367002419\\
2.84297176184823	-0.319256458036893\\
2.69793031738669	-0.589740078626076\\
2.52595624426856	-0.866727932818914\\
2.32713254218933	-1.1458000803693\\
2.10287500662487	-1.42174843885806\\
1.8560370038427	-1.68894570188325\\
1.59080879521382	-1.94184140203882\\
1.31239968144874	-2.17549463251815\\
1.02655471746895	-2.38602990610422\\
0.739008063001648	-2.57092037985953\\
0.454988272845025	-2.72905503920742\\
0.17886313749819	-2.86060861319044\\
-0.0860395711768357	-2.96677798302946\\
-0.33744884501812	-3.0494633517034\\
-0.574048963935673	-3.11096042749872\\
-0.795312543117972	-3.15370479378238\\
-1.00131324321407	-3.18008446073984\\
-1.19254879174875	-3.19231851186315\\
-1.36979058218974	-3.19239018933388\\
-1.53396526611143	-3.18201985703479\\
-1.68606707975293	-3.16266432210517\\
-1.82709638029388	-3.13553173401469\\
-1.95801885894119	-3.1016043038565\\
-2.07974014240425	-3.06166370936887\\
-2.19309129316794	-3.01631605259258\\
-2.29882165882822	-2.96601463753233\\
-2.3975963930293	-2.91107974953655\\
-2.48999669238132	-2.85171517738617\\
-2.57652134994271	-2.78802153705338\\
-2.65758863235173	-2.72000661991507\\
-2.73353777115685	-2.64759306006419\\
-2.80462954671893	-2.5706236378708\\
-2.87104555930817	-2.48886453900615\\
-2.93288584660048	-2.40200688999573\\
-2.99016453620524	-2.30966690859388\\
-3.04280323088502	-2.21138505388806\\
-3.09062182797518	-2.10662465187518\\
-3.13332649168657	-1.99477062473559\\
-3.17049455290271	-1.87512918724549\\
-3.20155624267972	-1.74692971627559\\
-3.22577342679063	-1.60933047455679\\
-3.24221597569766	-1.461430497007\\
-3.24973717941891	-1.30229072451559\\
-3.2469508233127	-1.13096834386344\\
-3.23221430086974	-0.946569114175136\\
-3.20362451552835	-0.748322915770596\\
-3.15903620107256	-0.535687288342782\\
-3.09611518507011	-0.308481479443459\\
-3.01244092568205	-0.0670484250269967\\
-2.90567144297647	0.187566820065364\\
-2.77377693185499	0.453446422817519\\
-2.61533349956532	0.727675555342042\\
-2.42984546794961	1.00631404478284\\
-2.21803827904919	1.28451198369756\\
-1.98204528315884	1.55679599914171\\
-1.72541491064481	1.81751033362336\\
-1.45289895432384	2.06134485117164\\
-1.17004172724966	2.28384340951517\\
-0.882650474864446	2.48178268065626\\
-0.596261176391486	2.65334877084781\\
-0.315705610257031	2.79809996170941\\
-0.0448430163955496	2.91676025414382\\
0.213533748640937	3.01091864183941\\
0.457649300011542	3.08270894153138\\
0.686605126289815	3.13452461130596\\
0.900197669369818	3.16879654796175\\
1.09873331760935	3.18783973202483\\
1.28286346141527	3.19376085633426\\
};
\addplot [color=mycolor2, forget plot]
  table[row sep=crcr]{%
1.89181837155654	1.47713763134783\\
2.13771553571714	1.46964939389994\\
2.33001012910751	1.45153211491801\\
2.48106192932662	1.42754820119153\\
2.60075866889671	1.40065512434169\\
2.69663740741938	1.3726112403047\\
2.77432332374972	1.34442024213581\\
2.83798709737352	1.31662450572021\\
2.89072519217482	1.28948916501309\\
2.93484951396149	1.26311474598279\\
2.97209959965753	1.23750518193163\\
3.00379547743106	1.21260852999437\\
3.03094720294237	1.18834111188663\\
3.05433340731742	1.16460159315882\\
3.07455783053367	1.14127892782587\\
3.0920901946036	1.11825652862231\\
3.10729586029809	1.0954140776154\\
3.12045735698308	1.07262781857304\\
3.13178992463875	1.04976982064082\\
3.141452537821	1.02670648217422\\
3.14955540266705	1.00329639964764\\
3.15616456487969	0.979387626220089\\
3.1613039916535	0.95481426681401\\
3.16495525780851	0.929392287372161\\
3.16705474608268	0.902914344806229\\
3.16748803510655	0.875143362056503\\
3.16608086437666	0.845804470789074\\
3.16258569349412	0.814574812976593\\
3.15666235788526	0.781070521381554\\
3.14785058585457	0.744829977260624\\
3.1355310666242	0.705292164790822\\
3.1188701800061	0.661768613519819\\
3.09674118814482	0.61340709041739\\
3.06761137190888	0.559145019828183\\
3.02938003899483	0.49765095619948\\
2.9791466974858	0.427254225025101\\
2.91288354275194	0.345868141978941\\
2.82498741508678	0.250925399102838\\
2.70770978260807	0.139373802808658\\
2.55054904516899	0.00784046109987624\\
2.33991897174694	-0.146824509493805\\
2.05988981018934	-0.326293957760779\\
1.69548212177068	-0.528509603760869\\
1.24004416179817	-0.744808228464494\\
0.70548881917923	-0.958493798790827\\
0.12783675969007	-1.14817471608847\\
-0.4416032325795	-1.29616971098341\\
-0.956740550574267	-1.39588042462719\\
-1.39228189895034	-1.45193873083072\\
-1.74441246300098	-1.47475135554755\\
-2.02232349506424	-1.47510894977225\\
-2.23973581762482	-1.46157145103386\\
-2.41001416968227	-1.44005415271611\\
-2.54430815062523	-1.41433334477799\\
-2.65128453981933	-1.38670069250552\\
-2.73746462382523	-1.35849200673027\\
-2.80769308911154	-1.33045148904568\\
-2.86556179250319	-1.30296510910786\\
-2.91374397688086	-1.27620490925257\\
-2.95424296342479	-1.25021663838917\\
-2.98857246774732	-1.22497243047476\\
-3.01788602285498	-1.20040220232462\\
-3.04306970540758	-1.17641214094034\\
-3.064808736466	-1.15289534086714\\
-3.08363552486989	-1.12973763674608\\
-3.09996447248869	-1.10682045932657\\
-3.11411724857011	-1.08402180768201\\
-3.12634110550271	-1.06121598213717\\
-3.1368220116024	-1.03827244474618\\
-3.14569381209809	-1.01505399705126\\
-3.15304422044083	-0.99141434594641\\
-3.15891813305506	-0.967195041532572\\
-3.16331851081537	-0.942221698869746\\
-3.16620484757075	-0.916299346538142\\
-3.16748902131662	-0.889206669386819\\
-3.16702806764512	-0.860688821992978\\
-3.16461309274635	-0.830448374032041\\
-3.15995310784245	-0.798133798868254\\
-3.15265195258697	-0.763324721426685\\
-3.14217558593179	-0.725512891445621\\
-3.12780572030924	-0.684077542100717\\
-3.10857386267558	-0.638253454802491\\
-3.08316704752093	-0.587089765742521\\
-3.04979262753941	-0.529397559975725\\
-3.0059843219301	-0.463685201190143\\
-2.94832597450879	-0.388083521184075\\
-2.87206625263878	-0.300271471378458\\
-2.77060661321309	-0.197432821785875\\
-2.63489222674009	-0.076317237101075\\
-2.4528799873791	0.0664401824931821\\
-2.20960200429706	0.233466482055127\\
-1.88896077904493	0.424913341965717\\
-1.47894830289929	0.635762259080663\\
-0.980951413796855	0.853286954305538\\
-0.41898401443042	1.05763974617386\\
0.161163149450842	1.22809747605534\\
0.708107111546855	1.35198310850992\\
1.18512109293991	1.42877601430675\\
1.57838917173825	1.46679056095548\\
1.89181837155654	1.47713763134783\\
};
\addplot [color=mycolor1, forget plot]
  table[row sep=crcr]{%
1.12337085347325	3.41944784996473\\
1.29798732958167	3.41396102716276\\
1.46196601803898	3.39835266926702\\
1.61598180314091	3.37375209585985\\
1.76072271522622	3.34109904968817\\
1.89686168659934	3.30115800815168\\
2.02503714076216	3.25453409588748\\
2.14584007538346	3.20168882359674\\
2.25980570258933	3.14295460284063\\
2.36740810261529	3.07854748545296\\
2.46905668700011	3.00857790509443\\
2.56509354175359	2.93305940788129\\
2.65579093071233	2.85191549030554\\
2.74134839309963	2.76498474793442\\
2.82188897843307	2.67202460225462\\
2.89745423864257	2.57271393484618\\
2.96799765449178	2.46665503396455\\
3.03337622540302	2.35337536315742\\
3.09334001561041	2.23232980883168\\
3.14751954699297	2.10290426749029\\
3.19541108883424	1.96442170598768\\
3.23636015534362	1.81615217714327\\
3.26954393191587	1.65732869446987\\
3.29395396865839	1.48717133646596\\
3.30838136349419	1.30492239406217\\
3.31140784697252	1.10989565819192\\
3.30140765839972	0.901542834665535\\
3.27656672623956	0.679539217503698\\
3.23492707925443	0.443888687781547\\
3.17446495544757	0.195044351376081\\
3.09320973934176	-0.0659645908731955\\
2.98940644267463	-0.337416511784161\\
2.86171601460507	-0.616815932541158\\
2.70943550282497	-0.900876370043321\\
2.53270626723075	-1.18559532876101\\
2.33266789204068	-1.46643785383327\\
2.1115144148106	-1.73862314181527\\
1.87242265834854	-1.99748102758807\\
1.61934908852423	-2.23882142629321\\
1.35672389053374	-2.45925068976833\\
1.08909666054117	-2.65637987051156\\
0.820797073047426	-2.82889748589837\\
0.555663663292206	-2.97651245966763\\
0.296870565317662	-3.09979918436642\\
0.0468560139137498	-3.19998884002555\\
-0.192663551417901	-3.27874919926749\\
-0.420619965366741	-3.33798418929181\\
-0.636487511273445	-3.3796707082659\\
-0.840167085322681	-3.40573820011202\\
-1.03187563778785	-3.41798833978931\\
-1.21204948593243	-3.41804793573803\\
-1.3812649278788	-3.40734686822138\\
-1.54017619937021	-3.38711338672114\\
-1.68946894170933	-3.35838042815215\\
-1.82982655532066	-3.32199816171223\\
-1.96190668838332	-3.27864937833709\\
-2.0863253406858	-3.22886548704013\\
-2.20364644186376	-3.17304173719398\\
-2.31437516862347	-3.11145089064738\\
-2.41895363492475	-3.04425497466228\\
-2.51775789681386	-2.97151501002056\\
-2.61109545485912	-2.89319877443323\\
-2.6992026176303	-2.8091867661675\\
-2.78224121963233	-2.7192766046861\\
-2.86029427845936	-2.62318616576988\\
-2.9333602412369	-2.52055581556418\\
-3.00134552312201	-2.41095019643518\\
-3.0640550961729	-2.29386014169077\\
-3.12118096441606	-2.16870547008987\\
-3.17228848545515	-2.03483964792995\\
-3.21680070377289	-1.8915576168803\\
-3.25398118948045	-1.73810847252857\\
-3.28291638252849	-1.57371512814135\\
-3.30249918624028	-1.39760356519063\\
-3.3114165882906	-1.20904465786337\\
-3.30814542905765	-1.00741168191251\\
-3.29096201643304	-0.792256187138982\\
-3.25797286635204	-0.563403515082209\\
-3.20717493015324	-0.321066389214163\\
-3.13655341623575	-0.0659702764881891\\
-3.04422258654089	0.200522420107314\\
-2.92860855717195	0.476310021275436\\
-2.7886626532819	0.758498504660572\\
-2.62408037933843	1.04342739569919\\
-2.43548813601527	1.32679813915937\\
-2.22455324917765	1.60391155894044\\
-1.99397861951217	1.86999526562393\\
-1.74736374460416	2.12057475325495\\
-1.48894469235357	2.35182434160235\\
-1.22325615775622	2.56083491987714\\
-0.954776947771264	2.74575579109426\\
-0.687619329080636	2.9057998932751\\
-0.425304744052455	3.0411326827088\\
-0.170642417405311	3.15268464004814\\
0.0742964523537774	3.24193210250864\\
0.308131369598009	3.31068389954316\\
0.530079690709056	3.36089818232796\\
0.739843429206208	3.39454053901598\\
0.937494593240998	3.41348430414743\\
1.12337085347325	3.41944784996473\\
};
\addplot [color=mycolor2, forget plot]
  table[row sep=crcr]{%
2.03026270353078	1.64575647587509\\
2.24201361397065	1.63929491819078\\
2.40988976796717	1.62346454151037\\
2.5439086404576	1.60217293817841\\
2.65192348194154	1.57789433711441\\
2.73991142031267	1.55215011587583\\
2.81236758836163	1.52585007471876\\
2.87266513834438	1.49951838258181\\
2.92334343033639	1.47343812077522\\
2.96632580056046	1.44774221252277\\
3.00307980332721	1.42247015209458\\
3.03473381366942	1.39760320969316\\
3.06216162049611	1.37308613323043\\
3.08604387326376	1.34884035107809\\
3.1069128492206	1.32477177949003\\
3.12518516010315	1.30077515426889\\
3.14118566099113	1.27673606856081\\
3.15516484960864	1.25253143538509\\
3.16731134706794	1.22802879750042\\
3.17776054716328	1.2030847121369\\
3.1866001477792	1.17754230256394\\
3.19387298775986	1.15122796591422\\
3.19957736951745	1.12394713932686\\
3.2036648216851	1.09547894128019\\
3.20603501902482	1.06556941121\\
3.2065272979622	1.03392295803928\\
3.20490784773303	1.00019148658358\\
3.20085116825551	0.963960488751789\\
3.19391369462968	0.924731153178388\\
3.18349649194873	0.88189725582094\\
3.16879247861024	0.834715253474649\\
3.14871154984905	0.782265661046414\\
3.12177402504919	0.723403600363448\\
3.08595887985921	0.65669674236961\\
3.03848848218452	0.580350616223418\\
2.97552755034422	0.492126403564453\\
2.89177588676445	0.389269007027596\\
2.77995604944656	0.268491300815408\\
2.63027155695706	0.126116785626424\\
2.43010358708565	-0.0414206862591693\\
2.16460932990471	-0.236405649579915\\
1.81944441184434	-0.457695773160881\\
1.38693674034512	-0.697839094321064\\
0.875038323293743	-0.941158665858655\\
0.313368717274706	-1.1659295814316\\
-0.252135552917663	-1.35184958146322\\
-0.776126199887151	-1.48819301966754\\
-1.22963821375558	-1.57605995670056\\
-1.60388412349005	-1.62425867750996\\
-1.9042365387354	-1.64371748521601\\
-2.1423340163231	-1.64401278743609\\
-2.33075772201044	-1.63226643854284\\
-2.48059210593731	-1.61331948699554\\
-2.60075362405107	-1.59029424094336\\
-2.69810967565792	-1.56513735762139\\
-2.77784760291292	-1.53902978813207\\
-2.84386080707416	-1.51266611354359\\
-2.89907416816592	-1.48643579068407\\
-2.94569574923572	-1.46053795843337\\
-2.98540399805985	-1.43505331283764\\
-3.01948454392525	-1.40998882547804\\
-3.04892951832121	-1.38530541457036\\
-3.07450962932375	-1.36093491127263\\
-3.09682658838307	-1.33679026387485\\
-3.11635136498169	-1.31277142156121\\
-3.13345215510356	-1.28876840528795\\
-3.14841479826139	-1.2646624896102\\
-3.16145755384025	-1.24032605003473\\
-3.17274155601308	-1.21562139138008\\
-3.18237783452297	-1.19039871137346\\
-3.19043146182922	-1.16449323731631\\
-3.19692312450592	-1.13772148055284\\
-3.20182818580579	-1.1098764686405\\
-3.20507307834332	-1.08072172691031\\
-3.20652861203835	-1.04998367925386\\
-3.20599946923685	-1.01734201226768\\
-3.20320874276366	-0.98241738645672\\
-3.19777579327309	-0.944755671769844\\
-3.18918487373995	-0.903807622805553\\
-3.17674076907669	-0.858902590472078\\
-3.15950595960833	-0.809214515661926\\
-3.13621132700539	-0.753718154746509\\
-3.10512897253988	-0.691133486978053\\
-3.06389128737052	-0.619857135842971\\
-3.00923570827716	-0.537882735252563\\
-2.9366524931571	-0.442720368567719\\
-2.83992194445915	-0.331344308054474\\
-2.71056965562503	-0.200238660869918\\
-2.53739135908002	-0.0456861773612729\\
-2.30648368509363	0.135441053243499\\
-2.00272025280522	0.344046356187461\\
-1.61408470738881	0.576195424583096\\
-1.13957746635919	0.820384013776223\\
-0.597773254972512	1.05727631209485\\
-0.0280177497212467	1.26470629064575\\
0.521625471572434	1.42639948348009\\
1.01267121040275	1.53774497708029\\
1.42656549795925	1.60443013244292\\
1.76263786922918	1.63692832306647\\
2.03026270353078	1.64575647587509\\
};
\addplot [color=mycolor1, forget plot]
  table[row sep=crcr]{%
1.01343528097655	3.63585201902066\\
1.19185337592755	3.63023677140222\\
1.36099770341136	3.61412819344363\\
1.52136225084341	3.58850545850642\\
1.67346151505693	3.55418471898578\\
1.81780658280804	3.51182884113409\\
1.95488752504785	3.4619585922676\\
2.08516054724753	3.40496396299583\\
2.20903855445196	3.34111480712293\\
2.32688401468736	3.27057034337028\\
2.43900320906891	3.19338731797567\\
2.54564112956951	3.10952680633696\\
2.64697642417426	3.0188597620073\\
2.74311589813743	2.92117152487511\\
2.83408816681036	2.81616559580879\\
2.9198361300294	2.70346708806976\\
3.00020801283658	2.58262638999081\\
3.07494680802332	2.45312373062445\\
3.14367808259371	2.31437553985711\\
3.20589629815401	2.16574374228541\\
3.26095007601892	2.00654941740021\\
3.30802724837493	1.83609258012509\\
3.3461411152807	1.6536801434531\\
3.37412010291755	1.45866433791852\\
3.39060399223019	1.25049384499662\\
3.39405099878186	1.02877944917842\\
3.38276106660261	0.793374855924018\\
3.35492146274524	0.544471159869154\\
3.30868060031537	0.28270004906654\\
3.24225428319321	0.00923621972990475\\
3.15406455666848	-0.274115814033056\\
3.04290471209723	-0.564870011927675\\
2.90811523997044	-0.859863697273204\\
2.74974649930657	-1.15533745256374\\
2.56867777004751	-1.44709857665007\\
2.36666285770699	-1.73076053886735\\
2.14628189544622	-2.00203209688494\\
1.91079657288267	-2.25701450242647\\
1.66392695151825	-2.49245974972126\\
1.40958524610206	-2.70595014654958\\
1.15160943172697	-2.89597743088331\\
0.893535292189428	-3.06192175638923\\
0.638432314978981	-3.20394967734436\\
0.388812268759267	-3.32286080230751\\
0.146604852015977	-3.41991411899592\\
-0.0868143097595534	-3.49665946835587\\
-0.310561319431775	-3.55479092968387\\
-0.524164133255402	-3.59603017651728\\
-0.727478342126641	-3.62204106531461\\
-0.920607476176883	-3.63437241270124\\
-1.10383281154742	-3.63442383470343\\
-1.2775546979489	-3.62342903307452\\
-1.44224545906707	-3.60245136822155\\
-1.59841278737773	-3.57238744537591\\
-1.74657203121279	-3.53397542842141\\
-1.8872256387729	-3.48780570303788\\
-2.02084811230786	-3.43433226123445\\
-2.14787501917248	-3.3738837591114\\
-2.26869483213253	-3.30667362660043\\
-2.38364258848127	-3.23280891210932\\
-2.49299454691819	-3.15229775837368\\
-2.59696317661036	-3.06505555737761\\
-2.69569193597136	-2.97090994650247\\
-2.7892493954591	-2.86960490543576\\
-2.87762233793308	-2.76080431079362\\
-2.96070754302555	-2.6440954174288\\
-3.03830204253425	-2.51899287473264\\
-3.11009173969587	-2.3849440637389\\
-3.17563843890095	-2.24133676433477\\
-3.23436556243217	-2.08751043345009\\
-3.28554317163611	-1.92277268588853\\
-3.32827339989067	-1.74642289138102\\
-3.36147807928184	-1.55778507495665\\
-3.38389122035015	-1.35625242644068\\
-3.3940600599788	-1.14134551988465\\
-3.39035951910219	-0.912785575902793\\
-3.37102586779912	-0.670582476113253\\
-3.33421575579759	-0.41513447650025\\
-3.27809591248885	-0.147332533287431\\
-3.20096602514468	0.131342905161217\\
-3.10141198482739	0.418749985897776\\
-2.97847885558869	0.712059402219527\\
-2.83184370894702	1.00779493438612\\
-2.6619604392095	1.30195494354937\\
-2.4701454483235	1.5902159480539\\
-2.25857794652784	1.86820131746095\\
-2.03020250578075	2.13178026081226\\
-1.78854156401486	2.37735127744792\\
-1.53744555744897	2.60206509794101\\
-1.28082125932745	2.8039553740791\\
-1.02238045525752	2.98196635995617\\
-0.765441767854556	3.13588812240772\\
-0.512802792854487	3.26622486838006\\
-0.266683626030244	3.37402778513509\\
-0.0287308580328474	3.46072126793759\\
0.199934997210314	3.52794382790228\\
0.418646483445373	3.57741593688789\\
0.627105777240416	3.61083916929073\\
0.825301884117939	3.62982544115539\\
1.01343528097654	3.63585201902066\\
};
\addplot [color=mycolor2, forget plot]
  table[row sep=crcr]{%
2.14516078266191	1.81183437647932\\
2.32810767202528	1.80624082044245\\
2.47498896724813	1.79237950465215\\
2.59392767073726	1.77347427500045\\
2.69121028702518	1.7516000183635\\
2.77162026450685	1.72806638687415\\
2.83877780008583	1.7036840880486\\
2.89542578857523	1.67894145181767\\
2.9436522231263	1.65411894328654\\
2.98505693235282	1.62936267187349\\
3.02087449597627	1.60473135783548\\
3.05206427187012	1.5802262386663\\
3.0793762728029	1.55580999860794\\
3.10339945256859	1.53141858573327\\
3.12459717608243	1.50696836088577\\
3.14333329276745	1.48236011877948\\
3.15989123768703	1.45748094377396\\
3.1744878629247	1.43220448859898\\
3.18728317556303	1.40639001567081\\
3.19838676732086	1.37988036802691\\
3.20786141780366	1.35249890818988\\
3.21572410207699	1.32404535679468\\
3.22194440374892	1.2942903628817\\
3.22644009903405	1.26296853170257\\
3.22906940570025	1.22976951211391\\
3.22961904749807	1.19432659237647\\
3.22778682211214	1.15620205798437\\
3.22315671266391	1.11486831569946\\
3.21516365575605	1.06968347569427\\
3.2030437427563	1.01985971386072\\
3.18576371422725	0.964422352420934\\
3.16192091548773	0.90215733690581\\
3.12960128344385	0.831545008064674\\
3.08617864417194	0.750679609819703\\
3.0280349472822	0.657178707510012\\
2.95018243305724	0.5480985257777\\
2.84578696967122	0.419897682430965\\
2.7056547387299	0.268545246798006\\
2.51790683589687	0.0899623815006545\\
2.26840203226445	-0.118893134541208\\
1.94296185496446	-0.357964425295355\\
1.53264287484255	-0.621143106950835\\
1.04185108331257	-0.893831953901765\\
0.494924609752802	-1.15403342101582\\
-0.066534518525441	-1.37894991663137\\
-0.597772602234644	-1.55378096771476\\
-1.06674192827408	-1.67591294678178\\
-1.46035076606654	-1.75222080015195\\
-1.78055759538858	-1.79347207767742\\
-2.03707648936046	-1.8100868393982\\
-2.24172853257791	-1.81033056486947\\
-2.40550246594102	-1.80010983010687\\
-2.53751880948657	-1.78340589518095\\
-2.64494597426942	-1.76281211984989\\
-2.73327449720316	-1.73998069484092\\
-2.80666650064496	-1.71594483032295\\
-2.8682718360826	-1.69133643129177\\
-2.92048189500717	-1.66652855324256\\
-2.96512276341988	-1.6417272257583\\
-3.00359856873776	-1.61703023471763\\
-3.03699671348769	-1.59246461859214\\
-3.0661648784366	-1.56801048741935\\
-3.09176740614943	-1.54361601670386\\
-3.11432667645284	-1.51920669087131\\
-3.13425352034404	-1.49469073780299\\
-3.15186955473061	-1.4699619740346\\
-3.16742347255512	-1.44490081626718\\
-3.18110270743932	-1.4193739110612\\
-3.19304143990479	-1.39323262886547\\
-3.20332557011713	-1.36631052109911\\
-3.21199500898494	-1.33841972365926\\
-3.21904340259761	-1.30934618874109\\
-3.22441517586908	-1.27884352534193\\
-3.22799953159585	-1.24662511534656\\
-3.22962073876998	-1.21235403508853\\
-3.22902364797926	-1.17563013964662\\
-3.2258528262045	-1.1359734462553\\
-3.21962292958437	-1.09280267290438\\
-3.20967681992079	-1.04540744492495\\
-3.1951263279035	-0.992912296007294\\
-3.17476828543654	-0.934230241279021\\
-3.14696530965509	-0.868003614028773\\
-3.10947680774259	-0.792530587011713\\
-3.05922140853847	-0.705678606652429\\
-2.99195001476793	-0.604793579562332\\
-2.90181625129863	-0.486631565879331\\
-2.78086674353476	-0.347377878770904\\
-2.61857755187232	-0.182890815216818\\
-2.40180375939587	0.0105804915569155\\
-2.11594162626837	0.234854610604258\\
-1.74855246596671	0.487240711876294\\
-1.29628392023436	0.757551460136748\\
-0.773087447884022	1.02700904971299\\
-0.213075259222931	1.27210066912647\\
0.338367057257413	1.47307141584576\\
0.841292482717871	1.62116207087161\\
1.27311059132485	1.71915089955838\\
1.62912389573152	1.77653729265007\\
1.91604217687101	1.80428448539298\\
2.14516078266191	1.81183437647932\\
};
\addplot [color=mycolor1, forget plot]
  table[row sep=crcr]{%
0.940529402382294	3.81671254728581\\
1.12221882568828	3.81098838439057\\
1.29553828688765	3.79447642475729\\
1.46088553715868	3.76805200798769\\
1.61867883079513	3.73244107004308\\
1.76933593247377	3.68822784529482\\
1.91325803670642	3.63586375726737\\
2.05081740222796	3.57567646725338\\
2.18234765030009	3.50787843672007\\
2.30813583155767	3.43257464778903\\
2.42841551097784	3.34976933939613\\
2.5433602463567	3.25937177792064\\
2.65307694136993	3.16120121065812\\
2.75759864377023	3.05499126627416\\
2.85687643993119	2.94039418355026\\
2.95077017923965	2.81698538133578\\
3.03903785950927	2.68426903942937\\
3.12132363527079	2.54168554995017\\
3.19714459664589	2.38862192459141\\
3.26587673431111	2.22442649953105\\
3.32674088588706	2.04842954703031\\
3.37878998011524	1.85997163851086\\
3.4208995765126	1.65844173226941\\
3.45176453222614	1.44332685920611\\
3.46990555407134	1.21427477949425\\
3.4736902636639	0.971169864968868\\
3.46137394331136	0.714220501283853\\
3.43116492062658	0.444053352368563\\
3.38131805752097	0.161805959730513\\
3.31025651689468	-0.130795159074461\\
3.21671664761649	-0.43138744706829\\
3.09990385796391	-0.736974296429673\\
2.95964010967504	-1.0439915601454\\
2.79647850519018	-1.3484462432788\\
2.61176009205239	-1.64612228743937\\
2.40759437810898	-1.93283329431207\\
2.18675786731902	-2.20468916462333\\
1.95252113303128	-2.45833775129871\\
1.70842956190959	-2.69114648965576\\
1.45807115625702	-2.90130167934605\\
1.20486434177597	-3.08782037619996\\
0.951890631527259	-3.25048609763232\\
0.701784809477116	-3.38973034605408\\
0.456683192328822	-3.50648564063056\\
0.218221550805198	-3.6020333187657\\
-0.0124304054884634	-3.67786331299615\\
-0.234511883913556	-3.73555603985411\\
-0.447603321986191	-3.77669032071695\\
-0.65155727176105	-3.8027767759739\\
-0.846433809638309	-3.81521349508384\\
-1.03244372230256	-3.81525964852368\\
-1.20990070786174	-3.80402259260859\\
-1.37918250494462	-3.78245448218111\\
-1.54070009742496	-3.75135511975611\\
-1.69487378229385	-3.71137852399006\\
-1.84211479056816	-3.66304138352832\\
-1.98281120528697	-3.6067321287988\\
-2.11731704942275	-3.54271979797231\\
-2.24594357092373	-3.47116220805626\\
-2.36895190407261	-3.392113189972\\
-2.48654642223561	-3.30552883098262\\
-2.59886821259585	-3.21127281082805\\
-2.70598820032301	-3.10912103854377\\
-2.80789953358164	-2.99876591164773\\
-2.90450892075347	-2.87982064227445\\
-2.99562669928613	-2.75182423785062\\
-3.08095552728283	-2.61424789649424\\
-3.16007774414687	-2.46650378520177\\
-3.23244166981062	-2.30795741139842\\
-3.29734743204461	-2.13794506389383\\
-3.35393335846368	-1.95579805791037\\
-3.40116456981079	-1.7608757129587\\
-3.43782617230095	-1.55260902490024\\
-3.4625243391138	-1.33055671819476\\
-3.47369949481505	-1.09447458398215\\
-3.4696565645729	-0.844397496185794\\
-3.4486174762433	-0.580731050508035\\
-3.40880032255423	-0.304346330489577\\
-3.3485272533785	-0.01666712280979\\
-3.26635885456223	0.280265268716509\\
-3.16124651288204	0.583767278568167\\
-3.03268692450557	0.890545754527654\\
-2.88085636606772	1.1968003018717\\
-2.70669926960498	1.49839709064195\\
-2.511948523168	1.79110154521217\\
-2.29906472086288	2.07084273240482\\
-2.0710966388664	2.33397236601865\\
-1.83148129387468	2.5774802038505\\
-1.58381386251731	2.79913625554401\\
-1.33162172825651	2.99754589709511\\
-1.07817232772824	3.17212143192531\\
-0.826333794313452	3.32298755848107\\
-0.578494768481339	3.45084547052904\\
-0.336538935920287	3.55682069245203\\
-0.101862995942506	3.6423151625029\\
0.124575853936526	3.7088772221073\\
0.342196161687663	3.75809635556959\\
0.550722883442389	3.79152412281218\\
0.750119970262207	3.81061919190203\\
0.940529402382293	3.81671254728581\\
};
\addplot [color=mycolor2, forget plot]
  table[row sep=crcr]{%
2.23886372154498	1.97054596626372\\
2.3971405473277	1.96569800633572\\
2.52565451777639	1.95356181822508\\
2.63101472075458	1.9368076774117\\
2.71829225436106	1.9171769431276\\
2.79134646283389	1.89579091537222\\
2.85311275745035	1.87336152224805\\
2.90583199922238	1.85033104357419\\
2.95122497346438	1.82696357625248\\
2.99062220768506	1.80340459198751\\
3.02505965735058	1.77971963312046\\
3.05534904636493	1.75591938705399\\
3.08212961471371	1.7319758194069\\
3.10590625968392	1.70783237231697\\
3.12707767721752	1.68341015198366\\
3.14595708030931	1.65861132931121\\
3.16278731704588	1.63332051994798\\
3.17775165866286	1.60740460463824\\
3.19098111743425	1.58071123920809\\
3.20255883808743	1.55306614754639\\
3.21252184778417	1.52426916483789\\
3.22086021804516	1.49408888349644\\
3.22751345957729	1.46225563596542\\
3.23236370895152	1.42845241350026\\
3.23522494030351	1.39230315482345\\
3.23582700102603	1.353357628782\\
3.23379266602203	1.31107186583288\\
3.22860504328548	1.26478275150053\\
3.21956142251217	1.21367497880448\\
3.20570787578924	1.15673809634543\\
3.18574640394614	1.09271099770805\\
3.15790302204014	1.02001118671122\\
3.11974100782698	0.93664727675023\\
3.06789961820395	0.840117152408607\\
2.99773845519081	0.727304690745195\\
2.90288113640783	0.594412126820719\\
2.77470249200837	0.437015500958592\\
2.60193895795226	0.250422457776838\\
2.37089293593623	0.0306421785226072\\
2.0671652834548	-0.22364673721209\\
1.68015815458778	-0.50804165113269\\
1.21061935927683	-0.809365687088379\\
0.677943225041865	-1.10554365855548\\
0.119994514024516	-1.37122039316945\\
-0.418603686536259	-1.58717161051174\\
-0.902562091961035	-1.74656914995138\\
-1.31454939143297	-1.85392677460347\\
-1.65327120545997	-1.91961872031168\\
-1.92668956910225	-1.95484601531965\\
-2.14600505565856	-1.96904561234524\\
-2.32220434948555	-1.96924705464741\\
-2.46466491470551	-1.96034786893389\\
-2.58087975605076	-1.94563555192347\\
-2.67664785983346	-1.92727015027268\\
-2.75639513382362	-1.9066511722555\\
-2.82348655693236	-1.88467393820216\\
-2.88048523384333	-1.86190155832319\\
-2.92935322944049	-1.83867809557897\\
-2.97160233631346	-1.81520240074319\\
-3.00840560076461	-1.79157612972125\\
-3.04067936767815	-1.76783490847551\\
-3.06914359976889	-1.74396847328966\\
-3.09436629393814	-1.71993353935428\\
-3.11679624346869	-1.6956618039765\\
-3.1367871984734	-1.67106462072325\\
-3.15461559476158	-1.64603531582131\\
-3.17049337645814	-1.62044974563793\\
-3.18457696299681	-1.59416544138832\\
-3.19697305294373	-1.56701950728727\\
-3.2077416738476	-1.53882530000431\\
-3.2168966457057	-1.50936779884364\\
-3.22440339703895	-1.47839746123413\\
-3.23017382918471	-1.44562223308809\\
-3.23405763531408	-1.4106972351741\\
-3.23582910716819	-1.37321146101976\\
-3.23516795230254	-1.33267058417853\\
-3.2316319243834	-1.28847466843858\\
-3.22461803537015	-1.23988919461343\\
-3.21330762930232	-1.18600737204386\\
-3.1965884725218	-1.12570125453857\\
-3.17294406804385	-1.05755892086571\\
-3.14029656852784	-0.979805398423894\\
-3.09578538824318	-0.890207250560821\\
-3.03546088218197	-0.785967333430171\\
-2.95387728181653	-0.663632354103263\\
-2.84359689196043	-0.5190712343952\\
-2.6947027328728	-0.347651706824467\\
-2.49462112019136	-0.144857159474346\\
-2.22894202869952	0.092288706523193\\
-1.88439266240653	0.362673847196198\\
-1.45498408819297	0.657793680364402\\
-0.950154585058804	0.959712068055632\\
-0.399297508665178	1.24364687671789\\
0.154306540059556	1.48615161318593\\
0.668936484537602	1.67386686455298\\
1.1179353871564	1.80617203399682\\
1.49270534993371	1.89125774695739\\
1.79746627586354	1.9403949720658\\
2.042394554795	1.96407939077264\\
2.23886372154498	1.97054596626372\\
};
\addplot [color=mycolor1, forget plot]
  table[row sep=crcr]{%
0.896441690959812	3.9441923617009\\
1.08051075693992	3.93838963927558\\
1.2567522619063	3.92159579500097\\
1.42551554186611	3.89462204154981\\
1.58716888950602	3.85813663492713\\
1.74208022106361	3.81267167666549\\
1.89060191874816	3.758630911541\\
2.03305881683847	3.69629765064671\\
2.16973841785594	3.62584228168578\\
2.30088255142231	3.54732908093015\\
2.42667980715507	3.46072223444294\\
2.54725817812176	3.36589112949966\\
2.66267744312509	3.26261510845942\\
2.77292089898392	3.15058800222867\\
2.87788613609048	3.02942289234418\\
2.97737464328587	2.8986577007422\\
3.07108014656033	2.75776238280855\\
3.15857574937347	2.60614870633934\\
3.23930017381126	2.44318383344733\\
3.31254372793733	2.26820916924529\\
3.3774350723429	2.08056616661775\\
3.43293044815691	1.87963091917116\\
3.47780776033545	1.66485933575901\\
3.51066874471785	1.4358443267926\\
3.52995327764365	1.19238556079154\\
3.53397050503074	0.934570769130519\\
3.520951542599	0.662865133062121\\
3.48912758734578	0.378201980646154\\
3.43683490925928	0.0820641606591533\\
3.36264404373427	-0.22345818841623\\
3.26550470313232	-0.53564992420845\\
3.14489133868984	-0.851208802312372\\
3.00092865089166	-1.16634927276106\\
2.83447398519056	-1.47697249275014\\
2.64713652401332	-1.77888975621129\\
2.4412221560901	-2.06807363835451\\
2.21960628443296	-2.34090274714581\\
1.98555095319482	-2.59436541308887\\
1.7424931426409	-2.82619562441075\\
1.49383473629253	-3.03492853999866\\
1.24276094183342	-3.21987843739294\\
0.99210489703689	-3.38105444633402\\
0.744265247213911	-3.51903616312019\\
0.501173856815033	-3.63483195176959\\
0.264304426877602	-3.72973898211346\\
0.0347100233669113	-3.80521812041878\\
-0.186922315934679	-3.86279069643553\\
-0.400209775873058	-3.90395918544841\\
-0.605011972928268	-3.93015044620767\\
-0.801373771193119	-3.9426782744144\\
-0.989475129968325	-3.94272132193596\\
-1.16958882528395	-3.93131248815339\\
-1.34204585640571	-3.90933636257011\\
-1.50720775825661	-3.87753193646439\\
-1.66544476555644	-3.83649845119894\\
-1.81711870285525	-3.78670283131044\\
-1.96256952083242	-3.72848763083526\\
-2.10210450369508	-3.66207879984237\\
-2.23598929741258	-3.58759286802539\\
-2.36444003225977	-3.50504336254576\\
-2.48761592541276	-3.41434644820779\\
-2.60561184747482	-3.31532591828493\\
-2.71845042333053	-3.20771779060812\\
-2.82607331901439	-3.09117489037264\\
-2.92833145207355	-2.96527194084203\\
-3.02497396639078	-2.82951184559114\\
-3.11563595065332	-2.68333403757056\\
-3.19982507385023	-2.52612599199289\\
-3.2769075865186	-2.35723924349641\\
-3.34609452020288	-2.17601148989445\\
-3.40642943436212	-1.98179655873834\\
-3.45677972219486	-1.77400407998747\\
-3.49583427838497	-1.55215052784587\\
-3.52211118323165	-1.31592270230808\\
-3.53397981522779	-1.06525351898234\\
-3.52970219987871	-0.800407976357051\\
-3.50749804443063	-0.522074279067044\\
-3.46563632196283	-0.231451448007957\\
-3.4025530281778	0.0696791348611318\\
-3.31698970315709	0.378913893937647\\
-3.20814095193239	0.693235425230204\\
-3.07579282076281	1.00908258119744\\
-2.92042958963605	1.32248811271824\\
-2.74328665053001	1.62927785135783\\
-2.54633316001776	1.9253118886953\\
-2.33217967750365	2.20673703088098\\
-2.10392029761606	2.4702150397221\\
-1.86493159586839	2.71309499905096\\
-1.61865801975323	2.93350962852924\\
-1.36841321339716	3.13039076137595\\
-1.1172199965142	3.30341369522077\\
-0.867701270448185	3.45288993022916\\
-0.622023503989817	3.5796314323213\\
-0.381886304879468	3.68480776815644\\
-0.148547037878136	3.76981232605801\\
0.0771317224625102	3.83614761341639\\
0.294622911237203	3.88533397713145\\
0.503672444507914	3.91884188328714\\
0.704239306475423	3.93804528486142\\
0.896441690959811	3.9441923617009\\
};
\addplot [color=mycolor2, forget plot]
  table[row sep=crcr]{%
2.31219753178786	2.11733153574279\\
2.44922341097804	2.11312775477941\\
2.56158996350191	2.10251021132567\\
2.65470000455284	2.08769859129554\\
2.73267510535881	2.07015547350546\\
2.79865320194544	2.05083686318459\\
2.85503036596741	2.0303609255361\\
2.90364646842595	2.00911982122122\\
2.94592401264342	1.98735336420384\\
2.98297073132288	1.96519745049244\\
3.01565511133232	1.94271587522494\\
3.04466200341872	1.9199211706665\\
3.07053364776049	1.89678811750632\\
3.09369999080958	1.8732622887459\\
3.11450107276432	1.84926514335361\\
3.13320345912556	1.8246966344936\\
3.15001210013918	1.79943592837321\\
3.16507856566243	1.77334057550395\\
3.17850627055307	1.74624429114838\\
3.19035303839031	1.71795335539022\\
3.20063111778286	1.68824151382186\\
3.20930453781797	1.65684313019624\\
3.21628343921874	1.62344419736214\\
3.22141471297521	1.58767063712821\\
3.22446787669025	1.54907309668509\\
3.22511456290194	1.50710716029286\\
3.22289920129302	1.4611075212188\\
3.21719733181251	1.41025418735661\\
3.20715632890983	1.35352823453838\\
3.19161094778317	1.28965404943067\\
3.16896282325786	1.21702465758599\\
3.13700882893513	1.13360723270606\\
3.09269866370287	1.03682870343218\\
3.03179975076724	0.923449784702938\\
2.94845514644759	0.789456527236652\\
2.83465694514752	0.630044135083091\\
2.67976544666188	0.439856708125415\\
2.470453671707	0.213788759349444\\
2.19190030976773	-0.0512092357059949\\
1.83149757056718	-0.353020390318954\\
1.38586064895721	-0.680636258290168\\
0.868979301433518	-1.01254253787914\\
0.314775059181264	-1.32092149657512\\
-0.231992170738203	-1.58148225610957\\
-0.732142184200613	-1.78216604951326\\
-1.16346924919793	-1.92431373598648\\
-1.52110294016983	-2.01754639030242\\
-1.81123723337378	-2.07382755476637\\
-2.04459157333144	-2.10389266620177\\
-2.23231835496362	-2.11604172030406\\
-2.38418957407001	-2.11620865195714\\
-2.50811759304301	-2.10846060918443\\
-2.61026815515519	-2.09552290855396\\
-2.69536337117685	-2.07919915628878\\
-2.76699906822492	-2.06067308728935\\
-2.82791570154822	-2.04071476579902\\
-2.88021121612825	-2.01981812605767\\
-2.92550194722388	-1.99829177037317\\
-2.96504207156239	-1.97631869958058\\
-2.99981164768789	-1.95399557306985\\
-3.03058141070106	-1.9313584754723\\
-3.05796052354452	-1.90839972850473\\
-3.08243183998244	-1.88507868464425\\
-3.10437796560589	-1.86132839610284\\
-3.12410046178291	-1.83705937156592\\
-3.14183384840193	-1.81216118317854\\
-3.15775555529696	-1.786502381649\\
-3.17199259332608	-1.75992896227229\\
-3.18462542050323	-1.73226146215457\\
-3.19568923178027	-1.70329063338424\\
-3.20517267385935	-1.67277150931253\\
-3.21301375125914	-1.64041554563228\\
-3.21909241698553	-1.6058803595666\\
-3.22321899405013	-1.5687563933435\\
-3.22511710383538	-1.52854957457582\\
-3.22439911527329	-1.48465871706912\\
-3.22053117725034	-1.43634598373198\\
-3.21278351860096	-1.38269821389076\\
-3.20015971378189	-1.32257633521977\\
-3.18129580814329	-1.25454957478905\\
-3.15431642558814	-1.17681114515829\\
-3.11663044452128	-1.08707347508828\\
-3.06464488273198	-0.982446092645308\\
-2.99337682270279	-0.859312703859403\\
-2.89596188345167	-0.713255221179613\\
-2.76312292999702	-0.539137347331856\\
-2.5828311924434	-0.331576630228325\\
-2.34073848640261	-0.0861895684435461\\
-2.02245974210641	0.197953775757708\\
-1.61894476666473	0.514714918158474\\
-1.1346272030851	0.847742484176509\\
-0.593813487202299	1.17140343647957\\
-0.037695774543649	1.45827588621514\\
0.489734952359442	1.68949633550006\\
0.95705966773936	1.86007193292024\\
1.35128361218501	1.97629591348342\\
1.67398237851932	2.04958304062953\\
1.93429493771153	2.0915584657179\\
2.14350538852695	2.1117855131171\\
2.31219753178786	2.11733153574279\\
};
\addplot [color=mycolor1, forget plot]
  table[row sep=crcr]{%
0.8754575855778	4.0104177677322\\
1.06078896386434	4.0045735522546\\
1.23854922283576	3.98763331937388\\
1.40906731527161	3.96037745627087\\
1.57269006917172	3.92344595010289\\
1.72976359699493	3.87734483702673\\
1.88061855825207	3.82245355026012\\
2.02555831002059	3.75903236969278\\
2.16484908783362	3.68722948504239\\
2.29871147429479	3.60708742245828\\
2.42731252133389	3.51854876960353\\
2.55075798927877	3.42146128496499\\
2.66908425269244	3.31558261015845\\
2.78224950425876	3.20058493399301\\
2.89012397248969	3.07606009659427\\
2.99247896873398	2.94152578111821\\
3.08897470971137	2.79643362639668\\
3.17914704345333	2.64018030796597\\
3.26239346283231	2.47212287057045\\
3.33795914786553	2.2915998319141\\
3.40492425999905	2.0979597736456\\
3.46219433283419	1.89059921987429\\
3.50849635386469	1.66901146600368\\
3.54238395366623	1.43284750920726\\
3.56225588019801	1.18198916153694\\
3.56639239984929	0.916632614519469\\
3.55301407953746	0.637378062993985\\
3.52036612351583	0.34531757374038\\
3.46682864144034	0.042109632465084\\
3.39104871394099	-0.269974385645804\\
3.29208421282323	-0.588047866480912\\
3.16954310200086	-0.908664677082647\\
3.02369724810034	-1.22794039218397\\
2.85554881414298	-1.5417357521964\\
2.66683175586405	-1.84588617090004\\
2.45994085201191	-2.13644959609351\\
2.23779398117415	-2.40993886934328\\
2.00364623041742	-2.66350649919343\\
1.76088292059494	-2.8950592542108\\
1.51282035604323	-3.10329417726252\\
1.26253818584371	-3.28766205539108\\
1.0127580451136	-3.44827510923211\\
0.765772890696933	-3.58578070169582\\
0.523422924806344	-3.70122243120848\\
0.28710867212055	-3.79590578646691\\
0.0578297533518808	-3.87127974547553\\
-0.16376158403674	-3.92884006165661\\
-0.377301125842935	-3.97005551774127\\
-0.582653973083151	-3.99631547460025\\
-0.779860478794216	-4.00889546709449\\
-0.969088921892505	-4.00893706080468\\
-1.15059559138525	-3.99743830836119\\
-1.32469204290745	-3.97525162052005\\
-1.49171877168303	-3.94308647587504\\
-1.65202430719974	-3.90151500087944\\
-1.80594867567897	-3.85097899069579\\
-1.95381022095887	-3.79179738648444\\
-2.09589487016749	-3.72417357645836\\
-2.23244704428052	-3.64820215943935\\
-2.36366152645962	-3.56387501880103\\
-2.48967570419104	-3.47108672039994\\
-2.61056169286983	-3.36963938799942\\
-2.72631793166803	-3.25924733949355\\
-2.83685992415061	-3.13954190057296\\
-2.94200988646712	-3.0100769607635\\
-3.04148517894587	-2.87033600869235\\
-3.13488555037737	-2.71974158361321\\
-3.22167944008226	-2.5576683066545\\
-3.30118988622717	-2.38346089509554\\
-3.37258100569672	-2.19645878626191\\
-3.43484656180295	-1.99602914905429\\
-3.48680282540288	-1.78161005091959\\
-3.52708873210587	-1.5527652451021\\
-3.5541771522628	-1.3092512763151\\
-3.56640173996894	-1.05109618213473\\
-3.56200401681776	-0.778686835306311\\
-3.53920466986621	-0.492858901596589\\
-3.49630104919356	-0.194979719730555\\
-3.43178919645336	0.112989202526469\\
-3.3445034434412	0.428465808248053\\
-3.2337603623416	0.748272731161792\\
-3.09948811511522	1.06872542460145\\
-2.94231915600336	1.38578570460051\\
-2.76362585630468	1.69527105654625\\
-2.56548592378343	1.99309733314492\\
-2.35057647791853	2.27552318459625\\
-2.12200924627566	2.53936219982267\\
-1.88313046792046	2.78213458674895\\
-1.63731435100178	3.00214257596618\\
-1.38777715533139	3.19846861629568\\
-1.13743151443992	3.37090840357856\\
-0.888790469792121	3.51985876913608\\
-0.643921028663922	3.64618260939846\\
-0.404440041536871	3.75107046116877\\
-0.171541574434479	3.83591307758739\\
0.0539556738293125	3.90219346612614\\
0.271550971989046	3.95140171852103\\
0.481001881327635	3.98497224848479\\
0.682267495359916	4.00424082343988\\
0.8754575855778	4.0104177677322\\
};
\addplot [color=mycolor2, forget plot]
  table[row sep=crcr]{%
2.36607513206161	2.24876922711939\\
2.48479706281508	2.2451218328169\\
2.58299896795016	2.23583800156676\\
2.66512219853627	2.22276998478633\\
2.73454184645255	2.2071481048122\\
2.79382894849366	2.18978550625748\\
2.84495288427685	2.17121476259944\\
2.88943286094029	2.15177833760319\\
2.92844974236369	2.13168839648718\\
2.96292821583002	2.11106637597001\\
2.99359717000535	2.08996915444847\\
3.0210341646257	2.06840627620922\\
3.04569826787349	2.04635111826338\\
3.06795432697702	2.02374786766238\\
3.088090849694	2.00051550832921\\
3.10633302831232	1.97654957172304\\
3.12285196427214	1.9517221016921\\
3.13777079597541	1.92588006614125\\
3.15116815280601	1.89884228016008\\
3.16307912154825	1.87039476134352\\
3.17349368874399	1.84028429845796\\
3.18235238645018	1.80820986219087\\
3.1895385885222	1.77381130435142\\
3.1948665410602	1.73665456055376\\
3.19806371075895	1.69621226889143\\
3.19874532252459	1.651838317919\\
3.1963779229547	1.60273431584277\\
3.19022729282418	1.54790531486466\\
3.17928382873872	1.48610135786744\\
3.16215537235939	1.41574068525492\\
3.13691319543502	1.33481019147983\\
3.1008716536058	1.24074009733246\\
3.0502774481935	1.13025550889201\\
2.97988581083231	0.999223520956103\\
2.88242297065914	0.842552786578786\\
2.74801189806255	0.654283612795641\\
2.56383956356074	0.42815369958251\\
2.31475351698567	0.159119005239815\\
1.98603619888372	-0.153648071589967\\
1.56964637155804	-0.502450673511533\\
1.07307920943061	-0.867685243885396\\
0.525045715892561	-1.21982070756272\\
-0.0300686417610128	-1.52892864853262\\
-0.54848079440757	-1.77614915002699\\
-1.00181111163593	-1.95815382094919\\
-1.38061631647323	-2.08304703639134\\
-1.68892329885739	-2.16344268568353\\
-1.93698816546921	-2.21156852013282\\
-2.13631241828859	-2.23724727348235\\
-2.29726209591901	-2.24765874981433\\
-2.42832874967536	-2.24779758371498\\
-2.53615694693746	-2.24105114662836\\
-2.62583786134551	-2.22968834031408\\
-2.70124308342754	-2.21521955092914\\
-2.7653167740668	-2.19864577489946\\
-2.82030728836154	-2.18062611348689\\
-2.86794271978042	-2.16158898882639\\
-2.90956118911193	-2.14180569951998\\
-2.94620672239702	-2.12143906531823\\
-2.97869967160048	-2.10057561443679\\
-3.00768851348651	-2.07924683740924\\
-3.03368805283886	-2.0574430954663\\
-3.05710765533064	-2.03512250741799\\
-3.07827209651661	-2.01221631374512\\
-3.09743685605133	-1.98863167260534\\
-3.1147991341797	-1.96425247643176\\
-3.13050545896176	-1.93893852243978\\
-3.14465643976465	-1.91252318125205\\
-3.157308968138	-1.88480955464968\\
-3.16847594086442	-1.85556497395101\\
-3.17812335416495	-1.8245135466138\\
-3.18616436406186	-1.79132629348262\\
-3.19244959157946	-1.7556082148057\\
-3.19675252762576	-1.71688135921992\\
-3.19874829759734	-1.67456262237005\\
-3.19798318852437	-1.62793454443151\\
-3.19383109034958	-1.57610678635818\\
-3.18543117438541	-1.51796524405418\\
-3.17159849099963	-1.45210498074569\\
-3.15069547410926	-1.37674258308107\\
-3.12044753174917	-1.28960389793679\\
-3.07768067158824	-1.18778615883497\\
-3.01795644245956	-1.06760331224818\\
-2.93508840278829	-0.924448472815183\\
-2.8205679718414	-0.752764040392522\\
-2.66305620321969	-0.546322006812653\\
-2.44839501627363	-0.299196532781962\\
-2.16110678567916	-0.00797435955711197\\
-1.78879993103978	0.324476345462676\\
-1.33009207596989	0.684705987266555\\
-0.802928758313889	1.04740134737392\\
-0.24540594588675	1.38129334496077\\
0.296099945092465	1.66082990681729\\
0.784276385241216	1.8749810740011\\
1.20050216946353	2.0269837269649\\
1.54301420737431	2.12799812665259\\
1.81975778976501	2.19086075125827\\
2.04205424924148	2.22670699051684\\
2.22100759983092	2.24400500522965\\
2.36607513206161	2.24876922711939\\
};
\addplot [color=mycolor1, forget plot]
  table[row sep=crcr]{%
0.873456719862334	4.01693373145583\\
1.05891327869839	4.01108540720015\\
1.23682310230941	3.9941307624143\\
1.40751329045177	3.96684723559204\\
1.5713287044198	3.92987209111072\\
1.72861344520434	3.88370883688959\\
1.8796961552078	3.8287345317985\\
2.02487818403272	3.76520719212756\\
2.16442376584047	3.69327281387803\\
2.29855146967883	3.61297176422278\\
2.42742629184089	3.52424447990469\\
2.55115185588764	3.42693656085289\\
2.66976227232088	3.32080348051212\\
2.78321329113623	3.20551526489463\\
2.89137246526386	3.08066163265594\\
2.99400814334819	2.94575824861939\\
3.09077724232353	2.80025492986278\\
3.18121193375922	2.64354685821892\\
3.26470563673946	2.47499008868385\\
3.34049907010301	2.29392287872273\\
3.40766760223331	2.09969455651876\\
3.46511176086235	1.89170372440407\\
3.51155351700004	1.66944744605708\\
3.54554177723419	1.4325825396255\\
3.56547127237876	1.18099900892871\\
3.56961947845612	0.914903810388281\\
3.55620599223771	0.634910472204568\\
3.52347746660291	0.342126651254164\\
3.46981837323584	0.0382279755197088\\
3.39388331735628	-0.274496619328825\\
3.29474071424938	-0.593144051856046\\
3.1720114463944	-0.914254534674148\\
3.02598151755879	-1.23393444664973\\
2.85766690030261	-1.54804101814983\\
2.66881335507843	-1.85241233716958\\
2.46182399016865	-2.14311481241083\\
2.23962059159806	-2.4166742889777\\
2.00545749895331	-2.67025899425647\\
1.76271511630104	-2.90179211307992\\
1.5147016951166	-3.10998598831279\\
1.26448700290808	-3.29430426487939\\
1.01478226487975	-3.45486885847563\\
0.767870580179866	-3.59233351054688\\
0.525583595809768	-3.70774515597869\\
0.289314989276039	-3.8024101063091\\
0.0600593545353311	-3.87777627277717\\
-0.161534365867956	-3.9353370560545\\
-0.375103765549384	-3.97655811533715\\
-0.580514536320754	-4.00282531612021\\
-0.777806689788507	-4.01541060842645\\
-0.967147489666844	-4.01545206439023\\
-1.1487917526378	-4.00394443509903\\
-1.32304927507435	-3.9817370633439\\
-1.49025863160819	-3.949536594988\\
-1.65076635603593	-3.90791253566546\\
-1.8049104570209	-3.85730423461502\\
-1.95300726548201	-3.79802831936665\\
-2.09534070557109	-3.73028595429559\\
-2.232153193876	-3.65416956580092\\
-2.36363748332244	-3.56966888501637\\
-2.48992887056574	-3.47667632423253\\
-2.6110972767274	-3.37499184311909\\
-2.72713879426336	-3.26432759100626\\
-2.8379663744662	-3.14431274546656\\
-2.94339942097207	-3.01449911656582\\
-3.04315216863044	-2.87436825898505\\
-3.13682088210797	-2.72334103518177\\
-3.22387012658132	-2.56079079956444\\
-3.30361866903396	-2.38606161298636\\
-3.37522598876934	-2.19849311809292\\
-3.43768093001322	-1.99745385297501\\
-3.489794721229	-1.78238476237182\\
-3.53020138305169	-1.55285435020383\\
-3.55736935648446	-1.30862613309598\\
-3.56962882114418	-1.04973761227755\\
-3.5652193417877	-0.776587727073905\\
-3.54236177337171	-0.490026666871447\\
-3.49935632124874	-0.191438251483621\\
-3.43470496054046	0.117198492103618\\
-3.34725110462667	0.43328428570102\\
-3.23632317040156	0.753626484098471\\
-3.10186302120267	1.07452892770211\\
-2.94451729347739	1.39194696563502\\
-2.76567137060142	1.70169765748539\\
-2.56741318493844	1.99970253001784\\
-2.35242605060152	2.28223116587682\\
-2.12382325714924	2.54611176537777\\
-1.88494811092087	2.78888085186346\\
-1.63916818199947	3.00885671507353\\
-1.38969059392881	3.20513601141079\\
-1.13941768184485	3.37752576543612\\
-0.890852237628319	3.52643083375097\\
-0.646051990474897	3.65271891601133\\
-0.406626059034493	3.75758255335449\\
-0.173762573571543	3.84241229809124\\
0.0517239108538937	3.90868937694286\\
0.269335186361097	3.95790108610335\\
0.478829977292244	3.99147848747363\\
0.680167467369668	4.01075377987807\\
0.873456719862334	4.01693373145583\\
};
\addplot [color=mycolor2, forget plot]
  table[row sep=crcr]{%
2.40210553937661	2.3630934927815\\
2.50510161403136	2.3599252946844\\
2.59093986907906	2.35180674191589\\
2.66329357058539	2.34029018666113\\
2.72494730336123	2.32641316944438\\
2.77802380521816	2.31086697042658\\
2.82415333884121	2.29410829369208\\
2.86459819850988	2.2764331562321\\
2.90034375025032	2.25802585522687\\
2.93216501324467	2.2389914814089\\
2.96067549452266	2.21937748795096\\
2.98636313825409	2.19918788625439\\
3.00961686072481	2.178392380323\\
3.03074612968263	2.15693193032603\\
3.04999531717496	2.13472169426589\\
3.06755402708116	2.11165193088673\\
3.08356420926532	2.08758718990546\\
3.0981245740805	2.06236392291898\\
3.11129257730319	2.03578648931227\\
3.12308402653721	2.00762138329789\\
3.13347013824599	1.9775893515926\\
3.1423716207288	1.94535488741487\\
3.1496490370355	1.91051235464126\\
3.1550882650705	1.87256769100378\\
3.15837925131487	1.83091423017277\\
3.15908534884755	1.78480063269962\\
3.15659919152362	1.73328818764031\\
3.15007907223557	1.67519382034874\\
3.13835688076933	1.60901406613714\\
3.11980449825242	1.53282429113761\\
3.09213997322834	1.44414731633822\\
3.05214840971314	1.33978830226637\\
2.99528834615863	1.21564299799959\\
2.915162391427	1.0665146585686\\
2.80287740920904	0.886041612979106\\
2.64646409737929	0.666974814346141\\
2.43087006477885	0.40227371715175\\
2.13964864643347	0.0877097586390446\\
1.75998295128699	-0.273604994523119\\
1.29162468022708	-0.666084542994651\\
0.755735709369115	-1.06045141329421\\
0.194381811289358	-1.42137835431777\\
-0.343993528887985	-1.72136328187118\\
-0.8230959741329	-1.94997123111311\\
-1.22706646442504	-2.11223255958094\\
-1.55674573595149	-2.22096273838767\\
-1.82167218307478	-2.29005839159569\\
-2.03382776263391	-2.33121944287373\\
-2.20441075259616	-2.35319302273665\\
-2.3427115930779	-2.36213552675717\\
-2.45602015570107	-2.36225150184665\\
-2.54991195846396	-2.35637325432495\\
-2.62861039286575	-2.34639861192798\\
-2.69531198456748	-2.33359694759486\\
-2.75244592225422	-2.3188156891715\\
-2.80187063105124	-2.30261759563426\\
-2.84501915808969	-2.28537159093286\\
-2.88300566451951	-2.26731289744304\\
-2.91670326267435	-2.24858293136569\\
-2.94680101222194	-2.22925579567656\\
-2.97384580138051	-2.20935580740452\\
-2.99827322614566	-2.18886893315524\\
-3.02043039130044	-2.16774999072742\\
-3.04059269726404	-2.14592680917192\\
-3.05897605708338	-2.12330209619639\\
-3.07574553608595	-2.09975345711912\\
-3.09102106812584	-2.07513178920378\\
-3.10488063548612	-2.04925810255761\\
-3.11736107164653	-2.0219186677147\\
-3.12845642910063	-1.99285823992896\\
-3.13811362046281	-1.96177094239724\\
-3.14622475857645	-1.92828818562114\\
-3.15261524936247	-1.89196273519221\\
-3.15702617264633	-1.85224768755824\\
-3.15908873808585	-1.80846863872752\\
-3.15828750318021	-1.75978669611459\\
-3.15390741000263	-1.70514915648678\\
-3.14495728852243	-1.64322365788887\\
-3.13005897651501	-1.57231053418962\\
-3.10728633676627	-1.49022740200106\\
-3.07393230130358	-1.39416092882715\\
-3.02617613796275	-1.28048634955713\\
-2.9586230951157	-1.14457270076291\\
-2.86371092077442	-0.980635813944171\\
-2.73106306753107	-0.781798681809504\\
-2.54709840982175	-0.540702677403615\\
-2.29568726130443	-0.251266722847516\\
-1.96130476006145	0.0877368265485396\\
-1.53613415024095	0.467498290295421\\
-1.02987232282991	0.865253094815708\\
-0.475123546990699	1.24715643189577\\
0.0803998746625715	1.58007232845334\\
0.592422884350118	1.84455614988675\\
1.03471808848556	2.03868190768345\\
1.40071126286749	2.17239124480139\\
1.69656031103102	2.25966486586898\\
1.93360764642683	2.31351636318777\\
2.12368640087881	2.34416651870353\\
2.27709645851434	2.35899206933955\\
2.40210553937661	2.3630934927815\\
};
\addplot [color=mycolor1, forget plot]
  table[row sep=crcr]{%
0.88748083090382	3.97200858962911\\
1.0720778919661	3.96618849790435\\
1.24895705497356	3.94933318180974\\
1.41845873996219	3.92224071305184\\
1.58094188710687	3.88556733999441\\
1.73676494485547	3.83983413351367\\
1.88627089384102	3.78543458741033\\
2.02977530398858	3.72264233455396\\
2.16755653586359	3.65161846206454\\
2.29984731849397	3.57241815496482\\
2.42682705012332	3.48499658733104\\
2.54861426999769	3.38921413209142\\
2.66525883878065	3.28484109258449\\
2.77673344728929	3.17156228599146\\
2.88292415622928	3.04898194384937\\
2.98361976516508	2.91662954878138\\
3.07849993246268	2.77396740708429\\
3.16712213882791	2.62040096697476\\
3.24890782883676	2.45529312748884\\
3.32312840391552	2.27798402600529\\
3.38889220239087	2.08781800617625\\
3.44513420522414	1.88417958710035\\
3.49061094602754	1.66654017533366\\
3.52390393434153	1.43451683673792\\
3.5434357040016	1.18794349033633\\
3.54750315362382	0.926953207658468\\
3.53433281244008	0.652067762950401\\
3.50216159819999	0.364287218810285\\
3.44934408098872	0.0651685085542933\\
3.37448295939045	-0.243121487807968\\
3.27657359760087	-0.557794681664479\\
3.15514702279179	-0.875487236086551\\
3.01039053558596	-1.19237084085232\\
2.84322332138081	-1.50432859235038\\
2.65530806361779	-1.80718120739027\\
2.44898895673042	-2.09693696378953\\
2.22715988255978	-2.37003130859836\\
1.9930801181445	-2.62352252760346\\
1.75016456792847	-2.85521852822322\\
1.50177832216072	-3.06372391945442\\
1.25106108174094	-3.24841164825857\\
1.00079784722329	-3.40933518809728\\
0.753341613742977	-3.54710327268561\\
0.510584667334669	-3.6627393695639\\
0.273969153648827	-3.75754413326148\\
0.0445251531774694	-3.83297319760561\\
-0.177075181045537	-3.89053676919391\\
-0.390456895440626	-3.93172272686753\\
-0.595481787898095	-3.95794172913841\\
-0.792192586781656	-3.97049108501902\\
-0.9807640315538	-3.97053350968948\\
-1.16146162361766	-3.95908697293618\\
-1.3346078374386	-3.93702232266239\\
-1.50055502356405	-3.90506599125295\\
-1.6596639758734	-3.86380572390053\\
-1.81228706908935	-3.81369783019103\\
-1.95875491793273	-3.7550749251003\\
-2.09936560985006	-3.68815349230441\\
-2.23437568316647	-3.61304088447291\\
-2.36399214139935	-3.52974159061903\\
-2.48836490267482	-3.43816276914898\\
-2.60757917840563	-3.338119185261\\
-2.72164736035928	-3.22933781906081\\
-2.83050007654208	-3.11146254031263\\
-2.93397616391571	-2.98405938907472\\
-3.03181141332334	-2.84662316795648\\
-3.12362608650336	-2.69858624707172\\
-3.20891140823143	-2.53933070666704\\
-3.28701552366608	-2.36820518468506\\
-3.35712980862818	-2.18454803109935\\
-3.41827695185381	-1.98771854765969\\
-3.46930290230872	-1.77713812720729\\
-3.50887556937882	-1.55234287588744\\
-3.53549400081329	-1.31304863658416\\
-3.54751247725296	-1.05922803781881\\
-3.54318427306414	-0.791197093373195\\
-3.52072934240834	-0.509705907386554\\
-3.47842843077109	-0.216024376266373\\
-3.41474269343404	0.0879899996174577\\
-3.32845275249314	0.399857903219787\\
-3.2188047962198	0.71649368801053\\
-3.08564520760163	1.03428315833578\\
-2.92952141661873	1.34922808352798\\
-2.7517274258466	1.65714986611883\\
-2.55427904466183	1.95393157086205\\
-2.3398156170525	2.23576713284225\\
-2.11143905370944	2.49938282954807\\
-1.87251307579609	2.74220082153557\\
-1.62645200166476	2.96242645934868\\
-1.37652754064401	3.15905625761151\\
-1.12571497113936	3.33181728725355\\
-0.876589749548109	3.48105775263239\\
-0.631275386710711	3.60761149658494\\
-0.391435775371032	3.71265703174771\\
-0.158301014886573	3.79758650411687\\
0.0672851495639999	3.86389391027349\\
0.284806868002934	3.91308646802211\\
0.494014904291655	3.94661904697569\\
0.694868100710875	3.96584912424045\\
0.887480830903819	3.97200858962911\\
};
\addplot [color=mycolor2, forget plot]
  table[row sep=crcr]{%
2.4225402142989	2.46018632662555\\
2.51205643688159	2.45742975989243\\
2.58715310402205	2.45032443935752\\
2.65088726236523	2.44017747558856\\
2.70557307792621	2.4278666981761\\
2.75297624258242	2.41398036768896\\
2.79445592127621	2.39890929225061\\
2.83106778148485	2.38290771886248\\
2.86363880157304	2.36613373705743\\
2.89282178268163	2.34867615047899\\
2.91913526192633	2.33057230571732\\
2.94299286602794	2.31181977576321\\
2.9647249495008	2.29238376667649\\
2.98459451106734	2.27220144467588\\
3.00280877580309	2.25118393410117\\
3.01952739229601	2.22921642937336\\
3.03486786757133	2.20615664079546\\
3.04890860593242	2.18183161751885\\
3.06168969865664	2.15603283440837\\
3.07321140170315	2.12850927126795\\
3.08343001054312	2.0989580327608\\
3.09225056289839	2.06701183348295\\
3.09951542959146	2.03222237847955\\
3.10498733118484	1.99403827093061\\
3.10832455512295	1.95177553164861\\
3.10904501145278	1.90457806492188\\
3.10647405557169	1.85136439391878\\
3.09966843494282	1.79075568108779\\
3.08730489620387	1.72097851095623\\
3.06751649839952	1.63973451932123\\
3.03765237194324	1.54402890082052\\
2.99392870120542	1.42995433931523\\
2.93093549525602	1.29244386400949\\
2.84098235827299	1.12505312272658\\
2.71335191405367	0.919943702626728\\
2.53377842768651	0.668462503356907\\
2.28502042078931	0.363048770166131\\
1.95020895826046	0.00136628808190746\\
1.52075402939387	-0.407433008979272\\
1.00742697991319	-0.837777316427742\\
0.446142849925326	-1.25106537212028\\
-0.111707244002825	-1.60996241837187\\
-0.620392649297839	-1.89356900100245\\
-1.05487621394852	-2.10098517856431\\
-1.41084644802164	-2.24401564095483\\
-1.69636065488845	-2.33820047087083\\
-1.92385592987469	-2.39753995321719\\
-2.10560952422855	-2.43280258185524\\
-2.25198836060394	-2.45165586010096\\
-2.37115351288484	-2.45935790680001\\
-2.46932748905164	-2.45945525895173\\
-2.55119603512821	-2.45432688174761\\
-2.62028102094775	-2.44556815644518\\
-2.67923988611994	-2.43425027528856\\
-2.73009188392266	-2.42109225869469\\
-2.77438441764409	-2.40657438532166\\
-2.81331400459103	-2.39101305145171\\
-2.84781404185429	-2.37461034956898\\
-2.8786186291562	-2.35748701566959\\
-2.90630918515096	-2.33970433405003\\
-2.93134866004977	-2.32127860593696\\
-2.95410673594858	-2.30219050947069\\
-2.97487839752585	-2.2823908489408\\
-2.99389753846539	-2.26180364522069\\
-3.0113467549276	-2.2403271507598\\
-3.02736410100977	-2.21783311276133\\
-3.0420472940977	-2.19416441213581\\
-3.0554556237838	-2.16913104219236\\
-3.0676096069072	-2.1425042361312\\
-3.07848821597365	-2.11400838576039\\
-3.08802325950052	-2.08331019481765\\
-3.09609017451912	-2.05000425497905\\
-3.10249405386319	-2.01359389111321\\
-3.1069491015482	-1.97346565571904\\
-3.10904877997025	-1.92885521122919\\
-3.10822251951847	-1.87880146499304\\
-3.10367276308135	-1.8220846630784\\
-3.09428297552815	-1.75714270577492\\
-3.07848264028545	-1.68195840127306\\
-3.05404884139697	-1.59390941537601\\
-3.01781608985064	-1.48957429658994\\
-2.96525931255143	-1.36449725176504\\
-2.88991926847882	-1.21294324366504\\
-2.78268341530166	-1.02774855362424\\
-2.63108638333475	-0.800533386004224\\
-2.41917822405782	-0.5228320884181\\
-2.12922877244577	-0.189017812821461\\
-1.7472170033862	0.198338271800076\\
-1.27289457392451	0.622144237570358\\
-0.729699692373591	1.04912960977973\\
-0.16354907825766	1.43911799032485\\
0.374313533081504	1.76164762221984\\
0.847581194730859	2.00624311026009\\
1.24235269675514	2.17958156815997\\
1.56166558886184	2.29626968605072\\
1.816559003061	2.37147371815747\\
2.01974994150058	2.4176362919499\\
2.18266083639468	2.44390408090305\\
2.31454294060768	2.45664623494165\\
2.4225402142989	2.46018632662555\\
};
\addplot [color=mycolor1, forget plot]
  table[row sep=crcr]{%
0.915533679617734	3.88713310961483\\
1.09852957586721	3.8813657567245\\
1.27346356547106	3.86469800603206\\
1.44070502679724	3.83796895114571\\
1.60064309186958	3.80187211215185\\
1.75366660945792	3.75696259991024\\
1.90014859260987	3.70366536152442\\
2.04043405159741	3.64228356855361\\
2.1748302425219	3.57300656400907\\
2.30359849977727	3.49591705144124\\
2.42694694982747	3.4109974107844\\
2.54502351717723	3.31813518200184\\
2.65790873062758	3.21712788786888\\
2.76560792312149	3.10768748793501\\
2.86804249941612	2.9894448810189\\
2.96504003340893	2.86195501551079\\
3.05632306563367	2.72470333476827\\
3.14149661968498	2.57711448481228\\
3.22003466697259	2.41856444310393\\
3.29126606904786	2.24839747987543\\
3.35436094428586	2.06594961004835\\
3.40831896547524	1.87058038140072\\
3.45196180628364	1.66171488462126\\
3.48393279307768	1.43889762943759\\
3.50270769798084	1.20185922764917\\
3.50662134749873	0.950595448240291\\
3.49391501100189	0.685455972113881\\
3.46280893822414	0.407237005046545\\
3.41160242848525	0.117268009256084\\
3.3388000232035	-0.182521169850204\\
3.24325677681965	-0.489569388934429\\
3.12432874896352	-0.800706281557046\\
2.98200847083346	-1.11223998463782\\
2.81702153450226	-1.42011406428428\\
2.63086202041295	-1.72012409561288\\
2.42575242199516	-2.0081703800757\\
2.20452699361871	-2.28051288937255\\
1.97045256789499	-2.53399172221552\\
1.72701316031923	-2.76618270743753\\
1.47769023896395	-2.97547140947847\\
1.22576810427838	-3.16104517817101\\
0.974185079409573	-3.32281700200052\\
0.725439687995681	-3.46130334842394\\
0.481550323893453	-3.57748007529576\\
0.244059467334431	-3.67263724172967\\
0.014069969409949	-3.74824762771832\\
-0.207699290829558	-3.80585726085608\\
-0.420849176249404	-3.84700075283503\\
-0.625235479465544	-3.87314040851484\\
-0.82090867815256	-3.88562588125512\\
-1.00806102171395	-3.8856702655477\\
-1.18698195644625	-3.8743385047252\\
-1.35802174485055	-3.85254446054366\\
-1.52156247283532	-3.82105366033698\\
-1.67799532803701	-3.78048942963511\\
-1.82770294948821	-3.73134074045006\\
-1.97104569765303	-3.67397062098299\\
-2.10835080773941	-3.60862437772012\\
-2.23990352580227	-3.53543718918442\\
-2.36593946219093	-3.45444086222352\\
-2.48663751863826	-3.36556971807446\\
-2.60211285033001	-3.26866571656902\\
-2.71240941474592	-3.16348305042861\\
-2.81749174097168	-3.04969256296791\\
-2.91723563586649	-2.92688647478541\\
-3.01141763956755	-2.79458405898928\\
-3.09970316870044	-2.65223908805077\\
-3.18163346197881	-2.49925009167618\\
-3.25661169457858	-2.3349747094033\\
-3.32388898286324	-2.15874967602557\\
-3.3825514872278	-1.96991820363688\\
-3.43151045682256	-1.76786665163131\\
-3.46949784105365	-1.55207229500711\\
-3.49507096764294	-1.32216355405645\\
-3.50663062600307	-1.07799303572134\\
-3.5024574558465	-0.819721949421519\\
-3.4807714497124	-0.547911752619291\\
-3.43981815001576	-0.263615298723845\\
-3.37798226858461	0.0315442947739052\\
-3.29392471123411	0.335322455088922\\
-3.18673262583612	0.644846778872989\\
-3.05606523351181	0.956670804307634\\
-2.90227288426035	1.26689664545942\\
-2.72646550439925	1.57136383997961\\
-2.53051133047817	1.86588772527822\\
-2.31695772449058	2.1465178955073\\
-2.08888061230601	2.40978030367758\\
-1.84968336974496	2.65286836215824\\
-1.60287524639162	2.87375881821858\\
-1.35186094488384	3.07124383893494\\
-1.0997670197766	3.24488656013633\\
-0.849320137476856	3.39491889635324\\
-0.602780744122213	3.52210551505497\\
-0.361926436947026	3.62759693510459\\
-0.128073852783262	3.71278975837447\\
0.0978738583301783	3.77920553430087\\
0.315365906003484	3.82839362228991\\
0.524138523843202	3.86185871657206\\
0.724151960402994	3.88101071234976\\
0.915533679617733	3.88713310961483\\
};
\addplot [color=mycolor2, forget plot]
  table[row sep=crcr]{%
2.4299081702385	2.54116882702592\\
2.50787494271559	2.53876558816946\\
2.57366262941401	2.53253896978367\\
2.62983071977584	2.52359475287463\\
2.67831546206082	2.51267833276714\\
2.72059573800434	2.50029126138849\\
2.75781232015285	2.48676781509864\\
2.79085365342427	2.47232557052543\\
2.82041782125844	2.45709897920762\\
2.84705757465151	2.44116170019426\\
2.87121325252423	2.42454138381536\\
2.89323696401346	2.40722927966503\\
2.91341038049567	2.389186191406\\
2.93195776796347	2.37034574422184\\
2.94905538221762	2.35061555615369\\
2.96483798082337	2.32987664103611\\
2.97940292712109	2.30798117168998\\
2.9928121358354	2.28474856519831\\
3.00509190739402	2.25995969292994\\
3.01623049197373	2.23334884492343\\
3.02617298565802	2.20459286898918\\
3.03481285392039	2.17329663289134\\
3.04197895192468	2.13897358873028\\
3.04741629281838	2.10101970552392\\
3.05075789123247	2.05867831603404\\
3.05148360533599	2.01099241587134\\
3.04885975085121	1.95673956443917\\
3.04184997123927	1.89434269996271\\
3.02898288471224	1.82174796300119\\
3.00815480790283	1.73625854312748\\
2.97633622607303	1.63431340182052\\
2.9291405632774	1.51120649421977\\
2.86021201401299	1.36076848408523\\
2.76042381261721	1.17510775670447\\
2.61702285655303	0.944686228955867\\
2.41326229334755	0.659356991790123\\
2.12990038377572	0.311457516252769\\
1.75089642057547	-0.0980187258714269\\
1.27463368388592	-0.551510957104492\\
0.725603479611226	-1.01200187265938\\
0.153423684980919	-1.43355393844155\\
-0.386895878096306	-1.781373752186\\
-0.857807998042484	-2.04405169615064\\
-1.24655035727424	-2.22970197490458\\
-1.55802675628121	-2.35488668960367\\
-1.80474892698442	-2.4362870199259\\
-2.00026827854185	-2.48728888963669\\
-2.15635972917608	-2.51757183527422\\
-2.28235123026508	-2.53379703128516\\
-2.3853325708139	-2.54045052172632\\
-2.47060400784085	-2.54053264189128\\
-2.54211426479498	-2.53605092303616\\
-2.60281571015797	-2.52835313116987\\
-2.65493194885777	-2.51834705799954\\
-2.70015307733974	-2.50664450911494\\
-2.7397763721719	-2.49365571111631\\
-2.77480739784914	-2.47965149699101\\
-2.80603286524599	-2.46480449290115\\
-2.83407341333794	-2.44921650261795\\
-2.85942208340816	-2.43293670021977\\
-2.8824725214897	-2.4159735910477\\
-2.90353972275058	-2.39830264290297\\
-2.92287527559821	-2.37987080314453\\
-2.94067846079959	-2.36059866261591\\
-2.95710412985306	-2.34038071534023\\
-2.97226796853908	-2.31908393637231\\
-2.98624950341748	-2.29654472067452\\
-2.99909299838002	-2.27256406567429\\
-3.01080618748486	-2.24690071671462\\
-3.02135657185462	-2.21926180642577\\
-3.03066474070404	-2.18929028214403\\
-3.03859381758708	-2.15654809985281\\
-3.04493362220341	-2.12049372869982\\
-3.0493773843031	-2.08045190262606\\
-3.05148770902473	-2.03557270286934\\
-3.05064675651646	-1.98477586828778\\
-3.04598293737177	-1.92667462167759\\
-3.03626237354422	-1.859471247099\\
-3.01972734727122	-1.78081438888453\\
-2.99385548589297	-1.68760656457908\\
-2.95500305891746	-1.57575272729209\\
-2.89788792633696	-1.43985494402061\\
-2.81487854563123	-1.27290378155379\\
-2.6951315525663	-1.06613497952922\\
-2.52387147871502	-0.809478334678318\\
-2.2827155102493	-0.493463210432633\\
-1.95295399038066	-0.113790440084776\\
-1.52408987804222	0.321166530106526\\
-1.0065467764534	0.783769472801461\\
-0.438781182587107	1.23030599286926\\
0.12365660080417	1.61796478898512\\
0.632364709248575	1.92317878057828\\
1.06242978314545	2.14554424480842\\
1.41124442532834	2.29875214256913\\
1.68861198838555	2.40013228843616\\
1.9081169061554	2.46490185274609\\
2.08260241895416	2.50454347441469\\
2.22262737845324	2.52711937482361\\
2.33635043410676	2.5381045734392\\
2.4299081702385	2.54116882702592\\
};
\addplot [color=mycolor1, forget plot]
  table[row sep=crcr]{%
0.956492163373761	3.77412205831772\\
1.13740076407384	3.76842378481052\\
1.30974206413514	3.75200626229156\\
1.47393318508464	3.72576781510803\\
1.6304120425485	3.69045468628328\\
1.77961572647687	3.64666912332949\\
1.9219640807902	3.59487871776386\\
2.05784721601137	3.53542590681148\\
2.18761584721043	3.46853695485647\\
2.3115735180329	3.39433003513442\\
2.42996992777755	3.31282225379333\\
2.54299471316223	3.22393562312583\\
2.65077114829901	3.12750212021302\\
2.75334931928612	3.02326808008332\\
2.85069841099577	2.91089828469517\\
2.94269782332709	2.78998023422623\\
3.02912692577029	2.66002923679982\\
3.10965337973529	2.52049513581231\\
3.18382012942104	2.37077171518533\\
3.25103141066373	2.21021007956218\\
3.31053848396032	2.03813758337323\\
3.36142629425119	1.85388414361534\\
3.40260292041893	1.65681794814161\\
3.43279450382072	1.44639255212235\\
3.45054929220922	1.22220697421679\\
3.45425537515231	0.984079443041352\\
3.44217736834696	0.732133662143365\\
3.41251732322385	0.466893666768205\\
3.36350395342279	0.189379534222988\\
3.29351130938284	-0.0988082424650076\\
3.20120293909346	-0.395432053992662\\
3.08569057875153	-0.697606752326291\\
2.94668872432899	-1.00185257784537\\
2.78464033521322	-1.30422162394327\\
2.60078732071264	-1.60049568740128\\
2.3971646973498	-1.88643777475042\\
2.17650964895159	-2.15806518450049\\
1.94209347366312	-2.41190418402993\\
1.69750048694484	-2.64518837669219\\
1.44638803650607	-2.85597474469135\\
1.19226280410582	-3.04316905185557\\
0.938301050414134	-3.20646988028905\\
0.687227916835214	-3.34625299598594\\
0.441257847553226	-3.46342270785044\\
0.202088153572837	-3.55925509628672\\
-0.0290677594005788	-3.63525195926824\\
-0.251422224679322	-3.69301689447977\\
-0.464543240306113	-3.73415822607492\\
-0.668282385146043	-3.76021855937426\\
-0.862707241732954	-3.77262779087261\\
-1.04804191430491	-3.77267508472103\\
-1.22461703881608	-3.76149514011564\\
-1.39282923260587	-3.74006453499351\\
-1.55310909471179	-3.70920467702781\\
-1.70589647410159	-3.66958869119952\\
-1.85162161649841	-3.62175029923513\\
-1.99069085949854	-3.56609334832289\\
-2.12347568568543	-3.50290111685884\\
-2.25030411081334	-3.43234487799468\\
-2.37145354827313	-3.35449146089099\\
-2.48714443705732	-3.26930973983096\\
-2.59753404351027	-3.17667612588477\\
-2.70270994881875	-3.07637925481466\\
-2.80268282012914	-2.96812417560813\\
-2.89737814214477	-2.85153646114373\\
-2.98662666980141	-2.72616679860698\\
-3.07015346630092	-2.59149678270892\\
-3.14756553372927	-2.44694683662406\\
-3.21833824985063	-2.29188742545607\\
-3.2818011233653	-2.12565499683434\\
-3.33712380274189	-1.94757435838472\\
-3.38330385000719	-1.75698943172239\\
-3.41915853623843	-1.55330441875053\\
-3.44332381209519	-1.33603724016791\\
-3.45426457288477	-1.10488646486432\\
-3.45030118873616	-0.859811606680412\\
-3.42965768085947	-0.601124392539394\\
-3.3905364155057	-0.329585284462418\\
-3.33122218023402	-0.0464953272455659\\
-3.25021449801282	0.24623106449233\\
-3.14638090878713	0.546030203512032\\
-3.01911638353627	0.849708542167321\\
-2.86848679485848	1.15353175153547\\
-2.69533015530887	1.45338808905218\\
-2.50129095670289	1.74501628182254\\
-2.28877187934261	2.02427253179527\\
-2.0608022087767	2.28739951598753\\
-1.82083942887876	2.53125710237021\\
-1.57253412315359	2.75348176980381\\
-1.31949403672393	2.95255719857406\\
-1.06507961242214	3.12779689068709\\
-0.812252728524991	3.27925516934623\\
-0.563487029195763	3.40759168605377\\
-0.320736368861388	3.51391591728976\\
-0.0854500579454544	3.59963385723916\\
0.141379654705287	3.66631205839245\\
0.359152010999243	3.71556690270327\\
0.567585643337569	3.74898109802236\\
0.766648128212963	3.76804547197061\\
0.95649216337376	3.77412205831772\\
};
\addplot [color=mycolor2, forget plot]
  table[row sep=crcr]{%
2.42663064814023	2.60785811664781\\
2.49468174008971	2.60575870030039\\
2.55239939390668	2.60029427736913\\
2.60193795001227	2.59240432776687\\
2.64492702754717	2.58272400429261\\
2.68261278465861	2.57168187990689\\
2.7159583805664	2.55956400668035\\
2.74571577241066	2.54655620235487\\
2.7724773872352	2.53277211909779\\
2.79671359330112	2.51827189344838\\
2.81880006068362	2.50307443871288\\
2.83903783407902	2.48716533789187\\
2.85766806721563	2.47050158520723\\
2.87488276140725	2.4530139569977\\
2.89083242168568	2.43460747402141\\
2.90563123071895	2.41516018695922\\
2.91936009944686	2.3945203352042\\
2.93206775108925	2.37250176715401\\
2.94376980445504	2.34887734481953\\
2.95444561662831	2.32336986413215\\
2.96403239338554	2.29563977935101\\
2.97241573727883	2.26526869274245\\
2.97941531880948	2.23173711437162\\
2.98476363463382	2.19439434819838\\
2.98807471584772	2.15241743103571\\
2.98879794433029	2.10475472069131\\
2.9861494745245	2.05004785320334\\
2.97900961012992	1.98652323679204\\
2.96576810702092	1.91184105925449\\
2.94408993137816	1.82288661116335\\
2.91056126141298	1.71548807070542\\
2.86016234749081	1.58405422656099\\
2.7855138056449	1.42116462835929\\
2.67590028700184	1.21725945971396\\
2.51630693803723	0.96085555711886\\
2.28733937387612	0.640253754606333\\
1.96811113829447	0.248309634107519\\
1.54507742635003	-0.208816149717553\\
1.02660002081288	-0.702678714632209\\
0.452566596117432	-1.18438280496429\\
-0.116845507547197	-1.60413285410775\\
-0.629111139445965	-1.93406654496179\\
-1.05822398949492	-2.17352895352723\\
-1.40273388675657	-2.33810426546856\\
-1.67410244944011	-2.44718986880552\\
-1.88715984915978	-2.51749020682078\\
-2.05545301269843	-2.56139108298409\\
-2.18985564718869	-2.58746499239997\\
-2.29861963965018	-2.60146966362994\\
-2.3878659616402	-2.60723369569702\\
-2.46210796987436	-2.60730327728648\\
-2.52468330161363	-2.60337981840676\\
-2.57807879986109	-2.59660701049621\\
-2.62416564382979	-2.58775721560065\\
-2.66436694812378	-2.57735252840745\\
-2.69977684240484	-2.56574383339848\\
-2.73124533318201	-2.553162806889\\
-2.75943915518276	-2.53975635217672\\
-2.7848857327831	-2.52560948615007\\
-2.80800517492384	-2.51076050824391\\
-2.82913370106633	-2.49521089982419\\
-2.84854084410645	-2.47893151898856\\
-2.86644204919181	-2.4618660817546\\
-2.88300777859459	-2.44393253682758\\
-2.89836986805199	-2.42502267264894\\
-2.91262560739631	-2.4050000935758\\
-2.92583980018095	-2.3836965333914\\
-2.93804486378162	-2.36090631348358\\
-2.94923883663002	-2.33637857752089\\
-2.95938093476062	-2.30980672078445\\
-2.96838401080529	-2.28081415192535\\
-2.97610286612645	-2.24893513973113\\
-2.98231677778854	-2.21358895431813\\
-2.98670371295673	-2.17404473597005\\
-2.98880233513725	-2.12937341183341\\
-2.9879557777621	-2.07838139629176\\
-2.98322783606726	-2.01951860794754\\
-2.97327707002324	-1.95075043755831\\
-2.95616649574139	-1.86937996465697\\
-2.92907538160684	-1.77180428933348\\
-2.88786595675086	-1.65319172469644\\
-2.8264483041122	-1.50708707923459\\
-2.73590657652846	-1.32502100773348\\
-2.60347279280227	-1.09638260105197\\
-2.41183305744973	-0.809216968254309\\
-2.1401807378859	-0.453250634669411\\
-1.76974184690201	-0.0267037097425988\\
-1.29601116923746	0.453883472638125\\
-0.742939856829561	0.94845834101123\\
-0.163493109629213	1.40443132492638\\
0.382433223343916	1.78091893864817\\
0.854599636609413	2.06434377873827\\
1.24050348687811	2.26394711433531\\
1.54663701409861	2.39844076975961\\
1.78701637795491	2.48631381412283\\
1.976156548392	2.5421269616579\\
2.12631849151321	2.57624209357592\\
2.24701642358105	2.59570022049178\\
2.34536835067793	2.60519857167455\\
2.42663064814023	2.60785811664781\\
};
\addplot [color=mycolor1, forget plot]
  table[row sep=crcr]{%
1.01008260529783	3.64344252016072\\
1.18863581493154	3.63782274867468\\
1.35795691941703	3.62169707538814\\
1.51853505039252	3.596039966336\\
1.67087992678567	3.56166356816446\\
1.81549807204968	3.51922733307875\\
1.95287527571584	3.46924908623552\\
2.08346376072164	3.41211623186198\\
2.20767273034483	3.34809628854992\\
2.32586118968549	3.27734630195263\\
2.43833213811751	3.19992093592348\\
2.54532739923415	3.11577922133519\\
2.64702249179135	3.02479007188445\\
2.74352105303856	2.92673678023746\\
2.83484841210704	2.82132080427773\\
2.92094398579676	2.70816525751865\\
3.0016522446899	2.58681864342859\\
3.07671208976739	2.45675953219305\\
3.14574460866834	2.31740307964802\\
3.20823937192019	2.16811053679873\\
3.26353971445634	2.00820319128623\\
3.31082786321983	1.83698250119168\\
3.34911135512656	1.6537584832036\\
3.37721296910294	1.45788861870122\\
3.39376736983027	1.2488295048429\\
3.39722876537074	1.02620299783167\\
3.38589494341176	0.78987740221178\\
3.35795373804042	0.540062057875569\\
3.31155776316144	0.277410246225679\\
3.24493144135332	0.00312072118151988\\
3.15651027593232	-0.280977070906215\\
3.04510563913755	-0.572373575603527\\
2.9100796338738	-0.86788681152822\\
2.75150571136334	-1.16374511301673\\
2.57028488793736	-1.45575285888894\\
2.36818821165584	-1.73953096665872\\
2.14780582530322	-2.01080537910998\\
1.91240056029732	-2.26570195682995\\
1.66568466521956	-2.50100121158749\\
1.41155505050136	-2.71431398248399\\
1.15382947508208	-2.90415714629118\\
0.896021605662616	-3.0699303069577\\
0.641179677015408	-3.2118128162806\\
0.391797123675682	-3.33061064331799\\
0.149789393445159	-3.42758371932104\\
-0.0834777132379029	-3.50427881045662\\
-0.307125970182486	-3.56238432952652\\
-0.520685487747275	-3.60361492375731\\
-0.724011280286682	-3.62962700154336\\
-0.9172043512485	-3.64196214467776\\
-1.10054218441131	-3.64201332125204\\
-1.27442062080241	-3.63100834590335\\
-1.43930716846449	-3.6100054885638\\
-1.59570468074986	-3.57989701029662\\
-1.7441238231301	-3.54141737984803\\
-1.88506261657972	-3.49515381977946\\
-2.01899143174947	-3.44155757175398\\
-2.14634199782038	-3.38095484324088\\
-2.26749921159166	-3.31355682009303\\
-2.3827947459713	-3.23946843078057\\
-2.49250164354792	-3.15869575998736\\
-2.59682923424688	-3.07115216037838\\
-2.69591783775042	-2.97666322589441\\
-2.78983280734277	-2.87497088806156\\
-2.87855755091879	-2.76573699535562\\
-2.96198523827919	-2.64854684907848\\
-3.03990898554357	-2.52291331009226\\
-3.11201041529539	-2.38828226978606\\
-3.17784664758319	-2.24404050335959\\
-3.23683601067323	-2.08952719546202\\
-3.28824310589124	-1.92405073777965\\
-3.33116435594089	-1.74691271601883\\
-3.36451584528456	-1.5574412687282\\
-3.38702614099515	-1.35503610420793\\
-3.39723783527421	-1.13922723364808\\
-3.39352266483723	-0.909748678869892\\
-3.37411599074839	-0.666626750771519\\
-3.33717673598115	-0.410279694791303\\
-3.28087795806842	-0.141621445498978\\
-3.20353036334593	0.1378428090178\\
-3.1037356896824	0.425946667307661\\
-2.98055904483278	0.71983904773537\\
-2.83370016585106	1.01602761806062\\
-2.66363574932995	1.31050293427032\\
-2.47170207179493	1.59894374716538\\
-2.2600922576455	1.87698593269571\\
-2.03175656787866	2.14051997763506\\
-1.79021401324247	2.38597142907599\\
-1.53930320431057	2.6105200170064\\
-1.28291278747925	2.81222659483883\\
-1.02473300661721	2.99005791525733\\
-0.768060490634631	3.14382020870676\\
-0.515672839772723	3.27402717373248\\
-0.269773766994071	3.3817334889122\\
-0.0319978264617681	3.46836230050207\\
0.196542153408091	3.5355475791765\\
0.415182535501067	3.58500331369573\\
0.623626184417784	3.61842374351327\\
0.821860461073582	3.63741338152024\\
1.01008260529783	3.64344252016072\\
};
\addplot [color=mycolor2, forget plot]
  table[row sep=crcr]{%
2.41476033068263	2.66229893813066\\
2.47426051877808	2.66046188142822\\
2.52496086636225	2.65566055223636\\
2.56868246992092	2.64869594138342\\
2.60680336812849	2.64011084184641\\
2.64037873090607	2.63027218455783\\
2.67022567673547	2.61942490430711\\
2.69698361012018	2.60772747837609\\
2.72115753202921	2.59527550289921\\
2.74314940113939	2.58211732487685\\
2.76328100985426	2.56826428139133\\
2.78181074564567	2.55369717031087\\
2.79894586309751	2.53836997930367\\
2.81485137716571	2.52221150387158\\
2.82965632375841	2.50512521041166\\
2.84345786519026	2.48698749574624\\
2.85632350665693	2.46764432527621\\
2.86829150595093	2.44690607167524\\
2.87936937632061	2.42454020182558\\
2.88953017480314	2.40026124729676\\
2.89870600245254	2.37371721339243\\
2.90677777235215	2.34447119331895\\
2.91355975624061	2.3119763996539\\
2.91877659097508	2.27554202075635\\
2.92202913523956	2.23428613260507\\
2.92274352947618	2.18707017433945\\
2.92009457053138	2.13240700466926\\
2.91288935500511	2.06833106433985\\
2.89938904279321	1.99221462611844\\
2.87703433081253	1.90050926551449\\
2.84202334551324	1.7883898674507\\
2.78867310206417	1.64929041000972\\
2.70849742357398	1.47437617373549\\
2.58902294051886	1.25216810687731\\
2.41272157600675	0.968961378919757\\
2.15740379551078	0.611488822489571\\
1.80110489897901	0.174007585924341\\
1.33491064908747	-0.329868142581518\\
0.780220632640025	-0.858437119498023\\
0.192569509838792	-1.35182905347143\\
-0.362109379256176	-1.76094095913193\\
-0.839173355705796	-2.06834567290477\\
-1.22537900019186	-2.28393895297461\\
-1.52855377354098	-2.42880161679264\\
-1.7643048112528	-2.52358245666273\\
-1.94827238565339	-2.58428805072776\\
-2.09334646337523	-2.62213200702853\\
-2.20933414641352	-2.6446321006958\\
-2.30345686127722	-2.65674983100523\\
-2.38097929309355	-2.66175499013238\\
-2.44574577706084	-2.66181416277486\\
-2.50058475857612	-2.65837441759815\\
-2.54759911312288	-2.65240981638321\\
-2.58837062757489	-2.6445796382357\\
-2.62410339759821	-2.63533052752175\\
-2.65572470420992	-2.62496301407602\\
-2.68395647334647	-2.61367522554978\\
-2.70936633781276	-2.60159182182315\\
-2.73240445392251	-2.58878320443428\\
-2.75343026534433	-2.57527820089079\\
-2.7727320785939	-2.5610722598607\\
-2.79054141304474	-2.54613245103991\\
-2.80704347033075	-2.53040007914688\\
-2.82238463652207	-2.51379139393238\\
-2.83667761962244	-2.49619664361787\\
-2.85000458909476	-2.47747753592046\\
-2.86241848981296	-2.45746300903009\\
-2.8739425229001	-2.43594305053212\\
-2.8845675945778	-2.41266011222196\\
-2.89424730217836	-2.38729742666149\\
-2.90288971521271	-2.35946320289597\\
-2.91034476235518	-2.32866921595343\\
-2.91638536501737	-2.29430163791519\\
-2.92067942563984	-2.25558098569283\\
-2.92274815937726	-2.2115066359581\\
-2.92190368942666	-2.16077928146949\\
-2.91715473495383	-2.10169173654886\\
-2.90706273852554	-2.03197446090965\\
-2.88952074197548	-1.94857728422476\\
-2.86141267148656	-1.84736477030459\\
-2.81809240408791	-1.72270557611584\\
-2.75260969633187	-1.56696416314431\\
-2.65464170389844	-1.37000355082238\\
-2.50927856523459	-1.11908505286392\\
-2.29642420195864	-0.800165548559755\\
-1.99294888695119	-0.402505596279578\\
-1.58128810879217	0.0715688300337349\\
-1.06580695917224	0.594675098125325\\
-0.486211284158695	1.11321633820615\\
0.0924395664717935	1.56881374821226\\
0.611840162879801	1.92719254817372\\
1.0434617323806	2.18638558471255\\
1.38643068898744	2.36383239917408\\
1.65383033355687	2.4813303299899\\
1.86188078388727	2.55739255487381\\
2.02499011504009	2.60552571919054\\
2.1544717399875	2.63494165601582\\
2.25876065635701	2.65175283414999\\
2.34402491715779	2.65998552639311\\
2.41476033068263	2.66229893813066\\
};
\addplot [color=mycolor1, forget plot]
  table[row sep=crcr]{%
1.07692127260879	3.503666468077\\
1.25300918281728	3.4981296553274\\
1.41903449901791	3.48232295184664\\
1.57559066828564	3.45731328419297\\
1.72328858336827	3.42399002181016\\
1.86273010346507	3.38307711742861\\
1.99448928975369	3.33514683294116\\
2.11909938364559	3.28063347005673\\
2.23704386587854	3.21984614712431\\
2.3487502458256	3.15298010112185\\
2.45458550625696	3.0801262906107\\
2.55485235559185	3.00127927236677\\
2.64978561667084	2.91634345581436\\
2.7395482141668	2.82513793286865\\
2.82422632106993	2.72740015821513\\
2.9038232990936	2.62278883421946\\
2.97825213075303	2.51088645081344\\
3.04732610688787	2.3912020576363\\
3.11074762068764	2.26317501621477\\
3.1680950512474	2.12618070521954\\
3.21880792679199	1.97953943919817\\
3.26217087868375	1.82253020947976\\
3.29729737848994	1.65441124709938\\
3.3231149407673	1.47444979335821\\
3.33835441273393	1.28196374684973\\
3.34154716145888	1.07637786772928\\
3.33103533186351	0.857296700760661\\
3.30500166178559	0.624594976529345\\
3.26152616108178	0.378523569503026\\
3.19867657908521	0.119824829489567\\
3.11463706764755	-0.150154690727738\\
3.00787389465825	-0.429372646421058\\
2.87732813737469	-0.715044489263038\\
2.72261396138454	-1.00366786439617\\
2.5441902657782	-1.29113798111643\\
2.34346776726496	-1.57295980618841\\
2.12281774905249	-1.84454133989462\\
1.88546479782922	-2.1015292855393\\
1.63527087347818	-2.3401328052854\\
1.37644393510446	-2.55738011708881\\
1.11322143137332	-2.7512680245816\\
0.84958107671147	-2.92079049449001\\
0.589018967382472	-3.06585923936307\\
0.334414531303806	-3.18714802934969\\
0.0879812010941502	-3.28589915315193\\
-0.148712951538626	-3.36372648548615\\
-0.374677009102123	-3.42243953622834\\
-0.589403634093858	-3.46390144626262\\
-0.792768060091346	-3.48992439196961\\
-0.984931880726181	-3.50219956417983\\
-1.16625854347781	-3.50225568023596\\
-1.33724335614839	-3.49143907338669\\
-1.49845810192408	-3.47090887109396\\
-1.65050882723953	-3.44164188608603\\
-1.7940046862125	-3.40444311702556\\
-1.92953558294803	-3.35995892643898\\
-2.05765650621834	-3.30869092286519\\
-2.17887673587479	-3.25100930325602\\
-2.29365241756244	-3.18716493616498\\
-2.40238129872309	-3.11729982803371\\
-2.50539867091555	-3.04145585700177\\
-2.60297376494856	-2.95958181897214\\
-2.69530599936835	-2.87153894005791\\
-2.7825205972781	-2.77710509254552\\
-2.86466317138413	-2.67597802777069\\
-2.94169294415755	-2.56777802516605\\
-3.01347433245063	-2.45205046662173\\
-3.07976670000611	-2.32826899249109\\
-3.14021218759816	-2.1958400922185\\
-3.19432169577425	-2.05411023834092\\
-3.24145935384523	-1.90237699139921\\
-3.28082620369488	-1.73990587678817\\
-3.31144440682784	-1.5659552316216\\
-3.33214409411165	-1.37981157025044\\
-3.34155604505053	-1.18083819109787\\
-3.33811467655641	-0.968539532429108\\
-3.32007719888221	-0.74264287264097\\
-3.28556593487212	-0.503196977428685\\
-3.23264112331153	-0.250683848211803\\
-3.15941020074127	0.0138653435980458\\
-3.06417561474539	0.28876538350546\\
-2.94561596149961	0.571604778973339\\
-2.80298488238984	0.859231691711481\\
-2.63630059119198	1.14782198589606\\
-2.44648999609513	1.43304350653639\\
-2.23545010493017	1.71031228038472\\
-2.00599954803418	1.97511307778672\\
-1.76171434840376	2.22333637142549\\
-1.50666874172212	2.45157480806186\\
-1.24512443377309	2.65732977749459\\
-0.981221667747762	2.83910042158708\\
-0.718719854451677	2.99635515654093\\
-0.46081804338262	3.12940944708454\\
-0.21006397215261	3.23924643367439\\
0.0316577094427846	3.32731793722812\\
0.263078101244961	3.39535570033236\\
0.483460592181638	3.44521145172643\\
0.692501675998655	3.47873364607262\\
0.890231168036171	3.49768077510133\\
1.07692127260879	3.503666468077\\
};
\addplot [color=mycolor2, forget plot]
  table[row sep=crcr]{%
2.39586003108064	2.70645250921799\\
2.44793921405715	2.7048434128308\\
2.4925056321078	2.7006219481352\\
2.53110320230635	2.69447266868954\\
2.56490106181823	2.68686034672239\\
2.59479573023351	2.67809952494233\\
2.62148280844775	2.66839996721084\\
2.64550781899207	2.65789663089041\\
2.66730263361219	2.64666953175667\\
2.68721182411929	2.63475687548339\\
2.70551186863031	2.6221635898773\\
2.72242520658815	2.6088666094499\\
2.73813050107971	2.59481775820078\\
2.75277002945031	2.57994473802309\\
2.76645481248171	2.56415049078378\\
2.77926786061602	2.54731101771482\\
2.79126572870291	2.52927157937977\\
2.80247840160814	2.50984103858069\\
2.81290735701722	2.48878392383097\\
2.82252144152823	2.46580955589587\\
2.83124991591435	2.44055725967451\\
2.83897162304683	2.41257622901504\\
2.84549862605467	2.38129794978588\\
2.85055172359163	2.34599810652856\\
2.85372375788986	2.30574343486008\\
2.85442423280694	2.25931679290446\\
2.85179486635119	2.20511047637028\\
2.84457937343576	2.140973103094\\
2.83092059535225	2.06398901685406\\
2.80804228925792	1.97016184831735\\
2.77175053613937	1.85396984851581\\
2.71566587753892	1.70777473990774\\
2.63010038706129	1.52114193105368\\
2.50062498549661	1.28037916993507\\
2.30690789344056	0.969241119782617\\
2.02384246864085	0.572942103193158\\
1.62923691139703	0.0883858147587356\\
1.12147624928202	-0.460565639479011\\
0.537853457168564	-1.0169569262796\\
-0.0519105584188836	-1.51239220851406\\
-0.581938954267458	-1.90352395875888\\
-1.01946440750957	-2.1855663139474\\
-1.36351965659421	-2.37768436448737\\
-1.62880917406411	-2.50446722807857\\
-1.83313334167097	-2.5866211214845\\
-1.99194137580707	-2.63902632364079\\
-2.11711539293686	-2.67167858606987\\
-2.21735901291246	-2.69112320755616\\
-2.29894325902303	-2.70162521662839\\
-2.36638328199093	-2.70597802925916\\
-2.42295348268433	-2.70602847514658\\
-2.47105477977271	-2.70301025935254\\
-2.51247004307102	-2.69775504555296\\
-2.5485406354134	-2.69082683301651\\
-2.5802887900039	-2.68260834336394\\
-2.608503090363	-2.6733571722273\\
-2.6337987661433	-2.66324266322134\\
-2.65666067324506	-2.65237030238575\\
-2.67747424204231	-2.64079788561504\\
-2.6965479567738	-2.6285461408226\\
-2.71412978233766	-2.61560550394926\\
-2.73041918415139	-2.60194012103131\\
-2.74557586123964	-2.58748973704813\\
-2.75972594573443	-2.57216985006503\\
-2.77296615566563	-2.555870301796\\
-2.78536618221272	-2.53845230665008\\
-2.7969694176343	-2.51974376357569\\
-2.80779196065837	-2.49953252520825\\
-2.81781964678537	-2.47755709259567\\
-2.82700261099537	-2.453493931178\\
-2.83524655702526	-2.42694022364257\\
-2.84239941590435	-2.397390327888\\
-2.84823132379159	-2.36420340393172\\
-2.85240466733144	-2.32655847699991\\
-2.85442905636686	-2.28339141396671\\
-2.85359303053655	-2.23330561876357\\
-2.84885934088247	-2.17444432624126\\
-2.83870260145156	-2.10430683505434\\
-2.82085533591204	-2.01948395812656\\
-2.79190930965565	-1.91528136423395\\
-2.74669449271135	-1.78520149587765\\
-2.67734125288136	-1.62029135042173\\
-2.57197791015913	-1.40850591963233\\
-2.41329867603964	-1.13464870014876\\
-2.17815769848986	-0.78237514111604\\
-1.84133030932303	-0.341010065890242\\
-1.38819237899538	0.180920934736513\\
-0.835208929213798	0.742291470905776\\
-0.239044017925271	1.27593051799716\\
0.32733697181048	1.72210942956557\\
0.812891499624408	2.05729133431892\\
1.20243775255801	2.29129848572901\\
1.50487719019694	2.44781159112029\\
1.73754441727249	2.55006142206413\\
1.91740163105295	2.61582044512544\\
2.05812274414373	2.65734719705302\\
2.16991322967182	2.68274295031821\\
2.26016704717165	2.69729025276353\\
2.33420250887109	2.70443730605496\\
2.39586003108064	2.70645250921799\\
};
\addplot [color=mycolor1, forget plot]
  table[row sep=crcr]{%
1.15864607231793	3.36165970522797\\
1.33225134403357	3.35620747412833\\
1.49478440075304	3.34073934225922\\
1.64698656841843	3.3164308911847\\
1.78960780066664	3.28425830398293\\
1.9233772092579	3.24501456805803\\
2.04898332032296	3.19932723382943\\
2.16706140026213	3.14767580996821\\
2.27818568495898	3.09040768498839\\
2.38286481120998	3.02775201263799\\
2.48153914528356	2.95983135156412\\
2.57457901793167	2.88667106749531\\
2.662283112096	2.80820663612774\\
2.74487642026642	2.72428906298787\\
2.82250730652801	2.63468868961239\\
2.89524328755055	2.53909770391505\\
2.96306520076115	2.43713173325252\\
3.0258594709254	2.32833098635054\\
3.08340823466972	2.21216153951883\\
3.1353771566508	2.08801754849827\\
3.18130089851176	1.95522542486019\\
3.22056641959446	1.81305135780606\\
3.25239464595539	1.66071399346844\\
3.27582160359392	1.49740459137472\\
3.28968094303244	1.32231751320041\\
3.29259094842626	1.13469435216573\\
3.28295064710913	0.933885175825142\\
3.25895143343967	0.719429892043736\\
3.21861241279621	0.491161172627005\\
3.15984885736914	0.249327100252536\\
3.08058273494598	-0.00527375221765975\\
2.97890085895925	-0.271159770129723\\
2.85325846718984	-0.546060096078085\\
2.70271349409731	-0.826866041703794\\
2.52716107812122	-1.10967315780278\\
2.32752360217438	-1.38993850604152\\
2.10584598766703	-1.66275658210955\\
1.86525579730255	-1.92322686588386\\
1.60977454229429	-2.16685595700952\\
1.34400348706045	-2.38992078133626\\
1.07274033607123	-2.58972555591923\\
0.800598780380851	-2.76471295661684\\
0.531695425490565	-2.91442778511639\\
0.269443497665235	-3.03936427842029\\
0.0164620121028157	-3.14074562674795\\
-0.225415782718549	-3.22028458620438\\
-0.455064137494796	-3.27996256296887\\
-0.671948205727318	-3.32184873132998\\
-0.875995320702246	-3.34796651731514\\
-1.0674721369507	-3.36020494960387\\
-1.24687767699094	-3.36026720629691\\
-1.41485619350969	-3.34964708085678\\
-1.57212977585483	-3.3296246480048\\
-1.71944847164187	-3.30127396546065\\
-1.85755482158101	-3.26547743932403\\
-1.98715961416919	-3.22294310732679\\
-2.10892597843005	-3.17422239821988\\
-2.22345940187905	-3.11972688940498\\
-2.33130174685321	-3.05974325486975\\
-2.43292777207422	-2.99444603774209\\
-2.52874302175775	-2.92390816047626\\
-2.61908221912585	-2.84810925437041\\
-2.70420750336871	-2.76694199026823\\
-2.78430599176549	-2.6802166549814\\
-2.85948624561622	-2.58766426650185\\
-2.92977328352482	-2.48893857372896\\
-2.99510183219263	-2.38361735876521\\
-3.05530754827947	-2.27120356644535\\
-3.11011600320062	-2.1511269414496\\
-3.15912931956567	-2.02274707328312\\
-3.20181051523691	-1.88535904787861\\
-3.23746589169837	-1.73820329135964\\
-3.26522625343941	-1.580481664241\\
-3.28402843150976	-1.41138239503501\\
-3.29259957665535	-1.23011695549039\\
-3.2894480336971	-1.03597232183191\\
-3.27286628917489	-0.828381968701454\\
-3.24095333148943	-0.607017992352561\\
-3.19166535112224	-0.371904411158706\\
-3.12290423878371	-0.123547372835805\\
-3.0326516010854	0.136928612253899\\
-2.91915057597159	0.407657775624189\\
-2.78112754703252	0.685951486808953\\
-2.61803134182142	0.968291107210478\\
-2.43025173714252	1.25042562042498\\
-2.21926821359736	1.52758956887821\\
-1.98768139377993	1.79483020803747\\
-1.73909823593737	2.04740097255614\\
-1.47787533741939	2.28115352612145\\
-1.20876162797556	2.49285493698564\\
-0.936507428543648	2.68037437802348\\
-0.665510906127175	2.84271842506794\\
-0.39955540906315	2.97993113850398\\
-0.141661611193392	3.09290102385607\\
0.105950536635203	3.18312551744094\\
0.341814414927393	3.25247713342869\\
0.565115580825522	3.303000844366\\
0.775566323367475	3.33675669571624\\
0.973278152500676	3.35570945870127\\
1.15864607231793	3.36165970522797\\
};
\addplot [color=mycolor2, forget plot]
  table[row sep=crcr]{%
2.37097245534212	2.7420316362117\\
2.41656189004161	2.74062210000145\\
2.45573036679667	2.73691111095205\\
2.48978854289773	2.73148430336595\\
2.5197299463119	2.72473991858471\\
2.54631766025098	2.71694762527161\\
2.57014492375801	2.70828693151035\\
2.59167800844056	2.69887250480756\\
2.61128690506718	2.68877093192561\\
2.6292675082278	2.67801175474508\\
2.64585777686604	2.66659456961682\\
2.66124954648495	2.65449331573003\\
2.67559712901856	2.64165844829721\\
2.68902346449365	2.62801740138216\\
2.70162432302637	2.61347353472187\\
2.71347085468897	2.59790359057999\\
2.72461061815595	2.58115353271565\\
2.73506706218434	2.56303247644746\\
2.7448372630099	2.54330422245611\\
2.7538875075488	2.52167564799242\\
2.76214601829786	2.49778084702536\\
2.76949168257558	2.47115938582597\\
2.77573698267025	2.4412262613363\\
2.78060227007522	2.40722997517465\\
2.7836768267455	2.36819334751416\\
2.7843593695785	2.32282895919441\\
2.78176603826495	2.26941694331938\\
2.77458623091866	2.20562662620142\\
2.76085399122843	2.12825471098869\\
2.73758242954271	2.03284186876362\\
2.70017911364511	1.91312183239206\\
2.64152756712036	1.76027240469776\\
2.5506233526731	1.56203952020427\\
2.41084116723899	1.30216273183048\\
2.19869982976661	0.961482342718005\\
1.88609986295266	0.523853307612389\\
1.45188103584108	-0.00941184988821951\\
0.905011412639907	-0.600841322450176\\
0.300984103015729	-1.17697437633149\\
-0.279639614986057	-1.66500002627316\\
-0.777141108438632	-2.03230438061452\\
-1.17291841174885	-2.28752439101081\\
-1.4766410052214	-2.45715990617252\\
-1.70755387179492	-2.56752878073292\\
-1.88417489769587	-2.63854841219933\\
-2.02113006664282	-2.68374289252458\\
-2.12912778760137	-2.71191375145814\\
-2.21579746346988	-2.72872411662436\\
-2.28654930912442	-2.73783043742404\\
-2.34524379108104	-2.74161762034185\\
-2.39466718845133	-2.74166067136464\\
-2.43685815704226	-2.73901242010449\\
-2.4733298827699	-2.73438371425431\\
-2.50522156088711	-2.72825746861326\\
-2.53340252940198	-2.7209617754234\\
-2.55854463909447	-2.71271737191783\\
-2.58117315185587	-2.70366879825636\\
-2.60170296423541	-2.69390499614592\\
-2.62046466944398	-2.68347292916575\\
-2.637723477896	-2.67238647518524\\
-2.65369303268425	-2.66063201051763\\
-2.66854549995222	-2.64817157435036\\
-2.68241886767288	-2.63494415067055\\
-2.69542207384251	-2.62086535994409\\
-2.70763835613721	-2.60582566737345\\
-2.71912703456596	-2.58968705650954\\
-2.72992377983627	-2.57227796133157\\
-2.74003925935967	-2.55338607341383\\
-2.74945586473644	-2.5327484173563\\
-2.75812197675783	-2.51003778313961\\
-2.76594286938916	-2.48484416956563\\
-2.77276681926595	-2.4566492548512\\
-2.77836415205913	-2.42479095535871\\
-2.78239562171843	-2.38841368543216\\
-2.78436434544	-2.34639771944883\\
-2.78354193410604	-2.297257676396\\
-2.77885350496149	-2.23899503538068\\
-2.76869638454963	-2.16888211395647\\
-2.75065120711101	-2.08314489972321\\
-2.72101924431481	-1.976501637464\\
-2.67408675857107	-1.8415136263038\\
-2.60099339294028	-1.66775062116819\\
-2.48814654562482	-1.44097153743184\\
-2.31553640753194	-1.14312435524285\\
-2.0566699437989	-0.755346434435791\\
-1.68465037679736	-0.267851466002901\\
-1.19012994655386	0.301873451191688\\
-0.60514456145037	0.895985809499205\\
-0.00307335409498148	1.43520501048875\\
0.540852901816076	1.86392021265926\\
0.987553308803798	2.1724095677462\\
1.3351415266425	2.38127196158624\\
1.59996349938777	2.51834233826423\\
1.80163750250165	2.60698012197524\\
1.95685829931324	2.66373366513709\\
2.07821045594123	2.69954427228426\\
2.17474736069969	2.72147367820844\\
2.25289193509988	2.73406788805195\\
2.31720883710598	2.74027552804658\\
2.37097245534212	2.7420316362117\\
};
\addplot [color=mycolor1, forget plot]
  table[row sep=crcr]{%
1.25819306153321	3.22316713074098\\
1.42932448156702	3.21780042499708\\
1.58817579819832	3.2026898549361\\
1.73569527099036	3.17913584564961\\
1.87282081932095	3.14820898628053\\
2.00044794120658	3.11077272316979\\
2.11940999752583	3.06750715761604\\
2.23046714131272	3.01893166092257\\
2.33430099158643	2.96542509142497\\
2.43151287638883	2.90724308576105\\
2.52262405474737	2.84453231054009\\
2.60807677090703	2.77734179199042\\
2.68823531434906	2.70563155834315\\
2.76338647949024	2.62927888099351\\
2.83373896250298	2.54808241930898\\
2.89942131942143	2.46176458372459\\
2.96047815656954	2.36997244992145\\
3.01686424639106	2.2722775981427\\
3.06843627388874	2.16817533053999\\
3.11494193831615	2.05708385197457\\
3.15600618418592	1.93834420480503\\
3.19111444712788	1.81122204761352\\
3.21959301982294	1.67491278418725\\
3.24058703633076	1.52855210026402\\
3.2530372271061	1.37123465250037\\
3.25565761918672	1.20204443800585\\
3.24691785729109	1.02010113697681\\
3.22503587243533	0.824627215083776\\
3.18798916524724	0.615040344528044\\
3.13355565748483	0.391074048366338\\
3.05939705864913	0.152925467254844\\
2.96319746704448	-0.0985781135962522\\
2.84286528066663	-0.361813147851201\\
2.69679515327729	-0.634226331552334\\
2.52416781458882	-0.912277211147875\\
2.32524184475379	-1.19150302866321\\
2.10157109850985	-1.46673833034004\\
1.85607672581616	-1.73248861197861\\
1.5929240733955	-1.98341096065366\\
1.31720186022244	-2.21481305199938\\
1.0344580862182	-2.42306622844925\\
0.750189036321172	-2.60584971916751\\
0.469385229877587	-2.76219307362953\\
0.196210292324188	-2.89233948726711\\
-0.0661579410990571	-2.99749091688168\\
-0.315538485073338	-3.07950661708148\\
-0.550651292773326	-3.14061457602759\\
-0.770962052237112	-3.18317238106405\\
-0.976508330981463	-3.20949148705748\\
-1.1677347744734	-3.22172275137178\\
-1.34535225314538	-3.22179255714557\\
-1.51022612358149	-3.21137621559538\\
-1.66329270619869	-3.19189623469262\\
-1.8055000413986	-3.16453548237547\\
-1.93776797815316	-3.13025800138574\\
-2.06096279273744	-3.08983262586188\\
-2.17588220806786	-3.0438563962646\\
-2.28324750945659	-2.9927760767148\\
-2.38370023495592	-2.93690694400996\\
-2.47780157576205	-2.87644855549136\\
-2.56603313501092	-2.81149751565592\\
-2.6487980723826	-2.74205742852956\\
-2.72642192920186	-2.6680463020594\\
-2.79915260797772	-2.58930170239171\\
-2.86715909316413	-2.50558396773426\\
-2.93052856475804	-2.41657780352222\\
-2.98926158898672	-2.32189260853916\\
-3.04326508547837	-2.22106193981792\\
-3.09234278368039	-2.11354262766093\\
-3.13618291239866	-1.99871421838422\\
-3.17434294200452	-1.87587967180223\\
-3.20623135758636	-1.74426859563808\\
-3.23108673821157	-1.60304478125932\\
-3.24795492865539	-1.45132042588722\\
-3.2556659137754	-1.28818017046075\\
-3.25281325462513	-1.11271887779823\\
-3.2377407180028	-0.924097747129222\\
-3.2085430476891	-0.721623565866524\\
-3.16309050763878	-0.504855062760248\\
-3.09908931177837	-0.273737620112226\\
-3.01419117581832	-0.0287620629975018\\
-2.90616309346768	0.228865828376097\\
-2.77312069140666	0.497070671546244\\
-2.61381332780608	0.772806139286641\\
-2.42792711446531	1.05205331312857\\
-2.21634862057458	1.32996047962264\\
-1.9813178241369	1.60114259875737\\
-1.72640633608714	1.86011730169536\\
-1.45629215031779	2.10180758900895\\
-1.1763571087301	2.32201043807557\\
-0.892185922484802	2.5177331366009\\
-0.609071712238378	2.68733707063445\\
-0.331621664611324	2.83048477424765\\
-0.0635163824766542	2.94793521683809\\
0.192571151958296	3.04125702685223\\
0.434927199860419	3.11252737579308\\
0.662665959751348	3.16406508076278\\
0.875560713711963	3.19822261541064\\
1.07387188834348	3.21724197812296\\
1.25819306153321	3.22316713074098\\
};
\addplot [color=mycolor2, forget plot]
  table[row sep=crcr]{%
2.34062091623357	2.77043391311226\\
2.38048810855438	2.76920049765774\\
2.41487063113053	2.76594225496986\\
2.44488053164076	2.76115988508774\\
2.47136191107579	2.75519433348111\\
2.49496423907643	2.74827649567593\\
2.51619347442919	2.74055966732415\\
2.53544818577876	2.73214092589803\\
2.55304538806354	2.72307526061283\\
2.56923921638794	2.71338483192463\\
2.58423452433188	2.70306485611385\\
2.59819681231562	2.69208705049761\\
2.61125943422324	2.680401208964\\
2.62352871490474	2.66793522629686\\
2.63508738407729	2.65459370329513\\
2.64599655732029	2.64025510948017\\
2.65629634560386	2.62476733086357\\
2.66600502845203	2.60794126391278\\
2.6751165591453	2.58954190826651\\
2.68359595434742	2.56927612719677\\
2.69137181364272	2.54677583926005\\
2.69832475175538	2.52157480550302\\
2.70426980120371	2.49307627195009\\
2.70892967619744	2.46050733885834\\
2.71189387130658	2.42285377253979\\
2.71255536591029	2.37876560454941\\
2.71001128895158	2.32641859378053\\
2.70290467892945	2.26330850689729\\
2.68916885789367	2.18594318845285\\
2.66561022175146	2.08938165487635\\
2.62722628319647	1.966555716939\\
2.56611036138608	1.80732423281948\\
2.46979588526204	1.59734325436237\\
2.3191579275618	1.31734250648625\\
2.08714632199668	0.944806770699193\\
1.74268782805652	0.462587824528167\\
1.26752298933528	-0.121065315080577\\
0.685174320015884	-0.751117827291476\\
0.0705983045612169	-1.33763520231911\\
-0.490095687381608	-1.80916846000086\\
-0.949033123034393	-2.14814942489748\\
-1.30235718224383	-2.37606131082363\\
-1.56808187960351	-2.52450123376009\\
-1.7679299976156	-2.62003143352293\\
-1.92007065852317	-2.68120975165639\\
-2.03792748199427	-2.72010163125482\\
-2.13097838217791	-2.74437261532731\\
-2.20583863974892	-2.75889124251424\\
-2.26714487892474	-2.76678071574231\\
-2.31818449186105	-2.7700729861553\\
-2.36132268812799	-2.77010970246252\\
-2.3982881179035	-2.7677887039666\\
-2.43036411232187	-2.76371721875338\\
-2.45851781959879	-2.75830844297451\\
-2.48348845348508	-2.75184334400764\\
-2.50584840985986	-2.7445107643716\\
-2.52604616859021	-2.73643374038363\\
-2.54443679848662	-2.72768688696962\\
-2.56130389714171	-2.71830785975126\\
-2.57687551520985	-2.70830478091108\\
-2.59133577674062	-2.69766081344432\\
-2.60483335046394	-2.68633661801923\\
-2.61748754855663	-2.67427112582108\\
-2.62939256346133	-2.66138084688549\\
-2.64062015615187	-2.64755776611475\\
-2.65122095017126	-2.63266572980567\\
-2.66122434073354	-2.61653507058767\\
-2.67063687479661	-2.59895503469838\\
-2.67943877086722	-2.57966333424822\\
-2.68757799217844	-2.55833180968797\\
-2.69496091224232	-2.53454669638471\\
-2.70143803483985	-2.50778125474477\\
-2.70678231334399	-2.47735740461383\\
-2.71065612191777	-2.44239127686218\\
-2.71256045774346	-2.4017149007247\\
-2.71175579375927	-2.35376202792122\\
-2.70713693658239	-2.29639953334494\\
-2.69703222541958	-2.22667588514445\\
-2.67887725147416	-2.14044412457706\\
-2.64868109923095	-2.0318006139283\\
-2.60015857138027	-1.89227550575164\\
-2.52336726231127	-1.70976638353818\\
-2.40277592972888	-1.46747869035947\\
-2.21528573365125	-1.14401469626111\\
-1.93073081033608	-0.717795814247614\\
-1.52128427408323	-0.181220092662584\\
-0.986029825337172	0.435612739739464\\
-0.376064190330708	1.0554004523323\\
0.220679833419737	1.59015249851199\\
0.733329590066657	1.99441882961407\\
1.1380104421674	2.27399244198054\\
1.44477007677968	2.45836496946707\\
1.67499107537872	2.57754226425367\\
1.84901930754608	2.65403439145904\\
1.98261292073439	2.7028810520088\\
2.08708479775557	2.73370957408162\\
2.17035470254002	2.75262407212757\\
2.23795475939865	2.76351769633521\\
2.29378247987082	2.76890493021635\\
2.34062091623356	2.77043391311226\\
};
\addplot [color=mycolor1, forget plot]
  table[row sep=crcr]{%
1.3803158105586	3.09362620152166\\
1.54893729354649	3.08834757461492\\
1.70385067175428	3.0736199824606\\
1.84629109518523	3.05088443207606\\
1.97744700638335	3.02131067122778\\
2.09842739246933	2.98583011929795\\
2.21024459821208	2.94516856333983\\
2.31380737873498	2.89987603999062\\
2.40992028466746	2.85035273838091\\
2.499286622764	2.79687059829109\\
2.58251310245332	2.7395907336281\\
2.6601148985556	2.67857702789487\\
2.73252028326624	2.61380632502745\\
2.8000742552139	2.545175640604\\
2.86304076030295	2.47250678588776\\
2.92160318910353	2.39554875609676\\
2.97586287170976	2.31397820155143\\
3.02583529045299	2.22739828867977\\
3.07144370765178	2.13533627934926\\
3.1125098733779	2.03724022582315\\
3.14874145372062	1.93247531338628\\
3.17971582729964	1.82032060830986\\
3.20485997404229	1.69996731790985\\
3.22342638361572	1.57052018167982\\
3.23446532983979	1.43100432997874\\
3.23679462022262	1.28038090100622\\
3.22896920905005	1.11757588981802\\
3.20925506229208	0.941528012539268\\
3.1756145649968	0.751262515788463\\
3.12571459433532	0.545998238420895\\
3.05697275890688	0.325293773351387\\
2.96666102541377	0.0892337103734397\\
2.85208654475064	-0.161354125139463\\
2.71086306896909	-0.42467583761746\\
2.54126855163561	-0.69778941448721\\
2.34265335777059	-0.976528947597679\\
2.11582407343387	-1.25560544923608\\
1.86329611123416	-1.52893099931784\\
1.58930778114802	-1.79015539430015\\
1.29953650539806	-2.033327807896\\
1.00054660757384	-2.25353593622805\\
0.699089460599829	-2.44736860510669\\
0.401423196147943	-2.61310444429223\\
0.112797733141857	-2.75062047786383\\
-0.162820833709833	-2.86109388236568\\
-0.422792905911817	-2.94660581462791\\
-0.66567587136833	-3.00974651590204\\
-0.891000432494973	-3.05328563427414\\
-1.09902013707136	-3.07993341126121\\
-1.29047876097123	-3.09219047278403\\
-1.46641776589206	-3.09226950456981\\
-1.6280297223777	-3.08206810833061\\
-1.77655415559569	-3.06317411338522\\
-1.91320817811311	-3.03688892090978\\
-2.03914348297347	-3.00425893086493\\
-2.15542213748432	-2.96610879480698\\
-2.26300508755094	-2.92307292285301\\
-2.36274878631352	-2.87562345824462\\
-2.45540665108207	-2.82409403305481\\
-2.54163305914944	-2.76869924496107\\
-2.62198833070582	-2.70955011621246\\
-2.69694366193025	-2.64666593257216\\
-2.76688531448405	-2.57998289280504\\
-2.83211758399588	-2.50935997984026\\
-2.8928641952833	-2.43458242540496\\
-2.94926783270839	-2.35536310126249\\
-3.00138752998868	-2.27134214650044\\
-3.04919363025445	-2.18208514363919\\
-3.09255999762963	-2.08708019949427\\
-3.13125313083228	-1.98573438583329\\
-3.16491781700242	-1.87737017147393\\
-3.19305899997689	-1.76122276026232\\
-3.21501966812754	-1.63643967473304\\
-3.22995486468682	-1.50208453506649\\
-3.23680249553661	-1.35714781644299\\
-3.23425260632015	-1.20056844175191\\
-3.22071841420947	-1.03127133462783\\
-3.19431481332988	-0.848227338451973\\
-3.15285345483355	-0.650542770593227\\
-3.09386769577832	-0.437585506196712\\
-3.01468497997496	-0.209151545640447\\
-2.91256677147678	0.0343312360156933\\
-2.78493380284614	0.291578628037772\\
-2.62968278173373	0.560240488330013\\
-2.44557614400683	0.836762832348514\\
-2.23264986763644	1.11639222886098\\
-1.99254607838426	1.3933852947127\\
-1.72865801087491	1.66144513354078\\
-1.44599765322294	1.91433606894612\\
-1.15076782531539	2.14655445462601\\
-0.849716631807568	2.35389657663218\\
-0.54942625744986	2.53379103644462\\
-0.255700867017199	2.68534239882346\\
0.0268326092836768	2.8091237895267\\
0.294878551857073	2.90681579248229\\
0.546419730517061	2.9808004118877\\
0.780531158675736	3.0337933296798\\
0.997135915631996	3.06855852011688\\
1.19675952955286	3.08771522176552\\
1.3803158105586	3.09362620152166\\
};
\addplot [color=mycolor2, forget plot]
  table[row sep=crcr]{%
2.30478646398007	2.79272230997285\\
2.33956694171741	2.79164558255829\\
2.36967416089367	2.78879189377679\\
2.39604915169522	2.78458826116053\\
2.41940738784067	2.77932580505079\\
2.44030047088818	2.773201622012\\
2.45915907372101	2.76634612415256\\
2.47632326503792	2.75884104672134\\
2.49206419846869	2.75073132446467\\
2.50659979070406	2.74203282882589\\
2.52010613681443	2.73273721255329\\
2.53272583618772	2.72281463555961\\
2.54457401700489	2.71221483445628\\
2.55574258032479	2.70086678072266\\
2.56630299095393	2.68867700627798\\
2.57630779005883	2.67552653084561\\
2.58579087030686	2.66126617950051\\
2.59476641722134	2.64570990850852\\
2.60322625758609	2.62862553629016\\
2.6111351372104	2.60972196762729\\
2.61842313156106	2.58863154881195\\
2.62497390203963	2.56488551457264\\
2.63060672793107	2.53787944840306\\
2.63504896247673	2.50682405511139\\
2.63789341837391	2.47067397545903\\
2.63853153897078	2.4280232661205\\
2.63604690669952	2.37694958087758\\
2.62904264638929	2.31477861271425\\
2.61535713588221	2.23772424957448\\
2.59158987092791	2.14033740317269\\
2.55230801849377	2.01467364838278\\
2.48874098609742	1.8491005504418\\
2.38676701709457	1.626837450619\\
2.22436620901246	1.32503913598231\\
1.97040697847379	0.917321547969514\\
1.59106365529649	0.386261610312481\\
1.07379118803747	-0.249267906989404\\
0.461322633711947	-0.912228345945979\\
-0.152439094359707	-1.49831803336745\\
-0.682985972155761	-1.94473859744379\\
-1.09888941523765	-2.252054648784\\
-1.41006759865393	-2.4528303265095\\
-1.64031122406817	-2.58146771345832\\
-1.81210247322955	-2.66359170219609\\
-1.9425173390933	-2.71603447836118\\
-2.04356549189356	-2.74937887236392\\
-2.12349951943009	-2.77022737719967\\
-2.18799236562903	-2.78273419074224\\
-2.24098691524362	-2.78955303187048\\
-2.28526666845532	-2.79240839789825\\
-2.32283069563633	-2.79243963015844\\
-2.35513972102892	-2.79041036313633\\
-2.38327901438151	-2.78683801838073\\
-2.40806762524164	-2.78207523978057\\
-2.43013272765714	-2.77636196288903\\
-2.44996099328966	-2.76985920138283\\
-2.46793461658927	-2.76267121917085\\
-2.48435692268607	-2.75486016166862\\
-2.49947078649536	-2.74645566828574\\
-2.51347200233559	-2.73746104186232\\
-2.52651903542685	-2.72785695947072\\
-2.53874011744027	-2.71760332751641\\
-2.5502383290549	-2.70663962616922\\
-2.5610950867276	-2.69488390063282\\
-2.57137228112513	-2.68223040464321\\
-2.58111317415563	-2.66854575916828\\
-2.59034202852945	-2.65366333410884\\
-2.59906229695985	-2.63737536881697\\
-2.60725301186582	-2.61942208769188\\
-2.61486275467235	-2.59947669586444\\
-2.62180019037119	-2.5771245895206\\
-2.62791953552688	-2.55183427868937\\
-2.63299832922247	-2.52291622416527\\
-2.63670322317505	-2.48946375149657\\
-2.63853671651824	-2.45026695988795\\
-2.63775297178264	-2.40368534204549\\
-2.63322252863105	-2.34745650548621\\
-2.62321120715757	-2.27840530799946\\
-2.60501339474685	-2.1919983213213\\
-2.57433840211125	-2.08166418718521\\
-2.52428849911235	-1.93778670175437\\
-2.44371639744693	-1.74634206498435\\
-2.31486598594063	-1.48752300588703\\
-2.11105659036845	-1.1359716084557\\
-1.79811766488291	-0.667275034316672\\
-1.34858501413813	-0.0780971683956526\\
-0.774238049795059	0.584032329267382\\
-0.148286129954118	1.22042159931633\\
0.43138878657625	1.74018191686149\\
0.905258255977768	2.11404264728545\\
1.26615647323833	2.36344830444132\\
1.53378736915116	2.52433371153796\\
1.7323137872215	2.62711401014371\\
1.88162743328096	2.69274559445242\\
1.996123787551	2.734609393255\\
2.08576827798602	2.76106149465897\\
2.15739694930724	2.77733056834636\\
2.21573094769686	2.78672991307819\\
2.26407619548579	2.79139417326984\\
2.30478646398007	2.79272230997285\\
};
\addplot [color=mycolor1, forget plot]
  table[row sep=crcr]{%
1.53253678575503	2.97922984659636\\
1.69848220691898	2.97404617197318\\
1.84905338789904	2.95974112755087\\
1.98587942337696	2.93791019857066\\
2.11047743894977	2.90982256838876\\
2.22422441894082	2.87646992265854\\
2.32834778659313	2.83861196188184\\
2.42392717619353	2.79681609187841\\
2.5119022713053	2.75149055169265\\
2.59308338133733	2.70291117802793\\
2.66816267662567	2.65124242894247\\
2.73772482749179	2.59655342193481\\
2.80225631409169	2.53882972272346\\
2.86215298532536	2.4779815333501\\
2.91772561499132	2.41384881780184\\
2.96920327876322	2.34620379492374\\
3.01673438752842	2.2747511352303\\
3.06038518124038	2.19912612837183\\
3.10013542583642	2.11889104851962\\
3.13587097367162	2.0335299450084\\
3.16737275505032	1.94244213940555\\
3.19430167865837	1.84493483995977\\
3.21617885567919	1.74021552379761\\
3.23236056803254	1.62738513522178\\
3.24200754825425	1.50543377229184\\
3.24404854847912	1.37324146874722\\
3.23713904166488	1.22958801733666\\
3.21961750629181	1.07317759260802\\
3.18946448203592	0.902686185291078\\
3.14427388092552	0.716842275389181\\
3.0812521866997	0.514552958629766\\
2.99726887340254	0.295087250864118\\
2.88898890014322	0.0583227105842752\\
2.75312121191095	-0.194952944280957\\
2.5868082044987	-0.462720065356516\\
2.38815062632492	-0.741457191260336\\
2.15680525510023	-1.02603128135291\\
1.89451953418333	-1.30986628440396\\
1.60541388803629	-1.58546199907443\\
1.29584000417588	-1.84522328545909\\
0.97376055486984	-2.08242099143707\\
0.647775025750236	-2.29202113725685\\
0.326058846345277	-2.47115338695656\\
0.0155010949859296	-2.61913164993093\\
-0.278783070476973	-2.73710304913087\\
-0.55354668355016	-2.82749834587453\\
-0.807211255219125	-2.89345983110686\\
-1.03951336708477	-2.93836411768581\\
-1.25111782291838	-2.96548646552457\\
-1.44327236002039	-2.97780172214899\\
-1.61753615327626	-2.97789196897808\\
-1.77558563503421	-2.96792588011471\\
-1.91908666014927	-2.94968000538996\\
-2.04961740702522	-2.92458054500986\\
-2.16862706998544	-2.89375195075539\\
-2.27741818531878	-2.858064557318\\
-2.37714358505926	-2.81817736357363\\
-2.46881171621756	-2.7745744601786\\
-2.55329617472792	-2.7275949154007\\
-2.63134681126098	-2.67745657822587\\
-2.70360078663693	-2.62427451367744\\
-2.77059261446044	-2.56807482875103\\
-2.83276263492102	-2.5088045865118\\
-2.8904635973038	-2.44633840269961\\
-2.94396514665196	-2.38048220756323\\
-2.99345605060642	-2.31097455363744\\
-3.0390439907467	-2.23748576780782\\
-3.08075269491288	-2.15961519038613\\
-3.11851611400772	-2.07688672281537\\
-3.15216925792774	-1.98874293037763\\
-3.18143521176219	-1.89453803467972\\
-3.20590777313985	-1.79353031021597\\
-3.22502911610769	-1.68487471079595\\
-3.23806195164112	-1.56761705291988\\
-3.24405591432524	-1.44069185098044\\
-3.24180851303236	-1.30292702486047\\
-3.22982217800392	-1.15306027114507\\
-3.20626105545628	-0.98977393705402\\
-3.16891466549225	-0.811757630175587\\
-3.11518074780994	-0.617810041760324\\
-3.04208664511558	-0.406992384778554\\
-2.94637653364231	-0.17884323886986\\
-2.82469784762952	0.0663449341509858\\
-2.67391843856611	0.327207594583538\\
-2.4915874111442	0.601000775136923\\
-2.27650873051731	0.883396115860219\\
-2.02932874848718	1.16849676839643\\
-1.75296947956007	1.44917774918503\\
-1.45271532510242	1.71777384233659\\
-1.13582816674967	1.96700459311098\\
-0.810723475202644	2.19090326270296\\
-0.485914798356431	2.38548715216578\\
-0.169021024921071	2.54900209472586\\
0.133920662571018	2.68173940921952\\
0.418738978475628	2.78556175444595\\
0.683059030323465	2.86332311555987\\
0.926008456463667	2.91833410460932\\
1.14783490818052	2.953953549418\\
1.3495335469166	2.97332388388875\\
1.53253678575503	2.97922984659636\\
};
\addplot [color=mycolor2, forget plot]
  table[row sep=crcr]{%
2.26280257423888	2.80960815564716\\
2.29302831305614	2.80867183074462\\
2.31929080247629	2.80618204287693\\
2.3423822844014	2.80250127951935\\
2.36290633502203	2.79787694028509\\
2.3813293543842	2.79247641197041\\
2.39801639079062	2.78640997886444\\
2.41325644043243	2.77974590780946\\
2.42728055074426	2.77252037269508\\
2.44027491297271	2.76474387369191\\
2.45239039602751	2.75640518306304\\
2.46374949358338	2.74747345215825\\
2.47445133409025	2.73789884971652\\
2.48457517910172	2.72761191317346\\
2.49418267040591	2.71652164580413\\
2.5033189539957	2.7045122571966\\
2.51201268803542	2.69143830082525\\
2.52027481314658	2.67711778764795\\
2.52809580404399	2.66132262019722\\
2.53544090084505	2.64376535696691\\
2.54224248849254	2.62408081970194\\
2.54838827505016	2.60180029685615\\
2.55370307958217	2.57631491169239\\
2.55792064149776	2.5468228411808\\
2.56063948195111	2.51225203425696\\
2.56125271337673	2.47114511402165\\
2.55883439128627	2.42148497715885\\
2.55195194722146	2.36042619470558\\
2.53835084075819	2.28387586047843\\
2.5144163927739	2.18583579277575\\
2.47425016722987	2.05738165567431\\
2.40811017730113	1.88515666246544\\
2.29995179407256	1.64947860721804\\
2.12432871806517	1.32318517052677\\
1.84541309187049	0.875462695164259\\
1.42728821133862	0.290077031820641\\
0.867424953745529	-0.3980128423087\\
0.232854072756388	-1.08529691844234\\
-0.366964603365829	-1.65843732329594\\
-0.857779150263658	-2.07163959142021\\
-1.22747996915798	-2.34491019091256\\
-1.4974781299389	-2.51915094733346\\
-1.69475300086089	-2.62937975121338\\
-1.84116903062546	-2.69937576089385\\
-1.95220622288503	-2.74402602797072\\
-2.03835733829138	-2.77245352777192\\
-2.10668852572652	-2.79027457647737\\
-2.16200402841573	-2.80100059504363\\
-2.2076240161277	-2.80686962964739\\
-2.24588648643235	-2.80933620065568\\
-2.27846958141515	-2.80936263900506\\
-2.30660028634828	-2.80759524637279\\
-2.33119146968984	-2.80447285739189\\
-2.35293335492057	-2.80029503494\\
-2.3723556209046	-2.79526567578926\\
-2.38987026671953	-2.78952131869115\\
-2.40580166404906	-2.7831497243389\\
-2.42040792605227	-2.77620212170264\\
-2.43389628536998	-2.76870121870028\\
-2.44643426042814	-2.76064628454099\\
-2.45815779743023	-2.75201611564983\\
-2.46917718349013	-2.74277037511218\\
-2.47958125896757	-2.73284957465974\\
-2.48944026641433	-2.7221738033032\\
-2.49880752772091	-2.71064016748868\\
-2.5077200168184	-2.69811877080593\\
-2.51619777295085	-2.6844469050665\\
-2.52424195866059	-2.66942092406263\\
-2.5318311813046	-2.65278499291078\\
-2.5389154290978	-2.63421549961047\\
-2.54540656145442	-2.61329930282569\\
-2.55116363630205	-2.58950304368931\\
-2.55597027620966	-2.5621292591992\\
-2.55949945480315	-2.53025264788471\\
-2.56125795447763	-2.49262596042441\\
-2.56049726233623	-2.44753862177759\\
-2.55606791828385	-2.39260070946658\\
-2.54617683975173	-2.32440785933653\\
-2.52797596916795	-2.23801615992016\\
-2.49685708013057	-2.12612016964453\\
-2.44524644183476	-1.97779984761483\\
-2.36061832539141	-1.77677482876997\\
-2.22260043664195	-1.4996131001525\\
-2.00028857914076	-1.11622173396557\\
-1.65524651627194	-0.59947096514045\\
-1.16265575560169	0.0462647387570705\\
-0.55270247609633	0.749777113952609\\
0.0775982456260898	1.39099568483034\\
0.627983461948721	1.8847859809947\\
1.05679567880607	2.22324449171281\\
1.37319110253725	2.4419540065758\\
1.60368096486637	2.58053233579598\\
1.77321390764975	2.66830794416033\\
1.90035718559736	2.72419513280603\\
1.99788638089036	2.75985431610817\\
2.07440788424436	2.78243293029504\\
2.13573790410978	2.79636172181024\\
2.18586127967366	2.80443709500679\\
2.22755746886345	2.80845902429587\\
2.26280257423888	2.80960815564716\\
};
\addplot [color=mycolor1, forget plot]
  table[row sep=crcr]{%
1.72691890166584	2.88847723179633\\
1.88977999741713	2.8834031973613\\
2.03536220504261	2.86958333143687\\
2.16582777796012	2.84877669185259\\
2.28311545746725	2.82234504184872\\
2.38892852063427	2.79132564942209\\
2.48474300968047	2.75649464867069\\
2.57182593624988	2.718419395954\\
2.65125727235224	2.67750026980201\\
2.72395218959132	2.63400319819502\\
2.79068166269082	2.58808440644126\\
2.85209053609323	2.53980879057863\\
2.90871270340877	2.4891631136742\\
2.96098333195722	2.43606498605994\\
3.00924818368783	2.38036836691099\\
3.05377010415965	2.32186612963826\\
3.09473271118193	2.26029007125134\\
3.13224123531534	2.19530861546202\\
3.1663203563447	2.12652235981508\\
3.19690874753616	2.05345754998075\\
3.2238498846893	1.97555753587078\\
3.2468785019026	1.89217228867236\\
3.26560188770085	1.80254616152805\\
3.27947503274202	1.70580430389986\\
3.28776850659022	1.60093856189606\\
3.28952794292936	1.48679442299365\\
3.28352431398367	1.36206175158604\\
3.26819506653325	1.22527392271579\\
3.24157815775366	1.07482273819487\\
3.20124481153415	0.909000389538981\\
3.14424341568583	0.726084617565809\\
3.06707742927126	0.524488235265126\\
2.96575475049153	0.302996821707431\\
2.83596244229785	0.0611134075880491\\
2.67343114881857	-0.200492162412919\\
2.4745413993529	-0.479478188643106\\
2.23716580664829	-0.77139298741713\\
1.96162194139772	-1.06950375809132\\
1.65145678651216	-1.36511457787493\\
1.31368992162837	-1.6484886224626\\
0.958235973900378	-1.91024059688417\\
0.596541749872809	-2.14279440193373\\
0.239848108588855	-2.3414111709203\\
-0.102343401094471	-2.50448188054882\\
-0.42317189979988	-2.633118866783\\
-0.718594149747866	-2.73033687201901\\
-0.987022555610887	-2.80016265072688\\
-1.2287068586837	-2.84690338107628\\
-1.44509216148004	-2.87465850432765\\
-1.63828267686112	-2.88705726953517\\
-1.81065220664195	-2.88716101599237\\
-1.96459092140898	-2.87746633983676\\
-2.10235789055556	-2.85995982776086\\
-2.22600690882601	-2.83619235396913\\
-2.33735892084216	-2.80735485227582\\
-2.4380017496072	-2.77434678579574\\
-2.52930430162641	-2.73783403942497\\
-2.61243724478097	-2.69829586975579\\
-2.68839544316936	-2.65606188520638\\
-2.75801954004429	-2.61134049812941\\
-2.82201536321552	-2.56424032138803\\
-2.8809705693721	-2.51478581828564\\
-2.93536834419476	-2.46292828524935\\
-2.98559816563916	-2.40855301349939\\
-3.03196370132581	-2.35148326563642\\
-3.07468789760506	-2.29148152437925\\
-3.11391525625967	-2.22824832451758\\
-3.14971120020528	-2.16141886424223\\
-3.18205830904091	-2.0905575082707\\
-3.21084906159126	-2.01515024616841\\
-3.23587455717597	-1.93459516483031\\
-3.25680850423628	-1.84819105386823\\
-3.27318557608884	-1.7551244211205\\
-3.28437306798325	-1.65445550928868\\
-3.28953470899622	-1.54510446245682\\
-3.28758560690962	-1.42583972519049\\
-3.27713785400533	-1.29527225199449\\
-3.25643767503956	-1.15186139365136\\
-3.2232977623087	-0.993941642129349\\
-3.17503349703523	-0.81978384387707\\
-3.1084201706057	-0.627709629035622\\
-3.01970090069941	-0.41628209193809\\
-2.9046910010042	-0.1845954521154\\
-2.75903948093035	0.0673254023635762\\
-2.57871015733816	0.338037514991398\\
-2.36071268601663	0.624187684126594\\
-2.10402531729576	0.920179220424164\\
-1.81050727508198	1.21822058644483\\
-1.48545896971827	1.50894404467598\\
-1.13747436990128	1.78259850877738\\
-0.777441183093496	2.03053725527135\\
-0.416920991136378	2.24651693100535\\
-0.0664435258140606	2.42737556252555\\
0.265744470379608	2.57294994399036\\
0.574208918267981	2.68541759064244\\
0.856202200836089	2.76840422224328\\
1.11113844155475	2.82615351570793\\
1.33994052889959	2.86291451272814\\
1.54444115644465	2.88257247184433\\
1.72691890166584	2.88847723179633\\
};
\addplot [color=mycolor2, forget plot]
  table[row sep=crcr]{%
2.21306861028515	2.82138468809461\\
2.23919421653483	2.82057483010069\\
2.26198296571115	2.81841389866629\\
2.28209648632592	2.81520741079904\\
2.30004012557273	2.81116410695579\\
2.31620543889458	2.80642506981829\\
2.33089976349982	2.80108275966494\\
2.34436712119354	2.79519354364591\\
2.35680319112508	2.78878591359092\\
2.36836614840949	2.78186575343349\\
2.37918455994482	2.77441950095214\\
2.3893631330347	2.76641571723579\\
2.39898684605211	2.75780535367829\\
2.40812380367975	2.74852084306517\\
2.41682701984414	2.73847400734123\\
2.42513521646558	2.72755264680445\\
2.43307261707491	2.71561553299119\\
2.44064759293219	2.70248534729842\\
2.44784986320374	2.6879388588166\\
2.45464572789424	2.67169327259211\\
2.46097047148896	2.65338713218998\\
2.46671653053907	2.63255331130523\\
2.47171511937966	2.60858028298763\\
2.4757074853344	2.58065567904366\\
2.47829932467659	2.54768257197588\\
2.47888721346337	2.50815292966782\\
2.47653745718111	2.45995258613374\\
2.46978225740399	2.40005495988158\\
2.45626945328734	2.32403227607062\\
2.43214990124387	2.22526860338397\\
2.39099721123487	2.09370287188537\\
2.32193164605888	1.91391676251592\\
2.20659445887431	1.66267101758591\\
2.01537655551612	1.30749113866626\\
1.70707307373937	0.81265146938813\\
1.24532057573136	0.166096340870068\\
0.644241666550335	-0.572997261016063\\
-0.000258562820945922	-1.27152011839544\\
-0.570947604965193	-1.81719050216002\\
-1.01313964274256	-2.18963851724364\\
-1.33455067673994	-2.42728370461455\\
-1.56475863857915	-2.57586851466407\\
-1.73148478174332	-2.6690337084946\\
-1.85490534393418	-2.72803682939888\\
-1.94857853884193	-2.76570357759088\\
-2.02144687313672	-2.78974676871039\\
-2.07944622325466	-2.80487199459892\\
-2.12658297327417	-2.81401105855275\\
-2.16561701663072	-2.81903194698809\\
-2.19849027082329	-2.82115039387458\\
-2.22659750116732	-2.82117260665784\\
-2.25096007245838	-2.8196414537099\\
-2.27233945660591	-2.81692643655401\\
-2.29131279918541	-2.81328021918748\\
-2.30832416989408	-2.80887480881956\\
-2.32371993871405	-2.80382506834013\\
-2.33777359401511	-2.79820415446371\\
-2.35070340680101	-2.79205367855204\\
-2.36268515580077	-2.78539031690628\\
-2.37386137428234	-2.77820994370734\\
-2.38434809181079	-2.77048994828268\\
-2.39423972083171	-2.76219012830864\\
-2.40361251632825	-2.75325236179538\\
-2.41252687705079	-2.74359911533113\\
-2.42102863209388	-2.73313071778383\\
-2.42914934676731	-2.72172119612913\\
-2.43690556889085	-2.70921231194465\\
-2.44429680113426	-2.69540522725624\\
-2.45130180048283	-2.68004892996989\\
-2.45787253219382	-2.66282410550147\\
-2.46392467675033	-2.64332046147067\\
-2.46932289096235	-2.62100444588785\\
-2.47385785781808	-2.59517259170852\\
-2.47721016019479	-2.56488293533537\\
-2.47889250869963	-2.52885233674563\\
-2.47815558144282	-2.48529976084126\\
-2.47383130451668	-2.43170242087853\\
-2.46406632740831	-2.36440952035556\\
-2.44585964082178	-2.27802228655166\\
-2.41424909041632	-2.16439669527417\\
-2.36088161752589	-2.01107690277439\\
-2.27159316119037	-1.79904741422875\\
-2.12283613807592	-1.50040313581385\\
-1.87864430103352	-1.07935997199548\\
-1.49635590393171	-0.506829834323151\\
-0.957923725922759	0.19921414448768\\
-0.31930233989347	0.936241581559927\\
0.300140911852266	1.56686264606333\\
0.8085892828741	2.02329470130406\\
1.18720792635716	2.32224961244474\\
1.45915746964589	2.51027569508556\\
1.65461761218222	2.62780476318544\\
1.79762993314633	2.70185160839854\\
1.90481495801368	2.74896531345433\\
1.98718679399244	2.779081264406\\
2.05201909311113	2.79820951733387\\
2.10417669754535	2.81005399951602\\
2.14697605630563	2.81694845203879\\
2.18272614636253	2.82039605103031\\
2.21306861028515	2.82138468809461\\
};
\addplot [color=mycolor1, forget plot]
  table[row sep=crcr]{%
1.98350190093304	2.83481109413524\\
2.14248056964784	2.82987367299531\\
2.28208118893736	2.81663430426115\\
2.40516899039759	2.79701459731132\\
2.51420403841414	2.7724512279593\\
2.61126839996874	2.7440035972167\\
2.69810891910516	2.71244079897124\\
2.77618358885115	2.67830928457756\\
2.8467054939859	2.64198422970282\\
2.9106817063769	2.60370782662248\\
2.96894633693247	2.56361736515773\\
3.02218784567284	2.5217654316552\\
3.07097109315635	2.47813402637803\\
3.11575472258553	2.43264394117089\\
3.15690442723473	2.38516036500903\\
3.19470255464972	2.33549538741588\\
3.22935436337866	2.2834078343478\\
3.2609910950986	2.22860068335032\\
3.28966985832646	2.1707161524134\\
3.31537013590001	2.10932843153631\\
3.33798651930749	2.0439339238606\\
3.35731702858033	1.97393878711947\\
3.37304608638084	1.89864352854998\\
3.38472087205928	1.81722443533096\\
3.39171938981744	1.72871177170863\\
3.39320817539114	1.63196503927921\\
3.38808722604025	1.52564634298641\\
3.37491967274232	1.40819430322313\\
3.35184434348169	1.27780342807982\\
3.31647150223608	1.132418022608\\
3.26576712772136	0.969756328464007\\
3.19594142625119	0.787390297779356\\
3.10237596899037	0.582918868903819\\
2.9796536297165	0.35428450928417\\
2.82179468067958	0.100284240976438\\
2.62283595679328	-0.178703088485548\\
2.37787712128811	-0.47984112537456\\
2.08459072102887	-0.797045633861037\\
1.74490067134976	-1.12070472977889\\
1.36616010011684	-1.4383845799873\\
0.961011440557227	-1.73669058147411\\
0.545540466365693	-2.00381031164518\\
0.136253891683127	-2.23172729541117\\
-0.252879637056443	-2.41719964416383\\
-0.612215416707421	-2.56131528530312\\
-0.936727653380476	-2.66814611224094\\
-1.22522905296017	-2.74323004872262\\
-1.47916627067631	-2.79237206539434\\
-1.70147892947065	-2.82091381775971\\
-1.89574084990639	-2.83340285743232\\
-2.06560850664664	-2.83352251146187\\
-2.21451442915048	-2.82415886677544\\
-2.34552614361502	-2.80752219717156\\
-2.46130458258079	-2.78527693510509\\
-2.56411606576263	-2.75865895339753\\
-2.65586928113545	-2.72857289609268\\
-2.73816091103867	-2.6956691735887\\
-2.81232128279681	-2.66040306853896\\
-2.87945597318512	-2.62307917768607\\
-2.94048180960857	-2.58388427271685\\
-2.99615700593859	-2.54291118335312\\
-3.04710577173586	-2.50017576111775\\
-3.0938379552933	-2.45562848446283\\
-3.13676430446855	-2.40916184966583\\
-3.17620785403866	-2.3606143575557\\
-3.21241182593122	-2.30977164153326\\
-3.24554428337353	-2.25636507247526\\
-3.27569962030643	-2.20006800733758\\
-3.30289679282638	-2.14048971016388\\
-3.32707400402481	-2.07716686053251\\
-3.34807932820653	-2.00955247449009\\
-3.36565649445862	-1.93700200381204\\
-3.3794247335105	-1.85875637028413\\
-3.38885122290928	-1.77392177183919\\
-3.39321425839128	-1.68144633904356\\
-3.39155488820183	-1.58009424993202\\
-3.38261450833846	-1.46841894084292\\
-3.36475612660729	-1.34473892226158\\
-3.33586825206194	-1.20712293827287\\
-3.29325374438244	-1.05339649906144\\
-3.2335133149152	-0.881189930367348\\
-3.1524474686113	-0.68805933509838\\
-3.04502470951272	-0.4717247658697\\
-2.90549908589009	-0.2304784019155\\
-2.72779972292002	0.0361951825770366\\
-2.50633242227956	0.326800387005326\\
-2.23727259392387	0.636955616243555\\
-1.92022039873202	0.958795871949097\\
-1.55973200992121	1.28113532185731\\
-1.16592980680383	1.59076497852406\\
-0.753506095799707	1.8747576300706\\
-0.33916317715618	2.122984674891\\
0.061546409408701	2.32978928870104\\
0.436689095220135	2.49422395791177\\
0.778975116482342	2.61906324192193\\
1.08543654134701	2.7092890691649\\
1.35635780230378	2.77069358498586\\
1.59405931410446	2.80891338869378\\
1.80188505015079	2.82891482601894\\
1.98350190093304	2.83481109413524\\
};
\addplot [color=mycolor2, forget plot]
  table[row sep=crcr]{%
2.15235170053101	2.82771016179676\\
2.17478115168786	2.8270143638408\\
2.19442963274587	2.82515076697225\\
2.21184331301689	2.82237430294861\\
2.22744071667035	2.81885934274927\\
2.24154721596132	2.81472355735665\\
2.25441907581603	2.81004353652318\\
2.26626047910532	2.80486506527101\\
2.27723574388257	2.79920983718986\\
2.287478180893	2.79307970635794\\
2.2970965514435	2.78645915794991\\
2.30617976561089	2.77931640431305\\
2.31480024420055	2.77160332548791\\
2.32301621447722	2.76325433179324\\
2.33087309285995	2.75418410478504\\
2.33840400822576	2.74428405092516\\
2.34562942082719	2.73341715966002\\
2.3525556768675	2.72141077046633\\
2.35917218516503	2.70804648849882\\
2.36544667654291	2.69304609484333\\
2.37131765367985	2.67605169284216\\
2.37668256454787	2.65659737944596\\
2.38137926771617	2.63406819591809\\
2.38515669445102	2.60763958760633\\
2.38762767356599	2.57618636589477\\
2.38819156745344	2.53814292479036\\
2.3859045295297	2.49128390585528\\
2.37925664472764	2.43237256993924\\
2.36577987128437	2.35658621879794\\
2.341343944372	2.25656583621438\\
2.29887835004151	2.12085217775264\\
2.22608704851716	1.93143401185242\\
2.10168592547686	1.66053185726397\\
1.89085754402466	1.26902361700542\\
1.54649503515728	0.716340999550197\\
1.03571087172451	0.000929336424112065\\
0.399348919740479	-0.782068038128535\\
-0.236420855862073	-1.47169274277777\\
-0.760509941703801	-1.9731442651855\\
-1.14599792097037	-2.29796994256754\\
-1.41795736062666	-2.49909633877957\\
-1.61004980181856	-2.6230915771769\\
-1.74853215088807	-2.7004757486327\\
-1.8510856437997	-2.74950170551074\\
-1.92914976045597	-2.78089014176923\\
-1.99012781130407	-2.80100848790145\\
-2.03888953927628	-2.81372339082726\\
-2.07870967972725	-2.82144283220785\\
-2.11184268359863	-2.82570383909892\\
-2.13987658965057	-2.82750974300448\\
-2.16395440534203	-2.82752820836332\\
-2.1849153903894	-2.82621036536572\\
-2.20338713933759	-2.82386418602627\\
-2.21984685321108	-2.82070066621658\\
-2.23466292048751	-2.81686344476888\\
-2.24812365619123	-2.81244808564602\\
-2.26045749776887	-2.80751474654026\\
-2.27184740571056	-2.80209650138798\\
-2.28244125532504	-2.79620471606388\\
-2.29235939745704	-2.78983234430576\\
-2.30170017193125	-2.78295567340977\\
-2.31054389541644	-2.77553482472919\\
-2.3189556644266	-2.76751315331924\\
-2.32698718181477	-2.75881556255123\\
-2.33467770906455	-2.74934563031601\\
-2.3420541496028	-2.7389813137606\\
-2.34913016391621	-2.72756883803061\\
-2.35590408628134	-2.71491415348089\\
-2.36235522843535	-2.70077102445467\\
-2.36843787494733	-2.68482432663472\\
-2.37407182667224	-2.66666637335596\\
-2.37912760666702	-2.64576288546655\\
-2.38340318009172	-2.62140325599939\\
-2.38658683527468	-2.59262649835452\\
-2.38819693135162	-2.55810873989928\\
-2.38748200409132	-2.51598860125734\\
-2.38325123982084	-2.4635902058119\\
-2.37357973587011	-2.39697463704155\\
-2.35528424240865	-2.31020156485598\\
-2.32297480791804	-2.19410716723213\\
-2.26733796727471	-2.03432483854576\\
-2.17215165550712	-1.8083671327232\\
-2.00985192138508	-1.48263115360921\\
-1.73835691289407	-1.01459181792908\\
-1.31180442026624	-0.375710526844209\\
-0.726543819406856	0.392110669737228\\
-0.0726552518913369	1.1473486329682\\
0.515994265558174	1.74708442268336\\
0.969581768645549	2.15449219604956\\
1.29396436142171	2.41070322055379\\
1.52212487005618	2.56847718214997\\
1.68471060334354	2.66624400461973\\
1.80346802395086	2.72773203965389\\
1.89264247285927	2.76692741400909\\
1.96142401905006	2.7920729024311\\
2.01580175575232	2.80811516354905\\
2.05975745345351	2.81809589920693\\
2.09600028640691	2.82393323818825\\
2.12641710976445	2.8268657627066\\
2.15235170053101	2.82771016179676\\
};
\addplot [color=mycolor1, forget plot]
  table[row sep=crcr]{%
2.33734640818134	2.84173773499851\\
2.49109306502754	2.83698081417114\\
2.62330022544378	2.82445653845541\\
2.73771631127657	2.80622994061618\\
2.83740682018188	2.78378031472703\\
2.92486031715751	2.75815635844025\\
3.00209113109873	2.73009192852672\\
3.07072932765729	2.70009040270244\\
3.13209557814148	2.66848528261135\\
3.18726167237742	2.63548328666928\\
3.23709857129096	2.60119468959842\\
3.28231411075317	2.56565438791066\\
3.32348228736911	2.52883617258919\\
3.36106573814055	2.49066194334983\\
3.39543268532002	2.45100704965754\\
3.42686930034134	2.40970253972035\\
3.45558815745043	2.36653479655119\\
3.48173319375229	2.32124280621342\\
3.50538135569285	2.27351311212424\\
3.52654087650361	2.22297234172502\\
3.54514587585504	2.169177033847\\
3.5610466794569	2.11160033706129\\
3.57399489574042	2.04961498688047\\
3.58362182697132	1.98247180702595\\
3.58940819663856	1.90927283688475\\
3.59064240796475	1.82893811365253\\
3.58636359083918	1.74016524118358\\
3.57528458424016	1.64138137476734\\
3.55568892486679	1.53068855771114\\
3.52529540174499	1.40580622725746\\
3.48108505982604	1.26402050241484\\
3.41909141725667	1.10216073546208\\
3.33417040425861	0.916642753393567\\
3.21980119327106	0.703648025466911\\
3.06803476626328	0.459547051737497\\
2.86981036427367	0.181706580406626\\
2.6159739292958	-0.130208155401798\\
2.29934194311722	-0.472513120888338\\
1.91782040438414	-0.835884415541505\\
1.4777126078581	-1.20491866160028\\
0.995289546876273	-1.56004776980447\\
0.494706089324506	-1.88187020607477\\
0.00238449151309759	-2.1560540933375\\
-0.459355597332624	-2.3761892341717\\
-0.876199631010727	-2.54343586600995\\
-1.24220440088729	-2.66399026487476\\
-1.55780494100006	-2.74618201507608\\
-1.8272057959172	-2.79836113551832\\
-2.05624254163932	-2.82780100467006\\
-2.25101548100488	-2.84034970404085\\
-2.41717577691034	-2.84048729330591\\
-2.5596376113602	-2.83154468445753\\
-2.68252577371285	-2.81595187433117\\
-2.78923445865596	-2.79545890038434\\
-2.88252731183531	-2.77131300475877\\
-2.96464362547795	-2.7443931490121\\
-3.03739528832324	-2.71530905464699\\
-3.1022492448699	-2.68447280866225\\
-3.16039499015794	-2.65215002671948\\
-3.21279858825657	-2.61849606107907\\
-3.26024529251385	-2.58358133702245\\
-3.30337281756184	-2.54740876374752\\
-3.34269704208471	-2.50992529935037\\
-3.37863158278195	-2.47102910840988\\
-3.41150234802259	-2.43057328017037\\
-3.44155787923642	-2.38836672707065\\
-3.46897602068764	-2.34417261896214\\
-3.49386721477555	-2.29770449848986\\
-3.51627448610095	-2.24862004575401\\
-3.53616993572033	-2.19651229908184\\
-3.55344729667083	-2.14089798164329\\
-3.56790977846952	-2.08120242353426\\
-3.57925202243482	-2.01674040514939\\
-3.58703446675794	-1.94669209089562\\
-3.59064774299446	-1.87007310494929\\
-3.58926386356524	-1.78569779700986\\
-3.58176991497527	-1.69213500956189\\
-3.56667883962573	-1.58765649062854\\
-3.54201099524978	-1.470180064693\\
-3.50514035916986	-1.3372137993762\\
-3.45260237602551	-1.18581540912352\\
-3.37987038390187	-1.01259563476494\\
-3.28113136886937	-0.813818484650238\\
-3.14914043914852	-0.585686455627477\\
-2.97531818455351	-0.324937813493696\\
-2.75037108609346	-0.0298939993925478\\
-2.46579913118171	0.297998214986816\\
-2.11652784717784	0.652394548293221\\
-1.70430281127496	1.02086053043047\\
-1.24041247959621	1.38549935232378\\
-0.745589454425837	1.72618573406833\\
-0.245957959688292	2.02551419754205\\
0.233410818537544	2.27295902153481\\
0.673940830231836	2.46611819797747\\
1.06564802589559	2.60904900756689\\
1.40610199972537	2.70934266685266\\
1.69793410542082	2.77553649714771\\
1.9463836188929	2.81552398599753\\
2.15754334599019	2.8358768856879\\
2.33734640818134	2.84173773499851\\
};
\addplot [color=mycolor2, forget plot]
  table[row sep=crcr]{%
2.07403866260917	2.8269543349669\\
2.09315924916392	2.82636066585801\\
2.10999318190359	2.82476358091409\\
2.12498416604991	2.82237301346102\\
2.1384737236779	2.81933272783761\\
2.15072854314413	2.81573951332741\\
2.16195960392293	2.81165577361651\\
2.17233576107662	2.80711780571264\\
2.18199352214924	2.80214117749728\\
2.19104415278104	2.79672407548457\\
2.19957886469313	2.79084915661269\\
2.20767258761167	2.78448421578705\\
2.21538665469627	2.77758182436031\\
2.2227706072954	2.7700779720829\\
2.22986322800465	2.76188963392558\\
2.23669282495772	2.75291106475939\\
2.24327670047541	2.74300847945363\\
2.24961962713554	2.73201257827272\\
2.25571100127946	2.71970809059483\\
2.26152011235544	2.70581907565178\\
2.26698859638195	2.68998804099299\\
2.27201852764331	2.67174585401969\\
2.27645355260126	2.6504676438978\\
2.28004862539231	2.62530691233002\\
2.28242057834992	2.59509496991008\\
2.28296560694686	2.55818389163297\\
2.28071807308664	2.51219529283697\\
2.27410238642777	2.45360860632651\\
2.26048518241586	2.37707127134637\\
2.23534788704536	2.27422554125036\\
2.19073871869562	2.13172032968132\\
2.11242614059813	1.92801461437831\\
1.97516213207316	1.62920486636794\\
1.73745884434234	1.18788523244802\\
1.34691053847526	0.561027491721929\\
0.783496643137142	-0.228530815809744\\
0.128514551779342	-1.03519522066827\\
-0.470022307285884	-1.68503886015666\\
-0.927825138946958	-2.12334570222652\\
-1.24955556419985	-2.39453726327172\\
-1.47174325929099	-2.55887721488673\\
-1.6276173784582	-2.65949588556988\\
-1.74007013041079	-2.72233230901795\\
-1.82370459021303	-2.76231110217522\\
-1.88773858188048	-2.78805580223589\\
-1.93807543546071	-2.80466143548468\\
-1.97858565442676	-2.81522326828103\\
-2.01187345695523	-2.82167525731679\\
-2.0397362836796	-2.82525763629486\\
-2.06344483303272	-2.82678421414579\\
-2.08391748886742	-2.82679934778336\\
-2.10183152825442	-2.82567259474096\\
-2.11769567143009	-2.82365721079403\\
-2.13189845770502	-2.820927108104\\
-2.14474116501868	-2.81760064446228\\
-2.15646063252991	-2.81375615329919\\
-2.16724534973486	-2.80944215691259\\
-2.17724696413185	-2.80468405816212\\
-2.18658860818404	-2.7994884187659\\
-2.1953709700189	-2.79384550823411\\
-2.203676722874	-2.78773053545745\\
-2.2115737211562	-2.78110379048918\\
-2.21911722628461	-2.77390978768314\\
-2.22635131743657	-2.7660753869192\\
-2.2333095527058	-2.75750675718095\\
-2.24001486011258	-2.74808491752273\\
-2.24647854030522	-2.73765942262243\\
-2.25269813468516	-2.72603952351195\\
-2.25865372597533	-2.71298178264141\\
-2.2643019470651	-2.69817258161117\\
-2.26956649912897	-2.68120310461785\\
-2.27432317971486	-2.66153299540644\\
-2.27837603427681	-2.63843659050964\\
-2.28141877576086	-2.61092174376639\\
-2.28297110720314	-2.57760452688373\\
-2.28227113129426	-2.53651120578387\\
-2.27808881038903	-2.48475757987664\\
-2.26839371171135	-2.41801737554461\\
-2.24974780590293	-2.32962364145465\\
-2.21617410986802	-2.209037813379\\
-2.15704627256193	-2.03929834557288\\
-2.05333502049323	-1.79319604265062\\
-1.87212667571019	-1.42961807547341\\
-1.56418794388763	-0.898801961181344\\
-1.08459312721087	-0.180259328262583\\
-0.458142264779495	0.642240804383108\\
0.185645057171374	1.3865479411452\\
0.717664498691321	1.92902099233589\\
1.10355717766559	2.27579000289122\\
1.37077553186421	2.48689678677427\\
1.55630717272349	2.61520166763647\\
1.68820173657598	2.69451217766848\\
1.78481611440694	2.74453254832799\\
1.85774317453226	2.77658392136132\\
1.91434049111072	2.79727288927779\\
1.95937293265292	2.81055649618859\\
1.99600504333981	2.81887305061057\\
2.02639365982054	2.82376652078023\\
2.05204572476894	2.82623889617718\\
2.07403866260917	2.82695433496691\\
};
\addplot [color=mycolor1, forget plot]
  table[row sep=crcr]{%
2.85378684355605	2.95375934774863\\
3.00036271091603	2.94924405278867\\
3.12344126758871	2.93759902399836\\
3.22778799871929	2.92098727686308\\
3.3170997188036	2.90088308953318\\
3.39424650393384	2.87828537605597\\
3.46146503151574	2.85386429021913\\
3.5205077004599	2.82806100926026\\
3.57275564791737	2.80115539161355\\
3.61930372218224	2.77331176433672\\
3.66102418476661	2.74460980903458\\
3.69861443570776	2.71506521735019\\
3.73263274860664	2.6846432222397\\
3.76352495212441	2.6532670549698\\
3.79164418608587	2.62082266241941\\
3.81726524580055	2.58716052734958\\
3.84059455760578	2.55209508554567\\
3.86177645753552	2.51540197328519\\
3.8808961356679	2.4768131280646\\
3.89797932807475	2.43600957663164\\
3.91298855500801	2.39261155440344\\
3.92581538456694	2.34616538903257\\
3.93626780564246	2.29612632827557\\
3.94405127046831	2.24183617786611\\
3.94874124478657	2.18249421862109\\
3.94974408471304	2.11711937884328\\
3.94624161194274	2.04450105336132\\
3.93711271718952	1.96313534372365\\
3.92082251357158	1.87114303712362\\
3.89526591422912	1.76616583920769\\
3.85754835703905	1.64523938113454\\
3.80368328564037	1.50464785205062\\
3.72818855730828	1.33978092653307\\
3.62358394602286	1.14504866135667\\
3.4798554152207	0.91397973008484\\
3.28410704277114	0.639748655323926\\
3.02092802527427	0.316534364168515\\
2.67443814677328	-0.0578283946622168\\
2.23315886346491	-0.477872185184208\\
1.6977294122587	-0.926613010670904\\
1.08809389975987	-1.37524073796466\\
0.443423744619842	-1.789656253505\\
-0.188919701103301	-2.1418787995415\\
-0.769190184836531	-2.41863570498246\\
-1.27496003117096	-2.62168455721952\\
-1.70082019331029	-2.76206254379989\\
-2.05243324705879	-2.85371812142474\\
-2.34036479104677	-2.90954929474476\\
-2.5760484738883	-2.939888827454\\
-2.76983699388432	-2.95240645345205\\
-2.93035907357098	-2.95256259839635\\
-3.06451310398694	-2.94415836419008\\
-3.17770308589254	-2.92980858807872\\
-3.2741269781873	-2.91130015429449\\
-3.35704103530817	-2.88984769035182\\
-3.42897740464765	-2.86627073504579\\
-3.49191444035656	-2.84111477549386\\
-3.5474067123776	-2.81473343561453\\
-3.59668306227496	-2.78734414827436\\
-3.6407201882803	-2.75906578463492\\
-3.680297784754	-2.72994395305715\\
-3.71603984576179	-2.69996777990496\\
-3.74844556194064	-2.66908069854847\\
-3.77791231563064	-2.63718690458173\\
-3.80475257411681	-2.60415454368636\\
-3.82920594423958	-2.5698162864352\\
-3.85144723589105	-2.53396764537538\\
-3.87159104604926	-2.49636315828672\\
-3.88969308395758	-2.45671036490989\\
-3.90574818005476	-2.41466131763696\\
-3.91968462439406	-2.369801168491\\
-3.93135412847148	-2.32163314528318\\
-3.94051625242322	-2.26955894890809\\
-3.94681552699829	-2.2128532506457\\
-3.94974864331101	-2.15063052461213\\
-3.94861786898256	-2.08180190868407\\
-3.94246512699939	-2.00501917332367\\
-3.92997876793897	-1.91860230349286\\
-3.90936183301247	-1.82044697762058\\
-3.87814661617138	-1.70790912168691\\
-3.83293637790775	-1.57766744265031\\
-3.76905389077023	-1.42557504030993\\
-3.68008561837665	-1.24653503414642\\
-3.55734757160428	-1.03448524106805\\
-3.38940075654997	-0.782670607277383\\
-3.16196877730482	-0.484526804717714\\
-2.8589974547303	-0.135636159747718\\
-2.46599260243726	0.262899932561106\\
-1.97647101926526	0.700218704365431\\
-1.40011050779229	1.15307365468656\\
-0.767179368858455	1.58875278686164\\
-0.122883733916398	1.9747623873091\\
0.487411681846658	2.28988111309565\\
1.03206660126965	2.52881822967954\\
1.49770385090727	2.69884339905605\\
1.88530306237435	2.81312253182019\\
2.2036337251602	2.88540035185023\\
2.46405326257749	2.92736787416042\\
2.67759798181907	2.94798892820503\\
2.85378684355605	2.95375934774863\\
};
\addplot [color=mycolor2, forget plot]
  table[row sep=crcr]{%
1.96331297825472	2.81411434692614\\
1.97955568460529	2.81360945989346\\
1.99394894145568	2.81224344495505\\
2.00684563657167	2.81018642729994\\
2.01851919858992	2.80755505797425\\
2.02918453054378	2.80442755317506\\
2.03901273452669	2.80085358909671\\
2.04814162857186	2.79686080122021\\
2.05668335343318	2.79245896553525\\
2.06472992354038	2.78764252681038\\
2.07235728995282	2.78239187592512\\
2.07962829257673	2.77667360100596\\
2.08659474688109	2.7704398071925\\
2.09329881290044	2.76362649201651\\
2.09977371479861	2.75615085852422\\
2.10604380432763	2.7479073287507\\
2.1121238779384	2.73876186553627\\
2.11801754870306	2.72854399284722\\
2.12371431648853	2.71703558071442\\
2.12918473361584	2.70395496073472\\
2.13437266031943	2.68893414383281\\
2.13918292172501	2.67148561889073\\
2.1434614872898	2.6509530546004\\
2.14696315807193	2.62643654424968\\
2.14929780981425	2.59667659095788\\
2.14983877505048	2.55986949608556\\
2.14756240313816	2.51336573228205\\
2.14075880220387	2.45316383993336\\
2.12649489728472	2.37304040554418\\
2.09959243129066	2.26303053274972\\
2.05066101712678	2.10679271522806\\
1.96241744398336	1.87735297786831\\
1.80373134755954	1.5320193803186\\
1.5249370184216	1.01443872907873\\
1.07434285272362	0.290876216853278\\
0.466845707927361	-0.561366935479028\\
-0.165834173365223	-1.34157313650627\\
-0.685139205075571	-1.90593960632484\\
-1.05566763729559	-2.26086599763237\\
-1.30803943883671	-2.4736311669978\\
-1.48098569504271	-2.60155014104102\\
-1.60277403995252	-2.68015961549638\\
-1.69139540992165	-2.72967353157353\\
-1.75798342375903	-2.76149930187783\\
-1.8095001932274	-2.78220824653105\\
-1.8504053136757	-2.79570011885863\\
-1.8836363370713	-2.80436241989564\\
-1.91118293182293	-2.80970035122978\\
-1.93442860552846	-2.81268811575503\\
-1.95435888882385	-2.81397064027515\\
-1.97169141749957	-2.81398282422971\\
-1.98695927691459	-2.81302198532718\\
-2.00056568939737	-2.81129297433242\\
-2.01282070314002	-2.80893688317258\\
-2.0239663129008	-2.80604963840826\\
-2.0341939807038	-2.8026941931011\\
-2.04365705814167	-2.79890855504601\\
-2.0524797183299	-2.79471102260153\\
-2.06076344870776	-2.79010347533507\\
-2.06859180064954	-2.78507323912646\\
-2.0760338591876	-2.77959383119857\\
-2.0831467385059	-2.77362474073185\\
-2.08997729655123	-2.76711028448372\\
-2.09656317542501	-2.7599774726838\\
-2.10293319858159	-2.7521327105695\\
-2.10910707836365	-2.74345702675666\\
-2.11509429371756	-2.73379933765209\\
-2.12089186851221	-2.72296699287375\\
-2.12648058522473	-2.71071244562911\\
-2.13181885532999	-2.69671426376555\\
-2.13683294532342	-2.68054968683557\\
-2.14140135889619	-2.66165426968956\\
-2.14532958657962	-2.63926134362759\\
-2.14830854448596	-2.61230916988259\\
-2.14984462122842	-2.57929506405428\\
-2.14913886684121	-2.53804021148533\\
-2.14487236280542	-2.48530023035206\\
-2.13481351048029	-2.41610339883878\\
-2.11507917243741	-2.322602075395\\
-2.07871661488327	-2.19206490860275\\
-2.01299166199431	-2.00347325007772\\
-1.89455311878832	-1.72253206278974\\
-1.68304146370885	-1.29824361601324\\
-1.32288788987185	-0.677317393127129\\
-0.784256047327998	0.130299107663984\\
-0.143052910286758	0.973226891250416\\
0.444335511621189	1.6531311090341\\
0.887744684682392	2.1055808885141\\
1.19402089518443	2.38089192911604\\
1.40239527567296	2.54552300786836\\
1.54695012512996	2.64548551289316\\
1.65041113146462	2.70769190997727\\
1.72693445679873	2.74730523772255\\
1.78530237107653	2.77295405043041\\
1.83106812686669	2.78968083596682\\
1.8678383551793	2.80052528515005\\
1.89802243479918	2.80737648435111\\
1.92327427133273	2.81144166910494\\
1.94475819050041	2.813511447244\\
1.96331297825472	2.81411434692614\\
};
\addplot [color=mycolor1, forget plot]
  table[row sep=crcr]{%
3.66143984979885	3.26030969750488\\
3.79889567075761	3.25609522771153\\
3.91145095592088	3.24545959918808\\
4.00488405111818	3.23059503687368\\
4.08344387248664	3.21291823573724\\
4.15028586095019	3.1933443182594\\
4.20777948300575	3.1724604713876\\
4.25772412466527	3.15063650206084\\
4.30150107266993	3.12809570627973\\
4.34018118660345	3.10496058427243\\
4.37460178831773	3.08128242180113\\
4.40542201274555	3.05706035845976\\
4.43316292657453	3.0322534630426\\
4.45823672388085	3.00678802538314\\
4.4809679471203	2.98056144301995\\
4.50160874510828	2.95344354072357\\
4.52034952145912	2.92527579532119\\
4.53732584811405	2.89586867450269\\
4.55262214951887	2.86499708846314\\
4.56627235198397	2.8323937631021\\
4.57825739685561	2.79774014511581\\
4.5884991936492	2.76065421558516\\
4.59685019208507	2.72067428857635\\
4.60307721600122	2.67723746494548\\
4.60683743426471	2.62965084430114\\
4.60764320176406	2.5770527932945\\
4.60481076479983	2.51836041969331\\
4.59738513955396	2.45219776821199\\
4.58402929444862	2.37679697110114\\
4.56285927616657	2.28986152667899\\
4.53119697906703	2.18837715999582\\
4.48519762792757	2.06835230804267\\
4.41928951640905	1.92447060357796\\
4.32534401180416	1.74965179478889\\
4.1914966368282	1.5345712889852\\
4.00063747017681	1.26734206152893\\
3.72897141283549	0.933929753391442\\
3.34608977710739	0.520567391559537\\
2.81996545962316	0.0201741504485785\\
2.13164049836243	-0.556265694742713\\
1.29875894103909	-1.16884591759495\\
0.38957392165057	-1.75320489136707\\
-0.498186453022081	-2.24783582138081\\
-1.28370138150545	-2.62273274160032\\
-1.93109871617218	-2.88288433040544\\
-2.44307834405449	-3.05183851426923\\
-2.84090750293457	-3.15567109193353\\
-3.14945119067209	-3.21558432071315\\
-3.39047657724607	-3.24666685292685\\
-3.5810051094907	-3.25901012604828\\
-3.73371646281758	-3.25918281434069\\
-3.85788271398113	-3.2514207524235\\
-3.96026221202168	-3.23845303017948\\
-4.04580440875587	-3.22204163116993\\
-4.11816598291268	-3.2033255661376\\
-4.18007743256084	-3.18303890969571\\
-4.23360073644704	-3.16164924418725\\
-4.28031038105244	-3.13944609773853\\
-4.32142116360366	-3.11659782607227\\
-4.35787908656229	-3.09318838386657\\
-4.39042653213797	-3.06924110152558\\
-4.41964935154084	-3.04473391305451\\
-4.44601108058334	-3.01960882400614\\
-4.4698778459518	-2.99377736739554\\
-4.4915363995599	-2.96712312797632\\
-4.51120693577764	-2.9395019735828\\
-4.52905178805701	-2.91074032479108\\
-4.5451806848625	-2.88063156228785\\
-4.55965291049511	-2.84893047509702\\
-4.57247641804878	-2.81534546188049\\
-4.58360363797199	-2.77952798499219\\
-4.59292337262107	-2.74105851436077\\
-4.60024771005255	-2.69942785082767\\
-4.60529225253164	-2.65401223980981\\
-4.60764702219832	-2.60404001140737\\
-4.60673400038473	-2.54854652175554\\
-4.60174509902098	-2.48631279899729\\
-4.59155101434331	-2.4157813597352\\
-4.5745662013086	-2.33493999874156\\
-4.54854715071333	-2.2411609192126\\
-4.51028900271617	-2.13097878721842\\
-4.45516831287353	-1.99978905489944\\
-4.37645900239336	-1.84145315801323\\
-4.26433536047132	-1.64782626160644\\
-4.10451173349945	-1.40831628899047\\
-3.87667368401192	-1.10982577643923\\
-3.55351055876737	-0.737953295795312\\
-3.10267774522643	-0.281144605569967\\
-2.49611814370087	0.260290711235291\\
-1.7301505261287	0.861715239181722\\
-0.84776523000926	1.46888239758364\\
0.0629472957477424	2.0145391317581\\
0.907080160091894	2.45060873194818\\
1.62516746798835	2.76588732943137\\
2.20290107764375	2.9770626761349\\
2.65467947580856	3.11042204052515\\
3.00487204212609	3.19003972564353\\
3.27722841285825	3.23399961853859\\
3.49116912628829	3.25470361386584\\
3.66143984979885	3.26030969750488\\
};
\addplot [color=mycolor2, forget plot]
  table[row sep=crcr]{%
1.781882940209	2.77322293973926\\
1.79588260529314	2.77278702308228\\
1.80841097659117	2.77159736069738\\
1.81974138446683	2.76978961057895\\
1.83008806950595	2.76745684099947\\
1.83962129977994	2.76466086069903\\
1.8484781286801	2.7614397057913\\
1.8567701602796	2.75781253548015\\
1.86458921862207	2.75378271177296\\
1.87201151627499	2.7493395386105\\
1.87910071970149	2.74445893837388\\
1.88591017453311	2.73910320409754\\
1.89248445759564	2.73321985568351\\
1.89886034778057	2.72673952788851\\
1.90506724197861	2.71957270969144\\
1.91112697391478	2.71160502049679\\
1.91705290988569	2.70269052456319\\
1.92284807805368	2.69264231552977\\
1.92850190848564	2.68121919108762\\
1.93398487063834	2.66810658861676\\
1.93923980603726	2.65288890189017\\
1.94416790475438	2.63500855607246\\
1.94860575705665	2.61370425569063\\
1.95228712340011	2.58791565836191\\
1.95477779378775	2.55613250846652\\
1.95536164810143	2.51614941394148\\
1.95283551157929	2.46465606258949\\
1.94512845155244	2.39653378330288\\
1.92857473641034	2.30362121407213\\
1.89649698060454	2.17253292220582\\
1.83645882975982	1.98093216469277\\
1.72533668269092	1.69210819871046\\
1.52302781961484	1.25184668083432\\
1.17606665822847	0.607304844084989\\
0.663584658507134	-0.216879808826287\\
0.0724679910453545	-1.04778855701782\\
-0.450405259793972	-1.69358161491012\\
-0.83620668374121	-2.11316810285577\\
-1.10040794624059	-2.36628461285566\\
-1.28021043293575	-2.51785112424728\\
-1.40550257695371	-2.61050165599441\\
-1.49571971739991	-2.6687177443404\\
-1.56287639808906	-2.7062287971572\\
-1.61442271385558	-2.73085860808695\\
-1.65508183507587	-2.74719838369212\\
-1.68793327545703	-2.75803071783548\\
-1.71504330725383	-2.76509519539211\\
-1.73783668099559	-2.76951037359158\\
-1.75732083822314	-2.77201338351128\\
-1.77422466624992	-2.77310013824509\\
-1.78908660957274	-2.77310975999451\\
-1.80231201968687	-2.77227676672347\\
-1.81421135362129	-2.77076408710107\\
-1.82502616337916	-2.76868436034322\\
-1.83494712521408	-2.76611388550085\\
-1.84412676764258	-2.76310183373794\\
-1.85268859658263	-2.75967631826468\\
-1.86073372247673	-2.75584830696964\\
-1.8683457190156	-2.75161398652662\\
-1.87559419994801	-2.74695594516419\\
-1.88253743820329	-2.74184337704484\\
-1.88922423869972	-2.73623138924254\\
-1.89569519247438	-2.73005938925365\\
-1.90198337088657	-2.72324842875773\\
-1.90811445297349	-2.71569726071395\\
-1.91410620476902	-2.70727671131428\\
-1.91996713155655	-2.69782174700842\\
-1.92569398016566	-2.68712028512562\\
-1.93126754139454	-2.67489728134176\\
-1.93664582735851	-2.66079180389336\\
-1.94175305655896	-2.64432345531191\\
-1.94646174838001	-2.62484223712471\\
-1.95056317919443	-2.6014520545482\\
-1.9537176319937	-2.57289118093351\\
-1.95536854304165	-2.53734057205639\\
-1.95459018715304	-2.49210796544402\\
-1.94980926975556	-2.43309269516865\\
-1.93828054844667	-2.35385588598758\\
-1.91507345139156	-2.24397836189112\\
-1.87109236299913	-2.08618419617105\\
-1.78932984678071	-1.85168026808045\\
-1.63883898093533	-1.49478489388947\\
-1.37068276603309	-0.956723276792841\\
-0.938084355572819	-0.210099538651129\\
-0.368170161346866	0.646002603077406\\
0.204498185789562	1.40023954540039\\
0.660626177511073	1.92880903266141\\
0.981154247279146	2.2560043438161\\
1.19872262954689	2.45157412015697\\
1.34824136987672	2.56968263421268\\
1.45410936810425	2.64287391127373\\
1.53163409357617	2.68947336578912\\
1.59025726812783	2.71981211549013\\
1.63589119605128	2.73985964858191\\
1.672335905501	2.75317592808163\\
1.7021050431826	2.76195291278141\\
1.72690876815852	2.76758092987106\\
1.74794154747409	2.77096544098006\\
1.76605796463212	2.77270965210016\\
1.781882940209	2.77322293973926\\
};
\addplot [color=mycolor1, forget plot]
  table[row sep=crcr]{%
4.99561704614432	3.93093778001022\\
5.12441715718723	3.92700594018682\\
5.22749359751082	3.91727727481723\\
5.31147431550237	3.90392417940075\\
5.38100855098295	3.88828361280772\\
5.4394192891387	3.87118259644813\\
5.48912451149793	3.853130710441\\
5.53191316501664	3.83443589164418\\
5.56912894315978	3.81527522357547\\
5.60179472704873	3.79573879287683\\
5.63069819801756	3.77585708918583\\
5.65645159570325	3.75561812131712\\
5.67953394460403	3.73497793935292\\
5.70032116110036	3.71386678908289\\
5.71910759992558	3.69219224168065\\
5.73612139531802	3.6698400907359\\
5.75153514953909	3.64667344917968\\
5.7654729662991	3.62253022618498\\
5.77801442013848	3.59721896598886\\
5.78919572749572	3.57051285010664\\
5.79900808959564	3.54214147165365\\
5.80739286480359	3.51177975532787\\
5.81423284683676	3.4790330822595\\
5.81933840613643	3.44341723417179\\
5.82242649126247	3.40433111923476\\
5.82308931925251	3.36101926251882\\
5.82074773116208	3.31251954517394\\
5.81458117655192	3.25758934794285\\
5.80342129046852	3.19459959207491\\
5.7855875771488	3.12138036097956\\
5.75862922647238	3.034992555466\\
5.71891201929251	2.93138554196882\\
5.66094598091564	2.80487890431387\\
5.57627654710481	2.64737706978255\\
5.45164906545852	2.44719972533898\\
5.26602389169874	2.18744287372611\\
4.98604568915978	1.8440724380056\\
4.56051953116549	1.385080343331\\
3.91861389651914	0.775217544816685\\
2.98722249841715	-0.0039109304312002\\
1.75058309585665	-0.912657960913425\\
0.328315571224359	-1.82653552024029\\
-1.04986979985048	-2.59477729602087\\
-2.19910607754332	-3.14384399650946\\
-3.06935198955974	-3.49401988568186\\
-3.7005382255588	-3.70261632449116\\
-4.15499533889883	-3.82140591624197\\
-4.48601544731098	-3.88578484184051\\
-4.7319291316338	-3.91755636433422\\
-4.91871783456797	-3.92969230616816\\
-5.06374868767354	-3.92987792432078\\
-5.1787004733027	-3.92270577353361\\
-5.27154301172934	-3.91095528868635\\
-5.34781428514036	-3.89632892311543\\
-5.4114358091086	-3.87987800734067\\
-5.46523604475778	-3.86225253481791\\
-5.51129068081925	-3.84385014702867\\
-5.55114736942836	-3.82490650236635\\
-5.58597664418855	-3.80555091596297\\
-5.61667493956786	-3.78584100285091\\
-5.64393599650679	-3.76578435055823\\
-5.66830102740274	-3.74535198599743\\
-5.69019434122226	-3.72448650066768\\
-5.70994881233312	-3.70310656504833\\
-5.72782408732873	-3.68110886870713\\
-5.7440194449165	-3.6583680807747\\
-5.75868256018706	-3.63473512752232\\
-5.77191495372025	-3.61003386380834\\
-5.78377454721926	-3.58405602983503\\
-5.79427544279993	-3.556554201649\\
-5.80338474493174	-3.52723223388266\\
-5.81101590439257	-3.49573242376901\\
-5.81701762371723	-3.46161825359495\\
-5.82115673986917	-3.42435103214597\\
-5.82309256192751	-3.38325795858695\\
-5.82233867563938	-3.3374879231185\\
-5.81820587077501	-3.28594949383599\\
-5.80971597542463	-3.22722262281281\\
-5.79546989489136	-3.15943099493215\\
-5.77344210597716	-3.08005461689224\\
-5.74065481114691	-2.98565064564301\\
-5.6926519686067	-2.87143251714126\\
-5.62263692343476	-2.73063152630738\\
-5.52004510004058	-2.55353455835752\\
-5.36819308466233	-2.32608261267201\\
-5.14054916221915	-2.02803325817305\\
-4.79549424832186	-1.63128596818774\\
-4.27157764880956	-1.10094716298679\\
-3.49246762041912	-0.406261979656713\\
-2.40332781996221	0.448023197706848\\
-1.04933145655836	1.37913204945238\\
0.380773929650926	2.23607011864938\\
1.65928387635527	2.89706136864684\\
2.66777140341069	3.34038976153233\\
3.41090951800739	3.61241114041432\\
3.94617389039209	3.77064798880819\\
4.33320444411984	3.85877485260488\\
4.61774349464813	3.90477787832598\\
4.83146612136199	3.92550583087053\\
4.99561704614432	3.93093778001022\\
};
\addplot [color=mycolor2, forget plot]
  table[row sep=crcr]{%
1.41336186657461	2.6452476547433\\
1.42661771092012	2.6448335028727\\
1.438711252913	2.6436839279649\\
1.44984844049261	2.64190594243159\\
1.46019496974537	2.6395722439554\\
1.46988592532474	2.63672910925978\\
1.47903283490991	2.63340160243155\\
1.48772889170151	2.62959686997006\\
1.49605285324855	2.62530599884741\\
1.50407196073714	2.62050471464515\\
1.51184410942881	2.61515305224684\\
1.51941941835957	2.60919401406851\\
1.52684128198338	2.60255111996953\\
1.53414692742136	2.59512463047509\\
1.54136743812141	2.58678607029672\\
1.54852712580497	2.57737046498136\\
1.55564201997255	2.56666538876986\\
1.56271706952095	2.55439544060589\\
1.56974136514771	2.54020000811172\\
1.57668020608856	2.52360095861468\\
1.58346198732936	2.50395488348667\\
1.58995636420672	2.48038113089167\\
1.59593735844811	2.45165102966057\\
1.60101980835639	2.41601348883995\\
1.60454742630167	2.37091401188447\\
1.60539083148236	2.31253186959236\\
1.60157464345816	2.23500407079215\\
1.58957687332405	2.12911610833666\\
1.56301143478508	1.9801432277843\\
1.5102532766994	1.76463817519863\\
1.41081821954095	1.44727205128183\\
1.23288596616825	0.984345860164834\\
0.943081411046606	0.352273078517919\\
0.54428566688769	-0.391140171268018\\
0.107904047733924	-1.09554150842607\\
-0.276044760851313	-1.63657136712392\\
-0.569184196744573	-1.99892328105359\\
-0.77995959606827	-2.22811087339792\\
-0.930346758973861	-2.37209991875548\\
-1.03947457874975	-2.46402444534223\\
-1.12072825956013	-2.52406730683613\\
-1.18290548049581	-2.56416308963738\\
-1.23174035972648	-2.59142346698548\\
-1.27101822274564	-2.61018011835085\\
-1.30329024585911	-2.62314182793299\\
-1.3303159546295	-2.63204789657639\\
-1.3533370722357	-2.63804298620778\\
-1.37324926014167	-2.64189713696169\\
-1.39071186785787	-2.64413816644854\\
-1.40621953027101	-2.64513332474643\\
-1.42014985394465	-2.64514082425715\\
-1.43279581960538	-2.64434303531236\\
-1.44438821946604	-2.64286824669118\\
-1.45511147030128	-2.64080511765368\\
-1.46511494034323	-2.63821233899968\\
-1.47452118274717	-2.63512506370108\\
-1.48343199755177	-2.63155908346106\\
-1.49193294054518	-2.62751335986359\\
-1.50009669731347	-2.62297127778391\\
-1.50798560497077	-2.61790082096134\\
-1.51565350807117	-2.61225374145689\\
-1.52314706266683	-2.6059636828706\\
-1.53050654150842	-2.59894310284286\\
-1.5377661337854	-2.59107870464897\\
-1.54495366368046	-2.5822249073791\\
-1.55208955895927	-2.57219462609524\\
-1.55918476169539	-2.56074624579452\\
-1.56623705092564	-2.54756507140195\\
-1.573224876085	-2.53223657740225\\
-1.58009716105139	-2.51420721748646\\
-1.5867564075857	-2.49272594887438\\
-1.59303037348538	-2.46675518971048\\
-1.59862377892076	-2.43483222422706\\
-1.60303420978513	-2.39484846990763\\
-1.60540220859596	-2.34368979276673\\
-1.60423757597317	-2.27663824466107\\
-1.59690900318884	-2.18636364362759\\
-1.57868173742978	-2.06123237273783\\
-1.54093102719399	-1.88261546525284\\
-1.46810249276303	-1.62137238292075\\
-1.33400160379627	-1.23656406937439\\
-1.10326361017691	-0.688505376792297\\
-0.754348062540462	0.0136382320891431\\
-0.324477087888203	0.758366136414363\\
0.0945266002362541	1.38980765530617\\
0.434093811825414	1.83775491680209\\
0.683522071725052	2.12686100309367\\
0.8613696880509	2.30832381252633\\
0.989104532384684	2.42306408013362\\
1.08294774530413	2.49713991499516\\
1.15378656170739	2.54608051781412\\
1.20871846719522	2.57907857516757\\
1.25239176239966	2.60166685117974\\
1.28790527516386	2.61725928393709\\
1.3173715371126	2.62801946673208\\
1.34226451525509	2.63535427264229\\
1.36363601743509	2.64020015530037\\
1.3822526363932	2.64319329085787\\
1.39868416267362	2.64477324354987\\
1.41336186657461	2.6452476547433\\
};
\addplot [color=mycolor1, forget plot]
  table[row sep=crcr]{%
6.84088886676008	4.99913208029814\\
6.96761917857306	4.99527446465504\\
7.06755331315122	4.98584932276291\\
7.14802261607397	4.97305911339979\\
7.21401975286968	4.95821728261124\\
7.26902978728066	4.94211411572269\\
7.31554052669344	4.92522402903217\\
7.35536390507287	4.90782600632662\\
7.38984331799043	4.89007512858044\\
7.41999042362753	4.87204581526907\\
7.4465772713189	4.85375826900637\\
7.47019948081834	4.83519467923873\\
7.49132023875733	4.81630899908268\\
7.51030129726572	4.79703254482266\\
7.52742495531667	4.77727674831561\\
7.54290961255317	4.75693383296865\\
7.5569205823831	4.73587582594179\\
7.56957724233974	4.71395207172416\\
7.58095716552604	4.69098521945959\\
7.59109753889791	4.66676548027988\\
7.5999938747872	4.64104276040482\\
7.60759571140593	4.61351603914356\\
7.61379862175913	4.58381903830862\\
7.6184313363036	4.55150076352845\\
7.62123602336253	4.51599879927524\\
7.62183858316341	4.47660216315016\\
7.61970388775799	4.43239882863183\\
7.61406769191097	4.38220029407415\\
7.60383144924698	4.32443108425763\\
7.58739662284487	4.25696352991992\\
7.56239769899336	4.1768652550648\\
7.52526100347198	4.08000429228032\\
7.47045574746095	3.96041707443767\\
7.38918722760005	3.80927485345324\\
7.26705835616366	3.61316623865749\\
7.07981294985811	3.35124052435176\\
6.78562883880548	2.99063248318109\\
6.31206390072326	2.4801833081198\\
5.53922240998824	1.74661149592949\\
4.30165628022422	0.712525630456547\\
2.48557351533814	-0.620688304855937\\
0.26402085414709	-2.04765566869058\\
-1.8687359569853	-3.2372039172557\\
-3.52815139697121	-4.03095814698191\\
-4.67630145924028	-4.49358687649038\\
-5.44305045173526	-4.7473157681752\\
-5.96001472258145	-4.8826087065026\\
-6.31831910812099	-4.95237714613542\\
-6.57479182115228	-4.98555661624868\\
-6.76422057514554	-4.99788833910659\\
-6.90818573778395	-4.99808676492652\\
-7.02040997793232	-4.99109349776442\\
-7.10986595001333	-4.97977722865751\\
-7.18258420021252	-4.96583597628711\\
-7.24272375925819	-4.95028803020279\\
-7.29322133832692	-4.93374643429515\\
-7.33619506303057	-4.91657652308681\\
-7.37320185482429	-4.8989885047452\\
-7.4054053934111	-4.88109297144779\\
-7.43368810964642	-4.86293468270704\\
-7.45872731796898	-4.84451327772963\\
-7.48104784848783	-4.82579590645338\\
-7.50105893333677	-4.80672470492933\\
-7.51908030042781	-4.78722084610942\\
-7.53536068244905	-4.76718618350808\\
-7.55009083275759	-4.74650306099718\\
-7.56341240252416	-4.72503256788109\\
-7.57542352380168	-4.70261130382071\\
-7.58618156519308	-4.67904653784841\\
-7.59570321451892	-4.65410946616483\\
-7.60396174412403	-4.62752606413669\\
-7.61088097851963	-4.59896475411069\\
-7.61632504963485	-4.56801972508052\\
-7.62008240431654	-4.53418817181828\\
-7.62184158267811	-4.49683885655334\\
-7.62115478012173	-4.4551680487666\\
-7.61738273052627	-4.40813675180086\\
-7.60961026522076	-4.35437963155908\\
-7.59651464829314	-4.29207025773525\\
-7.57615588237226	-4.21871742268348\\
-7.54563462676158	-4.13085028486\\
-7.50051934440824	-4.02352022392346\\
-7.43386033241791	-3.88949461361287\\
-7.33444657504917	-3.71792630922736\\
-7.18365457895993	-3.4921354354706\\
-6.94970302787466	-3.18596166290699\\
-6.57746887226027	-2.75822063891058\\
-5.97271943598497	-2.14656008491193\\
-4.98915068767605	-1.27049336483544\\
-3.4647769938628	-0.0761671675852215\\
-1.39846942708153	1.34372974891551\\
0.845794119376638	2.6886884672571\\
2.76685927831793	3.68281754683751\\
4.15924238449614	4.29571110156509\\
5.09825681476418	4.63989472459928\\
5.72610420536212	4.82573521306198\\
6.15479889280274	4.92346550355785\\
6.45671715430494	4.97233824045081\\
6.6763049251593	4.99366738199783\\
6.84088886676008	4.99913208029814\\
};
\addplot [color=mycolor2, forget plot]
  table[row sep=crcr]{%
0.523495985188513	2.21095812992688\\
0.543228512121635	2.21033588094888\\
0.562220859464411	2.20852524916379\\
0.58062348639751	2.20558242663241\\
0.59857107624324	2.20152953227212\\
0.616186173211566	2.19635696258384\\
0.6335821876191	2.19002390927179\\
0.650865900163185	2.18245712819836\\
0.668139552271565	2.1735478870409\\
0.685502566909632	2.16314685563153\\
0.703052896168851	2.15105651310548\\
0.720887929723092	2.13702040601345\\
0.739104808628203	2.12070827115511\\
0.757799850973994	2.10169559548784\\
0.777066575245817	2.07943556864067\\
0.796991447022335	2.05322052138817\\
0.817645880480452	2.02212875644123\\
0.839072043534282	1.98495110624118\\
0.861258399072199	1.94008964755294\\
0.884098302558147	1.88541917667427\\
0.907320920341571	1.81810169992184\\
0.930377965797729	1.73434916367601\\
0.952263194491732	1.62914930722371\\
0.971239355789409	1.4960241572878\\
0.984465785527444	1.3270165243409\\
0.987598438751026	1.11333546856789\\
0.974634416241003	0.847385341338023\\
0.938591949871787	0.526853328745189\\
0.873710716112844	0.160179721753477\\
0.778896374237475	-0.230310041134964\\
0.660066987294904	-0.612784015588797\\
0.528630790965447	-0.957467547825973\\
0.396777493547414	-1.24689744957127\\
0.273413498440531	-1.47776602135199\\
0.163031486303281	-1.6561500367302\\
0.0667094469319491	-1.79176889067705\\
-0.0163980118246023	-1.89431828124055\\
-0.0879334239880009	-1.97194566097251\\
-0.149699133735178	-2.0309668534891\\
-0.203365362938089	-2.07609011987583\\
-0.250369196107044	-2.11076539943113\\
-0.291903635828865	-2.13750761633118\\
-0.328942641967419	-2.15815283636643\\
-0.362276439623747	-2.17404814008715\\
-0.392546170745644	-2.18618812408863\\
-0.420273998952583	-2.19531176521621\\
-0.445887865276266	-2.20197089826355\\
-0.469741307298921	-2.20657867192898\\
-0.492129130424326	-2.20944393400595\\
-0.51329974773488	-2.21079569114687\\
-0.533464906173632	-2.21080050167315\\
-0.552807387145015	-2.20957476137215\\
-0.571487146611918	-2.2071932176451\\
-0.589646255366341	-2.20369460979581\\
-0.607412915652489	-2.19908502012294\\
-0.624904762922485	-2.19333928744042\\
-0.642231607068395	-2.18640064970422\\
-0.6594977214769	-2.1781786212345\\
-0.676803745849656	-2.16854495220916\\
-0.69424822429159	-2.15732734415148\\
-0.711928746334305	-2.14430038351473\\
-0.729942584674714	-2.12917287962482\\
-0.748386612912459	-2.11157041820197\\
-0.767356112690995	-2.09101142034797\\
-0.786941798226588	-2.06687426702275\\
-0.807223923936087	-2.0383520342906\\
-0.828261576687304	-2.00439000913586\\
-0.850073992278704	-1.96359939714325\\
-0.872608674221088	-1.91413867617342\\
-0.895687814770749	-1.85355268365995\\
-0.918919604020422	-1.77856110763088\\
-0.941554581125384	-1.6847986646771\\
-0.962261669610689	-1.56654291960342\\
-0.978803465833006	-1.41655046193693\\
-0.987632129793905	-1.22629992368928\\
-0.983559095423338	-0.987224987260853\\
-0.95992427292537	-0.693728548593595\\
-0.9099752585336	-0.348209238467667\\
-0.829844399657837	0.0339699757827997\\
-0.721851236154917	0.424610843801142\\
-0.595120868130539	0.791297011438935\\
-0.462094678343811	1.10958947459289\\
-0.333655905694098	1.3694178283083\\
-0.216470558877756	1.57294120700516\\
-0.113136514165607	1.72867233978889\\
-0.0235974610654696	1.84662458847378\\
0.0534986437817769	1.93581096326671\\
0.119928196544504	2.00345410619619\\
0.177449005002451	2.05502469963258\\
0.22761999764002	2.09455801098953\\
0.271754484623795	2.12499999714405\\
0.310931796261443	2.14849852256881\\
0.346029505257319	2.16662510211742\\
0.377759205625238	2.18053628765935\\
0.4066991226322	2.1910887605286\\
0.433321539256857	2.19892078291678\\
0.458014996438643	2.2045097844662\\
0.481101933251353	2.20821316547787\\
0.502852595133849	2.21029729226656\\
0.523495985188513	2.21095812992688\\
};
\addplot [color=mycolor1, forget plot]
  table[row sep=crcr]{%
6.86250796792293	5.01216327431788\\
6.98926231728329	5.00830501201048\\
7.0892040971588	4.99887920200613\\
7.1696723709199	4.98608919075086\\
7.2356639362686	4.97124863638639\\
7.29066611921641	4.95514778446123\\
7.33716798431924	4.93826093255989\\
7.37698216855827	4.92086693575256\\
7.41145245879218	4.90312076171995\\
7.44159072664735	4.8850967394826\\
7.46816913277388	4.86681500433133\\
7.49178334982416	4.84825769934232\\
7.51289658462578	4.82937874950955\\
7.53187059220708	4.8101094587785\\
7.5489876665659	4.790361260487\\
7.56446620043385	4.7700263919257\\
7.57847150229701	4.74897690563372\\
7.59112294956573	4.72706218251218\\
7.60249812227665	4.70410491905285\\
7.61263422345491	4.67989538484688\\
7.62152679279647	4.65418355600433\\
7.62912540959612	4.62666849341812\\
7.63532570446324	4.5969840120384\\
7.63995648517121	4.56467922102285\\
7.64276002044572	4.5291918154402\\
7.64336233688809	4.4898109229457\\
7.6412284595908	4.44562461062275\\
7.63559431755128	4.39544442207369\\
7.62536153911076	4.33769481597923\\
7.60893170603898	4.27024782131662\\
7.58393922779236	4.19017027676805\\
7.54680982482963	4.09332844236527\\
7.49201078226915	3.97375494172735\\
7.41074223820117	3.82261292741202\\
7.28859398844889	3.62647362156677\\
7.10127677227243	3.364448227552\\
6.80688429362408	3.0035862582056\\
6.33275976369098	2.49253691535988\\
5.55848013628344	1.75760613116161\\
4.31749294271232	0.720672658969268\\
2.49461507568364	-0.617516179954371\\
0.263345568688976	-2.05071979992666\\
-1.8785328447942	-3.245362799024\\
-3.5438740067486	-4.04196088661383\\
-4.69511885692381	-4.50584250495883\\
-5.46335414679059	-4.76006617664413\\
-5.98102449960256	-4.89554530075052\\
-6.33966919335148	-4.96538068726368\\
-6.59630802588342	-4.99858199844547\\
-6.78581731532883	-5.01091915477763\\
-6.92981960632722	-5.01111774157195\\
-7.04205839289073	-5.00412363463622\\
-7.1315169872655	-4.99280707632652\\
-7.20423156477816	-4.97886655640214\\
-7.26436420644563	-4.96332041832582\\
-7.31485330757953	-4.94678161368626\\
-7.35781793374445	-4.92961534825615\\
-7.3948155316173	-4.9120317075125\\
-7.42701007098371	-4.8941411814341\\
-7.45528413804713	-4.87598845082904\\
-7.48031512524895	-4.85757309838982\\
-7.50262789680656	-4.83886223714095\\
-7.52263169492403	-4.81979798318208\\
-7.54064624555851	-4.8003015042472\\
-7.55692027486454	-4.78027466167686\\
-7.57164452987791	-4.75959981905813\\
-7.58496065894079	-4.73813809662397\\
-7.59696679724758	-4.71572613590784\\
-7.60772032464288	-4.69217125881124\\
-7.617237950381	-4.66724472566727\\
-7.62549298070531	-4.64067258756152\\
-7.63240928909447	-4.61212335426371\\
-7.63785107451925	-4.58119131334615\\
-7.64160687227093	-4.54737376726125\\
-7.64336533556924	-4.51003958990812\\
-7.64267879968886	-4.46838515534314\\
-7.63890816347927	-4.42137154178832\\
-7.63113843688252	-4.3676334154589\\
-7.61804704135371	-4.30534418336554\\
-7.59769402623181	-4.23201213696242\\
-7.5671797709553	-4.1441652432129\\
-7.52207161887795	-4.03685227518945\\
-7.45541668020137	-3.90283505324939\\
-7.35599568951142	-3.73125459015708\\
-7.2051643343779	-3.50540534944489\\
-6.97108861704537	-3.19907014224744\\
-6.59851096802524	-2.77093651021558\\
-5.99285900701924	-2.15836741926386\\
-5.00703994044869	-1.28030472689602\\
-3.4776900948767	-0.0820928732758431\\
-1.40284278372905	1.34366140761687\\
0.85134661085044	2.69457031415574\\
2.78004453523949	3.69265884948226\\
4.17678575758048	4.3074785408114\\
5.11795459925128	4.65245605656726\\
5.74682533859377	4.83860148115811\\
6.17600915738681	4.93644425192202\\
6.47816500031136	4.98535593273646\\
6.69786876188027	5.00669659737774\\
6.86250796792293	5.01216327431788\\
};
\addplot [color=mycolor2, forget plot]
  table[row sep=crcr]{%
-0.733575921492504	1.31460374341902\\
-0.632596108039127	1.3113410779402\\
-0.520663702370314	1.3005890110674\\
-0.397263074514887	1.2807737843542\\
-0.262341623792816	1.25022656161158\\
-0.11652851656645	1.20733710372774\\
0.0386574440584269	1.15078286280803\\
0.200679110745031	1.07981345555107\\
0.366013660074528	0.994532769103677\\
0.530406661832258	0.896088323134107\\
0.689338991169707	0.786677146174762\\
0.838605846799504	0.6693274655713\\
0.974845201924293	0.547500688560829\\
1.095867236989	0.424630275227481\\
1.20072115263929	0.303728526329512\\
1.28953654070531	0.187146049008567\\
1.36323727351282	0.076498504484224\\
1.42322986832842	-0.0272778744932675\\
1.47113557370203	-0.12379976936684\\
1.50859537167969	-0.213089685086319\\
1.53714783701748	-0.295440389365299\\
1.5581653382286	-0.371305494710742\\
1.57283062677761	-0.441218943712308\\
1.582138147318	-0.505740022815961\\
1.58690854061136	-0.565418369279737\\
1.58780877458732	-0.620773443737429\\
1.58537339560882	-0.672283856281868\\
1.58002447744766	-0.720383054726347\\
1.57208914862459	-0.765458892081875\\
1.56181433196347	-0.807855384239133\\
1.54937872858848	-0.847875547553941\\
1.53490226248587	-0.88578460612647\\
1.51845326346976	-0.921813123485746\\
1.50005366306014	-0.956159781172203\\
1.47968244310377	-0.988993626619511\\
1.45727753006904	-1.02045566565407\\
1.43273627943341	-1.05065969493691\\
1.40591465056035	-1.0796922657676\\
1.37662513722201	-1.10761164827412\\
1.34463349691367	-1.13444562723217\\
1.30965431967807	-1.16018790954467\\
1.27134550445155	-1.18479286088385\\
1.22930178423368	-1.20816821926737\\
1.18304758612331	-1.23016536542467\\
1.13202976680905	-1.25056668239289\\
1.07561118349157	-1.26906954588727\\
1.01306671797197	-1.28526661767614\\
0.943584354520575	-1.29862247668178\\
0.866275297135439	-1.30844739031735\\
0.780198905747261	-1.31387044835238\\
0.684410242848602	-1.31381664823958\\
0.57803963763987	-1.3069960618567\\
0.460413542584466	-1.29191779453278\\
0.331221736278588	-1.26694601494493\\
0.190724547621685	-1.23041717488425\\
0.0399730394022008	-1.18083190374211\\
-0.119012709766546	-1.11711650257582\\
-0.283185610920674	-1.03891590754923\\
-0.448610699077059	-0.946841825316007\\
-0.610832441756026	-0.842579774836903\\
-0.765415585653945	-0.728782995388376\\
-0.908520369919128	-0.608753247524889\\
-1.03734588454356	-0.485994123255349\\
-1.15032942969175	-0.363769276058191\\
-1.2470907949759	-0.244779334817512\\
-1.32819617988291	-0.131007247923902\\
-1.39484812893842	-0.023716225666711\\
-1.44858972346348	0.0764520407182007\\
-1.49107166189482	0.169335335020153\\
-1.52389492325475	0.255106777306546\\
-1.54852001943411	0.334151396951494\\
-1.56622557067635	0.406971321366467\\
-1.57809896662283	0.47411865229053\\
-1.58504544890288	0.536151200579106\\
-1.58780615864932	0.593605370382616\\
-1.58697923972185	0.646981064609402\\
-1.5830406446606	0.696734560951497\\
-1.57636295534998	0.743276395173477\\
-1.567231530928	0.786972193951703\\
-1.5558578502988	0.828145082622651\\
-1.54239019348571	0.867078777152041\\
-1.52692191983323	0.904020796901718\\
-1.50949762482587	0.939185446898803\\
-1.49011743515793	0.972756349052911\\
-1.46873965915509	1.00488837603656\\
-1.4452819609737	1.03570887660166\\
-1.41962118013877	1.06531808828617\\
-1.39159187784586	1.09378861980928\\
-1.36098366219232	1.12116385507696\\
-1.32753733130181	1.14745508600307\\
-1.2909398841505	1.17263712414442\\
-1.25081849666718	1.19664207428242\\
-1.20673366580052	1.21935088267\\
-1.15817191813318	1.24058221179928\\
-1.1045388085493	1.2600781683996\\
-1.04515346275685	1.27748647074099\\
-0.979246726695719	1.29233887058039\\
-0.905966162524658	1.30402618397681\\
-0.824392727841595	1.31177134385085\\
-0.733575921492505	1.31460374341902\\
};
\addplot [color=mycolor1, forget plot]
  table[row sep=crcr]{%
4.82273228002968	3.83720887290188\\
4.95224699965661	3.83325358762484\\
5.05611778584741	3.82344889459213\\
5.14089177706792	3.80996895800022\\
5.21118152874184	3.79415795626632\\
5.27029535935223	3.77685073784405\\
5.32064745275308	3.75856365681821\\
5.36402814716335	3.73960996602969\\
5.40178480740073	3.72017066332652\\
5.43494478132041	3.70033854357982\\
5.46430026698535	3.68014580834553\\
5.49046771058867	3.65958136247208\\
5.51392987214387	3.63860147624619\\
5.53506587201003	3.61713604058954\\
5.55417272393345	3.59509176286132\\
5.57148068066435	3.57235310028877\\
5.58716392840006	3.54878136750211\\
5.60134761773018	3.52421220091616\\
5.61411181537904	3.49845136332545\\
5.62549263681689	3.47126869055441\\
5.63548052402979	3.4423897885073\\
5.64401531919005	3.41148485356697\\
5.65097740203965	3.3781536756374\\
5.65617363775665	3.34190544085221\\
5.65931612011444	3.30213130495984\\
5.65999052751267	3.2580667421089\\
5.65760906390394	3.20873920256256\\
5.65133996804031	3.15289434124797\\
5.64000063885971	3.0888905337548\\
5.62189313791285	3.01454582992094\\
5.59454673765195	2.92691277479322\\
5.55430807172166	2.82194311606146\\
5.49567845866715	2.69398488798758\\
5.41023066111837	2.5350299931434\\
5.28483775577398	2.33361496287792\\
5.09884485169681	2.07332964712349\\
4.81990247355855	1.73120544840765\\
4.39920671023066	1.27738149278302\\
3.77091717341521	0.680385231150418\\
2.86995021215103	-0.0733906543835908\\
1.68696680769167	-0.942802236466377\\
0.335242594647828	-1.8113829036668\\
-0.975850751363337	-2.54218196592073\\
-2.07722399358059	-3.06831338957807\\
-2.91971033690261	-3.40726568298498\\
-3.53681121128057	-3.61117424376579\\
-3.98482032351319	-3.72825990601219\\
-4.31328463216064	-3.79213151854898\\
-4.55853705626038	-3.82381182226538\\
-4.74555312990996	-3.83595915425882\\
-4.89120350145804	-3.8361435067725\\
-5.00692397817158	-3.82892209014364\\
-5.10056696603678	-3.81706943662609\\
-5.17761549046581	-3.80229342988045\\
-5.24196729344291	-3.7856532643666\\
-5.29644256945701	-3.76780633590028\\
-5.34311629806293	-3.74915634157748\\
-5.38353891570215	-3.72994353477927\\
-5.41888515330459	-3.71030051495013\\
-5.4500560003726	-3.69028708231215\\
-5.4777495889861	-3.66991210828325\\
-5.50251111476878	-3.649147162936\\
-5.52476836016115	-3.62793476012899\\
-5.54485713109196	-3.60619295469908\\
-5.56303946198115	-3.58381733326525\\
-5.5795164823624	-3.56068099779297\\
-5.5944371838535	-3.53663284196546\\
-5.60790385994208	-3.5114941991508\\
-5.61997463426678	-3.48505375427219\\
-5.63066318889856	-3.45706042796672\\
-5.63993550533676	-3.42721373096009\\
-5.64770308990145	-3.39515081727613\\
-5.65381171345032	-3.36042909447345\\
-5.65802406972734	-3.32250271645461\\
-5.65999381797042	-3.28069049628806\\
-5.65922701260224	-3.23413158650851\\
-5.65502458055458	-3.18172344909521\\
-5.64639567479162	-3.12203380256001\\
-5.63192534913398	-3.0531737927659\\
-5.6095692067967	-2.97261265973941\\
-5.57632924476756	-2.87690331670947\\
-5.52773361511053	-2.76127190280889\\
-5.45699012819906	-2.61900185992461\\
-5.35359971688748	-2.44051986601403\\
-5.20110634735667	-2.21209661788149\\
-4.97360823102457	-1.91421992438641\\
-4.63105640072421	-1.52031875631979\\
-4.11552958912839	-0.998417660583009\\
-3.35732869804741	-0.322290494927053\\
-2.30994690716635	0.499343820936391\\
-1.01998239562171	1.38648817003772\\
0.338534070191457	2.20051889899191\\
1.55857214384513	2.83121730922997\\
2.52985081867727	3.25812564683877\\
3.25294571500252	3.52276759202078\\
3.7785533816478	3.67812488620809\\
4.16142077655795	3.76529004425864\\
4.44452431404327	3.81105331235671\\
4.65811579031653	3.83176414287705\\
4.82273228002968	3.83720887290188\\
};
\addplot [color=mycolor2, forget plot]
  table[row sep=crcr]{%
-0.409091368008181	0.847807163764681\\
0.0363872777318144	0.833551248542955\\
0.4817250507643	0.791124827733238\\
0.888249773018939	0.726348923044037\\
1.23188738207708	0.649090012089854\\
1.50689824214324	0.568693910692423\\
1.71996212285585	0.491453974327295\\
1.88260397700221	0.420525765118745\\
2.00639290543533	0.356905955764647\\
2.10100336756632	0.300416597818771\\
2.17388392469238	0.250362608748835\\
2.2305535888983	0.205892398119981\\
2.27503824215836	0.166169537837721\\
2.31026475499303	0.130441927877661\\
2.33836800502053	0.0980607706484964\\
2.36091562510899	0.0684772729283638\\
2.37906807968813	0.0412308829428764\\
2.39369148912165	0.0159354581143135\\
2.40543701739583	-0.00773391989835962\\
2.41479689131733	-0.0300525119747493\\
2.42214411567697	-0.0512556852703417\\
2.42776075688613	-0.071547013686774\\
2.43185813181773	-0.0911049652894258\\
2.43459118001494	-0.110088443142424\\
2.43606856940276	-0.128641425045538\\
2.43635958053514	-0.146896915618108\\
2.43549845661457	-0.164980392485317\\
2.4334866433849	-0.183012901412597\\
2.43029313715	-0.201113934326957\\
2.42585298260436	-0.219404209021938\\
2.4200637916992	-0.238008458984596\\
2.41277996830448	-0.25705833460852\\
2.40380409637995	-0.276695510518368\\
2.39287465074495	-0.297075083466383\\
2.37964877707909	-0.318369323368185\\
2.36367830171839	-0.34077179180409\\
2.34437628718835	-0.364501740005801\\
2.32097022868314	-0.389808489635322\\
2.29243623898783	-0.416975085835033\\
2.25740613658626	-0.446319707981462\\
2.21403617770271	-0.478191786707268\\
2.15982264227314	-0.512956889308037\\
2.09134735282723	-0.550959152568429\\
2.00394095050149	-0.592440809000024\\
1.89127791599731	-0.637383660783571\\
1.74499861443862	-0.685218755545888\\
1.5546537323195	-0.73434281979572\\
1.30865514096727	-0.781435534003174\\
0.99738778007668	-0.820806041810157\\
0.619400593165738	-0.84450103591156\\
0.188952459819122	-0.844293417387755\\
-0.261696494489053	-0.815644863467068\\
-0.69179486850701	-0.760951163274546\\
-1.0686755863563	-0.688639413498638\\
-1.37774703986944	-0.608809307923861\\
-1.6204938645077	-0.529418433597329\\
-1.80683005131328	-0.455100938365687\\
-1.94869725624527	-0.387795309093099\\
-2.05682977985848	-0.327806702640065\\
-2.13977481288561	-0.27463742208824\\
-2.20396352356178	-0.227483314092969\\
-2.25411439088667	-0.18548691140633\\
-2.29365976976601	-0.147849323331971\\
-2.32509766189118	-0.113869568607998\\
-2.35025571162444	-0.0829500057831151\\
-2.37048128221355	-0.0545876561121953\\
-2.38677592984353	-0.0283608978100306\\
-2.39989003031857	-0.00391575783331045\\
-2.41038943186043	0.0190465212089434\\
-2.418702593801	0.0407801420528644\\
-2.42515408608277	0.0615036573149588\\
-2.42998848383307	0.0814073298086471\\
-2.43338741492735	0.100659214206848\\
-2.43548164095655	0.119410218566654\\
-2.43635944668375	0.137798375621783\\
-2.4360721894412	0.15595252101642\\
-2.43463755492039	0.173995546059013\\
-2.43204083528246	0.192047368630618\\
-2.42823435741328	0.210227747942568\\
-2.42313501838075	0.228659056246865\\
-2.4166197095217	0.24746911208251\\
-2.40851820720412	0.266794173188343\\
-2.39860284992443	0.286782179523592\\
-2.38657397207145	0.307596322070028\\
-2.37203957427972	0.329418980482539\\
-2.35448700745897	0.352456002023506\\
-2.33324343183192	0.376941147473959\\
-2.30742034857198	0.403140234953994\\
-2.2758354289806	0.431353934491819\\
-2.23690205457826	0.46191705234338\\
-2.18847352812088	0.495190032571688\\
-2.12762571293684	0.53153448224751\\
-2.05036208875032	0.571257487717239\\
-1.95123794787654	0.614497646681818\\
-1.8229482566059	0.661008445474991\\
-1.65605455441742	0.709778029129612\\
-1.43931539355192	0.758438100532923\\
-1.16155468723619	0.802541670170139\\
-0.816285878889351	0.835160767177662\\
-0.409091368008183	0.84780716376468\\
};
\addplot [color=mycolor1, forget plot]
  table[row sep=crcr]{%
3.32646336156982	3.11950096423544\\
3.4673413537824	3.11517424948534\\
3.58374016942398	3.10417038463928\\
3.68108791118803	3.08867944763077\\
3.76345208436014	3.07014401802894\\
3.83390080001395	3.04951196054361\\
3.89476794811634	3.02740122541234\\
3.94784519226144	3.00420724774397\\
3.994520472178	2.98017315169741\\
4.03587809210986	2.95543579797172\\
4.07277136817759	2.93005601422601\\
4.1058756325731	2.90403833369911\\
4.13572707628554	2.87734364507239\\
4.16275126496685	2.84989692604319\\
4.18728400030374	2.82159143566068\\
4.20958637530507	2.79229021225691\\
4.22985527769723	2.76182536074797\\
4.24823015242909	2.72999534805555\\
4.26479648414853	2.6965603133532\\
4.27958615827755	2.6612352082174\\
4.29257456593334	2.6236803840576\\
4.3036739942792	2.58348901526293\\
4.31272244412135	2.54017045864858\\
4.31946648004315	2.49312826914687\\
4.32353595839411	2.44163107470658\\
4.3244073661675	2.38477380401515\\
4.32135084517486	2.32142579094374\\
4.31335348121521	2.2501609803709\\
4.2990076892852	2.16916380032119\\
4.27634797012147	2.07610236547164\\
4.24261137132037	1.9679591029565\\
4.19388654870193	1.84080937778487\\
4.12460537108111	1.68954583103729\\
4.02682820492829	1.50757102235393\\
3.88930626934119	1.28654804716912\\
3.69644156479011	1.01645458273428\\
3.42765412937966	0.686491934124628\\
3.05850682441724	0.287838803306029\\
2.5661047161497	-0.180634561274986\\
1.94114780808001	-0.704170530007748\\
1.20371230481806	-1.2466740267169\\
0.4093959035994	-1.75723902570244\\
-0.367691869692907	-2.19015566537185\\
-1.06586467664211	-2.52327489742262\\
-1.65471167076941	-2.75980784569078\\
-2.13231136223699	-2.91734668343342\\
-2.51239515758807	-3.01649960533499\\
-2.81340129671657	-3.07491758745073\\
-3.05271283203895	-3.1057586216582\\
-3.24466165286656	-3.11818053282271\\
-3.40036685358521	-3.11834771041054\\
-3.52822452426209	-3.11034882452227\\
-3.63451407501058	-3.09688160317952\\
-3.72393128262089	-3.07972373122979\\
-3.80000535223279	-3.06004519381678\\
-3.86540922520523	-3.03861246415104\\
-3.92218542946281	-3.01592151010757\\
-3.97190915567376	-2.99228458614822\\
-4.01580594520563	-2.96788707091709\\
-4.05483690688043	-2.94282478363867\\
-4.08976073081524	-2.91712844319707\\
-4.12117904297665	-2.8907795265247\\
-4.14956968723582	-2.86372024646818\\
-4.17531113792668	-2.8358593807395\\
-4.19870026860253	-2.80707503683381\\
-4.21996500508263	-2.7772150024272\\
-4.23927288051749	-2.74609502346451\\
-4.25673611981222	-2.71349511828296\\
-4.2724135596746	-2.67915383784852\\
-4.28630941748958	-2.64276019064817\\
-4.29836861884402	-2.60394274067567\\
-4.30846803824527	-2.5622551323156\\
-4.31640254833021	-2.51715696644361\\
-4.32186413763414	-2.46798850942894\\
-4.32441144040951	-2.41393711139262\\
-4.32342566597318	-2.35399237957271\\
-4.31804688193366	-2.28688602589242\\
-4.30708154515252	-2.21101082677981\\
-4.28886759794017	-2.12431132081434\\
-4.26107676054016	-2.02413701363562\\
-4.2204244039374	-1.90704797673904\\
-4.16224618983675	-1.76856567553848\\
-4.07989195483255	-1.60287585146393\\
-3.96389676347797	-1.40253200632372\\
-3.80096093637043	-1.15831208656481\\
-3.57300992369185	-0.859605020688629\\
-3.25719868151819	-0.496092458045106\\
-2.82880022614507	-0.0618785465303164\\
-2.26979422053179	0.437272257365636\\
-1.58377242826251	0.976075726774439\\
-0.809107763453254	1.50920352669317\\
-0.0142167475127106	1.98545362547987\\
0.729332634606697	2.3694845312444\\
1.37460171844811	2.65269650078429\\
1.90677228309312	2.8471362738584\\
2.33340934238541	2.97301567219907\\
2.67163537044435	3.04987357854242\\
2.93979454170461	3.09313057854113\\
3.1538417162047	3.11382835567295\\
3.32646336156982	3.11950096423544\\
};
\addplot [color=mycolor2, forget plot]
  table[row sep=crcr]{%
0.560167015110667	0.853244053900496\\
1.06707393089475	0.837614568300724\\
1.47679824178267	0.799019827265189\\
1.78965952386821	0.749429636354734\\
2.02226091251516	0.697266211987803\\
2.1941524994748	0.647072601752934\\
2.32194042341777	0.600766346427937\\
2.41808474955201	0.558840228174232\\
2.49146704461956	0.521121780938159\\
2.54830649589018	0.487177503668587\\
2.59295308266754	0.456506966732256\\
2.62847138587656	0.428627544501524\\
2.65704463224663	0.403106100218799\\
2.68024800218523	0.379566373857723\\
2.69923259434205	0.357686050860287\\
2.71484974774595	0.337190063938117\\
2.72773576375632	0.317843067151704\\
2.73837023202117	0.299442302209988\\
2.74711659991187	0.281811278308091\\
2.75425064661735	0.264794329660224\\
2.75998059101263	0.24825196618472\\
2.76446130416895	0.232056879954383\\
2.76780426913872	0.216090458887398\\
2.77008437581847	0.200239663848504\\
2.77134425673718	0.184394133581673\\
2.77159659503874	0.168443387858922\\
2.77082462449488	0.152273999962742\\
2.76898086106356	0.135766603317522\\
2.76598393012495	0.118792582172281\\
2.76171315775648	0.101210270872707\\
2.75600034892523	0.0828604480786408\\
2.74861784118104	0.0635608583348252\\
2.73926144240195	0.0430994204588267\\
2.72752614849146	0.0212256878847664\\
2.71287145602787	-0.00235998808074396\\
2.69457142269093	-0.0280202621363476\\
2.67164205333901	-0.0561983212534008\\
2.64273460263223	-0.0874398238642967\\
2.60597729363336	-0.122419446196449\\
2.55873899524786	-0.16197063710764\\
2.49727644824871	-0.207112973876854\\
2.41621472705903	-0.25906134742367\\
2.30781294906669	-0.319178239043526\\
2.16103090376812	-0.388782584104714\\
1.96066602867023	-0.46864379618366\\
1.68754270561763	-0.557892575256645\\
1.32214543016994	-0.652168158065221\\
0.855139672833289	-0.741662557912\\
0.304000364149641	-0.811679733978347\\
-0.278731448373542	-0.848741899301714\\
-0.824668087485252	-0.849099418132795\\
-1.28465122185451	-0.820383506369671\\
-1.6444858391152	-0.77497166528901\\
-1.9147387932445	-0.72330745184811\\
-2.11466040306014	-0.671757944162537\\
-2.26268638490528	-0.623380385091793\\
-2.37333988344336	-0.579257040242056\\
-2.45718036768567	-0.539479019780047\\
-2.52164678111486	-0.503708930612821\\
-2.57193772497335	-0.471464138077679\\
-2.61169988858062	-0.442246757546901\\
-2.64351594898488	-0.415596919174394\\
-2.66923735914705	-0.391109861495715\\
-2.69020855922654	-0.368436997682488\\
-2.70741807953273	-0.347280568397312\\
-2.72160104643071	-0.327386299517498\\
-2.73330938581511	-0.308535984071788\\
-2.74296040698015	-0.290540731779852\\
-2.75087075690752	-0.273235092187402\\
-2.75728033616711	-0.256472024068654\\
-2.76236920991791	-0.240118592446481\\
-2.76626952842005	-0.224052247118856\\
-2.76907379540857	-0.208157535512617\\
-2.77084036426975	-0.192323110178293\\
-2.77159672029497	-0.1764388988405\\
-2.77134086887585	-0.160393308534797\\
-2.77004095713812	-0.144070332773498\\
-2.76763308176857	-0.127346420280503\\
-2.76401705350883	-0.110086943955474\\
-2.75904967228526	-0.092142077288357\\
-2.75253478283033	-0.0733418397797098\\
-2.74420898212786	-0.0534900098683465\\
-2.73372126704024	-0.032356520394106\\
-2.7206040346805	-0.00966784671947868\\
-2.70423150947638	0.014905222825905\\
-2.68375960308018	0.0417631560078521\\
-2.65803800694986	0.0713973388592425\\
-2.62548037540859	0.104414356051744\\
-2.58387101713977	0.141564491099818\\
-2.53007597315801	0.183771420326977\\
-2.45961364608305	0.232153476510912\\
-2.36603228770019	0.288011450466867\\
-2.24006518469893	0.352724342015756\\
-2.06866914692298	0.427428950372729\\
-1.83449282910597	0.512259192042037\\
-1.51738192768083	0.604872089801674\\
-1.10107594364602	0.698358248788268\\
-0.587556093100141	0.780086596003453\\
-0.0122557582588863	0.834865234224473\\
0.560167015110665	0.853244053900496\\
};
\addplot [color=mycolor1, forget plot]
  table[row sep=crcr]{%
2.4470975329268	2.85734050088096\\
2.59925956234538	2.85263748443431\\
2.72936239773121	2.84031623671661\\
2.84139874347278	2.82247155051008\\
2.93859222736187	2.80058642786175\\
3.0235313743472	2.77570091105713\\
3.09829197054803	2.74853551350831\\
3.16454044419819	2.71957967116439\\
3.22361771846211	2.68915438376271\\
3.27660568747186	2.6574562163269\\
3.32437921061262	2.62458795306691\\
3.36764642225098	2.59057968585606\\
3.40697974908862	2.55540298570755\\
3.44283955508847	2.51897998110674\\
3.47559189073129	2.48118857401167\\
3.50552143839477	2.44186459580276\\
3.53284041838165	2.40080138931511\\
3.5576939377613	2.35774706040887\\
3.58016200911911	2.31239944452605\\
3.60025821866181	2.2643986587764\\
3.61792476088008	2.21331694180605\\
3.63302325477194	2.15864530947419\\
3.64532038396865	2.09977636572004\\
3.65446692232179	2.03598240275462\\
3.65996807166613	1.96638771281183\\
3.66114219736428	1.88993384992122\\
3.6570639521016	1.80533650974888\\
3.64648641782961	1.7110329239271\\
3.62773537508888	1.60511957138689\\
3.59856753534355	1.48528233243545\\
3.55598465216144	1.34872632808139\\
3.49599945873118	1.19212300117946\\
3.41336285143547	1.01161127682071\\
3.3012949774795	0.802922570523745\\
3.15133200538093	0.561748137098855\\
2.95352110334314	0.284520029275175\\
2.6973552050403	-0.0302170952383606\\
2.37392335641798	-0.379827544265013\\
1.97946724983572	-0.755471413831684\\
1.51952588658696	-1.14109638579653\\
1.01136727122743	-1.51514517126524\\
0.482032224901594	-1.85544525334606\\
-0.0382838063201061	-2.14522947273086\\
-0.52410154924132	-2.37686303097922\\
-0.959488772986293	-2.55157186200705\\
-1.33839105970505	-2.67639513915615\\
-1.66205132699853	-2.76070307072847\\
-1.93580024564179	-2.81373776109643\\
-2.16654578772536	-2.84340740970685\\
-2.36125505045569	-2.85595956618366\\
-2.52621914730902	-2.8561017997265\\
-2.66679922833549	-2.84728154219588\\
-2.787420973919	-2.83197952277816\\
-2.89167569415667	-2.81196030141443\\
-2.98245314892876	-2.7884673805791\\
-3.06207131943148	-2.76236803217295\\
-3.13238964293222	-2.73425797963794\\
-3.19490246471729	-2.70453592010907\\
-3.25081385812493	-2.67345607439244\\
-3.30109649242935	-2.64116495686063\\
-3.34653745708976	-2.60772685290374\\
-3.38777365579463	-2.57314117569786\\
-3.42531892653787	-2.53735390476143\\
-3.45958457962228	-2.50026460922663\\
-3.49089463019454	-2.46173005540939\\
-3.51949664689044	-2.42156503154378\\
-3.54556883535683	-2.37954074729201\\
-3.56922370906653	-2.33538094838593\\
-3.59050845108607	-2.28875570263873\\
-3.60940181843321	-2.2392726437686\\
-3.6258071618823	-2.18646528955512\\
-3.63954080086355	-2.12977787016306\\
-3.65031457159793	-2.06854590523487\\
-3.65771081504082	-2.00197155739246\\
-3.66114733989249	-1.92909258526504\\
-3.65982893264023	-1.84874357670581\\
-3.65268075561092	-1.75950818944266\\
-3.63825750750274	-1.65966162877518\\
-3.61462074197262	-1.54710408232652\\
-3.57917593635714	-1.41928932570068\\
-3.52846249420908	-1.27316004705987\\
-3.45789767683273	-1.10511567818159\\
-3.36149714823051	-0.911064057090691\\
-3.23164370107576	-0.686649269870069\\
-3.0590694227229	-0.427801906604648\\
-2.83336215057245	-0.131797507300495\\
-2.54445295612327	0.201048934514697\\
-2.18548932838733	0.565232839046605\\
-1.756881214354	0.948298435946715\\
-1.26995300623213	1.33101269062124\\
-0.747435352721523	1.6907508679559\\
-0.218964618192301	2.00735905571867\\
0.286748605912422	2.26841802747823\\
0.748686704214283	2.47098556427373\\
1.15605475094327	2.61965277121247\\
1.50685469940157	2.72301328980851\\
1.80474754969108	2.7905970960028\\
2.05610271647609	2.83106393347599\\
2.26799126297232	2.85149584067307\\
2.4470975329268	2.85734050088096\\
};
\addplot [color=mycolor2, forget plot]
  table[row sep=crcr]{%
1.13400451818224	0.976808589286248\\
1.55591156704315	0.963967632646999\\
1.87844575220844	0.933648692297892\\
2.11936805604402	0.895471980855908\\
2.29861648452528	0.855264674972638\\
2.43291852387781	0.816033640430765\\
2.53480545668349	0.77909964354237\\
2.61323550840971	0.744886767314028\\
2.67451202995887	0.713381203705023\\
2.72306878728849	0.684375563701702\\
2.76204775962919	0.657592066074585\\
2.79370039348197	0.632741553186792\\
2.81966033819558	0.609549681145767\\
2.84112800312704	0.587766765388304\\
2.85899595219203	0.567169748121874\\
2.87393480527273	0.547560569004644\\
2.88645267285243	0.528763065252748\\
2.89693669235387	0.510619424756789\\
2.90568230974701	0.4929866544925\\
2.91291404181462	0.475733242668805\\
2.91880020475392	0.45873605083536\\
2.92346326744079	0.441877401398409\\
2.9269869300111	0.425042290128983\\
2.9294206417286	0.408115632972522\\
2.93078199156806	0.390979441224458\\
2.93105718626686	0.373509802692034\\
2.93019964118205	0.355573524530472\\
2.92812652187264	0.337024262310264\\
2.92471286181861	0.317697915544286\\
2.91978261199468	0.297407007473874\\
2.91309560767229	0.275933680108139\\
2.90432890298117	0.253020816484529\\
2.89305012649987	0.2283606419443\\
2.87867929825578	0.201579947253414\\
2.86043367930716	0.172220819462771\\
2.83724732590173	0.139715489795419\\
2.80765253239014	0.103353715016184\\
2.76960351029923	0.0622412937740589\\
2.72021268256465	0.0152496400797256\\
2.65535696720453	-0.0390393611120289\\
2.56909954942	-0.102377421932274\\
2.45287988771856	-0.176838945573836\\
2.2945089602085	-0.264649236867091\\
2.07732101696771	-0.367628681657688\\
1.78067588590787	-0.48587920660265\\
1.38454964462808	-0.615400748165417\\
0.881673507817823	-0.74534824221461\\
0.294918328976953	-0.858145393319198\\
-0.317126387879956	-0.936341921800144\\
-0.883847296346128	-0.972771788836004\\
-1.35810775058867	-0.973322344140018\\
-1.72865102211688	-0.950297867995204\\
-2.00781965414307	-0.915097530776598\\
-2.21555633660112	-0.875382765679109\\
-2.37050823287822	-0.835416192301556\\
-2.48728119065286	-0.797238697197828\\
-2.57650663409757	-0.761647304777937\\
-2.64570478272576	-0.728805858007367\\
-2.70015917313327	-0.698582512499448\\
-2.74359762199151	-0.6707242684433\\
-2.77867579716957	-0.64494284878883\\
-2.80730847419449	-0.620954518293437\\
-2.83089386719239	-0.598496660609891\\
-2.85046562760151	-0.57733295164336\\
-2.86679650790334	-0.557253161273629\\
-2.88046972309956	-0.538070608957517\\
-2.89192857699452	-0.519618756232101\\
-2.90151130391482	-0.501747631914064\\
-2.9094757140015	-0.484320386239284\\
-2.91601668742886	-0.467210069019795\\
-2.92127854785938	-0.450296626550061\\
-2.92536366807315	-0.433464061900805\\
-2.92833819837386	-0.416597676898128\\
-2.93023548133117	-0.399581297302019\\
-2.93105747144582	-0.382294367470811\\
-2.93077427789105	-0.364608782085508\\
-2.9293217633479	-0.346385296441374\\
-2.9265969355594	-0.3274693196282\\
-2.92245063167818	-0.307685842231661\\
-2.91667668191573	-0.286833176453542\\
-2.90899629612566	-0.264675084706324\\
-2.89903576598859	-0.240930734263808\\
-2.8862945941022	-0.215261731664662\\
-2.87009965741047	-0.187255256459951\\
-2.84953868532166	-0.156402040528551\\
-2.82336272243743	-0.122067683553979\\
-2.78984168857616	-0.0834557348325103\\
-2.74654882533352	-0.0395615824687034\\
-2.69003818636869	0.0108813368567756\\
-2.61536585856166	0.0694543294176981\\
-2.51539944447989	0.138078327170258\\
-2.37989458894041	0.218941624283059\\
-2.19448862572723	0.314173655445411\\
-1.94029548841821	0.424963981921096\\
-1.59600269760179	0.549723930217708\\
-1.14591807084824	0.681304947299176\\
-0.595828970644992	0.805112369640538\\
0.012534643673176	0.902349173444445\\
0.609958617098545	0.959665970740262\\
1.13400451818224	0.976808589286248\\
};
\addplot [color=mycolor1, forget plot]
  table[row sep=crcr]{%
1.90545647109768	2.84473020387397\\
2.06560614281006	2.83975180880271\\
2.20696629698615	2.82634185272677\\
2.3321915951789	2.80637841785645\\
2.44358977384658	2.7812802081649\\
2.54313495484306	2.75210344657922\\
2.63249879955004	2.71962180533369\\
2.71308776287006	2.68438966472592\\
2.78608015084079	2.64679082867193\\
2.85245993353386	2.60707528779914\\
2.91304609217289	2.56538646417426\\
2.96851724174839	2.52178098738162\\
3.01943172859427	2.47624262550686\\
3.06624357723105	2.42869160436426\\
3.10931468151508	2.37899021861514\\
3.14892357416526	2.32694537000167\\
3.18527100428818	2.27230845243878\\
3.21848242258461	2.21477283089051\\
3.24860732457302	2.15396902195747\\
3.27561523244261	2.08945757331174\\
3.29938790103738	2.02071955571186\\
3.31970710623613	1.94714453143744\\
3.33623710895942	1.86801586331425\\
3.34850058536363	1.78249331255182\\
3.35584648861182	1.68959310181112\\
3.35740800761322	1.58816609549383\\
3.3520486262168	1.47687564143641\\
3.33829449887801	1.35417819222828\\
3.3142524026042	1.2183124702892\\
3.27751522297935	1.06730717305853\\
3.22506265000043	0.899023567903594\\
3.15317550326068	0.71125798313182\\
3.05740019708648	0.501939051865713\\
2.9326263925163	0.269461325311117\\
2.77337167138756	0.0131902830470721\\
2.57438512912176	-0.265864250884355\\
2.33164968949332	-0.564299363001459\\
2.04372935410427	-0.875730608718146\\
1.71315341533049	-1.19073305175328\\
1.34725514404524	-1.49766199545704\\
0.957840581159799	-1.7843954061448\\
0.559476469129818	-2.04051966377913\\
0.166909292354912	-2.2591216772568\\
-0.20738281673389	-2.43751067242486\\
-0.554658638939878	-2.57677777831348\\
-0.870174254687224	-2.68063492454041\\
-1.15256142971836	-2.75411665693453\\
-1.40283481017664	-2.80254022426169\\
-1.62342322812576	-2.83085278857385\\
-1.81741597084171	-2.84331816002108\\
-1.98805916965749	-2.84343322803351\\
-2.13846090256022	-2.83397138670082\\
-2.27144346885171	-2.81708109928046\\
-2.38948818663578	-2.79439767208476\\
-2.49473284788786	-2.76714748807323\\
-2.58899587024454	-2.73623658309891\\
-2.67381156646907	-2.70232204470638\\
-2.75046784293595	-2.66586768325229\\
-2.82004187602089	-2.62718644856711\\
-2.88343176937095	-2.58647215117961\\
-2.94138353379713	-2.54382274343107\\
-2.99451340673076	-2.49925699574864\\
-3.04332582379704	-2.45272598968585\\
-3.0882274409002	-2.40412048827574\\
-3.12953757859625	-2.35327494589072\\
-3.16749537455486	-2.29996867912253\\
-3.20226381135857	-2.24392452729567\\
-3.23393064701117	-2.18480517625679\\
-3.26250611642261	-2.12220719462947\\
-3.28791709050908	-2.05565273435872\\
-3.30999716922801	-1.98457877940319\\
-3.32847193922572	-1.90832379869763\\
-3.34293834225436	-1.82611169612514\\
-3.35283678321713	-1.73703309658577\\
-3.35741428508422	-1.64002434273836\\
-3.35567674689123	-1.53384523447966\\
-3.34632834584585	-1.41705773592997\\
-3.32769667673192	-1.28800992744541\\
-3.29764396001492	-1.1448328491852\\
-3.25346864714091	-0.985463117106764\\
-3.19180969536969	-0.807711723896657\\
-3.10857992988814	-0.609408970823442\\
-2.99897728294295	-0.388664619236554\\
-2.85765247301142	-0.144284203070108\\
-2.6791393917441	0.123637165840428\\
-2.4586534840674	0.412984096921417\\
-2.19328667710712	0.718912984736189\\
-1.88343065048257	1.03347769843591\\
-1.53397275642332	1.34597848929993\\
-1.15461344762183	1.64426910082164\\
-0.758827839321173	1.91681242572131\\
-0.361611913312704	2.15477791315486\\
0.0231636257632512	2.35335165025311\\
0.384783706451366	2.51184807879158\\
0.716540098658869	2.63283491704342\\
1.01548918488673	2.7208374714998\\
1.28158359749607	2.78113776271335\\
1.51665497280379	2.81892605629222\\
1.72354052603863	2.83882994316678\\
1.90545647109768	2.84473020387397\\
};
\addplot [color=mycolor2, forget plot]
  table[row sep=crcr]{%
1.46855410137581	1.12707832297397\\
1.81746172861383	1.11648449734565\\
2.0832734524898	1.09148900817058\\
2.28420219676519	1.05963121359182\\
2.43667502329634	1.02541141897586\\
2.55355487106084	0.991253931079652\\
2.64431809237863	0.958339784161561\\
2.71577816800775	0.927157533362092\\
2.77280526976218	0.897829051554277\\
2.81889337956571	0.8702918262286\\
2.85657061328148	0.844397665828629\\
2.88768591502819	0.819964710924385\\
2.91360694503597	0.796803898388846\\
2.93535655415903	0.774731620325453\\
2.95370737147097	0.7535750041178\\
2.96924786678498	0.733173305273046\\
2.98242888329376	0.71337729747401\\
2.99359666860525	0.694047664479373\\
3.00301644870154	0.675052915197177\\
3.01088926630408	0.656267075791459\\
3.01736391707627	0.637567262531493\\
3.02254521232267	0.618831152111924\\
3.02649937460473	0.59993431215182\\
3.02925706552155	0.580747316169379\\
3.03081430616027	0.561132533969967\\
3.03113134592148	0.540940452919435\\
3.03012933592254	0.520005342248276\\
3.02768444097797	0.49814001572594\\
3.02361874677748	0.475129371377977\\
3.01768694324708	0.450722282586946\\
3.00955722904638	0.424621273347811\\
2.99878409207805	0.3964692207245\\
2.98476943299641	0.365832079889668\\
2.96670669226752	0.332176320531016\\
2.9434998840396	0.294839426073746\\
2.91364525751147	0.252991547337343\\
2.87505710623026	0.205586530339498\\
2.82481056570843	0.151301851642755\\
2.7587637319781	0.0884714203679338\\
2.6710140094552	0.015027152988835\\
2.55315732117441	-0.0715059378832148\\
2.39340687238096	-0.173853501639095\\
2.1759204274396	-0.294449287699812\\
1.8814106856495	-0.434129080814116\\
1.49134503928733	-0.589718938010167\\
0.998483596223561	-0.75106799166535\\
0.421858296720502	-0.900377107469046\\
-0.187265258270534	-1.01781985727513\\
-0.763463752921409	-1.09171592863421\\
-1.25816076574939	-1.12367349964627\\
-1.65447551481276	-1.12419096032058\\
-1.9596185164593	-1.10523445409195\\
-2.19074418828945	-1.07607638440074\\
-2.36560531342012	-1.04262754170864\\
-2.49890021152678	-1.00822970036258\\
-2.60172186414097	-0.974599376616324\\
-2.68211856458525	-0.942518623253538\\
-2.74585071655244	-0.91226261074259\\
-2.79703992585854	-0.883844558714942\\
-2.83865455548789	-0.857150338876948\\
-2.87285359067348	-0.832010381280273\\
-2.90122497915828	-0.808236977352645\\
-2.92494991100248	-0.785642778172455\\
-2.94491633958941	-0.764049184780701\\
-2.96179794748695	-0.743289371013107\\
-2.97610953260883	-0.723208506594891\\
-2.98824617924996	-0.703662559393909\\
-2.99851115012501	-0.684516404077932\\
-3.00713581596548	-0.66564160531801\\
-3.01429385668226	-0.646914043782802\\
-3.02011123698435	-0.62821143953346\\
-3.02467295600509	-0.60941075971465\\
-3.02802721267342	-0.59038545286311\\
-3.03018736051096	-0.571002417318382\\
-3.03113180758053	-0.551118577579125\\
-3.03080181858543	-0.530576903738905\\
-3.02909696891869	-0.509201659800999\\
-3.02586775522894	-0.486792600809011\\
-3.02090454690351	-0.463117749314465\\
-3.01392161641998	-0.43790426005694\\
-3.00453433777658	-0.41082671747549\\
-2.99222667527461	-0.381491992925512\\
-2.97630462145942	-0.349419510354504\\
-2.95582901110058	-0.314015440440706\\
-2.92951773672268	-0.274539022594774\\
-2.89560227628824	-0.23005908867203\\
-2.85161602955057	-0.179399434070764\\
-2.79408213382729	-0.121074198907511\\
-2.71805834339307	-0.0532218198776744\\
-2.61649598903557	0.026435103293064\\
-2.47941066892704	0.120533485531534\\
-2.29302953539182	0.231758171538483\\
-2.0395571501463	0.361972183384229\\
-1.69921188392751	0.510376156685171\\
-1.25735371882417	0.670635146156023\\
-0.718111284055277	0.828535287891957\\
-0.117225357787478	0.964113108564206\\
0.483218576030838	1.06040872070266\\
1.02278616318477	1.11239456361227\\
1.4685541013758	1.12707832297397\\
};
\addplot [color=mycolor1, forget plot]
  table[row sep=crcr]{%
1.5511141993765	2.96816970333686\\
1.71675358438934	2.96299691921139\\
1.86682539418795	2.94874046419969\\
2.00300869880171	2.92701307354291\\
2.1268612119055	2.89909435579508\\
2.23979214975311	2.86598173523314\\
2.34305412892067	2.82843762111414\\
2.43774626769991	2.78703031878428\\
2.52482323250256	2.74216802431532\\
2.6051068522611	2.69412618643373\\
2.67929821669458	2.64306893140237\\
2.74798902104091	2.58906536319621\\
2.81167144936764	2.53210151744128\\
2.87074620122304	2.47208864713597\\
2.92552843417927	2.40886839739986\\
2.97625146671294	2.34221530971119\\
3.02306809386465	2.27183699588217\\
3.06604933256557	2.19737224570959\\
3.10518034740554	2.11838728616227\\
3.14035322017697	2.03437040262416\\
3.17135612668665	1.94472517759859\\
3.19785838475057	1.84876272096299\\
3.21939075997375	1.74569349161575\\
3.23532040059592	1.63461969228791\\
3.24481988950191	1.51452982616827\\
3.24683026945662	1.38429792520045\\
3.24001870889995	1.24269129770411\\
3.22273302150037	1.08839248387585\\
3.19295793562874	0.920043451259204\\
3.14828230299302	0.736322673246078\\
3.08589270849307	0.536067880169691\\
3.00261702269694	0.318457344540068\\
2.8950497572107	0.0832576972436309\\
2.75979544183791	-0.168867570731735\\
2.59385881672152	-0.436021360781201\\
2.39518107659023	-0.714779306849007\\
2.16326296217152	-1.0000507108857\\
1.89973724771263	-1.2852211083055\\
1.60869142726366	-1.56266105395898\\
1.29655212742886	-1.82457122668823\\
0.971461435453503	-2.06398451866564\\
0.64226687823649	-2.27564747825153\\
0.317405648969006	-2.45653164756393\\
0.00399029116635881	-2.60587315260145\\
-0.292710800285464	-2.72481554129322\\
-0.569374827774472	-2.81583834360932\\
-0.824413429264136	-2.88215937495653\\
-1.05759927921737	-2.92723658466493\\
-1.2696590252546	-2.954419170429\\
-1.46191201754143	-2.96674238900522\\
-1.63598822105908	-2.96683397124905\\
-1.79362814199161	-2.95689494329898\\
-1.93655247395617	-2.93872345847997\\
-2.06638458222437	-2.91375925764997\\
-2.18460991573799	-2.88313463759206\\
-2.29255955355281	-2.84772397858109\\
-2.39140850588118	-2.80818795523389\\
-2.48218231230374	-2.76501099685318\\
-2.56576770136256	-2.7185318953432\\
-2.64292464567118	-2.66896809950429\\
-2.71429819888329	-2.61643447582328\\
-2.78042917454843	-2.56095734362065\\
-2.84176313728666	-2.50248451756251\\
-2.89865740951947	-2.44089197581842\\
-2.95138591191257	-2.37598765124789\\
-3.00014169247095	-2.30751273337605\\
-3.04503698345508	-2.2351407798849\\
-3.08610057314396	-2.15847487429311\\
-3.12327220164251	-2.0770430384803\\
-3.1563935950482	-1.99029212536997\\
-3.18519565057231	-1.89758049553535\\
-3.20928119314775	-1.79816994866221\\
-3.2281026712299	-1.69121767735085\\
-3.2409341989296	-1.5757694952191\\
-3.24683757518568	-1.45075634298661\\
-3.24462247064948	-1.31499719288961\\
-3.23280210809785	-1.16721305337147\\
-3.20954781988667	-1.00605887830981\\
-3.17264929489326	-0.83018271505137\\
-3.11949258355707	-0.638323940630935\\
-3.04707518900768	-0.429463817566223\\
-2.95208610020282	-0.203039647718152\\
-2.83108570188368	0.0407750039789764\\
-2.68082016772555	0.300741426695085\\
-2.49868774249454	0.574228932152976\\
-2.28333051547243	0.856982607028298\\
-2.03525425987047	1.14311012993192\\
-1.75730260272994	1.42540246962225\\
-1.45477926837949	1.69602394718267\\
-1.13507590084788	1.94746673630532\\
-0.806827848963815	2.17352897098816\\
-0.478810766230246	2.37003509442909\\
-0.158891672379683	2.53511231340366\\
0.146701500614249	2.66901330900869\\
0.433683425640174	2.7736265607983\\
0.699638558134577	2.85187124026548\\
0.943710310761243	2.90713854170722\\
1.166197624607	2.94286609588378\\
1.36816404860046	2.96226391390951\\
1.5511141993765	2.96816970333686\\
};
\addplot [color=mycolor2, forget plot]
  table[row sep=crcr]{%
1.69631441633007	1.28768207007386\\
1.99032662854603	1.27874738696554\\
2.2166701408415	1.2574447719624\\
2.39088211468516	1.22980388087852\\
2.52592030888927	1.19948059308081\\
2.6317285425257	1.16854555802824\\
2.71566106839321	1.13809809930942\\
2.78308250493548	1.10866994684214\\
2.83790007966455	1.08047124194359\\
2.88297368997499	1.0535348197561\\
2.92041396595002	1.02779904998385\\
2.95179366805963	1.00315466733232\\
2.97829637996918	0.979470777011036\\
3.00082110987031	0.956608880037873\\
3.02005620276326	0.934430016464237\\
3.03653189746237	0.912797952928076\\
3.0506579371935	0.891580089193716\\
3.06275061316424	0.870647035321154\\
3.07305223115758	0.849871390220485\\
3.0817450407718	0.829126003760703\\
3.0889610107669	0.808281853879328\\
3.09478837182346	0.787205574173226\\
3.09927551078203	0.765756600727119\\
3.10243253949991	0.743783853697696\\
3.10423064069486	0.721121818891428\\
3.10459908279546	0.697585839253715\\
3.10341956749125	0.672966358648663\\
3.10051729544469	0.647021773042724\\
3.09564776637187	0.619469428583003\\
3.08847781162818	0.589974151989877\\
3.07855860551277	0.558133495200779\\
3.06528728841541	0.523458614364247\\
3.04785217094134	0.485349384665519\\
3.02515399881607	0.44306201098606\\
2.99569207555459	0.395667149125735\\
2.95739875970142	0.341996730740528\\
2.90739882293268	0.280579102362855\\
2.84166243842304	0.209566654956785\\
2.75451725805986	0.126672070571172\\
2.63800260305594	0.0291573292462813\\
2.48113177255788	-0.0860210416188368\\
2.26937740142539	-0.22169921476775\\
1.98527052826225	-0.379277231781463\\
1.61192350068489	-0.55644225831266\\
1.14158841186434	-0.744225259582789\\
0.588057722722855	-0.925696180260258\\
-0.00675744185391754	-1.0800113283142\\
-0.584623958529031	-1.19167406956994\\
-1.09642539040285	-1.25745740352105\\
-1.51896416693009	-1.2848102027044\\
-1.8528008268099	-1.28525153808844\\
-2.11094151451397	-1.26919996062677\\
-2.30941040545633	-1.24414223451508\\
-2.4626089024133	-1.21481927141176\\
-2.58196490385806	-1.18400372538223\\
-2.67605415977655	-1.15321784723463\\
-2.75116272174692	-1.12323800969317\\
-2.81186746537437	-1.09441190091988\\
-2.86150824186208	-1.06684756239571\\
-2.90253908805109	-1.04052294570266\\
-2.93677969678039	-1.01534831669648\\
-2.96559277957666	-0.991201247053671\\
-2.9900087356399	-0.967945805144239\\
-3.01081351667094	-0.945442666432577\\
-3.02861090182525	-0.923554007263144\\
-3.04386692502897	-0.902145396894908\\
-3.05694175382007	-0.881085951917333\\
-3.06811263841632	-0.860247466272762\\
-3.07759040122891	-0.839502907414268\\
-3.08553114872227	-0.818724476404097\\
-3.09204433866097	-0.797781310465758\\
-3.09719794322897	-0.776536827601692\\
-3.10102115455803	-0.754845654470515\\
-3.10350484234504	-0.732550028067738\\
-3.10459976120126	-0.709475509870231\\
-3.10421228968941	-0.685425790570829\\
-3.10219723414302	-0.660176287121552\\
-3.09834691259617	-0.633466133528818\\
-3.09237529912219	-0.604988033344248\\
-3.08389538674237	-0.574375264372061\\
-3.07238701384697	-0.541184894215979\\
-3.05715103904924	-0.504875973594996\\
-3.03724371305329	-0.464781136540198\\
-3.01138206111722	-0.420069719438527\\
-2.97780665437066	-0.369700415319409\\
-2.93408197189791	-0.312362127070219\\
-2.87680689778822	-0.246404329456668\\
-2.80120128949956	-0.16976576083321\\
-2.7005384078293	-0.079928782593947\\
-2.56543493883457	0.0260318956802744\\
-2.3831591312352	0.15115536363008\\
-2.13750906636878	0.297774100114308\\
-1.8105897569183	0.465783528152239\\
-1.38866466012649	0.649891413810413\\
-0.87323150796197	0.837055758569039\\
-0.292272852787305	1.00745919807065\\
0.30138542793791	1.14168765120112\\
0.850896019481423	1.2300133132159\\
1.31925086546401	1.27523515772891\\
1.69631441633007	1.28768207007386\\
};
\addplot [color=mycolor1, forget plot]
  table[row sep=crcr]{%
1.30964269727806	3.1637962565165\\
1.4796617479317	3.15846843373279\\
1.63679061427442	3.14352532559141\\
1.78209449164426	3.12032833944566\\
1.91661469139635	3.08999203623042\\
2.04133587734847	3.05341084022803\\
2.15716704436191	3.01128640334179\\
2.26493187823667	2.96415318583326\\
2.36536518025283	2.91240103174314\\
2.45911293062631	2.85629427339921\\
2.54673426377869	2.79598733673527\\
2.62870414512695	2.73153704934776\\
2.70541590354361	2.66291195707374\\
2.77718301859967	2.58999898766238\\
2.84423971673922	2.51260779861417\\
2.90674001989998	2.4304731350626\\
2.96475493309473	2.34325552009698\\
3.01826746994878	2.25054061827051\\
3.06716521160309	2.1518376669181\\
3.11123009040846	2.04657747494767\\
3.15012510558999	1.93411066423918\\
3.18337774235946	1.81370709861271\\
3.21036002162971	1.68455783764647\\
3.23026541778701	1.54578149644089\\
3.24208343692729	1.39643760860752\\
3.24457356752618	1.23555047198094\\
3.23624174246731	1.06214793587486\\
3.21532452017037	0.875320475251067\\
3.17978895210455	0.674306296938357\\
3.12735936579723	0.458607445326076\\
3.05558538675182	0.228138886152453\\
2.96196699821337	-0.0165939040116348\\
2.84414987595434	-0.274304198788235\\
2.70019456494705	-0.542750437351112\\
2.52890392387698	-0.818625887665045\\
2.33016567717077	-1.09756710070662\\
2.10523835090943	-1.37432977635652\\
1.85689379064363	-1.64314957924953\\
1.58934378464163	-1.89825275383867\\
1.3079278520424	-2.13442513615714\\
1.01860977322162	-2.34751617881986\\
0.727390781441582	-2.53476738101936\\
0.439767930045209	-2.69490895567937\\
0.160339199500988	-2.82803826492504\\
-0.107399608192614	-2.93534672986096\\
-0.361078183550312	-3.0187811774478\\
-0.599343551116539	-3.08071390197956\\
-0.82167924151625	-3.12366813311058\\
-1.02820229942608	-3.15011728731852\\
-1.21947223478175	-3.16235593533549\\
-1.39632974112117	-3.16242964145298\\
-1.55977082001897	-3.15210762402562\\
-1.71085457163499	-3.1328834234634\\
-1.85063940386854	-3.10599186652311\\
-1.98014142579909	-3.07243399096331\\
-2.10030916868881	-3.03300448554837\\
-2.21200973246586	-2.98831837870147\\
-2.31602253244997	-2.93883521303982\\
-2.41303779837131	-2.88487991020164\\
-2.50365777336167	-2.82666011113997\\
-2.58839916531416	-2.76428009887243\\
-2.66769583949962	-2.69775156936529\\
-2.7419010420427	-2.62700157896064\\
-2.81128864048502	-2.5518780086988\\
-2.87605298716733	-2.47215287701608\\
-2.93630707519116	-2.38752382306851\\
-2.99207868262204	-2.29761408868848\\
-3.04330420322828	-2.20197136130826\\
-3.08981985611297	-2.10006591749712\\
-3.13134996953366	-1.99128864432507\\
-3.16749206989299	-1.87494973512688\\
-3.19769861039845	-1.7502791836144\\
-3.22125539743398	-1.61643066493288\\
-3.23725719218883	-1.47249102074309\\
-3.24458168556316	-1.31749836824648\\
-3.24186419916032	-1.15047279834493\\
-3.22747719906101	-0.970464595351572\\
-3.19952112736246	-0.77662561958305\\
-3.15583611401251	-0.568309413072392\\
-3.09404743511042	-0.345203854359003\\
-3.01166010965617	-0.107495624874144\\
-2.90621784361654	0.143942848398775\\
-2.77553580150244	0.407366412040965\\
-2.61800243985426	0.680008613328076\\
-2.432921803668	0.958023744307871\\
-2.22083819760022	1.23657425973828\\
-1.98376120673494	1.51009984708879\\
-1.72520699853086	1.77276127972635\\
-1.45000416550279	2.01899454967052\\
-1.16387551442911	2.24406316113598\\
-0.872876987209391	2.44448553178769\\
-0.582818051995086	2.61825016010458\\
-0.298783922300724	2.76479746668511\\
-0.0248348027380357	2.88481207518776\\
0.236102926428582	2.97990634957964\\
0.482187030483545	3.05227836659459\\
0.712507782295484	3.10440573556148\\
0.926890690971778	3.13880717167437\\
1.12569630509008	3.15787875025747\\
1.30964269727806	3.1637962565165\\
};
\addplot [color=mycolor2, forget plot]
  table[row sep=crcr]{%
1.87066321805417	1.45402756039734\\
2.12177093449327	1.44638275984957\\
2.317761693533	1.42791955610156\\
2.47135347756998	1.40353440201243\\
2.59275799155613	1.37625932425357\\
2.6897624708477	1.34788753229756\\
2.76817229032354	1.319434934294\\
2.83228382782909	1.29144458778211\\
2.88528007826317	1.26417714528765\\
2.92953233347676	1.23772686605507\\
2.96682099594705	1.21209130233767\\
2.99849438692525	1.1872127710152\\
3.025582317517	1.16300278176544\\
3.04887736169328	1.13935617858333\\
3.06899323549112	1.11615904872294\\
3.08640693070161	1.09329282421922\\
3.10148924371942	1.07063602288273\\
3.11452692052714	1.04806448560686\\
3.12573864511145	1.02545060684543\\
3.13528640133144	1.00266183085243\\
3.14328324184147	0.97955854127504\\
3.14979813366696	0.955991371688167\\
3.15485826910997	0.931797888189099\\
3.15844899574587	0.90679852796303\\
3.16051129998121	0.88079160930123\\
3.16093654653781	0.853547150047979\\
3.15955790023812	0.8247991342263\\
3.15613749773549	0.794235741298259\\
3.15034794215717	0.761486888763146\\
3.14174598657415	0.726108225906463\\
3.12973524113668	0.687560446781515\\
3.1135132220467	0.64518246836988\\
3.09199583371357	0.598156684667283\\
3.06370915644131	0.545464286952159\\
3.0266339377191	0.485828872524742\\
2.97798250492584	0.417648064197987\\
2.91388218623718	0.338917487750234\\
2.82893862932173	0.247163201907554\\
2.71567109994163	0.139425480376497\\
2.56388651868539	0.0123921444109677\\
2.36026722654807	-0.137122249369144\\
2.08890423110962	-0.311030939692938\\
1.73420883848114	-0.507843467702896\\
1.28787824980463	-0.719793581163217\\
0.759259933385833	-0.931073702832379\\
0.182177988575539	-1.12053266150978\\
-0.392171517511089	-1.26977533751479\\
-0.915575840684575	-1.37106932892531\\
-1.36004790937376	-1.42827021685356\\
-1.7199956170801	-1.45158832747825\\
-2.00398372858132	-1.45195535619048\\
-2.22580640137757	-1.43814554765677\\
-2.39915840499094	-1.41624205918801\\
-2.53554018619301	-1.39012326080988\\
-2.64390675165217	-1.36213304062674\\
-2.73099292835528	-1.33362899055591\\
-2.80179428767293	-1.30536071197632\\
-2.86000707011612	-1.27771170209651\\
-2.90837620006094	-1.2508483373094\\
-2.94895423287702	-1.22480990474502\\
-2.9832888558787	-1.19956242009131\\
-3.01255721582719	-1.17503050616074\\
-3.03766196113174	-1.15111603355987\\
-3.05930008373523	-1.12770876023663\\
-3.07801248582485	-1.10469210625135\\
-3.09421983150112	-1.08194593790484\\
-3.10824855146559	-1.05934747633484\\
-3.12034968056339	-1.03677098569893\\
-3.13071237649265	-1.01408661289765\\
-3.13947338120028	-0.991158571549518\\
-3.14672326345681	-0.967842743744772\\
-3.15250996389637	-0.9439836870027\\
-3.15683990998776	-0.919410963503353\\
-3.15967674486952	-0.893934642015034\\
-3.16093749189277	-0.867339750538167\\
-3.16048572690583	-0.839379370929042\\
-3.15812101853101	-0.809765956741145\\
-3.1535634778409	-0.77816031235226\\
-3.14643166941203	-0.744157484395111\\
-3.1362112833808	-0.707268575723716\\
-3.12221071816805	-0.666897194254896\\
-3.10349788402036	-0.622308911711488\\
-3.07880985193452	-0.572591804392801\\
-3.04642315041378	-0.516606090574266\\
-3.00396738607712	-0.452921596262271\\
-2.94815890970766	-0.37974451149767\\
-2.87442707678991	-0.294842389227502\\
-2.77641185225325	-0.195494294437964\\
-2.64535151344877	-0.0785320608051692\\
-2.46950778725791	0.0593858575997357\\
-2.23409419163594	0.221008435394047\\
-1.92277618699022	0.406878897255613\\
-1.52245210617972	0.612727365398179\\
-1.03231897555632	0.826789077549283\\
-0.473780599181381	1.02986037347607\\
0.108686675359043	1.20096726029463\\
0.662567557702798	1.32640137592083\\
1.14847974851045	1.40461533226047\\
1.55026489179016	1.44344976208378\\
1.87066321805417	1.45402756039734\\
};
\addplot [color=mycolor1, forget plot]
  table[row sep=crcr]{%
1.14198747225695	3.38832402736194\\
1.31605992222682	3.38285578958292\\
1.47926499672352	3.3673224594284\\
1.63231216590224	3.34287789641818\\
1.77592179996291	3.31048127019128\\
1.91079626334644	3.27091234371535\\
2.03760030921143	3.22478834037688\\
2.15694827157711	3.17258054332779\\
2.26939599959159	3.11462954399408\\
2.37543590802034	3.05115858110924\\
2.47549388692006	2.98228475333172\\
2.56992710848601	2.9080281024882\\
2.65902199295158	2.82831869522321\\
2.74299175804852	2.74300191243686\\
2.82197309033541	2.65184221413475\\
2.89602155468672	2.55452570213947\\
2.96510541366527	2.45066187094384\\
3.02909757560089	2.33978503259681\\
3.08776544561469	2.22135603930962\\
3.14075853869006	2.09476512195938\\
3.18759385630825	1.95933692768926\\
3.22763926585268	1.81433918576067\\
3.26009550419467	1.65899685866019\\
3.28397801391087	1.4925141257177\\
3.29810067759518	1.31410704126207\\
3.30106469525668	1.12305008611112\\
3.29125735525567	0.918739870152637\\
3.26686717093224	0.7007785980871\\
3.22592347601145	0.469078103296655\\
3.16636945965635	0.223981748356033\\
3.08617676437358	-0.0336041103607149\\
2.98350587120135	-0.302084954015504\\
2.85690839958484	-0.579083389716836\\
2.70555499087658	-0.861405365663716\\
2.52945757552058	-1.14509799818222\\
2.32964231475143	-1.42561972089645\\
2.10822613845917	-1.69812186340872\\
1.86836133044444	-1.957811514879\\
1.61403947829286	-2.20033847192607\\
1.34978085508433	-2.42213618717118\\
1.08026481735875	-2.62065529577265\\
0.809969103738292	-2.79445610767395\\
0.542877020464996	-2.94316206126457\\
0.282287201830255	-3.06730577664165\\
0.0307321921675561	-3.16811417651256\\
-0.210010359277831	-3.24727833304284\\
-0.438841755820347	-3.30674243380652\\
-0.655230912051859	-3.34853143793361\\
-0.85909190141726	-3.37462385965159\\
-1.05066706316558	-3.38686711003996\\
-1.23042500857762	-3.38692810863447\\
-1.39897720956761	-3.37627042460475\\
-1.55701316096008	-3.35614973643604\\
-1.70525208644774	-3.32762084659481\\
-1.84440832162054	-3.29155115765214\\
-1.97516739762051	-3.24863703808408\\
-2.09817011972688	-3.1994207324695\\
-2.21400235954319	-3.14430638303466\\
-2.32318872587667	-3.08357436830736\\
-2.42618868176004	-3.01739359026803\\
-2.52339400739865	-2.94583161297398\\
-2.61512676734915	-2.86886272311823\\
-2.70163713203072	-2.78637408546898\\
-2.78310054027091	-2.69817023325833\\
-2.85961378384395	-2.60397618789948\\
-2.93118965997596	-2.50343956182795\\
-2.99774988716902	-2.39613207818106\\
-3.05911602864777	-2.28155105592556\\
-3.11499823485627	-2.15912157337876\\
-3.16498172601717	-2.02820025121288\\
-3.20851112040985	-1.88808190062256\\
-3.24487301706631	-1.73801067006497\\
-3.27317771812207	-1.57719778735318\\
-3.29234169013923	-1.4048484969593\\
-3.30107337735163	-1.22020124874507\\
-3.29786632705124	-1.02258243707351\\
-3.28100522506141	-0.811479737863473\\
-3.2485921674592	-0.586635931031317\\
-3.19860185149968	-0.348162504129153\\
-3.12897453324035	-0.0966678067712452\\
-3.03775338926328	0.166612124818355\\
-2.92326702955321	0.43970012338347\\
-2.78434754524817	0.7198096240344\\
-2.62056039573655	1.00335319673104\\
-2.43240798435225	1.28605503841691\\
-2.22145995280244	1.56317868851008\\
-1.99036681856816	1.82985490562083\\
-1.74273320138674	2.08146498692662\\
-1.48285909721788	2.31401346585324\\
-1.2153915502866	2.52442163704885\\
-0.944951100570337	2.7106925546306\\
-0.675798932141263	2.8719316217413\\
-0.411592829482475	3.00824106676635\\
-0.155252198347039	3.12052944746041\\
0.0910740234581681	3.21028396949869\\
0.325960018222231	3.27934651147373\\
0.548605990898327	3.32972039217949\\
0.758718693172089	3.36342045582833\\
0.956390154658264	3.38236784241906\\
1.14198747225695	3.38832402736193\\
};
\addplot [color=mycolor2, forget plot]
  table[row sep=crcr]{%
2.01273186082157	1.62264188057517\\
2.22882063757858	1.61604978981954\\
2.39982292437263	1.5999264589979\\
2.53605024869709	1.57828560186922\\
2.64560455641858	1.55366230710075\\
2.73465289221244	1.52760891831352\\
2.80782868963228	1.50104856263404\\
2.86860364562702	1.47450913150948\\
2.91958676862407	1.44827262140604\\
2.96275059777069	1.42246876899394\\
2.99959758866234	1.39713323996872\\
3.0312810461154	1.37224358546903\\
3.05869273021849	1.34774130683866\\
3.08252639381533	1.32354521601587\\
3.10332400638802	1.2995592980853\\
3.12150948490404	1.27567705258614\\
3.13741333335669	1.25178352695749\\
3.15129057492953	1.22775577834639\\
3.16333363422859	1.20346219676488\\
3.17368130395596	1.17876092409924\\
3.18242454469853	1.15349746678444\\
3.18960956917868	1.12750149774532\\
3.19523841599993	1.10058275728146\\
3.19926699199864	1.0725258797971\\
3.20160032852801	1.04308388296492\\
3.20208452520021	1.01196994813214\\
3.20049450803311	0.978846985268764\\
3.19651625838312	0.943314301531696\\
3.18972150520314	0.9048904686354\\
3.17953191683756	0.862991203193945\\
3.16516843942157	0.816900741376229\\
3.14557941789191	0.765734845680667\\
3.11933827732454	0.708393357162769\\
3.08449765619282	0.64350043767493\\
3.03838211135335	0.569332132271484\\
2.97729712019097	0.483735409182865\\
2.89613254086358	0.384054181754214\\
2.78785645755072	0.267103344500391\\
2.64296078709071	0.129283819862817\\
2.44909508235137	-0.0329769279253369\\
2.19149803292758	-0.222156640087914\\
1.85540284097846	-0.437620556075269\\
1.43183220207253	-0.672781570954555\\
0.926529254475946	-0.91293714136645\\
0.366905965032531	-1.13685603092568\\
-0.201788153687179	-1.3237955378641\\
-0.732802230074641	-1.46194690177102\\
-1.19478592849004	-1.55144546914079\\
-1.57703266179045	-1.60067155688414\\
-1.88399309881306	-1.62055881991123\\
-2.12714244300369	-1.62086201768036\\
-2.3192602556512	-1.60888726103068\\
-2.47172727407397	-1.5896091371629\\
-2.59373537863074	-1.56623151723462\\
-2.69237111564201	-1.54074516975253\\
-2.77298453280887	-1.51435193789272\\
-2.83958597030085	-1.48775415463792\\
-2.89518319062102	-1.46134215062032\\
-2.94204278026816	-1.43531268450704\\
-2.98188459701558	-1.40974281616709\\
-3.01602373734729	-1.38463567926112\\
-3.04547347217946	-1.35994867978886\\
-3.07101983077187	-1.33561070585615\\
-3.09327577122147	-1.31153242886764\\
-3.11272065473482	-1.28761221449429\\
-3.12972907723809	-1.26373919332839\\
-3.14459190832569	-1.23979443853937\\
-3.15753152786689	-1.21565081849316\\
-3.16871263534396	-1.19117184813658\\
-3.1782495596713	-1.16620969956304\\
-3.18621066100641	-1.14060241545111\\
-3.19262014838458	-1.11417027688287\\
-3.19745740447247	-1.0867111941397\\
-3.20065368243093	-1.05799490378562\\
-3.20208579114637	-1.02775565754519\\
-3.20156608111895	-0.995682968167048\\
-3.19882764192543	-0.961409824006352\\
-3.19350306524589	-0.924497586282473\\
-3.1850943320687	-0.884416530974658\\
-3.17293023047192	-0.840520688268477\\
-3.15610603691663	-0.792015285376631\\
-3.13339778795653	-0.737914789304898\\
-3.10314010978443	-0.676989486867996\\
-3.0630521846915	-0.607699252841271\\
-3.00999156462662	-0.528115846321772\\
-2.93961266425308	-0.435842359447583\\
-2.8459134889737	-0.327955664745927\\
-2.72069030724748	-0.201034638583436\\
-2.55303012238377	-0.0514075447247277\\
-2.32923428621965	0.124137739729845\\
-2.03405456577611	0.326840235647845\\
-1.65465028027992	0.55345949443634\\
-1.18821921518954	0.793467136619876\\
-0.650950165933849	1.02834457263954\\
-0.0805694209150354	1.23597001051807\\
0.474456620282551	1.399221940702\\
0.973544504074251	1.51237661528221\\
1.39585274626457	1.58041146158093\\
1.73928748626154	1.61362058532941\\
2.01273186082157	1.62264188057517\\
};
\addplot [color=mycolor1, forget plot]
  table[row sep=crcr]{%
1.02611651957107	3.60772225884434\\
1.20403531250379	3.60212375025056\\
1.37252272406886	3.58607871275405\\
1.53209147433507	3.5605840611805\\
1.68327442490991	3.52647097361666\\
1.82660014241576	3.48441505698273\\
1.96257504267002	3.43494800021525\\
2.09167048293983	3.37846934578765\\
2.21431340202062	3.31525753296105\\
2.3308793479758	3.24547974201835\\
2.44168695081149	3.16920033169254\\
2.54699307979453	3.08638784461037\\
2.64698807085481	2.99692068599959\\
2.74179052285167	2.9005916824586\\
2.83144125016096	2.79711181942329\\
2.91589605313904	2.68611355417884\\
2.99501703994363	2.56715421982348\\
3.06856231834773	2.4397201868288\\
3.13617399414738	2.30323264312533\\
3.19736458859026	2.15705609778928\\
3.25150225242221	2.00051100716612\\
3.29779554621413	1.83289225205801\\
3.33527911587741	1.65349552358913\\
3.36280235276438	1.4616539294472\\
3.37902409905939	1.2567871829688\\
3.38241759486048	1.03846538671013\\
3.3712910142648	0.806488398293082\\
3.34382979549314	0.560979753679769\\
3.29816702173673	0.302490850827981\\
3.23248664958135	0.0321065160951329\\
3.14516065270606	-0.248462376543206\\
3.03491467713805	-0.536818690225567\\
2.90100793301208	-0.82987341863068\\
2.74340347468625	-1.12391471873249\\
2.56289793654726	-1.41476251714717\\
2.36117911643895	-1.6980036617741\\
2.14078838582012	-1.96928306110288\\
1.90498241059215	-2.22460940016504\\
1.65751055078008	-2.46062665077033\\
1.4023432160411	-2.67480853992983\\
1.14339564859588	-2.86555083124844\\
0.884288382218877	-3.03215927806954\\
0.628172427659431	-3.1747514530428\\
0.377629863914206	-3.29410266310233\\
0.134644947708159	-3.39146838897052\\
-0.0993692244464525	-3.46841037969752\\
-0.323507312636681	-3.52664453446868\\
-0.537289374400536	-3.56791950464855\\
-0.740573585424378	-3.59392766774999\\
-0.933473511872671	-3.60624546586133\\
-1.11628526502292	-3.60629781860137\\
-1.28942670714867	-3.59534075499257\\
-1.45338877731054	-3.5744568633044\\
-1.60869780134301	-3.54455908355186\\
-1.75588709590617	-3.50639940520567\\
-1.89547603925685	-3.46057998697967\\
-2.02795487972215	-3.40756500385319\\
-2.15377376160685	-3.34769213374202\\
-2.2733346897906	-3.28118304140237\\
-2.38698538549539	-3.20815253229395\\
-2.49501418637923	-3.12861626808435\\
-2.5976453079511	-3.04249708878699\\
-2.69503391204524	-2.94963009988326\\
-2.78726052797591	-2.84976677729476\\
-2.87432445198552	-2.74257843623574\\
-2.95613582189578	-2.62765951672197\\
-3.0325061402555	-2.50453127197162\\
-3.10313711807964	-2.37264661756504\\
-3.16760785464923	-2.23139711792507\\
-3.22536058499219	-2.08012335631199\\
-3.27568555028962	-1.91813024917339\\
-3.31770601716222	-1.74470920163683\\
-3.35036512836541	-1.55916930470951\\
-3.37241713470611	-1.36087994730133\\
-3.38242662245726	-1.14932709569401\\
-3.37878052110083	-0.924184843670482\\
-3.35971873119469	-0.685402357949884\\
-3.3233897443763	-0.43330372642119\\
-3.26793702021244	-0.168694272966524\\
-3.19161937754449	0.107038210307673\\
-3.09296357987329	0.391844357044387\\
-2.97093950589242	0.682977531095793\\
-2.82513876334528	0.977023428823052\\
-2.65592880016096	1.27001138997337\\
-2.46455021158412	1.55761110463077\\
-2.25312869633061	1.83540002987176\\
-2.02458646705947	2.09916774465816\\
-1.78245842128309	2.34521061155878\\
-1.53063974954406	2.57056923249898\\
-1.2731063612472	2.77317355736937\\
-1.01365249973888	2.95188184958125\\
-0.755681123047297	3.10642231749431\\
-0.502066530535087	3.2372628547284\\
-0.255091604879366	3.34544136045897\\
-0.0164489460183681	3.43238713735545\\
0.212711718933483	3.49975621592722\\
0.431708626818227	3.54929399415121\\
0.640241689089839	3.58273017202159\\
0.838306582515661	3.60170497518489\\
1.02611651957107	3.60772225884434\\
};
\addplot [color=mycolor2, forget plot]
  table[row sep=crcr]{%
2.13066155564074	1.78934952646239\\
2.31729935703494	1.78364460719312\\
2.46689620124448	1.76952844019794\\
2.58781105164417	1.75031033791726\\
2.68652254186893	1.72811583414305\\
2.76796031070307	1.70428226501363\\
2.83585227040171	1.67963405118568\\
2.89301957020744	1.65466520604145\\
2.94160681957125	1.6296575131894\\
2.98325487595012	1.6047562069412\\
3.01922820624035	1.58001819006828\\
3.05050810394904	1.55544264470556\\
3.07786083388366	1.53099034984978\\
3.10188752944395	1.50659570797689\\
3.12306081329798	1.4821740068146\\
3.14175170109824	1.45762550489228\\
3.15824931183566	1.43283733227823\\
3.17277515750889	1.4076838126804\\
3.18549323891532	1.38202555853736\\
3.19651677012285	1.35570751532808\\
3.20591204311225	1.32855600187292\\
3.21369968872289	1.30037468766754\\
3.21985335966386	1.27093935036681\\
3.22429562795569	1.23999115375056\\
3.22689062345455	1.20722806743009\\
3.22743260712858	1.17229390269361\\
3.22562922655149	1.13476425208792\\
3.22107757786229	1.09412838138774\\
3.21323030795922	1.04976582242581\\
3.20134770676748	1.00091605681835\\
3.18442989421847	0.946639300923022\\
3.16112060377793	0.885766123130951\\
3.12957055354315	0.816833764968077\\
3.08724411929405	0.73800835618114\\
3.03064910810781	0.64699640659248\\
2.95496974575675	0.540959600987511\\
2.85359820694491	0.416470990893481\\
2.71761568164062	0.269599915374646\\
2.53542181057784	0.0963008308050649\\
2.29302638863571	-0.106599668422552\\
1.97600821968529	-0.339474794327379\\
1.5744278205967	-0.597031272628531\\
1.09079846615224	-0.865714438789585\\
0.547271679730867	-1.12426677793354\\
-0.0156987973804935	-1.34975858100649\\
-0.552609514395057	-1.5264343025763\\
-1.02938785202724	-1.65058730346741\\
-1.43097456897238	-1.72843647162033\\
-1.75816612094634	-1.77058643378522\\
-2.02028675676074	-1.78756478900962\\
-2.22921263119039	-1.78781501291504\\
-2.39615822324344	-1.77739781919804\\
-2.53049340461053	-1.76040181520039\\
-2.63960205598709	-1.73948683427999\\
-2.72914325726933	-1.71634290147778\\
-2.80340494859105	-1.69202300667229\\
-2.86562888501474	-1.66716817309768\\
-2.9182730255947	-1.64215460857281\\
-2.96321163045467	-1.6171883613051\\
-3.00188381624089	-1.5923657554321\\
-3.0354025628448	-1.56771182882603\\
-3.06463440778332	-1.54320467631437\\
-3.09025774070329	-1.51879072927095\\
-3.11280553904297	-1.49439415253789\\
-3.13269675775084	-1.46992236302528\\
-3.15025937346884	-1.44526892684105\\
-3.16574720105731	-1.42031461319411\\
-3.17935196092714	-1.39492707139712\\
-3.19121160749489	-1.3689593874057\\
-3.20141557668629	-1.34224762723294\\
-3.21000733136552	-1.3146073592814\\
-3.21698434428276	-1.28582904753639\\
-3.2222954297344	-1.25567210867758\\
-3.22583508947928	-1.22385731664626\\
-3.22743424402313	-1.19005710671384\\
-3.22684633746756	-1.15388316582293\\
-3.22372727881572	-1.11487048454915\\
-3.21760693908985	-1.07245677716519\\
-3.20784885476026	-1.02595584522598\\
-3.19359324689341	-0.974523082798222\\
-3.17367626606242	-0.91711096815095\\
-3.14651532584494	-0.852412257800675\\
-3.10994643917659	-0.778789190780974\\
-3.06099512366542	-0.694189465974422\\
-2.99555986643644	-0.596056555476247\\
-2.90799288715264	-0.48125811924112\\
-2.79059380934311	-0.346091178417871\\
-2.63312516162728	-0.186489932883004\\
-2.42268081912405	0.00133018142641184\\
-2.14465351030925	0.219451212471439\\
-1.78602806334448	0.465803890576448\\
-1.34199072149221	0.731173209142785\\
-0.824317574584908	0.997756503334357\\
-0.265278147565503	1.24239048713434\\
0.28996029303649	1.44471801699729\\
0.799904615435649	1.59485812742332\\
1.23981820678715	1.69467540015643\\
1.60340305424167	1.75327938401769\\
1.8966275701755	1.78163647045096\\
2.13066155564074	1.78934952646239\\
};
\addplot [color=mycolor1]
  table[row sep=crcr]{%
0.948683298050514	3.79473319220206\\
1.12996891013208	3.78902241071535\\
1.3027838103604	3.77255915923418\\
1.46753552344581	3.74623053454947\\
1.62465224482371	3.71077288416293\\
1.77456152903207	3.66677970030574\\
1.91767403561802	3.61471073626391\\
2.05437110063153	3.55490128334521\\
2.18499505421831	3.4875709448019\\
2.30984136685525	3.41283153872394\\
2.42915185734506	3.33069397942336\\
2.54310832592638	3.24107414976038\\
2.65182608463458	3.14379790634541\\
2.75534694820212	3.03860547381077\\
2.8536313296431	2.92515559896292\\
2.9465491655584	2.80302996392973\\
3.0338694906698	2.67173851056894\\
3.11524860652315	2.53072651479627\\
3.19021696761039	2.3793844730251\\
3.25816516606007	2.21706211963439\\
3.31832976399147	2.0430881668871\\
3.36978023105031	1.85679760784185\\
3.41140891541445	1.65756857671014\\
3.44192680709927	1.44487070344488\\
3.45986879009691	1.21832646083672\\
3.46361298678702	0.977785965288651\\
3.45141941180798	0.723413822791541\\
3.4214930614047	0.455783738589644\\
3.37207522993717	0.175972765268488\\
3.30156372141992	-0.114357323644244\\
3.20865740947072	-0.412907981729085\\
3.0925136091427	-0.716739496073036\\
2.95289924526556	-1.02233055985344\\
2.79031115637498	-1.3257107709346\\
2.60603988731933	-1.6226624623078\\
2.40215712665582	-1.90897292279738\\
2.18141953804948	-2.18070442957845\\
1.94709823579504	-2.43444260737754\\
1.70275852491129	-2.66748659951055\\
1.45202370644265	-2.877956852001\\
1.19835703486066	-3.0648138241024\\
0.944888096380122	-3.22779787311511\\
0.694297500839398	-3.36731218002033\\
0.448761194715288	-3.48427490957016\\
0.209946190700892	-3.57996468520442\\
-0.0209555786883461	-3.65587741213202\\
-0.243169602253506	-3.71360522906444\\
-0.456269901983927	-3.7547419037936\\
-0.660108423391693	-3.78081428495334\\
-0.854748940791145	-3.79323662458319\\
-1.04040889805107	-3.79328335955738\\
-1.21741050351997	-3.78207578896275\\
-1.38614100991934	-3.76057854694004\\
-1.54702130890241	-3.72960250007616\\
-1.70048159286015	-3.6898114747465\\
-1.84694273429099	-3.64173092446581\\
-1.98680208954437	-3.58575723170233\\
-2.1204225680783	-3.52216679568137\\
-2.24812396922559	-3.45112440165093\\
-2.37017574651009	-3.37269062063008\\
-2.48679050050333	-3.2868281760587\\
-2.59811762050387	-3.19340735751618\\
-2.70423659456836	-3.09221068144532\\
-2.80514959232286	-2.98293711145668\\
-2.90077300437504	-2.86520627071399\\
-2.99092770791458	-2.73856321838213\\
-3.07532793559953	-2.60248453106578\\
-3.1535687736533	-2.45638663503084\\
-3.22511252958097	-2.29963757630341\\
-3.28927451889835	-2.13157368366421\\
-3.34520925490937	-1.95152284702656\\
-3.39189861348128	-1.7588363466938\\
-3.42814429814894	-1.5529312348605\\
-3.45256782598978	-1.33334504716437\\
-3.46362220116165	-1.0998039137315\\
-3.45962024522194	-0.852303713618117\\
-3.43878487541956	-0.591201557879526\\
-3.3993259814164	-0.317311504170883\\
-3.3395463824176	-0.0319942042762583\\
-3.25797518618734	0.262774139257094\\
-3.15352067565559	0.564371272065384\\
-3.02562738470585	0.869554764908886\\
-2.87441511977517	1.17455760801215\\
-2.70077402394131	1.47525676358082\\
-2.50639202484484	1.767403513355\\
-2.29370038309167	2.04688929885519\\
-2.06573813670639	2.31000991273335\\
-1.8259528645897	2.5536887433386\\
-1.57796799985389	2.77562773795799\\
-1.32535178123783	2.9743703475303\\
-1.07141885643386	3.14927865398809\\
-0.819084909373598	3.30044159903781\\
-0.570781691446472	3.42853925898355\\
-0.328428492520302	3.53468897189284\\
-0.093448740680559	3.62029468577783\\
0.13318251344032	3.68691393725669\\
0.350873927242831	3.73614982594868\\
0.55934713845833	3.76956970364108\\
0.758567824726602	3.78864856747597\\
0.948683298050513	3.79473319220206\\
};
\addlegendentry{Внешние аппроксимации}

\addplot [color=mycolor2, forget plot]
  table[row sep=crcr]{%
2.22721472973942	1.94935886896179\\
2.3886648425121	1.94441487511101\\
2.5195634483927	1.93205457572182\\
2.62670779849127	1.91501766628131\\
2.71531853364	1.89508786650117\\
2.78936864949369	1.87341097537522\\
2.85187814440773	1.85071228241933\\
2.90515032165589	1.82744075521499\\
2.9509520177711	1.80386332654994\\
2.99064783647547	1.78012618777659\\
3.02529909977035	1.7562945351756\\
3.0557365623221	1.73237827445755\\
3.08261387510984	1.7083485310185\\
3.10644696879361	1.68414807630787\\
3.12764310033082	1.65969765855822\\
3.14652223914943	1.63489950162782\\
3.16333268711427	1.60963876367151\\
3.17826225469757	1.5837834336332\\
3.19144589188063	1.55718292750329\\
3.20297034763997	1.52966548843736\\
3.2128761683912	1.50103436869378\\
3.2211571124149	1.47106265802835\\
3.22775682617807	1.43948650786522\\
3.23256237104755	1.40599637010381\\
3.23539387180292	1.3702257107414\\
3.23598913837185	1.33173645766061\\
3.23398152983158	1.29000018434938\\
3.22886850103885	1.24437370419161\\
3.21996707976175	1.19406735000614\\
3.20635080827997	1.13810376656035\\
3.18676025898309	1.07526465092859\\
3.15947593702972	1.0040228169715\\
3.12213828317303	0.922457920675385\\
3.07149545967596	0.828157721349833\\
3.00305879148097	0.718116213117772\\
2.91065699973037	0.588662002756959\\
2.785925244978	0.435496623074383\\
2.61788823849866	0.254008232649376\\
2.39306798253518	0.0401523940333445\\
2.09699730361163	-0.20771900183087\\
1.7183819635995	-0.485932899271695\\
1.25639683089422	-0.782386182967809\\
0.728319451607428	-1.07597695395358\\
0.17048677629427	-1.34156752261761\\
-0.372344302750219	-1.55919115860543\\
-0.863264272549092	-1.72086595365401\\
-1.28299919742863	-1.8302347441912\\
-1.62890152595063	-1.89731652670458\\
-1.90833376158261	-1.93331843320936\\
-2.13240654410545	-1.94782687646357\\
-2.3122537087605	-1.94803363323748\\
-2.45746747168844	-1.93896359662991\\
-2.57574556593285	-1.92399109460735\\
-2.67305603610964	-1.90533078490532\\
-2.75395564612754	-1.88441460443591\\
-2.82190758462804	-1.86215611801615\\
-2.87954766726586	-1.83912801979308\\
-2.92889157233296	-1.81567886095776\\
-2.97149068204595	-1.79200910352774\\
-3.00854741762708	-1.76822049003994\\
-3.04100007190525	-1.74434802267285\\
-3.0695851473103	-1.7203805908067\\
-3.09488323143276	-1.69627413213334\\
-3.11735281839452	-1.67195981527391\\
-3.13735524534064	-1.64734883071173\\
-3.15517299859553	-1.62233479306363\\
-3.17102297575804	-1.59679437416259\\
-3.18506579905015	-1.570586527561\\
-3.19741190629232	-1.54355048208521\\
-3.20812485602917	-1.51550254281508\\
-3.21722203851139	-1.48623162017347\\
-3.22467275544594	-1.45549329519237\\
-3.23039339092638	-1.42300210774249\\
-3.23423911335776	-1.38842161161915\\
-3.23599118590349	-1.35135156242766\\
-3.23533847055249	-1.31131137681683\\
-3.23185101774822	-1.2677187105326\\
-3.22494264006069	-1.21986163848029\\
-3.2138179373885	-1.16686249014231\\
-3.19739719631547	-1.10763095340298\\
-3.17420973909784	-1.0408037811618\\
-3.142242561071	-0.964668760862181\\
-3.09872682936675	-0.877072578179989\\
-3.0398417586358	-0.77531812311728\\
-2.96031892266926	-0.656071422287709\\
-2.85295425309884	-0.515330753928937\\
-2.70811161177175	-0.348574764250833\\
-2.51348980394932	-0.151314840167112\\
-2.25479111797749	0.0795960527623154\\
-1.91840618384291	0.343563906305597\\
-1.49721915275415	0.633014609293368\\
-0.998726638487032	0.931115827109969\\
-0.450328181715587	1.21375185121395\\
0.105465468340197	1.45718721810745\\
0.625945116714116	1.64701594620063\\
1.08251666630864	1.78154116777267\\
1.46486936946524	1.86834353319716\\
1.77626331377619	1.91854897142989\\
2.02656829443923	1.94275375393752\\
2.22721472973941	1.94935886896179\\
};
\addplot [color=mycolor2]
  table[row sep=crcr]{%
2.12132034355964	2.82842712474619\\
2.18582575000323	2.82640660874952\\
2.2459761483502	2.82067996094826\\
2.30316600770785	2.81153894842004\\
2.35833530419819	2.79908319427389\\
2.4121407651283	2.78328504568152\\
2.46505313939319	2.76402339892432\\
2.51741350921551	2.74110157112695\\
2.56946568278775	2.71425682483196\\
2.62137377470791	2.68316571665979\\
2.67322998558975	2.64744786912504\\
2.72505542741334	2.60667010622203\\
2.77679571750944	2.56035271147659\\
2.82831256449113	2.50797963523731\\
2.87937250576692	2.44901464186941\\
2.92963424895255	2.38292549191834\\
2.97863667364875	2.30921809960124\\
3.02579039162557	2.22748193169334\\
3.0703766721051	2.13744642550811\\
3.11155818595542	2.03904567139316\\
3.14840590189045	1.93248505458784\\
3.1799449834646	1.81829951655162\\
3.20521925970046	1.69738980361013\\
3.2233689191928	1.57102231442397\\
3.23371058284854	1.44078163681543\\
3.23580480244767	1.30847305653245\\
3.22949541157146	1.1759834831\\
3.21490900216393	1.04511946165368\\
3.19240995273442	0.917445713323391\\
3.16251366135309	0.794144173493542\\
3.12576385298257	0.675902054710668\\
3.08257536776515	0.562820516214273\\
3.03302813706062	0.454314775018543\\
2.97656426645917	0.348948817128694\\
2.91146921429886	0.244099783305823\\
2.83385680006128	0.135243682019644\\
2.73548220960228	0.0144130439331602\\
2.59871334341581	-0.133168683665202\\
2.38479638365891	-0.336180712113444\\
2.01057175466146	-0.648371525543042\\
1.34210041814469	-1.13791067535239\\
0.375352321166521	-1.75822148716338\\
-0.527448468220416	-2.26262801632552\\
-1.11365847778129	-2.5437869612038\\
-1.4567327767317	-2.68228003389134\\
-1.6691296256022	-2.75259678373195\\
-1.81428800439327	-2.79054335642282\\
-1.92320462744895	-2.81168897715777\\
-2.01132392805712	-2.82302365363059\\
-2.08680248126988	-2.82787464184343\\
-2.15422943553058	-2.82790871121207\\
-2.21634478307291	-2.8239827357917\\
-2.27487422572476	-2.8165264847195\\
-2.33095693947289	-2.80572465930185\\
-2.3853755403757	-2.79160756168836\\
-2.43868469976231	-2.77409782070174\\
-2.49128507004909	-2.75303495626884\\
-2.54346612600745	-2.7281884019911\\
-2.59543032955127	-2.69926454755406\\
-2.64730535029083	-2.66591103001887\\
-2.69914810041563	-2.62772046423944\\
-2.75094277189846	-2.58423541950896\\
-2.80259428647749	-2.53495641420315\\
-2.85391830194335	-2.47935483422118\\
-2.90462903903618	-2.41689283661657\\
-2.95432664759112	-2.34705230341292\\
-3.00248656709011	-2.26937453210355\\
-3.04845424185856	-2.18351130413452\\
-3.0914493816545	-2.08928598065547\\
-3.13058429431929	-1.98676020448289\\
-3.16490008683964	-1.87629788656397\\
-3.19342217916266	-1.75861429449935\\
-3.21523240499265	-1.63479579561046\\
-3.22954956680171	-1.50627700312196\\
-3.23580523135335	-1.37476797309193\\
-3.23369896357047	-1.2421341494958\\
-3.22321880671614	-1.11024302689198\\
-3.20461860896113	-0.980799507713277\\
-3.17835148049538	-0.85519274756288\\
-3.14496439092874	-0.734369602871579\\
-3.10495870881183	-0.618735254776386\\
-3.05861192744495	-0.508062722468019\\
-3.0057323937692	-0.401369642978784\\
-2.94527025004781	-0.296685462718974\\
-2.87460204254734	-0.190562899149911\\
-2.78805571026974	-0.0770301658026245\\
-2.67361421522119	0.0546866392302208\\
-2.50520482475694	0.225105040795738\\
-2.22541867988909	0.474087715816021\\
-1.7209533613293	0.868461990547897\\
-0.880186083660502	1.44500069250654\\
0.110323684998138	2.03883159526995\\
0.859259206598599	2.42737283484743\\
1.30741523004191	2.62511520505649\\
1.57426953542748	2.72304263504847\\
1.74766615441039	2.77435113512036\\
1.87207484559155	2.80265686669549\\
1.96924085114733	2.81832059141065\\
2.0502990590364	2.82612977760851\\
2.12132034355964	2.82842712474619\\
};
\addlegendentry{Внутренние аппроксимации}

\end{axis}

\begin{axis}[%
width=0.798\linewidth,
height=0.578\linewidth,
at={(-0.104\linewidth,-0.064\linewidth)},
scale only axis,
xmin=0,
xmax=1,
ymin=0,
ymax=1,
axis line style={draw=none},
ticks=none,
axis x line*=bottom,
axis y line*=left,
legend style={legend cell align=left, align=left, draw=white!15!black}
]
\end{axis}
\end{tikzpicture}%
        \caption{Эллипсоидальные аппроксимации для 50 направлений.}
\end{figure}
%%%%%%%%%%%%%%%%%%%%%%%%%%%%%%%%%%%%%%%%%%%%%%%%%%%%%%%%%%%%%%%%%%%%%%%%%%%%%%%%


%% Document end


%%%%%%%%%%%%%%%%%%%%%%%%%%%%%%%%%%%%%%%%%%%%%%%%%%%%%%%%%%%%%%%%%%%%%%%%%%%%%%%%
\clearpage
\begin{thebibliography}{9}
% Библиография
\bibitem{kurzh} Kurzhanski A. B., Varaiya P. \textit{Dynamics and Control of Trajectory Tubes.} Birkhauser, 2014.
\end{thebibliography}
\end{document}